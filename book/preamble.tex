% --- Page layout
\usepackage[a4paper,left=3cm,right=3cm,top=2cm,bottom=2cm]{geometry}
\usepackage{booktabs}
\usepackage{longtable}
\usepackage{array}
\usepackage{xcolor}

% --- Fonts (XeLaTeX)
\usepackage{fontspec}
\XeTeXinputnormalization=1
\defaultfontfeatures{Ligatures=TeX,Scale=MatchLowercase}
\IfFontExistsTF{TeX Gyre Termes}{\setmainfont{TeX Gyre Termes}}{\setmainfont{Times New Roman}}
\IfFontExistsTF{TeX Gyre Heros}{\setsansfont{TeX Gyre Heros}}{\setsansfont{Arial}}
\IfFontExistsTF{DejaVu Sans Mono}{\setmonofont{DejaVu Sans Mono}}{%
  \IfFontExistsTF{Menlo}{\setmonofont{Menlo}}{\setmonofont{Latin Modern Mono}}%
}

% --- Hyperlinks: colors + behaviors
\definecolor{linkblue}{HTML}{0645AD}

% Make sure hyperref picks our colors even if Pandoc touches it later
\PassOptionsToPackage{unicode=true}{hyperref}
\usepackage{hyperref}
\usepackage{xurl}
\urlstyle{same} % URLs use surrounding text font

% Ensure final link colors after everything is loaded
\makeatletter
\AtBeginDocument{%
  \hypersetup{
    colorlinks=true,     % color actual text (no boxes)
    linktoc=all,         % color ToC entries
    linkcolor=linkblue,  % internal links (sections, figures, ToC)
    citecolor=linkblue,  % in-text citations
    urlcolor=linkblue    % external URLs
  }%
  % Also force the internal color registers hyperref uses
  \def\@linkcolor{linkblue}\def\@anchorcolor{linkblue}%
  \def\@citecolor{linkblue}\def\@urlcolor{linkblue}%
}
\makeatother

% --- Underline helpers (for EXTERNAL links only)
\usepackage[normalem]{ulem} % \uline
\renewcommand{\ULdepth}{1.2pt}
\renewcommand{\ULthickness}{0.6pt}
% Keep PDF bookmarks clean
\pdfstringdefDisableCommands{%
  \def\uline#1{#1}%
  \def\textcolor#1#2{#2}%
}

% Wrappers
\makeatletter
% Blue+underline text (used for external links/urls only)
\DeclareRobustCommand{\EGV@ulineblue}[1]{\begingroup\color{linkblue}\uline{#1}\endgroup}
% Blue (no underline) text (used for internal links & citations)
\DeclareRobustCommand{\EGV@blue}[1]{\begingroup\color{linkblue}#1\endgroup}

% --- EXTERNAL links: underline + blue
\let\EGV@oldhref\href
\renewcommand{\href}[2]{\EGV@oldhref{#1}{\EGV@ulineblue{#2}}}
\let\EGV@oldurl\url
\renewcommand{\url}[1]{\EGV@ulineblue{\EGV@oldurl{#1}}}

% Provide a macro for code-links.lua (link literal code → blue+underlined)
\newcommand{\linkcode}[2]{\href{#2}{\EGV@ulineblue{\texttt{#1}}}}

% --- INTERNAL links (cross-refs, ToC, citations): blue, NO underline
% \hyperref[<label>]{<text>}  (cross-refs/ToC entries)
\let\EGV@oldhyperref\hyperref
\renewcommand{\hyperref}[2][]{\EGV@oldhyperref[#1]{\EGV@blue{#2}}}
% \hyperlink{<target>}{<text>} (citeproc often uses this for in-text citations)
\let\EGV@oldhyperlink\hyperlink
\renewcommand{\hyperlink}[2]{\EGV@oldhyperlink{#1}{\EGV@blue{#2}}}
% Plain \ref / \pageref / \autoref numbers also blue (no underline)
\let\EGV@oldref\ref
\renewcommand{\ref}[1]{\EGV@blue{\EGV@oldref{#1}}}
\let\EGV@oldpageref\pageref
\renewcommand{\pageref}[1]{\EGV@blue{\EGV@oldpageref{#1}}}
\@ifundefined{autoref}{}{%
  \let\EGV@oldautoref\autoref
  \renewcommand{\autoref}[1]{\EGV@blue{\EGV@oldautoref{#1}}}}
\makeatother

% IMPORTANT: Do NOT recolor the whole CSLReferences environment.
% We want only the URL/DOI inside entries to be blue+underlined (handled by \url/\href above),
% while the rest of the bibliography text remains black.

% --- Unicode superscripts/subscripts in text
\usepackage{newunicodechar}
\newunicodechar{⁰}{\textsuperscript{0}}
\newunicodechar{¹}{\textsuperscript{1}}
\newunicodechar{²}{\textsuperscript{2}}
\newunicodechar{³}{\textsuperscript{3}}
\newunicodechar{⁴}{\textsuperscript{4}}
\newunicodechar{⁵}{\textsuperscript{5}}
\newunicodechar{⁶}{\textsuperscript{6}}
\newunicodechar{⁷}{\textsuperscript{7}}
\newunicodechar{⁸}{\textsuperscript{8}}
\newunicodechar{⁹}{\textsuperscript{9}}
\newunicodechar{⁺}{\textsuperscript{+}}
\newunicodechar{⁻}{\textsuperscript{-}}
\newunicodechar{₀}{\textsubscript{0}}
\newunicodechar{₁}{\textsubscript{1}}
\newunicodechar{₂}{\textsubscript{2}}
\newunicodechar{₃}{\textsubscript{3}}
\newunicodechar{₄}{\textsubscript{4}}
\newunicodechar{₅}{\textsubscript{5}}
\newunicodechar{₆}{\textsubscript{6}}
\newunicodechar{₇}{\textsubscript{7}}
\newunicodechar{₈}{\textsubscript{8}}
\newunicodechar{₉}{\textsubscript{9}}
\newunicodechar{₊}{\textsubscript{+}}
\newunicodechar{₋}{\textsubscript{-}}

% --- Code blocks: framed + light background (no mdframed)
\definecolor{codebg}{HTML}{F7F7F7}
\definecolor{codeline}{HTML}{E0E0E0}
\usepackage{fvextra}
\DefineVerbatimEnvironment{Highlighting}{Verbatim}{%
  breaklines=true,
  breaksymbol=\tiny\ensuremath{\hookrightarrow},
  breakanywhere=true,
  commandchars=\\\{\},
  showtabs=false,
  showspaces=false,
  fontsize=\small,
  baselinestretch=1.0,
  formatcom=\ttfamily,
  frame=single,
  rulecolor=\color{codeline},
  framerule=0.4pt,
  framesep=6pt,
  bgcolor=codebg
}

% Neutralize Pandoc’s Shaded wrapper (avoid double boxes / errors)
\makeatletter
\@ifundefined{Shaded}{}{\renewenvironment{Shaded}{\ignorespaces}{\unskip}}
\makeatother

% Inline code small mono (no background)
\let\OldTexttt\texttt
\renewcommand{\texttt}[1]{\OldTexttt{\small #1}}

% --- Page style
\usepackage{fancyhdr}
\pagestyle{fancy}
\fancyhf{}
\fancyfoot[C]{\thepage}
\renewcommand{\headrulewidth}{0pt}
\renewcommand{\footrulewidth}{0pt}
\setlength{\headheight}{14pt}

% --- Pandoc helpers
\providecommand{\tightlist}{%
  \setlength{\itemsep}{0pt}\setlength{\parskip}{0pt}}
\setlength{\emergencystretch}{3em}
\sloppy
