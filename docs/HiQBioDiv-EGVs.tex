% Options for packages loaded elsewhere
\PassOptionsToPackage{unicode}{hyperref}
\PassOptionsToPackage{hyphens}{url}
\documentclass[
]{book}
\usepackage{xcolor}
\usepackage{amsmath,amssymb}
\setcounter{secnumdepth}{5}
\usepackage{iftex}
\ifPDFTeX
  \usepackage[T1]{fontenc}
  \usepackage[utf8]{inputenc}
  \usepackage{textcomp} % provide euro and other symbols
\else % if luatex or xetex
  \usepackage{unicode-math} % this also loads fontspec
  \defaultfontfeatures{Scale=MatchLowercase}
  \defaultfontfeatures[\rmfamily]{Ligatures=TeX,Scale=1}
\fi
\usepackage{lmodern}
\ifPDFTeX\else
  % xetex/luatex font selection
\fi
% Use upquote if available, for straight quotes in verbatim environments
\IfFileExists{upquote.sty}{\usepackage{upquote}}{}
\IfFileExists{microtype.sty}{% use microtype if available
  \usepackage[]{microtype}
  \UseMicrotypeSet[protrusion]{basicmath} % disable protrusion for tt fonts
}{}
\makeatletter
\@ifundefined{KOMAClassName}{% if non-KOMA class
  \IfFileExists{parskip.sty}{%
    \usepackage{parskip}
  }{% else
    \setlength{\parindent}{0pt}
    \setlength{\parskip}{6pt plus 2pt minus 1pt}}
}{% if KOMA class
  \KOMAoptions{parskip=half}}
\makeatother
\usepackage{color}
\usepackage{fancyvrb}
\newcommand{\VerbBar}{|}
\newcommand{\VERB}{\Verb[commandchars=\\\{\}]}
\DefineVerbatimEnvironment{Highlighting}{Verbatim}{commandchars=\\\{\}}
% Add ',fontsize=\small' for more characters per line
\usepackage{framed}
\definecolor{shadecolor}{RGB}{248,248,248}
\newenvironment{Shaded}{\begin{snugshade}}{\end{snugshade}}
\newcommand{\AlertTok}[1]{\textcolor[rgb]{0.94,0.16,0.16}{#1}}
\newcommand{\AnnotationTok}[1]{\textcolor[rgb]{0.56,0.35,0.01}{\textbf{\textit{#1}}}}
\newcommand{\AttributeTok}[1]{\textcolor[rgb]{0.13,0.29,0.53}{#1}}
\newcommand{\BaseNTok}[1]{\textcolor[rgb]{0.00,0.00,0.81}{#1}}
\newcommand{\BuiltInTok}[1]{#1}
\newcommand{\CharTok}[1]{\textcolor[rgb]{0.31,0.60,0.02}{#1}}
\newcommand{\CommentTok}[1]{\textcolor[rgb]{0.56,0.35,0.01}{\textit{#1}}}
\newcommand{\CommentVarTok}[1]{\textcolor[rgb]{0.56,0.35,0.01}{\textbf{\textit{#1}}}}
\newcommand{\ConstantTok}[1]{\textcolor[rgb]{0.56,0.35,0.01}{#1}}
\newcommand{\ControlFlowTok}[1]{\textcolor[rgb]{0.13,0.29,0.53}{\textbf{#1}}}
\newcommand{\DataTypeTok}[1]{\textcolor[rgb]{0.13,0.29,0.53}{#1}}
\newcommand{\DecValTok}[1]{\textcolor[rgb]{0.00,0.00,0.81}{#1}}
\newcommand{\DocumentationTok}[1]{\textcolor[rgb]{0.56,0.35,0.01}{\textbf{\textit{#1}}}}
\newcommand{\ErrorTok}[1]{\textcolor[rgb]{0.64,0.00,0.00}{\textbf{#1}}}
\newcommand{\ExtensionTok}[1]{#1}
\newcommand{\FloatTok}[1]{\textcolor[rgb]{0.00,0.00,0.81}{#1}}
\newcommand{\FunctionTok}[1]{\textcolor[rgb]{0.13,0.29,0.53}{\textbf{#1}}}
\newcommand{\ImportTok}[1]{#1}
\newcommand{\InformationTok}[1]{\textcolor[rgb]{0.56,0.35,0.01}{\textbf{\textit{#1}}}}
\newcommand{\KeywordTok}[1]{\textcolor[rgb]{0.13,0.29,0.53}{\textbf{#1}}}
\newcommand{\NormalTok}[1]{#1}
\newcommand{\OperatorTok}[1]{\textcolor[rgb]{0.81,0.36,0.00}{\textbf{#1}}}
\newcommand{\OtherTok}[1]{\textcolor[rgb]{0.56,0.35,0.01}{#1}}
\newcommand{\PreprocessorTok}[1]{\textcolor[rgb]{0.56,0.35,0.01}{\textit{#1}}}
\newcommand{\RegionMarkerTok}[1]{#1}
\newcommand{\SpecialCharTok}[1]{\textcolor[rgb]{0.81,0.36,0.00}{\textbf{#1}}}
\newcommand{\SpecialStringTok}[1]{\textcolor[rgb]{0.31,0.60,0.02}{#1}}
\newcommand{\StringTok}[1]{\textcolor[rgb]{0.31,0.60,0.02}{#1}}
\newcommand{\VariableTok}[1]{\textcolor[rgb]{0.00,0.00,0.00}{#1}}
\newcommand{\VerbatimStringTok}[1]{\textcolor[rgb]{0.31,0.60,0.02}{#1}}
\newcommand{\WarningTok}[1]{\textcolor[rgb]{0.56,0.35,0.01}{\textbf{\textit{#1}}}}
\usepackage{longtable,booktabs,array}
\usepackage{calc} % for calculating minipage widths
% Correct order of tables after \paragraph or \subparagraph
\usepackage{etoolbox}
\makeatletter
\patchcmd\longtable{\par}{\if@noskipsec\mbox{}\fi\par}{}{}
\makeatother
% Allow footnotes in longtable head/foot
\IfFileExists{footnotehyper.sty}{\usepackage{footnotehyper}}{\usepackage{footnote}}
\makesavenoteenv{longtable}
\usepackage{graphicx}
\makeatletter
\newsavebox\pandoc@box
\newcommand*\pandocbounded[1]{% scales image to fit in text height/width
  \sbox\pandoc@box{#1}%
  \Gscale@div\@tempa{\textheight}{\dimexpr\ht\pandoc@box+\dp\pandoc@box\relax}%
  \Gscale@div\@tempb{\linewidth}{\wd\pandoc@box}%
  \ifdim\@tempb\p@<\@tempa\p@\let\@tempa\@tempb\fi% select the smaller of both
  \ifdim\@tempa\p@<\p@\scalebox{\@tempa}{\usebox\pandoc@box}%
  \else\usebox{\pandoc@box}%
  \fi%
}
% Set default figure placement to htbp
\def\fps@figure{htbp}
\makeatother
% definitions for citeproc citations
\NewDocumentCommand\citeproctext{}{}
\NewDocumentCommand\citeproc{mm}{%
  \begingroup\def\citeproctext{#2}\cite{#1}\endgroup}
\makeatletter
 % allow citations to break across lines
 \let\@cite@ofmt\@firstofone
 % avoid brackets around text for \cite:
 \def\@biblabel#1{}
 \def\@cite#1#2{{#1\if@tempswa , #2\fi}}
\makeatother
\newlength{\cslhangindent}
\setlength{\cslhangindent}{1.5em}
\newlength{\csllabelwidth}
\setlength{\csllabelwidth}{3em}
\newenvironment{CSLReferences}[2] % #1 hanging-indent, #2 entry-spacing
 {\begin{list}{}{%
  \setlength{\itemindent}{0pt}
  \setlength{\leftmargin}{0pt}
  \setlength{\parsep}{0pt}
  % turn on hanging indent if param 1 is 1
  \ifodd #1
   \setlength{\leftmargin}{\cslhangindent}
   \setlength{\itemindent}{-1\cslhangindent}
  \fi
  % set entry spacing
  \setlength{\itemsep}{#2\baselineskip}}}
 {\end{list}}
\usepackage{calc}
\newcommand{\CSLBlock}[1]{\hfill\break\parbox[t]{\linewidth}{\strut\ignorespaces#1\strut}}
\newcommand{\CSLLeftMargin}[1]{\parbox[t]{\csllabelwidth}{\strut#1\strut}}
\newcommand{\CSLRightInline}[1]{\parbox[t]{\linewidth - \csllabelwidth}{\strut#1\strut}}
\newcommand{\CSLIndent}[1]{\hspace{\cslhangindent}#1}
\setlength{\emergencystretch}{3em} % prevent overfull lines
\providecommand{\tightlist}{%
  \setlength{\itemsep}{0pt}\setlength{\parskip}{0pt}}
% --- Page layout
\usepackage[a4paper,left=3cm,right=3cm,top=2cm,bottom=2cm]{geometry}
\usepackage{booktabs}
\usepackage{longtable}
\usepackage{array}
\usepackage{xcolor}

% --- Fonts (XeLaTeX)
\usepackage{fontspec}
\XeTeXinputnormalization=1
\defaultfontfeatures{Ligatures=TeX,Scale=MatchLowercase}
\IfFontExistsTF{TeX Gyre Termes}{\setmainfont{TeX Gyre Termes}}{\setmainfont{Times New Roman}}
\IfFontExistsTF{TeX Gyre Heros}{\setsansfont{TeX Gyre Heros}}{\setsansfont{Arial}}
\IfFontExistsTF{DejaVu Sans Mono}{\setmonofont{DejaVu Sans Mono}}{%
  \IfFontExistsTF{Menlo}{\setmonofont{Menlo}}{\setmonofont{Latin Modern Mono}}%
}

% --- Hyperlinks: colors + behaviors
\definecolor{linkblue}{HTML}{0645AD}

% Make sure hyperref picks our colors even if Pandoc touches it later
\PassOptionsToPackage{unicode=true}{hyperref}
\usepackage{hyperref}
\usepackage{xurl}
\urlstyle{same} % URLs use surrounding text font

% Ensure final link colors after everything is loaded
\makeatletter
\AtBeginDocument{%
  \hypersetup{
    colorlinks=true,     % color actual text (no boxes)
    linktoc=all,         % color ToC entries
    linkcolor=linkblue,  % internal links (sections, figures, ToC)
    citecolor=linkblue,  % in-text citations
    urlcolor=linkblue    % external URLs
  }%
  % Also force the internal color registers hyperref uses
  \def\@linkcolor{linkblue}\def\@anchorcolor{linkblue}%
  \def\@citecolor{linkblue}\def\@urlcolor{linkblue}%
}
\makeatother

% --- Underline helpers (for EXTERNAL links only)
\usepackage[normalem]{ulem} % \uline
\renewcommand{\ULdepth}{1.2pt}
\renewcommand{\ULthickness}{0.6pt}
% Keep PDF bookmarks clean
\pdfstringdefDisableCommands{%
  \def\uline#1{#1}%
  \def\textcolor#1#2{#2}%
}

% Wrappers
\makeatletter
% Blue+underline text (used for external links/urls only)
\DeclareRobustCommand{\EGV@ulineblue}[1]{\begingroup\color{linkblue}\uline{#1}\endgroup}
% Blue (no underline) text (used for internal links & citations)
\DeclareRobustCommand{\EGV@blue}[1]{\begingroup\color{linkblue}#1\endgroup}

% --- EXTERNAL links: underline + blue
\let\EGV@oldhref\href
\renewcommand{\href}[2]{\EGV@oldhref{#1}{\EGV@ulineblue{#2}}}
\let\EGV@oldurl\url
\renewcommand{\url}[1]{\EGV@ulineblue{\EGV@oldurl{#1}}}

% Provide a macro for code-links.lua (link literal code → blue+underlined)
\newcommand{\linkcode}[2]{\href{#2}{\EGV@ulineblue{\texttt{#1}}}}

% --- INTERNAL links (cross-refs, ToC, citations): blue, NO underline
% \hyperref[<label>]{<text>}  (cross-refs/ToC entries)
\let\EGV@oldhyperref\hyperref
\renewcommand{\hyperref}[2][]{\EGV@oldhyperref[#1]{\EGV@blue{#2}}}
% \hyperlink{<target>}{<text>} (citeproc often uses this for in-text citations)
\let\EGV@oldhyperlink\hyperlink
\renewcommand{\hyperlink}[2]{\EGV@oldhyperlink{#1}{\EGV@blue{#2}}}
% Plain \ref / \pageref / \autoref numbers also blue (no underline)
\let\EGV@oldref\ref
\renewcommand{\ref}[1]{\EGV@blue{\EGV@oldref{#1}}}
\let\EGV@oldpageref\pageref
\renewcommand{\pageref}[1]{\EGV@blue{\EGV@oldpageref{#1}}}
\@ifundefined{autoref}{}{%
  \let\EGV@oldautoref\autoref
  \renewcommand{\autoref}[1]{\EGV@blue{\EGV@oldautoref{#1}}}}
\makeatother

% IMPORTANT: Do NOT recolor the whole CSLReferences environment.
% We want only the URL/DOI inside entries to be blue+underlined (handled by \url/\href above),
% while the rest of the bibliography text remains black.

% --- Unicode superscripts/subscripts in text
\usepackage{newunicodechar}
\newunicodechar{⁰}{\textsuperscript{0}}
\newunicodechar{¹}{\textsuperscript{1}}
\newunicodechar{²}{\textsuperscript{2}}
\newunicodechar{³}{\textsuperscript{3}}
\newunicodechar{⁴}{\textsuperscript{4}}
\newunicodechar{⁵}{\textsuperscript{5}}
\newunicodechar{⁶}{\textsuperscript{6}}
\newunicodechar{⁷}{\textsuperscript{7}}
\newunicodechar{⁸}{\textsuperscript{8}}
\newunicodechar{⁹}{\textsuperscript{9}}
\newunicodechar{⁺}{\textsuperscript{+}}
\newunicodechar{⁻}{\textsuperscript{-}}
\newunicodechar{₀}{\textsubscript{0}}
\newunicodechar{₁}{\textsubscript{1}}
\newunicodechar{₂}{\textsubscript{2}}
\newunicodechar{₃}{\textsubscript{3}}
\newunicodechar{₄}{\textsubscript{4}}
\newunicodechar{₅}{\textsubscript{5}}
\newunicodechar{₆}{\textsubscript{6}}
\newunicodechar{₇}{\textsubscript{7}}
\newunicodechar{₈}{\textsubscript{8}}
\newunicodechar{₉}{\textsubscript{9}}
\newunicodechar{₊}{\textsubscript{+}}
\newunicodechar{₋}{\textsubscript{-}}

% --- Code blocks: framed + light background (no mdframed)
\definecolor{codebg}{HTML}{F7F7F7}
\definecolor{codeline}{HTML}{E0E0E0}
\usepackage{fvextra}
\DefineVerbatimEnvironment{Highlighting}{Verbatim}{%
  breaklines=true,
  breaksymbol=\tiny\ensuremath{\hookrightarrow},
  breakanywhere=true,
  commandchars=\\\{\},
  showtabs=false,
  showspaces=false,
  fontsize=\small,
  baselinestretch=1.0,
  formatcom=\ttfamily,
  frame=single,
  rulecolor=\color{codeline},
  framerule=0.4pt,
  framesep=6pt,
  bgcolor=codebg
}

% Neutralize Pandoc’s Shaded wrapper (avoid double boxes / errors)
\makeatletter
\@ifundefined{Shaded}{}{\renewenvironment{Shaded}{\ignorespaces}{\unskip}}
\makeatother

% Inline code small mono (no background)
\let\OldTexttt\texttt
\renewcommand{\texttt}[1]{\OldTexttt{\small #1}}

% --- Page style
\usepackage{fancyhdr}
\pagestyle{fancy}
\fancyhf{}
\fancyfoot[C]{\thepage}
\renewcommand{\headrulewidth}{0pt}
\renewcommand{\footrulewidth}{0pt}
\setlength{\headheight}{14pt}

% --- Pandoc helpers
\providecommand{\tightlist}{%
  \setlength{\itemsep}{0pt}\setlength{\parskip}{0pt}}
\setlength{\emergencystretch}{3em}
\sloppy
\usepackage{bookmark}
\IfFileExists{xurl.sty}{\usepackage{xurl}}{} % add URL line breaks if available
\urlstyle{same}
\hypersetup{
  pdftitle={High-resolution ecogeographical variables for species distribution modelling describing Latvia, 2024},
  pdfauthor={Andris Avotiņš; Jekaterīna Butkeviča; Betija Rubene; Solvita Rūsiņa; Rūta Starka; Vita Šakele; Vineta Vērpēja; Ivo Vinogradovs; Ainārs Auniņš},
  hidelinks,
  pdfcreator={LaTeX via pandoc}}

\title{High-resolution ecogeographical variables for species distribution modelling describing Latvia, 2024}
\author{Andris Avotiņš \and Jekaterīna Butkeviča \and Betija Rubene \and Solvita Rūsiņa \and Rūta Starka \and Vita Šakele \and Vineta Vērpēja \and Ivo Vinogradovs \and Ainārs Auniņš}
\date{2025-12-03}

\begin{document}
\maketitle

{
\setcounter{tocdepth}{1}
\tableofcontents
}
\chapter*{Preface}\label{preface}
\addcontentsline{toc}{chapter}{Preface}

Welcome! This book documents the geodata and processing workflows used to create
ecogeographical variables (EGVs) for species distribution modelling in Latvia (2024).

This material presents the results of three University of Latvia projects deeply
rooted in species distribution modelling and, more importantly, explains the workflow
and decisions made to ensure their repeatability and
reproducibility. These projects are:

\begin{itemize}
\item
  The project ``Preparation of a geospatial data layer covering existing
  protected areas for the implementation of the EU Biodiversity Strategy
  2030'' (No.~1-08/73/2023), funded by the Administrations of the Latvian
  Environmental Protection Fund;
\item
  Scientific research service project commissioned by the Joint Stock Company ``Latvijas valsts
  meži'' (Latvian State Forests) ``Improvement of the monitoring of the northern
  goshawk \emph{Accipiter gentilis} and creation of a spatial model of habitat
  suitability'' (Latvian State Forests document No.~5-5.5.1\_000r\_101\_23\_27\_6);
\item
  State research program ``Development of research specified in the Biodiversity
  Priority Action Program'' project ``High-resolution quantification of biodiversity
  for nature conservation and management: HiQBioDiv'' (VPP-VARAM-DABA-2024/1-0002).
\end{itemize}

The material was developed in R using \{bookdown\}. The data processing and analysis
described in the content was mainly performed in R, and one of the main reasons
for creating this material was to transfer the information necessary for
reproducing the work using verified command lines. A desirable side effect
is to promote openness and reproducibility in scientific practice and practical
science.

\begin{itemize}
\item
  Home repository of this material: \href{https://github.com/aavotins/HiQBioDiv_EGVs}{aavotins/HiQBioDiv\_EGVs}
\item
  Cite as needed using \texttt{book/book.bib}.
\end{itemize}

\section*{About this material}\label{about-this-material}
\addcontentsline{toc}{section}{About this material}

This material \textbf{is not}:

\begin{itemize}
\item
  \emph{an introduction to R or other programming language}. On the contrary, it will
  be most useful to those who already understand how to use command lines.
  However, it will also be informative for other users regarding the approaches used;
\item
  \emph{a tutorial on geoprocessing}. This material summarizes the approaches that,
  at the time of its development, were known to the authors as the most
  effective (in terms of processing time, RAM and hard disk space, performance
  guarantees, and reliability), but they are certainly not the only ones possible;
\item
  \emph{copy/paste ready product}. Although the use and publication of command lines
  tends to be intended for these purposes, in a situation where large amounts of
  data and, at least in part, restricted access data are used for the work, this
  is simply not possible. However, by ensuring data availability and placement in
  accordance with the file structure of this project (available at \texttt{root/Data} or
  by forking \href{https://github.com/aavotins/HiQBioDiv_FileTree}{template repository}), the
  command lines will be repeatable without changes and will produce the same results.
\end{itemize}

This material \textbf{has been} prepared to provide a reproducible workflow, describing
the decisions made and solutions implemented in the preparation of ecogeographical
variables for species distribution (habitat suitability) modelling for biodiversity
conservation planning.

For the most part, this material consists of:

\begin{itemize}
\item
  \emph{explanatory text}, which is recognizable as text;
\item
  \emph{command lines}, which are hidden by default to make the text easier to read.
  The locations of the command lines can be identified by the ``\textbar\textgreater{} Code'' visible
  on the left side of the page, just below this paragraph. Clicking on it will open
  the code area, where the text on a grey background is command lines, for example:
\end{itemize}

\begin{Shaded}
\begin{Highlighting}[]
\NormalTok{object}\OtherTok{=}\ControlFlowTok{function}\NormalTok{(arguments1,arguments2,}
\AttributeTok{path=}\StringTok{"./path/file/tree/object.extension"}\NormalTok{)}
\CommentTok{\# comment}
\end{Highlighting}
\end{Shaded}

In the example above, the first line creates an object (``object'') that is
the result of a function (``function()''). The function has three
arguments (``arguments1'', ``arguments2'' and ``path'') separated by commas (as with all
function arguments in R). The third argument is the path in the file tree. It is
on ``a new line'' but is a continuation of the function on the previous line, because
the parentheses are not closed. Note the beginning ``./'', which indicates a relative
path - the location in the file tree is relative to the project location.

The second line of the example above is a comment - everything after ``\#'' is a
comment. Anything in a command line before ``\#'' must be an executable function or
object. A comment can contain anything and be on the same line as an executable
function (at the end of it).

Command lines are the most important part of this material for reproducibility.
However, the person using them must ensure the availability of input data and
maintain correct paths in the file tree.

In this material code chunks are formatted as individual pieces to better pinpoint
commands used for a job described in the text around. However, in practical setting the
creation of ecogeographical variables will be much faster, if they will be combined
in loops or other batch processing setup. Command lines used in practice are available
in the \href{https://github.com/aavotins/HiQBioDiv_EGVs}{home repository} of this material at\\
\texttt{Data/RScripts\_final}, they can be executed in an alphanumeric order, if not
specified differently. We performed parts of the compute on the University of Latvia
Institute of Numerical Modelling HPC cluster with the same file tree as in this
material. Shell scripts used to run R commands are available in
the \href{https://github.com/aavotins/HiQBioDiv_EGVs}{home repository} of this material
at \texttt{Data/hpc\_io/Jobs\_shell/2024/EGVs}.

Sometimes we will refer to R packages in the text, we will put them in curly
brackets, for example, \{package\}.

\begin{itemize}
\item
  \emph{graphics} - occasional diagrams that describe the workflow or data
  characteristics and maps;
\item
  \emph{links to other resources}, especially to higher-level products and results
  created within the project, as well as any publicly available data. The results
  are intended for practical use.
\end{itemize}

Within reason, the material describes all data sets used and provides metadata
related to ensuring reproducibility. Since not all data sets are freely available,
they are not published as such, but in all cases information is provided on how
they were obtained for the development of this project.

\section*{Outline}\label{outline}
\addcontentsline{toc}{section}{Outline}

\begin{enumerate}
\def\labelenumi{\arabic{enumi}.}
\item
  \hyperref[Ch01]{Terminology and acronyms}
\item
  \hyperref[Ch02]{Utilities}
\item
  \hyperref[Ch03]{Template files}
\item
  \hyperref[Ch04]{Raw geodata}
\item
  \hyperref[Ch05]{Geodata products}
\item
  \hyperref[Ch06]{Ecogeographical variables}
\item
  \hyperref[Ch07]{Data access}
\end{enumerate}

\chapter{Terminology and acronyms}\label{Ch01}

Athough all georeferenced data can be considered \emph{geodata}, in this material we
use the following terms in the order listed below in our workflows:

\begin{itemize}
\item
  \textbf{raw geodata} - considered as raw data obtained for a harmonised description
  of the environment. This may include tables with coordinates, raster or vector data.
  It can be anything that has been or can be used to create \emph{ecogeographical variables},
  with or without slight processing.
\item
  \textbf{geodata product} - processed \emph{raw geodata} that have undegone heavy modifications, e.g.~
  spatial overlays and combinations of different sets of \emph{raw geodata}, and are used
  as \emph{input data}. In this document, \emph{geodata products} are categorical
  raster layers that match the \emph{CRS} and the pixel locations of \emph{input data}. When
  split by categories, they become \emph{input data}. The processing step of creating \emph{geodata products}
  is necessary when decisions about the order of spatial overlays are important. For example,
  in a high-resolution pixel, there can only be water or forest, if the edge between water and
  forest need to be calculated.
\item
  \textbf{input data} or \textbf{input layers} - very-high resolution (multiple times higher than that
  used for \emph{ecogeographical variables}) raster data that are the direct input for the creation
  of most of the \emph{ecogeographical variables}. The creation of such layers is particularly useful
  alongside \emph{geodata products}, as dealing with border misalignment or decisions regarding the
  order of spatial overlays, as well as simple geoprocessing, is much faster with raster
  data.
\item
  \textbf{ecogeographical variables} (EGVs) - this is the final product of the workflow
  describing environment for statistical analysis (e.g.~\emph{species distribution modelling}).
  They are suitable also for publishing due to standardisation of the values. In other
  words, these are standardised landscape ecological variables in the form of
  high-resolution raster layers (we use 1 ha cells). Each layer contains values
  representing the environment within the cell footprint or a summary of focal
  neighbours. In our case, each layer is of quantitative data describing a natural
  quantity (e.g.~timber volume, mean annual temperature), or quantified information of
  categories (e.g.~the fraction of class's area in an analysis cell or some neighbourhood,
  the number of pixels creating an edge of a certain class or between two classes in the
  analysis cell or some neighbourhood). The values of each layer are standardised:
  for each cell, the layer mean is subtracted and the result is divided
  by the root mean square error. Therefore, the values are more suitable for
  modelling, and the layers can be made publicly available as they do not directly
  provide exact sensitive information.
\end{itemize}

In this material, we use the term \emph{species distribution modelling} \textbf{(SDM)} as
a more broadly used term, that is synonymous with \emph{ecological niche analysis}
and \emph{ecological niche modelling}.

Tree species groups:

\textbf{coniferous} - following species (codes) as used in the national forest
stand-level-inventory database:

\begin{itemize}
\item
  pines (1, 14, 22)
\item
  spruces (3, 15)
\item
  larch (13)
\item
  firs (23, 28)
\end{itemize}

\textbf{boreal deciduous} - following species (codes) as used in the national forest
stand-level-inventory database:

\begin{itemize}
\item
  birches (4)
\item
  black alder (6)
\item
  aspens (8, 19, 68)
\item
  grey alder (9)
\item
  willows (20, 21)
\item
  rowan (32)
\item
  eve (35)
\end{itemize}

\textbf{temperate deciduous} - following species (codes) as used in the national forest
stand-level-inventory database:

\begin{itemize}
\item
  oaks (10, 61)
\item
  ashes (11, 64)
\item
  lindens (12, 62)
\item
  elms (16, 65)
\item
  beech (17)
\item
  hornbeam (18)
\item
  maples (24, 63)
\item
  cherry (25)
\item
  apple (26)
\item
  pear (27)
\item
  yew (29)
\item
  acacia (50)
\item
  walnut (66)
\item
  chestnut (67)
\item
  robinia (69)
\end{itemize}

Forest stand \textbf{age groups} \emph{(vgr)} as used in the national forest
stand-level-inventory database:

\begin{itemize}
\item
  young stands (vgr = 1) in coniferous trees, ashes and oaks - until 40 years,
  in grey alder - until 10 years, in other tree species - until 20 years;
\item
  medium aged stands (vgr = 2 or vgr = 3) are between young stands (vgr = 1) and legal rotation age;
\item
  old stands (vgr = 4 or vgr = 5) are stands exceeding legal rotation age. This is
  defined in \href{https://likumi.lv/ta/id/2825\#p9}{by law} based on tree species and
  site quality class (bonity). Generally for oaks, pines and larches it is 101 or 121 years,
  for spruces, ashes, limes, elms and maples it is 81 years, for birches it is 71 or 51 years,
  for black alder it is 71 years, for aspens it is 41 years. Currently, there is no minimum
  rotation age in grey alder. We used 35 years, as it is the age of the youngest
  stand registered as ``full grown'' in the databse. This was necessary for the
  harmonization of EGVs throughout forests.
\end{itemize}

Acronyms:

\textbf{CRS} - coordinate reference system

\textbf{DW} - \hyperref[Ch04.08]{Dynamic World}

\textbf{EDI} - Institute of Electronics and Computer Sciences

\textbf{EGV} - ecogeographical variables

\textbf{GEE} - Google Earth Engine

\textbf{SDM} - species distribution modelling

\textbf{SDMs} - species distribution models

\textbf{LAD} - Rural Support Service

\textbf{LĢIA} - Latvian Geospatial Information Agency

\textbf{LULC} - Land use and land cover

\textbf{LU} - University of Latvia

\textbf{LU ĢZZF} - University of Latvia Faculty of Geography and Earth Sciences

\textbf{LVM} - state owned Joint Stock Company ``Latvia's State Forests''

\textbf{LVMI Silava} - Latvian State Forest Research Institute ``Silava''

\textbf{NDMI} - normalized difference moisture index

\textbf{NDVI} - normalized difference vegetation index

\textbf{NDWI} - normalized difference water index

\textbf{MVR} - State Forest Service's stand level inventory database ``Forest State Registry''

\textbf{VMD} - State Forest Service

\chapter{Utilities}\label{Ch02}

This chapter provides a brief description of the utility functions used in this
material. Most of these functions are available in the R package \{egvtools\}, which
was created specifically for this work.

\section{R package egvtools}\label{Ch02.01}

\{egvtools\} provides a coherent set of wrappers and utilities that facilitate the
reproducible and efficient creation of large-scale EGVs on real datasets. The
package relies on robust building blocks --- \{terra\}, \{sf\}, \{sfarrow\}, \{exactextractr\}
and \{whitebox\} --- and standardises input/output data, naming conventions and multi-scale
zonal statistics, ensuring that the pipelines are repeatable across machines and
projects.

The package was developed for the project `HiQBioDiv: High-resolution
quantification of biodiversity for conservation and management', which was
funded by the Latvian Council of Science (Ref. No.~VPP-VARAM-DABA-2024/1-0002),
to simplify our work and to facilitate the reproduction of our results. Five of
the functions are strictly for replication, while others are useful for a wider
audience.

Package can be installed from \href{https://github.com/aavotins/egvtools}{GitHub} with:

\begin{Shaded}
\begin{Highlighting}[]
\CommentTok{\# install.packages("pak")}
\NormalTok{pak}\SpecialCharTok{::}\FunctionTok{pak}\NormalTok{(}\StringTok{"aavotins/egvtools"}\NormalTok{)}
\end{Highlighting}
\end{Shaded}

or obtained as a \href{https://hub.docker.com/repository/docker/aavotins/hiqbiodiv-container/general}{Docker container} with all the necessary system and software dependencies.

\subsection{Reproduction only functions}\label{Ch02.01.01}

These functions are small wrappers, that help to recreate our
working environments - template files and their locations in the file tree.

These functions are:

\begin{itemize}
\item
  \href{https://aavotins.github.io/egvtools/reference/download_raster_templates.html}{\texttt{download\_raster\_templates()}} --- fetch template rasters from Zenodo repository
  and place them in a user specified location on the disk, or by default - the
  place we used. By default this function links
  to the \href{https://zenodo.org/records/14497070}{version 2.0.0} of the dataset;
\item
  \href{https://aavotins.github.io/egvtools/reference/download_vector_templates.html}{\texttt{download\_vector\_templates()}} - fetch template vector grids/points from Zenodo
  repository and place them in a user specified location on the disk, or by default - the
  place we used. By default this function links
  to the \href{https://zenodo.org/records/14277114}{version 1.0.1} of the dataset;
\item
  \href{https://aavotins.github.io/egvtools/reference/radius_function.html}{radius\_function()} --- extracts
  summary statistics from raster layers using buffered polygon zones of multiple
  radii and rasterises them onto a common template grid. Internally hard coded to
  use filenames (first and second part in the result of tiling functions) as used
  in this project. If the filenames are kept, function can easily be used for
  other projects, regions etc. Function can be used to run sequentially, however much
  faster compute will be with parallel computing. If fast swap disk is available,
  this function needs only c.a. 5 GiBs of RAM per worker to perform tasks in this
  project. However, if the swap disk is not available, at least 20 GiBs of RAM per
  worker need to be assigned.
\end{itemize}

\subsection{General purpose functions}\label{Ch02.01.02}

Each of those functions are small workflows themselves that can be combined
into larger workflows and used more widely than for Latvia.

\begin{itemize}
\item
  \href{https://aavotins.github.io/egvtools/reference/tile_vector_grid.html}{\texttt{tile\_vector\_grid()}} --- tile template (vector) grid for chunked processing. The function internally is linked to our file naming
  convention. As long as it is maintained, function can be used to create tiled grid
  from any \{sfarrow\} parquet grid file;
\item
  \href{https://aavotins.github.io/egvtools/reference/tiled_buffers.html}{\texttt{tiled\_buffers()}} ---
  precompute buffered tiles for multiple radii around
  points. The function internally is linked to our file naming
  convention. As long as it is maintained, function can be used to create tiled
  polygons with buffers around points from any \{sfarrow\} parquet grid file. There
  are three buffering modes: \textbf{dense} (buffers the best-matching pts100*.parquet
  (prefers pts100\_sauzeme.parquet) for each tile by radii\_dense (default: 500,
  1250, 3000, 10000 m ensuring that every analysis grid cell has desired buffer.
  Computationally heavy in the following workflows), \textbf{sparse} (uses a file to
  radius mapping and is highly generalizable),
  and \textbf{specified} (the same as sparse, but with one single
  point file). \textbf{In our workflows we used the sparse mode with default mapping};
\item
  \href{https://aavotins.github.io/egvtools/reference/create_backgrounds.html}{create\_backgrounds()} --- a wrapper
  around \texttt{terra::ifel()} to build consistent background rasters. This function better
  guards coordinate reference system and how it is stored, while also guarding
  spatial cover, resolution, coordinate reference system, exact pixel matching, etc.
  Creation of layers with default background values is faster than recreating them
  several times in workflows preparing EGVs;
\item
  \href{}{polygon2input()} --- rasterise polygons to input layers. Handles only polygon data,
  other geometry types need to be buffered. Rasterizes polygon/multipolygon sf data to
  a raster aligned to a template GeoTIFF. Rasterization targets a raster::RasterLayer
  built from the template (so grids normally match). Projection is optional
  (project\_mode). Missing values are counted only over valid template cells. User
  may optionally restrict the result with a raster mask (restrict\_to) using numeric
  values or bracketed range strings (e.g., ``(0,5{]}'', ``{[}10,)''). Remaining NA cells
  can be filled by covering with a background raster (background\_raster) or a
  constant (background\_value). For large rasters, heavy steps (projection/mask/cover)
  can stream to disk via terra\_todisk=TRUE.
\item
  \href{https://aavotins.github.io/egvtools/reference/input2egv.html}{input2egv()} --- normalize/align
  a fine-resolution input raster to a (coarser) EGV template, optionally cover missing values and/or fill gaps (IDW via Whitebox), and write the result to disk. Designed for large runs: fast gap counting (inside template footprint only), optional filling, tuned GDAL write options, and controlled terra memory/temp behavior.
\item
  \href{}{downscale2egv()} --- downscale coarse rasters to a template grid (CRS,
  resolution, extent), masks to the template footprint, and optionally: (1) fills
  NoData gaps using WhiteboxTools' IDW-based fill\_missing\_data, and (2) applies
  IDW smoothing to reduce blockiness from low-resolution inputs.
\item
  \href{https://aavotins.github.io/egvtools/reference/distance2egv.html}{distance2egv()} --- computes
  Euclidean distance (in map units) from cells matching a set of class values in
  an input raster to all cells of an EGV template grid, then writes a Float32
  GeoTIFF aligned to the template. Designed to work with rasters produced
  by \texttt{polygon2input()}.
\item
  \href{https://aavotins.github.io/egvtools/reference/landscape_function.html}{landscape\_function()} --- computes a \{landscapemetrics\} metric (default ``lsm\_l\_shdi''), optionally with extra lm\_args,
  that yields one value per zone and per input layer. Runs tile-by-tile (by
  tile\_field), writes per-tile rasters, merges to final per-layer GeoTIFF(s),
  then performs gap analysis (NA count within the template footprint and optional
  maximum gap width) and optional IDW gap filling via WhiteboxTools. Returns a
  compact \texttt{data.frame} with per-layer stats and timing. Function can be used to run
  sequentially, however much faster compute will be with parallel computing. If fast
  swap disk is available, this function needs only 3 GiBs of RAM per worked
  to perform tasks in this project. However, if the swap disk is not available, at
  least 20 GiBs of RAM per worker need to be assigned.
\end{itemize}

\section{Other utility functions}\label{Ch02.02}

Other handy functions repeatedly used, not included in \{egvtools\} are stored
in \texttt{egvs02.02\_UtilityFunctions.R} file, located in \texttt{Data/RScipts\_final}.

\begin{itemize}
\tightlist
\item
  \texttt{ensure\_multipolygons()} - rather agressive function to
  create \texttt{MULTIPOLYGON} geometries from \texttt{GEOMETRYCOLLECTION}
\end{itemize}

\begin{Shaded}
\begin{Highlighting}[]
\ControlFlowTok{if}\NormalTok{(}\SpecialCharTok{!}\FunctionTok{require}\NormalTok{(sf)) \{}\FunctionTok{install.packages}\NormalTok{(}\StringTok{"sf"}\NormalTok{); }\FunctionTok{require}\NormalTok{(sf)\}}
\ControlFlowTok{if}\NormalTok{(}\SpecialCharTok{!}\FunctionTok{require}\NormalTok{(gdalUtilities)) \{}\FunctionTok{install.packages}\NormalTok{(}\StringTok{"gdalUtilities"}\NormalTok{); }\FunctionTok{require}\NormalTok{(gdalUtilities)\}}

\NormalTok{ensure\_multipolygons }\OtherTok{\textless{}{-}} \ControlFlowTok{function}\NormalTok{(X) \{}
  \FunctionTok{library}\NormalTok{(sf)}
  \FunctionTok{library}\NormalTok{(gdalUtilities)}
  
\NormalTok{  tmp1 }\OtherTok{\textless{}{-}} \FunctionTok{tempfile}\NormalTok{(}\AttributeTok{fileext =} \StringTok{".gpkg"}\NormalTok{)}
\NormalTok{  tmp2 }\OtherTok{\textless{}{-}} \FunctionTok{tempfile}\NormalTok{(}\AttributeTok{fileext =} \StringTok{".gpkg"}\NormalTok{)}
  \FunctionTok{st\_write}\NormalTok{(X, tmp1)}
  \FunctionTok{ogr2ogr}\NormalTok{(tmp1, tmp2, }\AttributeTok{f =} \StringTok{"GPKG"}\NormalTok{, }\AttributeTok{nlt =} \StringTok{"MULTIPOLYGON"}\NormalTok{)}
\NormalTok{  Y }\OtherTok{\textless{}{-}} \FunctionTok{st\_read}\NormalTok{(tmp2)}
  \FunctionTok{st\_sf}\NormalTok{(}\FunctionTok{st\_drop\_geometry}\NormalTok{(X), }\AttributeTok{geom =} \FunctionTok{st\_geometry}\NormalTok{(Y))}
\NormalTok{\}}
\end{Highlighting}
\end{Shaded}

\chapter{Templates files}\label{Ch03}

This chapter defines template files. They define the analysis space and ensure
harmonisation of georeferenced data creation, and facilitate connection with
other Latvian geodata.

\section{Vector data}\label{Ch03.01}

Baseline template (or reference) vector grid and point files are publiclly available
in the \href{https://zenodo.org/records/14277114}{HiQBioDiv's Zenodo repository}. The command lines
and data used to create these files are documented in
the HiQBioDiv main code repository's \href{https://github.com/aavotins/HiQBioDiv/blob/main/Templates/TemplateGrids_Vector.R}{file}.

The easiest way to obtain these files is to run determined
function \texttt{download\_vector\_templates()} from \{egvtools\}.

\begin{Shaded}
\begin{Highlighting}[]
\ControlFlowTok{if}\NormalTok{(}\SpecialCharTok{!}\FunctionTok{require}\NormalTok{(egvtools)) \{remotes}\SpecialCharTok{::}\FunctionTok{install\_github}\NormalTok{(}\StringTok{"aavotins/egvtools"}\NormalTok{); }\FunctionTok{require}\NormalTok{(egvtools)\}}

\FunctionTok{download\_vector\_templates}\NormalTok{(}
  \AttributeTok{url =} \StringTok{"https://zenodo.org/api/records/14277114/files{-}archive"}\NormalTok{,}
  \AttributeTok{grid\_dir =} \StringTok{"./Templates/TemplateGrids"}\NormalTok{,}
  \AttributeTok{points\_dir =} \StringTok{"./Templates/TemplateGridPoints"}\NormalTok{,}
  \AttributeTok{gpkg\_dir =} \StringTok{"./Templates"}\NormalTok{,}
  \AttributeTok{overwrite =} \ConstantTok{FALSE}\NormalTok{,}
  \AttributeTok{quiet =} \ConstantTok{FALSE}
\NormalTok{)}
\end{Highlighting}
\end{Shaded}

Once template vector data are downloaded and unarchived, they need to be tiled:

\begin{enumerate}
\def\labelenumi{\arabic{enumi}.}
\tightlist
\item
  Analysis grid is tiled in \texttt{tks50km} pages
\end{enumerate}

\begin{Shaded}
\begin{Highlighting}[]
\ControlFlowTok{if}\NormalTok{(}\SpecialCharTok{!}\FunctionTok{require}\NormalTok{(egvtools)) \{remotes}\SpecialCharTok{::}\FunctionTok{install\_github}\NormalTok{(}\StringTok{"aavotins/egvtools"}\NormalTok{); }\FunctionTok{require}\NormalTok{(egvtools)\}}

\FunctionTok{tile\_vector\_grid}\NormalTok{(}
  \AttributeTok{grid\_path =} \StringTok{"./Templates/TemplateGrids/tikls100\_sauzeme.parquet"}\NormalTok{,}
  \AttributeTok{out\_dir =} \StringTok{"./Templates/TemplateGrids/tiles"}\NormalTok{,}
  \AttributeTok{tile\_field =} \StringTok{"tks50km"}\NormalTok{,}
  \AttributeTok{chunk\_size =} \DecValTok{50000}\DataTypeTok{L}\NormalTok{,}
  \AttributeTok{overwrite =} \ConstantTok{FALSE}\NormalTok{,}
  \AttributeTok{quiet =} \ConstantTok{FALSE}
\NormalTok{)}
\end{Highlighting}
\end{Shaded}

Expect to see warning:
\texttt{This\ is\ an\ initial\ implementation\ of\ Parquet/Feather\ file\ support\ and\ geo\ metadata.\ This\ is\ tracking\ version\ 0.1.0\ of\ the\ metadata\ (https://github.com/geopandas/geo-arrow-spec).\ This\ metadata\ specification\ may\ change\ and\ does\ not\ yet\ make\ stability\ promises.\ \ We\ do\ not\ yet\ recommend\ using\ this\ in\ a\ production\ setting\ unless\ you\ are\ able\ to\ rewrite\ your\ Parquet/Feather\ files.}

\begin{enumerate}
\def\labelenumi{\arabic{enumi}.}
\setcounter{enumi}{1}
\item
  Point files are tiled and buffered. In the workflows creating EGVs described in this document,
  we used a ``sparse'' grid:

  \begin{itemize}
  \item
    500m buffers around every 100m grid cell's centre;
  \item
    1250m buffers around every 100m grid cell's centre;
  \item
    3000m buffers around every 300m grid cell's centre (to speed up neighbourhood analysis \textasciitilde9 times, while loosing \textless0.001\% of precission);
  \item
    10000m buffers around every 1000m grid cell's centre (to speed up neighbourhood analysis \textasciitilde100 times, while loosing \textless0.001\% of precission)
  \end{itemize}
\end{enumerate}

\begin{Shaded}
\begin{Highlighting}[]
\ControlFlowTok{if}\NormalTok{(}\SpecialCharTok{!}\FunctionTok{require}\NormalTok{(egvtools)) \{remotes}\SpecialCharTok{::}\FunctionTok{install\_github}\NormalTok{(}\StringTok{"aavotins/egvtools"}\NormalTok{); }\FunctionTok{require}\NormalTok{(egvtools)\}}

\FunctionTok{tiled\_buffers}\NormalTok{(}
  \AttributeTok{in\_dir =} \StringTok{"./Templates/TemplateGridPoints"}\NormalTok{,}
  \AttributeTok{out\_dir =} \StringTok{"./Templates/TemplateGridPoints/tiles"}\NormalTok{,}
  \AttributeTok{buffer\_mode =} \StringTok{"sparse"}\NormalTok{,}
  \AttributeTok{mapping\_sparse =} \FunctionTok{list}\NormalTok{(}\StringTok{"pts100\_sauzeme.parquet"} \OtherTok{=} \FunctionTok{c}\NormalTok{(}\DecValTok{500}\NormalTok{, }\DecValTok{1250}\NormalTok{), }
                        \StringTok{"pts300\_sauzeme.parquet"} \OtherTok{=} \DecValTok{3000}\NormalTok{, }
                        \StringTok{"pts1000\_sauzeme.parquet"} \OtherTok{=} \DecValTok{10000}\NormalTok{),}
  \AttributeTok{split\_field =} \StringTok{"tks50km"}\NormalTok{,}
  \AttributeTok{n\_workers =} \DecValTok{4}\NormalTok{,}
  \AttributeTok{future\_max\_mem\_gb =} \DecValTok{4}\NormalTok{,}
  \AttributeTok{overwrite =} \ConstantTok{FALSE}\NormalTok{,}
  \AttributeTok{quiet =} \ConstantTok{FALSE}
\NormalTok{)}
\end{Highlighting}
\end{Shaded}

Expect to see warning:
\texttt{This\ is\ an\ initial\ implementation\ of\ Parquet/Feather\ file\ support\ and\ geo\ metadata.\ This\ is\ tracking\ version\ 0.1.0\ of\ the\ metadata\ (https://github.com/geopandas/geo-arrow-spec).\ This\ metadata\ specification\ may\ change\ and\ does\ not\ yet\ make\ stability\ promises.\ \ We\ do\ not\ yet\ recommend\ using\ this\ in\ a\ production\ setting\ unless\ you\ are\ able\ to\ rewrite\ your\ Parquet/Feather\ files.}

Appearance of file \texttt{pts300\_r3000\_NA.parquet}, i.e.~without a tile number, is expected,
due to slight mismatch of 300 m grid with the 50 km one.

\section{Raster data}\label{Ch03.02}

Baseline template (or reference) raster grid and point files are publically available
in the \href{https://zenodo.org/records/14497070}{HiQBioDiv's Zenodo repository}. The command lines
and data used to create these files are documented in
the HiQBioDiv main code repository's \href{https://github.com/aavotins/HiQBioDiv/blob/main/Templates/TemplateGrids_Raster.R}{file}.

The easiest way to obtain these files is to run determined
function \texttt{download\_raster\_templates()} from \{egvtools\}.

\begin{Shaded}
\begin{Highlighting}[]
\ControlFlowTok{if}\NormalTok{(}\SpecialCharTok{!}\FunctionTok{require}\NormalTok{(egvtools)) \{remotes}\SpecialCharTok{::}\FunctionTok{install\_github}\NormalTok{(}\StringTok{"aavotins/egvtools"}\NormalTok{); }\FunctionTok{require}\NormalTok{(egvtools)\}}

\FunctionTok{download\_raster\_templates}\NormalTok{(}
  \AttributeTok{url =} \StringTok{"https://zenodo.org/api/records/14497070/files{-}archive"}\NormalTok{,}
  \AttributeTok{out\_dir =} \StringTok{"./Templates/TemplateRasters"}\NormalTok{,}
  \AttributeTok{overwrite =} \ConstantTok{TRUE}\NormalTok{,}
  \AttributeTok{quiet =} \ConstantTok{FALSE}
\NormalTok{)}
\end{Highlighting}
\end{Shaded}

During EGV creation, background filling to handle missing values may be
necessary. For all EGVs described in this document where such an exercise might
be required, the variables can be considered quantities of ratio scale, therefore backgrounds
with value \texttt{0} are created.

\begin{Shaded}
\begin{Highlighting}[]
\ControlFlowTok{if}\NormalTok{(}\SpecialCharTok{!}\FunctionTok{require}\NormalTok{(egvtools)) \{remotes}\SpecialCharTok{::}\FunctionTok{install\_github}\NormalTok{(}\StringTok{"aavotins/egvtools"}\NormalTok{); }\FunctionTok{require}\NormalTok{(egvtools)\}}

\FunctionTok{create\_backgrounds}\NormalTok{(}\AttributeTok{in\_dir=}\StringTok{"./Templates/TemplateRasters/"}\NormalTok{,}
                   \AttributeTok{out\_dir =} \StringTok{"./Templates/TemplateRasters/"}\NormalTok{,}
                   \AttributeTok{background\_value =} \DecValTok{0}\NormalTok{,}
                   \AttributeTok{out\_prefix =} \StringTok{"nulls\_"}\NormalTok{,}
                   \AttributeTok{overwrite=}\ConstantTok{TRUE}\NormalTok{)}
\end{Highlighting}
\end{Shaded}

\chapter{Raw geodata}\label{Ch04}

This chapter describes raw geodata used and the preliminary processing conducted on them.

\section{State Forest Service's State Forest Register}\label{Ch04.01}

The State Forest Service's State Forest Register database (ESRI file geodatabase),
which compiles indicators and spatial data characterizing forest compartments
(stand level inventory database), was received by the University of
Latvia on January 7, 2024, to support study and research processes. The structure
of the received database version corresponds to
the \href{https://www.vmd.gov.lv/lv/meza-valsts-registra-meza-inventarizacijas-failu-struktura}{State Forest Register Forest Inventory File Structure}, but
lowercase letters are used in field names.

After downloading, the CRS is guarded, geometries are checked and saved in
GeoParquet format.

Files are stored at \texttt{Geodata/2024/MVR/}.

\begin{Shaded}
\begin{Highlighting}[]
\CommentTok{\# libs}
\ControlFlowTok{if}\NormalTok{(}\SpecialCharTok{!}\FunctionTok{require}\NormalTok{(sf)) \{}\FunctionTok{install.packages}\NormalTok{(}\StringTok{"sf"}\NormalTok{); }\FunctionTok{require}\NormalTok{(sf)\}}
\ControlFlowTok{if}\NormalTok{(}\SpecialCharTok{!}\FunctionTok{require}\NormalTok{(arrow)) \{}\FunctionTok{install.packages}\NormalTok{(}\StringTok{"arrow"}\NormalTok{); }\FunctionTok{require}\NormalTok{(arrow)\}}
\ControlFlowTok{if}\NormalTok{(}\SpecialCharTok{!}\FunctionTok{require}\NormalTok{(sfarrow)) \{}\FunctionTok{install.packages}\NormalTok{(}\StringTok{"sfarrow"}\NormalTok{); }\FunctionTok{require}\NormalTok{(sfarrow)\}}
\ControlFlowTok{if}\NormalTok{(}\SpecialCharTok{!}\FunctionTok{require}\NormalTok{(gdalUtilities)) \{}\FunctionTok{install.packages}\NormalTok{(}\StringTok{"gdalUtilities"}\NormalTok{); }\FunctionTok{require}\NormalTok{(gdalUtilities)\}}

\CommentTok{\# database}
\NormalTok{nog}\OtherTok{=}\FunctionTok{read\_sf}\NormalTok{(}\StringTok{"./Geodata/2024/MVR/VMD.gdb/"}\NormalTok{,}\AttributeTok{layer=}\StringTok{"Nogabali\_pilna\_datubaze"}\NormalTok{)}

\CommentTok{\# ensuring geometries}
\FunctionTok{source}\NormalTok{(}\StringTok{"./RScripts\_final/egvs02.02\_UtilityFunctions.R"}\NormalTok{)}
\NormalTok{nogabali }\OtherTok{\textless{}{-}} \FunctionTok{ensure\_multipolygons}\NormalTok{(nog)}

\CommentTok{\# securing geometries}
\NormalTok{nogabali2 }\OtherTok{=}\NormalTok{ nogabali[}\SpecialCharTok{!}\FunctionTok{st\_is\_empty}\NormalTok{(nogabali),,drop}\OtherTok{=}\ConstantTok{FALSE}\NormalTok{] }\CommentTok{\# 108 tukšas ģeometrijas}
\NormalTok{validity}\OtherTok{=}\FunctionTok{st\_is\_valid}\NormalTok{(nogabali2) }
\FunctionTok{table}\NormalTok{(validity) }\CommentTok{\# 1733 invalid ģeometrijas}
\NormalTok{nogabali3}\OtherTok{=}\FunctionTok{st\_make\_valid}\NormalTok{(nogabali2)}

\CommentTok{\# transforming CRS}
\NormalTok{nogabali4}\OtherTok{=}\FunctionTok{st\_transform}\NormalTok{(nogabali3, }\AttributeTok{crs=}\DecValTok{3059}\NormalTok{)}

\CommentTok{\# saving}
\NormalTok{sfarrow}\SpecialCharTok{::}\FunctionTok{st\_write\_parquet}\NormalTok{(nogabali4, }\StringTok{"./Geodata/2024/MVR/nogabali\_2024janv.parquet"}\NormalTok{)}
\end{Highlighting}
\end{Shaded}

\section{Rural Support Service's information on declared fields}\label{Ch04.02}

The Rural Support Service maintains \href{https://data.gov.lv/dati/lv/organization/lad}{regularly updated information on their open
data portal}. An archive (since 2016) is
also available there, and the data of interest contain the keyword ``deklarētās platības''.

After downloading files to \texttt{Geodata/2024/LAD/downloads/}, they are unzipped and read into R.
Files are checked, empty geometries are deleted and the rest are validated. Then, all individual
files are combined into one, which is saved in GeoPackage and GeoParquet formats
at \texttt{Geodata/2024/LAD/}. At the end, downloaded files are unlinked.

\begin{Shaded}
\begin{Highlighting}[]
\CommentTok{\# libs}
\ControlFlowTok{if}\NormalTok{(}\SpecialCharTok{!}\FunctionTok{require}\NormalTok{(sf)) \{}\FunctionTok{install.packages}\NormalTok{(}\StringTok{"sf"}\NormalTok{); }\FunctionTok{require}\NormalTok{(sf)\}}
\ControlFlowTok{if}\NormalTok{(}\SpecialCharTok{!}\FunctionTok{require}\NormalTok{(arrow)) \{}\FunctionTok{install.packages}\NormalTok{(}\StringTok{"arrow"}\NormalTok{); }\FunctionTok{require}\NormalTok{(arrow)\}}
\ControlFlowTok{if}\NormalTok{(}\SpecialCharTok{!}\FunctionTok{require}\NormalTok{(sfarrow)) \{}\FunctionTok{install.packages}\NormalTok{(}\StringTok{"sfarrow"}\NormalTok{); }\FunctionTok{require}\NormalTok{(sfarrow)\}}
\ControlFlowTok{if}\NormalTok{(}\SpecialCharTok{!}\FunctionTok{require}\NormalTok{(gdalUtilities)) \{}\FunctionTok{install.packages}\NormalTok{(}\StringTok{"gdalUtilities"}\NormalTok{); }\FunctionTok{require}\NormalTok{(gdalUtilities)\}}

\CommentTok{\# reading all files}
\NormalTok{faili}\OtherTok{=}\FunctionTok{data.frame}\NormalTok{(}\AttributeTok{celi=}\FunctionTok{list.files}\NormalTok{(}\StringTok{"./Geodata/2024/LAD/downloads"}\NormalTok{,}\AttributeTok{full.names =} \ConstantTok{TRUE}\NormalTok{))}
\NormalTok{dati}\OtherTok{=}\FunctionTok{st\_read}\NormalTok{(faili}\SpecialCharTok{$}\NormalTok{celi[}\DecValTok{1}\NormalTok{])}
\ControlFlowTok{for}\NormalTok{(i }\ControlFlowTok{in} \DecValTok{2}\SpecialCharTok{:}\FunctionTok{length}\NormalTok{(faili}\SpecialCharTok{$}\NormalTok{celi))\{}
\NormalTok{  nakosais}\OtherTok{=}\FunctionTok{st\_read}\NormalTok{(faili}\SpecialCharTok{$}\NormalTok{celi[i])}
\NormalTok{  dati}\OtherTok{=}\FunctionTok{bind\_rows}\NormalTok{(dati,nakosais)}
  \FunctionTok{print}\NormalTok{(}\FunctionTok{nrow}\NormalTok{(dati))}
\NormalTok{\}}

\CommentTok{\# ensuring geometries}
\FunctionTok{source}\NormalTok{(}\StringTok{"./RScripts\_final/egvs02.02\_UtilityFunctions.R"}\NormalTok{)}
\NormalTok{nogabali }\OtherTok{\textless{}{-}} \FunctionTok{ensure\_multipolygons}\NormalTok{(nog)}
\NormalTok{dati2 }\OtherTok{\textless{}{-}} \FunctionTok{ensure\_multipolygons}\NormalTok{(dati)}
\NormalTok{dati3 }\OtherTok{=}\NormalTok{ dati2[}\SpecialCharTok{!}\FunctionTok{st\_is\_empty}\NormalTok{(dati2),,drop}\OtherTok{=}\ConstantTok{FALSE}\NormalTok{] }\CommentTok{\# viss kārtībā}
\FunctionTok{table}\NormalTok{(}\FunctionTok{st\_is\_valid}\NormalTok{(dati3)) }
\NormalTok{dati4}\OtherTok{=}\FunctionTok{st\_make\_valid}\NormalTok{(dati3)}
\FunctionTok{table}\NormalTok{(}\FunctionTok{st\_is\_valid}\NormalTok{(dati4))}
\NormalTok{dati5 }\OtherTok{\textless{}{-}} \FunctionTok{ensure\_multipolygons}\NormalTok{(dati4)}
\FunctionTok{table}\NormalTok{(}\FunctionTok{st\_is\_valid}\NormalTok{(dati5))}

\CommentTok{\# saving output}
\FunctionTok{st\_write}\NormalTok{(dati5,}\StringTok{"./Geodata/2024/LAD/Lauki\_2024.gpkg"}\NormalTok{,}\AttributeTok{append =} \ConstantTok{FALSE}\NormalTok{)}
\NormalTok{sfarrow}\SpecialCharTok{::}\FunctionTok{st\_write\_parquet}\NormalTok{(dati5,}\StringTok{"./Geodata/2024/LAD/Lauki\_2024.parquet"}\NormalTok{)}

\CommentTok{\# unlinking downloads}
\ControlFlowTok{for}\NormalTok{(i }\ControlFlowTok{in} \FunctionTok{seq\_along}\NormalTok{(faili}\SpecialCharTok{$}\NormalTok{celi))\{}
  \FunctionTok{unlink}\NormalTok{(faili}\SpecialCharTok{$}\NormalTok{celi[i])}
\NormalTok{\}}
\FunctionTok{rm}\NormalTok{(}\AttributeTok{list=}\FunctionTok{ls}\NormalTok{())}
\end{Highlighting}
\end{Shaded}

\section{Melioration Cadaster}\label{Ch04.03}

The Land Improvement Cadastre Information System database was downloaded layer
by layer from Geoserver. Geometries were tested and validated for each layer, and
layers were all combined into a single GeoPackage file stored at \texttt{Geodata/2024/MKIS/}.

Initially, no additional processing was performed on this data. It was used to
prepare \hyperref[Ch05]{Geodata products} - both \hyperref[Ch05.01]{Terrain products} and \hyperref[Ch05.03]{Landscape classification}.

\begin{Shaded}
\begin{Highlighting}[]
\CommentTok{\# libs}
\ControlFlowTok{if}\NormalTok{(}\SpecialCharTok{!}\FunctionTok{require}\NormalTok{(sf)) \{}\FunctionTok{install.packages}\NormalTok{(}\StringTok{"sf"}\NormalTok{); }\FunctionTok{require}\NormalTok{(sf)\}}
\ControlFlowTok{if}\NormalTok{(}\SpecialCharTok{!}\FunctionTok{require}\NormalTok{(tidyverse)) \{}\FunctionTok{install.packages}\NormalTok{(}\StringTok{"tidyverse"}\NormalTok{); }\FunctionTok{require}\NormalTok{(tidyverse)\}}
\ControlFlowTok{if}\NormalTok{(}\SpecialCharTok{!}\FunctionTok{require}\NormalTok{(httr)) \{}\FunctionTok{install.packages}\NormalTok{(}\StringTok{"httr"}\NormalTok{); }\FunctionTok{require}\NormalTok{(httr)\}}
\ControlFlowTok{if}\NormalTok{(}\SpecialCharTok{!}\FunctionTok{require}\NormalTok{(ows4R)) \{}\FunctionTok{install.packages}\NormalTok{(}\StringTok{"ows4R"}\NormalTok{); }\FunctionTok{require}\NormalTok{(ows4R)\}}

\CommentTok{\# basis information {-}{-}{-}{-}}
\NormalTok{link}\OtherTok{=}\StringTok{"https://lvmgeoserver.lvm.lv/geoserver/zmni/ows?"}
\NormalTok{url}\OtherTok{=}\FunctionTok{parse\_url}\NormalTok{(link)}
\NormalTok{url}\SpecialCharTok{$}\NormalTok{query }\OtherTok{\textless{}{-}} \FunctionTok{list}\NormalTok{(}\AttributeTok{service =} \StringTok{"wfs"}\NormalTok{,}
                  \CommentTok{\#version = "2.0.0", \# facultative}
                  \AttributeTok{request =} \StringTok{"GetCapabilities"}\NormalTok{)}
\NormalTok{request }\OtherTok{\textless{}{-}} \FunctionTok{build\_url}\NormalTok{(url)}
\NormalTok{request}
\NormalTok{bwk\_client }\OtherTok{\textless{}{-}}\NormalTok{ WFSClient}\SpecialCharTok{$}\FunctionTok{new}\NormalTok{(link, }
                            \AttributeTok{serviceVersion =} \StringTok{"2.0.0"}\NormalTok{)}
\NormalTok{bwk\_client}
\NormalTok{bwk\_client}\SpecialCharTok{$}\FunctionTok{getFeatureTypes}\NormalTok{(}\AttributeTok{pretty =} \ConstantTok{TRUE}\NormalTok{)}


\CommentTok{\# dams {-}{-}{-}{-}}

\NormalTok{bwk\_client}\SpecialCharTok{$}\FunctionTok{getFeatureTypes}\NormalTok{(}\AttributeTok{pretty =} \ConstantTok{TRUE}\NormalTok{)}
\NormalTok{url}\SpecialCharTok{$}\NormalTok{query }\OtherTok{\textless{}{-}} \FunctionTok{list}\NormalTok{(}\AttributeTok{service =} \StringTok{"wfs"}\NormalTok{,}
                  \AttributeTok{request =} \StringTok{"GetFeature"}\NormalTok{,}
                  \AttributeTok{srsName=}\StringTok{"EPSG:3059"}\NormalTok{,}
                  \AttributeTok{typename =} \StringTok{"zmni:zmni\_dam"}\NormalTok{)}
\NormalTok{request }\OtherTok{\textless{}{-}} \FunctionTok{build\_url}\NormalTok{(url)}
\NormalTok{aizsargdambji }\OtherTok{\textless{}{-}} \FunctionTok{read\_sf}\NormalTok{(request)}
\NormalTok{aizsargdambji }\OtherTok{=}\NormalTok{ aizsargdambji }\SpecialCharTok{\%\textgreater{}\%} \FunctionTok{st\_set\_crs}\NormalTok{(}\FunctionTok{st\_crs}\NormalTok{(}\DecValTok{3059}\NormalTok{))}
\NormalTok{aizsargdambji}\OtherTok{=}\FunctionTok{st\_cast}\NormalTok{(aizsargdambji,}\StringTok{"MULTILINESTRING"}\NormalTok{)}

\FunctionTok{ggplot}\NormalTok{(aizsargdambji)}\SpecialCharTok{+}\FunctionTok{geom\_sf}\NormalTok{()}

\FunctionTok{table}\NormalTok{(}\FunctionTok{st\_is\_valid}\NormalTok{(aizsargdambji))}

\FunctionTok{write\_sf}\NormalTok{(aizsargdambji,}
         \StringTok{"./Geodata/2024/MKIS/MKIS\_2025.gpkg"}\NormalTok{,}
         \AttributeTok{layer=}\StringTok{"Aizsargdambji"}\NormalTok{,}
         \AttributeTok{append=}\ConstantTok{FALSE}\NormalTok{)}
\FunctionTok{rm}\NormalTok{(aizsargdambji)}

\CommentTok{\# watercourses {-}{-}{-}{-}}

\NormalTok{bwk\_client}\SpecialCharTok{$}\FunctionTok{getFeatureTypes}\NormalTok{(}\AttributeTok{pretty =} \ConstantTok{TRUE}\NormalTok{)}
\NormalTok{url}\SpecialCharTok{$}\NormalTok{query }\OtherTok{\textless{}{-}} \FunctionTok{list}\NormalTok{(}\AttributeTok{service =} \StringTok{"wfs"}\NormalTok{,}
                  \AttributeTok{request =} \StringTok{"GetFeature"}\NormalTok{,}
                  \AttributeTok{srsName=}\StringTok{"EPSG:3059"}\NormalTok{,}
                  \AttributeTok{typename =} \StringTok{"zmni:zmni\_watercourses"}\NormalTok{)}
\NormalTok{request }\OtherTok{\textless{}{-}} \FunctionTok{build\_url}\NormalTok{(url)}

\NormalTok{DabiskasUdensteces }\OtherTok{\textless{}{-}} \FunctionTok{read\_sf}\NormalTok{(request)}
\NormalTok{DabiskasUdensteces }\OtherTok{=}\NormalTok{ DabiskasUdensteces }\SpecialCharTok{\%\textgreater{}\%} \FunctionTok{st\_set\_crs}\NormalTok{(}\FunctionTok{st\_crs}\NormalTok{(}\DecValTok{3059}\NormalTok{))}
\NormalTok{DabiskasUdensteces}\OtherTok{=}\FunctionTok{st\_cast}\NormalTok{(DabiskasUdensteces,}\StringTok{"MULTILINESTRING"}\NormalTok{)}

\FunctionTok{ggplot}\NormalTok{(DabiskasUdensteces)}\SpecialCharTok{+}\FunctionTok{geom\_sf}\NormalTok{()}

\FunctionTok{table}\NormalTok{(}\FunctionTok{st\_is\_valid}\NormalTok{(DabiskasUdensteces))}

\FunctionTok{write\_sf}\NormalTok{(DabiskasUdensteces,}
         \StringTok{"./Geodata/2024/MKIS/MKIS\_2025.gpkg"}\NormalTok{,}
         \AttributeTok{layer=}\StringTok{"DabiskasUdensteces"}\NormalTok{,}
         \AttributeTok{append=}\ConstantTok{FALSE}\NormalTok{)}
\FunctionTok{rm}\NormalTok{(DabiskasUdensteces)}



\CommentTok{\# dam pickets {-}{-}{-}{-}}


\NormalTok{bwk\_client}\SpecialCharTok{$}\FunctionTok{getFeatureTypes}\NormalTok{(}\AttributeTok{pretty =} \ConstantTok{TRUE}\NormalTok{)}
\NormalTok{url}\SpecialCharTok{$}\NormalTok{query }\OtherTok{\textless{}{-}} \FunctionTok{list}\NormalTok{(}\AttributeTok{service =} \StringTok{"wfs"}\NormalTok{,}
                  \AttributeTok{request =} \StringTok{"GetFeature"}\NormalTok{,}
                  \AttributeTok{srsName=}\StringTok{"EPSG:3059"}\NormalTok{,}
                  \AttributeTok{typename =} \StringTok{"zmni:zmni\_dampicket"}\NormalTok{)}
\NormalTok{request }\OtherTok{\textless{}{-}} \FunctionTok{build\_url}\NormalTok{(url)}

\NormalTok{DambjuPiketi }\OtherTok{\textless{}{-}} \FunctionTok{read\_sf}\NormalTok{(request)}
\NormalTok{DambjuPiketi }\OtherTok{=}\NormalTok{ DambjuPiketi }\SpecialCharTok{\%\textgreater{}\%} \FunctionTok{st\_set\_crs}\NormalTok{(}\FunctionTok{st\_crs}\NormalTok{(}\DecValTok{3059}\NormalTok{))}
\NormalTok{DambjuPiketi}\OtherTok{=}\FunctionTok{st\_cast}\NormalTok{(DambjuPiketi,}\StringTok{"POINT"}\NormalTok{)}

\FunctionTok{ggplot}\NormalTok{(DambjuPiketi)}\SpecialCharTok{+}\FunctionTok{geom\_sf}\NormalTok{()}

\FunctionTok{table}\NormalTok{(}\FunctionTok{st\_is\_valid}\NormalTok{(DambjuPiketi))}

\FunctionTok{write\_sf}\NormalTok{(DambjuPiketi,}
         \StringTok{"./Geodata/2024/MKIS/MKIS\_2025.gpkg"}\NormalTok{,}
         \AttributeTok{layer=}\StringTok{"DambjuPiketi"}\NormalTok{,}
         \AttributeTok{append=}\ConstantTok{FALSE}\NormalTok{)}
\FunctionTok{rm}\NormalTok{(DambjuPiketi)}


\CommentTok{\# drainpipes {-}{-}{-}{-}}

\NormalTok{bwk\_client}\SpecialCharTok{$}\FunctionTok{getFeatureTypes}\NormalTok{(}\AttributeTok{pretty =} \ConstantTok{TRUE}\NormalTok{)}

\NormalTok{base\_url }\OtherTok{\textless{}{-}} \StringTok{"https://lvmgeoserver.lvm.lv/geoserver/zmni/ows?"}
\NormalTok{type\_name }\OtherTok{\textless{}{-}} \StringTok{"zmni:zmni\_drainpipes"}
\NormalTok{crs\_code }\OtherTok{\textless{}{-}} \DecValTok{3059}
\NormalTok{chunk\_size }\OtherTok{\textless{}{-}} \DecValTok{100000}
\NormalTok{gpkg\_path }\OtherTok{\textless{}{-}} \StringTok{"./Geodata/2024/MKIS/temp\_MKIS\_2025.gpkg"}
\NormalTok{layer\_name }\OtherTok{\textless{}{-}} \StringTok{"temp\_Drenas"}
\NormalTok{i }\OtherTok{\textless{}{-}} \DecValTok{0}

\ControlFlowTok{repeat}\NormalTok{ \{}
  \FunctionTok{message}\NormalTok{(}\StringTok{"Fetching features "}\NormalTok{, i }\SpecialCharTok{*}\NormalTok{ chunk\_size }\SpecialCharTok{+} \DecValTok{1}\NormalTok{, }\StringTok{" to "}\NormalTok{, (i }\SpecialCharTok{+} \DecValTok{1}\NormalTok{) }\SpecialCharTok{*}\NormalTok{ chunk\_size, }\StringTok{"..."}\NormalTok{)}
  
\NormalTok{  query }\OtherTok{\textless{}{-}} \FunctionTok{list}\NormalTok{(}
    \AttributeTok{service =} \StringTok{"WFS"}\NormalTok{,}
    \AttributeTok{version =} \StringTok{"2.0.0"}\NormalTok{,}
    \AttributeTok{request =} \StringTok{"GetFeature"}\NormalTok{,}
    \AttributeTok{typename =}\NormalTok{ type\_name,}
    \AttributeTok{srsName =} \FunctionTok{paste0}\NormalTok{(}\StringTok{"EPSG:"}\NormalTok{, crs\_code),}
    \AttributeTok{count =}\NormalTok{ chunk\_size,}
    \AttributeTok{startIndex =}\NormalTok{ i }\SpecialCharTok{*}\NormalTok{ chunk\_size}
\NormalTok{  )}
  
\NormalTok{  req\_url }\OtherTok{\textless{}{-}} \FunctionTok{modify\_url}\NormalTok{(base\_url, }\AttributeTok{query =}\NormalTok{ query)}
  
  \FunctionTok{try}\NormalTok{(\{}
\NormalTok{    chunk }\OtherTok{\textless{}{-}} \FunctionTok{read\_sf}\NormalTok{(req\_url)}
    \ControlFlowTok{if}\NormalTok{ (}\FunctionTok{nrow}\NormalTok{(chunk) }\SpecialCharTok{==} \DecValTok{0}\NormalTok{) }\ControlFlowTok{break}
    
    \CommentTok{\# Set CRS and cast to MULTILINESTRING}
\NormalTok{    chunk }\OtherTok{\textless{}{-}}\NormalTok{ chunk }\SpecialCharTok{\%\textgreater{}\%}
      \FunctionTok{st\_set\_crs}\NormalTok{(}\FunctionTok{st\_crs}\NormalTok{(crs\_code)) }\SpecialCharTok{\%\textgreater{}\%}
      \FunctionTok{st\_cast}\NormalTok{(}\StringTok{"MULTILINESTRING"}\NormalTok{)}
    
    \CommentTok{\# Write chunk to GeoPackage (append mode after first)}
    \FunctionTok{st\_write}\NormalTok{(}
\NormalTok{      chunk, }
      \AttributeTok{dsn =}\NormalTok{ gpkg\_path,}
      \AttributeTok{layer =}\NormalTok{ layer\_name,}
      \AttributeTok{append =}\NormalTok{ i }\SpecialCharTok{!=} \DecValTok{0}\NormalTok{,}
      \AttributeTok{quiet =} \ConstantTok{FALSE}
\NormalTok{    )}
    
\NormalTok{    i }\OtherTok{\textless{}{-}}\NormalTok{ i }\SpecialCharTok{+} \DecValTok{1}
\NormalTok{  \}, }\AttributeTok{silent =} \ConstantTok{TRUE}\NormalTok{)}
\NormalTok{\}}

\FunctionTok{message}\NormalTok{(}\StringTok{"All chunks written to "}\NormalTok{, gpkg\_path, }\StringTok{" in layer "}\NormalTok{, layer\_name)}

\NormalTok{Drenas\_all}\OtherTok{=}\FunctionTok{st\_read}\NormalTok{(}\StringTok{"./Geodata/2024/MKIS/temp\_MKIS\_2025.gpkg"}\NormalTok{,}
                   \AttributeTok{layer=}\StringTok{"temp\_Drenas"}\NormalTok{)}
\NormalTok{Drenas\_all2 }\OtherTok{=}\NormalTok{ Drenas\_all[}\SpecialCharTok{!}\FunctionTok{st\_is\_empty}\NormalTok{(Drenas\_all),,drop}\OtherTok{=}\ConstantTok{FALSE}\NormalTok{] }\CommentTok{\# 1}
\FunctionTok{table}\NormalTok{(}\FunctionTok{st\_is\_valid}\NormalTok{(Drenas\_all2))}


\FunctionTok{write\_sf}\NormalTok{(Drenas\_all2,}
         \StringTok{"./Geodata/2024/MKIS/MKIS\_2025.gpkg"}\NormalTok{,}
         \AttributeTok{layer=}\StringTok{"Drenas"}\NormalTok{,}
         \AttributeTok{append=}\ConstantTok{FALSE}\NormalTok{)}
\FunctionTok{rm}\NormalTok{(}\AttributeTok{list=}\FunctionTok{ls}\NormalTok{())}




\CommentTok{\# drain collectors {-}{-}{-}{-}}


\NormalTok{bwk\_client}\SpecialCharTok{$}\FunctionTok{getFeatureTypes}\NormalTok{(}\AttributeTok{pretty =} \ConstantTok{TRUE}\NormalTok{)}

\CommentTok{\# geoms}
\NormalTok{bwk\_client}\SpecialCharTok{$}\FunctionTok{getFeatureTypes}\NormalTok{(}\AttributeTok{pretty =} \ConstantTok{TRUE}\NormalTok{)}
\NormalTok{url}\SpecialCharTok{$}\NormalTok{query }\OtherTok{\textless{}{-}} \FunctionTok{list}\NormalTok{(}\AttributeTok{service =} \StringTok{"wfs"}\NormalTok{,}
                  \AttributeTok{request =} \StringTok{"GetFeature"}\NormalTok{,}
                  \AttributeTok{srsName=}\StringTok{"EPSG:3059"}\NormalTok{,}
                  \AttributeTok{typename =} \StringTok{"zmni:zmni\_draincollectors"}\NormalTok{,}
                  \AttributeTok{count=}\DecValTok{1}\NormalTok{)}
\NormalTok{request }\OtherTok{\textless{}{-}} \FunctionTok{build\_url}\NormalTok{(url)}

\NormalTok{geometrijam }\OtherTok{\textless{}{-}} \FunctionTok{read\_sf}\NormalTok{(request)}
\NormalTok{geometrijam}

\CommentTok{\# count}
\NormalTok{url}\SpecialCharTok{$}\NormalTok{query }\OtherTok{\textless{}{-}} \FunctionTok{list}\NormalTok{(}\AttributeTok{service =} \StringTok{"wfs"}\NormalTok{,}
                  \AttributeTok{request =} \StringTok{"GetFeature"}\NormalTok{,}
                  \AttributeTok{srsName=}\StringTok{"EPSG:3059"}\NormalTok{,}
                  \AttributeTok{typename =} \StringTok{"zmni:zmni\_draincollectors"}\NormalTok{,}
                  \AttributeTok{resultType=}\StringTok{"hits"}\NormalTok{)}
\NormalTok{request }\OtherTok{\textless{}{-}} \FunctionTok{build\_url}\NormalTok{(url)}
\NormalTok{result }\OtherTok{\textless{}{-}} \FunctionTok{GET}\NormalTok{(request)}
\NormalTok{parsed }\OtherTok{\textless{}{-}}\NormalTok{ xml2}\SpecialCharTok{::}\FunctionTok{as\_list}\NormalTok{(}\FunctionTok{content}\NormalTok{(result, }\StringTok{"parsed"}\NormalTok{))}
\NormalTok{n\_features }\OtherTok{\textless{}{-}} \FunctionTok{attr}\NormalTok{(parsed}\SpecialCharTok{$}\NormalTok{FeatureCollection, }\StringTok{"numberMatched"}\NormalTok{)}
\NormalTok{n\_features}


\CommentTok{\# download}
\NormalTok{base\_url }\OtherTok{\textless{}{-}} \StringTok{"https://lvmgeoserver.lvm.lv/geoserver/zmni/ows?"}
\NormalTok{type\_name }\OtherTok{\textless{}{-}} \StringTok{"zmni:zmni\_draincollectors"}
\NormalTok{crs\_code }\OtherTok{\textless{}{-}} \DecValTok{3059}
\NormalTok{chunk\_size }\OtherTok{\textless{}{-}} \DecValTok{100000}
\NormalTok{gpkg\_path }\OtherTok{\textless{}{-}} \StringTok{"./Geodata/2024/MKIS/temp\_MKIS\_2025.gpkg"}
\NormalTok{layer\_name }\OtherTok{\textless{}{-}} \StringTok{"temp\_DrenuKolektori"}
\NormalTok{i }\OtherTok{\textless{}{-}} \DecValTok{0}

\ControlFlowTok{repeat}\NormalTok{ \{}
  \FunctionTok{message}\NormalTok{(}\StringTok{"Fetching features "}\NormalTok{, i }\SpecialCharTok{*}\NormalTok{ chunk\_size }\SpecialCharTok{+} \DecValTok{1}\NormalTok{, }\StringTok{" to "}\NormalTok{, (i }\SpecialCharTok{+} \DecValTok{1}\NormalTok{) }\SpecialCharTok{*}\NormalTok{ chunk\_size, }\StringTok{"..."}\NormalTok{)}
  
\NormalTok{  query }\OtherTok{\textless{}{-}} \FunctionTok{list}\NormalTok{(}
    \AttributeTok{service =} \StringTok{"WFS"}\NormalTok{,}
    \AttributeTok{version =} \StringTok{"2.0.0"}\NormalTok{,}
    \AttributeTok{request =} \StringTok{"GetFeature"}\NormalTok{,}
    \AttributeTok{typename =}\NormalTok{ type\_name,}
    \AttributeTok{srsName =} \FunctionTok{paste0}\NormalTok{(}\StringTok{"EPSG:"}\NormalTok{, crs\_code),}
    \AttributeTok{count =}\NormalTok{ chunk\_size,}
    \AttributeTok{startIndex =}\NormalTok{ i }\SpecialCharTok{*}\NormalTok{ chunk\_size}
\NormalTok{  )}
  
\NormalTok{  req\_url }\OtherTok{\textless{}{-}} \FunctionTok{modify\_url}\NormalTok{(base\_url, }\AttributeTok{query =}\NormalTok{ query)}
  
  \FunctionTok{try}\NormalTok{(\{}
\NormalTok{    chunk }\OtherTok{\textless{}{-}} \FunctionTok{read\_sf}\NormalTok{(req\_url)}
    \ControlFlowTok{if}\NormalTok{ (}\FunctionTok{nrow}\NormalTok{(chunk) }\SpecialCharTok{==} \DecValTok{0}\NormalTok{) }\ControlFlowTok{break}
    
    \CommentTok{\# Set CRS and cast to MULTILINESTRING}
\NormalTok{    chunk }\OtherTok{\textless{}{-}}\NormalTok{ chunk }\SpecialCharTok{\%\textgreater{}\%}
      \FunctionTok{st\_set\_crs}\NormalTok{(}\FunctionTok{st\_crs}\NormalTok{(crs\_code)) }\SpecialCharTok{\%\textgreater{}\%}
      \FunctionTok{st\_cast}\NormalTok{(}\StringTok{"MULTILINESTRING"}\NormalTok{)}
    
    \CommentTok{\# Write chunk to GeoPackage (append mode after first)}
    \FunctionTok{st\_write}\NormalTok{(}
\NormalTok{      chunk, }
      \AttributeTok{dsn =}\NormalTok{ gpkg\_path,}
      \AttributeTok{layer =}\NormalTok{ layer\_name,}
      \AttributeTok{append =}\NormalTok{ i }\SpecialCharTok{!=} \DecValTok{0}\NormalTok{,}
      \AttributeTok{quiet =} \ConstantTok{FALSE}
\NormalTok{    )}
    
\NormalTok{    i }\OtherTok{\textless{}{-}}\NormalTok{ i }\SpecialCharTok{+} \DecValTok{1}
\NormalTok{  \}, }\AttributeTok{silent =} \ConstantTok{TRUE}\NormalTok{)}
\NormalTok{\}}

\FunctionTok{message}\NormalTok{(}\StringTok{"All chunks written to "}\NormalTok{, gpkg\_path, }\StringTok{" in layer "}\NormalTok{, layer\_name)}

\NormalTok{DrenuKolektori\_all}\OtherTok{=}\FunctionTok{st\_read}\NormalTok{(}\StringTok{"./Geodata/2024/MKIS/temp\_MKIS\_2025.gpkg"}\NormalTok{,}
                           \AttributeTok{layer=}\StringTok{"temp\_DrenuKolektori"}\NormalTok{)}
\NormalTok{DrenuKolektori\_all2 }\OtherTok{=}\NormalTok{ DrenuKolektori\_all[}\SpecialCharTok{!}\FunctionTok{st\_is\_empty}\NormalTok{(DrenuKolektori\_all),,drop}\OtherTok{=}\ConstantTok{FALSE}\NormalTok{] }\CommentTok{\# 1}
\FunctionTok{table}\NormalTok{(}\FunctionTok{st\_is\_valid}\NormalTok{(DrenuKolektori\_all2))}


\FunctionTok{write\_sf}\NormalTok{(DrenuKolektori\_all2,}
         \StringTok{"./Geodata/2024/MKIS/MKIS\_2025.gpkg"}\NormalTok{,}
         \AttributeTok{layer=}\StringTok{"DrenuKolektori"}\NormalTok{,}
         \AttributeTok{append=}\ConstantTok{FALSE}\NormalTok{)}
\FunctionTok{rm}\NormalTok{(}\AttributeTok{list=}\FunctionTok{ls}\NormalTok{())}




\CommentTok{\# drenage network structures {-}{-}{-}{-}}

\NormalTok{link}\OtherTok{=}\StringTok{"https://lvmgeoserver.lvm.lv/geoserver/zmni/ows?"}
\NormalTok{url}\OtherTok{=}\FunctionTok{parse\_url}\NormalTok{(link)}

\NormalTok{url}\SpecialCharTok{$}\NormalTok{query }\OtherTok{\textless{}{-}} \FunctionTok{list}\NormalTok{(}\AttributeTok{service =} \StringTok{"wfs"}\NormalTok{,}\AttributeTok{request =} \StringTok{"GetCapabilities"}\NormalTok{)}
\NormalTok{request }\OtherTok{\textless{}{-}} \FunctionTok{build\_url}\NormalTok{(url)}
\NormalTok{bwk\_client }\OtherTok{\textless{}{-}}\NormalTok{ WFSClient}\SpecialCharTok{$}\FunctionTok{new}\NormalTok{(link,}\AttributeTok{serviceVersion =} \StringTok{"2.0.0"}\NormalTok{)}

\NormalTok{bwk\_client}\SpecialCharTok{$}\FunctionTok{getFeatureTypes}\NormalTok{(}\AttributeTok{pretty =} \ConstantTok{TRUE}\NormalTok{)}

\CommentTok{\# geoms}

\NormalTok{url}\SpecialCharTok{$}\NormalTok{query }\OtherTok{\textless{}{-}} \FunctionTok{list}\NormalTok{(}\AttributeTok{service =} \StringTok{"wfs"}\NormalTok{,}
                  \AttributeTok{request =} \StringTok{"GetFeature"}\NormalTok{,}
                  \AttributeTok{srsName=}\StringTok{"EPSG:3059"}\NormalTok{,}
                  \AttributeTok{typename =} \StringTok{"zmni:zmni\_networkstructures"}\NormalTok{,}
                  \AttributeTok{count=}\DecValTok{1}\NormalTok{)}
\NormalTok{request }\OtherTok{\textless{}{-}} \FunctionTok{build\_url}\NormalTok{(url)}

\NormalTok{geometrijam }\OtherTok{\textless{}{-}} \FunctionTok{read\_sf}\NormalTok{(request)}
\NormalTok{geometrijam}



\CommentTok{\# download}
\NormalTok{base\_url }\OtherTok{\textless{}{-}} \StringTok{"https://lvmgeoserver.lvm.lv/geoserver/zmni/ows?"}
\NormalTok{type\_name }\OtherTok{\textless{}{-}} \StringTok{"zmni:zmni\_networkstructures"}
\NormalTok{crs\_code }\OtherTok{\textless{}{-}} \DecValTok{3059}
\NormalTok{chunk\_size }\OtherTok{\textless{}{-}} \DecValTok{100000}
\NormalTok{gpkg\_path }\OtherTok{\textless{}{-}} \StringTok{"./Geodata/2024/MKIS/temp\_MKIS\_2025.gpkg"}
\NormalTok{layer\_name }\OtherTok{\textless{}{-}} \StringTok{"temp\_DrenazasTiklaBuves"}
\NormalTok{i }\OtherTok{\textless{}{-}} \DecValTok{0}

\ControlFlowTok{repeat}\NormalTok{ \{}
  \FunctionTok{message}\NormalTok{(}\StringTok{"Fetching features "}\NormalTok{, i }\SpecialCharTok{*}\NormalTok{ chunk\_size }\SpecialCharTok{+} \DecValTok{1}\NormalTok{, }\StringTok{" to "}\NormalTok{, (i }\SpecialCharTok{+} \DecValTok{1}\NormalTok{) }\SpecialCharTok{*}\NormalTok{ chunk\_size, }\StringTok{"..."}\NormalTok{)}
  
\NormalTok{  query }\OtherTok{\textless{}{-}} \FunctionTok{list}\NormalTok{(}
    \AttributeTok{service =} \StringTok{"WFS"}\NormalTok{,}
    \AttributeTok{version =} \StringTok{"2.0.0"}\NormalTok{,}
    \AttributeTok{request =} \StringTok{"GetFeature"}\NormalTok{,}
    \AttributeTok{typename =}\NormalTok{ type\_name,}
    \AttributeTok{srsName =} \FunctionTok{paste0}\NormalTok{(}\StringTok{"EPSG:"}\NormalTok{, crs\_code),}
    \AttributeTok{count =}\NormalTok{ chunk\_size,}
    \AttributeTok{startIndex =}\NormalTok{ i }\SpecialCharTok{*}\NormalTok{ chunk\_size}
\NormalTok{  )}
  
\NormalTok{  req\_url }\OtherTok{\textless{}{-}} \FunctionTok{modify\_url}\NormalTok{(base\_url, }\AttributeTok{query =}\NormalTok{ query)}
  
  \FunctionTok{try}\NormalTok{(\{}
\NormalTok{    chunk }\OtherTok{\textless{}{-}} \FunctionTok{read\_sf}\NormalTok{(req\_url)}
    \ControlFlowTok{if}\NormalTok{ (}\FunctionTok{nrow}\NormalTok{(chunk) }\SpecialCharTok{==} \DecValTok{0}\NormalTok{) }\ControlFlowTok{break}
    
    \CommentTok{\# Set CRS and cast to MULTILINESTRING}
\NormalTok{    chunk }\OtherTok{\textless{}{-}}\NormalTok{ chunk }\SpecialCharTok{\%\textgreater{}\%}
      \FunctionTok{st\_set\_crs}\NormalTok{(}\FunctionTok{st\_crs}\NormalTok{(crs\_code)) }\SpecialCharTok{\%\textgreater{}\%}
      \FunctionTok{st\_cast}\NormalTok{(}\StringTok{"POINT"}\NormalTok{)}
    
    \CommentTok{\# Write chunk to GeoPackage (append mode after first)}
    \FunctionTok{st\_write}\NormalTok{(}
\NormalTok{      chunk, }
      \AttributeTok{dsn =}\NormalTok{ gpkg\_path,}
      \AttributeTok{layer =}\NormalTok{ layer\_name,}
      \AttributeTok{append =}\NormalTok{ i }\SpecialCharTok{!=} \DecValTok{0}\NormalTok{,}
      \AttributeTok{quiet =} \ConstantTok{FALSE}
\NormalTok{    )}
    
\NormalTok{    i }\OtherTok{\textless{}{-}}\NormalTok{ i }\SpecialCharTok{+} \DecValTok{1}
\NormalTok{  \}, }\AttributeTok{silent =} \ConstantTok{TRUE}\NormalTok{)}
\NormalTok{\}}

\FunctionTok{message}\NormalTok{(}\StringTok{"All chunks written to "}\NormalTok{, gpkg\_path, }\StringTok{" in layer "}\NormalTok{, layer\_name)}

\NormalTok{DrenazasTiklaBuves\_all}\OtherTok{=}\FunctionTok{st\_read}\NormalTok{(}\StringTok{"./Geodata/2024/MKIS/temp\_MKIS\_2025.gpkg"}\NormalTok{,}
                               \AttributeTok{layer=}\StringTok{"temp\_DrenazasTiklaBuves"}\NormalTok{)}
\NormalTok{DrenazasTiklaBuves\_all2 }\OtherTok{=}\NormalTok{ DrenazasTiklaBuves\_all[}\SpecialCharTok{!}\FunctionTok{st\_is\_empty}\NormalTok{(DrenazasTiklaBuves\_all),,drop}\OtherTok{=}\ConstantTok{FALSE}\NormalTok{] }\CommentTok{\# 0}
\FunctionTok{table}\NormalTok{(}\FunctionTok{st\_is\_valid}\NormalTok{(DrenazasTiklaBuves\_all2))}


\FunctionTok{write\_sf}\NormalTok{(DrenazasTiklaBuves\_all2,}
         \StringTok{"./Geodata/2024/MKIS/MKIS\_2025.gpkg"}\NormalTok{,}
         \AttributeTok{layer=}\StringTok{"DrenazasTiklaBuves"}\NormalTok{,}
         \AttributeTok{append=}\ConstantTok{FALSE}\NormalTok{)}
\FunctionTok{rm}\NormalTok{(}\AttributeTok{list=}\FunctionTok{ls}\NormalTok{())}




\CommentTok{\# dithces {-}{-}{-}{-}{-}}


\NormalTok{link}\OtherTok{=}\StringTok{"https://lvmgeoserver.lvm.lv/geoserver/zmni/ows?"}
\NormalTok{url}\OtherTok{=}\FunctionTok{parse\_url}\NormalTok{(link)}

\NormalTok{url}\SpecialCharTok{$}\NormalTok{query }\OtherTok{\textless{}{-}} \FunctionTok{list}\NormalTok{(}\AttributeTok{service =} \StringTok{"wfs"}\NormalTok{,}\AttributeTok{request =} \StringTok{"GetCapabilities"}\NormalTok{)}
\NormalTok{request }\OtherTok{\textless{}{-}} \FunctionTok{build\_url}\NormalTok{(url)}
\NormalTok{bwk\_client }\OtherTok{\textless{}{-}}\NormalTok{ WFSClient}\SpecialCharTok{$}\FunctionTok{new}\NormalTok{(link,}\AttributeTok{serviceVersion =} \StringTok{"2.0.0"}\NormalTok{)}

\NormalTok{bwk\_client}\SpecialCharTok{$}\FunctionTok{getFeatureTypes}\NormalTok{(}\AttributeTok{pretty =} \ConstantTok{TRUE}\NormalTok{)}

\CommentTok{\# geoms}

\NormalTok{url}\SpecialCharTok{$}\NormalTok{query }\OtherTok{\textless{}{-}} \FunctionTok{list}\NormalTok{(}\AttributeTok{service =} \StringTok{"wfs"}\NormalTok{,}
                  \AttributeTok{request =} \StringTok{"GetFeature"}\NormalTok{,}
                  \AttributeTok{srsName=}\StringTok{"EPSG:3059"}\NormalTok{,}
                  \AttributeTok{typename =} \StringTok{"zmni:zmni\_ditches"}\NormalTok{,}
                  \AttributeTok{count=}\DecValTok{100}\NormalTok{)}
\NormalTok{request }\OtherTok{\textless{}{-}} \FunctionTok{build\_url}\NormalTok{(url)}

\NormalTok{geometrijam }\OtherTok{\textless{}{-}} \FunctionTok{read\_sf}\NormalTok{(request)}
\NormalTok{geometrijam}



\CommentTok{\# download}
\NormalTok{base\_url }\OtherTok{\textless{}{-}} \StringTok{"https://lvmgeoserver.lvm.lv/geoserver/zmni/ows?"}
\NormalTok{type\_name }\OtherTok{\textless{}{-}} \StringTok{"zmni:zmni\_ditches"}
\NormalTok{crs\_code }\OtherTok{\textless{}{-}} \DecValTok{3059}
\NormalTok{chunk\_size }\OtherTok{\textless{}{-}} \DecValTok{100000}
\NormalTok{gpkg\_path }\OtherTok{\textless{}{-}} \StringTok{"./Geodata/2024/MKIS/temp\_MKIS\_2025.gpkg"}
\NormalTok{layer\_name }\OtherTok{\textless{}{-}} \StringTok{"temp\_Gravji"}
\NormalTok{i }\OtherTok{\textless{}{-}} \DecValTok{0}

\ControlFlowTok{repeat}\NormalTok{ \{}
  \FunctionTok{message}\NormalTok{(}\StringTok{"Fetching features "}\NormalTok{, i }\SpecialCharTok{*}\NormalTok{ chunk\_size }\SpecialCharTok{+} \DecValTok{1}\NormalTok{, }\StringTok{" to "}\NormalTok{, (i }\SpecialCharTok{+} \DecValTok{1}\NormalTok{) }\SpecialCharTok{*}\NormalTok{ chunk\_size, }\StringTok{"..."}\NormalTok{)}
  
\NormalTok{  query }\OtherTok{\textless{}{-}} \FunctionTok{list}\NormalTok{(}
    \AttributeTok{service =} \StringTok{"WFS"}\NormalTok{,}
    \AttributeTok{version =} \StringTok{"2.0.0"}\NormalTok{,}
    \AttributeTok{request =} \StringTok{"GetFeature"}\NormalTok{,}
    \AttributeTok{typename =}\NormalTok{ type\_name,}
    \AttributeTok{srsName =} \FunctionTok{paste0}\NormalTok{(}\StringTok{"EPSG:"}\NormalTok{, crs\_code),}
    \AttributeTok{count =}\NormalTok{ chunk\_size,}
    \AttributeTok{startIndex =}\NormalTok{ i }\SpecialCharTok{*}\NormalTok{ chunk\_size}
\NormalTok{  )}
  
\NormalTok{  req\_url }\OtherTok{\textless{}{-}} \FunctionTok{modify\_url}\NormalTok{(base\_url, }\AttributeTok{query =}\NormalTok{ query)}
  
  \FunctionTok{try}\NormalTok{(\{}
\NormalTok{    chunk }\OtherTok{\textless{}{-}} \FunctionTok{read\_sf}\NormalTok{(req\_url)}
    \ControlFlowTok{if}\NormalTok{ (}\FunctionTok{nrow}\NormalTok{(chunk) }\SpecialCharTok{==} \DecValTok{0}\NormalTok{) }\ControlFlowTok{break}
    
    \CommentTok{\# Set CRS and cast to MULTILINESTRING}
\NormalTok{    chunk }\OtherTok{\textless{}{-}}\NormalTok{ chunk }\SpecialCharTok{\%\textgreater{}\%}
      \FunctionTok{st\_set\_crs}\NormalTok{(}\FunctionTok{st\_crs}\NormalTok{(crs\_code)) }\SpecialCharTok{\%\textgreater{}\%}
      \FunctionTok{st\_cast}\NormalTok{(}\StringTok{"MULTILINESTRING"}\NormalTok{)}
    
    \CommentTok{\# Write chunk to GeoPackage (append mode after first)}
    \FunctionTok{st\_write}\NormalTok{(}
\NormalTok{      chunk, }
      \AttributeTok{dsn =}\NormalTok{ gpkg\_path,}
      \AttributeTok{layer =}\NormalTok{ layer\_name,}
      \AttributeTok{append =}\NormalTok{ i }\SpecialCharTok{!=} \DecValTok{0}\NormalTok{,}
      \AttributeTok{quiet =} \ConstantTok{FALSE}
\NormalTok{    )}
    
\NormalTok{    i }\OtherTok{\textless{}{-}}\NormalTok{ i }\SpecialCharTok{+} \DecValTok{1}
\NormalTok{  \}, }\AttributeTok{silent =} \ConstantTok{TRUE}\NormalTok{)}
\NormalTok{\}}

\FunctionTok{message}\NormalTok{(}\StringTok{"All chunks written to "}\NormalTok{, gpkg\_path, }\StringTok{" in layer "}\NormalTok{, layer\_name)}

\NormalTok{Gravji\_all}\OtherTok{=}\FunctionTok{st\_read}\NormalTok{(}\StringTok{"./Geodata/2024/MKIS/temp\_MKIS\_2025.gpkg"}\NormalTok{,}
                   \AttributeTok{layer=}\StringTok{"temp\_Gravji"}\NormalTok{)}
\NormalTok{Gravji\_all2 }\OtherTok{=}\NormalTok{ Gravji\_all[}\SpecialCharTok{!}\FunctionTok{st\_is\_empty}\NormalTok{(Gravji\_all),,drop}\OtherTok{=}\ConstantTok{FALSE}\NormalTok{] }\CommentTok{\# 0}
\FunctionTok{table}\NormalTok{(}\FunctionTok{st\_is\_valid}\NormalTok{(Gravji\_all2))}


\FunctionTok{write\_sf}\NormalTok{(Gravji\_all2,}
         \StringTok{"./Geodata/2024/MKIS/MKIS\_2025.gpkg"}\NormalTok{,}
         \AttributeTok{layer=}\StringTok{"Gravji"}\NormalTok{,}
         \AttributeTok{append=}\ConstantTok{FALSE}\NormalTok{)}
\FunctionTok{rm}\NormalTok{(}\AttributeTok{list=}\FunctionTok{ls}\NormalTok{())}




\CommentTok{\# hydrometric posts {-}{-}{-}{-}}


\NormalTok{link}\OtherTok{=}\StringTok{"https://lvmgeoserver.lvm.lv/geoserver/zmni/ows?"}
\NormalTok{url}\OtherTok{=}\FunctionTok{parse\_url}\NormalTok{(link)}

\NormalTok{url}\SpecialCharTok{$}\NormalTok{query }\OtherTok{\textless{}{-}} \FunctionTok{list}\NormalTok{(}\AttributeTok{service =} \StringTok{"wfs"}\NormalTok{,}\AttributeTok{request =} \StringTok{"GetCapabilities"}\NormalTok{)}
\NormalTok{request }\OtherTok{\textless{}{-}} \FunctionTok{build\_url}\NormalTok{(url)}
\NormalTok{bwk\_client }\OtherTok{\textless{}{-}}\NormalTok{ WFSClient}\SpecialCharTok{$}\FunctionTok{new}\NormalTok{(link,}\AttributeTok{serviceVersion =} \StringTok{"2.0.0"}\NormalTok{)}

\NormalTok{bwk\_client}\SpecialCharTok{$}\FunctionTok{getFeatureTypes}\NormalTok{(}\AttributeTok{pretty =} \ConstantTok{TRUE}\NormalTok{)}

\CommentTok{\# geoms}

\NormalTok{url}\SpecialCharTok{$}\NormalTok{query }\OtherTok{\textless{}{-}} \FunctionTok{list}\NormalTok{(}\AttributeTok{service =} \StringTok{"wfs"}\NormalTok{,}
                  \AttributeTok{request =} \StringTok{"GetFeature"}\NormalTok{,}
                  \AttributeTok{srsName=}\StringTok{"EPSG:3059"}\NormalTok{,}
                  \AttributeTok{typename =} \StringTok{"zmni:zmni\_hydropost"}\NormalTok{,}
                  \AttributeTok{count=}\DecValTok{100}\NormalTok{)}
\NormalTok{request }\OtherTok{\textless{}{-}} \FunctionTok{build\_url}\NormalTok{(url)}

\NormalTok{geometrijam }\OtherTok{\textless{}{-}} \FunctionTok{read\_sf}\NormalTok{(request)}
\NormalTok{geometrijam}



\CommentTok{\# download}
\NormalTok{base\_url }\OtherTok{\textless{}{-}} \StringTok{"https://lvmgeoserver.lvm.lv/geoserver/zmni/ows?"}
\NormalTok{type\_name }\OtherTok{\textless{}{-}} \StringTok{"zmni:zmni\_hydropost"}
\NormalTok{crs\_code }\OtherTok{\textless{}{-}} \DecValTok{3059}
\NormalTok{chunk\_size }\OtherTok{\textless{}{-}} \DecValTok{100000}
\NormalTok{gpkg\_path }\OtherTok{\textless{}{-}} \StringTok{"./IevadesDati/MKIS/temp\_MKIS\_2025.gpkg"}
\NormalTok{layer\_name }\OtherTok{\textless{}{-}} \StringTok{"temp\_HidrometriskiePosteni"}
\NormalTok{i }\OtherTok{\textless{}{-}} \DecValTok{0}

\ControlFlowTok{repeat}\NormalTok{ \{}
  \FunctionTok{message}\NormalTok{(}\StringTok{"Fetching features "}\NormalTok{, i }\SpecialCharTok{*}\NormalTok{ chunk\_size }\SpecialCharTok{+} \DecValTok{1}\NormalTok{, }\StringTok{" to "}\NormalTok{, (i }\SpecialCharTok{+} \DecValTok{1}\NormalTok{) }\SpecialCharTok{*}\NormalTok{ chunk\_size, }\StringTok{"..."}\NormalTok{)}
  
\NormalTok{  query }\OtherTok{\textless{}{-}} \FunctionTok{list}\NormalTok{(}
    \AttributeTok{service =} \StringTok{"WFS"}\NormalTok{,}
    \AttributeTok{version =} \StringTok{"2.0.0"}\NormalTok{,}
    \AttributeTok{request =} \StringTok{"GetFeature"}\NormalTok{,}
    \AttributeTok{typename =}\NormalTok{ type\_name,}
    \AttributeTok{srsName =} \FunctionTok{paste0}\NormalTok{(}\StringTok{"EPSG:"}\NormalTok{, crs\_code),}
    \AttributeTok{count =}\NormalTok{ chunk\_size,}
    \AttributeTok{startIndex =}\NormalTok{ i }\SpecialCharTok{*}\NormalTok{ chunk\_size}
\NormalTok{  )}
  
\NormalTok{  req\_url }\OtherTok{\textless{}{-}} \FunctionTok{modify\_url}\NormalTok{(base\_url, }\AttributeTok{query =}\NormalTok{ query)}
  
  \FunctionTok{try}\NormalTok{(\{}
\NormalTok{    chunk }\OtherTok{\textless{}{-}} \FunctionTok{read\_sf}\NormalTok{(req\_url)}
    \ControlFlowTok{if}\NormalTok{ (}\FunctionTok{nrow}\NormalTok{(chunk) }\SpecialCharTok{==} \DecValTok{0}\NormalTok{) }\ControlFlowTok{break}
    
    \CommentTok{\# Set CRS and cast to MULTILINESTRING, POINT, MULTIPOLYGON}
\NormalTok{    chunk }\OtherTok{\textless{}{-}}\NormalTok{ chunk }\SpecialCharTok{\%\textgreater{}\%}
      \FunctionTok{st\_set\_crs}\NormalTok{(}\FunctionTok{st\_crs}\NormalTok{(crs\_code)) }\SpecialCharTok{\%\textgreater{}\%}
      \FunctionTok{st\_cast}\NormalTok{(}\StringTok{"POINT"}\NormalTok{)}
    
    \CommentTok{\# Write chunk to GeoPackage (append mode after first)}
    \FunctionTok{st\_write}\NormalTok{(}
\NormalTok{      chunk, }
      \AttributeTok{dsn =}\NormalTok{ gpkg\_path,}
      \AttributeTok{layer =}\NormalTok{ layer\_name,}
      \AttributeTok{append =}\NormalTok{ i }\SpecialCharTok{!=} \DecValTok{0}\NormalTok{,}
      \AttributeTok{quiet =} \ConstantTok{FALSE}
\NormalTok{    )}
    
\NormalTok{    i }\OtherTok{\textless{}{-}}\NormalTok{ i }\SpecialCharTok{+} \DecValTok{1}
\NormalTok{  \}, }\AttributeTok{silent =} \ConstantTok{TRUE}\NormalTok{)}
\NormalTok{\}}

\FunctionTok{message}\NormalTok{(}\StringTok{"All chunks written to "}\NormalTok{, gpkg\_path, }\StringTok{" in layer "}\NormalTok{, layer\_name)}

\NormalTok{HidrometriskiePosteni\_all}\OtherTok{=}\FunctionTok{st\_read}\NormalTok{(}\StringTok{"./Geodata/2024/MKIS/temp\_MKIS\_2025.gpkg"}\NormalTok{,}
                                  \AttributeTok{layer=}\StringTok{"temp\_HidrometriskiePosteni"}\NormalTok{)}
\NormalTok{HidrometriskiePosteni\_all2 }\OtherTok{=}\NormalTok{ HidrometriskiePosteni\_all[}\SpecialCharTok{!}\FunctionTok{st\_is\_empty}\NormalTok{(HidrometriskiePosteni\_all),,drop}\OtherTok{=}\ConstantTok{FALSE}\NormalTok{] }\CommentTok{\# 0}
\FunctionTok{table}\NormalTok{(}\FunctionTok{st\_is\_valid}\NormalTok{(HidrometriskiePosteni\_all2))}


\FunctionTok{write\_sf}\NormalTok{(HidrometriskiePosteni\_all2,}
         \StringTok{"./Geodata/2024/MKIS/MKIS\_2025.gpkg"}\NormalTok{,}
         \AttributeTok{layer=}\StringTok{"HidrometriskiePosteni"}\NormalTok{,}
         \AttributeTok{append=}\ConstantTok{FALSE}\NormalTok{)}
\FunctionTok{rm}\NormalTok{(}\AttributeTok{list=}\FunctionTok{ls}\NormalTok{())}


\CommentTok{\# large diameter drain collectors {-}{-}{-}{-}}


\NormalTok{link}\OtherTok{=}\StringTok{"https://lvmgeoserver.lvm.lv/geoserver/zmni/ows?"}
\NormalTok{url}\OtherTok{=}\FunctionTok{parse\_url}\NormalTok{(link)}

\NormalTok{url}\SpecialCharTok{$}\NormalTok{query }\OtherTok{\textless{}{-}} \FunctionTok{list}\NormalTok{(}\AttributeTok{service =} \StringTok{"wfs"}\NormalTok{,}\AttributeTok{request =} \StringTok{"GetCapabilities"}\NormalTok{)}
\NormalTok{request }\OtherTok{\textless{}{-}} \FunctionTok{build\_url}\NormalTok{(url)}
\NormalTok{bwk\_client }\OtherTok{\textless{}{-}}\NormalTok{ WFSClient}\SpecialCharTok{$}\FunctionTok{new}\NormalTok{(link,}\AttributeTok{serviceVersion =} \StringTok{"2.0.0"}\NormalTok{)}

\NormalTok{bwk\_client}\SpecialCharTok{$}\FunctionTok{getFeatureTypes}\NormalTok{(}\AttributeTok{pretty =} \ConstantTok{TRUE}\NormalTok{)}

\CommentTok{\# geoms}

\NormalTok{url}\SpecialCharTok{$}\NormalTok{query }\OtherTok{\textless{}{-}} \FunctionTok{list}\NormalTok{(}\AttributeTok{service =} \StringTok{"wfs"}\NormalTok{,}
                  \AttributeTok{request =} \StringTok{"GetFeature"}\NormalTok{,}
                  \AttributeTok{srsName=}\StringTok{"EPSG:3059"}\NormalTok{,}
                  \AttributeTok{typename =} \StringTok{"zmni:zmni\_bigdraincollectors"}\NormalTok{,}
                  \AttributeTok{count=}\DecValTok{100}\NormalTok{)}
\NormalTok{request }\OtherTok{\textless{}{-}} \FunctionTok{build\_url}\NormalTok{(url)}

\NormalTok{geometrijam }\OtherTok{\textless{}{-}} \FunctionTok{read\_sf}\NormalTok{(request)}
\NormalTok{geometrijam}



\CommentTok{\# download}
\NormalTok{base\_url }\OtherTok{\textless{}{-}} \StringTok{"https://lvmgeoserver.lvm.lv/geoserver/zmni/ows?"}
\NormalTok{type\_name }\OtherTok{\textless{}{-}} \StringTok{"zmni:zmni\_bigdraincollectors"}
\NormalTok{crs\_code }\OtherTok{\textless{}{-}} \DecValTok{3059}
\NormalTok{chunk\_size }\OtherTok{\textless{}{-}} \DecValTok{100000}
\NormalTok{gpkg\_path }\OtherTok{\textless{}{-}} \StringTok{"./Geodata/2024/MKIS/temp\_MKIS\_2025.gpkg"}
\NormalTok{layer\_name }\OtherTok{\textless{}{-}} \StringTok{"temp\_LielaDiametraKolektori"}
\NormalTok{i }\OtherTok{\textless{}{-}} \DecValTok{0}

\ControlFlowTok{repeat}\NormalTok{ \{}
  \FunctionTok{message}\NormalTok{(}\StringTok{"Fetching features "}\NormalTok{, i }\SpecialCharTok{*}\NormalTok{ chunk\_size }\SpecialCharTok{+} \DecValTok{1}\NormalTok{, }\StringTok{" to "}\NormalTok{, (i }\SpecialCharTok{+} \DecValTok{1}\NormalTok{) }\SpecialCharTok{*}\NormalTok{ chunk\_size, }\StringTok{"..."}\NormalTok{)}
  
\NormalTok{  query }\OtherTok{\textless{}{-}} \FunctionTok{list}\NormalTok{(}
    \AttributeTok{service =} \StringTok{"WFS"}\NormalTok{,}
    \AttributeTok{version =} \StringTok{"2.0.0"}\NormalTok{,}
    \AttributeTok{request =} \StringTok{"GetFeature"}\NormalTok{,}
    \AttributeTok{typename =}\NormalTok{ type\_name,}
    \AttributeTok{srsName =} \FunctionTok{paste0}\NormalTok{(}\StringTok{"EPSG:"}\NormalTok{, crs\_code),}
    \AttributeTok{count =}\NormalTok{ chunk\_size,}
    \AttributeTok{startIndex =}\NormalTok{ i }\SpecialCharTok{*}\NormalTok{ chunk\_size}
\NormalTok{  )}
  
\NormalTok{  req\_url }\OtherTok{\textless{}{-}} \FunctionTok{modify\_url}\NormalTok{(base\_url, }\AttributeTok{query =}\NormalTok{ query)}
  
  \FunctionTok{try}\NormalTok{(\{}
\NormalTok{    chunk }\OtherTok{\textless{}{-}} \FunctionTok{read\_sf}\NormalTok{(req\_url)}
    \ControlFlowTok{if}\NormalTok{ (}\FunctionTok{nrow}\NormalTok{(chunk) }\SpecialCharTok{==} \DecValTok{0}\NormalTok{) }\ControlFlowTok{break}
    
    \CommentTok{\# Set CRS and cast to MULTILINESTRING, POINT, MULTIPOLYGON}
\NormalTok{    chunk }\OtherTok{\textless{}{-}}\NormalTok{ chunk }\SpecialCharTok{\%\textgreater{}\%}
      \FunctionTok{st\_set\_crs}\NormalTok{(}\FunctionTok{st\_crs}\NormalTok{(crs\_code)) }\SpecialCharTok{\%\textgreater{}\%}
      \FunctionTok{st\_cast}\NormalTok{(}\StringTok{"MULTILINESTRING"}\NormalTok{)}
    
    \CommentTok{\# Write chunk to GeoPackage (append mode after first)}
    \FunctionTok{st\_write}\NormalTok{(}
\NormalTok{      chunk, }
      \AttributeTok{dsn =}\NormalTok{ gpkg\_path,}
      \AttributeTok{layer =}\NormalTok{ layer\_name,}
      \AttributeTok{append =}\NormalTok{ i }\SpecialCharTok{!=} \DecValTok{0}\NormalTok{,}
      \AttributeTok{quiet =} \ConstantTok{FALSE}
\NormalTok{    )}
    
\NormalTok{    i }\OtherTok{\textless{}{-}}\NormalTok{ i }\SpecialCharTok{+} \DecValTok{1}
\NormalTok{  \}, }\AttributeTok{silent =} \ConstantTok{TRUE}\NormalTok{)}
\NormalTok{\}}

\FunctionTok{message}\NormalTok{(}\StringTok{"All chunks written to "}\NormalTok{, gpkg\_path, }\StringTok{" in layer "}\NormalTok{, layer\_name)}

\NormalTok{LielaDiametraKolektori\_all}\OtherTok{=}\FunctionTok{st\_read}\NormalTok{(}\StringTok{"./Geodata/2024/MKIS/temp\_MKIS\_2025.gpkg"}\NormalTok{,}
                                   \AttributeTok{layer=}\StringTok{"temp\_LielaDiametraKolektori"}\NormalTok{)}
\NormalTok{LielaDiametraKolektori\_all2 }\OtherTok{=}\NormalTok{ LielaDiametraKolektori\_all[}\SpecialCharTok{!}\FunctionTok{st\_is\_empty}\NormalTok{(LielaDiametraKolektori\_all),,drop}\OtherTok{=}\ConstantTok{FALSE}\NormalTok{] }\CommentTok{\# 0}
\FunctionTok{table}\NormalTok{(}\FunctionTok{st\_is\_valid}\NormalTok{(LielaDiametraKolektori\_all2))}


\FunctionTok{write\_sf}\NormalTok{(LielaDiametraKolektori\_all2,}
         \StringTok{"./Geodata/2024/MKIS/MKIS\_2025.gpkg"}\NormalTok{,}
         \AttributeTok{layer=}\StringTok{"LielaDiametraKolektori"}\NormalTok{,}
         \AttributeTok{append=}\ConstantTok{FALSE}\NormalTok{)}
\FunctionTok{rm}\NormalTok{(}\AttributeTok{list=}\FunctionTok{ls}\NormalTok{())}




\CommentTok{\# river pickets {-}{-}{-}{-}}



\NormalTok{link}\OtherTok{=}\StringTok{"https://lvmgeoserver.lvm.lv/geoserver/zmni/ows?"}
\NormalTok{url}\OtherTok{=}\FunctionTok{parse\_url}\NormalTok{(link)}

\NormalTok{url}\SpecialCharTok{$}\NormalTok{query }\OtherTok{\textless{}{-}} \FunctionTok{list}\NormalTok{(}\AttributeTok{service =} \StringTok{"wfs"}\NormalTok{,}\AttributeTok{request =} \StringTok{"GetCapabilities"}\NormalTok{)}
\NormalTok{request }\OtherTok{\textless{}{-}} \FunctionTok{build\_url}\NormalTok{(url)}
\NormalTok{bwk\_client }\OtherTok{\textless{}{-}}\NormalTok{ WFSClient}\SpecialCharTok{$}\FunctionTok{new}\NormalTok{(link,}\AttributeTok{serviceVersion =} \StringTok{"2.0.0"}\NormalTok{)}

\NormalTok{bwk\_client}\SpecialCharTok{$}\FunctionTok{getFeatureTypes}\NormalTok{(}\AttributeTok{pretty =} \ConstantTok{TRUE}\NormalTok{)}

\CommentTok{\# geoms}

\NormalTok{url}\SpecialCharTok{$}\NormalTok{query }\OtherTok{\textless{}{-}} \FunctionTok{list}\NormalTok{(}\AttributeTok{service =} \StringTok{"wfs"}\NormalTok{,}
                  \AttributeTok{request =} \StringTok{"GetFeature"}\NormalTok{,}
                  \AttributeTok{srsName=}\StringTok{"EPSG:3059"}\NormalTok{,}
                  \AttributeTok{typename =} \StringTok{"zmni:zmni\_stateriverspickets"}\NormalTok{,}
                  \AttributeTok{count=}\DecValTok{100}\NormalTok{)}
\NormalTok{request }\OtherTok{\textless{}{-}} \FunctionTok{build\_url}\NormalTok{(url)}

\NormalTok{geometrijam }\OtherTok{\textless{}{-}} \FunctionTok{read\_sf}\NormalTok{(request)}
\NormalTok{geometrijam}



\CommentTok{\# download}
\NormalTok{base\_url }\OtherTok{\textless{}{-}} \StringTok{"https://lvmgeoserver.lvm.lv/geoserver/zmni/ows?"}
\NormalTok{type\_name }\OtherTok{\textless{}{-}} \StringTok{"zmni:zmni\_stateriverspickets"}
\NormalTok{crs\_code }\OtherTok{\textless{}{-}} \DecValTok{3059}
\NormalTok{chunk\_size }\OtherTok{\textless{}{-}} \DecValTok{100000}
\NormalTok{gpkg\_path }\OtherTok{\textless{}{-}} \StringTok{"./Geodata/2024/MKIS/temp\_MKIS\_2025.gpkg"}
\NormalTok{layer\_name }\OtherTok{\textless{}{-}} \StringTok{"temp\_Piketi"}
\NormalTok{i }\OtherTok{\textless{}{-}} \DecValTok{0}

\ControlFlowTok{repeat}\NormalTok{ \{}
  \FunctionTok{message}\NormalTok{(}\StringTok{"Fetching features "}\NormalTok{, i }\SpecialCharTok{*}\NormalTok{ chunk\_size }\SpecialCharTok{+} \DecValTok{1}\NormalTok{, }\StringTok{" to "}\NormalTok{, (i }\SpecialCharTok{+} \DecValTok{1}\NormalTok{) }\SpecialCharTok{*}\NormalTok{ chunk\_size, }\StringTok{"..."}\NormalTok{)}
  
\NormalTok{  query }\OtherTok{\textless{}{-}} \FunctionTok{list}\NormalTok{(}
    \AttributeTok{service =} \StringTok{"WFS"}\NormalTok{,}
    \AttributeTok{version =} \StringTok{"2.0.0"}\NormalTok{,}
    \AttributeTok{request =} \StringTok{"GetFeature"}\NormalTok{,}
    \AttributeTok{typename =}\NormalTok{ type\_name,}
    \AttributeTok{srsName =} \FunctionTok{paste0}\NormalTok{(}\StringTok{"EPSG:"}\NormalTok{, crs\_code),}
    \AttributeTok{count =}\NormalTok{ chunk\_size,}
    \AttributeTok{startIndex =}\NormalTok{ i }\SpecialCharTok{*}\NormalTok{ chunk\_size}
\NormalTok{  )}
  
\NormalTok{  req\_url }\OtherTok{\textless{}{-}} \FunctionTok{modify\_url}\NormalTok{(base\_url, }\AttributeTok{query =}\NormalTok{ query)}
  
  \FunctionTok{try}\NormalTok{(\{}
\NormalTok{    chunk }\OtherTok{\textless{}{-}} \FunctionTok{read\_sf}\NormalTok{(req\_url)}
    \ControlFlowTok{if}\NormalTok{ (}\FunctionTok{nrow}\NormalTok{(chunk) }\SpecialCharTok{==} \DecValTok{0}\NormalTok{) }\ControlFlowTok{break}
    
    \CommentTok{\# Set CRS and cast to MULTILINESTRING, POINT, MULTIPOLYGON}
\NormalTok{    chunk }\OtherTok{\textless{}{-}}\NormalTok{ chunk }\SpecialCharTok{\%\textgreater{}\%}
      \FunctionTok{st\_set\_crs}\NormalTok{(}\FunctionTok{st\_crs}\NormalTok{(crs\_code)) }\SpecialCharTok{\%\textgreater{}\%}
      \FunctionTok{st\_cast}\NormalTok{(}\StringTok{"POINT"}\NormalTok{)}
    
    \CommentTok{\# Write chunk to GeoPackage (append mode after first)}
    \FunctionTok{st\_write}\NormalTok{(}
\NormalTok{      chunk, }
      \AttributeTok{dsn =}\NormalTok{ gpkg\_path,}
      \AttributeTok{layer =}\NormalTok{ layer\_name,}
      \AttributeTok{append =}\NormalTok{ i }\SpecialCharTok{!=} \DecValTok{0}\NormalTok{,}
      \AttributeTok{quiet =} \ConstantTok{FALSE}
\NormalTok{    )}
    
\NormalTok{    i }\OtherTok{\textless{}{-}}\NormalTok{ i }\SpecialCharTok{+} \DecValTok{1}
\NormalTok{  \}, }\AttributeTok{silent =} \ConstantTok{TRUE}\NormalTok{)}
\NormalTok{\}}

\FunctionTok{message}\NormalTok{(}\StringTok{"All chunks written to "}\NormalTok{, gpkg\_path, }\StringTok{" in layer "}\NormalTok{, layer\_name)}

\NormalTok{Piketi\_all}\OtherTok{=}\FunctionTok{st\_read}\NormalTok{(}\StringTok{"./Geodata/2024/MKIS/temp\_MKIS\_2025.gpkg"}\NormalTok{,}
                   \AttributeTok{layer=}\StringTok{"temp\_Piketi"}\NormalTok{)}
\NormalTok{Piketi\_all2 }\OtherTok{=}\NormalTok{ Piketi\_all[}\SpecialCharTok{!}\FunctionTok{st\_is\_empty}\NormalTok{(Piketi\_all),,drop}\OtherTok{=}\ConstantTok{FALSE}\NormalTok{] }\CommentTok{\# 0}
\FunctionTok{table}\NormalTok{(}\FunctionTok{st\_is\_valid}\NormalTok{(Piketi\_all2))}


\FunctionTok{write\_sf}\NormalTok{(Piketi\_all2,}
         \StringTok{"./Geodata/2024/MKIS/MKIS\_2025.gpkg"}\NormalTok{,}
         \AttributeTok{layer=}\StringTok{"Piketi"}\NormalTok{,}
         \AttributeTok{append=}\ConstantTok{FALSE}\NormalTok{)}
\FunctionTok{rm}\NormalTok{(}\AttributeTok{list=}\FunctionTok{ls}\NormalTok{())}


\CommentTok{\# polder pumping stations {-}{-}{-}{-}{-}}


\NormalTok{link}\OtherTok{=}\StringTok{"https://lvmgeoserver.lvm.lv/geoserver/zmni/ows?"}
\NormalTok{url}\OtherTok{=}\FunctionTok{parse\_url}\NormalTok{(link)}

\NormalTok{url}\SpecialCharTok{$}\NormalTok{query }\OtherTok{\textless{}{-}} \FunctionTok{list}\NormalTok{(}\AttributeTok{service =} \StringTok{"wfs"}\NormalTok{,}\AttributeTok{request =} \StringTok{"GetCapabilities"}\NormalTok{)}
\NormalTok{request }\OtherTok{\textless{}{-}} \FunctionTok{build\_url}\NormalTok{(url)}
\NormalTok{bwk\_client }\OtherTok{\textless{}{-}}\NormalTok{ WFSClient}\SpecialCharTok{$}\FunctionTok{new}\NormalTok{(link,}\AttributeTok{serviceVersion =} \StringTok{"2.0.0"}\NormalTok{)}

\NormalTok{bwk\_client}\SpecialCharTok{$}\FunctionTok{getFeatureTypes}\NormalTok{(}\AttributeTok{pretty =} \ConstantTok{TRUE}\NormalTok{)}

\CommentTok{\# geoms}

\NormalTok{url}\SpecialCharTok{$}\NormalTok{query }\OtherTok{\textless{}{-}} \FunctionTok{list}\NormalTok{(}\AttributeTok{service =} \StringTok{"wfs"}\NormalTok{,}
                  \AttributeTok{request =} \StringTok{"GetFeature"}\NormalTok{,}
                  \AttributeTok{srsName=}\StringTok{"EPSG:3059"}\NormalTok{,}
                  \AttributeTok{typename =} \StringTok{"zmni:zmni\_polderpumpingstation"}\NormalTok{,}
                  \AttributeTok{count=}\DecValTok{100}\NormalTok{)}
\NormalTok{request }\OtherTok{\textless{}{-}} \FunctionTok{build\_url}\NormalTok{(url)}

\NormalTok{geometrijam }\OtherTok{\textless{}{-}} \FunctionTok{read\_sf}\NormalTok{(request)}
\NormalTok{geometrijam}



\CommentTok{\# download}
\NormalTok{base\_url }\OtherTok{\textless{}{-}} \StringTok{"https://lvmgeoserver.lvm.lv/geoserver/zmni/ows?"}
\NormalTok{type\_name }\OtherTok{\textless{}{-}} \StringTok{"zmni:zmni\_polderpumpingstation"}
\NormalTok{crs\_code }\OtherTok{\textless{}{-}} \DecValTok{3059}
\NormalTok{chunk\_size }\OtherTok{\textless{}{-}} \DecValTok{100000}
\NormalTok{gpkg\_path }\OtherTok{\textless{}{-}} \StringTok{"./Geodata/2024/MKIS/temp\_MKIS\_2025.gpkg"}
\NormalTok{layer\_name }\OtherTok{\textless{}{-}} \StringTok{"temp\_PolderuSuknuStacijas"}
\NormalTok{i }\OtherTok{\textless{}{-}} \DecValTok{0}

\ControlFlowTok{repeat}\NormalTok{ \{}
  \FunctionTok{message}\NormalTok{(}\StringTok{"Fetching features "}\NormalTok{, i }\SpecialCharTok{*}\NormalTok{ chunk\_size }\SpecialCharTok{+} \DecValTok{1}\NormalTok{, }\StringTok{" to "}\NormalTok{, (i }\SpecialCharTok{+} \DecValTok{1}\NormalTok{) }\SpecialCharTok{*}\NormalTok{ chunk\_size, }\StringTok{"..."}\NormalTok{)}
  
\NormalTok{  query }\OtherTok{\textless{}{-}} \FunctionTok{list}\NormalTok{(}
    \AttributeTok{service =} \StringTok{"WFS"}\NormalTok{,}
    \AttributeTok{version =} \StringTok{"2.0.0"}\NormalTok{,}
    \AttributeTok{request =} \StringTok{"GetFeature"}\NormalTok{,}
    \AttributeTok{typename =}\NormalTok{ type\_name,}
    \AttributeTok{srsName =} \FunctionTok{paste0}\NormalTok{(}\StringTok{"EPSG:"}\NormalTok{, crs\_code),}
    \AttributeTok{count =}\NormalTok{ chunk\_size,}
    \AttributeTok{startIndex =}\NormalTok{ i }\SpecialCharTok{*}\NormalTok{ chunk\_size}
\NormalTok{  )}
  
\NormalTok{  req\_url }\OtherTok{\textless{}{-}} \FunctionTok{modify\_url}\NormalTok{(base\_url, }\AttributeTok{query =}\NormalTok{ query)}
  
  \FunctionTok{try}\NormalTok{(\{}
\NormalTok{    chunk }\OtherTok{\textless{}{-}} \FunctionTok{read\_sf}\NormalTok{(req\_url)}
    \ControlFlowTok{if}\NormalTok{ (}\FunctionTok{nrow}\NormalTok{(chunk) }\SpecialCharTok{==} \DecValTok{0}\NormalTok{) }\ControlFlowTok{break}
    
    \CommentTok{\# Set CRS and cast to MULTILINESTRING, POINT, MULTIPOLYGON}
\NormalTok{    chunk }\OtherTok{\textless{}{-}}\NormalTok{ chunk }\SpecialCharTok{\%\textgreater{}\%}
      \FunctionTok{st\_set\_crs}\NormalTok{(}\FunctionTok{st\_crs}\NormalTok{(crs\_code)) }\SpecialCharTok{\%\textgreater{}\%}
      \FunctionTok{st\_cast}\NormalTok{(}\StringTok{"POINT"}\NormalTok{)}
    
    \CommentTok{\# Write chunk to GeoPackage (append mode after first)}
    \FunctionTok{st\_write}\NormalTok{(}
\NormalTok{      chunk, }
      \AttributeTok{dsn =}\NormalTok{ gpkg\_path,}
      \AttributeTok{layer =}\NormalTok{ layer\_name,}
      \AttributeTok{append =}\NormalTok{ i }\SpecialCharTok{!=} \DecValTok{0}\NormalTok{,}
      \AttributeTok{quiet =} \ConstantTok{FALSE}
\NormalTok{    )}
    
\NormalTok{    i }\OtherTok{\textless{}{-}}\NormalTok{ i }\SpecialCharTok{+} \DecValTok{1}
\NormalTok{  \}, }\AttributeTok{silent =} \ConstantTok{TRUE}\NormalTok{)}
\NormalTok{\}}

\FunctionTok{message}\NormalTok{(}\StringTok{"All chunks written to "}\NormalTok{, gpkg\_path, }\StringTok{" in layer "}\NormalTok{, layer\_name)}

\NormalTok{PolderuSuknuStacijas\_all}\OtherTok{=}\FunctionTok{st\_read}\NormalTok{(}\StringTok{"./Geodata/2024/MKIS/temp\_MKIS\_2025.gpkg"}\NormalTok{,}
                                 \AttributeTok{layer=}\StringTok{"temp\_PolderuSuknuStacijas"}\NormalTok{)}
\NormalTok{PolderuSuknuStacijas\_all2 }\OtherTok{=}\NormalTok{ PolderuSuknuStacijas\_all[}\SpecialCharTok{!}\FunctionTok{st\_is\_empty}\NormalTok{(PolderuSuknuStacijas\_all),,drop}\OtherTok{=}\ConstantTok{FALSE}\NormalTok{] }\CommentTok{\# 0}
\FunctionTok{table}\NormalTok{(}\FunctionTok{st\_is\_valid}\NormalTok{(PolderuSuknuStacijas\_all2))}


\FunctionTok{write\_sf}\NormalTok{(PolderuSuknuStacijas\_all2,}
         \StringTok{"./Geodata/2024/MKIS/MKIS\_2025.gpkg"}\NormalTok{,}
         \AttributeTok{layer=}\StringTok{"PolderuSuknuStacijas"}\NormalTok{,}
         \AttributeTok{append=}\ConstantTok{FALSE}\NormalTok{)}
\FunctionTok{rm}\NormalTok{(}\AttributeTok{list=}\FunctionTok{ls}\NormalTok{())}



\CommentTok{\# polders {-}{-}{-}{-}{-}}


\NormalTok{link}\OtherTok{=}\StringTok{"https://lvmgeoserver.lvm.lv/geoserver/zmni/ows?"}
\NormalTok{url}\OtherTok{=}\FunctionTok{parse\_url}\NormalTok{(link)}

\NormalTok{url}\SpecialCharTok{$}\NormalTok{query }\OtherTok{\textless{}{-}} \FunctionTok{list}\NormalTok{(}\AttributeTok{service =} \StringTok{"wfs"}\NormalTok{,}\AttributeTok{request =} \StringTok{"GetCapabilities"}\NormalTok{)}
\NormalTok{request }\OtherTok{\textless{}{-}} \FunctionTok{build\_url}\NormalTok{(url)}
\NormalTok{bwk\_client }\OtherTok{\textless{}{-}}\NormalTok{ WFSClient}\SpecialCharTok{$}\FunctionTok{new}\NormalTok{(link,}\AttributeTok{serviceVersion =} \StringTok{"2.0.0"}\NormalTok{)}

\NormalTok{bwk\_client}\SpecialCharTok{$}\FunctionTok{getFeatureTypes}\NormalTok{(}\AttributeTok{pretty =} \ConstantTok{TRUE}\NormalTok{)}

\CommentTok{\# geoms}

\NormalTok{url}\SpecialCharTok{$}\NormalTok{query }\OtherTok{\textless{}{-}} \FunctionTok{list}\NormalTok{(}\AttributeTok{service =} \StringTok{"wfs"}\NormalTok{,}
                  \AttributeTok{request =} \StringTok{"GetFeature"}\NormalTok{,}
                  \AttributeTok{srsName=}\StringTok{"EPSG:3059"}\NormalTok{,}
                  \AttributeTok{typename =} \StringTok{"zmni:zmni\_polderterritory"}\NormalTok{,}
                  \AttributeTok{count=}\DecValTok{100}\NormalTok{)}
\NormalTok{request }\OtherTok{\textless{}{-}} \FunctionTok{build\_url}\NormalTok{(url)}

\NormalTok{geometrijam }\OtherTok{\textless{}{-}} \FunctionTok{read\_sf}\NormalTok{(request)}
\NormalTok{geometrijam}

\NormalTok{geometrijas}\OtherTok{=}\FunctionTok{st\_set\_crs}\NormalTok{(geometrijam,}\FunctionTok{st\_crs}\NormalTok{(}\DecValTok{3059}\NormalTok{))}

\FunctionTok{library}\NormalTok{(gdalUtilities)}
\NormalTok{ensure\_multipolygons }\OtherTok{\textless{}{-}} \ControlFlowTok{function}\NormalTok{(X) \{}
\NormalTok{  tmp1 }\OtherTok{\textless{}{-}} \FunctionTok{tempfile}\NormalTok{(}\AttributeTok{fileext =} \StringTok{".gpkg"}\NormalTok{)}
\NormalTok{  tmp2 }\OtherTok{\textless{}{-}} \FunctionTok{tempfile}\NormalTok{(}\AttributeTok{fileext =} \StringTok{".gpkg"}\NormalTok{)}
  \FunctionTok{st\_write}\NormalTok{(X, tmp1)}
  \FunctionTok{ogr2ogr}\NormalTok{(tmp1, tmp2, }\AttributeTok{f =} \StringTok{"GPKG"}\NormalTok{, }\AttributeTok{nlt =} \StringTok{"MULTIPOLYGON"}\NormalTok{)}
\NormalTok{  Y }\OtherTok{\textless{}{-}} \FunctionTok{st\_read}\NormalTok{(tmp2)}
  \FunctionTok{st\_sf}\NormalTok{(}\FunctionTok{st\_drop\_geometry}\NormalTok{(X), }\AttributeTok{geom =} \FunctionTok{st\_geometry}\NormalTok{(Y))}
\NormalTok{\}}
\NormalTok{poligoni }\OtherTok{\textless{}{-}} \FunctionTok{ensure\_multipolygons}\NormalTok{(geometrijas)}
\NormalTok{PolderuTeritorijas\_all2 }\OtherTok{=}\NormalTok{ poligoni[}\SpecialCharTok{!}\FunctionTok{st\_is\_empty}\NormalTok{(poligoni),,drop}\OtherTok{=}\ConstantTok{FALSE}\NormalTok{] }\CommentTok{\# 0}
\FunctionTok{table}\NormalTok{(}\FunctionTok{st\_is\_valid}\NormalTok{(PolderuTeritorijas\_all2))}


\CommentTok{\# download}
\NormalTok{base\_url }\OtherTok{\textless{}{-}} \StringTok{"https://lvmgeoserver.lvm.lv/geoserver/zmni/ows?"}
\NormalTok{type\_name }\OtherTok{\textless{}{-}} \StringTok{"zmni:zmni\_polderterritory"}
\NormalTok{crs\_code }\OtherTok{\textless{}{-}} \DecValTok{3059}
\NormalTok{chunk\_size }\OtherTok{\textless{}{-}} \DecValTok{100000}
\NormalTok{gpkg\_path }\OtherTok{\textless{}{-}} \StringTok{"./Geodata/2024/MKIS/temp\_MKIS\_2025.gpkg"}
\NormalTok{layer\_name }\OtherTok{\textless{}{-}} \StringTok{"temp\_PolderuTeritorijas"}
\NormalTok{i }\OtherTok{\textless{}{-}} \DecValTok{0}

\ControlFlowTok{repeat}\NormalTok{ \{}
  \FunctionTok{message}\NormalTok{(}\StringTok{"Fetching features "}\NormalTok{, i }\SpecialCharTok{*}\NormalTok{ chunk\_size }\SpecialCharTok{+} \DecValTok{1}\NormalTok{, }\StringTok{" to "}\NormalTok{, (i }\SpecialCharTok{+} \DecValTok{1}\NormalTok{) }\SpecialCharTok{*}\NormalTok{ chunk\_size, }\StringTok{"..."}\NormalTok{)}
  
\NormalTok{  query }\OtherTok{\textless{}{-}} \FunctionTok{list}\NormalTok{(}
    \AttributeTok{service =} \StringTok{"WFS"}\NormalTok{,}
    \AttributeTok{version =} \StringTok{"2.0.0"}\NormalTok{,}
    \AttributeTok{request =} \StringTok{"GetFeature"}\NormalTok{,}
    \AttributeTok{typename =}\NormalTok{ type\_name,}
    \AttributeTok{srsName =} \FunctionTok{paste0}\NormalTok{(}\StringTok{"EPSG:"}\NormalTok{, crs\_code),}
    \AttributeTok{count =}\NormalTok{ chunk\_size,}
    \AttributeTok{startIndex =}\NormalTok{ i }\SpecialCharTok{*}\NormalTok{ chunk\_size}
\NormalTok{  )}
  
\NormalTok{  req\_url }\OtherTok{\textless{}{-}} \FunctionTok{modify\_url}\NormalTok{(base\_url, }\AttributeTok{query =}\NormalTok{ query)}
  
  \FunctionTok{try}\NormalTok{(\{}
\NormalTok{    chunk }\OtherTok{\textless{}{-}} \FunctionTok{read\_sf}\NormalTok{(req\_url)}
    \ControlFlowTok{if}\NormalTok{ (}\FunctionTok{nrow}\NormalTok{(chunk) }\SpecialCharTok{==} \DecValTok{0}\NormalTok{) }\ControlFlowTok{break}
    
    \CommentTok{\# Set CRS and cast to MULTILINESTRING, POINT, MULTIPOLYGON}
\NormalTok{    chunk }\OtherTok{\textless{}{-}}\NormalTok{ chunk }\SpecialCharTok{\%\textgreater{}\%}
      \FunctionTok{st\_set\_crs}\NormalTok{(}\FunctionTok{st\_crs}\NormalTok{(crs\_code)) }\SpecialCharTok{\%\textgreater{}\%}
      \FunctionTok{st\_cast}\NormalTok{(}\StringTok{"MULTIPOLYGON"}\NormalTok{)}
    
    \CommentTok{\# Write chunk to GeoPackage (append mode after first)}
    \FunctionTok{st\_write}\NormalTok{(}
\NormalTok{      chunk, }
      \AttributeTok{dsn =}\NormalTok{ gpkg\_path,}
      \AttributeTok{layer =}\NormalTok{ layer\_name,}
      \AttributeTok{append =}\NormalTok{ i }\SpecialCharTok{!=} \DecValTok{0}\NormalTok{,}
      \AttributeTok{quiet =} \ConstantTok{FALSE}
\NormalTok{    )}
    
\NormalTok{    i }\OtherTok{\textless{}{-}}\NormalTok{ i }\SpecialCharTok{+} \DecValTok{1}
\NormalTok{  \}, }\AttributeTok{silent =} \ConstantTok{TRUE}\NormalTok{)}
  \FunctionTok{Sys.sleep}\NormalTok{(}\FloatTok{0.5}\NormalTok{)}
\NormalTok{\}}

\FunctionTok{message}\NormalTok{(}\StringTok{"All chunks written to "}\NormalTok{, gpkg\_path, }\StringTok{" in layer "}\NormalTok{, layer\_name)}


\NormalTok{PolderuTeritorijas\_all}\OtherTok{=}\FunctionTok{st\_read}\NormalTok{(}\StringTok{"./Geodata/2024/MKIS/temp\_MKIS\_2025.gpkg"}\NormalTok{,}
                               \AttributeTok{layer=}\StringTok{"temp\_PolderuTeritorijas"}\NormalTok{)}
\NormalTok{PolderuTeritorijas\_all2 }\OtherTok{=}\NormalTok{ PolderuTeritorijas\_all[}\SpecialCharTok{!}\FunctionTok{st\_is\_empty}\NormalTok{(PolderuTeritorijas\_all),,drop}\OtherTok{=}\ConstantTok{FALSE}\NormalTok{] }\CommentTok{\# 0}
\FunctionTok{table}\NormalTok{(}\FunctionTok{st\_is\_valid}\NormalTok{(PolderuTeritorijas\_all2))}


\FunctionTok{write\_sf}\NormalTok{(PolderuTeritorijas\_all2,}
         \StringTok{"./Geodata/2024/MKIS/MKIS\_2025.gpkg"}\NormalTok{,}
         \AttributeTok{layer=}\StringTok{"PolderuTeritorijas"}\NormalTok{,}
         \AttributeTok{append=}\ConstantTok{FALSE}\NormalTok{)}
\FunctionTok{rm}\NormalTok{(}\AttributeTok{list=}\FunctionTok{ls}\NormalTok{())}


\CommentTok{\# catchment basins {-}{-}{-}{-}}


\NormalTok{link}\OtherTok{=}\StringTok{"https://lvmgeoserver.lvm.lv/geoserver/zmni/ows?"}
\NormalTok{url}\OtherTok{=}\FunctionTok{parse\_url}\NormalTok{(link)}

\NormalTok{url}\SpecialCharTok{$}\NormalTok{query }\OtherTok{\textless{}{-}} \FunctionTok{list}\NormalTok{(}\AttributeTok{service =} \StringTok{"wfs"}\NormalTok{,}\AttributeTok{request =} \StringTok{"GetCapabilities"}\NormalTok{)}
\NormalTok{request }\OtherTok{\textless{}{-}} \FunctionTok{build\_url}\NormalTok{(url)}
\NormalTok{bwk\_client }\OtherTok{\textless{}{-}}\NormalTok{ WFSClient}\SpecialCharTok{$}\FunctionTok{new}\NormalTok{(link,}\AttributeTok{serviceVersion =} \StringTok{"2.0.0"}\NormalTok{)}

\NormalTok{bwk\_client}\SpecialCharTok{$}\FunctionTok{getFeatureTypes}\NormalTok{(}\AttributeTok{pretty =} \ConstantTok{TRUE}\NormalTok{)}

\CommentTok{\# geoms}

\NormalTok{url}\SpecialCharTok{$}\NormalTok{query }\OtherTok{\textless{}{-}} \FunctionTok{list}\NormalTok{(}\AttributeTok{service =} \StringTok{"wfs"}\NormalTok{,}
                  \AttributeTok{request =} \StringTok{"GetFeature"}\NormalTok{,}
                  \AttributeTok{srsName=}\StringTok{"EPSG:3059"}\NormalTok{,}
                  \AttributeTok{typename =} \StringTok{"zmni:zmni\_catchment"}\NormalTok{,}
                  \AttributeTok{count=}\DecValTok{100}\NormalTok{)}
\NormalTok{request }\OtherTok{\textless{}{-}} \FunctionTok{build\_url}\NormalTok{(url)}

\NormalTok{geometrijam }\OtherTok{\textless{}{-}} \FunctionTok{read\_sf}\NormalTok{(request)}
\NormalTok{geometrijam}



\CommentTok{\# download}
\NormalTok{base\_url }\OtherTok{\textless{}{-}} \StringTok{"https://lvmgeoserver.lvm.lv/geoserver/zmni/ows?"}
\NormalTok{type\_name }\OtherTok{\textless{}{-}} \StringTok{"zmni:zmni\_catchment"}
\NormalTok{crs\_code }\OtherTok{\textless{}{-}} \DecValTok{3059}
\NormalTok{chunk\_size }\OtherTok{\textless{}{-}} \DecValTok{100000}
\NormalTok{gpkg\_path }\OtherTok{\textless{}{-}} \StringTok{"./Geodata/2024/MKIS/temp\_MKIS\_2025.gpkg"}
\NormalTok{layer\_name }\OtherTok{\textless{}{-}} \StringTok{"temp\_SatecesBaseini"}
\NormalTok{i }\OtherTok{\textless{}{-}} \DecValTok{0}

\ControlFlowTok{repeat}\NormalTok{ \{}
  \FunctionTok{message}\NormalTok{(}\StringTok{"Fetching features "}\NormalTok{, i }\SpecialCharTok{*}\NormalTok{ chunk\_size }\SpecialCharTok{+} \DecValTok{1}\NormalTok{, }\StringTok{" to "}\NormalTok{, (i }\SpecialCharTok{+} \DecValTok{1}\NormalTok{) }\SpecialCharTok{*}\NormalTok{ chunk\_size, }\StringTok{"..."}\NormalTok{)}
  
\NormalTok{  query }\OtherTok{\textless{}{-}} \FunctionTok{list}\NormalTok{(}
    \AttributeTok{service =} \StringTok{"WFS"}\NormalTok{,}
    \AttributeTok{version =} \StringTok{"2.0.0"}\NormalTok{,}
    \AttributeTok{request =} \StringTok{"GetFeature"}\NormalTok{,}
    \AttributeTok{typename =}\NormalTok{ type\_name,}
    \AttributeTok{srsName =} \FunctionTok{paste0}\NormalTok{(}\StringTok{"EPSG:"}\NormalTok{, crs\_code),}
    \AttributeTok{count =}\NormalTok{ chunk\_size,}
    \AttributeTok{startIndex =}\NormalTok{ i }\SpecialCharTok{*}\NormalTok{ chunk\_size}
\NormalTok{  )}
  
\NormalTok{  req\_url }\OtherTok{\textless{}{-}} \FunctionTok{modify\_url}\NormalTok{(base\_url, }\AttributeTok{query =}\NormalTok{ query)}
  
  \FunctionTok{try}\NormalTok{(\{}
\NormalTok{    chunk }\OtherTok{\textless{}{-}} \FunctionTok{read\_sf}\NormalTok{(req\_url)}
    \ControlFlowTok{if}\NormalTok{ (}\FunctionTok{nrow}\NormalTok{(chunk) }\SpecialCharTok{==} \DecValTok{0}\NormalTok{) }\ControlFlowTok{break}
    
    \CommentTok{\# Set CRS and cast to MULTILINESTRING, POINT, MULTIPOLYGON}
\NormalTok{    chunk }\OtherTok{\textless{}{-}}\NormalTok{ chunk }\SpecialCharTok{\%\textgreater{}\%}
      \FunctionTok{st\_set\_crs}\NormalTok{(}\FunctionTok{st\_crs}\NormalTok{(crs\_code))}
    
\NormalTok{    ensure\_multipolygons }\OtherTok{\textless{}{-}} \ControlFlowTok{function}\NormalTok{(X) \{}
\NormalTok{      tmp1 }\OtherTok{\textless{}{-}} \FunctionTok{tempfile}\NormalTok{(}\AttributeTok{fileext =} \StringTok{".gpkg"}\NormalTok{)}
\NormalTok{      tmp2 }\OtherTok{\textless{}{-}} \FunctionTok{tempfile}\NormalTok{(}\AttributeTok{fileext =} \StringTok{".gpkg"}\NormalTok{)}
      \FunctionTok{st\_write}\NormalTok{(X, tmp1)}
      \FunctionTok{ogr2ogr}\NormalTok{(tmp1, tmp2, }\AttributeTok{f =} \StringTok{"GPKG"}\NormalTok{, }\AttributeTok{nlt =} \StringTok{"MULTIPOLYGON"}\NormalTok{)}
\NormalTok{      Y }\OtherTok{\textless{}{-}} \FunctionTok{st\_read}\NormalTok{(tmp2)}
      \FunctionTok{st\_sf}\NormalTok{(}\FunctionTok{st\_drop\_geometry}\NormalTok{(X), }\AttributeTok{geom =} \FunctionTok{st\_geometry}\NormalTok{(Y))}
\NormalTok{    \}}
\NormalTok{    chunk }\OtherTok{\textless{}{-}} \FunctionTok{ensure\_multipolygons}\NormalTok{(chunk)}
    
    
    \CommentTok{\# Write chunk to GeoPackage (append mode after first)}
    \FunctionTok{st\_write}\NormalTok{(}
\NormalTok{      chunk, }
      \AttributeTok{dsn =}\NormalTok{ gpkg\_path,}
      \AttributeTok{layer =}\NormalTok{ layer\_name,}
      \AttributeTok{append =}\NormalTok{ i }\SpecialCharTok{!=} \DecValTok{0}\NormalTok{,}
      \AttributeTok{quiet =} \ConstantTok{FALSE}
\NormalTok{    )}
    
\NormalTok{    i }\OtherTok{\textless{}{-}}\NormalTok{ i }\SpecialCharTok{+} \DecValTok{1}
\NormalTok{  \}, }\AttributeTok{silent =} \ConstantTok{TRUE}\NormalTok{)}
  \FunctionTok{Sys.sleep}\NormalTok{(}\FloatTok{0.5}\NormalTok{)}
\NormalTok{\}}

\FunctionTok{message}\NormalTok{(}\StringTok{"All chunks written to "}\NormalTok{, gpkg\_path, }\StringTok{" in layer "}\NormalTok{, layer\_name)}


\NormalTok{SatecesBaseini\_all}\OtherTok{=}\FunctionTok{st\_read}\NormalTok{(}\StringTok{"./Geodata/2024/MKIS/temp\_MKIS\_2025.gpkg"}\NormalTok{,}
                           \AttributeTok{layer=}\StringTok{"temp\_SatecesBaseini"}\NormalTok{)}
\NormalTok{SatecesBaseini\_all2 }\OtherTok{=}\NormalTok{ SatecesBaseini\_all[}\SpecialCharTok{!}\FunctionTok{st\_is\_empty}\NormalTok{(SatecesBaseini\_all),,drop}\OtherTok{=}\ConstantTok{FALSE}\NormalTok{] }\CommentTok{\# 0}
\FunctionTok{table}\NormalTok{(}\FunctionTok{st\_is\_valid}\NormalTok{(SatecesBaseini\_all2))}

\NormalTok{SatecesBaseini\_all3}\OtherTok{=}\FunctionTok{st\_make\_valid}\NormalTok{(SatecesBaseini\_all2)}
\FunctionTok{table}\NormalTok{(}\FunctionTok{st\_is\_valid}\NormalTok{(SatecesBaseini\_all3))}
\NormalTok{SatecesBaseini\_all3}

\FunctionTok{write\_sf}\NormalTok{(SatecesBaseini\_all3,}
         \StringTok{"./Geodata/2024/MKIS/MKIS\_2025.gpkg"}\NormalTok{,}
         \AttributeTok{layer=}\StringTok{"SatecesBaseini"}\NormalTok{,}
         \AttributeTok{append=}\ConstantTok{FALSE}\NormalTok{)}
\FunctionTok{rm}\NormalTok{(}\AttributeTok{list=}\FunctionTok{ls}\NormalTok{())}


\CommentTok{\# drenage connection points {-}{-}{-}{-}}


\NormalTok{link}\OtherTok{=}\StringTok{"https://lvmgeoserver.lvm.lv/geoserver/zmni/ows?"}
\NormalTok{url}\OtherTok{=}\FunctionTok{parse\_url}\NormalTok{(link)}

\NormalTok{url}\SpecialCharTok{$}\NormalTok{query }\OtherTok{\textless{}{-}} \FunctionTok{list}\NormalTok{(}\AttributeTok{service =} \StringTok{"wfs"}\NormalTok{,}\AttributeTok{request =} \StringTok{"GetCapabilities"}\NormalTok{)}
\NormalTok{request }\OtherTok{\textless{}{-}} \FunctionTok{build\_url}\NormalTok{(url)}
\NormalTok{bwk\_client }\OtherTok{\textless{}{-}}\NormalTok{ WFSClient}\SpecialCharTok{$}\FunctionTok{new}\NormalTok{(link,}\AttributeTok{serviceVersion =} \StringTok{"2.0.0"}\NormalTok{)}

\NormalTok{bwk\_client}\SpecialCharTok{$}\FunctionTok{getFeatureTypes}\NormalTok{(}\AttributeTok{pretty =} \ConstantTok{TRUE}\NormalTok{)}

\CommentTok{\# geoms}

\NormalTok{url}\SpecialCharTok{$}\NormalTok{query }\OtherTok{\textless{}{-}} \FunctionTok{list}\NormalTok{(}\AttributeTok{service =} \StringTok{"wfs"}\NormalTok{,}
                  \AttributeTok{request =} \StringTok{"GetFeature"}\NormalTok{,}
                  \AttributeTok{srsName=}\StringTok{"EPSG:3059"}\NormalTok{,}
                  \AttributeTok{typename =} \StringTok{"zmni:zmni\_connectionpoints"}\NormalTok{,}
                  \AttributeTok{count=}\DecValTok{100}\NormalTok{)}
\NormalTok{request }\OtherTok{\textless{}{-}} \FunctionTok{build\_url}\NormalTok{(url)}

\NormalTok{geometrijam }\OtherTok{\textless{}{-}} \FunctionTok{read\_sf}\NormalTok{(request)}
\NormalTok{geometrijam}



\CommentTok{\# download}
\NormalTok{base\_url }\OtherTok{\textless{}{-}} \StringTok{"https://lvmgeoserver.lvm.lv/geoserver/zmni/ows?"}
\NormalTok{type\_name }\OtherTok{\textless{}{-}} \StringTok{"zmni:zmni\_connectionpoints"}
\NormalTok{crs\_code }\OtherTok{\textless{}{-}} \DecValTok{3059}
\NormalTok{chunk\_size }\OtherTok{\textless{}{-}} \DecValTok{100000}
\NormalTok{gpkg\_path }\OtherTok{\textless{}{-}} \StringTok{"./Geodata/2024/MKIS/temp\_MKIS\_2025.gpkg"}
\NormalTok{layer\_name }\OtherTok{\textless{}{-}} \StringTok{"temp\_Savienojumi"}
\NormalTok{i }\OtherTok{\textless{}{-}} \DecValTok{0}

\ControlFlowTok{repeat}\NormalTok{ \{}
  \FunctionTok{message}\NormalTok{(}\StringTok{"Fetching features "}\NormalTok{, i }\SpecialCharTok{*}\NormalTok{ chunk\_size }\SpecialCharTok{+} \DecValTok{1}\NormalTok{, }\StringTok{" to "}\NormalTok{, (i }\SpecialCharTok{+} \DecValTok{1}\NormalTok{) }\SpecialCharTok{*}\NormalTok{ chunk\_size, }\StringTok{"..."}\NormalTok{)}
  
\NormalTok{  query }\OtherTok{\textless{}{-}} \FunctionTok{list}\NormalTok{(}
    \AttributeTok{service =} \StringTok{"WFS"}\NormalTok{,}
    \AttributeTok{version =} \StringTok{"2.0.0"}\NormalTok{,}
    \AttributeTok{request =} \StringTok{"GetFeature"}\NormalTok{,}
    \AttributeTok{typename =}\NormalTok{ type\_name,}
    \AttributeTok{srsName =} \FunctionTok{paste0}\NormalTok{(}\StringTok{"EPSG:"}\NormalTok{, crs\_code),}
    \AttributeTok{count =}\NormalTok{ chunk\_size,}
    \AttributeTok{startIndex =}\NormalTok{ i }\SpecialCharTok{*}\NormalTok{ chunk\_size}
\NormalTok{  )}
  
\NormalTok{  req\_url }\OtherTok{\textless{}{-}} \FunctionTok{modify\_url}\NormalTok{(base\_url, }\AttributeTok{query =}\NormalTok{ query)}
  
  \FunctionTok{try}\NormalTok{(\{}
\NormalTok{    chunk }\OtherTok{\textless{}{-}} \FunctionTok{read\_sf}\NormalTok{(req\_url)}
    \ControlFlowTok{if}\NormalTok{ (}\FunctionTok{nrow}\NormalTok{(chunk) }\SpecialCharTok{==} \DecValTok{0}\NormalTok{) }\ControlFlowTok{break}
    
    \CommentTok{\# Set CRS and cast to MULTILINESTRING, POINT, MULTIPOLYGON}
\NormalTok{    chunk }\OtherTok{\textless{}{-}}\NormalTok{ chunk }\SpecialCharTok{\%\textgreater{}\%}
      \FunctionTok{st\_set\_crs}\NormalTok{(}\FunctionTok{st\_crs}\NormalTok{(crs\_code))}
    
\NormalTok{    chunk}\OtherTok{=}\FunctionTok{st\_cast}\NormalTok{(chunk,}\StringTok{"POINT"}\NormalTok{)}
    
    \CommentTok{\# Write chunk to GeoPackage (append mode after first)}
    \FunctionTok{st\_write}\NormalTok{(}
\NormalTok{      chunk, }
      \AttributeTok{dsn =}\NormalTok{ gpkg\_path,}
      \AttributeTok{layer =}\NormalTok{ layer\_name,}
      \AttributeTok{append =}\NormalTok{ i }\SpecialCharTok{!=} \DecValTok{0}\NormalTok{,}
      \AttributeTok{quiet =} \ConstantTok{FALSE}
\NormalTok{    )}
    
\NormalTok{    i }\OtherTok{\textless{}{-}}\NormalTok{ i }\SpecialCharTok{+} \DecValTok{1}
\NormalTok{  \}, }\AttributeTok{silent =} \ConstantTok{TRUE}\NormalTok{)}
  \FunctionTok{Sys.sleep}\NormalTok{(}\FloatTok{0.5}\NormalTok{)}
\NormalTok{\}}

\FunctionTok{message}\NormalTok{(}\StringTok{"All chunks written to "}\NormalTok{, gpkg\_path, }\StringTok{" in layer "}\NormalTok{, layer\_name)}


\NormalTok{Savienojumi\_all}\OtherTok{=}\FunctionTok{st\_read}\NormalTok{(}\StringTok{"./Geodata/2024/MKIS/temp\_MKIS\_2025.gpkg"}\NormalTok{,}
                        \AttributeTok{layer=}\StringTok{"temp\_Savienojumi"}\NormalTok{)}
\NormalTok{Savienojumi\_all2 }\OtherTok{=}\NormalTok{ Savienojumi\_all[}\SpecialCharTok{!}\FunctionTok{st\_is\_empty}\NormalTok{(Savienojumi\_all),,drop}\OtherTok{=}\ConstantTok{FALSE}\NormalTok{] }\CommentTok{\# 0}
\FunctionTok{table}\NormalTok{(}\FunctionTok{st\_is\_valid}\NormalTok{(Savienojumi\_all2))}


\FunctionTok{write\_sf}\NormalTok{(Savienojumi\_all2,}
         \StringTok{"./Geodata/2024/MKIS/MKIS\_2025.gpkg"}\NormalTok{,}
         \AttributeTok{layer=}\StringTok{"Savienojumi"}\NormalTok{,}
         \AttributeTok{append=}\ConstantTok{FALSE}\NormalTok{)}
\FunctionTok{rm}\NormalTok{(}\AttributeTok{list=}\FunctionTok{ls}\NormalTok{())}


\CommentTok{\# state controlled rivers {-}{-}{-}{-}{-}}


\NormalTok{link}\OtherTok{=}\StringTok{"https://lvmgeoserver.lvm.lv/geoserver/zmni/ows?"}
\NormalTok{url}\OtherTok{=}\FunctionTok{parse\_url}\NormalTok{(link)}

\NormalTok{url}\SpecialCharTok{$}\NormalTok{query }\OtherTok{\textless{}{-}} \FunctionTok{list}\NormalTok{(}\AttributeTok{service =} \StringTok{"wfs"}\NormalTok{,}\AttributeTok{request =} \StringTok{"GetCapabilities"}\NormalTok{)}
\NormalTok{request }\OtherTok{\textless{}{-}} \FunctionTok{build\_url}\NormalTok{(url)}
\NormalTok{bwk\_client }\OtherTok{\textless{}{-}}\NormalTok{ WFSClient}\SpecialCharTok{$}\FunctionTok{new}\NormalTok{(link,}\AttributeTok{serviceVersion =} \StringTok{"2.0.0"}\NormalTok{)}

\NormalTok{bwk\_client}\SpecialCharTok{$}\FunctionTok{getFeatureTypes}\NormalTok{(}\AttributeTok{pretty =} \ConstantTok{TRUE}\NormalTok{)}

\CommentTok{\# geoms}

\NormalTok{url}\SpecialCharTok{$}\NormalTok{query }\OtherTok{\textless{}{-}} \FunctionTok{list}\NormalTok{(}\AttributeTok{service =} \StringTok{"wfs"}\NormalTok{,}
                  \AttributeTok{request =} \StringTok{"GetFeature"}\NormalTok{,}
                  \AttributeTok{srsName=}\StringTok{"EPSG:3059"}\NormalTok{,}
                  \AttributeTok{typename =} \StringTok{"zmni:zmni\_statecontrolledrivers"}\NormalTok{,}
                  \AttributeTok{count=}\DecValTok{100}\NormalTok{)}
\NormalTok{request }\OtherTok{\textless{}{-}} \FunctionTok{build\_url}\NormalTok{(url)}

\NormalTok{geometrijam }\OtherTok{\textless{}{-}} \FunctionTok{read\_sf}\NormalTok{(request)}
\NormalTok{geometrijam}



\CommentTok{\# download}
\NormalTok{base\_url }\OtherTok{\textless{}{-}} \StringTok{"https://lvmgeoserver.lvm.lv/geoserver/zmni/ows?"}
\NormalTok{type\_name }\OtherTok{\textless{}{-}} \StringTok{"zmni:zmni\_statecontrolledrivers"}
\NormalTok{crs\_code }\OtherTok{\textless{}{-}} \DecValTok{3059}
\NormalTok{chunk\_size }\OtherTok{\textless{}{-}} \DecValTok{100000}
\NormalTok{gpkg\_path }\OtherTok{\textless{}{-}} \StringTok{"./Geodata/2024/MKIS/temp\_MKIS\_2025.gpkg"}
\NormalTok{layer\_name }\OtherTok{\textless{}{-}} \StringTok{"temp\_ValstsNozimesUdensnotekas"}
\NormalTok{i }\OtherTok{\textless{}{-}} \DecValTok{0}

\ControlFlowTok{repeat}\NormalTok{ \{}
  \FunctionTok{message}\NormalTok{(}\StringTok{"Fetching features "}\NormalTok{, i }\SpecialCharTok{*}\NormalTok{ chunk\_size }\SpecialCharTok{+} \DecValTok{1}\NormalTok{, }\StringTok{" to "}\NormalTok{, (i }\SpecialCharTok{+} \DecValTok{1}\NormalTok{) }\SpecialCharTok{*}\NormalTok{ chunk\_size, }\StringTok{"..."}\NormalTok{)}
  
\NormalTok{  query }\OtherTok{\textless{}{-}} \FunctionTok{list}\NormalTok{(}
    \AttributeTok{service =} \StringTok{"WFS"}\NormalTok{,}
    \AttributeTok{version =} \StringTok{"2.0.0"}\NormalTok{,}
    \AttributeTok{request =} \StringTok{"GetFeature"}\NormalTok{,}
    \AttributeTok{typename =}\NormalTok{ type\_name,}
    \AttributeTok{srsName =} \FunctionTok{paste0}\NormalTok{(}\StringTok{"EPSG:"}\NormalTok{, crs\_code),}
    \AttributeTok{count =}\NormalTok{ chunk\_size,}
    \AttributeTok{startIndex =}\NormalTok{ i }\SpecialCharTok{*}\NormalTok{ chunk\_size}
\NormalTok{  )}
  
\NormalTok{  req\_url }\OtherTok{\textless{}{-}} \FunctionTok{modify\_url}\NormalTok{(base\_url, }\AttributeTok{query =}\NormalTok{ query)}
  
  \FunctionTok{try}\NormalTok{(\{}
\NormalTok{    chunk }\OtherTok{\textless{}{-}} \FunctionTok{read\_sf}\NormalTok{(req\_url)}
    \ControlFlowTok{if}\NormalTok{ (}\FunctionTok{nrow}\NormalTok{(chunk) }\SpecialCharTok{==} \DecValTok{0}\NormalTok{) }\ControlFlowTok{break}
    
    \CommentTok{\# Set CRS and cast to MULTILINESTRING, POINT, MULTIPOLYGON}
\NormalTok{    chunk }\OtherTok{\textless{}{-}}\NormalTok{ chunk }\SpecialCharTok{\%\textgreater{}\%}
      \FunctionTok{st\_set\_crs}\NormalTok{(}\FunctionTok{st\_crs}\NormalTok{(crs\_code))}
    
\NormalTok{    chunk}\OtherTok{=}\FunctionTok{st\_cast}\NormalTok{(chunk,}\StringTok{"MULTILINESTRING"}\NormalTok{)}
    
    \CommentTok{\# Write chunk to GeoPackage (append mode after first)}
    \FunctionTok{st\_write}\NormalTok{(}
\NormalTok{      chunk, }
      \AttributeTok{dsn =}\NormalTok{ gpkg\_path,}
      \AttributeTok{layer =}\NormalTok{ layer\_name,}
      \AttributeTok{append =}\NormalTok{ i }\SpecialCharTok{!=} \DecValTok{0}\NormalTok{,}
      \AttributeTok{quiet =} \ConstantTok{FALSE}
\NormalTok{    )}
    
\NormalTok{    i }\OtherTok{\textless{}{-}}\NormalTok{ i }\SpecialCharTok{+} \DecValTok{1}
\NormalTok{  \}, }\AttributeTok{silent =} \ConstantTok{TRUE}\NormalTok{)}
  \FunctionTok{Sys.sleep}\NormalTok{(}\FloatTok{0.5}\NormalTok{)}
\NormalTok{\}}

\FunctionTok{message}\NormalTok{(}\StringTok{"All chunks written to "}\NormalTok{, gpkg\_path, }\StringTok{" in layer "}\NormalTok{, layer\_name)}


\NormalTok{ValstsNozimesUdensnotekas\_all}\OtherTok{=}\FunctionTok{st\_read}\NormalTok{(}\StringTok{"./Geodata/2024/MKIS/temp\_MKIS\_2025.gpkg"}\NormalTok{,}
                                      \AttributeTok{layer=}\StringTok{"temp\_ValstsNozimesUdensnotekas"}\NormalTok{)}
\NormalTok{ValstsNozimesUdensnotekas\_all2 }\OtherTok{=}\NormalTok{ ValstsNozimesUdensnotekas\_all[}\SpecialCharTok{!}\FunctionTok{st\_is\_empty}\NormalTok{(ValstsNozimesUdensnotekas\_all),,drop}\OtherTok{=}\ConstantTok{FALSE}\NormalTok{] }\CommentTok{\# 0}
\FunctionTok{table}\NormalTok{(}\FunctionTok{st\_is\_valid}\NormalTok{(ValstsNozimesUdensnotekas\_all2))}


\FunctionTok{write\_sf}\NormalTok{(ValstsNozimesUdensnotekas\_all2,}
         \StringTok{"./Geodata/2024/MKIS/MKIS\_2025.gpkg"}\NormalTok{,}
         \AttributeTok{layer=}\StringTok{"ValstsNozimesUdensnotekas"}\NormalTok{,}
         \AttributeTok{append=}\ConstantTok{FALSE}\NormalTok{)}
\FunctionTok{rm}\NormalTok{(}\AttributeTok{list=}\FunctionTok{ls}\NormalTok{())}


\CommentTok{\# zmni regions {-}{-}{-}{-}}


\NormalTok{link}\OtherTok{=}\StringTok{"https://lvmgeoserver.lvm.lv/geoserver/zmni/ows?"}
\NormalTok{url}\OtherTok{=}\FunctionTok{parse\_url}\NormalTok{(link)}

\NormalTok{url}\SpecialCharTok{$}\NormalTok{query }\OtherTok{\textless{}{-}} \FunctionTok{list}\NormalTok{(}\AttributeTok{service =} \StringTok{"wfs"}\NormalTok{,}\AttributeTok{request =} \StringTok{"GetCapabilities"}\NormalTok{)}
\NormalTok{request }\OtherTok{\textless{}{-}} \FunctionTok{build\_url}\NormalTok{(url)}
\NormalTok{bwk\_client }\OtherTok{\textless{}{-}}\NormalTok{ WFSClient}\SpecialCharTok{$}\FunctionTok{new}\NormalTok{(link,}\AttributeTok{serviceVersion =} \StringTok{"2.0.0"}\NormalTok{)}

\NormalTok{bwk\_client}\SpecialCharTok{$}\FunctionTok{getFeatureTypes}\NormalTok{(}\AttributeTok{pretty =} \ConstantTok{TRUE}\NormalTok{)}

\CommentTok{\# geoms}

\NormalTok{url}\SpecialCharTok{$}\NormalTok{query }\OtherTok{\textless{}{-}} \FunctionTok{list}\NormalTok{(}\AttributeTok{service =} \StringTok{"wfs"}\NormalTok{,}
                  \AttributeTok{request =} \StringTok{"GetFeature"}\NormalTok{,}
                  \AttributeTok{srsName=}\StringTok{"EPSG:3059"}\NormalTok{,}
                  \AttributeTok{typename =} \StringTok{"zmni:zmni\_zmniregion"}\NormalTok{,}
                  \AttributeTok{count=}\DecValTok{100}\NormalTok{)}
\NormalTok{request }\OtherTok{\textless{}{-}} \FunctionTok{build\_url}\NormalTok{(url)}

\NormalTok{geometrijam }\OtherTok{\textless{}{-}} \FunctionTok{read\_sf}\NormalTok{(request)}
\NormalTok{geometrijam}


\FunctionTok{library}\NormalTok{(gdalUtilities)}


\CommentTok{\# download}
\NormalTok{base\_url }\OtherTok{\textless{}{-}} \StringTok{"https://lvmgeoserver.lvm.lv/geoserver/zmni/ows?"}
\NormalTok{type\_name }\OtherTok{\textless{}{-}} \StringTok{"zmni:zmni\_zmniregion"}
\NormalTok{crs\_code }\OtherTok{\textless{}{-}} \DecValTok{3059}
\NormalTok{chunk\_size }\OtherTok{\textless{}{-}} \DecValTok{100000}
\NormalTok{gpkg\_path }\OtherTok{\textless{}{-}} \StringTok{"./Geodata/2024/MKIS/temp\_MKIS\_2025.gpkg"}
\NormalTok{layer\_name }\OtherTok{\textless{}{-}} \StringTok{"temp\_ZMNIRegions"}
\NormalTok{i }\OtherTok{\textless{}{-}} \DecValTok{0}

\ControlFlowTok{repeat}\NormalTok{ \{}
  \FunctionTok{message}\NormalTok{(}\StringTok{"Fetching features "}\NormalTok{, i }\SpecialCharTok{*}\NormalTok{ chunk\_size }\SpecialCharTok{+} \DecValTok{1}\NormalTok{, }\StringTok{" to "}\NormalTok{, (i }\SpecialCharTok{+} \DecValTok{1}\NormalTok{) }\SpecialCharTok{*}\NormalTok{ chunk\_size, }\StringTok{"..."}\NormalTok{)}
  
\NormalTok{  query }\OtherTok{\textless{}{-}} \FunctionTok{list}\NormalTok{(}
    \AttributeTok{service =} \StringTok{"WFS"}\NormalTok{,}
    \AttributeTok{version =} \StringTok{"2.0.0"}\NormalTok{,}
    \AttributeTok{request =} \StringTok{"GetFeature"}\NormalTok{,}
    \AttributeTok{typename =}\NormalTok{ type\_name,}
    \AttributeTok{srsName =} \FunctionTok{paste0}\NormalTok{(}\StringTok{"EPSG:"}\NormalTok{, crs\_code),}
    \AttributeTok{count =}\NormalTok{ chunk\_size,}
    \AttributeTok{startIndex =}\NormalTok{ i }\SpecialCharTok{*}\NormalTok{ chunk\_size}
\NormalTok{  )}
  
\NormalTok{  req\_url }\OtherTok{\textless{}{-}} \FunctionTok{modify\_url}\NormalTok{(base\_url, }\AttributeTok{query =}\NormalTok{ query)}
  
  \FunctionTok{try}\NormalTok{(\{}
\NormalTok{    chunk }\OtherTok{\textless{}{-}} \FunctionTok{read\_sf}\NormalTok{(req\_url)}
    \ControlFlowTok{if}\NormalTok{ (}\FunctionTok{nrow}\NormalTok{(chunk) }\SpecialCharTok{==} \DecValTok{0}\NormalTok{) }\ControlFlowTok{break}
    
    \CommentTok{\# Set CRS and cast to MULTILINESTRING, POINT, MULTIPOLYGON}
\NormalTok{    chunk }\OtherTok{\textless{}{-}}\NormalTok{ chunk }\SpecialCharTok{\%\textgreater{}\%}
      \FunctionTok{st\_set\_crs}\NormalTok{(}\FunctionTok{st\_crs}\NormalTok{(crs\_code))}
    
\NormalTok{    ensure\_multipolygons }\OtherTok{\textless{}{-}} \ControlFlowTok{function}\NormalTok{(X) \{}
\NormalTok{      tmp1 }\OtherTok{\textless{}{-}} \FunctionTok{tempfile}\NormalTok{(}\AttributeTok{fileext =} \StringTok{".gpkg"}\NormalTok{)}
\NormalTok{      tmp2 }\OtherTok{\textless{}{-}} \FunctionTok{tempfile}\NormalTok{(}\AttributeTok{fileext =} \StringTok{".gpkg"}\NormalTok{)}
      \FunctionTok{st\_write}\NormalTok{(X, tmp1)}
      \FunctionTok{ogr2ogr}\NormalTok{(tmp1, tmp2, }\AttributeTok{f =} \StringTok{"GPKG"}\NormalTok{, }\AttributeTok{nlt =} \StringTok{"MULTIPOLYGON"}\NormalTok{)}
\NormalTok{      Y }\OtherTok{\textless{}{-}} \FunctionTok{st\_read}\NormalTok{(tmp2)}
      \FunctionTok{st\_sf}\NormalTok{(}\FunctionTok{st\_drop\_geometry}\NormalTok{(X), }\AttributeTok{geom =} \FunctionTok{st\_geometry}\NormalTok{(Y))}
\NormalTok{    \}}
\NormalTok{    chunk }\OtherTok{\textless{}{-}} \FunctionTok{ensure\_multipolygons}\NormalTok{(chunk)}
    
    
    \CommentTok{\# Write chunk to GeoPackage (append mode after first)}
    \FunctionTok{st\_write}\NormalTok{(}
\NormalTok{      chunk, }
      \AttributeTok{dsn =}\NormalTok{ gpkg\_path,}
      \AttributeTok{layer =}\NormalTok{ layer\_name,}
      \AttributeTok{append =}\NormalTok{ i }\SpecialCharTok{!=} \DecValTok{0}\NormalTok{,}
      \AttributeTok{quiet =} \ConstantTok{FALSE}
\NormalTok{    )}
    
\NormalTok{    i }\OtherTok{\textless{}{-}}\NormalTok{ i }\SpecialCharTok{+} \DecValTok{1}
\NormalTok{  \}, }\AttributeTok{silent =} \ConstantTok{TRUE}\NormalTok{)}
  \FunctionTok{Sys.sleep}\NormalTok{(}\FloatTok{0.5}\NormalTok{)}
\NormalTok{\}}

\FunctionTok{message}\NormalTok{(}\StringTok{"All chunks written to "}\NormalTok{, gpkg\_path, }\StringTok{" in layer "}\NormalTok{, layer\_name)}


\NormalTok{ZMNIRegions\_all}\OtherTok{=}\FunctionTok{st\_read}\NormalTok{(}\StringTok{"./Geodata/2024/MKIS/temp\_MKIS\_2025.gpkg"}\NormalTok{,}
                        \AttributeTok{layer=}\StringTok{"temp\_ZMNIRegions"}\NormalTok{)}
\NormalTok{ZMNIRegions\_all2 }\OtherTok{=}\NormalTok{ ZMNIRegions\_all[}\SpecialCharTok{!}\FunctionTok{st\_is\_empty}\NormalTok{(ZMNIRegions\_all),,drop}\OtherTok{=}\ConstantTok{FALSE}\NormalTok{] }\CommentTok{\# 0}
\FunctionTok{table}\NormalTok{(}\FunctionTok{st\_is\_valid}\NormalTok{(ZMNIRegions\_all2))}


\FunctionTok{write\_sf}\NormalTok{(ZMNIRegions\_all2,}
         \StringTok{"./Geodata/2024/MKIS/MKIS\_2025.gpkg"}\NormalTok{,}
         \AttributeTok{layer=}\StringTok{"ZMNIRegions"}\NormalTok{,}
         \AttributeTok{append=}\ConstantTok{FALSE}\NormalTok{)}
\FunctionTok{rm}\NormalTok{(}\AttributeTok{list=}\FunctionTok{ls}\NormalTok{())}




\CommentTok{\# water drenage ditches {-}{-}{-}{-}{-}}


\NormalTok{link}\OtherTok{=}\StringTok{"https://lvmgeoserver.lvm.lv/geoserver/zmni/ows?"}
\NormalTok{url}\OtherTok{=}\FunctionTok{parse\_url}\NormalTok{(link)}

\NormalTok{url}\SpecialCharTok{$}\NormalTok{query }\OtherTok{\textless{}{-}} \FunctionTok{list}\NormalTok{(}\AttributeTok{service =} \StringTok{"wfs"}\NormalTok{,}\AttributeTok{request =} \StringTok{"GetCapabilities"}\NormalTok{)}
\NormalTok{request }\OtherTok{\textless{}{-}} \FunctionTok{build\_url}\NormalTok{(url)}
\NormalTok{bwk\_client }\OtherTok{\textless{}{-}}\NormalTok{ WFSClient}\SpecialCharTok{$}\FunctionTok{new}\NormalTok{(link,}\AttributeTok{serviceVersion =} \StringTok{"2.0.0"}\NormalTok{)}

\NormalTok{bwk\_client}\SpecialCharTok{$}\FunctionTok{getFeatureTypes}\NormalTok{(}\AttributeTok{pretty =} \ConstantTok{TRUE}\NormalTok{)}

\CommentTok{\# geoms}

\NormalTok{url}\SpecialCharTok{$}\NormalTok{query }\OtherTok{\textless{}{-}} \FunctionTok{list}\NormalTok{(}\AttributeTok{service =} \StringTok{"wfs"}\NormalTok{,}
                  \AttributeTok{request =} \StringTok{"GetFeature"}\NormalTok{,}
                  \AttributeTok{srsName=}\StringTok{"EPSG:3059"}\NormalTok{,}
                  \AttributeTok{typename =} \StringTok{"zmni:zmni\_waterdrainditches"}\NormalTok{,}
                  \AttributeTok{count=}\DecValTok{100}\NormalTok{)}
\NormalTok{request }\OtherTok{\textless{}{-}} \FunctionTok{build\_url}\NormalTok{(url)}

\NormalTok{geometrijam }\OtherTok{\textless{}{-}} \FunctionTok{read\_sf}\NormalTok{(request)}
\NormalTok{geometrijam}



\CommentTok{\# download}
\NormalTok{base\_url }\OtherTok{\textless{}{-}} \StringTok{"https://lvmgeoserver.lvm.lv/geoserver/zmni/ows?"}
\NormalTok{type\_name }\OtherTok{\textless{}{-}} \StringTok{"zmni:zmni\_waterdrainditches"}
\NormalTok{crs\_code }\OtherTok{\textless{}{-}} \DecValTok{3059}
\NormalTok{chunk\_size }\OtherTok{\textless{}{-}} \DecValTok{100000}
\NormalTok{gpkg\_path }\OtherTok{\textless{}{-}} \StringTok{"./Geodata/2024/MKIS/temp\_MKIS\_2025.gpkg"}
\NormalTok{layer\_name }\OtherTok{\textless{}{-}} \StringTok{"temp\_UdensnotekasNovadgravji"}
\NormalTok{i }\OtherTok{\textless{}{-}} \DecValTok{0}

\ControlFlowTok{repeat}\NormalTok{ \{}
  \FunctionTok{message}\NormalTok{(}\StringTok{"Fetching features "}\NormalTok{, i }\SpecialCharTok{*}\NormalTok{ chunk\_size }\SpecialCharTok{+} \DecValTok{1}\NormalTok{, }\StringTok{" to "}\NormalTok{, (i }\SpecialCharTok{+} \DecValTok{1}\NormalTok{) }\SpecialCharTok{*}\NormalTok{ chunk\_size, }\StringTok{"..."}\NormalTok{)}
  
\NormalTok{  query }\OtherTok{\textless{}{-}} \FunctionTok{list}\NormalTok{(}
    \AttributeTok{service =} \StringTok{"WFS"}\NormalTok{,}
    \AttributeTok{version =} \StringTok{"2.0.0"}\NormalTok{,}
    \AttributeTok{request =} \StringTok{"GetFeature"}\NormalTok{,}
    \AttributeTok{typename =}\NormalTok{ type\_name,}
    \AttributeTok{srsName =} \FunctionTok{paste0}\NormalTok{(}\StringTok{"EPSG:"}\NormalTok{, crs\_code),}
    \AttributeTok{count =}\NormalTok{ chunk\_size,}
    \AttributeTok{startIndex =}\NormalTok{ i }\SpecialCharTok{*}\NormalTok{ chunk\_size}
\NormalTok{  )}
  
\NormalTok{  req\_url }\OtherTok{\textless{}{-}} \FunctionTok{modify\_url}\NormalTok{(base\_url, }\AttributeTok{query =}\NormalTok{ query)}
  
  \FunctionTok{try}\NormalTok{(\{}
\NormalTok{    chunk }\OtherTok{\textless{}{-}} \FunctionTok{read\_sf}\NormalTok{(req\_url)}
    \ControlFlowTok{if}\NormalTok{ (}\FunctionTok{nrow}\NormalTok{(chunk) }\SpecialCharTok{==} \DecValTok{0}\NormalTok{) }\ControlFlowTok{break}
    
    \CommentTok{\# Set CRS and cast to MULTILINESTRING, POINT, MULTIPOLYGON}
\NormalTok{    chunk }\OtherTok{\textless{}{-}}\NormalTok{ chunk }\SpecialCharTok{\%\textgreater{}\%}
      \FunctionTok{st\_set\_crs}\NormalTok{(}\FunctionTok{st\_crs}\NormalTok{(crs\_code))}
    
\NormalTok{    chunk}\OtherTok{=}\FunctionTok{st\_cast}\NormalTok{(chunk,}\StringTok{"MULTILINESTRING"}\NormalTok{)}
    
    \CommentTok{\# Write chunk to GeoPackage (append mode after first)}
    \FunctionTok{st\_write}\NormalTok{(}
\NormalTok{      chunk, }
      \AttributeTok{dsn =}\NormalTok{ gpkg\_path,}
      \AttributeTok{layer =}\NormalTok{ layer\_name,}
      \AttributeTok{append =}\NormalTok{ i }\SpecialCharTok{!=} \DecValTok{0}\NormalTok{,}
      \AttributeTok{quiet =} \ConstantTok{FALSE}
\NormalTok{    )}
    
\NormalTok{    i }\OtherTok{\textless{}{-}}\NormalTok{ i }\SpecialCharTok{+} \DecValTok{1}
\NormalTok{  \}, }\AttributeTok{silent =} \ConstantTok{TRUE}\NormalTok{)}
  \FunctionTok{Sys.sleep}\NormalTok{(}\FloatTok{0.5}\NormalTok{)}
\NormalTok{\}}

\FunctionTok{message}\NormalTok{(}\StringTok{"All chunks written to "}\NormalTok{, gpkg\_path, }\StringTok{" in layer "}\NormalTok{, layer\_name)}


\NormalTok{UdensnotekasNovadgravji\_all}\OtherTok{=}\FunctionTok{st\_read}\NormalTok{(}\StringTok{"./Geodata/2024/MKIS/temp\_MKIS\_2025.gpkg"}\NormalTok{,}
                                    \AttributeTok{layer=}\StringTok{"temp\_UdensnotekasNovadgravji"}\NormalTok{)}
\NormalTok{UdensnotekasNovadgravji\_all2 }\OtherTok{=}\NormalTok{ UdensnotekasNovadgravji\_all[}\SpecialCharTok{!}\FunctionTok{st\_is\_empty}\NormalTok{(UdensnotekasNovadgravji\_all),,drop}\OtherTok{=}\ConstantTok{FALSE}\NormalTok{] }\CommentTok{\# 0}
\FunctionTok{table}\NormalTok{(}\FunctionTok{st\_is\_valid}\NormalTok{(UdensnotekasNovadgravji\_all2))}


\FunctionTok{write\_sf}\NormalTok{(UdensnotekasNovadgravji\_all2,}
         \StringTok{"./Geodata/2024/MKIS/MKIS\_2025.gpkg"}\NormalTok{,}
         \AttributeTok{layer=}\StringTok{"UdensnotekasNovadgravji"}\NormalTok{,}
         \AttributeTok{append=}\ConstantTok{FALSE}\NormalTok{)}
\FunctionTok{rm}\NormalTok{(}\AttributeTok{list=}\FunctionTok{ls}\NormalTok{())}




\CommentTok{\# ditch pickets {-}{-}{-}{-}}



\NormalTok{link}\OtherTok{=}\StringTok{"https://lvmgeoserver.lvm.lv/geoserver/zmni/ows?"}
\NormalTok{url}\OtherTok{=}\FunctionTok{parse\_url}\NormalTok{(link)}

\NormalTok{url}\SpecialCharTok{$}\NormalTok{query }\OtherTok{\textless{}{-}} \FunctionTok{list}\NormalTok{(}\AttributeTok{service =} \StringTok{"wfs"}\NormalTok{,}\AttributeTok{request =} \StringTok{"GetCapabilities"}\NormalTok{)}
\NormalTok{request }\OtherTok{\textless{}{-}} \FunctionTok{build\_url}\NormalTok{(url)}
\NormalTok{bwk\_client }\OtherTok{\textless{}{-}}\NormalTok{ WFSClient}\SpecialCharTok{$}\FunctionTok{new}\NormalTok{(link,}\AttributeTok{serviceVersion =} \StringTok{"2.0.0"}\NormalTok{)}

\NormalTok{bwk\_client}\SpecialCharTok{$}\FunctionTok{getFeatureTypes}\NormalTok{(}\AttributeTok{pretty =} \ConstantTok{TRUE}\NormalTok{)}

\CommentTok{\# geoms}

\NormalTok{url}\SpecialCharTok{$}\NormalTok{query }\OtherTok{\textless{}{-}} \FunctionTok{list}\NormalTok{(}\AttributeTok{service =} \StringTok{"wfs"}\NormalTok{,}
                  \AttributeTok{request =} \StringTok{"GetFeature"}\NormalTok{,}
                  \AttributeTok{srsName=}\StringTok{"EPSG:3059"}\NormalTok{,}
                  \AttributeTok{typename =} \StringTok{"zmni:zmni\_ditchpicket"}\NormalTok{,}
                  \AttributeTok{count=}\DecValTok{100}\NormalTok{)}
\NormalTok{request }\OtherTok{\textless{}{-}} \FunctionTok{build\_url}\NormalTok{(url)}

\NormalTok{geometrijam }\OtherTok{\textless{}{-}} \FunctionTok{read\_sf}\NormalTok{(request)}
\NormalTok{geometrijam}



\CommentTok{\# download}
\NormalTok{base\_url }\OtherTok{\textless{}{-}} \StringTok{"https://lvmgeoserver.lvm.lv/geoserver/zmni/ows?"}
\NormalTok{type\_name }\OtherTok{\textless{}{-}} \StringTok{"zmni:zmni\_ditchpicket"}
\NormalTok{crs\_code }\OtherTok{\textless{}{-}} \DecValTok{3059}
\NormalTok{chunk\_size }\OtherTok{\textless{}{-}} \DecValTok{100000}
\NormalTok{gpkg\_path }\OtherTok{\textless{}{-}} \StringTok{"./Geodata/2024/MKIS/temp\_MKIS\_2025.gpkg"}
\NormalTok{layer\_name }\OtherTok{\textless{}{-}} \StringTok{"temp\_UdensnotekuNovadgravjuPiketi"}
\NormalTok{i }\OtherTok{\textless{}{-}} \DecValTok{0}

\ControlFlowTok{repeat}\NormalTok{ \{}
  \FunctionTok{message}\NormalTok{(}\StringTok{"Fetching features "}\NormalTok{, i }\SpecialCharTok{*}\NormalTok{ chunk\_size }\SpecialCharTok{+} \DecValTok{1}\NormalTok{, }\StringTok{" to "}\NormalTok{, (i }\SpecialCharTok{+} \DecValTok{1}\NormalTok{) }\SpecialCharTok{*}\NormalTok{ chunk\_size, }\StringTok{"..."}\NormalTok{)}
  
\NormalTok{  query }\OtherTok{\textless{}{-}} \FunctionTok{list}\NormalTok{(}
    \AttributeTok{service =} \StringTok{"WFS"}\NormalTok{,}
    \AttributeTok{version =} \StringTok{"2.0.0"}\NormalTok{,}
    \AttributeTok{request =} \StringTok{"GetFeature"}\NormalTok{,}
    \AttributeTok{typename =}\NormalTok{ type\_name,}
    \AttributeTok{srsName =} \FunctionTok{paste0}\NormalTok{(}\StringTok{"EPSG:"}\NormalTok{, crs\_code),}
    \AttributeTok{count =}\NormalTok{ chunk\_size,}
    \AttributeTok{startIndex =}\NormalTok{ i }\SpecialCharTok{*}\NormalTok{ chunk\_size}
\NormalTok{  )}
  
\NormalTok{  req\_url }\OtherTok{\textless{}{-}} \FunctionTok{modify\_url}\NormalTok{(base\_url, }\AttributeTok{query =}\NormalTok{ query)}
  
  \FunctionTok{try}\NormalTok{(\{}
\NormalTok{    chunk }\OtherTok{\textless{}{-}} \FunctionTok{read\_sf}\NormalTok{(req\_url)}
    \ControlFlowTok{if}\NormalTok{ (}\FunctionTok{nrow}\NormalTok{(chunk) }\SpecialCharTok{==} \DecValTok{0}\NormalTok{) }\ControlFlowTok{break}
    
    \CommentTok{\# Set CRS and cast to MULTILINESTRING, POINT, MULTIPOLYGON}
\NormalTok{    chunk }\OtherTok{\textless{}{-}}\NormalTok{ chunk }\SpecialCharTok{\%\textgreater{}\%}
      \FunctionTok{st\_set\_crs}\NormalTok{(}\FunctionTok{st\_crs}\NormalTok{(crs\_code)) }\SpecialCharTok{\%\textgreater{}\%}
      \FunctionTok{st\_cast}\NormalTok{(}\StringTok{"POINT"}\NormalTok{)}
    
    \CommentTok{\# Write chunk to GeoPackage (append mode after first)}
    \FunctionTok{st\_write}\NormalTok{(}
\NormalTok{      chunk, }
      \AttributeTok{dsn =}\NormalTok{ gpkg\_path,}
      \AttributeTok{layer =}\NormalTok{ layer\_name,}
      \AttributeTok{append =}\NormalTok{ i }\SpecialCharTok{!=} \DecValTok{0}\NormalTok{,}
      \AttributeTok{quiet =} \ConstantTok{FALSE}
\NormalTok{    )}
    
\NormalTok{    i }\OtherTok{\textless{}{-}}\NormalTok{ i }\SpecialCharTok{+} \DecValTok{1}
\NormalTok{  \}, }\AttributeTok{silent =} \ConstantTok{TRUE}\NormalTok{)}
\NormalTok{\}}

\FunctionTok{message}\NormalTok{(}\StringTok{"All chunks written to "}\NormalTok{, gpkg\_path, }\StringTok{" in layer "}\NormalTok{, layer\_name)}

\NormalTok{UdensnotekuNovadgravjuPiketi\_all}\OtherTok{=}\FunctionTok{st\_read}\NormalTok{(}\StringTok{"./Geodata/2024/MKIS/temp\_MKIS\_2025.gpkg"}\NormalTok{,}
                                         \AttributeTok{layer=}\StringTok{"temp\_UdensnotekuNovadgravjuPiketi"}\NormalTok{)}
\NormalTok{UdensnotekuNovadgravjuPiketi\_all2 }\OtherTok{=}\NormalTok{ UdensnotekuNovadgravjuPiketi\_all[}\SpecialCharTok{!}\FunctionTok{st\_is\_empty}\NormalTok{(UdensnotekuNovadgravjuPiketi\_all),,drop}\OtherTok{=}\ConstantTok{FALSE}\NormalTok{] }\CommentTok{\# 0}
\FunctionTok{table}\NormalTok{(}\FunctionTok{st\_is\_valid}\NormalTok{(UdensnotekuNovadgravjuPiketi\_all2))}


\FunctionTok{write\_sf}\NormalTok{(UdensnotekuNovadgravjuPiketi\_all2,}
         \StringTok{"./Geodata/2024/MKIS/MKIS\_2025.gpkg"}\NormalTok{,}
         \AttributeTok{layer=}\StringTok{"UdensnotekuNovadgravjuPiketi"}\NormalTok{,}
         \AttributeTok{append=}\ConstantTok{FALSE}\NormalTok{)}
\FunctionTok{rm}\NormalTok{(}\AttributeTok{list=}\FunctionTok{ls}\NormalTok{())}



\CommentTok{\# state river axis {-}{-}{-}{-}}


\NormalTok{link}\OtherTok{=}\StringTok{"https://lvmgeoserver.lvm.lv/geoserver/zmni/ows?"}
\NormalTok{url}\OtherTok{=}\FunctionTok{parse\_url}\NormalTok{(link)}

\NormalTok{url}\SpecialCharTok{$}\NormalTok{query }\OtherTok{\textless{}{-}} \FunctionTok{list}\NormalTok{(}\AttributeTok{service =} \StringTok{"wfs"}\NormalTok{,}\AttributeTok{request =} \StringTok{"GetCapabilities"}\NormalTok{)}
\NormalTok{request }\OtherTok{\textless{}{-}} \FunctionTok{build\_url}\NormalTok{(url)}
\NormalTok{bwk\_client }\OtherTok{\textless{}{-}}\NormalTok{ WFSClient}\SpecialCharTok{$}\FunctionTok{new}\NormalTok{(link,}\AttributeTok{serviceVersion =} \StringTok{"2.0.0"}\NormalTok{)}

\NormalTok{bwk\_client}\SpecialCharTok{$}\FunctionTok{getFeatureTypes}\NormalTok{(}\AttributeTok{pretty =} \ConstantTok{TRUE}\NormalTok{)}

\CommentTok{\# geoms}

\NormalTok{url}\SpecialCharTok{$}\NormalTok{query }\OtherTok{\textless{}{-}} \FunctionTok{list}\NormalTok{(}\AttributeTok{service =} \StringTok{"wfs"}\NormalTok{,}
                  \AttributeTok{request =} \StringTok{"GetFeature"}\NormalTok{,}
                  \AttributeTok{srsName=}\StringTok{"EPSG:3059"}\NormalTok{,}
                  \AttributeTok{typename =} \StringTok{"zmni:zmni\_stateriversline"}\NormalTok{,}
                  \AttributeTok{count=}\DecValTok{100}\NormalTok{)}
\NormalTok{request }\OtherTok{\textless{}{-}} \FunctionTok{build\_url}\NormalTok{(url)}

\NormalTok{geometrijam }\OtherTok{\textless{}{-}} \FunctionTok{read\_sf}\NormalTok{(request)}
\NormalTok{geometrijam}



\CommentTok{\# download}
\NormalTok{base\_url }\OtherTok{\textless{}{-}} \StringTok{"https://lvmgeoserver.lvm.lv/geoserver/zmni/ows?"}
\NormalTok{type\_name }\OtherTok{\textless{}{-}} \StringTok{"zmni:zmni\_stateriversline"}
\NormalTok{crs\_code }\OtherTok{\textless{}{-}} \DecValTok{3059}
\NormalTok{chunk\_size }\OtherTok{\textless{}{-}} \DecValTok{100000}
\NormalTok{gpkg\_path }\OtherTok{\textless{}{-}} \StringTok{"./Geodata/2024/MKIS/temp\_MKIS\_2025.gpkg"}
\NormalTok{layer\_name }\OtherTok{\textless{}{-}} \StringTok{"temp\_UdenstecuAsis"}
\NormalTok{i }\OtherTok{\textless{}{-}} \DecValTok{0}

\ControlFlowTok{repeat}\NormalTok{ \{}
  \FunctionTok{message}\NormalTok{(}\StringTok{"Fetching features "}\NormalTok{, i }\SpecialCharTok{*}\NormalTok{ chunk\_size }\SpecialCharTok{+} \DecValTok{1}\NormalTok{, }\StringTok{" to "}\NormalTok{, (i }\SpecialCharTok{+} \DecValTok{1}\NormalTok{) }\SpecialCharTok{*}\NormalTok{ chunk\_size, }\StringTok{"..."}\NormalTok{)}
  
\NormalTok{  query }\OtherTok{\textless{}{-}} \FunctionTok{list}\NormalTok{(}
    \AttributeTok{service =} \StringTok{"WFS"}\NormalTok{,}
    \AttributeTok{version =} \StringTok{"2.0.0"}\NormalTok{,}
    \AttributeTok{request =} \StringTok{"GetFeature"}\NormalTok{,}
    \AttributeTok{typename =}\NormalTok{ type\_name,}
    \AttributeTok{srsName =} \FunctionTok{paste0}\NormalTok{(}\StringTok{"EPSG:"}\NormalTok{, crs\_code),}
    \AttributeTok{count =}\NormalTok{ chunk\_size,}
    \AttributeTok{startIndex =}\NormalTok{ i }\SpecialCharTok{*}\NormalTok{ chunk\_size}
\NormalTok{  )}
  
\NormalTok{  req\_url }\OtherTok{\textless{}{-}} \FunctionTok{modify\_url}\NormalTok{(base\_url, }\AttributeTok{query =}\NormalTok{ query)}
  
  \FunctionTok{try}\NormalTok{(\{}
\NormalTok{    chunk }\OtherTok{\textless{}{-}} \FunctionTok{read\_sf}\NormalTok{(req\_url)}
    \ControlFlowTok{if}\NormalTok{ (}\FunctionTok{nrow}\NormalTok{(chunk) }\SpecialCharTok{==} \DecValTok{0}\NormalTok{) }\ControlFlowTok{break}
    
    \CommentTok{\# Set CRS and cast to MULTILINESTRING, POINT, MULTIPOLYGON}
\NormalTok{    chunk }\OtherTok{\textless{}{-}}\NormalTok{ chunk }\SpecialCharTok{\%\textgreater{}\%}
      \FunctionTok{st\_set\_crs}\NormalTok{(}\FunctionTok{st\_crs}\NormalTok{(crs\_code))}
    
\NormalTok{    chunk}\OtherTok{=}\FunctionTok{st\_cast}\NormalTok{(chunk,}\StringTok{"MULTILINESTRING"}\NormalTok{)}
    
    \CommentTok{\# Write chunk to GeoPackage (append mode after first)}
    \FunctionTok{st\_write}\NormalTok{(}
\NormalTok{      chunk, }
      \AttributeTok{dsn =}\NormalTok{ gpkg\_path,}
      \AttributeTok{layer =}\NormalTok{ layer\_name,}
      \AttributeTok{append =}\NormalTok{ i }\SpecialCharTok{!=} \DecValTok{0}\NormalTok{,}
      \AttributeTok{quiet =} \ConstantTok{FALSE}
\NormalTok{    )}
    
\NormalTok{    i }\OtherTok{\textless{}{-}}\NormalTok{ i }\SpecialCharTok{+} \DecValTok{1}
\NormalTok{  \}, }\AttributeTok{silent =} \ConstantTok{TRUE}\NormalTok{)}
  \FunctionTok{Sys.sleep}\NormalTok{(}\FloatTok{0.5}\NormalTok{)}
\NormalTok{\}}

\FunctionTok{message}\NormalTok{(}\StringTok{"All chunks written to "}\NormalTok{, gpkg\_path, }\StringTok{" in layer "}\NormalTok{, layer\_name)}


\NormalTok{UdenstecuAsis\_all}\OtherTok{=}\FunctionTok{st\_read}\NormalTok{(}\StringTok{"./Geodata/2024/MKIS/temp\_MKIS\_2025.gpkg"}\NormalTok{,}
                          \AttributeTok{layer=}\StringTok{"temp\_UdenstecuAsis"}\NormalTok{)}
\NormalTok{UdenstecuAsis\_all2 }\OtherTok{=}\NormalTok{ UdenstecuAsis\_all[}\SpecialCharTok{!}\FunctionTok{st\_is\_empty}\NormalTok{(UdenstecuAsis\_all),,drop}\OtherTok{=}\ConstantTok{FALSE}\NormalTok{] }\CommentTok{\# 0}
\FunctionTok{table}\NormalTok{(}\FunctionTok{st\_is\_valid}\NormalTok{(UdenstecuAsis\_all2))}


\FunctionTok{write\_sf}\NormalTok{(UdenstecuAsis\_all2,}
         \StringTok{"./Geodata/2024/MKIS/MKIS\_2025.gpkg"}\NormalTok{,}
         \AttributeTok{layer=}\StringTok{"UdenstecuAsis"}\NormalTok{,}
         \AttributeTok{append=}\ConstantTok{FALSE}\NormalTok{)}
\FunctionTok{rm}\NormalTok{(}\AttributeTok{list=}\FunctionTok{ls}\NormalTok{())}



\CommentTok{\# river surface {-}{-}{-}{-}}


\NormalTok{link}\OtherTok{=}\StringTok{"https://lvmgeoserver.lvm.lv/geoserver/zmni/ows?"}
\NormalTok{url}\OtherTok{=}\FunctionTok{parse\_url}\NormalTok{(link)}

\NormalTok{url}\SpecialCharTok{$}\NormalTok{query }\OtherTok{\textless{}{-}} \FunctionTok{list}\NormalTok{(}\AttributeTok{service =} \StringTok{"wfs"}\NormalTok{,}\AttributeTok{request =} \StringTok{"GetCapabilities"}\NormalTok{)}
\NormalTok{request }\OtherTok{\textless{}{-}} \FunctionTok{build\_url}\NormalTok{(url)}
\NormalTok{bwk\_client }\OtherTok{\textless{}{-}}\NormalTok{ WFSClient}\SpecialCharTok{$}\FunctionTok{new}\NormalTok{(link,}\AttributeTok{serviceVersion =} \StringTok{"2.0.0"}\NormalTok{)}

\NormalTok{bwk\_client}\SpecialCharTok{$}\FunctionTok{getFeatureTypes}\NormalTok{(}\AttributeTok{pretty =} \ConstantTok{TRUE}\NormalTok{)}

\CommentTok{\# geoms}

\NormalTok{url}\SpecialCharTok{$}\NormalTok{query }\OtherTok{\textless{}{-}} \FunctionTok{list}\NormalTok{(}\AttributeTok{service =} \StringTok{"wfs"}\NormalTok{,}
                  \AttributeTok{request =} \StringTok{"GetFeature"}\NormalTok{,}
                  \AttributeTok{srsName=}\StringTok{"EPSG:3059"}\NormalTok{,}
                  \AttributeTok{typename =} \StringTok{"zmni:zmni\_stateriverspolygon"}\NormalTok{,}
                  \AttributeTok{count=}\DecValTok{100}\NormalTok{)}
\NormalTok{request }\OtherTok{\textless{}{-}} \FunctionTok{build\_url}\NormalTok{(url)}

\NormalTok{geometrijam }\OtherTok{\textless{}{-}} \FunctionTok{read\_sf}\NormalTok{(request)}
\NormalTok{geometrijam}


\CommentTok{\# download}
\NormalTok{base\_url }\OtherTok{\textless{}{-}} \StringTok{"https://lvmgeoserver.lvm.lv/geoserver/zmni/ows?"}
\NormalTok{type\_name }\OtherTok{\textless{}{-}} \StringTok{"zmni:zmni\_stateriverspolygon"}
\NormalTok{crs\_code }\OtherTok{\textless{}{-}} \DecValTok{3059}
\NormalTok{chunk\_size }\OtherTok{\textless{}{-}} \DecValTok{100000}
\NormalTok{gpkg\_path }\OtherTok{\textless{}{-}} \StringTok{"./Geodata/2024/MKIS/temp\_MKIS\_2025.gpkg"}
\NormalTok{layer\_name }\OtherTok{\textless{}{-}} \StringTok{"temp\_UdenstecuVirsmasLaukumi"}
\NormalTok{i }\OtherTok{\textless{}{-}} \DecValTok{0}

\ControlFlowTok{repeat}\NormalTok{ \{}
  \FunctionTok{message}\NormalTok{(}\StringTok{"Fetching features "}\NormalTok{, i }\SpecialCharTok{*}\NormalTok{ chunk\_size }\SpecialCharTok{+} \DecValTok{1}\NormalTok{, }\StringTok{" to "}\NormalTok{, (i }\SpecialCharTok{+} \DecValTok{1}\NormalTok{) }\SpecialCharTok{*}\NormalTok{ chunk\_size, }\StringTok{"..."}\NormalTok{)}
  
\NormalTok{  query }\OtherTok{\textless{}{-}} \FunctionTok{list}\NormalTok{(}
    \AttributeTok{service =} \StringTok{"WFS"}\NormalTok{,}
    \AttributeTok{version =} \StringTok{"2.0.0"}\NormalTok{,}
    \AttributeTok{request =} \StringTok{"GetFeature"}\NormalTok{,}
    \AttributeTok{typename =}\NormalTok{ type\_name,}
    \AttributeTok{srsName =} \FunctionTok{paste0}\NormalTok{(}\StringTok{"EPSG:"}\NormalTok{, crs\_code),}
    \AttributeTok{count =}\NormalTok{ chunk\_size,}
    \AttributeTok{startIndex =}\NormalTok{ i }\SpecialCharTok{*}\NormalTok{ chunk\_size}
\NormalTok{  )}
  
\NormalTok{  req\_url }\OtherTok{\textless{}{-}} \FunctionTok{modify\_url}\NormalTok{(base\_url, }\AttributeTok{query =}\NormalTok{ query)}
  
  \FunctionTok{try}\NormalTok{(\{}
\NormalTok{    chunk }\OtherTok{\textless{}{-}} \FunctionTok{read\_sf}\NormalTok{(req\_url)}
    \ControlFlowTok{if}\NormalTok{ (}\FunctionTok{nrow}\NormalTok{(chunk) }\SpecialCharTok{==} \DecValTok{0}\NormalTok{) }\ControlFlowTok{break}
    
    \CommentTok{\# Set CRS and cast to MULTILINESTRING, POINT, MULTIPOLYGON}
\NormalTok{    chunk }\OtherTok{\textless{}{-}}\NormalTok{ chunk }\SpecialCharTok{\%\textgreater{}\%}
      \FunctionTok{st\_set\_crs}\NormalTok{(}\FunctionTok{st\_crs}\NormalTok{(crs\_code))}
    
\NormalTok{    ensure\_multipolygons }\OtherTok{\textless{}{-}} \ControlFlowTok{function}\NormalTok{(X) \{}
\NormalTok{      tmp1 }\OtherTok{\textless{}{-}} \FunctionTok{tempfile}\NormalTok{(}\AttributeTok{fileext =} \StringTok{".gpkg"}\NormalTok{)}
\NormalTok{      tmp2 }\OtherTok{\textless{}{-}} \FunctionTok{tempfile}\NormalTok{(}\AttributeTok{fileext =} \StringTok{".gpkg"}\NormalTok{)}
      \FunctionTok{st\_write}\NormalTok{(X, tmp1)}
      \FunctionTok{ogr2ogr}\NormalTok{(tmp1, tmp2, }\AttributeTok{f =} \StringTok{"GPKG"}\NormalTok{, }\AttributeTok{nlt =} \StringTok{"MULTIPOLYGON"}\NormalTok{)}
\NormalTok{      Y }\OtherTok{\textless{}{-}} \FunctionTok{st\_read}\NormalTok{(tmp2)}
      \FunctionTok{st\_sf}\NormalTok{(}\FunctionTok{st\_drop\_geometry}\NormalTok{(X), }\AttributeTok{geom =} \FunctionTok{st\_geometry}\NormalTok{(Y))}
\NormalTok{    \}}
\NormalTok{    chunk }\OtherTok{\textless{}{-}} \FunctionTok{ensure\_multipolygons}\NormalTok{(chunk)}
    
    
    \CommentTok{\# Write chunk to GeoPackage (append mode after first)}
    \FunctionTok{st\_write}\NormalTok{(}
\NormalTok{      chunk, }
      \AttributeTok{dsn =}\NormalTok{ gpkg\_path,}
      \AttributeTok{layer =}\NormalTok{ layer\_name,}
      \AttributeTok{append =}\NormalTok{ i }\SpecialCharTok{!=} \DecValTok{0}\NormalTok{,}
      \AttributeTok{quiet =} \ConstantTok{FALSE}
\NormalTok{    )}
    
\NormalTok{    i }\OtherTok{\textless{}{-}}\NormalTok{ i }\SpecialCharTok{+} \DecValTok{1}
\NormalTok{  \}, }\AttributeTok{silent =} \ConstantTok{TRUE}\NormalTok{)}
  \FunctionTok{Sys.sleep}\NormalTok{(}\FloatTok{0.5}\NormalTok{)}
\NormalTok{\}}

\FunctionTok{message}\NormalTok{(}\StringTok{"All chunks written to "}\NormalTok{, gpkg\_path, }\StringTok{" in layer "}\NormalTok{, layer\_name)}


\NormalTok{UdenstecuVirsmasLaukumi\_all}\OtherTok{=}\FunctionTok{st\_read}\NormalTok{(}\StringTok{"./Geodata/2024/MKIS/temp\_MKIS\_2025.gpkg"}\NormalTok{,}
                                    \AttributeTok{layer=}\StringTok{"temp\_UdenstecuVirsmasLaukumi"}\NormalTok{)}
\NormalTok{UdenstecuVirsmasLaukumi\_all2 }\OtherTok{=}\NormalTok{ UdenstecuVirsmasLaukumi\_all[}\SpecialCharTok{!}\FunctionTok{st\_is\_empty}\NormalTok{(UdenstecuVirsmasLaukumi\_all),,drop}\OtherTok{=}\ConstantTok{FALSE}\NormalTok{] }\CommentTok{\# 0}
\FunctionTok{table}\NormalTok{(}\FunctionTok{st\_is\_valid}\NormalTok{(UdenstecuVirsmasLaukumi\_all2))}

\NormalTok{UdenstecuVirsmasLaukumi\_all3}\OtherTok{=}\FunctionTok{st\_make\_valid}\NormalTok{(UdenstecuVirsmasLaukumi\_all2)}
\FunctionTok{table}\NormalTok{(}\FunctionTok{st\_is\_valid}\NormalTok{(UdenstecuVirsmasLaukumi\_all3))}

\FunctionTok{write\_sf}\NormalTok{(UdenstecuVirsmasLaukumi\_all3,}
         \StringTok{"./Geodata/2024/MKIS/MKIS\_2025.gpkg"}\NormalTok{,}
         \AttributeTok{layer=}\StringTok{"UdenstecuVirsmasLaukumi"}\NormalTok{,}
         \AttributeTok{append=}\ConstantTok{FALSE}\NormalTok{)}
\FunctionTok{rm}\NormalTok{(}\AttributeTok{list=}\FunctionTok{ls}\NormalTok{())}
\end{Highlighting}
\end{Shaded}

\section{Topographic Map}\label{Ch04.04}

To support research process at the University of Latvia, the third (completed
by January 1, 2018) and fourth (unfinished) versions of the Latvian Geospatial
Information Agency's topographic map M:10000 vector geodatabase were received.
The most recent version is available for \href{https://kartes.lgia.gov.lv/karte/?x=311986.74&y=506887.35&zoom=3&basemap=topokarte}{public viewing},
but access to the vector data is restricted.

For the purposes of this project, the ESRI geodatabase has been converted to a
GeoPackage file. As part of the file format change, geometries (empty, their
validity checked and corrected where necessary) and coordinate system have
been checked.

Files were stored at \texttt{Geodata/2024/TopographicMap/}.

After processing each database separately, we combined the layers used in this
project, selecting the most recent layer per map sheet. These layers are:

\begin{itemize}
\item
  \texttt{brigde\_L}, describing bridges as lines;
\item
  \texttt{bridge\_P}, describing bridges as points;
\item
  \texttt{hidro\_A}, describing waterbodies as polygons;
\item
  \texttt{hidro\_L}, describing ditches and small rivers as lines;
\item
  \texttt{landus\_A}, describing LULC as polygons;
\item
  \texttt{road\_A}, describing larger roads as polygons;
\item
  \texttt{road\_L}, including very small or disused ones, as lines;
\item
  \texttt{swamp\_A}, describing bogs as polygons;
\item
  \texttt{flora\_L}, describing linear tree and shrub formations;
\item
  \texttt{build\_A}, describing types of built-up areas. Version 4 available at the
  University of Latvia does not include all the classes present in Version 3,
  therefore version 3 is used.
\end{itemize}

\begin{Shaded}
\begin{Highlighting}[]
\CommentTok{\# libs {-}{-}{-}{-}}
\ControlFlowTok{if}\NormalTok{(}\SpecialCharTok{!}\FunctionTok{require}\NormalTok{(sf)) \{}\FunctionTok{install.packages}\NormalTok{(}\StringTok{"sf"}\NormalTok{); }\FunctionTok{require}\NormalTok{(sf)\}}
\ControlFlowTok{if}\NormalTok{(}\SpecialCharTok{!}\FunctionTok{require}\NormalTok{(openxlsx)) \{}\FunctionTok{install.packages}\NormalTok{(}\StringTok{"openxlsx"}\NormalTok{); }\FunctionTok{require}\NormalTok{(openxlsx)\}}
\ControlFlowTok{if}\NormalTok{(}\SpecialCharTok{!}\FunctionTok{require}\NormalTok{(tidyverse)) \{}\FunctionTok{install.packages}\NormalTok{(}\StringTok{"tidyverse"}\NormalTok{); }\FunctionTok{require}\NormalTok{(tidyverse)\}}

\CommentTok{\# v4 {-}{-}{-}{-}}
\NormalTok{slani\_v4}\OtherTok{=}\FunctionTok{st\_layers}\NormalTok{(}\StringTok{"./Geodata/2024/TopographicMap/Latvija\_LKS92\_v4\_20250703.gdb/"}\NormalTok{)}
\FunctionTok{write.xlsx}\NormalTok{(slani\_v4,}\StringTok{"./Geodata/2024/TopographicMap/slani\_v4partial.xlsx"}\NormalTok{)}

\NormalTok{slani\_v4}\SpecialCharTok{$}\NormalTok{geometrijai}\OtherTok{=}\FunctionTok{as.character}\NormalTok{(slani\_v4}\SpecialCharTok{$}\NormalTok{geomtype)}
\FunctionTok{table}\NormalTok{(slani\_v4}\SpecialCharTok{$}\NormalTok{geometrijai)}

\NormalTok{slani\_v4}\SpecialCharTok{$}\NormalTok{geometrijai2}\OtherTok{=}\FunctionTok{ifelse}\NormalTok{(slani\_v4}\SpecialCharTok{$}\NormalTok{geometrijai}\SpecialCharTok{==}\StringTok{"3D Point"}\NormalTok{,}\StringTok{"POINT"}\NormalTok{,}
                                   \FunctionTok{ifelse}\NormalTok{(slani\_v4}\SpecialCharTok{$}\NormalTok{geometrijai}\SpecialCharTok{==}\StringTok{"Multi Polygon"}\NormalTok{,}
                                          \StringTok{"MULTIPOLYGON"}\NormalTok{,}
                                          \FunctionTok{ifelse}\NormalTok{(slani\_v4}\SpecialCharTok{$}\NormalTok{geometrijai}\SpecialCharTok{==}\StringTok{"3D Multi Line String"}\NormalTok{,}
                                                 \StringTok{"MULTILINESTRING"}\NormalTok{,}
                                                 \FunctionTok{ifelse}\NormalTok{(slani\_v4}\SpecialCharTok{$}\NormalTok{geometrijai}\SpecialCharTok{==}\StringTok{"3D Multi Polygon"}\NormalTok{,}
                                                        \StringTok{"MULTIPOLYGON"}\NormalTok{,}\ConstantTok{NA}\NormalTok{))))}

\NormalTok{slani4x}\OtherTok{=}\FunctionTok{data.frame}\NormalTok{(}\AttributeTok{name=}\NormalTok{slani\_v4}\SpecialCharTok{$}\NormalTok{name,}
                   \AttributeTok{geometrija=}\NormalTok{slani\_v4}\SpecialCharTok{$}\NormalTok{geometrijai2)}

\NormalTok{ciklam4x}\OtherTok{=}\FunctionTok{levels}\NormalTok{(}\FunctionTok{factor}\NormalTok{(slani4x}\SpecialCharTok{$}\NormalTok{name))}
\ControlFlowTok{for}\NormalTok{(i }\ControlFlowTok{in} \FunctionTok{seq\_along}\NormalTok{(ciklam4x))\{}
  \FunctionTok{print}\NormalTok{(i)}
\NormalTok{  sakums}\OtherTok{=}\FunctionTok{Sys.time}\NormalTok{()}
\NormalTok{  nosaukums}\OtherTok{=}\NormalTok{ciklam4x[i]}
\NormalTok{  objekts}\OtherTok{=}\NormalTok{slani4x }\SpecialCharTok{\%\textgreater{}\%} 
    \FunctionTok{filter}\NormalTok{(name}\SpecialCharTok{==}\NormalTok{nosaukums)}
  \FunctionTok{print}\NormalTok{(nosaukums)}
\NormalTok{  slanis}\OtherTok{=}\FunctionTok{read\_sf}\NormalTok{(}\StringTok{"./Geodata/2024/TopographicMap/topo10v4/Latvija\_LKS92\_v4\_20250703.gdb/"}\NormalTok{,}
                 \AttributeTok{layer=}\NormalTok{nosaukums)}
\NormalTok{  slanisZM}\OtherTok{=}\FunctionTok{st\_zm}\NormalTok{(slanis)}
\NormalTok{  slanis2}\OtherTok{=}\FunctionTok{st\_cast}\NormalTok{(slanisZM,}\AttributeTok{to=}\NormalTok{objekts}\SpecialCharTok{$}\NormalTok{geometrija)}
  \FunctionTok{write\_sf}\NormalTok{(slanis2,}\StringTok{"./Geodata/2024/TopographicMap/LGIAtopo10K\_v4partial.gpkg"}\NormalTok{,}
           \AttributeTok{layer=}\NormalTok{nosaukums,}
           \AttributeTok{append=}\ConstantTok{FALSE}\NormalTok{)}
\NormalTok{  ilgums}\OtherTok{=}\FunctionTok{Sys.time}\NormalTok{()}\SpecialCharTok{{-}}\NormalTok{sakums}
  \FunctionTok{print}\NormalTok{(ilgums)}
\NormalTok{\}}



\CommentTok{\# v3 {-}{-}{-}{-}}
\NormalTok{slani\_v3}\OtherTok{=}\FunctionTok{st\_layers}\NormalTok{(}\StringTok{"./Geodata/2024/TopographicMap/Latvija\_LKS92\_v3\_pilnais.gdb/"}\NormalTok{)}
\FunctionTok{write.xlsx}\NormalTok{(slani\_v3,}\StringTok{"./Geodata/2024/TopographicMap/slani\_v3.xlsx"}\NormalTok{)}

\NormalTok{slani\_v3}\SpecialCharTok{$}\NormalTok{geometrijai}\OtherTok{=}\FunctionTok{as.character}\NormalTok{(slani\_v3}\SpecialCharTok{$}\NormalTok{geomtype)}
\FunctionTok{table}\NormalTok{(slani\_v3}\SpecialCharTok{$}\NormalTok{geometrijai)}

\NormalTok{slani\_v3}\SpecialCharTok{$}\NormalTok{geometrijai2}\OtherTok{=}\FunctionTok{ifelse}\NormalTok{(slani\_v3}\SpecialCharTok{$}\NormalTok{geometrijai}\SpecialCharTok{==}\StringTok{"3D Point"}\NormalTok{,}\StringTok{"POINT"}\NormalTok{,}
                                   \FunctionTok{ifelse}\NormalTok{(slani\_v3}\SpecialCharTok{$}\NormalTok{geometrijai}\SpecialCharTok{==}\StringTok{"Multi Polygon"}\NormalTok{,}
                                          \StringTok{"MULTIPOLYGON"}\NormalTok{,}
                                          \FunctionTok{ifelse}\NormalTok{(slani\_v3}\SpecialCharTok{$}\NormalTok{geometrijai}\SpecialCharTok{==}\StringTok{"3D Multi Line String"}\NormalTok{,}
                                                 \StringTok{"MULTILINESTRING"}\NormalTok{,}
                                                 \FunctionTok{ifelse}\NormalTok{(slani\_v3}\SpecialCharTok{$}\NormalTok{geometrijai}\SpecialCharTok{==}\StringTok{"3D Multi Polygon"}\NormalTok{,}
                                                        \StringTok{"MULTIPOLYGON"}\NormalTok{,}
                                                        \FunctionTok{ifelse}\NormalTok{(slani\_v3}\SpecialCharTok{$}\NormalTok{geometrijai}\SpecialCharTok{==}\StringTok{"Point"}\NormalTok{,}\StringTok{"POINT"}\NormalTok{,}
                                                               \FunctionTok{ifelse}\NormalTok{(slani\_v3}\SpecialCharTok{$}\NormalTok{geometrijai}\SpecialCharTok{==}\StringTok{"Multi Line String"}\NormalTok{,}
                                                                      \StringTok{"MULTILINESTRING"}\NormalTok{,}
                                                                      \FunctionTok{ifelse}\NormalTok{(slani\_v3}\SpecialCharTok{$}\NormalTok{geometrijai}\SpecialCharTok{==}\StringTok{"3D Measured Point"}\NormalTok{,}
                                                                             \StringTok{"POINT"}\NormalTok{,}
                                                                             \ConstantTok{NA}\NormalTok{)))))))}

\NormalTok{slani3x}\OtherTok{=}\FunctionTok{data.frame}\NormalTok{(}\AttributeTok{name=}\NormalTok{slani\_v3}\SpecialCharTok{$}\NormalTok{name,}
                   \AttributeTok{geometrija=}\NormalTok{slani\_v3}\SpecialCharTok{$}\NormalTok{geometrijai2)}

\NormalTok{ciklam3x}\OtherTok{=}\FunctionTok{levels}\NormalTok{(}\FunctionTok{factor}\NormalTok{(slani3x}\SpecialCharTok{$}\NormalTok{name))}
\ControlFlowTok{for}\NormalTok{(i }\ControlFlowTok{in} \FunctionTok{seq\_along}\NormalTok{(ciklam3x))\{}
  \FunctionTok{print}\NormalTok{(i)}
\NormalTok{  sakums}\OtherTok{=}\FunctionTok{Sys.time}\NormalTok{()}
\NormalTok{  nosaukums}\OtherTok{=}\NormalTok{ciklam3x[i]}
\NormalTok{  objekts}\OtherTok{=}\NormalTok{slani3x }\SpecialCharTok{\%\textgreater{}\%} 
    \FunctionTok{filter}\NormalTok{(name}\SpecialCharTok{==}\NormalTok{nosaukums)}
  \FunctionTok{print}\NormalTok{(nosaukums)}
\NormalTok{  slanis}\OtherTok{=}\FunctionTok{read\_sf}\NormalTok{(}\StringTok{"./Geodata/2024/TopographicMap/Latvija\_LKS92\_v3\_pilnais.gdb/"}\NormalTok{,}
                 \AttributeTok{layer=}\NormalTok{nosaukums)}
\NormalTok{  slanisZM}\OtherTok{=}\FunctionTok{st\_zm}\NormalTok{(slanis)}
\NormalTok{  slanis2}\OtherTok{=}\FunctionTok{st\_cast}\NormalTok{(slanisZM,}\AttributeTok{to=}\NormalTok{objekts}\SpecialCharTok{$}\NormalTok{geometrija)}
  \FunctionTok{write\_sf}\NormalTok{(slanis2,}\StringTok{"./Geodata/2024/TopographicMap/LGIAtopo10K\_v3.gpkg"}\NormalTok{,}
           \AttributeTok{layer=}\NormalTok{nosaukums,}
           \AttributeTok{append=}\ConstantTok{FALSE}\NormalTok{)}
\NormalTok{  ilgums}\OtherTok{=}\FunctionTok{Sys.time}\NormalTok{()}\SpecialCharTok{{-}}\NormalTok{sakums}
  \FunctionTok{print}\NormalTok{(ilgums)}
\NormalTok{\}}


\CommentTok{\# combination {-}{-}{-}{-}}
\FunctionTok{st\_layers}\NormalTok{(}\StringTok{"./Geodata/2024/TopographicMap/LGIAtopo10K\_v3.gpkg"}\NormalTok{)}

\NormalTok{pages4}\OtherTok{=}\FunctionTok{st\_read}\NormalTok{(}\StringTok{"./Geodata/2024/TopographicMap/LGIAtopo10K\_v4partial.gpkg"}\NormalTok{,}
               \AttributeTok{layer=}\StringTok{"Topo10\_lapas"}\NormalTok{)}
\NormalTok{pages4\_united}\OtherTok{=}\FunctionTok{st\_union}\NormalTok{(pages4)}
\FunctionTok{ggplot}\NormalTok{(pages4\_united)}\SpecialCharTok{+}\FunctionTok{geom\_sf}\NormalTok{()}

\CommentTok{\# landus\_A}
\NormalTok{landus\_3}\OtherTok{=}\FunctionTok{st\_read}\NormalTok{(}\StringTok{"./Geodata/2024/TopographicMap/LGIAtopo10K\_v3.gpkg"}\NormalTok{,}
                 \AttributeTok{layer=}\StringTok{"landus\_A"}\NormalTok{)}
\NormalTok{landus\_not4}\OtherTok{=}\FunctionTok{st\_difference}\NormalTok{(landus\_3,pages4\_united)}
\NormalTok{landus\_not4}\OtherTok{=}\NormalTok{landus\_not4 }\SpecialCharTok{\%\textgreater{}\%} 
\NormalTok{  dplyr}\SpecialCharTok{::}\FunctionTok{select}\NormalTok{(FNAME,FCODE)}
\NormalTok{landus\_4}\OtherTok{=}\FunctionTok{st\_read}\NormalTok{(}\StringTok{"./Geodata/2024/TopographicMap/LGIAtopo10K\_v4partial.gpkg"}\NormalTok{,}
                 \AttributeTok{layer=}\StringTok{"landus\_A"}\NormalTok{)}
\NormalTok{landus\_4}\OtherTok{=}\NormalTok{landus\_4 }\SpecialCharTok{\%\textgreater{}\%} 
\NormalTok{  dplyr}\SpecialCharTok{::}\FunctionTok{select}\NormalTok{(FNAME,FCODE)}

\NormalTok{landus\_new}\OtherTok{=}\FunctionTok{rbind}\NormalTok{(landus\_not4,landus\_4)}
\NormalTok{sfarrow}\SpecialCharTok{::}\FunctionTok{st\_write\_parquet}\NormalTok{(landus\_new,}\StringTok{"./Geodata/2024/TopographicMap/LandusA\_COMB.parquet"}\NormalTok{)}

\CommentTok{\# bridge\_L}
\NormalTok{data\_3}\OtherTok{=}\FunctionTok{st\_read}\NormalTok{(}\StringTok{"./Geodata/2024/TopographicMap/LGIAtopo10K\_v3.gpkg"}\NormalTok{,}
               \AttributeTok{layer=}\StringTok{"bridge\_L"}\NormalTok{)}
\NormalTok{data\_not4}\OtherTok{=}\FunctionTok{st\_difference}\NormalTok{(data\_3,pages4\_united)}
\NormalTok{data\_not4}\OtherTok{=}\NormalTok{data\_not4 }\SpecialCharTok{\%\textgreater{}\%} 
\NormalTok{  dplyr}\SpecialCharTok{::}\FunctionTok{select}\NormalTok{(FNAME,FCODE)}
\NormalTok{data\_4}\OtherTok{=}\FunctionTok{st\_read}\NormalTok{(}\StringTok{"./Geodata/2024/TopographicMap/LGIAtopo10K\_v4partial.gpkg"}\NormalTok{,}
               \AttributeTok{layer=}\StringTok{"bridge\_L"}\NormalTok{)}
\NormalTok{data\_4}\OtherTok{=}\NormalTok{data\_4 }\SpecialCharTok{\%\textgreater{}\%} 
\NormalTok{  dplyr}\SpecialCharTok{::}\FunctionTok{select}\NormalTok{(FNAME,FCODE)}

\NormalTok{data\_new}\OtherTok{=}\FunctionTok{rbind}\NormalTok{(data\_not4,data\_4)}
\NormalTok{sfarrow}\SpecialCharTok{::}\FunctionTok{st\_write\_parquet}\NormalTok{(data\_new,}\StringTok{"./Geodata/2024/TopographicMap/BridgeL\_COMB.parquet"}\NormalTok{)}

\CommentTok{\# bridge\_P}
\NormalTok{data\_3}\OtherTok{=}\FunctionTok{st\_read}\NormalTok{(}\StringTok{"./Geodata/2024/TopographicMap/LGIAtopo10K\_v3.gpkg"}\NormalTok{,}
               \AttributeTok{layer=}\StringTok{"bridge\_P"}\NormalTok{)}
\NormalTok{data\_not4}\OtherTok{=}\FunctionTok{st\_difference}\NormalTok{(data\_3,pages4\_united)}
\NormalTok{data\_not4}\OtherTok{=}\NormalTok{data\_not4 }\SpecialCharTok{\%\textgreater{}\%} 
\NormalTok{  dplyr}\SpecialCharTok{::}\FunctionTok{select}\NormalTok{(FNAME,FCODE)}
\NormalTok{data\_4}\OtherTok{=}\FunctionTok{st\_read}\NormalTok{(}\StringTok{"./Geodata/2024/TopographicMap/LGIAtopo10K\_v4partial.gpkg"}\NormalTok{,}
               \AttributeTok{layer=}\StringTok{"bridge\_P"}\NormalTok{)}
\NormalTok{data\_4}\OtherTok{=}\NormalTok{data\_4 }\SpecialCharTok{\%\textgreater{}\%} 
\NormalTok{  dplyr}\SpecialCharTok{::}\FunctionTok{select}\NormalTok{(FNAME,FCODE)}

\NormalTok{data\_new}\OtherTok{=}\FunctionTok{rbind}\NormalTok{(data\_not4,data\_4)}
\NormalTok{sfarrow}\SpecialCharTok{::}\FunctionTok{st\_write\_parquet}\NormalTok{(data\_new,}\StringTok{"./Geodata/2024/TopographicMap/BridgeP\_COMB.parquet"}\NormalTok{)}

\CommentTok{\# hidro\_A}
\NormalTok{data\_3}\OtherTok{=}\FunctionTok{st\_read}\NormalTok{(}\StringTok{"./Geodata/2024/TopographicMap/LGIAtopo10K\_v3.gpkg"}\NormalTok{,}
               \AttributeTok{layer=}\StringTok{"hidro\_A"}\NormalTok{)}
\NormalTok{data\_not4}\OtherTok{=}\FunctionTok{st\_difference}\NormalTok{(data\_3,pages4\_united)}
\NormalTok{data\_not4}\OtherTok{=}\NormalTok{data\_not4 }\SpecialCharTok{\%\textgreater{}\%} 
\NormalTok{  dplyr}\SpecialCharTok{::}\FunctionTok{select}\NormalTok{(FNAME,FCODE)}
\NormalTok{data\_4}\OtherTok{=}\FunctionTok{st\_read}\NormalTok{(}\StringTok{"./Geodata/2024/TopographicMap/LGIAtopo10K\_v4partial.gpkg"}\NormalTok{,}
               \AttributeTok{layer=}\StringTok{"hidro\_A"}\NormalTok{)}
\NormalTok{data\_4}\OtherTok{=}\NormalTok{data\_4 }\SpecialCharTok{\%\textgreater{}\%} 
\NormalTok{  dplyr}\SpecialCharTok{::}\FunctionTok{select}\NormalTok{(FNAME,FCODE)}

\NormalTok{data\_new}\OtherTok{=}\FunctionTok{rbind}\NormalTok{(data\_not4,data\_4)}
\NormalTok{sfarrow}\SpecialCharTok{::}\FunctionTok{st\_write\_parquet}\NormalTok{(data\_new,}\StringTok{"./Geodata/2024/TopographicMap/HidroA\_COMB.parquet"}\NormalTok{)}

\CommentTok{\# hidro\_L}
\NormalTok{data\_3}\OtherTok{=}\FunctionTok{st\_read}\NormalTok{(}\StringTok{"./Geodata/2024/TopographicMap/LGIAtopo10K\_v3.gpkg"}\NormalTok{,}
               \AttributeTok{layer=}\StringTok{"hidro\_L"}\NormalTok{)}
\NormalTok{data\_not4}\OtherTok{=}\FunctionTok{st\_difference}\NormalTok{(data\_3,pages4\_united)}
\NormalTok{data\_not4}\OtherTok{=}\NormalTok{data\_not4 }\SpecialCharTok{\%\textgreater{}\%} 
\NormalTok{  dplyr}\SpecialCharTok{::}\FunctionTok{select}\NormalTok{(FNAME,FCODE)}
\NormalTok{data\_4}\OtherTok{=}\FunctionTok{st\_read}\NormalTok{(}\StringTok{"./Geodata/2024/TopographicMap/LGIAtopo10K\_v4partial.gpkg"}\NormalTok{,}
               \AttributeTok{layer=}\StringTok{"hidro\_L"}\NormalTok{)}
\NormalTok{data\_4}\OtherTok{=}\NormalTok{data\_4 }\SpecialCharTok{\%\textgreater{}\%} 
\NormalTok{  dplyr}\SpecialCharTok{::}\FunctionTok{select}\NormalTok{(FNAME,FCODE)}

\NormalTok{data\_new}\OtherTok{=}\FunctionTok{rbind}\NormalTok{(data\_not4,data\_4)}
\NormalTok{sfarrow}\SpecialCharTok{::}\FunctionTok{st\_write\_parquet}\NormalTok{(data\_new,}\StringTok{"./Geodata/2024/TopographicMap/HidroL\_COMB.parquet"}\NormalTok{)}


\CommentTok{\# road\_A}
\NormalTok{data\_3}\OtherTok{=}\FunctionTok{st\_read}\NormalTok{(}\StringTok{"./Geodata/2024/TopographicMap/LGIAtopo10K\_v3.gpkg"}\NormalTok{,}
               \AttributeTok{layer=}\StringTok{"road\_A"}\NormalTok{)}
\NormalTok{data\_not4}\OtherTok{=}\FunctionTok{st\_difference}\NormalTok{(data\_3,pages4\_united)}
\NormalTok{data\_not4}\OtherTok{=}\NormalTok{data\_not4 }\SpecialCharTok{\%\textgreater{}\%} 
\NormalTok{  dplyr}\SpecialCharTok{::}\FunctionTok{select}\NormalTok{(FNAME,FCODE)}
\NormalTok{data\_4}\OtherTok{=}\FunctionTok{st\_read}\NormalTok{(}\StringTok{"./Geodata/2024/TopographicMap/LGIAtopo10K\_v4partial.gpkg"}\NormalTok{,}
               \AttributeTok{layer=}\StringTok{"road\_A"}\NormalTok{)}
\NormalTok{data\_4}\OtherTok{=}\NormalTok{data\_4 }\SpecialCharTok{\%\textgreater{}\%} 
\NormalTok{  dplyr}\SpecialCharTok{::}\FunctionTok{select}\NormalTok{(FNAME,FCODE)}

\NormalTok{data\_new}\OtherTok{=}\FunctionTok{rbind}\NormalTok{(data\_not4,data\_4)}
\NormalTok{sfarrow}\SpecialCharTok{::}\FunctionTok{st\_write\_parquet}\NormalTok{(data\_new,}\StringTok{"./Geodata/2024/TopographicMap/RoadA\_COMB.parquet"}\NormalTok{)}



\CommentTok{\# road\_L}
\NormalTok{data\_3}\OtherTok{=}\FunctionTok{st\_read}\NormalTok{(}\StringTok{"./Geodata/2024/TopographicMap/LGIAtopo10K\_v3.gpkg"}\NormalTok{,}
               \AttributeTok{layer=}\StringTok{"road\_L"}\NormalTok{)}
\NormalTok{data\_not4}\OtherTok{=}\FunctionTok{st\_difference}\NormalTok{(data\_3,pages4\_united)}
\NormalTok{data\_not4}\OtherTok{=}\NormalTok{data\_not4 }\SpecialCharTok{\%\textgreater{}\%} 
\NormalTok{  dplyr}\SpecialCharTok{::}\FunctionTok{select}\NormalTok{(FNAME,FCODE)}
\NormalTok{data\_4}\OtherTok{=}\FunctionTok{st\_read}\NormalTok{(}\StringTok{"./Geodata/2024/TopographicMap/LGIAtopo10K\_v4partial.gpkg"}\NormalTok{,}
               \AttributeTok{layer=}\StringTok{"road\_L"}\NormalTok{)}
\NormalTok{data\_4}\OtherTok{=}\NormalTok{data\_4 }\SpecialCharTok{\%\textgreater{}\%} 
\NormalTok{  dplyr}\SpecialCharTok{::}\FunctionTok{select}\NormalTok{(FNAME,FCODE)}

\NormalTok{data\_new}\OtherTok{=}\FunctionTok{rbind}\NormalTok{(data\_not4,data\_4)}
\NormalTok{sfarrow}\SpecialCharTok{::}\FunctionTok{st\_write\_parquet}\NormalTok{(data\_new,}\StringTok{"./Geodata/2024/TopographicMap/RoadL\_COMB.parquet"}\NormalTok{)}



\CommentTok{\# swamp\_A}
\NormalTok{data\_3}\OtherTok{=}\FunctionTok{st\_read}\NormalTok{(}\StringTok{"./Geodata/2024/TopographicMap/LGIAtopo10K\_v3.gpkg"}\NormalTok{,}
               \AttributeTok{layer=}\StringTok{"swamp\_A"}\NormalTok{)}
\NormalTok{data\_not4}\OtherTok{=}\FunctionTok{st\_difference}\NormalTok{(data\_3,pages4\_united)}
\NormalTok{data\_not4}\OtherTok{=}\NormalTok{data\_not4 }\SpecialCharTok{\%\textgreater{}\%} 
\NormalTok{  dplyr}\SpecialCharTok{::}\FunctionTok{select}\NormalTok{(FNAME,FCODE)}
\NormalTok{data\_4}\OtherTok{=}\FunctionTok{st\_read}\NormalTok{(}\StringTok{"./Geodata/2024/TopographicMap/LGIAtopo10K\_v4partial.gpkg"}\NormalTok{,}
               \AttributeTok{layer=}\StringTok{"swamp\_A"}\NormalTok{)}
\NormalTok{data\_4}\OtherTok{=}\NormalTok{data\_4 }\SpecialCharTok{\%\textgreater{}\%} 
\NormalTok{  dplyr}\SpecialCharTok{::}\FunctionTok{select}\NormalTok{(FNAME,FCODE)}

\NormalTok{data\_new}\OtherTok{=}\FunctionTok{rbind}\NormalTok{(data\_not4,data\_4)}
\NormalTok{sfarrow}\SpecialCharTok{::}\FunctionTok{st\_write\_parquet}\NormalTok{(data\_new,}\StringTok{"./Geodata/2024/TopographicMap/SwampA\_COMB.parquet"}\NormalTok{)}



\CommentTok{\# flora\_L}
\NormalTok{data\_3}\OtherTok{=}\FunctionTok{st\_read}\NormalTok{(}\StringTok{"./Geodata/2024/TopographicMap/LGIAtopo10K\_v3.gpkg"}\NormalTok{,}
               \AttributeTok{layer=}\StringTok{"flora\_L"}\NormalTok{)}
\NormalTok{data\_not4}\OtherTok{=}\FunctionTok{st\_difference}\NormalTok{(data\_3,pages4\_united)}
\NormalTok{data\_not4}\OtherTok{=}\NormalTok{data\_not4 }\SpecialCharTok{\%\textgreater{}\%} 
\NormalTok{  dplyr}\SpecialCharTok{::}\FunctionTok{select}\NormalTok{(FNAME,FCODE)}
\NormalTok{data\_4}\OtherTok{=}\FunctionTok{st\_read}\NormalTok{(}\StringTok{"./Geodata/2024/TopographicMap/LGIAtopo10K\_v4partial.gpkg"}\NormalTok{,}
               \AttributeTok{layer=}\StringTok{"flora\_L"}\NormalTok{)}
\NormalTok{data\_4}\OtherTok{=}\NormalTok{data\_4 }\SpecialCharTok{\%\textgreater{}\%} 
\NormalTok{  dplyr}\SpecialCharTok{::}\FunctionTok{select}\NormalTok{(FNAME,FCODE)}

\NormalTok{data\_new}\OtherTok{=}\FunctionTok{rbind}\NormalTok{(data\_not4,data\_4)}
\NormalTok{sfarrow}\SpecialCharTok{::}\FunctionTok{st\_write\_parquet}\NormalTok{(data\_new,}\StringTok{"./Geodata/2024/TopographicMap/FloraL\_COMB.parquet"}\NormalTok{)}


\CommentTok{\# build\_A}
\NormalTok{data\_3}\OtherTok{=}\FunctionTok{st\_read}\NormalTok{(}\StringTok{"./Geodata/2024/TopographicMap/LGIAtopo10K\_v3.gpkg"}\NormalTok{,}
               \AttributeTok{layer=}\StringTok{"build\_A"}\NormalTok{)}
\NormalTok{data\_3}\OtherTok{=}\NormalTok{data\_3 }\SpecialCharTok{\%\textgreater{}\%} 
\NormalTok{  dplyr}\SpecialCharTok{::}\FunctionTok{select}\NormalTok{(FNAME,FCODE)}
\NormalTok{sfarrow}\SpecialCharTok{::}\FunctionTok{st\_write\_parquet}\NormalTok{(data\_3,}\StringTok{"./Geodata/2024/TopographicMap/BuildA\_v3.parquet"}\NormalTok{)}
\end{Highlighting}
\end{Shaded}

\section{Corine Land Cover 2018}\label{Ch04.05}

Corine Land Cover is a publicly available geodata that characterizes land cover
and land use (LULC) across Europe over a long period of time using a generally
consistent (comparable) \href{https://land.copernicus.eu/content/corine-land-cover-nomenclature-guidelines/docs/pdf/CLC2018_Nomenclature_illustrated_guide_20190510.pdf}{methodology},
providing results for \href{https://land.copernicus.eu/en/products/corine-land-cover}{individual years} - 1990, 2000, 2006, 2012,
2018.
Although the dataset has a coarse resolution -- the mapping unit is 25 ha areas
that are at least 100 m wide -- it provides sufficient information for general
use, such as workflow testing and observation filtering. This project uses
data from 2018.

The downloaded data set has been transformed into the Latvian coordinate
system (EPSG:3059), and the file format has been changed to GeoParquet to
facilitate and speed up further work. As part of the file format change,
geometries (empty, valid) have been checked.

Data are stored at \texttt{Geodata/2024/CLC/}.

\begin{Shaded}
\begin{Highlighting}[]
\CommentTok{\# libs {-}{-}{-}{-}}
\ControlFlowTok{if}\NormalTok{(}\SpecialCharTok{!}\FunctionTok{require}\NormalTok{(sf)) \{}\FunctionTok{install.packages}\NormalTok{(}\StringTok{"sf"}\NormalTok{); }\FunctionTok{require}\NormalTok{(sf)\}}
\ControlFlowTok{if}\NormalTok{(}\SpecialCharTok{!}\FunctionTok{require}\NormalTok{(arrow)) \{}\FunctionTok{install.packages}\NormalTok{(}\StringTok{"arrow"}\NormalTok{); }\FunctionTok{require}\NormalTok{(arrow)\}}
\ControlFlowTok{if}\NormalTok{(}\SpecialCharTok{!}\FunctionTok{require}\NormalTok{(sfarrow)) \{}\FunctionTok{install.packages}\NormalTok{(}\StringTok{"sfarrow"}\NormalTok{); }\FunctionTok{require}\NormalTok{(sfarrow)\}}

\CommentTok{\# downloaded data}
\NormalTok{clcLV}\OtherTok{=}\FunctionTok{st\_read}\NormalTok{(}\StringTok{"./Geodata/2024/CLC/clcLV.gpkg"}\NormalTok{,}\AttributeTok{layer=}\StringTok{"clcLV"}\NormalTok{)}

\CommentTok{\# empty geoms}
\NormalTok{clcLV2 }\OtherTok{=}\NormalTok{ clcLV[}\SpecialCharTok{!}\FunctionTok{st\_is\_empty}\NormalTok{(clcLV),,drop}\OtherTok{=}\ConstantTok{FALSE}\NormalTok{] }\CommentTok{\# OK}

\CommentTok{\# validation}
\NormalTok{validity}\OtherTok{=}\FunctionTok{st\_is\_valid}\NormalTok{(clcLV2) }
\FunctionTok{table}\NormalTok{(validity) }\CommentTok{\# 3 non{-}valid}
\NormalTok{clcLV3}\OtherTok{=}\FunctionTok{st\_make\_valid}\NormalTok{(clcLV2)}

\CommentTok{\# crs}
\NormalTok{clcLV3}\OtherTok{=}\FunctionTok{st\_transform}\NormalTok{(clcLV3,}\AttributeTok{crs=}\DecValTok{3059}\NormalTok{)}

\CommentTok{\# saving}
\NormalTok{sfarrow}\SpecialCharTok{::}\FunctionTok{st\_write\_parquet}\NormalTok{(clcLV3, }\StringTok{"./Geodata/2024/CLC/CLC\_LV\_2018.parquet"}\NormalTok{)}
\end{Highlighting}
\end{Shaded}

\section{Publicly available LVM data}\label{Ch04.06}

\href{https://data.gov.lv/dati/lv/dataset/as-latvijas-valsts-mezi-mezsaimniecibas-infrastruktura}{Latvian State Forests geospatial data on forest infrastructure and its description}. The
following datasets were used in the project:

\begin{itemize}
\item
  roads:

  \begin{itemize}
  \item
    forest roads;
  \item
    forest roads to be developed;
  \item
    turning areas;
  \item
    changeover areas;
  \item
    driveways;
  \end{itemize}
\item
  drainage systems:

  \begin{itemize}
  \item
    ditches;
  \item
    drainage systems;
  \item
    renovated drainage facilities.
  \end{itemize}
\end{itemize}

Initially, no additional processing of this data was performed. It was used to
prepare \hyperref[Ch05]{geodata products} (more specifically, \hyperref[Ch05.03]{Landscape classification}).

Data were downloaded to \texttt{Geodata/2024/LVM\_OpenData}

\section{Soil data}\label{Ch04.07}

Directory \texttt{Geodata/2024/Soils/} contains various soil related datasets that need
to be combined (soil texture) or can be used individually (soil chemistry). These
datasets and their location in the file tree are documented in following subchapters.

\subsection{Soil chemistry}\label{Ch04.07.01}

Data on soil chemistry are obtained from \href{https://esdac.jrc.ec.europa.eu/}{European Soil Data Centre's} European
Soil database (\citeproc{ref-esdac2}{Panagos et al., 2022}). Dataset decribing soil chemistry is derived from \href{https://esdac.jrc.ec.europa.eu/content/chemical-properties-european-scale-based-lucas-topsoil-data}{LUCAS
2009/2012 topsoil data}. There are several chemical properties available for
download, however not all of them were choser by experts for SDM:

\begin{itemize}
\item
  ``P'': used;
\item
  ``N'': used;
\item
  ``K'': used;
\item
  ``CEC'': not used;
\item
  ``CN'': used;
\item
  ``pH\_CaCl'': not used;
\item
  ``ph\_H2O\_ration\_ph\_CaCl'': not used;
\item
  ``pH\_H2O'': used;
\item
  ``CaCO3'': used.
\end{itemize}

Files were downloaded to \texttt{Geodata/2024/Soils/ESDAC/chemistry/} and no preprocessing
was carried out.

\subsection{Soil texture: Europe}\label{Ch04.07.02}

Data on soil texture were obtained from \href{https://esdac.jrc.ec.europa.eu/}{European Soil Data Centre's} European
Soil database (\citeproc{ref-esdac2}{Panagos et al., 2022}). Dataset is available as \href{https://esdac.jrc.ec.europa.eu/content/european-soil-database-v2-raster-library-1kmx1km}{European Soil Database v2 Raster Library 1kmx1km}. There
are several properties available for download, \texttt{TXT} was used to
create \hyperref[Ch05.02]{soil texture product}. Files were downloaded to \texttt{Geodata/2024/Soils/ESDAC/texture/}.

During the preprocessing (see code below) the layer was
projected to match the 10 m template with ``near'' as interpolation method, value \texttt{0}
substituted with \texttt{NA} and the result wars masked and cropped to the template.
Result was saved for further processing.

\begin{Shaded}
\begin{Highlighting}[]
\CommentTok{\# libs {-}{-}{-}{-}}
\ControlFlowTok{if}\NormalTok{(}\SpecialCharTok{!}\FunctionTok{require}\NormalTok{(terra)) \{}\FunctionTok{install.packages}\NormalTok{(}\StringTok{"terra"}\NormalTok{); }\FunctionTok{require}\NormalTok{(terra)\}}
\ControlFlowTok{if}\NormalTok{(}\SpecialCharTok{!}\FunctionTok{require}\NormalTok{(sf)) \{}\FunctionTok{install.packages}\NormalTok{(}\StringTok{"sf"}\NormalTok{); }\FunctionTok{require}\NormalTok{(sf)\}}
\ControlFlowTok{if}\NormalTok{(}\SpecialCharTok{!}\FunctionTok{require}\NormalTok{(tidyverse)) \{}\FunctionTok{install.packages}\NormalTok{(}\StringTok{"tidyverse"}\NormalTok{); }\FunctionTok{require}\NormalTok{(tidyverse)\}}

\CommentTok{\# Template {-}{-}{-}{-}}
\NormalTok{template10}\OtherTok{=}\FunctionTok{rast}\NormalTok{(}\StringTok{"./Templates/TemplateRasters/LV10m\_10km.tif"}\NormalTok{)}

\CommentTok{\# ESDAC texture {-}{-}{-}{-}}

\NormalTok{sdTEXT}\OtherTok{=}\FunctionTok{rast}\NormalTok{(}\FunctionTok{paste0}\NormalTok{(}\StringTok{"./Geodata/2024/Soils/ESDAC/texture/SoilDatabaseV2\_raster/"}\NormalTok{,}
                   \StringTok{"ESDB{-}Raster{-}Library{-}1k{-}GeoTIFF{-}20240507/TEXT/TEXT.tif"}\NormalTok{))}
\FunctionTok{plot}\NormalTok{(sdTEXT)}

\NormalTok{sdTEXT}\OtherTok{=}\FunctionTok{project}\NormalTok{(sdTEXT,template10,}\AttributeTok{method=}\StringTok{"near"}\NormalTok{)}
\FunctionTok{plot}\NormalTok{(sdTEXT)}

\NormalTok{sdTEXT}\OtherTok{=}\FunctionTok{subst}\NormalTok{(sdTEXT,}\DecValTok{0}\NormalTok{,}\ConstantTok{NA}\NormalTok{)}
\FunctionTok{plot}\NormalTok{(sdTEXT)}

\NormalTok{sdTEXT2}\OtherTok{=}\FunctionTok{mask}\NormalTok{(sdTEXT,template10,}
             \AttributeTok{filename=}\StringTok{"./RasterGrids\_10m/2024/SoilTXT\_ESDAC.tif"}\NormalTok{,}
             \AttributeTok{overwrite=}\ConstantTok{TRUE}\NormalTok{)}
\FunctionTok{plot}\NormalTok{(sdTEXT2)}
\end{Highlighting}
\end{Shaded}

\subsection{Soil texture: Farmland}\label{Ch04.07.03}

Topsoil characteristics in Latvia were mapped in the mid-20th century, almost
exclusively in farmlands. With time, data were digitised and combined with some
other information resulting in artefacts. Therefore preprocessing was necessary. The
version we used was obtained from the project ``GOODWATER'' C1D1\_Deliverable\_R2.

File is stored at \texttt{Geodata/2024/Soils/TopSoil\_LV/}.

Preprocessing included:

\begin{itemize}
\item
  reclassification based on the field \texttt{GrSast}:

  \begin{itemize}
  \item
    \texttt{sand} (1): ``mS'', ``mSp'', ``S'', ``sS'', ``iS'', ``Gr'', ``mGr'', ``D'';
  \item
    \texttt{silt} (2): ``sM'', ``sMp'', ``sM2'', ``sMp2'', ``sM3'', ``sMp3'';
  \item
    \texttt{clay} (3): ``M'',``M1'',``Mp'',``M2'',``sM1'',``sMp1'';
  \item
    \texttt{organic} (4): ``l'', ``vd'', ``vj'', ``n'',``T'';
  \item
    and other categories were left unclassified.
  \end{itemize}
\item
  coordinate transformation to EPSG:3059;
\item
  invsestigation of the resulting layer looking for anomalies by scrolling in interactive
  GIS, which led to exclusion of land parcels larger than 200 ha.
\item
  rasterisation to match the 10 m template with the highest class code prevailing.
\end{itemize}

\begin{Shaded}
\begin{Highlighting}[]
\CommentTok{\# libs {-}{-}{-}{-}}
\ControlFlowTok{if}\NormalTok{(}\SpecialCharTok{!}\FunctionTok{require}\NormalTok{(terra)) \{}\FunctionTok{install.packages}\NormalTok{(}\StringTok{"terra"}\NormalTok{); }\FunctionTok{require}\NormalTok{(terra)\}}
\ControlFlowTok{if}\NormalTok{(}\SpecialCharTok{!}\FunctionTok{require}\NormalTok{(sf)) \{}\FunctionTok{install.packages}\NormalTok{(}\StringTok{"sf"}\NormalTok{); }\FunctionTok{require}\NormalTok{(sf)\}}
\ControlFlowTok{if}\NormalTok{(}\SpecialCharTok{!}\FunctionTok{require}\NormalTok{(tidyverse)) \{}\FunctionTok{install.packages}\NormalTok{(}\StringTok{"tidyverse"}\NormalTok{); }\FunctionTok{require}\NormalTok{(tidyverse)\}}

\CommentTok{\# Template {-}{-}{-}{-}}
\NormalTok{template10}\OtherTok{=}\FunctionTok{rast}\NormalTok{(}\StringTok{"./Templates/TemplateRasters/LV10m\_10km.tif"}\NormalTok{)}

\CommentTok{\# Farmland soil texture {-}{-}{-}{-}}

\NormalTok{augsnes}\OtherTok{=}\FunctionTok{st\_read}\NormalTok{(}\StringTok{"./Geodata/2024/Soils/TopSoil\_LV/soil.gpkg"}\NormalTok{,}\AttributeTok{layer=}\StringTok{"soilunion"}\NormalTok{)}

\CommentTok{\# calculate parcels area}
\NormalTok{augsnes}\SpecialCharTok{$}\NormalTok{platiba\_ha}\OtherTok{=}\FunctionTok{as.numeric}\NormalTok{(}\FunctionTok{st\_area}\NormalTok{(augsnes))}\SpecialCharTok{/}\DecValTok{10000}

\CommentTok{\# only parcels with existing information on texture}
\NormalTok{tuksas}\OtherTok{=}\NormalTok{augsnes }\SpecialCharTok{\%\textgreater{}\%} 
  \FunctionTok{filter}\NormalTok{(GrSast}\SpecialCharTok{==}\StringTok{""}\NormalTok{)}

\CommentTok{\# classification}
\NormalTok{clay}\OtherTok{=}\FunctionTok{c}\NormalTok{(}\StringTok{"M"}\NormalTok{,}\StringTok{"M1"}\NormalTok{,}\StringTok{"Mp"}\NormalTok{,}\StringTok{"M2"}\NormalTok{,}\StringTok{"sM1"}\NormalTok{,}\StringTok{"sMp1"}\NormalTok{)}
\NormalTok{silt}\OtherTok{=}\FunctionTok{c}\NormalTok{(}\StringTok{"sM"}\NormalTok{, }\StringTok{"sMp"}\NormalTok{, }\StringTok{"sM2"}\NormalTok{, }\StringTok{"sMp2"}\NormalTok{, }\StringTok{"sM3"}\NormalTok{, }\StringTok{"sMp3"}\NormalTok{)}
\NormalTok{sand}\OtherTok{=}\FunctionTok{c}\NormalTok{(}\StringTok{"mS"}\NormalTok{, }\StringTok{"mSp"}\NormalTok{, }\StringTok{"S"}\NormalTok{, }\StringTok{"sS"}\NormalTok{, }\StringTok{"iS"}\NormalTok{, }\StringTok{"Gr"}\NormalTok{, }\StringTok{"mGr"}\NormalTok{, }\StringTok{"D"}\NormalTok{)}
\NormalTok{peat}\OtherTok{=}\FunctionTok{c}\NormalTok{(}\StringTok{"l"}\NormalTok{, }\StringTok{"vd"}\NormalTok{, }\StringTok{"vj"}\NormalTok{, }\StringTok{"n"}\NormalTok{,}\StringTok{"T"}\NormalTok{)}
\NormalTok{augsnes}\OtherTok{=}\NormalTok{augsnes }\SpecialCharTok{\%\textgreater{}\%} 
  \FunctionTok{mutate}\NormalTok{(}\AttributeTok{grupas=}\FunctionTok{case\_when}\NormalTok{(GrSast }\SpecialCharTok{\%in\%}\NormalTok{ sand}\SpecialCharTok{\textasciitilde{}}\StringTok{"Sand"}\NormalTok{,}
\NormalTok{                          GrSast }\SpecialCharTok{\%in\%}\NormalTok{ silt}\SpecialCharTok{\textasciitilde{}}\StringTok{"Silt"}\NormalTok{,}
\NormalTok{                          GrSast }\SpecialCharTok{\%in\%}\NormalTok{ clay}\SpecialCharTok{\textasciitilde{}}\StringTok{"Clay"}\NormalTok{,}
\NormalTok{                          GrSast }\SpecialCharTok{\%in\%}\NormalTok{ peat}\SpecialCharTok{\textasciitilde{}}\StringTok{"organika"}\NormalTok{,}
                          \AttributeTok{.default=}\ConstantTok{NA}\NormalTok{)) }\SpecialCharTok{\%\textgreater{}\%} 
  \FunctionTok{mutate}\NormalTok{(}\AttributeTok{grupas\_num=}\FunctionTok{case\_when}\NormalTok{(GrSast }\SpecialCharTok{\%in\%}\NormalTok{ sand}\SpecialCharTok{\textasciitilde{}}\StringTok{"1"}\NormalTok{,}
\NormalTok{                              GrSast }\SpecialCharTok{\%in\%}\NormalTok{ silt}\SpecialCharTok{\textasciitilde{}}\StringTok{"2"}\NormalTok{,}
\NormalTok{                              GrSast }\SpecialCharTok{\%in\%}\NormalTok{ clay}\SpecialCharTok{\textasciitilde{}}\StringTok{"3"}\NormalTok{,}
\NormalTok{                              GrSast }\SpecialCharTok{\%in\%}\NormalTok{ peat}\SpecialCharTok{\textasciitilde{}}\StringTok{"4"}\NormalTok{,}
                              \AttributeTok{.default=}\ConstantTok{NA}\NormalTok{))}

\CommentTok{\# crs}
\NormalTok{augsnes\_3059}\OtherTok{=}\FunctionTok{st\_transform}\NormalTok{(augsnes,}\AttributeTok{crs=}\DecValTok{3059}\NormalTok{)}

\CommentTok{\# only existing texture classification}
\NormalTok{augsnes\_3059}\OtherTok{=}\NormalTok{augsnes\_3059 }\SpecialCharTok{\%\textgreater{}\%} 
  \FunctionTok{filter}\NormalTok{(}\SpecialCharTok{!}\FunctionTok{is.na}\NormalTok{(grupas\_num))}

\CommentTok{\# parcels up to 200 ha}
\NormalTok{augsnes\_3059small}\OtherTok{=}\NormalTok{augsnes\_3059 }\SpecialCharTok{\%\textgreater{}\%} 
  \FunctionTok{filter}\NormalTok{(}\SpecialCharTok{!}\FunctionTok{is.na}\NormalTok{(grupas\_num)) }\SpecialCharTok{\%\textgreater{}\%} 
  \FunctionTok{filter}\NormalTok{(platiba\_ha}\SpecialCharTok{\textless{}}\DecValTok{200}\NormalTok{)}

\CommentTok{\# rasterisation}
\NormalTok{virsaugsnem2}\OtherTok{=}\FunctionTok{rasterize}\NormalTok{(augsnes\_3059small,template10,}\AttributeTok{field=}\StringTok{"grupas\_num"}\NormalTok{,}\AttributeTok{fun=}\StringTok{"max"}\NormalTok{,}
                       \AttributeTok{filename=}\StringTok{"./RasterGrids\_10m/2024/SoilTXT\_topSoilLV.tif"}\NormalTok{,}
                       \AttributeTok{overwrite=}\ConstantTok{TRUE}\NormalTok{)}
\FunctionTok{plot}\NormalTok{(virsaugsnem2)}
\end{Highlighting}
\end{Shaded}

\subsection{Soil texture: Quaternary}\label{Ch04.07.04}

Data on Quaternary Geology are digitised and stored by the University of Latvia
Department of Geology.

File is stored at \texttt{Geodata/2024/Soils/QuaternaryGeology\_LV/}.

Preprocessing included:

\begin{itemize}
\item
  reclassification based on field \texttt{Litologija}:

  \begin{itemize}
  \item
    \texttt{sand} (1): ``smilts'', ``smilts\_aleiritiska'',
    ``smilts\_dunjaina'', ``smilts\_grants'', ``smilts\_grants\_oli'', ``smilts\_grants\_oli\_aleirits'', ``smilts\_kudraina'',
    ``smilts\_videjgraudaina, malsmilts'', ``smilts\_videjgraudaina''\textasciitilde{}``Sand'';
  \item
    \texttt{silt} (2): ``aleirits'', ``aleirits\_malains'',
    ``morena'', ``smilts\_aleirits\_mals'', ``smilts\_aleirits\_sapropelis'', ``smilts\_malaina\_dazadgraudaina, malsmilts'';
  \item
    \texttt{clay} (3): ``mals'', ``mals\_aleiritisks'';
  \item
    \texttt{organic} (4): ``dunjas'', ``kudra'';
  \end{itemize}
\item
  coordinate transformation to EPSG:3059;
\item
  rasterisation to match the 10 m template with the highest class code prevailing.
\end{itemize}

\begin{Shaded}
\begin{Highlighting}[]
\CommentTok{\# libs {-}{-}{-}{-}}
\ControlFlowTok{if}\NormalTok{(}\SpecialCharTok{!}\FunctionTok{require}\NormalTok{(terra)) \{}\FunctionTok{install.packages}\NormalTok{(}\StringTok{"terra"}\NormalTok{); }\FunctionTok{require}\NormalTok{(terra)\}}
\ControlFlowTok{if}\NormalTok{(}\SpecialCharTok{!}\FunctionTok{require}\NormalTok{(sf)) \{}\FunctionTok{install.packages}\NormalTok{(}\StringTok{"sf"}\NormalTok{); }\FunctionTok{require}\NormalTok{(sf)\}}
\ControlFlowTok{if}\NormalTok{(}\SpecialCharTok{!}\FunctionTok{require}\NormalTok{(tidyverse)) \{}\FunctionTok{install.packages}\NormalTok{(}\StringTok{"tidyverse"}\NormalTok{); }\FunctionTok{require}\NormalTok{(tidyverse)\}}

\CommentTok{\# Template {-}{-}{-}{-}}
\NormalTok{template10}\OtherTok{=}\FunctionTok{rast}\NormalTok{(}\StringTok{"./Templates/TemplateRasters/LV10m\_10km.tif"}\NormalTok{)}

\CommentTok{\# Quarternary geology {-}{-}{-}{-}}

\NormalTok{kvartars}\OtherTok{=}\NormalTok{sfarrow}\SpecialCharTok{::}\FunctionTok{st\_read\_parquet}\NormalTok{(}\StringTok{"./Geodata/2024/Soils/QuaternaryGeology\_LV/Kvartargeologija.parquet"}\NormalTok{)}

\CommentTok{\# reclassification}
\NormalTok{kvartars}\OtherTok{=}\NormalTok{kvartars }\SpecialCharTok{\%\textgreater{}\%} 
  \FunctionTok{mutate}\NormalTok{(}\AttributeTok{grupas =} \FunctionTok{case\_when}\NormalTok{(Litologija}\SpecialCharTok{==}\StringTok{"aleirits"}\SpecialCharTok{\textasciitilde{}}\StringTok{"Silt"}\NormalTok{,}
\NormalTok{                            Litologija}\SpecialCharTok{==}\StringTok{"aleirits\_malains"}\SpecialCharTok{\textasciitilde{}}\StringTok{"Silt"}\NormalTok{,}
\NormalTok{                            Litologija}\SpecialCharTok{==}\StringTok{"dunjas"}\SpecialCharTok{\textasciitilde{}}\StringTok{"organika"}\NormalTok{,}
\NormalTok{                            Litologija}\SpecialCharTok{==}\StringTok{"kudra"}\SpecialCharTok{\textasciitilde{}}\StringTok{"organika"}\NormalTok{,}
\NormalTok{                            Litologija}\SpecialCharTok{==}\StringTok{"mals"}\SpecialCharTok{\textasciitilde{}}\StringTok{"Clay"}\NormalTok{,}
\NormalTok{                            Litologija}\SpecialCharTok{==}\StringTok{"mals\_aleiritisks"}\SpecialCharTok{\textasciitilde{}}\StringTok{"Clay"}\NormalTok{,}
\NormalTok{                            Litologija}\SpecialCharTok{==}\StringTok{"morena"}\SpecialCharTok{\textasciitilde{}}\StringTok{"Silt"}\NormalTok{,}
\NormalTok{                            Litologija}\SpecialCharTok{==}\StringTok{"smilts"}\SpecialCharTok{\textasciitilde{}}\StringTok{"Sand"}\NormalTok{,}
\NormalTok{                            Litologija}\SpecialCharTok{==}\StringTok{"smilts\_aleiritiska"}\SpecialCharTok{\textasciitilde{}}\StringTok{"Sand"}\NormalTok{,}
\NormalTok{                            Litologija}\SpecialCharTok{==}\StringTok{"smilts\_aleirits\_mals"}\SpecialCharTok{\textasciitilde{}}\StringTok{"Silt"}\NormalTok{,}
\NormalTok{                            Litologija}\SpecialCharTok{==}\StringTok{"smilts\_aleirits\_sapropelis"}\SpecialCharTok{\textasciitilde{}}\StringTok{"Silt"}\NormalTok{,}
\NormalTok{                            Litologija}\SpecialCharTok{==}\StringTok{"smilts\_dunjaina"}\SpecialCharTok{\textasciitilde{}}\StringTok{"Sand"}\NormalTok{,}
\NormalTok{                            Litologija}\SpecialCharTok{==}\StringTok{"smilts\_grants"}\SpecialCharTok{\textasciitilde{}}\StringTok{"Sand"}\NormalTok{,}
\NormalTok{                            Litologija}\SpecialCharTok{==}\StringTok{"smilts\_grants\_oli"}\SpecialCharTok{\textasciitilde{}}\StringTok{"Sand"}\NormalTok{,}
\NormalTok{                            Litologija}\SpecialCharTok{==}\StringTok{"smilts\_grants\_oli\_aleirits"}\SpecialCharTok{\textasciitilde{}}\StringTok{"Sand"}\NormalTok{,}
\NormalTok{                            Litologija}\SpecialCharTok{==}\StringTok{"smilts\_kudraina"}\SpecialCharTok{\textasciitilde{}}\StringTok{"Sand"}\NormalTok{,}
\NormalTok{                            Litologija}\SpecialCharTok{==}\StringTok{"smilts\_malaina\_dazadgraudaina, malsmilts"}\SpecialCharTok{\textasciitilde{}}\StringTok{"Silt"}\NormalTok{,}
\NormalTok{                            Litologija}\SpecialCharTok{==}\StringTok{"smilts\_videjgraudaina, malsmilts"}\SpecialCharTok{\textasciitilde{}}\StringTok{"Sand"}\NormalTok{,}
\NormalTok{                            Litologija}\SpecialCharTok{==}\StringTok{"smilts\_videjgraudaina"}\SpecialCharTok{\textasciitilde{}}\StringTok{"Sand"}\NormalTok{,}
                            \AttributeTok{.default=}\ConstantTok{NA}\NormalTok{))}
\CommentTok{\# numeric codes}
\NormalTok{kvartars}\OtherTok{=}\NormalTok{kvartars }\SpecialCharTok{\%\textgreater{}\%} 
  \FunctionTok{mutate}\NormalTok{(}\AttributeTok{grupas\_num=}\FunctionTok{case\_when}\NormalTok{(grupas }\SpecialCharTok{==} \StringTok{"Sand"} \SpecialCharTok{\textasciitilde{}}\StringTok{"1"}\NormalTok{,}
\NormalTok{                              grupas }\SpecialCharTok{==} \StringTok{"Silt"} \SpecialCharTok{\textasciitilde{}}\StringTok{"2"}\NormalTok{,}
\NormalTok{                              grupas }\SpecialCharTok{==} \StringTok{"Clay"} \SpecialCharTok{\textasciitilde{}}\StringTok{"3"}\NormalTok{,}
\NormalTok{                              grupas }\SpecialCharTok{==} \StringTok{"organika"} \SpecialCharTok{\textasciitilde{}}\StringTok{"4"}\NormalTok{,}
                              \AttributeTok{.default=}\ConstantTok{NA}\NormalTok{))}

\CommentTok{\# crs transformation}
\NormalTok{kvartars\_3059}\OtherTok{=}\FunctionTok{st\_transform}\NormalTok{(kvartars,}\AttributeTok{crs=}\DecValTok{3059}\NormalTok{)}

\CommentTok{\# nonmissing classes}
\NormalTok{kvartars\_3059}\OtherTok{=}\NormalTok{kvartars\_3059 }\SpecialCharTok{\%\textgreater{}\%} 
  \FunctionTok{filter}\NormalTok{(}\SpecialCharTok{!}\FunctionTok{is.na}\NormalTok{(grupas\_num))}

\CommentTok{\# rasterisation}
\NormalTok{apaksaugsnem}\OtherTok{=}\FunctionTok{rasterize}\NormalTok{(kvartars\_3059,template10,}\AttributeTok{field=}\StringTok{"grupas\_num"}\NormalTok{,}\AttributeTok{fun=}\StringTok{"max"}\NormalTok{,}
                       \AttributeTok{filename=}\StringTok{"./RasterGrids\_10m/2024/SoilTXT\_QuarternaryLV.tif"}\NormalTok{,}
                       \AttributeTok{overwrite=}\ConstantTok{TRUE}\NormalTok{)}
\FunctionTok{plot}\NormalTok{(apaksaugsnem)}
\end{Highlighting}
\end{Shaded}

\subsection{Organic soils: SILAVA}\label{Ch04.07.05}

The distribution of organic soils was modelled under the EU LIFE Programme project
``Demonstration of climate change mitigation potential of nutrients rich organic
soils in Baltic States and Finland'' at the scientific institue SILAVA. Results
were downloaded and stored at \texttt{Geodata/2024/Soils/OrganicSoils\_SILAVA/}.

Even though the layer covers all of Latvia, it has visible inconsistencies,
particularly stripes. These were digitised manually (as vector polygons) and masked
out as a part of preprocessing.

For further soil texture analysis we saved a GeoTIFF file with only presences.

\begin{Shaded}
\begin{Highlighting}[]
\CommentTok{\# libs {-}{-}{-}{-}}
\ControlFlowTok{if}\NormalTok{(}\SpecialCharTok{!}\FunctionTok{require}\NormalTok{(terra)) \{}\FunctionTok{install.packages}\NormalTok{(}\StringTok{"terra"}\NormalTok{); }\FunctionTok{require}\NormalTok{(terra)\}}
\ControlFlowTok{if}\NormalTok{(}\SpecialCharTok{!}\FunctionTok{require}\NormalTok{(sf)) \{}\FunctionTok{install.packages}\NormalTok{(}\StringTok{"sf"}\NormalTok{); }\FunctionTok{require}\NormalTok{(sf)\}}
\ControlFlowTok{if}\NormalTok{(}\SpecialCharTok{!}\FunctionTok{require}\NormalTok{(tidyverse)) \{}\FunctionTok{install.packages}\NormalTok{(}\StringTok{"tidyverse"}\NormalTok{); }\FunctionTok{require}\NormalTok{(tidyverse)\}}

\CommentTok{\# Template {-}{-}{-}{-}}
\NormalTok{template10}\OtherTok{=}\FunctionTok{rast}\NormalTok{(}\StringTok{"./Templates/TemplateRasters/LV10m\_10km.tif"}\NormalTok{)}

\CommentTok{\# Organic Soils SILAVA {-}{-}{-}{-}}

\NormalTok{organika\_silava}\OtherTok{=}\FunctionTok{rast}\NormalTok{(}\StringTok{"./Geodata/2024/Soils/OrganicSoils\_SILAVA/Silava\_OrgSoils.tif"}\NormalTok{)}
\FunctionTok{plot}\NormalTok{(organika\_silava)}
\CommentTok{\# visible stripes}

\CommentTok{\# only 40+ cm deep}
\NormalTok{organika\_silava}\OtherTok{=}\FunctionTok{ifel}\NormalTok{(organika\_silava}\SpecialCharTok{==}\DecValTok{2}\NormalTok{,}\DecValTok{1}\NormalTok{,}\ConstantTok{NA}\NormalTok{)}
\NormalTok{organika\_silavaLV}\OtherTok{=}\FunctionTok{project}\NormalTok{(organika\_silava,template10)}

\CommentTok{\# stripes drawn manually, rasterisation}
\NormalTok{silavas\_telpai}\OtherTok{=}\FunctionTok{st\_read}\NormalTok{(}\StringTok{"./Geodata/2024/Soils/OrganicSoils\_SILAVA/stripam.gpkg"}\NormalTok{,}
                       \AttributeTok{layer=}\StringTok{"stripam"}\NormalTok{)}
\NormalTok{silavas\_telpai}\OtherTok{=}\FunctionTok{st\_transform}\NormalTok{(silavas\_telpai,}\AttributeTok{crs=}\DecValTok{3059}\NormalTok{)}
\NormalTok{silavas\_telpai}\SpecialCharTok{$}\NormalTok{yes}\OtherTok{=}\DecValTok{1}
\NormalTok{SilavasTelpa\_10}\OtherTok{=}\FunctionTok{rasterize}\NormalTok{(silavas\_telpai,template10,}\AttributeTok{field=}\StringTok{"yes"}\NormalTok{)}

\CommentTok{\# presence{-}only layer without stripes}
\NormalTok{silava\_BezStripam1}\OtherTok{=}\FunctionTok{ifel}\NormalTok{(organika\_silavaLV}\SpecialCharTok{==}\DecValTok{1}\SpecialCharTok{\&}\NormalTok{SilavasTelpa\_10}\SpecialCharTok{==}\DecValTok{1}\NormalTok{,}\DecValTok{1}\NormalTok{,}\ConstantTok{NA}\NormalTok{)}
\NormalTok{silava\_BezStripam}\OtherTok{=}\FunctionTok{mask}\NormalTok{(silava\_BezStripam1,template10)}
\FunctionTok{plot}\NormalTok{(silava\_BezStripam)}
\FunctionTok{writeRaster}\NormalTok{(silava\_BezStripam,}
            \StringTok{"./RasterGrids\_10m/2024/SoilTXT\_OrganicSilava.tif"}\NormalTok{,}
            \AttributeTok{overwrite=}\ConstantTok{TRUE}\NormalTok{)}
\end{Highlighting}
\end{Shaded}

\subsection{Organic soils: LU}\label{Ch04.07.06}

The distribution of organic soils in farmlands was modelled by the University of
Latvia project ``Improvement of sustainable soil resource management in agriculture: E2SOILAGRI''.

From all the results we used layer \texttt{YN\_prognozes\_smooth.tif} stored
at \texttt{Geodata/2024/Soils/OrganicSoils\_LU/}.

Preprocessing consisted of projecting the layer to match the 10 m template. Both presences
and absences were saved for further processing.

\begin{Shaded}
\begin{Highlighting}[]
\CommentTok{\# libs {-}{-}{-}{-}}
\ControlFlowTok{if}\NormalTok{(}\SpecialCharTok{!}\FunctionTok{require}\NormalTok{(terra)) \{}\FunctionTok{install.packages}\NormalTok{(}\StringTok{"terra"}\NormalTok{); }\FunctionTok{require}\NormalTok{(terra)\}}
\ControlFlowTok{if}\NormalTok{(}\SpecialCharTok{!}\FunctionTok{require}\NormalTok{(sf)) \{}\FunctionTok{install.packages}\NormalTok{(}\StringTok{"sf"}\NormalTok{); }\FunctionTok{require}\NormalTok{(sf)\}}
\ControlFlowTok{if}\NormalTok{(}\SpecialCharTok{!}\FunctionTok{require}\NormalTok{(tidyverse)) \{}\FunctionTok{install.packages}\NormalTok{(}\StringTok{"tidyverse"}\NormalTok{); }\FunctionTok{require}\NormalTok{(tidyverse)\}}

\CommentTok{\# Template {-}{-}{-}{-}}
\NormalTok{template10}\OtherTok{=}\FunctionTok{rast}\NormalTok{(}\StringTok{"./Templates/TemplateRasters/LV10m\_10km.tif"}\NormalTok{)}

\CommentTok{\# Organic Soils LU {-}{-}{-}{-}}


\NormalTok{kudra\_norvegi}\OtherTok{=}\FunctionTok{rast}\NormalTok{(}\StringTok{"./Geodata/2024/Soils/OrganicSoils\_LU/YN\_prognozes\_smooth.tif"}\NormalTok{)}
\NormalTok{kudra\_norvLV}\OtherTok{=}\FunctionTok{project}\NormalTok{(kudra\_norvegi,template10)}
\FunctionTok{plot}\NormalTok{(kudra\_norvLV)}

\FunctionTok{writeRaster}\NormalTok{(kudra\_norvLV,}
            \StringTok{"./RasterGrids\_10m/2024/SoilTXT\_OrganicLU.tif"}\NormalTok{,}
            \AttributeTok{overwrite=}\ConstantTok{TRUE}\NormalTok{)}
\end{Highlighting}
\end{Shaded}

\section{Dynamic World data}\label{Ch04.08}

Dynamic World (DW) is a relatively new Earth observation system product that
classifies land cover and land use (LULC) into nine categories (0=water, 1=trees,
2=grass, 3=flooded\_vegetation, 4=crops, 5=shrub\_and\_scrub, 6=built, 7=bare,
8=snow\_and\_ice), for each ESA Copernicus Sentinel-2 image with identified
cloudiness ≤35, allowing for filtering and various aggregations (\citeproc{ref-DynWorld}{Brown et al., 2022}).

DW input information - raster layer for each season in each year - was prepared on
the Google Earth Engine (GEE) platform (\citeproc{ref-GEEpaper}{Gorelick et al., 2017}) using
a \href{https://code.earthengine.google.com/0f9fd61ee41af11d218ce8692abebe9b}{replication script}.
To use this script, you need a \href{https://console.cloud.google.com/earth-engine/welcome}{GEE account and project}
and sufficient space on Google Drive. When executing the command lines, a download
will be offered for a file covering the time period from the value in row 7 to
the value in row 8 (the file name should be specified in row 32, its
description in row 33 and the directory on Google Drive in row 31, or
all of this can be specified by confirming the save). This script is not optimized
for preparing all seasonal periods for all years, so in order to reproduce or
expand this study, it is necessary to change it manually.

Downloaded files are to be stored at \texttt{Geodata/2024/DynamicWorld/RAW/}.

During download, it can be seen that each layer covering all of Latvia is
divided into several sheets. This is because, in order to ensure a true zero
class (class ``water'' rather than background), the layers are encoded as \texttt{Float}
rather than integers. All of these tiles need to be downloaded, and the following
R command lines combine them, ensuring that the coordinate system and pixels
correspond to the reference raster.

\begin{Shaded}
\begin{Highlighting}[]
\CommentTok{\# libs {-}{-}{-}{-}}
\ControlFlowTok{if}\NormalTok{(}\SpecialCharTok{!}\FunctionTok{require}\NormalTok{(terra)) \{}\FunctionTok{install.packages}\NormalTok{(}\StringTok{"terra"}\NormalTok{); }\FunctionTok{require}\NormalTok{(terra)\}}
\ControlFlowTok{if}\NormalTok{(}\SpecialCharTok{!}\FunctionTok{require}\NormalTok{(tidyverse)) \{}\FunctionTok{install.packages}\NormalTok{(}\StringTok{"tidyverse"}\NormalTok{); }\FunctionTok{require}\NormalTok{(tidyverse)\}}

\CommentTok{\# 10 m template {-}{-}{-}{-}}
\NormalTok{template10}\OtherTok{=}\FunctionTok{rast}\NormalTok{(}\StringTok{"./Templates/TemplateRasters/LV10m\_10km.tif"}\NormalTok{)}

\CommentTok{\# DW export no GEE {-}{-}{-}{-}}
\NormalTok{faili}\OtherTok{=}\FunctionTok{data.frame}\NormalTok{(}\AttributeTok{faili=}\FunctionTok{list.files}\NormalTok{(}\StringTok{"./Geodata/2024/DynamicWorld/RAW/"}\NormalTok{))}
\NormalTok{faili}\SpecialCharTok{$}\NormalTok{celi\_sakums}\OtherTok{=}\FunctionTok{paste0}\NormalTok{(}\StringTok{"./Geodata/2024/DynamicWorld/RAW/"}\NormalTok{,faili}\SpecialCharTok{$}\NormalTok{faili)}

\CommentTok{\# prepping {-}{-}{-}{-}}
\NormalTok{faili}\OtherTok{=}\NormalTok{faili }\SpecialCharTok{\%\textgreater{}\%} 
  \FunctionTok{separate}\NormalTok{(faili,}\AttributeTok{into=}\FunctionTok{c}\NormalTok{(}\StringTok{"DW"}\NormalTok{,}\StringTok{"gads"}\NormalTok{,}\StringTok{"periods"}\NormalTok{,}\StringTok{"parejais"}\NormalTok{),}\AttributeTok{sep=}\StringTok{"\_"}\NormalTok{,}\AttributeTok{remove =} \ConstantTok{FALSE}\NormalTok{) }\SpecialCharTok{\%\textgreater{}\%} 
  \FunctionTok{mutate}\NormalTok{(}\AttributeTok{unikalais=}\FunctionTok{paste0}\NormalTok{(DW,}\StringTok{"\_"}\NormalTok{,gads,}\StringTok{"\_"}\NormalTok{,periods),}
         \AttributeTok{mosaic\_name=}\FunctionTok{paste0}\NormalTok{(unikalais,}\StringTok{".tif"}\NormalTok{),}
         \AttributeTok{masaic\_cels=}\FunctionTok{paste0}\NormalTok{(}\StringTok{"./Geodata/2024/DynamicWorld/"}\NormalTok{,mosaic\_name))}

\CommentTok{\# every layer consists of two tiles}
\NormalTok{unikalie}\OtherTok{=}\FunctionTok{levels}\NormalTok{(}\FunctionTok{factor}\NormalTok{(faili}\SpecialCharTok{$}\NormalTok{unikalais))}
\FunctionTok{min}\NormalTok{(}\FunctionTok{table}\NormalTok{(faili}\SpecialCharTok{$}\NormalTok{unikalais))}
\FunctionTok{max}\NormalTok{(}\FunctionTok{table}\NormalTok{(faili}\SpecialCharTok{$}\NormalTok{unikalais))}

\CommentTok{\# job}
\ControlFlowTok{for}\NormalTok{(i }\ControlFlowTok{in} \FunctionTok{seq\_along}\NormalTok{(unikalie))\{}
\NormalTok{  unikalais}\OtherTok{=}\NormalTok{faili }\SpecialCharTok{\%\textgreater{}\%} \FunctionTok{filter}\NormalTok{(unikalais}\SpecialCharTok{==}\NormalTok{unikalie[i])}
\NormalTok{  beigu\_cels}\OtherTok{=}\FunctionTok{unique}\NormalTok{(unikalais}\SpecialCharTok{$}\NormalTok{masaic\_cels)}
  
  \FunctionTok{print}\NormalTok{(i)}
  
\NormalTok{  viens}\OtherTok{=}\FunctionTok{rast}\NormalTok{(unikalais}\SpecialCharTok{$}\NormalTok{celi\_sakums[}\DecValTok{1}\NormalTok{])}
\NormalTok{  divi}\OtherTok{=}\FunctionTok{rast}\NormalTok{(unikalais}\SpecialCharTok{$}\NormalTok{celi\_sakums[}\DecValTok{2}\NormalTok{])}
  
\NormalTok{  viens2}\OtherTok{=}\FunctionTok{project}\NormalTok{(viens,template10)}
\NormalTok{  divi2}\OtherTok{=}\FunctionTok{project}\NormalTok{(divi,template10)}
  
\NormalTok{  mozaika}\OtherTok{=}\FunctionTok{mosaic}\NormalTok{(viens2,divi2,}\AttributeTok{fun=}\StringTok{"first"}\NormalTok{)}
\NormalTok{  maskets}\OtherTok{=}\FunctionTok{mask}\NormalTok{(mozaika,template10,}
               \AttributeTok{filename=}\NormalTok{beigu\_cels,}
               \AttributeTok{overwrite=}\ConstantTok{TRUE}\NormalTok{)}
  
  \FunctionTok{print}\NormalTok{(beigu\_cels)}
\NormalTok{\}}
\end{Highlighting}
\end{Shaded}

\section{The Global Forest Watch}\label{Ch04.09}

The Global Forest Watch (GFW) is a widely known product that describes tree
canopy cover in 2000, its annual growth from 2001 to 2012, and its annual
loss from 2001 to the current version, which is updated annually (\citeproc{ref-theGFW}{Hansen et al., 2013}). The
data is available both on the \href{https://data.globalforestwatch.org/documents/941f17325a494ed78c4817f9bb20f33a/explore}{project website}
and on \href{https://developers.google.com/earth-engine/datasets/catalog/UMD_hansen_global_forest_change_2024_v1_12}{GEE}, where
it was developed. This project used v1.12, in which the last year of tree loss
dating was 2024, preparing it for download on the GEE platform with
this \href{https://code.earthengine.google.com/4a12b7504ceafe7f422dd7efbe804b67}{replication script}.
To use this script, you need a \href{https://console.cloud.google.com/earth-engine/welcome}{GEE account and project}
and sufficient space on Google Drive. When executing the command lines, you will
be offered to download the file, which you need to save to Google Drive.

After executing the command lines and preparing the results in Google Drive,
four files become available for download. The location to download them is
\texttt{Geodata/2024/Trees/GFW/RAW/}. After download, these files need to be projected
to match the reference raster.

\begin{Shaded}
\begin{Highlighting}[]
\CommentTok{\# libs {-}{-}{-}{-}}
\ControlFlowTok{if}\NormalTok{(}\SpecialCharTok{!}\FunctionTok{require}\NormalTok{(terra)) \{}\FunctionTok{install.packages}\NormalTok{(}\StringTok{"terra"}\NormalTok{); }\FunctionTok{require}\NormalTok{(terra)\}}

\CommentTok{\# 10 m rastra template {-}{-}{-}{-}}
\NormalTok{template10}\OtherTok{=}\FunctionTok{rast}\NormalTok{(}\StringTok{"./Templates/TemplateRasters/LV10m\_10km.tif"}\NormalTok{)}

\CommentTok{\# TreeCoverLoss {-}{-}{-}{-}}
\NormalTok{treecoverloss}\OtherTok{=}\FunctionTok{rast}\NormalTok{(}\StringTok{"./Geodata/2024/Trees/GFW/RAW/TreeCoverLoss\_v1\_12.tif"}\NormalTok{)}

\NormalTok{tcl}\OtherTok{=}\FunctionTok{ifel}\NormalTok{(treecoverloss}\SpecialCharTok{\textless{}}\DecValTok{1}\NormalTok{,}\ConstantTok{NA}\NormalTok{,treecoverloss)}

\NormalTok{tcl2}\OtherTok{=}\NormalTok{terra}\SpecialCharTok{::}\FunctionTok{project}\NormalTok{(tcl,paraugs)}
\NormalTok{tcl3}\OtherTok{=}\FunctionTok{mask}\NormalTok{(tcl2,paraugs,}\AttributeTok{filename=}\StringTok{"./Geodata/2024/Trees/GFW/TreeCoverLoss\_v1\_12.tif"}\NormalTok{,}\AttributeTok{overwrite=}\ConstantTok{TRUE}\NormalTok{)}
\end{Highlighting}
\end{Shaded}

\section{Palsar}\label{Ch04.10}

The Palsar Forests resource is based on PALSAR-2 synthetic aperture radar (SAR)
reflectance classification of forest and non-forest land with a pixel
resolution of 25 m. Forests are classified as areas of at least 0.5 ha covered
with trees, where tree cover (at least 5 m high) is at least 10\% (\citeproc{ref-PALSARForest}{Shimada et al., 2013}).
The data is available at \href{https://developers.google.com/earth-engine/datasets/catalog/JAXA_ALOS_PALSAR_YEARLY_FNF4}{GEE}.
This project used a 4-class version (1=Dense Forest, 2=Non-dense Forest,
3=Non-Forest, 4=Water), in which the last tree cover dating year was 2020,
prepared for download on the GEE platform with this
\href{https://code.earthengine.google.com/3ec78ab057e6c8910cb1546002132b34}{replication script}.
To use this script, you need a \href{https://console.cloud.google.com/earth-engine/welcome}{GEE account and project}
and sufficient space on Google Drive. When executing the command lines, you will
be offered to download the file, which you need to save to Google Drive.

After executing the command lines and preparing the results in Google Drive,
four files become available for download. The location to download them is
\texttt{Geodata/2024/Trees/Palsar/RAW/}. After download, these files need to be projected
to match the reference raster and merged. In this resource, trees are coded into
two groups: 1=Dense Forest and 2=Non-dense Forest, which need to be merged and
the rest converted to missing values (see code below).

Although this resource reflects conditions in 2020 rather
than 2024, we used it because \hyperref[Ch04.09]{The Global Forest Watch data} provides reliable data on canopy loss, but the appearance
of canopy cover is not so rapid that there would be significant changes over a
four-year period.

\begin{Shaded}
\begin{Highlighting}[]
\CommentTok{\# libs {-}{-}{-}{-}}
\ControlFlowTok{if}\NormalTok{(}\SpecialCharTok{!}\FunctionTok{require}\NormalTok{(terra)) \{}\FunctionTok{install.packages}\NormalTok{(}\StringTok{"terra"}\NormalTok{); }\FunctionTok{require}\NormalTok{(terra)\}}

\CommentTok{\# 10 m rastra template {-}{-}{-}{-}}
\NormalTok{template10}\OtherTok{=}\FunctionTok{rast}\NormalTok{(}\StringTok{"./Templates/TemplateRasters/LV10m\_10km.tif"}\NormalTok{)}


\CommentTok{\# PALSAR Forests {-}{-}{-}{-}}

\NormalTok{fnf1}\OtherTok{=}\FunctionTok{rast}\NormalTok{(}\StringTok{"./Geodata/2024/Trees/Palsar/RAW/ForestNonForest{-}0000023296{-}0000023296.tif"}\NormalTok{)}
\NormalTok{fnf2}\OtherTok{=}\FunctionTok{rast}\NormalTok{(}\StringTok{"./Geodata/2024/Trees/Palsar/RAW/ForestNonForest{-}0000023296{-}0000000000.tif"}\NormalTok{)}
\NormalTok{fnf3}\OtherTok{=}\FunctionTok{rast}\NormalTok{(}\StringTok{"./Geodata/2024/Trees/Palsar/RAW/ForestNonForest{-}0000000000{-}0000023296.tif"}\NormalTok{)}
\NormalTok{fnf4}\OtherTok{=}\FunctionTok{rast}\NormalTok{(}\StringTok{"./Geodata/2024/Trees/Palsar/RAW/ForestNonForest{-}0000000000{-}0000000000.tif"}\NormalTok{)}

\NormalTok{fnf1p}\OtherTok{=}\NormalTok{terra}\SpecialCharTok{::}\FunctionTok{project}\NormalTok{(fnf1,template10)}
\NormalTok{fnf2p}\OtherTok{=}\NormalTok{terra}\SpecialCharTok{::}\FunctionTok{project}\NormalTok{(fnf2,template10)}
\NormalTok{fnf3p}\OtherTok{=}\NormalTok{terra}\SpecialCharTok{::}\FunctionTok{project}\NormalTok{(fnf3,template10)}
\NormalTok{fnf4p}\OtherTok{=}\NormalTok{terra}\SpecialCharTok{::}\FunctionTok{project}\NormalTok{(fnf4,template10)}

\NormalTok{fnfA}\OtherTok{=}\NormalTok{terra}\SpecialCharTok{::}\FunctionTok{merge}\NormalTok{(fnf1p,fnf2p)}
\NormalTok{fnfB}\OtherTok{=}\NormalTok{terra}\SpecialCharTok{::}\FunctionTok{merge}\NormalTok{(fnfA,fnf3p)}
\NormalTok{fnfC}\OtherTok{=}\NormalTok{terra}\SpecialCharTok{::}\FunctionTok{merge}\NormalTok{(fnfB,fnf4p)}
\FunctionTok{plot}\NormalTok{(fnfC)}

\NormalTok{fnf\_X}\OtherTok{=}\FunctionTok{ifel}\NormalTok{(fnfC}\SpecialCharTok{\textless{}=}\DecValTok{2}\SpecialCharTok{\&}\NormalTok{fnfC}\SpecialCharTok{\textgreater{}=}\DecValTok{1}\NormalTok{,}\DecValTok{1}\NormalTok{,}\ConstantTok{NA}\NormalTok{)}
\FunctionTok{plot}\NormalTok{(fnf\_X)}

\NormalTok{fnf\_XX}\OtherTok{=}\FunctionTok{mask}\NormalTok{(fnf\_X,template10,}
            \AttributeTok{filename=}\StringTok{"./Geodata/2024/Trees/Palsar/Palsar\_Forests.tif"}\NormalTok{,}
            \AttributeTok{overwrite=}\ConstantTok{TRUE}\NormalTok{)}
\end{Highlighting}
\end{Shaded}

\section{CHELSA v2.1}\label{Ch04.11}

Climatologies at high resolution for the Earth's land surface areas (CHELSA) is a
30 arc second global downscaled climate data set (\citeproc{ref-CHELSA}{Karger et al., 2017}). The temperature algorithm
is based on statistical downscaling of atmospheric temperatures. The precipitation
algorithm incorporates orographic predictors including wind fields, valley
exposition, and boundary layer height, with a subsequent bias correction. CHELSA
climatological data has a similar accuracy as other products for temperature, but its predictions of precipitation patterns are better (\citeproc{ref-CHELSA}{Karger et al., 2017}). Data
(1980-2010 baseline) are freely available for download
from the \href{https://chelsa-climate.org/}{homepage} forwarding
to download server, with download links for selected products available. There is also technical
specification available. In this project we used version 2.1.

The download links we used together with the renaming scheme are \href{https://github.com/aavotins/HiQBioDiv_EGVs/blob/main/Data/Geodata/2024/CHELSA/CHELSAdownload_rename.csv}{included}
with this document. The following command lines perform download, crop to the
extent of Latvia (using 1 km vector grid) and save the files for further processing
described with other \hyperref[Ch06]{EGVs}.

\begin{Shaded}
\begin{Highlighting}[]
\CommentTok{\# libs {-}{-}{-}{-}}
\ControlFlowTok{if}\NormalTok{(}\SpecialCharTok{!}\FunctionTok{require}\NormalTok{(terra)) \{}\FunctionTok{install.packages}\NormalTok{(}\StringTok{"terra"}\NormalTok{); }\FunctionTok{require}\NormalTok{(terra)\}}
\ControlFlowTok{if}\NormalTok{(}\SpecialCharTok{!}\FunctionTok{require}\NormalTok{(sf)) \{}\FunctionTok{install.packages}\NormalTok{(}\StringTok{"sf"}\NormalTok{); }\FunctionTok{require}\NormalTok{(sf)\}}
\ControlFlowTok{if}\NormalTok{(}\SpecialCharTok{!}\FunctionTok{require}\NormalTok{(sfarrow)) \{}\FunctionTok{install.packages}\NormalTok{(}\StringTok{"sfarrow"}\NormalTok{); }\FunctionTok{require}\NormalTok{(sfarrow)\}}
\ControlFlowTok{if}\NormalTok{(}\SpecialCharTok{!}\FunctionTok{require}\NormalTok{(tidyverse)) \{}\FunctionTok{install.packages}\NormalTok{(}\StringTok{"tidyverse"}\NormalTok{); }\FunctionTok{require}\NormalTok{(tidyverse)\}}
\ControlFlowTok{if}\NormalTok{(}\SpecialCharTok{!}\FunctionTok{require}\NormalTok{(curl)) \{}\FunctionTok{install.packages}\NormalTok{(}\StringTok{"curl"}\NormalTok{); }\FunctionTok{require}\NormalTok{(curl)\}}

\CommentTok{\# templates {-}{-}{-}{-}}
\CommentTok{\# 1km grid}
\NormalTok{tikls1km}\OtherTok{=}\NormalTok{sfarrow}\SpecialCharTok{::}\FunctionTok{st\_read\_parquet}\NormalTok{(}\StringTok{"./Templates/TemplateGrids/tikls1km\_sauzeme.parquet"}\NormalTok{)}
\NormalTok{telpai}\OtherTok{=}\NormalTok{tikls1km }\SpecialCharTok{\%\textgreater{}\%} 
  \FunctionTok{mutate}\NormalTok{(}\AttributeTok{yes=}\DecValTok{1}\NormalTok{) }\SpecialCharTok{\%\textgreater{}\%} 
  \FunctionTok{summarise}\NormalTok{(}\AttributeTok{yes=}\FunctionTok{max}\NormalTok{(yes)) }\SpecialCharTok{\%\textgreater{}\%} 
  \FunctionTok{st\_buffer}\NormalTok{(.,}\AttributeTok{dist=}\DecValTok{10000}\NormalTok{)}

\CommentTok{\# download and crop {-}{-}{-}{-}}

\NormalTok{links\_names}\OtherTok{=}\FunctionTok{read\_csv}\NormalTok{(}\StringTok{"./Geodata/2024/CHELSA/CHELSAdownload\_rename.csv"}\NormalTok{)}
\NormalTok{links\_names}\OtherTok{=}\NormalTok{links\_names }\SpecialCharTok{\%\textgreater{}\%} 
  \FunctionTok{filter}\NormalTok{(todownload}\SpecialCharTok{==}\DecValTok{1}\NormalTok{)}

\ControlFlowTok{for}\NormalTok{(i }\ControlFlowTok{in} \FunctionTok{seq\_along}\NormalTok{(links\_names}\SpecialCharTok{$}\NormalTok{localname))\{}
  \FunctionTok{print}\NormalTok{(i)}
\NormalTok{  sakums}\OtherTok{=}\FunctionTok{Sys.time}\NormalTok{()}
\NormalTok{  links}\OtherTok{=}\NormalTok{links\_names}\SpecialCharTok{$}\NormalTok{weblocation[i]}
\NormalTok{  saving1}\OtherTok{=}\StringTok{"./Geodata/2024/CHELSA/draza.tif"}
\NormalTok{  saving2}\OtherTok{=}\FunctionTok{paste0}\NormalTok{(}\StringTok{"./Geodata/2024/CHELSA/"}\NormalTok{,links\_names}\SpecialCharTok{$}\NormalTok{localname[i])}
  
  \FunctionTok{curl\_download}\NormalTok{(}\AttributeTok{url=}\NormalTok{links,}\AttributeTok{destfile =}\NormalTok{ saving1,}\AttributeTok{quiet =} \ConstantTok{FALSE}\NormalTok{)}
\NormalTok{  fails}\OtherTok{=}\FunctionTok{rast}\NormalTok{(saving1)}
\NormalTok{  telpa}\OtherTok{=}\FunctionTok{st\_transform}\NormalTok{(telpai,}\AttributeTok{crs=}\FunctionTok{st\_crs}\NormalTok{(fails))}
\NormalTok{  nogriezts}\OtherTok{=}\FunctionTok{crop}\NormalTok{(fails,telpa,}
                 \AttributeTok{filename=}\NormalTok{saving2,}
                 \AttributeTok{overwrite=}\ConstantTok{TRUE}\NormalTok{)}
  \FunctionTok{unlink}\NormalTok{(saving1)}
\NormalTok{  beigas}\OtherTok{=}\FunctionTok{Sys.time}\NormalTok{()}
\NormalTok{  ilgums}\OtherTok{=}\NormalTok{beigas}\SpecialCharTok{{-}}\NormalTok{sakums}
  \FunctionTok{print}\NormalTok{(ilgums)}
\NormalTok{\}}
\end{Highlighting}
\end{Shaded}

\section{HydroClim data}\label{Ch04.12}

HydroClim is a near-global freshwater-specific environmental variable dataset,
created for biodiversity analysis at 1 km resolution (\citeproc{ref-HydroClim}{Domisch et al., 2015}). Dataset
contains many different variables along the HydroSHEDS river
network (\citeproc{ref-HydroSheds}{Lehner et al., 2008}), including upstream climate recalculated from
worldclim (\citeproc{ref-worldclim_hijmans}{Hijmans et al., 2005}). We downloaded (to \texttt{Geodata/2024/HydroClim/})
averaged upstream climate from \href{https://zenodo.org/records/5089529}{Zenodo repository}
(available also from \href{https://datadryad.org/dataset/doi:10.5061/dryad.dv920}{Dryad})
and cropped to the extent of Latvia and renamed files for further processing
with the code below. Renaming scheme is \href{https://github.com/aavotins/HiQBioDiv_EGVs/blob/main/Data/Geodata/2024/HydroClim/HydroClim_renaming.csv}{published with document}.

\begin{Shaded}
\begin{Highlighting}[]
\CommentTok{\# libs {-}{-}{-}{-}}
\ControlFlowTok{if}\NormalTok{(}\SpecialCharTok{!}\FunctionTok{require}\NormalTok{(terra)) \{}\FunctionTok{install.packages}\NormalTok{(}\StringTok{"terra"}\NormalTok{); }\FunctionTok{require}\NormalTok{(terra)\}}
\ControlFlowTok{if}\NormalTok{(}\SpecialCharTok{!}\FunctionTok{require}\NormalTok{(sf)) \{}\FunctionTok{install.packages}\NormalTok{(}\StringTok{"sf"}\NormalTok{); }\FunctionTok{require}\NormalTok{(sf)\}}
\ControlFlowTok{if}\NormalTok{(}\SpecialCharTok{!}\FunctionTok{require}\NormalTok{(sfarrow)) \{}\FunctionTok{install.packages}\NormalTok{(}\StringTok{"sfarrow"}\NormalTok{); }\FunctionTok{require}\NormalTok{(sfarrow)\}}
\ControlFlowTok{if}\NormalTok{(}\SpecialCharTok{!}\FunctionTok{require}\NormalTok{(tidyverse)) \{}\FunctionTok{install.packages}\NormalTok{(}\StringTok{"tidyverse"}\NormalTok{); }\FunctionTok{require}\NormalTok{(tidyverse)\}}

\CommentTok{\# templates {-}{-}{-}{-}}
\NormalTok{template100}\OtherTok{=}\FunctionTok{rast}\NormalTok{(}\StringTok{"./Templates/TemplateRasters/LV100m\_10km.tif"}\NormalTok{)}
\NormalTok{tikls1km}\OtherTok{=}\NormalTok{sfarrow}\SpecialCharTok{::}\FunctionTok{st\_read\_parquet}\NormalTok{(}\StringTok{"./Templates/TemplateGrids/tikls1km\_sauzeme.parquet"}\NormalTok{)}

\CommentTok{\# reading HydroClim {-}{-}{-}{-}}
\NormalTok{videjie}\OtherTok{=}\NormalTok{terra}\SpecialCharTok{::}\FunctionTok{rast}\NormalTok{(}\StringTok{"./Geodata/2024/HydroClim/hydroclim\_average+sum.nc"}\NormalTok{)}

\CommentTok{\# reading dictionary {-}{-}{-}{-}{-}}
\NormalTok{slanu\_nosaukumi}\OtherTok{=}\FunctionTok{read\_csv}\NormalTok{(}\StringTok{"./Geodata/2024/HydroClim/HydroClim\_renaming.csv"}\NormalTok{)}

\CommentTok{\# cropping {-}{-}{-}}
\NormalTok{tikls1km\_reproj}\OtherTok{=}\FunctionTok{st\_transform}\NormalTok{(tikls1km,}\AttributeTok{crs=}\FunctionTok{st\_crs}\NormalTok{(videjie))}
\NormalTok{telpai}\OtherTok{=}\NormalTok{tikls1km }\SpecialCharTok{\%\textgreater{}\%} 
  \FunctionTok{mutate}\NormalTok{(}\AttributeTok{yes=}\DecValTok{1}\NormalTok{) }\SpecialCharTok{\%\textgreater{}\%} 
  \FunctionTok{summarise}\NormalTok{(}\AttributeTok{yes=}\FunctionTok{max}\NormalTok{(yes)) }\SpecialCharTok{\%\textgreater{}\%} 
  \FunctionTok{st\_buffer}\NormalTok{(.,}\AttributeTok{dist=}\DecValTok{10000}\NormalTok{) }\SpecialCharTok{\%\textgreater{}\%} 
  \FunctionTok{st\_transform}\NormalTok{(.,}\AttributeTok{crs=}\FunctionTok{st\_crs}\NormalTok{(videjie))}
\NormalTok{videjie}\OtherTok{=}\NormalTok{terra}\SpecialCharTok{::}\FunctionTok{crop}\NormalTok{(videjie,telpai)}

\CommentTok{\# layer names {-}{-}{-}{-}}
\FunctionTok{names}\NormalTok{(videjie)}\OtherTok{=}\NormalTok{slanu\_nosaukumi}\SpecialCharTok{$}\NormalTok{local\_name}

\CommentTok{\# saving files {-}{-}{-}{-}}
\ControlFlowTok{for}\NormalTok{(i }\ControlFlowTok{in} \FunctionTok{seq\_along}\NormalTok{(slanu\_nosaukumi}\SpecialCharTok{$}\NormalTok{local\_name))\{}
\NormalTok{  nosaukumam}\OtherTok{=}\NormalTok{slanu\_nosaukumi}\SpecialCharTok{$}\NormalTok{local\_name[i]}
  \FunctionTok{writeRaster}\NormalTok{(videjie[[i]],}
              \FunctionTok{paste0}\NormalTok{(}\StringTok{"./Geodata/2024/HydroClim/"}\NormalTok{,nosaukumam),}
              \AttributeTok{overwrite=}\ConstantTok{TRUE}\NormalTok{)}
\NormalTok{\}}
\end{Highlighting}
\end{Shaded}

The raster dataset contains values only where large enough rivers are detected
in HydroSHEDS. However, for species distribution modelling in this project
we need continuously covered raster surfaces. For necessary geoprocessing to create such surfaces, we downloaded also HydroBASINS (\citeproc{ref-HydroBasins}{Lehner and Grill, 2013}) \href{https://www.hydrosheds.org/products/hydrobasins}{dataset} to \texttt{Geodata/2024/HydroClim/}.
These procedures were EGV-specific and are described with other \hyperref[Ch06]{EGVs}.

\section{Sentinel-2 indices}\label{Ch04.13}

The European Space Agency (ESA) Copernicus program's Sentinel-2 mission is a
constellation of two (three since 09/05/2024) identical satellites orbiting in
the same orbit. The first satellite, Sentinel-2A, entered its orbit and
underwent calibration tests on 2015-06-23, the second (Sentinel-2B) on 2017-03-07,
with the first images available earlier. Each satellite captures high-resolution
images (from 10 m (at the equator) pixel resolution) in 13 spectral channels
with a return time of up to 5 days (more frequently closer to the poles) (\url{https://www.esa.int/Applications/Observing_the_Earth/Copernicus/Sentinel-2}). The
data from this mission is freely available, including on the Google Earth Engine
platform (\citeproc{ref-GEEpaper}{Gorelick et al., 2017}) for various large-scale pre-processing and analysis. We used
the harmonized Level-2A (\url{https://developers.google.com/earth-engine/datasets/catalog/COPERNICUS_S2_SR_HARMONIZED\#description}) product, applying a cloud mask that includes not only cloud filtering but also
shadow filtering. For each filtered image (cloud-free, April-October, 2020-2024), we computed the normalized difference vegetation
index (NDVI), the normalized difference moisture index (NDMI), and the
normalized difference water index (NDWI) as well as various metrics.
A \href{https://code.earthengine.google.com/78024b3354cccb526159fd865b214771}{replication script}
can be used to prepare the data. To use this script,
you need a \href{https://console.cloud.google.com/earth-engine/welcome}{GEE account and project}
and sufficient space on Google Drive. When executing the command lines, the
following files will be offered for download:

\begin{itemize}
\item
  \texttt{NDVI\_median-ST-{[}runtag,\ 20250820\ by\ default{]}} - NDVI short-term median (2020-2024) of annual medians (April to October)
\item
  \texttt{NDVI\_p25-ST-{[}runtag,\ 20250820\ by\ default{]}} - NDVI short-term median (2020-2024) of annual 25th percentiles (April to October)
\item
  \texttt{NDVI\_p75-ST-{[}runtag,\ 20250820\ by\ default{]}}- NDVI short-term median (2020-2024) of annual 75th percentiles (April to October)
\item
  \texttt{NDVI\_iqr-ST-{[}runtag,\ 20250820\ by\ default{]}} - NDVI short-term median (2020-2024) of inter-quartile ranges (April to October)
\item
  \texttt{NDVI\_median-LY-{[}runtag,\ 20250820\ by\ default{]}} - NDVI last-years (2024) median (April to October)
\item
  \texttt{NDMI\_median-ST-{[}runtag,\ 20250820\ by\ default{]}} - NDMI short-term median (2020-2024) of annual medians (April to October)
\item
  \texttt{NDMI\_p25-ST-{[}runtag,\ 20250820\ by\ default{]}} - NDMI short-term median (2020-2024) of annual 25th percentiles (April to October)
\item
  \texttt{NDMI\_p75-ST-{[}runtag,\ 20250820\ by\ default{]}} - NDMI short-term median (2020-2024) of annual 75th percentiles (April to October)
\item
  \texttt{NDMI\_iqr-ST-{[}runtag,\ 20250820\ by\ default{]}} - NDMI short-term median (2020-2024) of inter-quartile ranges (April to October)
\item
  \texttt{NDMI\_median-LY-{[}runtag,\ 20250820\ by\ default{]}} - NDMI last-years (2024) median (April to October)
\item
  \texttt{NDWI\_median-ST-{[}runtag,\ 20250820\ by\ default{]}} - NDMI short-term median (2020-2024) of annual medians (April to October)
\item
  \texttt{NDWI\_p25-ST-{[}runtag,\ 20250820\ by\ default{]}} - NDWI short-term median (2020-2024) of annual 25th percentiles (April to October)
\item
  \texttt{NDWI\_p75-ST-{[}runtag,\ 20250820\ by\ default{]}} - NDWI short-term median (2020-2024) of annual 75th percentiles (April to October)
\item
  \texttt{NDWI\_iqr-ST-{[}runtag,\ 20250820\ by\ default{]}} - NDWI short-term median (2020-2024) of inter-quartile ranges (April to October)
\item
  \texttt{NDWI\_median-LY-{[}runtag,\ 20250820\ by\ default{]}} - NDWI last-years (2024) median (April to October)
\end{itemize}

After executing the command lines and preparing the results in Google Drive, it
can be seen that each layer covering all of Latvia is divided into
several tiles. This is because the layers are encoded as Float and exceed 4 GB
in size before GeoTIFF compression. All of these files need to be downloaded and
located at \texttt{Geodata/2024/S2indices/RAW}. The following R commands combine them,
ensuring the coordinate systems and its naming, and pixels matching to the reference
raster, while renaming files to \texttt{EO\_{[}index{]}-{[}term:\ ST\ or\ LY{]}{[}statistic{]}}.

\begin{Shaded}
\begin{Highlighting}[]
\CommentTok{\# libs {-}{-}{-}{-}}
\ControlFlowTok{if}\NormalTok{(}\SpecialCharTok{!}\FunctionTok{require}\NormalTok{(terra)) \{}\FunctionTok{install.packages}\NormalTok{(}\StringTok{"terra"}\NormalTok{); }\FunctionTok{require}\NormalTok{(terra)\}}
\ControlFlowTok{if}\NormalTok{(}\SpecialCharTok{!}\FunctionTok{require}\NormalTok{(tidyverse)) \{}\FunctionTok{install.packages}\NormalTok{(}\StringTok{"tidyverse"}\NormalTok{); }\FunctionTok{require}\NormalTok{(tidyverse)\}}


\CommentTok{\# 10 m raster template {-}{-}{-}{-}}
\NormalTok{template10}\OtherTok{=}\FunctionTok{rast}\NormalTok{(}\StringTok{"./Templates/TemplateRasters/LV10m\_10km.tif"}\NormalTok{)}

\CommentTok{\# Fails as exported from GEE {-}{-}{-}{-}}
\NormalTok{faili}\OtherTok{=}\FunctionTok{data.frame}\NormalTok{(}\AttributeTok{fails=}\FunctionTok{list.files}\NormalTok{(}\StringTok{"./Geodata/2024/S2indices/RAW/"}\NormalTok{,}\AttributeTok{pattern =} \StringTok{".tif"}\NormalTok{))}
\NormalTok{faili}\SpecialCharTok{$}\NormalTok{celi\_sakums}\OtherTok{=}\FunctionTok{paste0}\NormalTok{(}\StringTok{"./Geodata/2024/S2indices/RAW/"}\NormalTok{,faili}\SpecialCharTok{$}\NormalTok{fails)}


\CommentTok{\# file names {-}{-}{-}{-}}
\NormalTok{faili}\OtherTok{=}\NormalTok{faili }\SpecialCharTok{\%\textgreater{}\%} 
  \FunctionTok{separate}\NormalTok{(fails,}\AttributeTok{into=}\FunctionTok{c}\NormalTok{(}\StringTok{"nosaukums"}\NormalTok{,}\StringTok{"vidus"}\NormalTok{,}\StringTok{"beigas"}\NormalTok{),}\AttributeTok{sep=}\StringTok{"{-}"}\NormalTok{,}\AttributeTok{remove =} \ConstantTok{FALSE}\NormalTok{) }\SpecialCharTok{\%\textgreater{}\%} 
  \FunctionTok{mutate}\NormalTok{(}\AttributeTok{mosaic\_name=}\FunctionTok{paste0}\NormalTok{(}\StringTok{"EO\_"}\NormalTok{,nosaukums,}\StringTok{"{-}"}\NormalTok{,beigas,}\FunctionTok{tolower}\NormalTok{(vidus),}\StringTok{".tif"}\NormalTok{),}
         \AttributeTok{masaic\_cels=}\FunctionTok{paste0}\NormalTok{(}\StringTok{"./Geodata/2024/S2indices/Mosaics/"}\NormalTok{,mosaic\_name))}


\NormalTok{unikalie}\OtherTok{=}\FunctionTok{levels}\NormalTok{(}\FunctionTok{factor}\NormalTok{(faili}\SpecialCharTok{$}\NormalTok{mosaic\_name))}
\FunctionTok{min}\NormalTok{(}\FunctionTok{table}\NormalTok{(faili}\SpecialCharTok{$}\NormalTok{mosaic\_name))}
\FunctionTok{max}\NormalTok{(}\FunctionTok{table}\NormalTok{(faili}\SpecialCharTok{$}\NormalTok{mosaic\_name))}

\CommentTok{\# preparation of mosaics {-}{-}{-}{-}}
\ControlFlowTok{for}\NormalTok{(i }\ControlFlowTok{in} \FunctionTok{seq\_along}\NormalTok{(unikalie))\{}
\NormalTok{  sakums}\OtherTok{=}\FunctionTok{Sys.time}\NormalTok{()}
\NormalTok{  unikalais}\OtherTok{=}\NormalTok{faili }\SpecialCharTok{\%\textgreater{}\%} \FunctionTok{filter}\NormalTok{(mosaic\_name}\SpecialCharTok{==}\NormalTok{unikalie[i])}
\NormalTok{  beigu\_cels}\OtherTok{=}\FunctionTok{unique}\NormalTok{(unikalais}\SpecialCharTok{$}\NormalTok{masaic\_cels)}
  
  \FunctionTok{print}\NormalTok{(i)}
  
  \CommentTok{\# there are exactly 2 tiles per file}
\NormalTok{  viens}\OtherTok{=}\FunctionTok{rast}\NormalTok{(unikalais}\SpecialCharTok{$}\NormalTok{celi\_sakums[}\DecValTok{1}\NormalTok{])}
\NormalTok{  divi}\OtherTok{=}\FunctionTok{rast}\NormalTok{(unikalais}\SpecialCharTok{$}\NormalTok{celi\_sakums[}\DecValTok{2}\NormalTok{])}
  
\NormalTok{  viens2}\OtherTok{=}\NormalTok{terra}\SpecialCharTok{::}\FunctionTok{project}\NormalTok{(viens,template10)}
\NormalTok{  divi2}\OtherTok{=}\NormalTok{terra}\SpecialCharTok{::}\FunctionTok{project}\NormalTok{(divi,template10)}
  
\NormalTok{  mozaika}\OtherTok{=}\NormalTok{terra}\SpecialCharTok{::}\FunctionTok{merge}\NormalTok{(viens2,divi2)}
\NormalTok{  maskets}\OtherTok{=}\FunctionTok{mask}\NormalTok{(mozaika,template10,}
               \AttributeTok{filename=}\NormalTok{beigu\_cels,}\AttributeTok{overwrite=}\ConstantTok{TRUE}\NormalTok{,}
               \AttributeTok{gdal=}\FunctionTok{c}\NormalTok{(}\StringTok{"COMPRESS=LZW"}\NormalTok{,}\StringTok{"TILED=YES"}\NormalTok{,}\StringTok{"BIGTIFF=IF\_SAFER"}\NormalTok{),}
               \AttributeTok{datatype=}\StringTok{"FLT4S"}\NormalTok{,}
               \AttributeTok{NAflag=}\ConstantTok{NA}\NormalTok{)}
  
  \FunctionTok{plot}\NormalTok{(maskets,}\AttributeTok{main=}\NormalTok{unikalie[i])}
  \FunctionTok{print}\NormalTok{(beigu\_cels)}
\NormalTok{  beigas}\OtherTok{=}\FunctionTok{Sys.time}\NormalTok{()}
\NormalTok{  ilgums}\OtherTok{=}\NormalTok{beigas}\SpecialCharTok{{-}}\NormalTok{sakums}
  \FunctionTok{print}\NormalTok{(ilgums)}
\NormalTok{\}}
\end{Highlighting}
\end{Shaded}

\section{Waste and garbage disposal sites, landfills}\label{Ch04.14}

Information on landfills has been compiled from \href{https://www.varam.gov.lv/sites/varam/files/content/files/atkritumu_poligoni_lv_karte.pdf}{The Ministry of Smart Administration and Regional Development} and
Latvian Environment, Geology and Meteorology Centre's report,\\
\href{https://videscentrs.lvgmc.lv/files/Vide/Atkritumi_un_radiacijas_objekti/Nr_3_parskats_par_atkritumiem/3Atkritumi_kopsavilkums_2023.pdf}{``Report on landfills in Latvia in 2023''} listed landfills and their addresses. The coordinates required
for the preparation of EGVs were obtained by combining the
resources \url{https://www.google.com/maps} and \url{https://balticmaps.eu/}. In addition to
the resources mentioned above, an object was added at the address
``Dardedzes C, Mārupes pag., Mārupes nov., Latvia, LV-2166''.

In addition, information from the \href{https://skiroviegli.lv/\#/}{State Environmental Service on
separate waste and deposit packaging collection points}
was used, exporting it to an Excel file.

Both data sets were combined into a single file
and \href{https://github.com/aavotins/HiQBioDiv_EGVs/blob/main/Data/Geodata/2024/GarbageWasteLandfills/Atkritumi.xlsx}{added} to this material.

\section{Digital elevation/terrain models}\label{Ch04.15}

With the publication of continuous aerial laser scanning data for the territory of Latvia (\url{https://www.lgia.gov.lv/lv/digitalie-augstuma-modeli-0}), various
high-resolution (1 m and higher) digital surface models (DSM) and
digital elevation models (DEM) have been developed. Since the input data was the
same in all cases, the values of these (corresponding) models were identical
across almost the entire territory of the country. However, airborne laser
scanning data (1) was not available for the entire territory of the country,
and (2) there were differences between the models in terms of filling (availability
of values) outside inland waters and (3) filling of water bodies themselves.
However, for areas covered by data on land, the values were almost
identical. Pearson's correlation coefficients between the DEMs developed
by LU ĢZZF, LVMI Silava, and LĢIA were greater than 0.999999.

The two DEMs (LU ĢZZF and LVMI Silava) were combined (arithmetic mean) within
the University of Latvia project ``Improvement of sustainable soil resource management
in agriculture: E2SOILAGRI'', was used as the working DEM. The resolution of this DEM is 1 m,
which is too detailed for species distribution modeling input data, therefore
the layer was designed to correspond to the reference 10 m raster.

When investigating the combined DEM, there were clearly
visible areas with no data. This has been solved by using
the DEM with a resolution of 10 m developed by Māris Nartišs (LU ĢZZF) in 2018,
which covers the entire territory of Latvia without gaps. To avoid sharp
edges and ensure smooth transitions, we created an arithmetic mean
layer covering all of Latvia and aligned to the
reference raster.

A slope layer has also been created from this raster, which is designed in
accordance with the reference. The slope is expressed in degrees and calculated
using the 8-neighbor approach. The same applies to the aspect or slope
direction.

\begin{Shaded}
\begin{Highlighting}[]
\CommentTok{\# libs}
\ControlFlowTok{if}\NormalTok{(}\SpecialCharTok{!}\FunctionTok{require}\NormalTok{(terra)) \{}\FunctionTok{install.packages}\NormalTok{(}\StringTok{"terra"}\NormalTok{); }\FunctionTok{require}\NormalTok{(terra)\}}
\ControlFlowTok{if}\NormalTok{(}\SpecialCharTok{!}\FunctionTok{require}\NormalTok{(sf)) \{}\FunctionTok{install.packages}\NormalTok{(}\StringTok{"sf"}\NormalTok{); }\FunctionTok{require}\NormalTok{(sf)\}}

\CommentTok{\# reference}
\NormalTok{template10}\OtherTok{=}\FunctionTok{rast}\NormalTok{(}\StringTok{"./Templates/TemplateRasters/LV10m\_10km.tif"}\NormalTok{)}

\CommentTok{\# LiDAR DEM 1 m to 10 m }

\NormalTok{lapas\_1m}\OtherTok{=}\FunctionTok{data.frame}\NormalTok{(}\AttributeTok{faili=}\FunctionTok{list.files}\NormalTok{(}\StringTok{"./Geodata/2024/DEM/meanDEM\_1mOLD/"}\NormalTok{,}\AttributeTok{pattern=}\StringTok{"*.tif$"}\NormalTok{))}
\NormalTok{lapas\_1m}\SpecialCharTok{$}\NormalTok{numurs}\OtherTok{=}\FunctionTok{substr}\NormalTok{(lapas\_1m}\SpecialCharTok{$}\NormalTok{faili,}\DecValTok{10}\NormalTok{,}\DecValTok{13}\NormalTok{)}
\NormalTok{lapas\_1m}\SpecialCharTok{$}\NormalTok{cels1}\OtherTok{=}\FunctionTok{paste0}\NormalTok{(}\StringTok{"./Geodata/2024/DEM/meanDEM\_1mOLD/"}\NormalTok{,lapas\_1m}\SpecialCharTok{$}\NormalTok{faili)}
\NormalTok{lapas\_1m}\SpecialCharTok{$}\NormalTok{cels2}\OtherTok{=}\FunctionTok{paste0}\NormalTok{(}\StringTok{"./Geodata/2024/DEM/meanDEM\_10mOLD/"}\NormalTok{,lapas\_1m}\SpecialCharTok{$}\NormalTok{faili)}

\NormalTok{kvadrati}\OtherTok{=}\FunctionTok{st\_read}\NormalTok{(}\AttributeTok{dsn=}\StringTok{"GIS\_Latvija10.2.gdb"}\NormalTok{,}\AttributeTok{layer=}\StringTok{"tks93\_50000"}\NormalTok{)}
\NormalTok{kvadrati}\SpecialCharTok{$}\NormalTok{name}\OtherTok{=}\FunctionTok{as.character}\NormalTok{(kvadrati}\SpecialCharTok{$}\NormalTok{num50tk)}

\NormalTok{moz2}\OtherTok{=}\FunctionTok{rast}\NormalTok{(}\StringTok{"./Geodata/2024/DEM/Nartiss\_visa\_Latvija/dem10\_20\_kopa.tif"}\NormalTok{)}

\ControlFlowTok{for}\NormalTok{(i }\ControlFlowTok{in} \DecValTok{1}\SpecialCharTok{:}\FunctionTok{length}\NormalTok{(kvadrati}\SpecialCharTok{$}\NormalTok{name))\{}
\NormalTok{  kvadrats}\OtherTok{=}\NormalTok{kvadrati[i,]}
\NormalTok{  nosaukums}\OtherTok{=}\NormalTok{kvadrats}\SpecialCharTok{$}\NormalTok{name}
\NormalTok{  telpa}\OtherTok{=}\NormalTok{terra}\SpecialCharTok{::}\FunctionTok{ext}\NormalTok{(kvadrats)}
  
\NormalTok{  paraugs}\OtherTok{=}\FunctionTok{crop}\NormalTok{(template10,telpa)}
\NormalTok{  nart}\OtherTok{=}\FunctionTok{crop}\NormalTok{(moz2,telpa)}
\NormalTok{  nart2}\OtherTok{=}\FunctionTok{project}\NormalTok{(nart,paraugs,}\AttributeTok{mask=}\ConstantTok{TRUE}\NormalTok{)}
  
\NormalTok{  dem1m}\OtherTok{=}\NormalTok{lapas\_1m[lapas\_1m}\SpecialCharTok{$}\NormalTok{numurs}\SpecialCharTok{==}\NormalTok{kvadrats}\SpecialCharTok{$}\NormalTok{name,]}
  \ControlFlowTok{if}\NormalTok{(}\FunctionTok{nrow}\NormalTok{(dem1m)}\SpecialCharTok{\textgreater{}}\DecValTok{0}\NormalTok{)\{}
\NormalTok{    sakumcels}\OtherTok{=}\NormalTok{dem1m}\SpecialCharTok{$}\NormalTok{cels1}
\NormalTok{    dem}\OtherTok{=}\FunctionTok{rast}\NormalTok{(sakumcels)}
\NormalTok{    reproj}\OtherTok{=}\FunctionTok{project}\NormalTok{(dem,paraugs,}\AttributeTok{mask=}\ConstantTok{TRUE}\NormalTok{,}\AttributeTok{method=}\StringTok{"bilinear"}\NormalTok{,}\AttributeTok{use\_gdal=}\ConstantTok{TRUE}\NormalTok{)}
\NormalTok{    videjais }\OtherTok{\textless{}{-}} \FunctionTok{ifel}\NormalTok{(}\FunctionTok{is.na}\NormalTok{(nart2),nart2,}\FunctionTok{ifel}\NormalTok{(}\FunctionTok{is.na}\NormalTok{(reproj),nart2,}
                                             \FunctionTok{app}\NormalTok{(}\FunctionTok{c}\NormalTok{(nart2,reproj), mean)))}
    \FunctionTok{writeRaster}\NormalTok{(videjais,}\AttributeTok{overwrite=}\ConstantTok{TRUE}\NormalTok{,}
                \AttributeTok{filename=}\FunctionTok{paste0}\NormalTok{(}\StringTok{"./Geodata/2024/DEM/meanDEM\_10m/"}\NormalTok{,}\StringTok{"vidDEM\_"}\NormalTok{,}
\NormalTok{                                nosaukums,}\StringTok{".tif"}\NormalTok{))}
\NormalTok{  \}}
  \ControlFlowTok{else}\NormalTok{\{}
    \FunctionTok{writeRaster}\NormalTok{(nart2,}\AttributeTok{overwrite=}\ConstantTok{TRUE}\NormalTok{,}
                \AttributeTok{filename=}\FunctionTok{paste0}\NormalTok{(}\StringTok{"./Geodata/2024/DEM/meanDEM\_10m/"}\NormalTok{,}\StringTok{"vidDEM\_"}\NormalTok{,}
\NormalTok{                                nosaukums,}\StringTok{".tif"}\NormalTok{))}
\NormalTok{  \}}
\NormalTok{\}}

\CommentTok{\# vrt un mosaic}
\NormalTok{lapas\_10}\OtherTok{=}\FunctionTok{data.frame}\NormalTok{(}\AttributeTok{faili=}\FunctionTok{list.files}\NormalTok{(}\StringTok{"./Geodata/2024/DEM/meanDEM\_10m/"}\NormalTok{,}\AttributeTok{pattern=}\StringTok{"*.tif$"}\NormalTok{))}
\NormalTok{lapas\_10}\SpecialCharTok{$}\NormalTok{celi1}\OtherTok{=}\FunctionTok{paste0}\NormalTok{(}\StringTok{"./Geodata/2024/DEM/meanDEM\_10m/"}\NormalTok{,lapas\_10}\SpecialCharTok{$}\NormalTok{faili)}
\NormalTok{mozaikai}\OtherTok{=}\FunctionTok{vrt}\NormalTok{(lapas\_10}\SpecialCharTok{$}\NormalTok{celi1,}\AttributeTok{overwrite=}\ConstantTok{TRUE}\NormalTok{,}
             \AttributeTok{filename=}\StringTok{"./Geodata/2024/DEM/vrtDEM\_10m.tif"}\NormalTok{)}
\NormalTok{mozaika}\OtherTok{=}\FunctionTok{rast}\NormalTok{(}\StringTok{"./Geodata/2024/DEM/vrtDEM\_10m.tif"}\NormalTok{)}
\FunctionTok{writeRaster}\NormalTok{(mozaika,}\StringTok{"./Geodata/2024/DEM/mozDEM\_10m.tif"}\NormalTok{)}


\DocumentationTok{\#\# slope}
\NormalTok{reljefs}\OtherTok{=}\FunctionTok{rast}\NormalTok{(}\StringTok{"./Geodata/2024/DEM/mozDEM\_10m.tif"}\NormalTok{)}
\NormalTok{slipumi}\OtherTok{=}\FunctionTok{terrain}\NormalTok{(reljefs, }\AttributeTok{v=}\StringTok{"slope"}\NormalTok{, }\AttributeTok{neighbors=}\DecValTok{8}\NormalTok{, }\AttributeTok{unit=}\StringTok{"degrees"}\NormalTok{, }
                \AttributeTok{filename=}\StringTok{"./Geodata/2024/DEM/Terrain\_Slope\_10m.tif"}\NormalTok{, }\AttributeTok{overwrite=}\ConstantTok{TRUE}\NormalTok{)  }

\DocumentationTok{\#\# aspect }
\NormalTok{reljefs}\OtherTok{=}\FunctionTok{rast}\NormalTok{(}\StringTok{"./Geodata/2024/DEM/mozDEM\_10m.tif"}\NormalTok{)}
\NormalTok{virzieni}\OtherTok{=}\FunctionTok{terrain}\NormalTok{(reljefs, }\AttributeTok{v=}\StringTok{"aspect"}\NormalTok{, }\AttributeTok{neighbors=}\DecValTok{8}\NormalTok{, }\AttributeTok{unit=}\StringTok{"degrees"}\NormalTok{, }
                 \AttributeTok{filename=}\StringTok{"./Geodata/2024/DEM/Terrain\_Aspect\_10m.tif"}\NormalTok{, }\AttributeTok{overwrite=}\ConstantTok{TRUE}\NormalTok{)}
\end{Highlighting}
\end{Shaded}

\section{Latvian Exclusive Economic Zone polygon}\label{Ch04.16}

The waters of Latvia's Exclusive Economic Zone were obtained from
the \href{https://maps.helcom.fi/website/mapservice/?datasetID=ae58c373-674c-45d1-be0f-1ff69a59f9ba}{HELCOM map and data service}. After downloading, this line file was
analogically connected to the coastline file obtained from the same resource.

\section{Bogs and Mires: EDI}\label{Ch04.17}

Data (training and classification) used in project ``Remote Sensing and Machine
Learning for Peatland Habitat Monitoring (PurvEO)'' by the Institute of electronics
and computer science (EDI) were stored at \texttt{Geodata/2024/Bogs\_EDI}.

Preprocessing was carried out to create two layers:

\begin{itemize}
\item
  \texttt{EDI\_BogsYN.tif}: training and classification results on open raised bogs (EU
  protected habitat codes 7110 and 7120) and locations where one of those overlapped
  with transitional mires (EU protected habitat code 7140);
\item
  \texttt{EDI\_TransitionalMiresYN.tif}: training and classification results on
  transitional mires (EU protected habitat code 7140) with no overlap with open
  rised bogs.
\end{itemize}

\begin{Shaded}
\begin{Highlighting}[]
\CommentTok{\# libs}
\ControlFlowTok{if}\NormalTok{(}\SpecialCharTok{!}\FunctionTok{require}\NormalTok{(sf)) \{}\FunctionTok{install.packages}\NormalTok{(}\StringTok{"sf"}\NormalTok{); }\FunctionTok{require}\NormalTok{(sf)\}}
\ControlFlowTok{if}\NormalTok{(}\SpecialCharTok{!}\FunctionTok{require}\NormalTok{(tidyverse)) \{}\FunctionTok{install.packages}\NormalTok{(}\StringTok{"tidyverse"}\NormalTok{); }\FunctionTok{require}\NormalTok{(tidyverse)\}}
\ControlFlowTok{if}\NormalTok{(}\SpecialCharTok{!}\FunctionTok{require}\NormalTok{(terra)) \{}\FunctionTok{install.packages}\NormalTok{(}\StringTok{"terra"}\NormalTok{); }\FunctionTok{require}\NormalTok{(terra)\}}


\CommentTok{\# Templates {-}{-}{-}{-}}
\NormalTok{template10}\OtherTok{=}\FunctionTok{rast}\NormalTok{(}\StringTok{"./Templates/TemplateRasters/LV10m\_10km.tif"}\NormalTok{)}
\NormalTok{template100}\OtherTok{=}\FunctionTok{rast}\NormalTok{(}\StringTok{"./Templates/TemplateRasters/LV100m\_10km.tif"}\NormalTok{)}

\NormalTok{nulles10}\OtherTok{=}\FunctionTok{rast}\NormalTok{(}\StringTok{"./Templates/TemplateRasters/nulls\_LV10m\_10km.tif"}\NormalTok{)}


\CommentTok{\# Bogs {-}{-}{-}{-}}
\NormalTok{neatklata71107120}\OtherTok{=}\FunctionTok{rast}\NormalTok{(}\FunctionTok{paste0}\NormalTok{(}\StringTok{"./Geodata/2024/Bogs\_EDI/purvi\_EDI\_projekts/"}\NormalTok{,}
                              \StringTok{"purvi/!LV\_kopa\_apv1020\_30\_05\_2022/"}\NormalTok{,}
                              \StringTok{"!LV\_kopa\_apv1020\_30\_05\_2022/"}\NormalTok{,}
                              \StringTok{"Neatklata\_purviem\_raksturiga\_zemsedze\_7110\_7120.tif"}\NormalTok{))}
\NormalTok{neatklata71107120}\OtherTok{=}\FunctionTok{ifel}\NormalTok{(neatklata71107120}\SpecialCharTok{\textgreater{}}\DecValTok{0}\NormalTok{,}\DecValTok{1}\NormalTok{,}\ConstantTok{NA}\NormalTok{)}
\FunctionTok{plot}\NormalTok{(neatklata71107120)}
\NormalTok{neatklata7140}\OtherTok{=}\FunctionTok{rast}\NormalTok{(}\FunctionTok{paste0}\NormalTok{(}\StringTok{"./Geodata/2024/Bogs\_EDI/purvi\_EDI\_projekts/"}\NormalTok{,}
                          \StringTok{"purvi/!LV\_kopa\_apv1020\_30\_05\_2022/"}\NormalTok{,}
                          \StringTok{"!LV\_kopa\_apv1020\_30\_05\_2022/"}\NormalTok{,}
                          \StringTok{"Neatklata\_purviem\_raksturiga\_zemsedze\_7140.tif"}\NormalTok{))}
\NormalTok{neatklata7140}\OtherTok{=}\FunctionTok{ifel}\NormalTok{(neatklata7140}\SpecialCharTok{\textgreater{}}\DecValTok{0}\NormalTok{,}\DecValTok{1}\NormalTok{,}\ConstantTok{NA}\NormalTok{)}

\NormalTok{raskturiga71107120}\OtherTok{=}\FunctionTok{rast}\NormalTok{(}\FunctionTok{paste0}\NormalTok{(}\StringTok{"./Geodata/2024/Bogs\_EDI/purvi\_EDI\_projekts/"}\NormalTok{,}
                               \StringTok{"purvi/!LV\_kopa\_apv1020\_30\_05\_2022/"}\NormalTok{,}
                               \StringTok{"!LV\_kopa\_apv1020\_30\_05\_2022/"}\NormalTok{,}
                               \StringTok{"Purviem\_neraksturiga\_zemsedze\_7110\_7120.tif"}\NormalTok{))}
\NormalTok{raskturiga71107120}\OtherTok{=}\FunctionTok{ifel}\NormalTok{(raskturiga71107120}\SpecialCharTok{\textgreater{}}\DecValTok{0}\NormalTok{,}\DecValTok{1}\NormalTok{,}\ConstantTok{NA}\NormalTok{)}
\NormalTok{raksturiga7140}\OtherTok{=}\FunctionTok{rast}\NormalTok{(}\FunctionTok{paste0}\NormalTok{(}\StringTok{"./Geodata/2024/Bogs\_EDI/purvi\_EDI\_projekts/"}\NormalTok{,}
                           \StringTok{"purvi/!LV\_kopa\_apv1020\_30\_05\_2022/"}\NormalTok{,}
                           \StringTok{"!LV\_kopa\_apv1020\_30\_05\_2022/"}\NormalTok{,}
                           \StringTok{"Purviem\_neraksturiga\_zemsedze\_7140.tif"}\NormalTok{))}
\NormalTok{raksturiga7140}\OtherTok{=}\FunctionTok{ifel}\NormalTok{(raksturiga7140}\SpecialCharTok{\textgreater{}}\DecValTok{0}\NormalTok{,}\DecValTok{1}\NormalTok{,}\ConstantTok{NA}\NormalTok{)}

\NormalTok{labels71107120}\OtherTok{=}\FunctionTok{rast}\NormalTok{(}\FunctionTok{paste0}\NormalTok{(}\StringTok{"./Geodata/2024/Bogs\_EDI/purvi\_EDI\_projekts/"}\NormalTok{,}
                           \StringTok{"purvi/!LV\_kopa\_apv1020\_30\_05\_2022/"}\NormalTok{,}
                           \StringTok{"!LV\_kopa\_apv1020\_30\_05\_2022/"}\NormalTok{,}
                           \StringTok{"latvija\_Labels\_B7110\_7120.tif"}\NormalTok{))}
\NormalTok{labels71107120}\OtherTok{=}\FunctionTok{ifel}\NormalTok{(labels71107120}\SpecialCharTok{\textgreater{}}\DecValTok{0}\NormalTok{,}\DecValTok{1}\NormalTok{,}\ConstantTok{NA}\NormalTok{)}
\NormalTok{labels7140}\OtherTok{=}\FunctionTok{rast}\NormalTok{(}\FunctionTok{paste0}\NormalTok{(}\StringTok{"./Geodata/2024/Bogs\_EDI/purvi\_EDI\_projekts/"}\NormalTok{,}
                       \StringTok{"purvi/!LV\_kopa\_apv1020\_30\_05\_2022/"}\NormalTok{,}
                       \StringTok{"!LV\_kopa\_apv1020\_30\_05\_2022/"}\NormalTok{,}\StringTok{"}
\StringTok{                       latvija\_Labels\_B7140.tif"}\NormalTok{))}
\NormalTok{labels7140}\OtherTok{=}\FunctionTok{ifel}\NormalTok{(labels7140}\SpecialCharTok{\textgreater{}}\DecValTok{0}\NormalTok{,}\DecValTok{1}\NormalTok{,}\ConstantTok{NA}\NormalTok{)}

\NormalTok{augstie}\OtherTok{=}\FunctionTok{cover}\NormalTok{(}\FunctionTok{cover}\NormalTok{(neatklata71107120,raskturiga71107120),labels71107120)}
\NormalTok{parejas}\OtherTok{=}\FunctionTok{cover}\NormalTok{(}\FunctionTok{cover}\NormalTok{(neatklata7140,raksturiga7140),labels7140)}
\NormalTok{tikai\_parejas}\OtherTok{=}\FunctionTok{ifel}\NormalTok{(parejas}\SpecialCharTok{==}\DecValTok{1}\SpecialCharTok{\&}\NormalTok{augstie}\SpecialCharTok{==}\DecValTok{1}\NormalTok{,}\ConstantTok{NA}\NormalTok{,parejas)}
\NormalTok{sunainie}\OtherTok{=}\FunctionTok{ifel}\NormalTok{(parejas}\SpecialCharTok{==}\DecValTok{1}\SpecialCharTok{\&}\NormalTok{augstie}\SpecialCharTok{==}\DecValTok{1}\NormalTok{,parejas,}\ConstantTok{NA}\NormalTok{)}

\NormalTok{sunu\_purvi}\OtherTok{=}\FunctionTok{cover}\NormalTok{(augstie,sunainie)}

\NormalTok{sunu\_proj}\OtherTok{=}\FunctionTok{project}\NormalTok{(sunu\_purvi,template10)}
\NormalTok{sunuYN}\OtherTok{=}\FunctionTok{cover}\NormalTok{(sunu\_proj,nulles10)}
\FunctionTok{plot}\NormalTok{(sunuYN)}
\FunctionTok{writeRaster}\NormalTok{(sunuYN,}
            \AttributeTok{overwrite=}\ConstantTok{TRUE}\NormalTok{,}
            \AttributeTok{filename=}\StringTok{"./RasterGrids\_10m/2024/EDI\_BogsYN.tif"}\NormalTok{)}



\CommentTok{\# Transitional mires {-}{-}{-}{-}}
\NormalTok{parejas\_proj}\OtherTok{=}\FunctionTok{project}\NormalTok{(tikai\_parejas,template10)}
\NormalTok{parejasYN}\OtherTok{=}\FunctionTok{cover}\NormalTok{(parejas\_proj,nulles10)}
\FunctionTok{plot}\NormalTok{(parejasYN)}
\FunctionTok{writeRaster}\NormalTok{(parejasYN,}
            \AttributeTok{overwrite=}\ConstantTok{TRUE}\NormalTok{,}
            \AttributeTok{filename=}\StringTok{"./RasterGrids\_10m/2024/EDI\_TransitionalMiresYN.tif"}\NormalTok{)}
\end{Highlighting}
\end{Shaded}

\chapter{Geodata products}\label{Ch05}

Some raw data need extensive processing prior to EGVs creation. Often, EGVs
relay on transforming raw geodata into intermediate products; in other cases, an EGV
itself could be created from raw geodata, but it has
to be spatially restricted to certain locations. This chapter describes these geodata
products and the procedures involved in creating them.

\section{Terrain products}\label{Ch05.01}

In order to develop the topographic wetness index (TWI) and non-drainage depressions,
it was necessary to address
water flow in the environment. This is a multi-step procedure that is logical
and reliable in mountainous areas and in environments with little hydrological
impact. However, in the context of Latvia, this was challenging. These challenges
can be addressed in various ways. For example, if reliable (accurate) information
on the exact locations of rivers and ditches were available, it could be
incorporated into the terrain. Unfortunately, there is no sufficiently
accurate information available. Therefore, information about network
structures from the \hyperref[Ch04.03]{Melioration Cadastre Information System database}
buffered by 10 m, bridges from the \hyperref[Ch04.04]{topographic map} and transport structures and bridges
from \hyperref[Ch04.06]{LVM Open Data} were used to address the challenges (both buffered
by 10 m). Information about the minimum height above sea level was incorporated into the
DEM to be used in further processing.

\begin{Shaded}
\begin{Highlighting}[]
\CommentTok{\# libs}
\ControlFlowTok{if}\NormalTok{(}\SpecialCharTok{!}\FunctionTok{require}\NormalTok{(terra)) \{}\FunctionTok{install.packages}\NormalTok{(}\StringTok{"terra"}\NormalTok{); }\FunctionTok{require}\NormalTok{(terra)\}}
\ControlFlowTok{if}\NormalTok{(}\SpecialCharTok{!}\FunctionTok{require}\NormalTok{(sf)) \{}\FunctionTok{install.packages}\NormalTok{(}\StringTok{"sf"}\NormalTok{); }\FunctionTok{require}\NormalTok{(sf)\}}
\ControlFlowTok{if}\NormalTok{(}\SpecialCharTok{!}\FunctionTok{require}\NormalTok{(tidyverse)) \{}\FunctionTok{install.packages}\NormalTok{(}\StringTok{"tidyverse"}\NormalTok{); }\FunctionTok{require}\NormalTok{(tidyverse)\}}
\ControlFlowTok{if}\NormalTok{(}\SpecialCharTok{!}\FunctionTok{require}\NormalTok{(arrow)) \{}\FunctionTok{install.packages}\NormalTok{(}\StringTok{"arrow"}\NormalTok{); }\FunctionTok{require}\NormalTok{(arrow)\}}
\ControlFlowTok{if}\NormalTok{(}\SpecialCharTok{!}\FunctionTok{require}\NormalTok{(sfarrow)) \{}\FunctionTok{install.packages}\NormalTok{(}\StringTok{"sfarrow"}\NormalTok{); }\FunctionTok{require}\NormalTok{(sfarrow)\}}
\ControlFlowTok{if}\NormalTok{(}\SpecialCharTok{!}\FunctionTok{require}\NormalTok{(exactextractr))\{}\FunctionTok{install.packages}\NormalTok{(}\StringTok{"exactextractr"}\NormalTok{);}\FunctionTok{require}\NormalTok{(exactextractr)\}}

\CommentTok{\# reference}
\NormalTok{template}\OtherTok{=}\FunctionTok{rast}\NormalTok{(}\StringTok{"./Templates/TemplateRasters/LV10m\_10km.tif"}\NormalTok{)}

\CommentTok{\# part one {-}{-}{-}{-}}

\CommentTok{\# dem raster}
\NormalTok{reljefs}\OtherTok{=}\FunctionTok{rast}\NormalTok{(}\StringTok{"./Geodata/2024/DEM/mozDEM\_10m.tif"}\NormalTok{)}

\CommentTok{\# drainage network structures}
\FunctionTok{st\_layers}\NormalTok{(}\StringTok{"./Geodata/2024/MKIS/MKIS\_2025.gpkg"}\NormalTok{)}

\NormalTok{dtb}\OtherTok{=}\FunctionTok{st\_read}\NormalTok{(}\StringTok{"./Geodata/2024/MKIS/MKIS\_2025.gpkg"}\NormalTok{,}\AttributeTok{layer=}\StringTok{"DrenazasTiklaBuves"}\NormalTok{)}
\NormalTok{dtb\_buffer}\OtherTok{=}\FunctionTok{st\_buffer}\NormalTok{(dtb,}\AttributeTok{dist=}\DecValTok{10}\NormalTok{)}

\CommentTok{\# bridges }
\NormalTok{tiltiL}\OtherTok{=}\NormalTok{sfarrow}\SpecialCharTok{::}\FunctionTok{st\_read\_parquet}\NormalTok{(}\StringTok{"./Geodata/2024/TopographicMap/BridgeL\_COMB.parquet"}\NormalTok{)}
\NormalTok{tiltiL\_buffer}\OtherTok{=}\FunctionTok{st\_buffer}\NormalTok{(tiltiL,}\AttributeTok{dist=}\DecValTok{30}\NormalTok{)}
\NormalTok{tiltiP}\OtherTok{=}\NormalTok{sfarrow}\SpecialCharTok{::}\FunctionTok{st\_read\_parquet}\NormalTok{(}\StringTok{"./Geodata/2024/TopographicMap/BridgeL\_COMB.parquet"}\NormalTok{)}
\NormalTok{tiltiP\_buffer}\OtherTok{=}\FunctionTok{st\_buffer}\NormalTok{(tiltiP,}\AttributeTok{dist=}\DecValTok{30}\NormalTok{)}

\CommentTok{\# LVM}
\NormalTok{lvm\_caurtekas}\OtherTok{=}\FunctionTok{st\_read}\NormalTok{(}\StringTok{"./Geodata/2024/LVM\_OpenData/LVM\_CAURTEKAS/LVM\_CAURTEKAS\_Shape.shp"}\NormalTok{)}
\NormalTok{lvm\_buffer}\OtherTok{=}\FunctionTok{st\_buffer}\NormalTok{(lvm\_caurtekas,}\AttributeTok{dist=}\DecValTok{30}\NormalTok{)}


\CommentTok{\# buffers}
\FunctionTok{st\_geometry}\NormalTok{(dtb\_buffer)}\OtherTok{=}\StringTok{"geometry"}
\FunctionTok{st\_geometry}\NormalTok{(tiltiL\_buffer)}\OtherTok{=}\StringTok{"geometry"}
\FunctionTok{st\_geometry}\NormalTok{(tiltiP\_buffer)}\OtherTok{=}\StringTok{"geometry"}
\FunctionTok{st\_geometry}\NormalTok{(lvm\_buffer)}\OtherTok{=}\StringTok{"geometry"}
\NormalTok{visi\_buferi}\OtherTok{=}\FunctionTok{bind\_rows}\NormalTok{(dtb\_buffer,tiltiL\_buffer,tiltiP\_buffer,lvm\_buffer)}

\CommentTok{\# incorporation in DEM}
\NormalTok{visi\_buferi}\SpecialCharTok{$}\NormalTok{vertiba}\OtherTok{=}\NormalTok{exactextractr}\SpecialCharTok{::}\FunctionTok{exact\_extract}\NormalTok{(reljefs,visi\_buferi,}\StringTok{"min"}\NormalTok{)}

\NormalTok{caurumi}\OtherTok{=}\NormalTok{fasterize}\SpecialCharTok{::}\FunctionTok{fasterize}\NormalTok{(visi\_buferi,templis,}\AttributeTok{field=}\StringTok{"vertiba"}\NormalTok{)}
\NormalTok{caurumi2}\OtherTok{=}\FunctionTok{rast}\NormalTok{(caurumi)}
\NormalTok{caurumains}\OtherTok{=}\FunctionTok{app}\NormalTok{(}\FunctionTok{c}\NormalTok{(reljefs,caurumi2),}\AttributeTok{fun=}\StringTok{"min"}\NormalTok{,}\AttributeTok{na.rm=}\ConstantTok{TRUE}\NormalTok{,}
               \AttributeTok{overwrite=}\ConstantTok{TRUE}\NormalTok{,}
               \AttributeTok{filename=}\StringTok{"./Geodata/2024/DEM/caurtDEM\_10m.tif"}\NormalTok{)}

\CommentTok{\# cleaning}
\FunctionTok{rm}\NormalTok{(caurumi)}
\FunctionTok{rm}\NormalTok{(caurumi2)}
\FunctionTok{rm}\NormalTok{(dtb)}
\FunctionTok{rm}\NormalTok{(dtb\_buffer)}
\FunctionTok{rm}\NormalTok{(lvm\_buffer)}
\FunctionTok{rm}\NormalTok{(lvm\_caurtekas)}
\FunctionTok{rm}\NormalTok{(reljefs)}
\FunctionTok{rm}\NormalTok{(tiltiL)}
\FunctionTok{rm}\NormalTok{(tiltiL\_buffer)}
\FunctionTok{rm}\NormalTok{(tiltiP)}
\FunctionTok{rm}\NormalTok{(tiltiP\_buffer)}
\FunctionTok{rm}\NormalTok{(visi\_buferi)}
\FunctionTok{rm}\NormalTok{(caurumains)}
\end{Highlighting}
\end{Shaded}

This DEM was then used for geoprocessing to identify terrain depressions and
determine the topographic wetness index (TWI):

\begin{enumerate}
\def\labelenumi{\arabic{enumi}.}
\item
  drainage depressions and their depth layers were prepared after
  incorporating flow breaks;
\item
  to calculate the topographic wetness index, terrain depressions without
  runoff were reviewed, allowing up to ten cell breaks in areas of lower
  resistance; the rest were filled in;
\item
  for additional security, the procedure of the second step was repeated to
  search for and fill in terrain depressions (\citeproc{ref-WangLiu2006}{Wang and Liu, 2006});
\item
  the result of the third step was used to determine the specific catchment
  area using D-infinity flow direction;
\item
  by combining the specific catchment area layer with the slope layer,
  the topographic wetness index was calculated. A graphical evaluation revealed
  individual extreme values, which were limited to \textbf{20}.
\end{enumerate}

\begin{Shaded}
\begin{Highlighting}[]
\CommentTok{\# libs}
\ControlFlowTok{if}\NormalTok{(}\SpecialCharTok{!}\FunctionTok{require}\NormalTok{(terra)) \{}\FunctionTok{install.packages}\NormalTok{(}\StringTok{"terra"}\NormalTok{); }\FunctionTok{require}\NormalTok{(terra)\}}
\ControlFlowTok{if}\NormalTok{(}\SpecialCharTok{!}\FunctionTok{require}\NormalTok{(whitebox))\{}\FunctionTok{install.packages}\NormalTok{(}\StringTok{"whitebox"}\NormalTok{);}\FunctionTok{require}\NormalTok{(whitebox)\}}

\CommentTok{\# reference}
\NormalTok{template}\OtherTok{=}\FunctionTok{rast}\NormalTok{(}\StringTok{"./Templates/TemplateRasters/LV10m\_10km.tif"}\NormalTok{)}

\CommentTok{\# part two {-}{-}{-}{-}}

\CommentTok{\# DEM}
\NormalTok{caurumainis}\OtherTok{=}\FunctionTok{rast}\NormalTok{(}\StringTok{"./Geodata/2024/DEM/caurtDEM\_10m.tif"}\NormalTok{)}

\CommentTok{\# Sinks}
\DocumentationTok{\#\# breached sinks and depth in sinks}
\FunctionTok{wbt\_breach\_depressions\_least\_cost}\NormalTok{(}
  \AttributeTok{dem =} \StringTok{"./Geodata/2024/DEM/caurtDEM\_10m.tif"}\NormalTok{,}
  \AttributeTok{output =} \StringTok{"./Geodata/2024/DEM/caurtDEM\_breachedNF.tif"}\NormalTok{,}
  \AttributeTok{dist =} \DecValTok{10}\NormalTok{,}
  \AttributeTok{fill =} \ConstantTok{FALSE}\NormalTok{)}
\FunctionTok{wbt\_depth\_in\_sink}\NormalTok{(}\AttributeTok{dem=}\StringTok{"./Geodata/2024/DEM/caurtDEM\_breachedNF.tif"}\NormalTok{,}
                  \AttributeTok{output=}\StringTok{"./Geodata/2024/DEM/Terrain\_DiS\_breached\_10m.tif"}\NormalTok{,}
                  \AttributeTok{zero\_background =} \ConstantTok{TRUE}\NormalTok{)}
\FunctionTok{wbt\_sink}\NormalTok{(}\AttributeTok{input =} \StringTok{"./Geodata/2024/DEM/caurtDEM\_breachedNF.tif"}\NormalTok{,}
         \AttributeTok{output =} \StringTok{"./Geodata/2024/DEM/Terrain\_Sink\_breached\_10m.tif"}\NormalTok{,}
         \AttributeTok{verbose\_mode =} \ConstantTok{FALSE}\NormalTok{,}\AttributeTok{zero\_background =} \ConstantTok{TRUE}\NormalTok{)}
\NormalTok{sinks}\OtherTok{=}\FunctionTok{rast}\NormalTok{(}\StringTok{"./Geodata/2024/DEM/Terrain\_Sink\_breached\_10m.tif"}\NormalTok{)}

\NormalTok{sinks2 }\OtherTok{\textless{}{-}} \FunctionTok{ifel}\NormalTok{(sinks }\SpecialCharTok{\textgreater{}=} \DecValTok{1}\NormalTok{, }\DecValTok{1}\NormalTok{, sinks,}
               \AttributeTok{filename=}\StringTok{"./Geodata/2024/DEM/Terrain\_SinkYN\_breached\_10m.tif"}\NormalTok{)}
\FunctionTok{plot}\NormalTok{(sinks2)}
\FunctionTok{unlink}\NormalTok{(}\StringTok{"./Geodata/2024/DEM/Terrain\_Sink\_breached\_10m.tif"}\NormalTok{)}

\CommentTok{\# TWI}
\DocumentationTok{\#\# breaching}
\FunctionTok{wbt\_breach\_depressions\_least\_cost}\NormalTok{(}
  \AttributeTok{dem =} \StringTok{"./Geodata/2024/DEM/caurtDEM\_10m.tif"}\NormalTok{,}
  \AttributeTok{output =} \StringTok{"./Geodata/2024/DEM/caurtDEM\_breachedF.tif"}\NormalTok{,}
  \AttributeTok{dist =} \DecValTok{10}\NormalTok{,}
  \AttributeTok{fill =} \ConstantTok{TRUE}\NormalTok{)}

\DocumentationTok{\#\#\# filling}
\FunctionTok{wbt\_fill\_depressions\_wang\_and\_liu}\NormalTok{(}
  \AttributeTok{dem =} \StringTok{"./Geodata/2024/DEM/caurtDEM\_breachedF.tif"}\NormalTok{,}
  \AttributeTok{output =} \StringTok{"./Geodata/2024/DEM/caurtDEM\_BreachFill.tif"}
\NormalTok{)}

\DocumentationTok{\#\#\# (d inf) flow direction}
\FunctionTok{wbt\_d\_inf\_flow\_accumulation}\NormalTok{(}\AttributeTok{input =} \StringTok{"./Geodata/2024/DEM/caurtDEM\_BreachFill.tif"}\NormalTok{,}
                            \AttributeTok{output =} \StringTok{"./Geodata/2024/DEM/caurtDEM\_DInfAccu\_SCA.tif"}\NormalTok{,}
                            \AttributeTok{out\_type =} \StringTok{"Specific Contributing Area"}\NormalTok{)}

\DocumentationTok{\#\#\# twi}
\FunctionTok{wbt\_wetness\_index}\NormalTok{(}\AttributeTok{sca =} \StringTok{"./Geodata/2024/DEM/caurtDEM\_DInfAccu\_SCA.tif"}\NormalTok{,}
                  \AttributeTok{slope =} \StringTok{"./Geodata/2024/DEM/Terrain\_Slope\_10m.tif"}\NormalTok{,}
                  \AttributeTok{output =} \StringTok{"./Geodata/2024/DEM/TWI\_caurtDEM.tif"}\NormalTok{)}
\NormalTok{twi}\OtherTok{=}\FunctionTok{rast}\NormalTok{(}\StringTok{"./Geodata/2024/DEM/TWI\_caurtDEM.tif"}\NormalTok{)}
\FunctionTok{hist}\NormalTok{(twi) }\CommentTok{\# excessively large values}
\FunctionTok{plot}\NormalTok{(twi)}
\NormalTok{twi2}\OtherTok{=}\FunctionTok{ifel}\NormalTok{(twi}\SpecialCharTok{\textgreater{}}\DecValTok{20}\NormalTok{,}\DecValTok{20}\NormalTok{,twi)}
\FunctionTok{plot}\NormalTok{(twi2)}
\NormalTok{twi2x}\OtherTok{=}\FunctionTok{ifel}\NormalTok{(}\FunctionTok{is.na}\NormalTok{(twi2)}\SpecialCharTok{\&!}\FunctionTok{is.na}\NormalTok{(template),}\DecValTok{20}\NormalTok{,twi2) }\CommentTok{\# Lake Burtnieks}

\FunctionTok{writeRaster}\NormalTok{(twi2x,}\AttributeTok{filename=}\StringTok{"./Geodata/2024/DEM/Terrain\_TWI\_lim20\_caurtDEM.tif"}\NormalTok{)}

\CommentTok{\# cleaning}
\FunctionTok{rm}\NormalTok{(sinks)}
\FunctionTok{rm}\NormalTok{(sinks2)}
\FunctionTok{rm}\NormalTok{(caurumainis)}
\FunctionTok{rm}\NormalTok{(twi)}
\FunctionTok{rm}\NormalTok{(twi2)}
\end{Highlighting}
\end{Shaded}

Since the initial DEM input was created by filling in water bodies using
interpolation methods, the water bodies show a pronounced terrain, which had
to be removed. This was done by overlaying arithmetic mean values of these polygons.

\begin{Shaded}
\begin{Highlighting}[]
\CommentTok{\# libs}
\ControlFlowTok{if}\NormalTok{(}\SpecialCharTok{!}\FunctionTok{require}\NormalTok{(terra)) \{}\FunctionTok{install.packages}\NormalTok{(}\StringTok{"terra"}\NormalTok{); }\FunctionTok{require}\NormalTok{(terra)\}}
\ControlFlowTok{if}\NormalTok{(}\SpecialCharTok{!}\FunctionTok{require}\NormalTok{(sf)) \{}\FunctionTok{install.packages}\NormalTok{(}\StringTok{"sf"}\NormalTok{); }\FunctionTok{require}\NormalTok{(sf)\}}
\ControlFlowTok{if}\NormalTok{(}\SpecialCharTok{!}\FunctionTok{require}\NormalTok{(tidyverse)) \{}\FunctionTok{install.packages}\NormalTok{(}\StringTok{"tidyverse"}\NormalTok{); }\FunctionTok{require}\NormalTok{(tidyverse)\}}
\ControlFlowTok{if}\NormalTok{(}\SpecialCharTok{!}\FunctionTok{require}\NormalTok{(arrow)) \{}\FunctionTok{install.packages}\NormalTok{(}\StringTok{"arrow"}\NormalTok{); }\FunctionTok{require}\NormalTok{(arrow)\}}
\ControlFlowTok{if}\NormalTok{(}\SpecialCharTok{!}\FunctionTok{require}\NormalTok{(sfarrow)) \{}\FunctionTok{install.packages}\NormalTok{(}\StringTok{"sfarrow"}\NormalTok{); }\FunctionTok{require}\NormalTok{(sfarrow)\}}
\ControlFlowTok{if}\NormalTok{(}\SpecialCharTok{!}\FunctionTok{require}\NormalTok{(exactextractr))\{}\FunctionTok{install.packages}\NormalTok{(}\StringTok{"exactextractr"}\NormalTok{);}\FunctionTok{require}\NormalTok{(exactextractr)\}}

\CommentTok{\# reference}
\NormalTok{template}\OtherTok{=}\FunctionTok{rast}\NormalTok{(}\StringTok{"./Templates/TemplateRasters/LV10m\_10km.tif"}\NormalTok{)}
\CommentTok{\# third part {-}{-}{-}{-}}


\CommentTok{\#  dealing with waterbodies }
\NormalTok{udeni}\OtherTok{=}\NormalTok{sfarrow}\SpecialCharTok{::}\FunctionTok{st\_read\_parquet}\NormalTok{(}\StringTok{"./Geodata/2024/TopographicMap/HidroA\_COMB.parquet"}\NormalTok{)}

\NormalTok{slope}\OtherTok{=}\FunctionTok{rast}\NormalTok{(}\StringTok{"./Geodata/2024/DEM/Terrain\_Slope\_10m.tif"}\NormalTok{)}
\NormalTok{aspect}\OtherTok{=}\FunctionTok{rast}\NormalTok{(}\StringTok{"./Geodata/2024/DEM/Terrain\_Aspect\_10m.tif"}\NormalTok{)}
\NormalTok{twi}\OtherTok{=}\FunctionTok{rast}\NormalTok{(}\StringTok{"./Geodata/2024/DEM/Terrain\_TWI\_lim20\_caurtDEM.tif"}\NormalTok{)}
\NormalTok{dis}\OtherTok{=}\FunctionTok{rast}\NormalTok{(}\StringTok{"./Geodata/2024/DEM/Terrain\_DiS\_breached\_10m.tif"}\NormalTok{)}


\CommentTok{\# average per waterbody}
\NormalTok{udeni}\SpecialCharTok{$}\NormalTok{slopes}\OtherTok{=}\NormalTok{exactextractr}\SpecialCharTok{::}\FunctionTok{exact\_extract}\NormalTok{(slope,udeni,}\StringTok{"mean"}\NormalTok{)}
\NormalTok{caurumi\_slope}\OtherTok{=}\NormalTok{fasterize}\SpecialCharTok{::}\FunctionTok{fasterize}\NormalTok{(udeni,templis,}\AttributeTok{field=}\StringTok{"slopes"}\NormalTok{)}
\NormalTok{caurumi\_slope2}\OtherTok{=}\FunctionTok{rast}\NormalTok{(caurumi\_slope)}
\NormalTok{caurumains\_slope}\OtherTok{=}\FunctionTok{app}\NormalTok{(}\FunctionTok{c}\NormalTok{(caurumi\_slope2,slope),}\AttributeTok{fun=}\StringTok{"first"}\NormalTok{,}\AttributeTok{na.rm=}\ConstantTok{TRUE}\NormalTok{,}
                     \AttributeTok{overwrite=}\ConstantTok{TRUE}\NormalTok{,}
                     \AttributeTok{filename=}\StringTok{"./Geodata/2024/DEM/Terrain\_Slope\_udeni\_10m.tif"}\NormalTok{)}
\NormalTok{caurumains\_slope}\OtherTok{=}\NormalTok{terra}\SpecialCharTok{::}\FunctionTok{rast}\NormalTok{(}\StringTok{"./Geodata/2024/DEM/Terrain\_Slope\_udeni\_10m.tif"}\NormalTok{)}
\NormalTok{caurumains\_slope2}\OtherTok{=}\NormalTok{terra}\SpecialCharTok{::}\FunctionTok{mask}\NormalTok{(caurumains\_slope,template,}
                              \AttributeTok{overwrite=}\ConstantTok{TRUE}\NormalTok{,}
                              \AttributeTok{filename=}\StringTok{"./RasterGrids\_10m/2024/Terrain\_Slope\_udeni2\_10m.tif"}\NormalTok{)}
\FunctionTok{rm}\NormalTok{(slope)}
\FunctionTok{rm}\NormalTok{(caurumi\_slope)}
\FunctionTok{rm}\NormalTok{(caurumi\_slope2)}
\FunctionTok{rm}\NormalTok{(caurumains\_slope)}
\FunctionTok{rm}\NormalTok{(caurumains\_slope2)}


\NormalTok{udeni}\SpecialCharTok{$}\NormalTok{aspect}\OtherTok{=}\NormalTok{exactextractr}\SpecialCharTok{::}\FunctionTok{exact\_extract}\NormalTok{(aspect,udeni,}\StringTok{"mean"}\NormalTok{)}
\NormalTok{caurumi\_aspect}\OtherTok{=}\NormalTok{fasterize}\SpecialCharTok{::}\FunctionTok{fasterize}\NormalTok{(udeni,templis,}\AttributeTok{field=}\StringTok{"aspect"}\NormalTok{)}
\NormalTok{caurumi\_aspect2}\OtherTok{=}\FunctionTok{rast}\NormalTok{(caurumi\_aspect)}
\NormalTok{caurumi\_aspect}\OtherTok{=}\FunctionTok{app}\NormalTok{(}\FunctionTok{c}\NormalTok{(caurumi\_aspect2,aspect),}\AttributeTok{fun=}\StringTok{"first"}\NormalTok{,}\AttributeTok{na.rm=}\ConstantTok{TRUE}\NormalTok{,}
                   \AttributeTok{overwrite=}\ConstantTok{TRUE}\NormalTok{,}
                   \AttributeTok{filename=}\StringTok{"./Geodata/2024/DEM/Terrain\_Aspect\_udeni\_10m.tif"}\NormalTok{)}
\NormalTok{caurumains\_aspect}\OtherTok{=}\NormalTok{terra}\SpecialCharTok{::}\FunctionTok{rast}\NormalTok{(}\StringTok{"./Geodata/2024/DEM/Terrain\_Aspect\_udeni\_10m.tif"}\NormalTok{)}
\NormalTok{caurumains\_aspect2}\OtherTok{=}\NormalTok{terra}\SpecialCharTok{::}\FunctionTok{mask}\NormalTok{(caurumains\_aspect,template,}
                               \AttributeTok{overwrite=}\ConstantTok{TRUE}\NormalTok{,}
                               \AttributeTok{filename=}\StringTok{"./RasterGrids\_10m/2024/Terrain\_Aspect\_udeni2\_10m.tif"}\NormalTok{)}
\FunctionTok{rm}\NormalTok{(aspect)}
\FunctionTok{rm}\NormalTok{(caurumi\_aspect)}
\FunctionTok{rm}\NormalTok{(caurumi\_aspect2)}
\FunctionTok{rm}\NormalTok{(caurumains\_aspect)}
\FunctionTok{rm}\NormalTok{(caurumains\_aspect2)}



\NormalTok{udeni}\SpecialCharTok{$}\NormalTok{twis}\OtherTok{=}\NormalTok{exactextractr}\SpecialCharTok{::}\FunctionTok{exact\_extract}\NormalTok{(twi,udeni,}\StringTok{"mean"}\NormalTok{)}
\NormalTok{caurumi\_TWI}\OtherTok{=}\NormalTok{fasterize}\SpecialCharTok{::}\FunctionTok{fasterize}\NormalTok{(udeni,templis,}\AttributeTok{field=}\StringTok{"twis"}\NormalTok{)}
\NormalTok{caurumi\_TWI2}\OtherTok{=}\FunctionTok{rast}\NormalTok{(caurumi\_TWI)}
\NormalTok{caurumains\_TWI}\OtherTok{=}\FunctionTok{app}\NormalTok{(}\FunctionTok{c}\NormalTok{(caurumi\_TWI2,twi),}\AttributeTok{fun=}\StringTok{"first"}\NormalTok{,}\AttributeTok{na.rm=}\ConstantTok{TRUE}\NormalTok{,}
                   \AttributeTok{overwrite=}\ConstantTok{TRUE}\NormalTok{,}
                   \AttributeTok{filename=}\StringTok{"./Geodata/2024/DEM/Terrain\_TWI\_udeni\_10m.tif"}\NormalTok{)}
\NormalTok{caurumains\_TWI}\OtherTok{=}\NormalTok{terra}\SpecialCharTok{::}\FunctionTok{rast}\NormalTok{(}\StringTok{"./Geodata/2024/DEM/Terrain\_TWI\_udeni\_10m.tif"}\NormalTok{)}
\NormalTok{caurumains\_TWI2}\OtherTok{=}\NormalTok{terra}\SpecialCharTok{::}\FunctionTok{mask}\NormalTok{(caurumains\_TWI,template,}
                            \AttributeTok{overwrite=}\ConstantTok{TRUE}\NormalTok{,}
                            \AttributeTok{filename=}\StringTok{"./RasterGrids\_10m/2024/Terrain\_TWI\_udeni2\_10m.tif"}\NormalTok{)}
\FunctionTok{rm}\NormalTok{(twi)}
\FunctionTok{rm}\NormalTok{(caurumi\_TWI)}
\FunctionTok{rm}\NormalTok{(caurumi\_TWI2)}
\FunctionTok{rm}\NormalTok{(caurumains\_TWI)}
\FunctionTok{rm}\NormalTok{(caurumains\_TWI2)}


\NormalTok{udeni}\SpecialCharTok{$}\NormalTok{disi}\OtherTok{=}\NormalTok{exactextractr}\SpecialCharTok{::}\FunctionTok{exact\_extract}\NormalTok{(dis,udeni,}\StringTok{"mean"}\NormalTok{)}
\NormalTok{caurumi\_DiS}\OtherTok{=}\NormalTok{fasterize}\SpecialCharTok{::}\FunctionTok{fasterize}\NormalTok{(udeni,templis,}\AttributeTok{field=}\StringTok{"disi"}\NormalTok{)}
\NormalTok{caurumi\_DiS2}\OtherTok{=}\FunctionTok{rast}\NormalTok{(caurumi\_DiS)}
\NormalTok{caurumains\_DiS}\OtherTok{=}\FunctionTok{app}\NormalTok{(}\FunctionTok{c}\NormalTok{(caurumi\_DiS2,dis),}\AttributeTok{fun=}\StringTok{"first"}\NormalTok{,}\AttributeTok{na.rm=}\ConstantTok{TRUE}\NormalTok{,}
                   \AttributeTok{overwrite=}\ConstantTok{TRUE}\NormalTok{,}
                   \AttributeTok{filename=}\StringTok{"./Geodata/2024/DEM/Terrain\_DiS\_udeni\_10m.tif"}\NormalTok{)}
\NormalTok{caurumains\_DiS}\OtherTok{=}\NormalTok{terra}\SpecialCharTok{::}\FunctionTok{rast}\NormalTok{(}\StringTok{"./Geodata/2024/DEM/Terrain\_DiS\_udeni\_10m.tif"}\NormalTok{)}
\NormalTok{caurumains\_DiS2}\OtherTok{=}\NormalTok{terra}\SpecialCharTok{::}\FunctionTok{mask}\NormalTok{(caurumains\_DiS,template,}
                            \AttributeTok{overwrite=}\ConstantTok{TRUE}\NormalTok{,}
                            \AttributeTok{filename=}\StringTok{"./RasterGrids\_10m/2024/Terrain\_DiS\_udeni2\_10m.tif"}\NormalTok{)}
\FunctionTok{rm}\NormalTok{(udeni)}
\FunctionTok{rm}\NormalTok{(dis)}
\FunctionTok{rm}\NormalTok{(caurumi\_DiS)}
\FunctionTok{rm}\NormalTok{(caurumi\_DiS2)}
\FunctionTok{rm}\NormalTok{(caurumains\_DiS)}
\FunctionTok{rm}\NormalTok{(caurumains\_DiS2)}


\CommentTok{\# cleaning}
\FunctionTok{unlink}\NormalTok{(}\StringTok{"./Geodata/2024/DEM/caurtDEM\_breachedF.tif"}\NormalTok{)}
\FunctionTok{unlink}\NormalTok{(}\StringTok{"./Geodata/2024/DEM/caurtDEM\_breachedNF.tif"}\NormalTok{)}
\FunctionTok{unlink}\NormalTok{(}\StringTok{"./Geodata/2024/DEM/caurtDEM\_BreachFill.tif"}\NormalTok{)}
\FunctionTok{unlink}\NormalTok{(}\StringTok{"./Geodata/2024/DEM/caurtDEM\_DInfAccu\_SCA.tif"}\NormalTok{)}

\FunctionTok{unlink}\NormalTok{(}\StringTok{"./Geodata/2024/DEM/Terrain\_Slope\_udeni\_10m.tif"}\NormalTok{)}
\FunctionTok{unlink}\NormalTok{(}\StringTok{"./Geodata/2024/DEM/Terrain\_Aspect\_udeni\_10m.tif"}\NormalTok{)}
\FunctionTok{unlink}\NormalTok{(}\StringTok{"./Geodata/2024/DEM/Terrain\_DiS\_udeni\_10m.tif"}\NormalTok{)}
\FunctionTok{unlink}\NormalTok{(}\StringTok{"./Geodata/2024/DEM/Terrain\_TWI\_udeni\_10m.tif"}\NormalTok{)}
\end{Highlighting}
\end{Shaded}

\section{Soil texture product}\label{Ch05.02}

In this section, a unified layer describing categorised soil texture (sand = 1,
silt = 2, clay = 3, organic = 4) was created from multiple preprocessed soil texture
data sources. The creation of the soil texture product consisted of multiple overlay steps.
These steps, along with the processed geodata used, are illustrated as follows:

\begin{enumerate}
\def\labelenumi{\arabic{enumi}.}
\item
  the base soil texture source was \hyperref[Ch04.07.02]{Soil texture layer from the European Soil Database}.
  This layer had to be reclassified to match the other layers, as this was not performed
  during preprocessing;
\item
  the layer from the first step was overlaid with the \hyperref[Ch04.07.04]{Latvian Quarternary geology data}
  coded as numeric starting with 1;
\item
  the layer from the second step was overlaid with the \hyperref[Ch04.07.03]{20th century topsoil data in Latvian farmland}
  coded as numeric starting with 1;
\item
  the layer from \hyperref[Ch04.07.05]{Organic soils as modelled by the LVMI Silava} (presence-only)
  was overlaid with the \hyperref[Ch04.07.06]{Organic soils as modelled by the University of Latvia}
  (presence-absence). After the overlay, it was classified as presence-only;
\item
  the layer from the third step was the overlaid with the layer from the fourth step and
  saved for EGV creation.
\end{enumerate}

\begin{Shaded}
\begin{Highlighting}[]
\CommentTok{\# libs {-}{-}{-}{-}}
\ControlFlowTok{if}\NormalTok{(}\SpecialCharTok{!}\FunctionTok{require}\NormalTok{(terra)) \{}\FunctionTok{install.packages}\NormalTok{(}\StringTok{"terra"}\NormalTok{); }\FunctionTok{require}\NormalTok{(terra)\}}

\CommentTok{\# step 1}
\NormalTok{step1}\OtherTok{=}\FunctionTok{rast}\NormalTok{(}\StringTok{"./RasterGrids\_10m/2024/SoilTXT\_ESDAC.tif"}\NormalTok{)}
\NormalTok{step1x}\OtherTok{=}\FunctionTok{ifel}\NormalTok{(step1}\SpecialCharTok{==}\DecValTok{1}\NormalTok{,}\DecValTok{1}\NormalTok{,}
            \FunctionTok{ifel}\NormalTok{(step1}\SpecialCharTok{==}\DecValTok{2}\NormalTok{,}\DecValTok{2}\NormalTok{,}
                 \FunctionTok{ifel}\NormalTok{(step1}\SpecialCharTok{==}\DecValTok{3}\NormalTok{,}\DecValTok{2}\NormalTok{,}
                      \FunctionTok{ifel}\NormalTok{(step1}\SpecialCharTok{==}\DecValTok{4}\NormalTok{,}\DecValTok{3}\NormalTok{,}
                           \FunctionTok{ifel}\NormalTok{(step1}\SpecialCharTok{==}\DecValTok{8}\NormalTok{,}\DecValTok{4}\NormalTok{,}\ConstantTok{NA}\NormalTok{)))))}
\FunctionTok{plot}\NormalTok{(step1x)}
\NormalTok{step1xy}\OtherTok{=}\FunctionTok{as.numeric}\NormalTok{(step1x)}
\FunctionTok{plot}\NormalTok{(step1xy)}


\CommentTok{\# step 2}
\NormalTok{step2a}\OtherTok{=}\FunctionTok{rast}\NormalTok{(}\StringTok{"./RasterGrids\_10m/2024/SoilTXT\_QuarternaryLV.tif"}\NormalTok{)}
\NormalTok{step2a}\OtherTok{=}\FunctionTok{as.numeric}\NormalTok{(step2a)}\SpecialCharTok{+}\DecValTok{1}
\FunctionTok{plot}\NormalTok{(step2a)}

\NormalTok{step2}\OtherTok{=}\FunctionTok{cover}\NormalTok{(step2a,step1x)}
\FunctionTok{plot}\NormalTok{(step2)}

\CommentTok{\# step 3}
\NormalTok{step3a}\OtherTok{=}\FunctionTok{rast}\NormalTok{(}\StringTok{"./RasterGrids\_10m/2024/SoilTXT\_topSoilLV.tif"}\NormalTok{)}
\NormalTok{step3a}\OtherTok{=}\FunctionTok{as.numeric}\NormalTok{(step3a)}\SpecialCharTok{+}\DecValTok{1}
\FunctionTok{plot}\NormalTok{(step3a)}

\NormalTok{step3}\OtherTok{=}\FunctionTok{cover}\NormalTok{(step3a,step2)}
\FunctionTok{plot}\NormalTok{(step3)}

\CommentTok{\# step 4}
\NormalTok{step4a}\OtherTok{=}\FunctionTok{rast}\NormalTok{(}\StringTok{"./RasterGrids\_10m/2024/SoilTXT\_OrganicLU.tif"}\NormalTok{)}
\NormalTok{step4b}\OtherTok{=}\FunctionTok{rast}\NormalTok{(}\StringTok{"./RasterGrids\_10m/2024/SoilTXT\_OrganicSilava.tif"}\NormalTok{)}

\NormalTok{step4c}\OtherTok{=}\FunctionTok{cover}\NormalTok{(step4a,step4b)}

\NormalTok{step4}\OtherTok{=}\FunctionTok{ifel}\NormalTok{(step4c}\SpecialCharTok{==}\DecValTok{1}\NormalTok{,}\DecValTok{4}\NormalTok{,}\ConstantTok{NA}\NormalTok{)}
\FunctionTok{plot}\NormalTok{(step4)}

\CommentTok{\# step 5}

\NormalTok{step5}\OtherTok{=}\FunctionTok{cover}\NormalTok{(step4,step3)}
\FunctionTok{plot}\NormalTok{(step5)}

\FunctionTok{writeRaster}\NormalTok{(step5,}
           \StringTok{"./RasterGrids\_10m/2024/SoilTXT\_combined.tif"}\NormalTok{,}
           \AttributeTok{overwrite=}\ConstantTok{TRUE}\NormalTok{)}
\end{Highlighting}
\end{Shaded}

\section{Landscape classification}\label{Ch05.03}

In this exercise, ``landscape'' refers to the representation of different types
of land cover and land use classes. The order in which these classes are
drawn is important because spatial data from different sources often have
mismatching boundaries. This requires addressing both their overlap (1) and
filling in gaps where no database information is available (2), as well as deciding
how to emphasize objects through certain processing steps, such as buffering. Some
elements that are important for characterizing the environment (especially
edge effects) may be so small or poorly positioned that they disappear during
the rasterisation process (3).

The general landscape layer also serves as a
mask for the preparation of further environmental descriptions. This section
describes the development of a general (simple) landscape and, in the following
document, its enrichment with more specific environmental ecogeographical
variables. The general landscape is stored in the file \texttt{Ainava\_vienk\_mask.tif}.
The classes in the order of overlay are as follow:

\begin{itemize}
\item
  Class \texttt{100} - Roads;
\item
  (Subclass \texttt{720} - Reed, Sedge, Rush beds;)
\item
  Class \texttt{200} - Waters;
\item
  Class \texttt{300} - Farmlands;
\item
  Class \texttt{400} - Allotment gardens, Orchards and Cottages;
\item
  Class \texttt{500} - Built-up;
\item
  Class \texttt{600} - Forests, Shrublands, Clearings;
\item
  Class \texttt{700} - Wetlands;
\item
  Class \texttt{800} - Bare Soil and Quarries.
\end{itemize}

The procedures for their creation are described below:

\begin{itemize}
\item
  Class \texttt{100} - \textbf{Roads}: roads from various sources. The following sources have been
  combined to create this class:

  \begin{itemize}
  \item
    layers \texttt{RoadA\_COMB} and \texttt{RoadL\_COMB} (except the smallest size groups) from
    \hyperref[Ch04.04]{topographic map}, buffered by 10 m before rasterisation;
  \item
    \hyperref[Ch04.06]{LVM open data} layers \texttt{LVM\_MEZA\_AUTOCELI}, \texttt{LVM\_ATTISTAMIE\_AUTOCELI},
    \texttt{LVM\_APGRIESANAS\_LAUKUMI}, \texttt{LVM\_IZMAINISANAS\_VIETAS}, and \texttt{LVM\_NOBRAUKTUVES}
    buffered by 10 m;
  \item
    information from the State Forest Register on unpaved forest tracks has not
    been used, as these roads do not usually form a continuous break in the canopy.
    Information on roads from this register is also available in other
    resources and has not been duplicated.
  \end{itemize}
\end{itemize}

The command lines below create a layer with landscape class \texttt{100}, which is
saved in the file \texttt{SimpleLandscape\_class100\_celi.tif} for further processing.

\begin{Shaded}
\begin{Highlighting}[]
\CommentTok{\# Libs {-}{-}{-}{-}}
\ControlFlowTok{if}\NormalTok{(}\SpecialCharTok{!}\FunctionTok{require}\NormalTok{(tidyverse)) \{}\FunctionTok{install.packages}\NormalTok{(}\StringTok{"tidyverse"}\NormalTok{); }\FunctionTok{require}\NormalTok{(tidyverse)\}}
\ControlFlowTok{if}\NormalTok{(}\SpecialCharTok{!}\FunctionTok{require}\NormalTok{(sf)) \{}\FunctionTok{install.packages}\NormalTok{(}\StringTok{"sf"}\NormalTok{); }\FunctionTok{require}\NormalTok{(sf)\}}
\ControlFlowTok{if}\NormalTok{(}\SpecialCharTok{!}\FunctionTok{require}\NormalTok{(arrow)) \{}\FunctionTok{install.packages}\NormalTok{(}\StringTok{"arrow"}\NormalTok{); }\FunctionTok{require}\NormalTok{(arrow)\}}
\ControlFlowTok{if}\NormalTok{(}\SpecialCharTok{!}\FunctionTok{require}\NormalTok{(sfarrow)) \{}\FunctionTok{install.packages}\NormalTok{(}\StringTok{"sfarrow"}\NormalTok{); }\FunctionTok{require}\NormalTok{(sfarrow)\}}
\ControlFlowTok{if}\NormalTok{(}\SpecialCharTok{!}\FunctionTok{require}\NormalTok{(terra)) \{}\FunctionTok{install.packages}\NormalTok{(}\StringTok{"terra"}\NormalTok{); }\FunctionTok{require}\NormalTok{(terra)\}}
\ControlFlowTok{if}\NormalTok{(}\SpecialCharTok{!}\FunctionTok{require}\NormalTok{(raster)) \{}\FunctionTok{install.packages}\NormalTok{(}\StringTok{"raster"}\NormalTok{); }\FunctionTok{require}\NormalTok{(raster)\}}
\ControlFlowTok{if}\NormalTok{(}\SpecialCharTok{!}\FunctionTok{require}\NormalTok{(fasterize)) \{}\FunctionTok{install.packages}\NormalTok{(}\StringTok{"fasterize"}\NormalTok{); }\FunctionTok{require}\NormalTok{(fasterize)\}}
\ControlFlowTok{if}\NormalTok{(}\SpecialCharTok{!}\FunctionTok{require}\NormalTok{(gdalUtilities))\{}\FunctionTok{install.packages}\NormalTok{(}\StringTok{"gdalUtilities"}\NormalTok{);}\FunctionTok{require}\NormalTok{(gdalUtilities)\}}
\ControlFlowTok{if}\NormalTok{(}\SpecialCharTok{!}\FunctionTok{require}\NormalTok{(readxl)) \{}\FunctionTok{install.packages}\NormalTok{(}\StringTok{"readxl"}\NormalTok{); }\FunctionTok{require}\NormalTok{(readxl)\}}

\CommentTok{\# templates {-}{-}{-}{-}}
\NormalTok{template\_t}\OtherTok{=}\FunctionTok{rast}\NormalTok{(}\StringTok{"./Templates/TemplateRasters/LV10m\_10km.tif"}\NormalTok{)}
\NormalTok{template\_r}\OtherTok{=}\FunctionTok{raster}\NormalTok{(template\_t)}


\CommentTok{\# class 100 {-}{-}{-}{-}}

\CommentTok{\#poly}
\NormalTok{celi\_topo}\OtherTok{=}\FunctionTok{st\_read\_parquet}\NormalTok{(}\StringTok{"./Geodata/2024/TopographicMap/RoadA\_COMB.parquet"}\NormalTok{)}
\NormalTok{celi\_topo}\OtherTok{=}\NormalTok{celi\_topo }\SpecialCharTok{\%\textgreater{}\%} 
  \FunctionTok{mutate}\NormalTok{(}\AttributeTok{yes=}\DecValTok{100}\NormalTok{) }\SpecialCharTok{\%\textgreater{}\%} 
\NormalTok{  dplyr}\SpecialCharTok{::}\FunctionTok{select}\NormalTok{(yes)}
\NormalTok{ctb}\OtherTok{=}\FunctionTok{st\_buffer}\NormalTok{(celi\_topo,}\AttributeTok{dist=}\DecValTok{10}\NormalTok{)}
\NormalTok{r\_celi\_topo}\OtherTok{=}\FunctionTok{fasterize}\NormalTok{(ctb,template\_r,}\AttributeTok{field=}\StringTok{"yes"}\NormalTok{)}

\CommentTok{\# pts}
\NormalTok{nobrauktuves}\OtherTok{=}\FunctionTok{st\_read}\NormalTok{(}\StringTok{"./Geodata/2024/LVM\_OpenData/LVM\_NOBRAUKTUVES/LVM\_NOBRAUKTUVES\_Shape.shp"}\NormalTok{)}
\NormalTok{nobrauktuves}\OtherTok{=}\NormalTok{nobrauktuves }\SpecialCharTok{\%\textgreater{}\%} 
  \FunctionTok{mutate}\NormalTok{(}\AttributeTok{yes=}\DecValTok{100}\NormalTok{) }\SpecialCharTok{\%\textgreater{}\%} 
\NormalTok{  dplyr}\SpecialCharTok{::}\FunctionTok{select}\NormalTok{(yes)}
\NormalTok{izmainisanas}\OtherTok{=}\FunctionTok{st\_read}\NormalTok{(}\StringTok{"./Geodata/2024/LVM\_OpenData/LVM\_IZMAINISANAS\_VIETAS/LVM\_IZMAINISANAS\_VIETAS\_Shape.shp"}\NormalTok{)}
\NormalTok{izmainisanas}\OtherTok{=}\NormalTok{izmainisanas }\SpecialCharTok{\%\textgreater{}\%} 
  \FunctionTok{mutate}\NormalTok{(}\AttributeTok{yes=}\DecValTok{100}\NormalTok{) }\SpecialCharTok{\%\textgreater{}\%} 
\NormalTok{  dplyr}\SpecialCharTok{::}\FunctionTok{select}\NormalTok{(yes)}
\NormalTok{apgriesanas}\OtherTok{=}\FunctionTok{st\_read}\NormalTok{(}\StringTok{"./Geodata/2024/LVM\_OpenData/LVM\_APGRIESANAS\_LAUKUMI/LVM\_APGRIESANAS\_LAUKUMI\_Shape.shp"}\NormalTok{)}
\NormalTok{apgriesanas}\OtherTok{=}\NormalTok{apgriesanas }\SpecialCharTok{\%\textgreater{}\%} 
  \FunctionTok{mutate}\NormalTok{(}\AttributeTok{yes=}\DecValTok{100}\NormalTok{) }\SpecialCharTok{\%\textgreater{}\%} 
\NormalTok{  dplyr}\SpecialCharTok{::}\FunctionTok{select}\NormalTok{(yes)}
\NormalTok{cp}\OtherTok{=}\FunctionTok{rbind}\NormalTok{(nobrauktuves,izmainisanas,apgriesanas)}
\NormalTok{cpb}\OtherTok{=}\FunctionTok{st\_buffer}\NormalTok{(cp,}\AttributeTok{dist=}\DecValTok{10}\NormalTok{)}
\NormalTok{r\_celi\_pts}\OtherTok{=}\FunctionTok{fasterize}\NormalTok{(cpb,template\_r,}\AttributeTok{field=}\StringTok{"yes"}\NormalTok{)}


\CommentTok{\# lines}
\NormalTok{meza\_autoceli}\OtherTok{=}\FunctionTok{st\_read}\NormalTok{(}\StringTok{"./Geodata/2024/LVM\_OpenData/LVM\_MEZA\_AUTOCELI/LVM\_MEZA\_AUTOCELI\_Shape.shp"}\NormalTok{)}
\NormalTok{meza\_autoceli}\OtherTok{=}\NormalTok{meza\_autoceli }\SpecialCharTok{\%\textgreater{}\%} 
  \FunctionTok{mutate}\NormalTok{(}\AttributeTok{yes=}\DecValTok{100}\NormalTok{) }\SpecialCharTok{\%\textgreater{}\%} 
\NormalTok{  dplyr}\SpecialCharTok{::}\FunctionTok{select}\NormalTok{(yes)}
\NormalTok{attistamie}\OtherTok{=}\FunctionTok{st\_read}\NormalTok{(}\StringTok{"./Geodata/2024/LVM\_OpenData/LVM\_ATTISTAMIE\_AUTOCELI/LVM\_ATTISTAMIE\_AUTOCELI\_Shape.shp"}\NormalTok{)}
\NormalTok{attistamie}\OtherTok{=}\NormalTok{attistamie }\SpecialCharTok{\%\textgreater{}\%} 
  \FunctionTok{mutate}\NormalTok{(}\AttributeTok{yes=}\DecValTok{100}\NormalTok{) }\SpecialCharTok{\%\textgreater{}\%} 
\NormalTok{  dplyr}\SpecialCharTok{::}\FunctionTok{select}\NormalTok{(yes)}
\NormalTok{topo\_lines}\OtherTok{=}\FunctionTok{st\_read\_parquet}\NormalTok{(}\StringTok{"./Geodata/2024/TopographicMap/RoadL\_COMB.parquet"}\NormalTok{)}
\NormalTok{topo\_lines}\OtherTok{=}\NormalTok{topo\_lines }\SpecialCharTok{\%\textgreater{}\%} 
  \FunctionTok{mutate}\NormalTok{(}\AttributeTok{yes=}\DecValTok{100}\NormalTok{) }\SpecialCharTok{\%\textgreater{}\%} 
\NormalTok{  dplyr}\SpecialCharTok{::}\FunctionTok{select}\NormalTok{(yes)}
\NormalTok{cl}\OtherTok{=}\FunctionTok{bind\_rows}\NormalTok{(meza\_autoceli,attistamie,topo\_lines)}
\NormalTok{cl}\OtherTok{=}\NormalTok{cl }\SpecialCharTok{\%\textgreater{}\%} 
\NormalTok{  dplyr}\SpecialCharTok{::}\FunctionTok{select}\NormalTok{(yes)}
\NormalTok{clb}\OtherTok{=}\FunctionTok{st\_buffer}\NormalTok{(cl,}\AttributeTok{dist=}\DecValTok{10}\NormalTok{)}
\NormalTok{r\_celi\_lines}\OtherTok{=}\FunctionTok{fasterize}\NormalTok{(clb,template\_r,}\AttributeTok{field=}\StringTok{"yes"}\NormalTok{)}

\CommentTok{\# cleaning}
\FunctionTok{rm}\NormalTok{(apgriesanas)}
\FunctionTok{rm}\NormalTok{(attistamie)}
\FunctionTok{rm}\NormalTok{(celi\_topo)}
\FunctionTok{rm}\NormalTok{(topo\_lines)}
\FunctionTok{rm}\NormalTok{(ctb)}
\FunctionTok{rm}\NormalTok{(cl)}
\FunctionTok{rm}\NormalTok{(clb)}
\FunctionTok{rm}\NormalTok{(cp)}
\FunctionTok{rm}\NormalTok{(cpb)}
\FunctionTok{rm}\NormalTok{(izmainisanas)}
\FunctionTok{rm}\NormalTok{(meza\_autoceli)}
\FunctionTok{rm}\NormalTok{(nobrauktuves)}

\CommentTok{\# to terra}
\NormalTok{t\_celi\_topo}\OtherTok{=}\FunctionTok{rast}\NormalTok{(r\_celi\_topo)}
\NormalTok{t\_celi\_pts}\OtherTok{=}\FunctionTok{rast}\NormalTok{(r\_celi\_pts)}
\NormalTok{t\_celi\_lines}\OtherTok{=}\FunctionTok{rast}\NormalTok{(r\_celi\_lines)}

\CommentTok{\# cleaning}
\FunctionTok{rm}\NormalTok{(r\_celi\_lines)}
\FunctionTok{rm}\NormalTok{(r\_celi\_pts)}
\FunctionTok{rm}\NormalTok{(r\_celi\_topo)}

\CommentTok{\# union}
\FunctionTok{plot}\NormalTok{(t\_celi\_topo)}

\NormalTok{road\_union1}\OtherTok{=}\FunctionTok{cover}\NormalTok{(t\_celi\_topo,t\_celi\_pts)}
\NormalTok{road\_union2}\OtherTok{=}\FunctionTok{cover}\NormalTok{(road\_union1,t\_celi\_lines,}
                  \AttributeTok{filename=}\StringTok{"./RasterGrids\_10m/2024/SimpleLandscape\_class100\_celi.tif"}\NormalTok{,}
                  \AttributeTok{overwrite=}\ConstantTok{TRUE}\NormalTok{)}

\CommentTok{\# cleaning}
\FunctionTok{rm}\NormalTok{(t\_celi\_topo)}
\FunctionTok{rm}\NormalTok{(t\_celi\_pts)}
\FunctionTok{rm}\NormalTok{(t\_celi\_lines)}
\FunctionTok{rm}\NormalTok{(road\_union1)}
\FunctionTok{rm}\NormalTok{(road\_union2)}
\end{Highlighting}
\end{Shaded}

\begin{itemize}
\item
  Class \texttt{200} - \textbf{Waters}: water bodies from various sources. The following are
  combined to create this class:

  -- \hyperref[Ch04.04]{topographic map} layers \texttt{HidroA\_COMB} and \texttt{HidroL\_COMB} (buffered by 5 m);

  -- \hyperref[Ch04.03]{MKIS} layer \texttt{Gravji}, buffered by 3 m;

  -- \hyperref[Ch04.06]{LVM open data} layers \texttt{LVM\_GRAVJI}, buffered by 5 m.

  -- information about ditches from the State Forest Register was not used,
  as it is either already available in other resources or consists of structures
  so small that they do not cause a continuous break in the tree canopy.
\end{itemize}

The command lines below create a layer with landscape class \texttt{200}, which is
saved in the file \texttt{SimpleLandscape\_class200\_udens\_premask.tif} for further processing.

\begin{Shaded}
\begin{Highlighting}[]
\CommentTok{\# Libs {-}{-}{-}{-}}
\ControlFlowTok{if}\NormalTok{(}\SpecialCharTok{!}\FunctionTok{require}\NormalTok{(tidyverse)) \{}\FunctionTok{install.packages}\NormalTok{(}\StringTok{"tidyverse"}\NormalTok{); }\FunctionTok{require}\NormalTok{(tidyverse)\}}
\ControlFlowTok{if}\NormalTok{(}\SpecialCharTok{!}\FunctionTok{require}\NormalTok{(sf)) \{}\FunctionTok{install.packages}\NormalTok{(}\StringTok{"sf"}\NormalTok{); }\FunctionTok{require}\NormalTok{(sf)\}}
\ControlFlowTok{if}\NormalTok{(}\SpecialCharTok{!}\FunctionTok{require}\NormalTok{(arrow)) \{}\FunctionTok{install.packages}\NormalTok{(}\StringTok{"arrow"}\NormalTok{); }\FunctionTok{require}\NormalTok{(arrow)\}}
\ControlFlowTok{if}\NormalTok{(}\SpecialCharTok{!}\FunctionTok{require}\NormalTok{(sfarrow)) \{}\FunctionTok{install.packages}\NormalTok{(}\StringTok{"sfarrow"}\NormalTok{); }\FunctionTok{require}\NormalTok{(sfarrow)\}}
\ControlFlowTok{if}\NormalTok{(}\SpecialCharTok{!}\FunctionTok{require}\NormalTok{(terra)) \{}\FunctionTok{install.packages}\NormalTok{(}\StringTok{"terra"}\NormalTok{); }\FunctionTok{require}\NormalTok{(terra)\}}
\ControlFlowTok{if}\NormalTok{(}\SpecialCharTok{!}\FunctionTok{require}\NormalTok{(raster)) \{}\FunctionTok{install.packages}\NormalTok{(}\StringTok{"raster"}\NormalTok{); }\FunctionTok{require}\NormalTok{(raster)\}}
\ControlFlowTok{if}\NormalTok{(}\SpecialCharTok{!}\FunctionTok{require}\NormalTok{(fasterize)) \{}\FunctionTok{install.packages}\NormalTok{(}\StringTok{"fasterize"}\NormalTok{); }\FunctionTok{require}\NormalTok{(fasterize)\}}
\ControlFlowTok{if}\NormalTok{(}\SpecialCharTok{!}\FunctionTok{require}\NormalTok{(gdalUtilities))\{}\FunctionTok{install.packages}\NormalTok{(}\StringTok{"gdalUtilities"}\NormalTok{);}\FunctionTok{require}\NormalTok{(gdalUtilities)\}}
\ControlFlowTok{if}\NormalTok{(}\SpecialCharTok{!}\FunctionTok{require}\NormalTok{(readxl)) \{}\FunctionTok{install.packages}\NormalTok{(}\StringTok{"readxl"}\NormalTok{); }\FunctionTok{require}\NormalTok{(readxl)\}}

\CommentTok{\# templates {-}{-}{-}{-}}
\NormalTok{template\_t}\OtherTok{=}\FunctionTok{rast}\NormalTok{(}\StringTok{"./Templates/TemplateRasters/LV10m\_10km.tif"}\NormalTok{)}
\NormalTok{template\_r}\OtherTok{=}\FunctionTok{raster}\NormalTok{(template\_t)}


\CommentTok{\# class 200 {-}{-}{-}{-}}

\CommentTok{\# topo}
\NormalTok{topo\_udens\_poly}\OtherTok{=}\FunctionTok{st\_read\_parquet}\NormalTok{(}\StringTok{"./Geodata/2024/TopographicMap/HidroA\_COMB.parquet"}\NormalTok{)}
\NormalTok{topo\_udens\_poly}\OtherTok{=}\NormalTok{topo\_udens\_poly }\SpecialCharTok{\%\textgreater{}\%} 
  \FunctionTok{mutate}\NormalTok{(}\AttributeTok{yes=}\DecValTok{200}\NormalTok{) }\SpecialCharTok{\%\textgreater{}\%} 
\NormalTok{  dplyr}\SpecialCharTok{::}\FunctionTok{select}\NormalTok{(yes) }\SpecialCharTok{\%\textgreater{}\%} 
  \FunctionTok{st\_transform}\NormalTok{(}\AttributeTok{crs=}\DecValTok{3059}\NormalTok{)}
\NormalTok{topo\_udens\_lines}\OtherTok{=}\FunctionTok{st\_read\_parquet}\NormalTok{(}\StringTok{"./Geodata/2024/TopographicMap/HidroL\_COMB.parquet"}\NormalTok{)}
\NormalTok{topo\_udens\_lines}\OtherTok{=}\NormalTok{topo\_udens\_lines }\SpecialCharTok{\%\textgreater{}\%} 
  \FunctionTok{mutate}\NormalTok{(}\AttributeTok{yes=}\DecValTok{200}\NormalTok{) }\SpecialCharTok{\%\textgreater{}\%} 
  \FunctionTok{st\_buffer}\NormalTok{(}\AttributeTok{dist=}\DecValTok{5}\NormalTok{) }\SpecialCharTok{\%\textgreater{}\%} 
\NormalTok{  dplyr}\SpecialCharTok{::}\FunctionTok{select}\NormalTok{(yes) }\SpecialCharTok{\%\textgreater{}\%} 
  \FunctionTok{st\_transform}\NormalTok{(}\AttributeTok{crs=}\DecValTok{3059}\NormalTok{)}
\NormalTok{topo\_udens}\OtherTok{=}\FunctionTok{rbind}\NormalTok{(topo\_udens\_poly,topo\_udens\_lines)}
\NormalTok{r\_topo\_udens}\OtherTok{=}\FunctionTok{fasterize}\NormalTok{(topo\_udens,template\_r,}\AttributeTok{field=}\StringTok{"yes"}\NormalTok{)}
\NormalTok{raster}\SpecialCharTok{::}\FunctionTok{writeRaster}\NormalTok{(r\_topo\_udens,}
                    \StringTok{"./RasterGrids\_10m/2024/SimpleLandscape\_class200\_topo.tif"}\NormalTok{,}
                    \AttributeTok{progress=}\StringTok{"text"}\NormalTok{)}
\CommentTok{\# cleaning}
\FunctionTok{rm}\NormalTok{(topo\_udens\_lines)}
\FunctionTok{rm}\NormalTok{(topo\_udens\_poly)}
\FunctionTok{rm}\NormalTok{(topo\_udens)}
\FunctionTok{rm}\NormalTok{(r\_topo\_udens)}

\CommentTok{\# mkis}
\FunctionTok{st\_layers}\NormalTok{(}\StringTok{"./Geodata/2024/MKIS/MKIS\_2025.gpkg"}\NormalTok{)}
\NormalTok{mkis\_gravji}\OtherTok{=}\FunctionTok{st\_read}\NormalTok{(}\StringTok{"./Geodata/2024/MKIS/MKIS\_2025.gpkg"}\NormalTok{,}\AttributeTok{layer=}\StringTok{"Gravji"}\NormalTok{)}

\NormalTok{mkis\_gravji}\OtherTok{=}\NormalTok{mkis\_gravji }\SpecialCharTok{\%\textgreater{}\%} 
  \FunctionTok{mutate}\NormalTok{(}\AttributeTok{yes=}\DecValTok{200}\NormalTok{) }\SpecialCharTok{\%\textgreater{}\%} 
  \FunctionTok{st\_buffer}\NormalTok{(}\AttributeTok{dist=}\DecValTok{3}\NormalTok{) }\SpecialCharTok{\%\textgreater{}\%} 
\NormalTok{  dplyr}\SpecialCharTok{::}\FunctionTok{select}\NormalTok{(yes)}
\NormalTok{r\_mkis\_udens}\OtherTok{=}\FunctionTok{fasterize}\NormalTok{(mkis\_gravji,template\_r,}\AttributeTok{field=}\StringTok{"yes"}\NormalTok{)}
\NormalTok{raster}\SpecialCharTok{::}\FunctionTok{writeRaster}\NormalTok{(r\_mkis\_udens,}
                    \StringTok{"./RasterGrids\_10m/2024/SimpleLandscape\_class200\_mkis.tif"}\NormalTok{,}
                    \AttributeTok{progress=}\StringTok{"text"}\NormalTok{)}
\CommentTok{\# cleaning}
\FunctionTok{rm}\NormalTok{(mkis\_gravji)}
\FunctionTok{rm}\NormalTok{(mkis\_gravji2)}
\FunctionTok{rm}\NormalTok{(mkis\_gravji3)}
\FunctionTok{rm}\NormalTok{(r\_mkis\_udens)}

\CommentTok{\# lvm}
\NormalTok{lvm\_gravji}\OtherTok{=}\FunctionTok{st\_read}\NormalTok{(}\StringTok{"./Geodata/2024/LVM\_OpenData/LVM\_GRAVJI/LVM\_GRAVJI\_Shape.shp"}\NormalTok{)}
\NormalTok{lvm\_gravji}\OtherTok{=}\NormalTok{lvm\_gravji }\SpecialCharTok{\%\textgreater{}\%} 
  \FunctionTok{mutate}\NormalTok{(}\AttributeTok{yes=}\DecValTok{200}\NormalTok{) }\SpecialCharTok{\%\textgreater{}\%} 
  \FunctionTok{st\_buffer}\NormalTok{(}\AttributeTok{dist=}\DecValTok{5}\NormalTok{) }\SpecialCharTok{\%\textgreater{}\%} 
\NormalTok{  dplyr}\SpecialCharTok{::}\FunctionTok{select}\NormalTok{(yes)}
\NormalTok{r\_lvm\_gravji}\OtherTok{=}\FunctionTok{fasterize}\NormalTok{(lvm\_gravji,template\_r,}\AttributeTok{field=}\StringTok{"yes"}\NormalTok{)}
\NormalTok{raster}\SpecialCharTok{::}\FunctionTok{writeRaster}\NormalTok{(r\_lvm\_gravji,}
                    \StringTok{"./RasterGrids\_10m/2024/SimpleLandscape\_class200\_lvm.tif"}\NormalTok{,}
                    \AttributeTok{progress=}\StringTok{"text"}\NormalTok{,}
                    \AttributeTok{overwrite=}\ConstantTok{TRUE}\NormalTok{)}
\CommentTok{\# cleaning}
\FunctionTok{rm}\NormalTok{(lvm\_gravji)}
\FunctionTok{rm}\NormalTok{(r\_lvm\_gravji)}


\CommentTok{\# merging}
\NormalTok{a200}\OtherTok{=}\FunctionTok{rast}\NormalTok{(}\StringTok{"./RasterGrids\_10m/2024/SimpleLandscape\_class200\_topo.tif"}\NormalTok{)}
\NormalTok{b200}\OtherTok{=}\FunctionTok{rast}\NormalTok{(}\StringTok{"./RasterGrids\_10m/2024/SimpleLandscape\_class200\_mkis.tif"}\NormalTok{)}
\NormalTok{c200}\OtherTok{=}\FunctionTok{rast}\NormalTok{(}\StringTok{"./RasterGrids\_10m/2024/SimpleLandscape\_class200\_lvm.tif"}\NormalTok{)}

\NormalTok{udens\_cover1}\OtherTok{=}\FunctionTok{cover}\NormalTok{(a200,b200)}
\NormalTok{udens\_cover2}\OtherTok{=}\FunctionTok{cover}\NormalTok{(udens\_cover1,c200,}
                   \AttributeTok{filename=}\StringTok{"./RasterGrids\_10m/2024/SimpleLandscape\_class200\_udens\_premask.tif"}\NormalTok{,}
                   \AttributeTok{overwrite=}\ConstantTok{TRUE}\NormalTok{)}

\CommentTok{\# cleaning}
\FunctionTok{rm}\NormalTok{(a200)}
\FunctionTok{rm}\NormalTok{(b200)}
\FunctionTok{rm}\NormalTok{(c200)}
\FunctionTok{rm}\NormalTok{(udens\_cover1)}
\FunctionTok{rm}\NormalTok{(udens\_cover2)}
\FunctionTok{unlink}\NormalTok{(}\StringTok{"./RasterGrids\_10m/2024/SimpleLandscape\_class200\_topo.tif"}\NormalTok{)}
\FunctionTok{unlink}\NormalTok{(}\StringTok{"./RasterGrids\_10m/2024/SimpleLandscape\_class200\_mkis.tif"}\NormalTok{)}
\FunctionTok{unlink}\NormalTok{(}\StringTok{"./RasterGrids\_10m/2024/SimpleLandscape\_class200\_lvm.tif"}\NormalTok{)}
\end{Highlighting}
\end{Shaded}

\begin{itemize}
\item
  Class \texttt{300} - \textbf{Farmland}: agricultural land from the LAD database. The following
  sources are combined to create this class:

  -- \hyperref[Ch04.02]{LAD database}, which, following the decision on grouping (classes are
  available \href{https://github.com/aavotins/HiQBioDiv_EGVs/blob/main/Data/Geodata/2024/LAD/KulturuKodi_2024.xlsx}{here}),
  is divided into three broad groups (in the order of overlap with lower
  number dominating):

\begin{verbatim}
  – arable land with class code `310`;

  – fallow land with class code `320`;

  – grassland with class code `330`;

  – orchards and perennial shrub plantations in the general landscape are 
  part of other landscape classes.
\end{verbatim}
\end{itemize}

The command lines below create a layer with landscape class \texttt{300} and its
subclasses, which are saved in the file \texttt{SimpleLandscape\_class300\_lauki\_premask.tif}
for further processing.

\begin{Shaded}
\begin{Highlighting}[]
\CommentTok{\# Libs {-}{-}{-}{-}}
\ControlFlowTok{if}\NormalTok{(}\SpecialCharTok{!}\FunctionTok{require}\NormalTok{(tidyverse)) \{}\FunctionTok{install.packages}\NormalTok{(}\StringTok{"tidyverse"}\NormalTok{); }\FunctionTok{require}\NormalTok{(tidyverse)\}}
\ControlFlowTok{if}\NormalTok{(}\SpecialCharTok{!}\FunctionTok{require}\NormalTok{(sf)) \{}\FunctionTok{install.packages}\NormalTok{(}\StringTok{"sf"}\NormalTok{); }\FunctionTok{require}\NormalTok{(sf)\}}
\ControlFlowTok{if}\NormalTok{(}\SpecialCharTok{!}\FunctionTok{require}\NormalTok{(arrow)) \{}\FunctionTok{install.packages}\NormalTok{(}\StringTok{"arrow"}\NormalTok{); }\FunctionTok{require}\NormalTok{(arrow)\}}
\ControlFlowTok{if}\NormalTok{(}\SpecialCharTok{!}\FunctionTok{require}\NormalTok{(sfarrow)) \{}\FunctionTok{install.packages}\NormalTok{(}\StringTok{"sfarrow"}\NormalTok{); }\FunctionTok{require}\NormalTok{(sfarrow)\}}
\ControlFlowTok{if}\NormalTok{(}\SpecialCharTok{!}\FunctionTok{require}\NormalTok{(terra)) \{}\FunctionTok{install.packages}\NormalTok{(}\StringTok{"terra"}\NormalTok{); }\FunctionTok{require}\NormalTok{(terra)\}}
\ControlFlowTok{if}\NormalTok{(}\SpecialCharTok{!}\FunctionTok{require}\NormalTok{(raster)) \{}\FunctionTok{install.packages}\NormalTok{(}\StringTok{"raster"}\NormalTok{); }\FunctionTok{require}\NormalTok{(raster)\}}
\ControlFlowTok{if}\NormalTok{(}\SpecialCharTok{!}\FunctionTok{require}\NormalTok{(fasterize)) \{}\FunctionTok{install.packages}\NormalTok{(}\StringTok{"fasterize"}\NormalTok{); }\FunctionTok{require}\NormalTok{(fasterize)\}}
\ControlFlowTok{if}\NormalTok{(}\SpecialCharTok{!}\FunctionTok{require}\NormalTok{(gdalUtilities))\{}\FunctionTok{install.packages}\NormalTok{(}\StringTok{"gdalUtilities"}\NormalTok{);}\FunctionTok{require}\NormalTok{(gdalUtilities)\}}
\ControlFlowTok{if}\NormalTok{(}\SpecialCharTok{!}\FunctionTok{require}\NormalTok{(readxl)) \{}\FunctionTok{install.packages}\NormalTok{(}\StringTok{"readxl"}\NormalTok{); }\FunctionTok{require}\NormalTok{(readxl)\}}

\CommentTok{\# templates {-}{-}{-}{-}}
\NormalTok{template\_t}\OtherTok{=}\FunctionTok{rast}\NormalTok{(}\StringTok{"./Templates/TemplateRasters/LV10m\_10km.tif"}\NormalTok{)}
\NormalTok{template\_r}\OtherTok{=}\FunctionTok{raster}\NormalTok{(template\_t)}


\CommentTok{\# class 300 {-}{-}{-}{-}}

\CommentTok{\# lad}
\NormalTok{lad\_klasem}\OtherTok{=}\FunctionTok{read\_excel}\NormalTok{(}\StringTok{"./Geodata/2024/LAD/KulturuKodi\_2024.xlsx"}\NormalTok{)}
\NormalTok{lad}\OtherTok{=}\FunctionTok{st\_read\_parquet}\NormalTok{(}\StringTok{"./Geodata/2024/LAD/Lauki\_2024.parquet"}\NormalTok{)}


\DocumentationTok{\#\# arable}
\NormalTok{amazemem}\OtherTok{=}\NormalTok{lad\_klasem }\SpecialCharTok{\%\textgreater{}\%} 
  \FunctionTok{filter}\NormalTok{(}\FunctionTok{str\_detect}\NormalTok{(SDM\_grupa\_sakums,}\StringTok{"aramz"}\NormalTok{))}
\NormalTok{aramzemes}\OtherTok{=}\NormalTok{lad }\SpecialCharTok{\%\textgreater{}\%} 
  \FunctionTok{filter}\NormalTok{(PRODUCT\_CODE }\SpecialCharTok{\%in\%}\NormalTok{ amazemem}\SpecialCharTok{$}\NormalTok{kods) }\SpecialCharTok{\%\textgreater{}\%} 
  \FunctionTok{mutate}\NormalTok{(}\AttributeTok{yes=}\DecValTok{310}\NormalTok{) }\SpecialCharTok{\%\textgreater{}\%} 
\NormalTok{  dplyr}\SpecialCharTok{::}\FunctionTok{select}\NormalTok{(yes)}
\NormalTok{r\_aramzemes\_lad}\OtherTok{=}\FunctionTok{fasterize}\NormalTok{(aramzemes,template\_r,}\AttributeTok{field=}\StringTok{"yes"}\NormalTok{)}
\NormalTok{raster}\SpecialCharTok{::}\FunctionTok{writeRaster}\NormalTok{(r\_aramzemes\_lad,}
                    \StringTok{"./RasterGrids\_10m/2024/SimpleLandscape\_class310\_aramzemes\_lad.tif"}\NormalTok{,}
                    \AttributeTok{progress=}\StringTok{"text"}\NormalTok{,}
                    \AttributeTok{overwrite=}\ConstantTok{TRUE}\NormalTok{)}
\CommentTok{\# cleaning}
\FunctionTok{rm}\NormalTok{(amazemem)}
\FunctionTok{rm}\NormalTok{(aramzemes)}
\FunctionTok{rm}\NormalTok{(r\_aramzemes\_lad)}


\DocumentationTok{\#\# fallow}
\NormalTok{papuvem}\OtherTok{=}\NormalTok{lad\_klasem }\SpecialCharTok{\%\textgreater{}\%} 
  \FunctionTok{filter}\NormalTok{(}\FunctionTok{str\_detect}\NormalTok{(SDM\_grupa\_sakums,}\StringTok{"papuv"}\NormalTok{))}
\NormalTok{papuves}\OtherTok{=}\NormalTok{lad }\SpecialCharTok{\%\textgreater{}\%} 
  \FunctionTok{filter}\NormalTok{(PRODUCT\_CODE }\SpecialCharTok{\%in\%}\NormalTok{ papuvem}\SpecialCharTok{$}\NormalTok{kods) }\SpecialCharTok{\%\textgreater{}\%} 
  \FunctionTok{mutate}\NormalTok{(}\AttributeTok{yes=}\DecValTok{320}\NormalTok{) }\SpecialCharTok{\%\textgreater{}\%} 
\NormalTok{  dplyr}\SpecialCharTok{::}\FunctionTok{select}\NormalTok{(yes)}
\NormalTok{r\_papuves\_lad}\OtherTok{=}\FunctionTok{fasterize}\NormalTok{(papuves,template\_r,}\AttributeTok{field=}\StringTok{"yes"}\NormalTok{)}
\NormalTok{raster}\SpecialCharTok{::}\FunctionTok{writeRaster}\NormalTok{(r\_papuves\_lad,}
                    \StringTok{"./RasterGrids\_10m/2024/SimpleLandscape\_class320\_papuves\_lad.tif"}\NormalTok{,}
                    \AttributeTok{progress=}\StringTok{"text"}\NormalTok{,}
                    \AttributeTok{overwrite=}\ConstantTok{TRUE}\NormalTok{)}
\CommentTok{\# cleaning}
\FunctionTok{rm}\NormalTok{(papuvem)}
\FunctionTok{rm}\NormalTok{(papuves)}
\FunctionTok{rm}\NormalTok{(r\_papuves\_lad)}

\DocumentationTok{\#\# grassland}
\NormalTok{zalajiem}\OtherTok{=}\NormalTok{lad\_klasem }\SpecialCharTok{\%\textgreater{}\%} 
  \FunctionTok{filter}\NormalTok{(}\FunctionTok{str\_detect}\NormalTok{(SDM\_grupa\_sakums,}\StringTok{"zālā"}\NormalTok{))}
\NormalTok{zalaji}\OtherTok{=}\NormalTok{lad }\SpecialCharTok{\%\textgreater{}\%} 
  \FunctionTok{filter}\NormalTok{(PRODUCT\_CODE }\SpecialCharTok{\%in\%}\NormalTok{ zalajiem}\SpecialCharTok{$}\NormalTok{kods) }\SpecialCharTok{\%\textgreater{}\%} 
  \FunctionTok{mutate}\NormalTok{(}\AttributeTok{yes=}\DecValTok{330}\NormalTok{) }\SpecialCharTok{\%\textgreater{}\%} 
\NormalTok{  dplyr}\SpecialCharTok{::}\FunctionTok{select}\NormalTok{(yes)}
\NormalTok{r\_zalaji\_lad}\OtherTok{=}\FunctionTok{fasterize}\NormalTok{(zalaji,template\_r,}\AttributeTok{field=}\StringTok{"yes"}\NormalTok{)}
\NormalTok{raster}\SpecialCharTok{::}\FunctionTok{writeRaster}\NormalTok{(r\_zalaji\_lad,}
                    \StringTok{"./RasterGrids\_10m/2024/SimpleLandscape\_class330\_zalaji\_lad.tif"}\NormalTok{,}
                    \AttributeTok{progress=}\StringTok{"text"}\NormalTok{,}
                    \AttributeTok{overwrite=}\ConstantTok{TRUE}\NormalTok{)}
\CommentTok{\# cleaning}
\FunctionTok{rm}\NormalTok{(zalajiem)}
\FunctionTok{rm}\NormalTok{(zalaji)}
\FunctionTok{rm}\NormalTok{(r\_zalaji\_lad)}

\CommentTok{\# merging}
\NormalTok{a300}\OtherTok{=}\FunctionTok{rast}\NormalTok{(}\StringTok{"./RasterGrids\_10m/2024/SimpleLandscape\_class310\_aramzemes\_lad.tif"}\NormalTok{)}
\NormalTok{b300}\OtherTok{=}\FunctionTok{rast}\NormalTok{(}\StringTok{"./RasterGrids\_10m/2024/SimpleLandscape\_class320\_papuves\_lad.tif"}\NormalTok{)}
\NormalTok{c300}\OtherTok{=}\FunctionTok{rast}\NormalTok{(}\StringTok{"./RasterGrids\_10m/2024/SimpleLandscape\_class330\_zalaji\_lad.tif"}\NormalTok{)}

\NormalTok{farmland\_cover1}\OtherTok{=}\FunctionTok{cover}\NormalTok{(a300,b300)}
\NormalTok{farmland\_cover2}\OtherTok{=}\FunctionTok{cover}\NormalTok{(farmland\_cover1,}
\NormalTok{                      c300,}
                      \AttributeTok{filename=}\FunctionTok{paste0}\NormalTok{(}\StringTok{"./RasterGrids\_10m/2024/"}\NormalTok{,}
                                      \StringTok{"SimpleLandscape\_class300\_lauki\_premask.tif"}\NormalTok{),}
                      \AttributeTok{overwrite=}\ConstantTok{TRUE}\NormalTok{)}
\CommentTok{\# cleaning}
\FunctionTok{rm}\NormalTok{(lad)}
\FunctionTok{rm}\NormalTok{(lad\_klasem)}
\FunctionTok{rm}\NormalTok{(a300)}
\FunctionTok{rm}\NormalTok{(b300)}
\FunctionTok{rm}\NormalTok{(c300)}
\FunctionTok{rm}\NormalTok{(farmland\_cover1)}
\FunctionTok{rm}\NormalTok{(farmland\_cover2)}
\FunctionTok{unlink}\NormalTok{(}\StringTok{"./RasterGrids\_10m/2024/SimpleLandscape\_class310\_aramzemes\_lad.tif"}\NormalTok{)}
\FunctionTok{unlink}\NormalTok{(}\StringTok{"./RasterGrids\_10m/2024/SimpleLandscape\_class320\_papuves\_lad.tif"}\NormalTok{)}
\FunctionTok{unlink}\NormalTok{(}\StringTok{"./RasterGrids\_10m/2024/SimpleLandscape\_class330\_zalaji\_lad.tif"}\NormalTok{)}
\end{Highlighting}
\end{Shaded}

\begin{itemize}
\item
  Class \texttt{400} - \textbf{Allotment Gardens, Orchards and Cottages}. To create this
  class, the following are combined (in order of overlap):

  -- \hyperref[Ch04.04]{topographic map} layer \texttt{BuildA\_v3} values: ``poligons\_Vasarnīcu\_apbūve'',
  ``poligons\_Viensētu\_apbūve'', coded as \texttt{410};

  -- \hyperref[Ch04.04]{topographic map} layer \texttt{LandusA\_COMB} values: ``poligons\_Augludarzs'',
  ``poligons\_Augļudārzs'', ``poligons\_Sakņudārzs'', ``poligons\_Ogulājs'', ``poligons\_Ogulajs'',
  ``poligons\_Saknudarzs'', coded as \texttt{420};

  -- \hyperref[Ch04.02]{LAD database} rural information layer group (classes are
  available \href{https://github.com/aavotins/HiQBioDiv_EGVs/blob/main/Data/Geodata/2024/LAD/KulturuKodi_2024.xlsx}{here})
  ``augļudārzi'', the result of which is coded as \texttt{420}.
\end{itemize}

The command lines below create a layer with landscape class \texttt{400}, which is saved
in the file \texttt{SimpleLandscape\_class400\_vasarnicas\_premask.tif} for further
processing.

\begin{Shaded}
\begin{Highlighting}[]
\CommentTok{\# Libs {-}{-}{-}{-}}
\ControlFlowTok{if}\NormalTok{(}\SpecialCharTok{!}\FunctionTok{require}\NormalTok{(tidyverse)) \{}\FunctionTok{install.packages}\NormalTok{(}\StringTok{"tidyverse"}\NormalTok{); }\FunctionTok{require}\NormalTok{(tidyverse)\}}
\ControlFlowTok{if}\NormalTok{(}\SpecialCharTok{!}\FunctionTok{require}\NormalTok{(sf)) \{}\FunctionTok{install.packages}\NormalTok{(}\StringTok{"sf"}\NormalTok{); }\FunctionTok{require}\NormalTok{(sf)\}}
\ControlFlowTok{if}\NormalTok{(}\SpecialCharTok{!}\FunctionTok{require}\NormalTok{(arrow)) \{}\FunctionTok{install.packages}\NormalTok{(}\StringTok{"arrow"}\NormalTok{); }\FunctionTok{require}\NormalTok{(arrow)\}}
\ControlFlowTok{if}\NormalTok{(}\SpecialCharTok{!}\FunctionTok{require}\NormalTok{(sfarrow)) \{}\FunctionTok{install.packages}\NormalTok{(}\StringTok{"sfarrow"}\NormalTok{); }\FunctionTok{require}\NormalTok{(sfarrow)\}}
\ControlFlowTok{if}\NormalTok{(}\SpecialCharTok{!}\FunctionTok{require}\NormalTok{(terra)) \{}\FunctionTok{install.packages}\NormalTok{(}\StringTok{"terra"}\NormalTok{); }\FunctionTok{require}\NormalTok{(terra)\}}
\ControlFlowTok{if}\NormalTok{(}\SpecialCharTok{!}\FunctionTok{require}\NormalTok{(raster)) \{}\FunctionTok{install.packages}\NormalTok{(}\StringTok{"raster"}\NormalTok{); }\FunctionTok{require}\NormalTok{(raster)\}}
\ControlFlowTok{if}\NormalTok{(}\SpecialCharTok{!}\FunctionTok{require}\NormalTok{(fasterize)) \{}\FunctionTok{install.packages}\NormalTok{(}\StringTok{"fasterize"}\NormalTok{); }\FunctionTok{require}\NormalTok{(fasterize)\}}
\ControlFlowTok{if}\NormalTok{(}\SpecialCharTok{!}\FunctionTok{require}\NormalTok{(readxl)) \{}\FunctionTok{install.packages}\NormalTok{(}\StringTok{"readxl"}\NormalTok{); }\FunctionTok{require}\NormalTok{(readxl)\}}

\CommentTok{\# templates {-}{-}{-}{-}}
\NormalTok{template\_t}\OtherTok{=}\FunctionTok{rast}\NormalTok{(}\StringTok{"./Templates/TemplateRasters/LV10m\_10km.tif"}\NormalTok{)}
\NormalTok{template\_r}\OtherTok{=}\FunctionTok{raster}\NormalTok{(template\_t)}


\CommentTok{\# class 400 {-}{-}{-}{-}}


\CommentTok{\# topo built{-}up}
\NormalTok{viensvasar}\OtherTok{=}\FunctionTok{st\_read\_parquet}\NormalTok{(}\StringTok{"./Geodata/2024/TopographicMap/BuildA\_v3.parquet"}\NormalTok{)}
\FunctionTok{table}\NormalTok{(viensvasar}\SpecialCharTok{$}\NormalTok{FNAME,}\AttributeTok{useNA=}\StringTok{"always"}\NormalTok{)}
\NormalTok{viensvasar}\OtherTok{=}\NormalTok{viensvasar }\SpecialCharTok{\%\textgreater{}\%} 
  \FunctionTok{filter}\NormalTok{(FNAME }\SpecialCharTok{\%in\%} \FunctionTok{c}\NormalTok{(}\StringTok{"poligons\_Vasarnīcu\_apbūve"}\NormalTok{,}\StringTok{"poligons\_Viensētu\_apbūve"}\NormalTok{)) }\SpecialCharTok{\%\textgreater{}\%} 
  \FunctionTok{mutate}\NormalTok{(}\AttributeTok{yes=}\DecValTok{410}\NormalTok{) }\SpecialCharTok{\%\textgreater{}\%} 
\NormalTok{  dplyr}\SpecialCharTok{::}\FunctionTok{select}\NormalTok{(yes)}
\NormalTok{r\_viensetasvasarnicas}\OtherTok{=}\FunctionTok{fasterize}\NormalTok{(viensvasar,template\_r,}\AttributeTok{field=}\StringTok{"yes"}\NormalTok{)}
\NormalTok{raster}\SpecialCharTok{::}\FunctionTok{writeRaster}\NormalTok{(r\_viensetasvasarnicas,}
                    \StringTok{"./RasterGrids\_10m/2024/SimpleLandscape\_class410\_vasarnicasviensetas\_topo.tif"}\NormalTok{,}
                    \AttributeTok{progress=}\StringTok{"text"}\NormalTok{,}
                    \AttributeTok{overwrite=}\ConstantTok{TRUE}\NormalTok{)}
\CommentTok{\# cleaning}
\FunctionTok{rm}\NormalTok{(viensvasar)}
\FunctionTok{rm}\NormalTok{(r\_darzini\_topo)}

\CommentTok{\# topo}
\NormalTok{darzini\_topo}\OtherTok{=}\FunctionTok{st\_read\_parquet}\NormalTok{(}\StringTok{"./Geodata/2024/TopographicMap/LandusA\_COMB.parquet"}\NormalTok{)}
\FunctionTok{table}\NormalTok{(darzini\_topo}\SpecialCharTok{$}\NormalTok{FNAME,}\AttributeTok{useNA=}\StringTok{"always"}\NormalTok{)}
\NormalTok{darzini\_topo}\OtherTok{=}\NormalTok{darzini\_topo }\SpecialCharTok{\%\textgreater{}\%} 
  \FunctionTok{filter}\NormalTok{(FNAME }\SpecialCharTok{\%in\%} \FunctionTok{c}\NormalTok{(}\StringTok{"poligons\_Augludarzs"}\NormalTok{,}\StringTok{"poligons\_Augļudārzs"}\NormalTok{,}\StringTok{"poligons\_Sakņudārzs"}\NormalTok{,}
                      \StringTok{"poligons\_Ogulājs"}\NormalTok{,}\StringTok{"poligons\_Ogulajs"}\NormalTok{,}\StringTok{"poligons\_Saknudarzs"}\NormalTok{)) }\SpecialCharTok{\%\textgreater{}\%} 
  \FunctionTok{mutate}\NormalTok{(}\AttributeTok{yes=}\DecValTok{410}\NormalTok{) }\SpecialCharTok{\%\textgreater{}\%} 
\NormalTok{  dplyr}\SpecialCharTok{::}\FunctionTok{select}\NormalTok{(yes)}
\NormalTok{r\_darzini\_topo}\OtherTok{=}\FunctionTok{fasterize}\NormalTok{(darzini\_topo,template\_r,}\AttributeTok{field=}\StringTok{"yes"}\NormalTok{)}
\NormalTok{raster}\SpecialCharTok{::}\FunctionTok{writeRaster}\NormalTok{(r\_darzini\_topo,}
                    \StringTok{"./RasterGrids\_10m/2024/SimpleLandscape\_class410\_darzini\_topo.tif"}\NormalTok{,}
                    \AttributeTok{progress=}\StringTok{"text"}\NormalTok{,}
                    \AttributeTok{overwrite=}\ConstantTok{TRUE}\NormalTok{)}
\CommentTok{\# cleaning}
\FunctionTok{rm}\NormalTok{(darzini\_topo)}
\FunctionTok{rm}\NormalTok{(r\_darzini\_topo)}

\CommentTok{\# lad}
\NormalTok{lad\_klasem}\OtherTok{=}\FunctionTok{read\_excel}\NormalTok{(}\StringTok{"./Geodata/2024/LAD/KulturuKodi\_2024.xlsx"}\NormalTok{)}
\FunctionTok{table}\NormalTok{(lad\_klasem}\SpecialCharTok{$}\NormalTok{SDM\_grupa\_sakums,}\AttributeTok{useNA=}\StringTok{"always"}\NormalTok{)}
\NormalTok{augludarziem}\OtherTok{=}\NormalTok{lad\_klasem }\SpecialCharTok{\%\textgreater{}\%} 
  \FunctionTok{filter}\NormalTok{(SDM\_grupa\_sakums}\SpecialCharTok{==}\StringTok{"augļudārzi"}\NormalTok{)}
\NormalTok{lad}\OtherTok{=}\FunctionTok{st\_read\_parquet}\NormalTok{(}\StringTok{"./Geodata/2024/LAD/Lauki\_2024.parquet"}\NormalTok{)}
\NormalTok{lad}\OtherTok{=}\NormalTok{lad }\SpecialCharTok{\%\textgreater{}\%} 
  \FunctionTok{filter}\NormalTok{(PRODUCT\_CODE }\SpecialCharTok{\%in\%}\NormalTok{ augludarziem}\SpecialCharTok{$}\NormalTok{kods) }\SpecialCharTok{\%\textgreater{}\%} 
  \FunctionTok{mutate}\NormalTok{(}\AttributeTok{yes=}\DecValTok{420}\NormalTok{) }\SpecialCharTok{\%\textgreater{}\%} 
\NormalTok{  dplyr}\SpecialCharTok{::}\FunctionTok{select}\NormalTok{(yes)}
\NormalTok{r\_darzini\_lad}\OtherTok{=}\FunctionTok{fasterize}\NormalTok{(lad,template\_r,}\AttributeTok{field=}\StringTok{"yes"}\NormalTok{)}
\NormalTok{raster}\SpecialCharTok{::}\FunctionTok{writeRaster}\NormalTok{(r\_darzini\_lad,}
                    \StringTok{"./RasterGrids\_10m/2024/SimpleLandscape\_class420\_darzini\_lad.tif"}\NormalTok{,}
                    \AttributeTok{progress=}\StringTok{"text"}\NormalTok{,}
                    \AttributeTok{overwrite=}\ConstantTok{TRUE}\NormalTok{)}
\CommentTok{\# cleaning}
\FunctionTok{rm}\NormalTok{(lad\_klasem)}
\FunctionTok{rm}\NormalTok{(augludarziem)}
\FunctionTok{rm}\NormalTok{(lad)}
\FunctionTok{rm}\NormalTok{(r\_darzini\_lad)}

\CommentTok{\# merging}
\NormalTok{a400}\OtherTok{=}\FunctionTok{rast}\NormalTok{(}\StringTok{"./RasterGrids\_10m/2024/SimpleLandscape\_class410\_vasarnicasviensetas\_topo.tif"}\NormalTok{)}
\NormalTok{b400}\OtherTok{=}\FunctionTok{rast}\NormalTok{(}\StringTok{"./RasterGrids\_10m/2024/SimpleLandscape\_class420\_darzini\_topo.tif"}\NormalTok{)}
\NormalTok{c400}\OtherTok{=}\FunctionTok{rast}\NormalTok{(}\StringTok{"./RasterGrids\_10m/2024/SimpleLandscape\_class420\_darzini\_lad.tif"}\NormalTok{)}

\NormalTok{allotment\_cover}\OtherTok{=}\FunctionTok{cover}\NormalTok{(a400,}
\NormalTok{                      b400,}
                     \AttributeTok{filename=}\FunctionTok{paste0}\NormalTok{(}\StringTok{"./RasterGrids\_10m/2024/"}\NormalTok{,}
                                     \StringTok{"SimpleLandscape\_class400\_varnicas\_premask.tif"}\NormalTok{),}
                     \AttributeTok{overwrite=}\ConstantTok{TRUE}\NormalTok{)}

\CommentTok{\# cleaning}
\FunctionTok{rm}\NormalTok{(a400)}
\FunctionTok{rm}\NormalTok{(b400)}
\FunctionTok{rm}\NormalTok{(allotment\_cover)}
\FunctionTok{unlink}\NormalTok{(}\StringTok{"./RasterGrids\_10m/2024/SimpleLandscape\_class410\_vasarnicasviensetas\_topo.tif"}\NormalTok{)}
\FunctionTok{unlink}\NormalTok{(}\StringTok{"./RasterGrids\_10m/2024/SimpleLandscape\_class420\_darzini\_topo.tif"}\NormalTok{)}
\FunctionTok{unlink}\NormalTok{(}\StringTok{"./RasterGrids\_10m/2024/SimpleLandscape\_class420\_darzini\_lad.tif"}\NormalTok{)}
\end{Highlighting}
\end{Shaded}

\begin{itemize}
\item
  Class \texttt{500} - \textbf{Built-up}: built-up areas, no particular layer or data source used.
  Filled in at the end (see section ``merging and filling'' of this chapter) using
  information from the Dynamic World for places not covered by other classes.
\item
  Class \texttt{600} - \textbf{Forests, Shrublands, Clearings}: areas covered with trees and
  shrubs, clearings, and dead forest stands. The following sources have been
  combined to create this class (in order of overlap):

  -- \hyperref[Ch04.09]{The Global Forest Watch} layer records of tree canopy cover
  loss since 2020, coded as \texttt{610};

  -- \hyperref[Ch04.01]{Forest State Register} clearings and dead forest stands, the result
  of which is coded as \texttt{610};

  -- \hyperref[Ch04.01]{Forest State Register} marked forest stands that are lower than 5 m
  and seed production plantations, the result of which is coded as \texttt{620};

  -- \hyperref[Ch04.04]{topographic map} layer \texttt{FloraL\_COMB} classes related to shrubs,
  buffered by 10 m, coded as \texttt{620};

  -- \hyperref[Ch04.04]{topographic map} layers \texttt{LandusA\_COMB} classes: ``poligons\_Krūmājs'',
  ``poligons\_Krumajs'', ``poligons\_Krūmaugu\_plant'', ``poligons\_Plantacija\_krum'', coded as \texttt{620};

  -- \hyperref[Ch04.02]{LAD database} group (classes are
  available \href{https://github.com/aavotins/HiQBioDiv_EGVs/blob/main/Data/Geodata/2024/LAD/KulturuKodi_2024.xlsx}{here})
  ``krūmveida ilggadīgie stādījumi'', the result of which is coded with \texttt{620};

  -- \hyperref[Ch04.01]{Forest State Register} forest stands with a height of at least 5 m,
  coded as \texttt{630};

  -- \hyperref[Ch04.04]{topographic map} layer \texttt{LandusA\_COMB} classes: ``poligons\_Parks'',
  ``poligons\_Meza\_kapi'', ``poligons\_Kapi'', ``poligons\_Kapi\_meza'', the result of
  which is coded as \texttt{640};

  -- \hyperref[Ch04.04]{topographic map} layer \texttt{FloraL\_COMB} with tree-related classes,
  buffered by 10 m, coded as \texttt{640};

  -- \hyperref[Ch04.10]{PALSAR Forests} layer, coded as \texttt{630}.
\end{itemize}

The command lines below create a layer with landscape class \texttt{600}, which is saved
in the file \texttt{SimpleLandscape\_class600\_meziem\_premask.tif} for further processing.

\begin{Shaded}
\begin{Highlighting}[]
\CommentTok{\# Libs {-}{-}{-}{-}}
\ControlFlowTok{if}\NormalTok{(}\SpecialCharTok{!}\FunctionTok{require}\NormalTok{(tidyverse)) \{}\FunctionTok{install.packages}\NormalTok{(}\StringTok{"tidyverse"}\NormalTok{); }\FunctionTok{require}\NormalTok{(tidyverse)\}}
\ControlFlowTok{if}\NormalTok{(}\SpecialCharTok{!}\FunctionTok{require}\NormalTok{(sf)) \{}\FunctionTok{install.packages}\NormalTok{(}\StringTok{"sf"}\NormalTok{); }\FunctionTok{require}\NormalTok{(sf)\}}
\ControlFlowTok{if}\NormalTok{(}\SpecialCharTok{!}\FunctionTok{require}\NormalTok{(arrow)) \{}\FunctionTok{install.packages}\NormalTok{(}\StringTok{"arrow"}\NormalTok{); }\FunctionTok{require}\NormalTok{(arrow)\}}
\ControlFlowTok{if}\NormalTok{(}\SpecialCharTok{!}\FunctionTok{require}\NormalTok{(sfarrow)) \{}\FunctionTok{install.packages}\NormalTok{(}\StringTok{"sfarrow"}\NormalTok{); }\FunctionTok{require}\NormalTok{(sfarrow)\}}
\ControlFlowTok{if}\NormalTok{(}\SpecialCharTok{!}\FunctionTok{require}\NormalTok{(terra)) \{}\FunctionTok{install.packages}\NormalTok{(}\StringTok{"terra"}\NormalTok{); }\FunctionTok{require}\NormalTok{(terra)\}}
\ControlFlowTok{if}\NormalTok{(}\SpecialCharTok{!}\FunctionTok{require}\NormalTok{(raster)) \{}\FunctionTok{install.packages}\NormalTok{(}\StringTok{"raster"}\NormalTok{); }\FunctionTok{require}\NormalTok{(raster)\}}
\ControlFlowTok{if}\NormalTok{(}\SpecialCharTok{!}\FunctionTok{require}\NormalTok{(fasterize)) \{}\FunctionTok{install.packages}\NormalTok{(}\StringTok{"fasterize"}\NormalTok{); }\FunctionTok{require}\NormalTok{(fasterize)\}}
\ControlFlowTok{if}\NormalTok{(}\SpecialCharTok{!}\FunctionTok{require}\NormalTok{(readxl)) \{}\FunctionTok{install.packages}\NormalTok{(}\StringTok{"readxl"}\NormalTok{); }\FunctionTok{require}\NormalTok{(readxl)\}}

\CommentTok{\# templates {-}{-}{-}{-}}
\NormalTok{template\_t}\OtherTok{=}\FunctionTok{rast}\NormalTok{(}\StringTok{"./Templates/TemplateRasters/LV10m\_10km.tif"}\NormalTok{)}
\NormalTok{template\_r}\OtherTok{=}\FunctionTok{raster}\NormalTok{(template\_t)}


\CommentTok{\# class 600 {-}{-}{-}{-}}

\CommentTok{\# mvr }
\NormalTok{mvr}\OtherTok{=}\FunctionTok{st\_read\_parquet}\NormalTok{(}\StringTok{"./Geodata/2024/MVR/nogabali\_2024janv.parquet"}\NormalTok{)}

\CommentTok{\# clearcuts}
\NormalTok{izcirtumi}\OtherTok{=}\NormalTok{mvr }\SpecialCharTok{\%\textgreater{}\%} 
  \FunctionTok{filter}\NormalTok{(zkat }\SpecialCharTok{\%in\%} \FunctionTok{c}\NormalTok{(}\StringTok{"12"}\NormalTok{,}\StringTok{"14"}\NormalTok{)) }\SpecialCharTok{\%\textgreater{}\%} 
  \FunctionTok{mutate}\NormalTok{(}\AttributeTok{yes=}\DecValTok{610}\NormalTok{) }\SpecialCharTok{\%\textgreater{}\%} 
\NormalTok{  dplyr}\SpecialCharTok{::}\FunctionTok{select}\NormalTok{(yes)}
\NormalTok{r\_izcirtumi\_mvr}\OtherTok{=}\FunctionTok{fasterize}\NormalTok{(izcirtumi,template\_r,}\AttributeTok{field=}\StringTok{"yes"}\NormalTok{)}
\NormalTok{raster}\SpecialCharTok{::}\FunctionTok{writeRaster}\NormalTok{(r\_izcirtumi\_mvr,}
                    \StringTok{"./RasterGrids\_10m/2024/SimpleLandscape\_class610\_izcirtumi\_mvr.tif"}\NormalTok{,}
                    \AttributeTok{progress=}\StringTok{"text"}\NormalTok{,}
                    \AttributeTok{overwrite=}\ConstantTok{TRUE}\NormalTok{)}
\CommentTok{\# cleaning}
\FunctionTok{rm}\NormalTok{(izcirtumi)}
\FunctionTok{rm}\NormalTok{(r\_izcirtumi\_mvr)}

\CommentTok{\# low stands}
\CommentTok{\# also zkat 16}
\NormalTok{zemas\_audzes}\OtherTok{=}\NormalTok{mvr }\SpecialCharTok{\%\textgreater{}\%} 
  \FunctionTok{filter}\NormalTok{((zkat }\SpecialCharTok{==}\StringTok{"10"} \SpecialCharTok{\&}\NormalTok{ h10}\SpecialCharTok{\textless{}}\DecValTok{5}\NormalTok{)}\SpecialCharTok{|}\NormalTok{zkat}\SpecialCharTok{==}\StringTok{"16"}\NormalTok{) }\SpecialCharTok{\%\textgreater{}\%} 
  \FunctionTok{mutate}\NormalTok{(}\AttributeTok{yes=}\DecValTok{620}\NormalTok{) }\SpecialCharTok{\%\textgreater{}\%} 
\NormalTok{  dplyr}\SpecialCharTok{::}\FunctionTok{select}\NormalTok{(yes)}
\NormalTok{r\_zemas\_mvr}\OtherTok{=}\FunctionTok{fasterize}\NormalTok{(zemas\_audzes,template\_r,}\AttributeTok{field=}\StringTok{"yes"}\NormalTok{)}
\NormalTok{raster}\SpecialCharTok{::}\FunctionTok{writeRaster}\NormalTok{(r\_zemas\_mvr,}
                    \StringTok{"./RasterGrids\_10m/2024/SimpleLandscape\_class620\_zemas\_mvr.tif"}\NormalTok{,}
                    \AttributeTok{progress=}\StringTok{"text"}\NormalTok{,}
                    \AttributeTok{overwrite=}\ConstantTok{TRUE}\NormalTok{)}
\CommentTok{\# cleaning}
\FunctionTok{rm}\NormalTok{(zemas\_audzes)}
\FunctionTok{rm}\NormalTok{(r\_zemas\_mvr)}


\CommentTok{\# high stands}
\NormalTok{augstas\_audzes}\OtherTok{=}\NormalTok{mvr }\SpecialCharTok{\%\textgreater{}\%} 
  \FunctionTok{filter}\NormalTok{(zkat }\SpecialCharTok{==}\StringTok{"10"} \SpecialCharTok{\&}\NormalTok{ h10}\SpecialCharTok{\textgreater{}=}\DecValTok{5}\NormalTok{) }\SpecialCharTok{\%\textgreater{}\%} 
  \FunctionTok{mutate}\NormalTok{(}\AttributeTok{yes=}\DecValTok{630}\NormalTok{) }\SpecialCharTok{\%\textgreater{}\%} 
\NormalTok{  dplyr}\SpecialCharTok{::}\FunctionTok{select}\NormalTok{(yes)}
\NormalTok{r\_augstas\_mvr}\OtherTok{=}\FunctionTok{fasterize}\NormalTok{(augstas\_audzes,template\_r,}\AttributeTok{field=}\StringTok{"yes"}\NormalTok{)}
\NormalTok{raster}\SpecialCharTok{::}\FunctionTok{writeRaster}\NormalTok{(r\_augstas\_mvr,}
                    \StringTok{"./RasterGrids\_10m/2024/SimpleLandscape\_class630\_augstas\_mvr.tif"}\NormalTok{,}
                    \AttributeTok{progress=}\StringTok{"text"}\NormalTok{,}
                    \AttributeTok{overwrite=}\ConstantTok{TRUE}\NormalTok{)}
\CommentTok{\# cleaning}
\FunctionTok{rm}\NormalTok{(augstas\_audzes)}
\FunctionTok{rm}\NormalTok{(r\_augstas\_mvr)}
\FunctionTok{rm}\NormalTok{(mvr)}

\CommentTok{\# tcl {-} since 2020}
\NormalTok{tcl}\OtherTok{=}\FunctionTok{rast}\NormalTok{(}\StringTok{"./Geodata/2024/Trees/GFW/TreeCoverLoss\_v1\_12.tif"}\NormalTok{)}
\NormalTok{tcl2}\OtherTok{=}\FunctionTok{ifel}\NormalTok{(tcl}\SpecialCharTok{\textless{}}\DecValTok{20}\NormalTok{,}\ConstantTok{NA}\NormalTok{,}\DecValTok{610}\NormalTok{,}
          \AttributeTok{filename=}\StringTok{"./RasterGrids\_10m/2024/SimpleLandscape\_class610\_TCL.tif"}\NormalTok{,}
          \AttributeTok{overwrite=}\ConstantTok{TRUE}\NormalTok{)}
\CommentTok{\# cleaning}
\FunctionTok{rm}\NormalTok{(tcl)}
\FunctionTok{rm}\NormalTok{(tcl2)}

\CommentTok{\# palsar}
\NormalTok{palsar}\OtherTok{=}\FunctionTok{rast}\NormalTok{(}\StringTok{"./Geodata/2024/Trees/Palsar/Palsar\_Forests.tif"}\NormalTok{)}
\NormalTok{palsar2}\OtherTok{=}\FunctionTok{ifel}\NormalTok{(palsar}\SpecialCharTok{==}\DecValTok{1}\NormalTok{,}\DecValTok{630}\NormalTok{,}\ConstantTok{NA}\NormalTok{,}
             \AttributeTok{filename=}\StringTok{"./RasterGrids\_10m/2024/SimpleLandscape\_class630\_Palsar.tif"}\NormalTok{,}
             \AttributeTok{overwrite=}\ConstantTok{TRUE}\NormalTok{)}
\CommentTok{\# cleaning}
\FunctionTok{rm}\NormalTok{(palsar)}
\FunctionTok{rm}\NormalTok{(palsar2)}


\CommentTok{\# lad}
\NormalTok{lad\_klasem}\OtherTok{=}\FunctionTok{read\_excel}\NormalTok{(}\StringTok{"./Geodata/2024/LAD/KulturuKodi\_2024.xlsx"}\NormalTok{)}
\FunctionTok{table}\NormalTok{(lad\_klasem}\SpecialCharTok{$}\NormalTok{SDM\_grupa\_sakums,}\AttributeTok{useNA=}\StringTok{"always"}\NormalTok{)}
\NormalTok{lad}\OtherTok{=}\FunctionTok{st\_read\_parquet}\NormalTok{(}\StringTok{"./Geodata/2024/LAD/Lauki\_2024.parquet"}\NormalTok{)}
\NormalTok{krumiem}\OtherTok{=}\NormalTok{lad\_klasem }\SpecialCharTok{\%\textgreater{}\%} 
  \FunctionTok{filter}\NormalTok{(}\FunctionTok{str\_detect}\NormalTok{(SDM\_grupa\_sakums,}\StringTok{"krūmv"}\NormalTok{))}
\NormalTok{krumi}\OtherTok{=}\NormalTok{lad }\SpecialCharTok{\%\textgreater{}\%} 
  \FunctionTok{filter}\NormalTok{(PRODUCT\_CODE }\SpecialCharTok{\%in\%}\NormalTok{ krumiem}\SpecialCharTok{$}\NormalTok{kods) }\SpecialCharTok{\%\textgreater{}\%} 
  \FunctionTok{mutate}\NormalTok{(}\AttributeTok{yes=}\DecValTok{620}\NormalTok{) }\SpecialCharTok{\%\textgreater{}\%} 
\NormalTok{  dplyr}\SpecialCharTok{::}\FunctionTok{select}\NormalTok{(yes)}
\NormalTok{r\_krumi\_lad}\OtherTok{=}\FunctionTok{fasterize}\NormalTok{(krumi,template\_r,}\AttributeTok{field=}\StringTok{"yes"}\NormalTok{)}
\NormalTok{raster}\SpecialCharTok{::}\FunctionTok{writeRaster}\NormalTok{(r\_krumi\_lad,}
                    \StringTok{"./RasterGrids\_10m/2024/SimpleLandscape\_class620\_krumi\_lad.tif"}\NormalTok{,}
                    \AttributeTok{progress=}\StringTok{"text"}\NormalTok{,}
                    \AttributeTok{overwrite=}\ConstantTok{TRUE}\NormalTok{)}
\CommentTok{\# cleaning}
\FunctionTok{rm}\NormalTok{(lad\_klasem)}
\FunctionTok{rm}\NormalTok{(lad)}
\FunctionTok{rm}\NormalTok{(krumiem)}
\FunctionTok{rm}\NormalTok{(krumi)}
\FunctionTok{rm}\NormalTok{(r\_krumi\_lad)}

\CommentTok{\# topo {-} pkk}
\NormalTok{pkk\_topo}\OtherTok{=}\FunctionTok{st\_read\_parquet}\NormalTok{(}\StringTok{"./Geodata/2024/TopographicMap/LandusA\_COMB.parquet"}\NormalTok{)}
\FunctionTok{table}\NormalTok{(pkk\_topo}\SpecialCharTok{$}\NormalTok{FNAME,}\AttributeTok{useNA=}\StringTok{"always"}\NormalTok{)}
\NormalTok{pkk\_topo}\OtherTok{=}\NormalTok{pkk\_topo }\SpecialCharTok{\%\textgreater{}\%} 
  \FunctionTok{filter}\NormalTok{(FNAME }\SpecialCharTok{\%in\%} \FunctionTok{c}\NormalTok{(}\StringTok{"poligons\_Parks"}\NormalTok{,}\StringTok{"poligons\_Meza\_kapi"}\NormalTok{,}\StringTok{"poligons\_Kapi"}\NormalTok{,}
                      \StringTok{"poligons\_Kapi\_meza"}\NormalTok{)) }\SpecialCharTok{\%\textgreater{}\%} 
  \FunctionTok{mutate}\NormalTok{(}\AttributeTok{yes=}\DecValTok{640}\NormalTok{) }\SpecialCharTok{\%\textgreater{}\%} 
\NormalTok{  dplyr}\SpecialCharTok{::}\FunctionTok{select}\NormalTok{(yes)}
\NormalTok{r\_pkk\_topo}\OtherTok{=}\FunctionTok{fasterize}\NormalTok{(pkk\_topo,template\_r,}\AttributeTok{field=}\StringTok{"yes"}\NormalTok{)}
\NormalTok{raster}\SpecialCharTok{::}\FunctionTok{writeRaster}\NormalTok{(r\_pkk\_topo,}
                    \StringTok{"./RasterGrids\_10m/2024/SimpleLandscape\_class640\_pkk\_topo.tif"}\NormalTok{,}
                    \AttributeTok{progress=}\StringTok{"text"}\NormalTok{,}
                    \AttributeTok{overwrite=}\ConstantTok{TRUE}\NormalTok{)}
\CommentTok{\# cleaning}
\FunctionTok{rm}\NormalTok{(pkk\_topo)}
\FunctionTok{rm}\NormalTok{(r\_pkk\_topo)}

\CommentTok{\# topo {-} shrubs}
\NormalTok{krumi\_topo}\OtherTok{=}\FunctionTok{st\_read\_parquet}\NormalTok{(}\StringTok{"./Geodata/2024/TopographicMap/LandusA\_COMB.parquet"}\NormalTok{)}
\FunctionTok{table}\NormalTok{(krumi\_topo}\SpecialCharTok{$}\NormalTok{FNAME,}\AttributeTok{useNA=}\StringTok{"always"}\NormalTok{)}
\NormalTok{krumi\_topo}\OtherTok{=}\NormalTok{krumi\_topo }\SpecialCharTok{\%\textgreater{}\%} 
  \FunctionTok{filter}\NormalTok{(FNAME }\SpecialCharTok{\%in\%} \FunctionTok{c}\NormalTok{(}\StringTok{"poligons\_Krūmājs"}\NormalTok{,}\StringTok{"poligons\_Krumajs"}\NormalTok{,}
                      \StringTok{"poligons\_Krūmaugu\_plant"}\NormalTok{,}\StringTok{"poligons\_Plantacija\_krum"}\NormalTok{)) }\SpecialCharTok{\%\textgreater{}\%} 
  \FunctionTok{mutate}\NormalTok{(}\AttributeTok{yes=}\DecValTok{620}\NormalTok{) }\SpecialCharTok{\%\textgreater{}\%} 
\NormalTok{  dplyr}\SpecialCharTok{::}\FunctionTok{select}\NormalTok{(yes)}
\NormalTok{r\_krumi\_topo}\OtherTok{=}\FunctionTok{fasterize}\NormalTok{(krumi\_topo,template\_r,}\AttributeTok{field=}\StringTok{"yes"}\NormalTok{)}
\NormalTok{raster}\SpecialCharTok{::}\FunctionTok{writeRaster}\NormalTok{(r\_krumi\_topo,}
                    \StringTok{"./RasterGrids\_10m/2024/SimpleLandscape\_class620\_krumi\_topo.tif"}\NormalTok{,}
                    \AttributeTok{progress=}\StringTok{"text"}\NormalTok{,}
                    \AttributeTok{overwrite=}\ConstantTok{TRUE}\NormalTok{)}
\CommentTok{\# cleaning}
\FunctionTok{rm}\NormalTok{(krumi\_topo)}
\FunctionTok{rm}\NormalTok{(r\_krumi\_topo)}

\CommentTok{\# topo {-} linear vegetation}
\NormalTok{linijas\_topo}\OtherTok{=}\FunctionTok{st\_read\_parquet}\NormalTok{(}\StringTok{"./Geodata/2024/TopographicMap/FloraL\_COMB.parquet"}\NormalTok{)}
\FunctionTok{table}\NormalTok{(linijas\_topo}\SpecialCharTok{$}\NormalTok{FNAME,}\AttributeTok{useNA=}\StringTok{"always"}\NormalTok{)}

\CommentTok{\# linear shrubs}
\NormalTok{krumu\_linijas\_topo}\OtherTok{=}\NormalTok{linijas\_topo }\SpecialCharTok{\%\textgreater{}\%} 
  \FunctionTok{filter}\NormalTok{(FNAME}\SpecialCharTok{==}\StringTok{"Krūmu rinda dzīvzogs"}\SpecialCharTok{|}\NormalTok{FNAME}\SpecialCharTok{==}\StringTok{"Krūmu rinda gar ceļiem upēm"}\SpecialCharTok{|}
\NormalTok{           FNAME}\SpecialCharTok{==}\StringTok{"Krumu\_rinda\_dzivzogs"}\SpecialCharTok{|}\NormalTok{FNAME}\SpecialCharTok{==}\StringTok{"Krumu\_rinda\_gar\_celiem\_upem"}\NormalTok{) }\SpecialCharTok{\%\textgreater{}\%} 
  \FunctionTok{mutate}\NormalTok{(}\AttributeTok{yes=}\DecValTok{620}\NormalTok{) }\SpecialCharTok{\%\textgreater{}\%} 
  \FunctionTok{st\_buffer}\NormalTok{(}\AttributeTok{dist=}\DecValTok{10}\NormalTok{) }\SpecialCharTok{\%\textgreater{}\%} 
\NormalTok{  dplyr}\SpecialCharTok{::}\FunctionTok{select}\NormalTok{(yes)}
\NormalTok{r\_krumu\_linijas\_topo}\OtherTok{=}\FunctionTok{fasterize}\NormalTok{(krumu\_linijas\_topo,template\_r,}\AttributeTok{field=}\StringTok{"yes"}\NormalTok{)}
\NormalTok{raster}\SpecialCharTok{::}\FunctionTok{writeRaster}\NormalTok{(r\_krumu\_linijas\_topo,}
                    \StringTok{"./RasterGrids\_10m/2024/SimpleLandscape\_class620\_KrumuLinijas\_topo.tif"}\NormalTok{,}
                    \AttributeTok{progress=}\StringTok{"text"}\NormalTok{,}
                    \AttributeTok{overwrite=}\ConstantTok{TRUE}\NormalTok{)}
\CommentTok{\# cleaning}
\FunctionTok{rm}\NormalTok{(krumu\_linijas\_topo)}
\FunctionTok{rm}\NormalTok{(r\_krumu\_linijas\_topo)}

\CommentTok{\# linear trees}
\NormalTok{koku\_linijas\_topo}\OtherTok{=}\NormalTok{linijas\_topo }\SpecialCharTok{\%\textgreater{}\%} 
  \FunctionTok{filter}\NormalTok{(}\FunctionTok{str\_detect}\NormalTok{(FNAME,}\StringTok{"Koku"}\NormalTok{)) }\SpecialCharTok{\%\textgreater{}\%} 
  \FunctionTok{mutate}\NormalTok{(}\AttributeTok{yes=}\DecValTok{640}\NormalTok{) }\SpecialCharTok{\%\textgreater{}\%} 
  \FunctionTok{st\_buffer}\NormalTok{(}\AttributeTok{dist=}\DecValTok{10}\NormalTok{) }\SpecialCharTok{\%\textgreater{}\%} 
\NormalTok{  dplyr}\SpecialCharTok{::}\FunctionTok{select}\NormalTok{(yes)}
\NormalTok{r\_koku\_linijas\_topo}\OtherTok{=}\FunctionTok{fasterize}\NormalTok{(koku\_linijas\_topo,template\_r,}\AttributeTok{field=}\StringTok{"yes"}\NormalTok{)}
\NormalTok{raster}\SpecialCharTok{::}\FunctionTok{writeRaster}\NormalTok{(r\_koku\_linijas\_topo,}
                    \StringTok{"./RasterGrids\_10m/2024/SimpleLandscape\_class640\_KokuLinijas\_topo.tif"}\NormalTok{,}
                    \AttributeTok{progress=}\StringTok{"text"}\NormalTok{,}
                    \AttributeTok{overwrite=}\ConstantTok{TRUE}\NormalTok{)}
\CommentTok{\# cleaning}
\FunctionTok{rm}\NormalTok{(koku\_linijas\_topo)}
\FunctionTok{rm}\NormalTok{(r\_koku\_linijas\_topo)}
\FunctionTok{rm}\NormalTok{(linijas\_topo)}

\CommentTok{\# merging}
\NormalTok{r\_krumi\_lad}\OtherTok{=}\FunctionTok{rast}\NormalTok{(}\StringTok{"./RasterGrids\_10m/2024/SimpleLandscape\_class620\_krumi\_lad.tif"}\NormalTok{)}
\NormalTok{r\_pkk\_topo}\OtherTok{=}\FunctionTok{rast}\NormalTok{(}\StringTok{"./RasterGrids\_10m/2024/SimpleLandscape\_class640\_pkk\_topo.tif"}\NormalTok{)}
\NormalTok{r\_krumi\_topo}\OtherTok{=}\FunctionTok{rast}\NormalTok{(}\StringTok{"./RasterGrids\_10m/2024/SimpleLandscape\_class620\_krumi\_topo.tif"}\NormalTok{)}
\NormalTok{r\_krumu\_linijas\_topo}\OtherTok{=}\FunctionTok{rast}\NormalTok{(}\StringTok{"./RasterGrids\_10m/2024/SimpleLandscape\_class620\_KrumuLinijas\_topo.tif"}\NormalTok{)}
\NormalTok{r\_koku\_linijas\_topo}\OtherTok{=}\FunctionTok{rast}\NormalTok{(}\StringTok{"./RasterGrids\_10m/2024/SimpleLandscape\_class640\_KokuLinijas\_topo.tif"}\NormalTok{)}
\NormalTok{r\_palsar}\OtherTok{=}\FunctionTok{rast}\NormalTok{(}\StringTok{"./RasterGrids\_10m/2024/SimpleLandscape\_class630\_palsar.tif"}\NormalTok{)}
\NormalTok{r\_tcl}\OtherTok{=}\FunctionTok{rast}\NormalTok{(}\StringTok{"./RasterGrids\_10m/2024/SimpleLandscape\_class610\_TCL.tif"}\NormalTok{)}
\NormalTok{r\_augstas\_mvr}\OtherTok{=}\FunctionTok{rast}\NormalTok{(}\StringTok{"./RasterGrids\_10m/2024/SimpleLandscape\_class630\_augstas\_mvr.tif"}\NormalTok{)}
\NormalTok{r\_zemas\_mvr}\OtherTok{=}\FunctionTok{rast}\NormalTok{(}\StringTok{"./RasterGrids\_10m/2024/SimpleLandscape\_class620\_zemas\_mvr.tif"}\NormalTok{)}
\NormalTok{r\_izcirtumi\_mvr}\OtherTok{=}\FunctionTok{rast}\NormalTok{(}\StringTok{"./RasterGrids\_10m/2024/SimpleLandscape\_class610\_izcirtumi\_mvr.tif"}\NormalTok{)}


\NormalTok{mezu\_cover}\OtherTok{=}\FunctionTok{cover}\NormalTok{(r\_tcl,r\_izcirtumi\_mvr)}
\NormalTok{mezu\_cover}\OtherTok{=}\FunctionTok{cover}\NormalTok{(mezu\_cover,r\_zemas\_mvr)}
\NormalTok{mezu\_cover}\OtherTok{=}\FunctionTok{cover}\NormalTok{(mezu\_cover,r\_krumu\_linijas\_topo)}
\NormalTok{mezu\_cover}\OtherTok{=}\FunctionTok{cover}\NormalTok{(mezu\_cover,r\_krumi\_topo)}
\NormalTok{mezu\_cover}\OtherTok{=}\FunctionTok{cover}\NormalTok{(mezu\_cover,r\_krumi\_lad)}
\NormalTok{mezu\_cover}\OtherTok{=}\FunctionTok{cover}\NormalTok{(mezu\_cover,r\_augstas\_mvr)}
\NormalTok{mezu\_cover}\OtherTok{=}\FunctionTok{cover}\NormalTok{(mezu\_cover,r\_pkk\_topo)}
\NormalTok{mezu\_cover}\OtherTok{=}\FunctionTok{cover}\NormalTok{(mezu\_cover,r\_koku\_linijas\_topo)}
\NormalTok{mezu\_cover}\OtherTok{=}\FunctionTok{cover}\NormalTok{(mezu\_cover,r\_palsar,}
                 \AttributeTok{filename=}\StringTok{"./RasterGrids\_10m/2024/SimpleLandscape\_class600\_meziem\_premask.tif"}\NormalTok{,}
                 \AttributeTok{overwrite=}\ConstantTok{TRUE}\NormalTok{)}


\CommentTok{\# cleaning}
\FunctionTok{rm}\NormalTok{(r\_krumi\_lad)}
\FunctionTok{rm}\NormalTok{(r\_pkk\_topo)}
\FunctionTok{rm}\NormalTok{(r\_krumi\_topo)}
\FunctionTok{rm}\NormalTok{(r\_krumu\_linijas\_topo)}
\FunctionTok{rm}\NormalTok{(r\_koku\_linijas\_topo)}
\FunctionTok{rm}\NormalTok{(r\_palsar)}
\FunctionTok{rm}\NormalTok{(r\_tcl)}
\FunctionTok{rm}\NormalTok{(r\_augstas\_mvr)}
\FunctionTok{rm}\NormalTok{(r\_zemas\_mvr)}
\FunctionTok{rm}\NormalTok{(r\_izcirtumi\_mvr)}
\FunctionTok{rm}\NormalTok{(mezu\_cover)}

\FunctionTok{unlink}\NormalTok{(}\StringTok{"./RasterGrids\_10m/2024/SimpleLandscape\_class620\_krumi\_lad.tif"}\NormalTok{)}
\FunctionTok{unlink}\NormalTok{(}\StringTok{"./RasterGrids\_10m/2024/SimpleLandscape\_class640\_pkk\_topo.tif"}\NormalTok{)}
\FunctionTok{unlink}\NormalTok{(}\StringTok{"./RasterGrids\_10m/2024/SimpleLandscape\_class620\_krumi\_topo.tif"}\NormalTok{)}
\FunctionTok{unlink}\NormalTok{(}\StringTok{"./RasterGrids\_10m/2024/SimpleLandscape\_class620\_KrumuLinijas\_topo.tif"}\NormalTok{)}
\FunctionTok{unlink}\NormalTok{(}\StringTok{"./RasterGrids\_10m/2024/SimpleLandscape\_class640\_KokuLinijas\_topo.tif"}\NormalTok{)}
\FunctionTok{unlink}\NormalTok{(}\StringTok{"./RasterGrids\_10m/2024/SimpleLandscape\_class630\_palsar.tif"}\NormalTok{)}
\FunctionTok{unlink}\NormalTok{(}\StringTok{"./RasterGrids\_10m/2024/SimpleLandscape\_class610\_TCL.tif"}\NormalTok{)}
\FunctionTok{unlink}\NormalTok{(}\StringTok{"./RasterGrids\_10m/2024/SimpleLandscape\_class630\_augstas\_mvr.tif"}\NormalTok{)}
\FunctionTok{unlink}\NormalTok{(}\StringTok{"./RasterGrids\_10m/2024/SimpleLandscape\_class620\_zemas\_mvr.tif"}\NormalTok{)}
\FunctionTok{unlink}\NormalTok{(}\StringTok{"./RasterGrids\_10m/2024/SimpleLandscape\_class610\_izcirtumi\_mvr.tif"}\NormalTok{)}
\end{Highlighting}
\end{Shaded}

\begin{itemize}
\item
  Class \texttt{700} - \textbf{Wetlands}: combining geospatial data related to reed, sedge and rush beds,
  marshes, mires, and bogs, \textbf{filled in order except class \texttt{720} that dominates over waters}.
  To create this class, the following sources are combined (in order of overlap):

  -- \hyperref[Ch04.04]{topographic map} layer \texttt{LandusA\_COMB} classes: ``Meldrājs\_ūdenī\_poligons'',
  ``poligons\_Grislajs'', ``poligons\_Grīslājs'', ``poligons\_Meldrajs'', ``poligons\_Meldrājs'',
  ``poligons\_Meldrajs\_udeni'', ``poligons\_Nec\_purvs\_grīslājs'', ``poligons\_Nec\_purvs\_meldrājs'',
  ``Sēklis\_poligons'', the result of which is coded with \texttt{720};

  -- \hyperref[Ch04.04]{topographic map} layer \texttt{LandusA\_COMB} classes: ``poligons\_Nec\_purvs\_sūnājs'',
  ``poligons\_Sunajs'', ``poligons\_Sūnājs'', the result of which is coded with \texttt{710};

  -- \hyperref[Ch04.04]{topographic map} layer \texttt{SwampA\_COMB}, the result of which is coded
  as \texttt{710};

  -- land categories ``21'', ``22'', and ``23'' marked in the \hyperref[Ch04.01]{State Forest Register},
  the result of which is coded as \texttt{710};

  -- land categories ``41'' and ``42'' marked in the \hyperref[Ch04.01]{State Forest Register},
  the result of which is coded as \texttt{730};

  \begin{itemize}
  \item
    bogs from \hyperref[Ch04.17]{Bogs and Mires: EDI};
  \item
    transitional mires from \hyperref[Ch04.17]{Bogs and Mires: EDI};
  \end{itemize}
\end{itemize}

The command lines below create a layer with landscape class \texttt{700}, which is saved
in the file \texttt{SimpleLandscape\_class700\_mitraji\_premask.tif} for further processing.

\begin{Shaded}
\begin{Highlighting}[]
\CommentTok{\# Libs {-}{-}{-}{-}}
\ControlFlowTok{if}\NormalTok{(}\SpecialCharTok{!}\FunctionTok{require}\NormalTok{(tidyverse)) \{}\FunctionTok{install.packages}\NormalTok{(}\StringTok{"tidyverse"}\NormalTok{); }\FunctionTok{require}\NormalTok{(tidyverse)\}}
\ControlFlowTok{if}\NormalTok{(}\SpecialCharTok{!}\FunctionTok{require}\NormalTok{(sf)) \{}\FunctionTok{install.packages}\NormalTok{(}\StringTok{"sf"}\NormalTok{); }\FunctionTok{require}\NormalTok{(sf)\}}
\ControlFlowTok{if}\NormalTok{(}\SpecialCharTok{!}\FunctionTok{require}\NormalTok{(arrow)) \{}\FunctionTok{install.packages}\NormalTok{(}\StringTok{"arrow"}\NormalTok{); }\FunctionTok{require}\NormalTok{(arrow)\}}
\ControlFlowTok{if}\NormalTok{(}\SpecialCharTok{!}\FunctionTok{require}\NormalTok{(sfarrow)) \{}\FunctionTok{install.packages}\NormalTok{(}\StringTok{"sfarrow"}\NormalTok{); }\FunctionTok{require}\NormalTok{(sfarrow)\}}
\ControlFlowTok{if}\NormalTok{(}\SpecialCharTok{!}\FunctionTok{require}\NormalTok{(terra)) \{}\FunctionTok{install.packages}\NormalTok{(}\StringTok{"terra"}\NormalTok{); }\FunctionTok{require}\NormalTok{(terra)\}}
\ControlFlowTok{if}\NormalTok{(}\SpecialCharTok{!}\FunctionTok{require}\NormalTok{(raster)) \{}\FunctionTok{install.packages}\NormalTok{(}\StringTok{"raster"}\NormalTok{); }\FunctionTok{require}\NormalTok{(raster)\}}
\ControlFlowTok{if}\NormalTok{(}\SpecialCharTok{!}\FunctionTok{require}\NormalTok{(fasterize)) \{}\FunctionTok{install.packages}\NormalTok{(}\StringTok{"fasterize"}\NormalTok{); }\FunctionTok{require}\NormalTok{(fasterize)\}}
\ControlFlowTok{if}\NormalTok{(}\SpecialCharTok{!}\FunctionTok{require}\NormalTok{(readxl)) \{}\FunctionTok{install.packages}\NormalTok{(}\StringTok{"readxl"}\NormalTok{); }\FunctionTok{require}\NormalTok{(readxl)\}}

\CommentTok{\# templates {-}{-}{-}{-}}
\NormalTok{template\_t}\OtherTok{=}\FunctionTok{rast}\NormalTok{(}\StringTok{"./Templates/TemplateRasters/LV10m\_10km.tif"}\NormalTok{)}
\NormalTok{template\_r}\OtherTok{=}\FunctionTok{raster}\NormalTok{(template\_t)}


\CommentTok{\# class 700 {-}{-}{-}{-}}

\CommentTok{\# topo}
\NormalTok{topo}\OtherTok{=}\FunctionTok{st\_read\_parquet}\NormalTok{(}\StringTok{"./Geodata/2024/TopographicMap/LandusA\_COMB.parquet"}\NormalTok{)}
\FunctionTok{table}\NormalTok{(topo}\SpecialCharTok{$}\NormalTok{FNAME,}\AttributeTok{useNA=}\StringTok{"always"}\NormalTok{)}

\DocumentationTok{\#\# ReedSedgeRush}
\NormalTok{niedraji\_topo}\OtherTok{=}\NormalTok{topo }\SpecialCharTok{\%\textgreater{}\%} 
  \FunctionTok{filter}\NormalTok{(FNAME }\SpecialCharTok{\%in\%} \FunctionTok{c}\NormalTok{(}\StringTok{"Meldrājs\_ūdenī\_poligons"}\NormalTok{,}\StringTok{"poligons\_Grislajs"}\NormalTok{,}\StringTok{"poligons\_Grīslājs"}\NormalTok{,}
                      \StringTok{"poligons\_Meldrajs"}\NormalTok{,}\StringTok{"poligons\_Meldrājs"}\NormalTok{,}\StringTok{"poligons\_Meldrajs\_udeni"}\NormalTok{,}
                      \StringTok{"poligons\_Nec\_purvs\_grīslājs"}\NormalTok{,}
                      \StringTok{"poligons\_Nec\_purvs\_meldrājs"}\NormalTok{,}
                      \StringTok{"Sēklis\_poligons"}\NormalTok{)) }\SpecialCharTok{\%\textgreater{}\%} 
  \FunctionTok{mutate}\NormalTok{(}\AttributeTok{yes=}\DecValTok{720}\NormalTok{) }\SpecialCharTok{\%\textgreater{}\%} 
\NormalTok{  dplyr}\SpecialCharTok{::}\FunctionTok{select}\NormalTok{(yes)}
\NormalTok{r\_niedraji\_topo}\OtherTok{=}\FunctionTok{fasterize}\NormalTok{(niedraji\_topo,template\_r,}\AttributeTok{field=}\StringTok{"yes"}\NormalTok{)}
\NormalTok{raster}\SpecialCharTok{::}\FunctionTok{writeRaster}\NormalTok{(r\_niedraji\_topo,}
                    \StringTok{"./RasterGrids\_10m/2024/SimpleLandscape\_class720\_niedraji\_topo.tif"}\NormalTok{,}
                    \AttributeTok{progress=}\StringTok{"text"}\NormalTok{,}
                    \AttributeTok{overwrite=}\ConstantTok{TRUE}\NormalTok{)}
\CommentTok{\# cleaning}
\FunctionTok{rm}\NormalTok{(niedraji\_topo)}
\FunctionTok{rm}\NormalTok{(r\_niedraji\_topo)}


\DocumentationTok{\#\# bogs}
\NormalTok{purvi\_topo}\OtherTok{=}\NormalTok{topo }\SpecialCharTok{\%\textgreater{}\%} 
  \FunctionTok{filter}\NormalTok{(FNAME }\SpecialCharTok{\%in\%} \FunctionTok{c}\NormalTok{(}\StringTok{"poligons\_Nec\_purvs\_sūnājs"}\NormalTok{,}
                      \StringTok{"poligons\_Sunajs"}\NormalTok{,}\StringTok{"poligons\_Sūnājs"}\NormalTok{)) }\SpecialCharTok{\%\textgreater{}\%} 
  \FunctionTok{mutate}\NormalTok{(}\AttributeTok{yes=}\DecValTok{710}\NormalTok{) }\SpecialCharTok{\%\textgreater{}\%} 
\NormalTok{  dplyr}\SpecialCharTok{::}\FunctionTok{select}\NormalTok{(yes)}
\NormalTok{topo\_purvi}\OtherTok{=}\FunctionTok{st\_read\_parquet}\NormalTok{(}\StringTok{"./Geodata/2024/TopographicMap/SwampA\_COMB.parquet"}\NormalTok{)}
\NormalTok{topo\_purvi}\OtherTok{=}\NormalTok{topo\_purvi }\SpecialCharTok{\%\textgreater{}\%} 
  \FunctionTok{mutate}\NormalTok{(}\AttributeTok{yes=}\DecValTok{710}\NormalTok{) }\SpecialCharTok{\%\textgreater{}\%} 
\NormalTok{  dplyr}\SpecialCharTok{::}\FunctionTok{select}\NormalTok{(yes)}
\NormalTok{purvi}\OtherTok{=}\FunctionTok{rbind}\NormalTok{(purvi\_topo,topo\_purvi)}
\NormalTok{r\_purvi\_topo}\OtherTok{=}\FunctionTok{fasterize}\NormalTok{(purvi,template\_r,}\AttributeTok{field=}\StringTok{"yes"}\NormalTok{)}
\NormalTok{raster}\SpecialCharTok{::}\FunctionTok{writeRaster}\NormalTok{(r\_purvi\_topo,}
                    \StringTok{"./RasterGrids\_10m/2024/SimpleLandscape\_class710\_purvi\_topo.tif"}\NormalTok{,}
                    \AttributeTok{progress=}\StringTok{"text"}\NormalTok{,}
                    \AttributeTok{overwrite=}\ConstantTok{TRUE}\NormalTok{)}
\CommentTok{\# cleaning}
\FunctionTok{rm}\NormalTok{(purvi\_topo)}
\FunctionTok{rm}\NormalTok{(topo\_purvi)}
\FunctionTok{rm}\NormalTok{(purvi)}
\FunctionTok{rm}\NormalTok{(r\_purvi\_topo)}


\CommentTok{\# mvr}
\NormalTok{mvr}\OtherTok{=}\FunctionTok{st\_read\_parquet}\NormalTok{(}\StringTok{"./Geodata/2024/MVR/nogabali\_2024janv.parquet"}\NormalTok{)}

\CommentTok{\# bogs and mires}
\NormalTok{mvr\_purvi}\OtherTok{=}\NormalTok{mvr }\SpecialCharTok{\%\textgreater{}\%} 
  \FunctionTok{filter}\NormalTok{(zkat }\SpecialCharTok{\%in\%} \FunctionTok{c}\NormalTok{(}\StringTok{"21"}\NormalTok{,}\StringTok{"22"}\NormalTok{,}\StringTok{"23"}\NormalTok{)) }\SpecialCharTok{\%\textgreater{}\%} 
  \FunctionTok{mutate}\NormalTok{(}\AttributeTok{yes=}\DecValTok{710}\NormalTok{) }\SpecialCharTok{\%\textgreater{}\%} 
\NormalTok{  dplyr}\SpecialCharTok{::}\FunctionTok{select}\NormalTok{(yes)}
\NormalTok{r\_purvi\_mvr}\OtherTok{=}\FunctionTok{fasterize}\NormalTok{(mvr\_purvi,template\_r,}\AttributeTok{field=}\StringTok{"yes"}\NormalTok{)}
\NormalTok{raster}\SpecialCharTok{::}\FunctionTok{writeRaster}\NormalTok{(r\_purvi\_mvr,}
                    \StringTok{"./RasterGrids\_10m/2024/SimpleLandscape\_class710\_purvi\_mvr.tif"}\NormalTok{,}
                    \AttributeTok{progress=}\StringTok{"text"}\NormalTok{,}
                    \AttributeTok{overwrite=}\ConstantTok{TRUE}\NormalTok{)}
\CommentTok{\# cleaning}
\FunctionTok{rm}\NormalTok{(mvr\_purvi)}
\FunctionTok{rm}\NormalTok{(r\_purvi\_mvr)}

\CommentTok{\# beavers}
\NormalTok{mvr\_bebri}\OtherTok{=}\NormalTok{mvr }\SpecialCharTok{\%\textgreater{}\%} 
  \FunctionTok{filter}\NormalTok{(zkat }\SpecialCharTok{\%in\%} \FunctionTok{c}\NormalTok{(}\StringTok{"41"}\NormalTok{,}\StringTok{"42"}\NormalTok{)) }\SpecialCharTok{\%\textgreater{}\%} 
  \FunctionTok{mutate}\NormalTok{(}\AttributeTok{yes=}\DecValTok{730}\NormalTok{) }\SpecialCharTok{\%\textgreater{}\%} 
\NormalTok{  dplyr}\SpecialCharTok{::}\FunctionTok{select}\NormalTok{(yes)}
\NormalTok{r\_bebri\_mvr}\OtherTok{=}\FunctionTok{fasterize}\NormalTok{(mvr\_bebri,template\_r,}\AttributeTok{field=}\StringTok{"yes"}\NormalTok{)}
\NormalTok{raster}\SpecialCharTok{::}\FunctionTok{writeRaster}\NormalTok{(r\_bebri\_mvr,}
                    \StringTok{"./RasterGrids\_10m/2024/SimpleLandscape\_class730\_bebri\_mvr.tif"}\NormalTok{,}
                    \AttributeTok{progress=}\StringTok{"text"}\NormalTok{,}
                    \AttributeTok{overwrite=}\ConstantTok{TRUE}\NormalTok{)}
\CommentTok{\# cleaning}
\FunctionTok{rm}\NormalTok{(mvr\_bebri)}
\FunctionTok{rm}\NormalTok{(r\_bebri\_mvr)}
\FunctionTok{rm}\NormalTok{(mvr)}



\CommentTok{\# merging}
\NormalTok{r\_niedraji\_topo}\OtherTok{=}\FunctionTok{rast}\NormalTok{(}\StringTok{"./RasterGrids\_10m/2024/SimpleLandscape\_class720\_niedraji\_topo.tif"}\NormalTok{)}
\NormalTok{r\_purvi\_topo}\OtherTok{=}\FunctionTok{rast}\NormalTok{(}\StringTok{"./RasterGrids\_10m/2024/SimpleLandscape\_class710\_purvi\_topo.tif"}\NormalTok{)}
\NormalTok{r\_purvi\_mvr}\OtherTok{=}\FunctionTok{rast}\NormalTok{(}\StringTok{"./RasterGrids\_10m/2024/SimpleLandscape\_class710\_purvi\_mvr.tif"}\NormalTok{)}
\NormalTok{r\_bebri\_mvr}\OtherTok{=}\FunctionTok{rast}\NormalTok{(}\StringTok{"./RasterGrids\_10m/2024/SimpleLandscape\_class730\_bebri\_mvr.tif"}\NormalTok{)}
\NormalTok{mires}\OtherTok{=}\FunctionTok{rast}\NormalTok{(}\StringTok{"./RasterGrids\_10m/2024/EDI\_TransitionalMiresYN.tif"}\NormalTok{)}
\NormalTok{miresY}\OtherTok{=}\FunctionTok{ifel}\NormalTok{(mires}\SpecialCharTok{==}\DecValTok{1}\NormalTok{,}\DecValTok{710}\NormalTok{,}\ConstantTok{NA}\NormalTok{)}
\NormalTok{bogs}\OtherTok{=}\FunctionTok{rast}\NormalTok{(}\StringTok{"./RasterGrids\_10m/2024/EDI\_BogsYN.tif"}\NormalTok{)}
\NormalTok{bogsY}\OtherTok{=}\FunctionTok{ifel}\NormalTok{(bogs}\SpecialCharTok{==}\DecValTok{1}\NormalTok{,}\DecValTok{710}\NormalTok{,}\ConstantTok{NA}\NormalTok{)}

\NormalTok{wetlands\_cover}\OtherTok{=}\FunctionTok{cover}\NormalTok{(r\_niedraji\_topo,r\_purvi\_topo)}
\NormalTok{wetlands\_cover}\OtherTok{=}\FunctionTok{cover}\NormalTok{(wetlands\_cover,r\_purvi\_mvr)}
\NormalTok{wetlands\_cover}\OtherTok{=}\FunctionTok{cover}\NormalTok{(wetlands\_cover,r\_bebri\_mvr)}
\NormalTok{wetlands\_cover}\OtherTok{=}\FunctionTok{cover}\NormalTok{(wetlands\_cover,miresY)}
\NormalTok{wetlands\_cover}\OtherTok{=}\FunctionTok{cover}\NormalTok{(wetlands\_cover,}
\NormalTok{                     bogsY,}
                     \AttributeTok{filename=}\FunctionTok{paste0}\NormalTok{(}\StringTok{"./RasterGrids\_10m/2024/"}\NormalTok{,}
                                     \StringTok{"SimpleLandscape\_class700\_mitraji\_premask.tif"}\NormalTok{),}
                            \AttributeTok{overwrite=}\ConstantTok{TRUE}\NormalTok{)}
\CommentTok{\# cleaning}
\FunctionTok{rm}\NormalTok{(r\_niedraji\_topo)}
\FunctionTok{rm}\NormalTok{(r\_purvi\_topo)}
\FunctionTok{rm}\NormalTok{(r\_purvi\_mvr)}
\FunctionTok{rm}\NormalTok{(r\_bebri\_mvr)}
\FunctionTok{rm}\NormalTok{(bogs)}
\FunctionTok{rm}\NormalTok{(bogsY)}
\FunctionTok{rm}\NormalTok{(mires)}
\FunctionTok{rm}\NormalTok{(miresY)}
\FunctionTok{rm}\NormalTok{(topo)}
\FunctionTok{rm}\NormalTok{(wetlands\_cover)}

\FunctionTok{unlink}\NormalTok{(}\StringTok{"./RasterGrids\_10m/2024/SimpleLandscape\_class710\_purvi\_topo.tif"}\NormalTok{)}
\FunctionTok{unlink}\NormalTok{(}\StringTok{"./RasterGrids\_10m/2024/SimpleLandscape\_class710\_purvi\_mvr.tif"}\NormalTok{)}
\FunctionTok{unlink}\NormalTok{(}\StringTok{"./RasterGrids\_10m/2024/SimpleLandscape\_class730\_bebri\_mvr.tif"}\NormalTok{)}
\end{Highlighting}
\end{Shaded}

\begin{itemize}
\item
  Class \texttt{800} - \textbf{Bare Soil and Quarries}: combining layers related to bare soil,
  heaths, and quarries. The following have been
  combined to create this class (in order of overlap):

  -- \hyperref[Ch04.04]{topographic map} layer \texttt{LandusA\_COMB} classes:
  ``poligons\_Smiltājs'', ``poligons\_Smiltajs'', ``poligons\_Grants'', ``poligons\_Kūdra'',
  ``poligons\_Virsajs'', the result of which is coded with \texttt{800};

  -- land categories ``33'' and ``34'' marked in the \hyperref[Ch04.01]{State Forest Register},
  the result of which is coded as \texttt{800}.
\end{itemize}

The command lines below create a layer with landscape class \texttt{800}, which is saved
in the file \texttt{SimpleLandscape\_class800\_smiltaji\_premask.tif} for further processing.

\begin{Shaded}
\begin{Highlighting}[]
\CommentTok{\# Libs {-}{-}{-}{-}}
\ControlFlowTok{if}\NormalTok{(}\SpecialCharTok{!}\FunctionTok{require}\NormalTok{(tidyverse)) \{}\FunctionTok{install.packages}\NormalTok{(}\StringTok{"tidyverse"}\NormalTok{); }\FunctionTok{require}\NormalTok{(tidyverse)\}}
\ControlFlowTok{if}\NormalTok{(}\SpecialCharTok{!}\FunctionTok{require}\NormalTok{(sf)) \{}\FunctionTok{install.packages}\NormalTok{(}\StringTok{"sf"}\NormalTok{); }\FunctionTok{require}\NormalTok{(sf)\}}
\ControlFlowTok{if}\NormalTok{(}\SpecialCharTok{!}\FunctionTok{require}\NormalTok{(arrow)) \{}\FunctionTok{install.packages}\NormalTok{(}\StringTok{"arrow"}\NormalTok{); }\FunctionTok{require}\NormalTok{(arrow)\}}
\ControlFlowTok{if}\NormalTok{(}\SpecialCharTok{!}\FunctionTok{require}\NormalTok{(sfarrow)) \{}\FunctionTok{install.packages}\NormalTok{(}\StringTok{"sfarrow"}\NormalTok{); }\FunctionTok{require}\NormalTok{(sfarrow)\}}
\ControlFlowTok{if}\NormalTok{(}\SpecialCharTok{!}\FunctionTok{require}\NormalTok{(terra)) \{}\FunctionTok{install.packages}\NormalTok{(}\StringTok{"terra"}\NormalTok{); }\FunctionTok{require}\NormalTok{(terra)\}}
\ControlFlowTok{if}\NormalTok{(}\SpecialCharTok{!}\FunctionTok{require}\NormalTok{(raster)) \{}\FunctionTok{install.packages}\NormalTok{(}\StringTok{"raster"}\NormalTok{); }\FunctionTok{require}\NormalTok{(raster)\}}
\ControlFlowTok{if}\NormalTok{(}\SpecialCharTok{!}\FunctionTok{require}\NormalTok{(fasterize)) \{}\FunctionTok{install.packages}\NormalTok{(}\StringTok{"fasterize"}\NormalTok{); }\FunctionTok{require}\NormalTok{(fasterize)\}}
\ControlFlowTok{if}\NormalTok{(}\SpecialCharTok{!}\FunctionTok{require}\NormalTok{(readxl)) \{}\FunctionTok{install.packages}\NormalTok{(}\StringTok{"readxl"}\NormalTok{); }\FunctionTok{require}\NormalTok{(readxl)\}}

\CommentTok{\# templates {-}{-}{-}{-}}
\NormalTok{template\_t}\OtherTok{=}\FunctionTok{rast}\NormalTok{(}\StringTok{"./Templates/TemplateRasters/LV10m\_10km.tif"}\NormalTok{)}
\NormalTok{template\_r}\OtherTok{=}\FunctionTok{raster}\NormalTok{(template\_t)}


\CommentTok{\# class 800 {-}{-}{-}{-}}

\NormalTok{smiltaji\_topo}\OtherTok{=}\FunctionTok{st\_read\_parquet}\NormalTok{(}\StringTok{"./Geodata/2024/TopographicMap/LandusA\_COMB.parquet"}\NormalTok{)}
\FunctionTok{table}\NormalTok{(smiltaji\_topo}\SpecialCharTok{$}\NormalTok{FNAME,}\AttributeTok{useNA=}\StringTok{"always"}\NormalTok{)}
\NormalTok{smiltaji\_topo}\OtherTok{=}\NormalTok{smiltaji\_topo }\SpecialCharTok{\%\textgreater{}\%} 
  \FunctionTok{filter}\NormalTok{(FNAME }\SpecialCharTok{\%in\%} \FunctionTok{c}\NormalTok{(}\StringTok{"poligons\_Smiltājs"}\NormalTok{,}\StringTok{"poligons\_Smiltajs"}\NormalTok{,}\StringTok{"poligons\_Grants"}\NormalTok{,}
                      \StringTok{"poligons\_Kūdra"}\NormalTok{,}\StringTok{"poligons\_Virsajs"}\NormalTok{)) }\SpecialCharTok{\%\textgreater{}\%} 
  \FunctionTok{mutate}\NormalTok{(}\AttributeTok{yes=}\DecValTok{800}\NormalTok{) }\SpecialCharTok{\%\textgreater{}\%} 
\NormalTok{  dplyr}\SpecialCharTok{::}\FunctionTok{select}\NormalTok{(yes)}
\NormalTok{r\_smiltaji\_topo}\OtherTok{=}\FunctionTok{fasterize}\NormalTok{(smiltaji\_topo,template\_r,}\AttributeTok{field=}\StringTok{"yes"}\NormalTok{)}
\NormalTok{raster}\SpecialCharTok{::}\FunctionTok{writeRaster}\NormalTok{(r\_smiltaji\_topo,}
                    \StringTok{"./RasterGrids\_10m/2024/SimpleLandscape\_class800\_SmiltajiKudra\_topo.tif"}\NormalTok{,}
                    \AttributeTok{progress=}\StringTok{"text"}\NormalTok{)}
\CommentTok{\# cleaning}
\FunctionTok{rm}\NormalTok{(smiltaji\_topo)}
\FunctionTok{rm}\NormalTok{(r\_smiltaji\_topo)}

\CommentTok{\# mvr zkat 33 un 34}
\NormalTok{mvr}\OtherTok{=}\FunctionTok{st\_read\_parquet}\NormalTok{(}\StringTok{"./Geodata/2024/MVR/nogabali\_2024janv.parquet"}\NormalTok{)}

\NormalTok{smiltajiem}\OtherTok{=}\NormalTok{mvr }\SpecialCharTok{\%\textgreater{}\%} 
  \FunctionTok{filter}\NormalTok{(zkat }\SpecialCharTok{\%in\%} \FunctionTok{c}\NormalTok{(}\StringTok{"33"}\NormalTok{,}\StringTok{"34"}\NormalTok{)) }\SpecialCharTok{\%\textgreater{}\%} 
  \FunctionTok{mutate}\NormalTok{(}\AttributeTok{yes=}\DecValTok{800}\NormalTok{) }\SpecialCharTok{\%\textgreater{}\%} 
\NormalTok{  dplyr}\SpecialCharTok{::}\FunctionTok{select}\NormalTok{(yes)}
\NormalTok{r\_smiltaji\_mvr}\OtherTok{=}\FunctionTok{fasterize}\NormalTok{(smiltajiem,template\_r,}\AttributeTok{field=}\StringTok{"yes"}\NormalTok{)}
\NormalTok{raster}\SpecialCharTok{::}\FunctionTok{writeRaster}\NormalTok{(r\_smiltaji\_mvr,}
                    \StringTok{"./RasterGrids\_10m/2024/SimpleLandscape\_class800\_SmiltVirs\_mvr.tif"}\NormalTok{,}
                    \AttributeTok{progress=}\StringTok{"text"}\NormalTok{,}
                    \AttributeTok{overwrite=}\ConstantTok{TRUE}\NormalTok{)}
\CommentTok{\# cleaning}
\FunctionTok{rm}\NormalTok{(mvr)}
\FunctionTok{rm}\NormalTok{(smiltajiem)}
\FunctionTok{rm}\NormalTok{(r\_smiltaji\_mvr)}

\CommentTok{\# merging}
\NormalTok{r\_smiltaji\_topo}\OtherTok{=}\FunctionTok{rast}\NormalTok{(}\StringTok{"./RasterGrids\_10m/2024/SimpleLandscape\_class800\_SmiltajiKudra\_topo.tif"}\NormalTok{)}
\NormalTok{r\_smiltaji\_mvr}\OtherTok{=}\FunctionTok{rast}\NormalTok{(}\StringTok{"./RasterGrids\_10m/2024/SimpleLandscape\_class800\_SmiltVirs\_mvr.tif"}\NormalTok{)}

\NormalTok{bare\_cover}\OtherTok{=}\NormalTok{terra}\SpecialCharTok{::}\FunctionTok{merge}\NormalTok{(r\_smiltaji\_topo,}
\NormalTok{                        r\_smiltaji\_mvr,}
                        \AttributeTok{filename=}\FunctionTok{paste0}\NormalTok{(}\StringTok{"./RasterGrids\_10m/2024/"}\NormalTok{,}
                                        \StringTok{"SimpleLandscape\_class800\_smiltaji\_premask.tif"}\NormalTok{),}
                               \AttributeTok{overwrite=}\ConstantTok{TRUE}\NormalTok{)}
\CommentTok{\# cleaning}
\FunctionTok{rm}\NormalTok{(r\_smiltaji\_topo)}
\FunctionTok{rm}\NormalTok{(r\_smiltaji\_mvr)}
\FunctionTok{rm}\NormalTok{(bare\_cover)}

\FunctionTok{unlink}\NormalTok{(}\StringTok{"./RasterGrids\_10m/2024/SimpleLandscape\_class800\_SmiltajiKudra\_topo.tif"}\NormalTok{)}
\FunctionTok{unlink}\NormalTok{(}\StringTok{"./RasterGrids\_10m/2024/SimpleLandscape\_class800\_SmiltVirs\_mvr.tif"}\NormalTok{)}
\end{Highlighting}
\end{Shaded}

\textbf{Merging and filling}

The command lines below combine the previously created layers with the landscape
classes in the correct order and ensure that gaps are filled with the appropriately
classified Dynamic World composite for April-August 2024. After masking to match\\
the analysis space, the layer is saved in the file \texttt{Ainava\_vienk\_mask.tif}
for further processing.

\begin{Shaded}
\begin{Highlighting}[]
\CommentTok{\# Libs {-}{-}{-}{-}}
\ControlFlowTok{if}\NormalTok{(}\SpecialCharTok{!}\FunctionTok{require}\NormalTok{(terra)) \{}\FunctionTok{install.packages}\NormalTok{(}\StringTok{"terra"}\NormalTok{); }\FunctionTok{require}\NormalTok{(terra)\}}

\CommentTok{\# templates {-}{-}{-}{-}}
\NormalTok{template\_t}\OtherTok{=}\FunctionTok{rast}\NormalTok{(}\StringTok{"./Templates/TemplateRasters/LV10m\_10km.tif"}\NormalTok{)}
\NormalTok{template\_r}\OtherTok{=}\FunctionTok{raster}\NormalTok{(template\_t)}


\CommentTok{\# final merging and covering {-}{-}{-}{-}}

\CommentTok{\# DW  }
\NormalTok{dynworld}\OtherTok{=}\FunctionTok{rast}\NormalTok{(}\StringTok{"Geodata/2024/DynamicWorld/DW\_2024\_apraug.tif"}\NormalTok{)}
\NormalTok{klases}\OtherTok{=}\FunctionTok{matrix}\NormalTok{(}\FunctionTok{c}\NormalTok{(}\DecValTok{0}\NormalTok{,}\DecValTok{200}\NormalTok{,}
                \DecValTok{1}\NormalTok{,}\DecValTok{620}\NormalTok{,}
                \DecValTok{2}\NormalTok{,}\DecValTok{330}\NormalTok{,}
                \DecValTok{3}\NormalTok{,}\DecValTok{720}\NormalTok{,}
                \DecValTok{4}\NormalTok{,}\DecValTok{310}\NormalTok{,}
                \DecValTok{5}\NormalTok{,}\DecValTok{710}\NormalTok{,}
                \DecValTok{6}\NormalTok{,}\DecValTok{500}\NormalTok{,}
                \DecValTok{7}\NormalTok{,}\DecValTok{800}\NormalTok{,}
                \DecValTok{8}\NormalTok{,}\DecValTok{500}\NormalTok{),}\AttributeTok{ncol=}\DecValTok{2}\NormalTok{,}\AttributeTok{byrow=}\ConstantTok{TRUE}\NormalTok{)}
\NormalTok{dw2}\OtherTok{=}\NormalTok{terra}\SpecialCharTok{::}\FunctionTok{classify}\NormalTok{(dynworld,klases)}
\FunctionTok{writeRaster}\NormalTok{(dw2,}
            \StringTok{"./RasterGrids\_10m/2024/DW\_reclass.tif"}\NormalTok{,}
            \AttributeTok{overwrite=}\ConstantTok{TRUE}\NormalTok{)}
\CommentTok{\# other layers}
\NormalTok{celi}\OtherTok{=}\FunctionTok{rast}\NormalTok{(}\StringTok{"./RasterGrids\_10m/2024/SimpleLandscape\_class100\_celi.tif"}\NormalTok{)}
\FunctionTok{plot}\NormalTok{(celi)}

\NormalTok{niedraji}\OtherTok{=}\FunctionTok{rast}\NormalTok{(}\StringTok{"RasterGrids\_10m/2024/SimpleLandscape\_class720\_niedraji\_topo.tif"}\NormalTok{)}
\FunctionTok{plot}\NormalTok{(niedraji)}

\NormalTok{udeni}\OtherTok{=}\FunctionTok{rast}\NormalTok{(}\StringTok{"./RasterGrids\_10m/2024/SimpleLandscape\_class200\_udens\_premask.tif"}\NormalTok{)}
\FunctionTok{plot}\NormalTok{(udeni)}

\NormalTok{lauki}\OtherTok{=}\FunctionTok{rast}\NormalTok{(}\StringTok{"./RasterGrids\_10m/2024/SimpleLandscape\_class300\_lauki\_premask.tif"}\NormalTok{)}
\FunctionTok{plot}\NormalTok{(lauki)}

\NormalTok{vasarnicas}\OtherTok{=}\FunctionTok{rast}\NormalTok{(}\StringTok{"./RasterGrids\_10m/2024/SimpleLandscape\_class400\_varnicas\_premask.tif"}\NormalTok{)}
\FunctionTok{plot}\NormalTok{(vasarnicas)}

\NormalTok{mezi}\OtherTok{=}\FunctionTok{rast}\NormalTok{(}\StringTok{"./RasterGrids\_10m/2024/SimpleLandscape\_class600\_meziem\_premask.tif"}\NormalTok{)}
\FunctionTok{plot}\NormalTok{(mezi)}

\NormalTok{mitraji}\OtherTok{=}\FunctionTok{rast}\NormalTok{(}\StringTok{"./RasterGrids\_10m/2024/SimpleLandscape\_class700\_mitraji\_premask.tif"}\NormalTok{)}
\FunctionTok{plot}\NormalTok{(mitraji)}

\NormalTok{smiltaji}\OtherTok{=}\FunctionTok{rast}\NormalTok{(}\StringTok{"./RasterGrids\_10m/2024/SimpleLandscape\_class800\_smiltaji\_premask.tif"}\NormalTok{)}
\FunctionTok{plot}\NormalTok{(smiltaji)}

\NormalTok{dw2}\OtherTok{=}\FunctionTok{rast}\NormalTok{(}\StringTok{"./RasterGrids\_10m/2024/DW\_reclass.tif"}\NormalTok{)}
\FunctionTok{plot}\NormalTok{(dw2)}

\CommentTok{\# covering in correct order}
\NormalTok{rastri\_ainavai}\OtherTok{=}\FunctionTok{cover}\NormalTok{(celi,niedraji)}
\NormalTok{rastri\_ainavai}\OtherTok{=}\FunctionTok{cover}\NormalTok{(rastri\_ainavai,udeni)}
\NormalTok{rastri\_ainavai}\OtherTok{=}\FunctionTok{cover}\NormalTok{(rastri\_ainavai,lauki)}
\NormalTok{rastri\_ainavai}\OtherTok{=}\FunctionTok{cover}\NormalTok{(rastri\_ainavai,vasarnicas)}
\NormalTok{rastri\_ainavai}\OtherTok{=}\FunctionTok{cover}\NormalTok{(rastri\_ainavai,mezi)}
\NormalTok{rastri\_ainavai}\OtherTok{=}\FunctionTok{cover}\NormalTok{(rastri\_ainavai,mitraji)}
\NormalTok{rastri\_ainavai}\OtherTok{=}\FunctionTok{cover}\NormalTok{(rastri\_ainavai,smiltaji)}
\NormalTok{rastri\_ainavai}\OtherTok{=}\FunctionTok{cover}\NormalTok{(rastri\_ainavai,dw2,}
                           \AttributeTok{filename=}\StringTok{"./RasterGrids\_10m/2024/Ainava\_vienkarsa.tif"}\NormalTok{,}
                           \AttributeTok{overwrite=}\ConstantTok{TRUE}\NormalTok{)}
\FunctionTok{plot}\NormalTok{(rastri\_ainavai)}

\CommentTok{\# cleaning}
\FunctionTok{rm}\NormalTok{(celi)}
\FunctionTok{rm}\NormalTok{(niedraji)}
\FunctionTok{rm}\NormalTok{(udeni)}
\FunctionTok{rm}\NormalTok{(lauki)}
\FunctionTok{rm}\NormalTok{(vasarnicas)}
\FunctionTok{rm}\NormalTok{(mezi)}
\FunctionTok{rm}\NormalTok{(mitraji)}
\FunctionTok{rm}\NormalTok{(smiltaji)}
\FunctionTok{rm}\NormalTok{(klases)}
\FunctionTok{rm}\NormalTok{(dynworld)}
\FunctionTok{rm}\NormalTok{(dw2)}
\FunctionTok{rm}\NormalTok{(rastri\_ainavai)}

\CommentTok{\# masking}
\NormalTok{rastrs\_ainava}\OtherTok{=}\FunctionTok{rast}\NormalTok{(}\StringTok{"./RasterGrids\_10m/2024/Ainava\_vienkarsa.tif"}\NormalTok{)}
\FunctionTok{plot}\NormalTok{(rastrs\_ainava)}
\FunctionTok{freq}\NormalTok{(rastrs\_ainava)}

\NormalTok{masketa\_ainava}\OtherTok{=}\NormalTok{terra}\SpecialCharTok{::}\FunctionTok{mask}\NormalTok{(rastrs\_ainava,}
\NormalTok{                           template\_t,}
                           \AttributeTok{filename=}\StringTok{"./RasterGrids\_10m/2024/Ainava\_vienk\_mask.tif"}\NormalTok{,}
                           \AttributeTok{overwrite=}\ConstantTok{TRUE}\NormalTok{)}
\FunctionTok{plot}\NormalTok{(masketa\_ainava)}

\CommentTok{\# cleaning}
\FunctionTok{rm}\NormalTok{(rastrs\_ainava)}
\FunctionTok{rm}\NormalTok{(masketa\_ainava)}
\end{Highlighting}
\end{Shaded}

\section{Landscape diversity}\label{Ch05.04}

This subsection summarizes the input products related to the landscape described
in the previous section -- raster layers prepared at a 10 m resolution, which
characterize the classes found in the landscape (environment), as well as the subsequent
preprocessing for the preparation of the EGVs.

The calculations of the Shannon diversity index are so computationally intensive that it is not
rationally possible to perform them at every landscape scale around each analysis
cell (EGV-cell). Furthermore, they cannot be directly aggregated to speed up the calculation. Therefore,
a decision has been made on the raster cell size, which:

\begin{itemize}
\item
  is formed as a multiplication of the EGV-cell by an integer;
\item
  is large enough to account for environmental variability. Therefore, the EGV-cell
  itself (or multiplication by 1) is not suitable - there is very little
  variability in land cover and land use within an area of 1 ha. Consequently, the
  raster cell size for calculation of Shannon index should be as large as possible
  without becoming so large that it artificially inflates spatial autocorrelation
  and loose spatial relevance;
\item
  allows to build every landscape scale from several diversity-index--level cells.
\end{itemize}

Since we use spatially weighted zonal statistics in the preparation of EGVs,
and the smallest landscape scale is r = 500 m around the centre of the EGV-cell,
it has been decided to calculate the landscape diversity index for individual cells
with a side length of 500 m (i.e., 25 ha landscapes). This means that the
smallest number of units used for the development of the EGVs is nine (for a
landscape scale of r = 500 m around the centre of the EGV-cell).

Three principal environments are described using diversity indices: overall landscape,
farmland, and forests. To make them easier to reproduce and locate, each
is described in a separate section below.

\subsection{Overall landscape}\label{Ch05.04.01}

Combination of three layers is involved to describe overall landscape diversity:

\begin{itemize}
\item
  as the lowest in hierarchy is \texttt{Ainava\_vienk\_mask.tif}, prepared in section
  \hyperref[Ch05.03]{Landscape classification};
\item
  farmland diversity as the top layer in the hierarhy. Prepared based on relatively
  broad \href{https://github.com/aavotins/HiQBioDiv_EGVs/blob/main/Data/Geodata/2024/LAD/KulturuKodi_2024.xlsx}{agricultural codes (field - SDM\_grupa\_sakums)} from \hyperref[Ch04.02]{Rural Support Service's information on declared fields}. Only cells
  corresponding to declared fields contain values; others are empty will inherit values
  from other layers during overlay. Codes used range from 351 to 362;
\item
  forest diversity is the second layer in hierarchy. This layer describes dominant tree
  species groups in each stand with stand, derived from stand-level inventory data combined with
  age group as used in forestry practice. Values used in this classification are
  available from \href{https://www.vmd.gov.lv/lv/meza-valsts-registra-meza-inventarizacijas-failu-struktura}{database description}.

  \begin{itemize}
  \item
    tree species groups:

    \begin{itemize}
    \item
      coniferous species codes: ``1'', ``3'', ``13'', ``14'', ``15'', ``22'', ``23'', ``28'';
    \item
      boreal deciduous species codes: ``4'', ``6'', ``8'', ``9'', ``19'', ``20'', ``21'',
      ``32'', ``35'', ``68'';
    \item
      temperate deciduous species codes: ``10'', ``11'', ``12'', ``16'', ``17'', ``18'',
      ``24'', ``25'', ``26'', ``27'', ``28'', ``29'', ``50'', ``61'', ``62'', ``63'', ``64'', ``65'', ``66'',
      ``67'', ``69'';
    \item
      classification: a forest is considered coniferous if timber volume of
      coniferous species in the top tree layer constitutes at least 75\% of the total timer
      volume. Otherwise, it can be considered boreal deciduous if the respective
      proportion is at least 75\%, or temperate deciduous
      if the respective proportion is at least 50\%; else it is considered mixed.
    \end{itemize}
  \item
    tree age groups:

    \begin{itemize}
    \item
      forests are considered young if they are registered with age groups
      ``1'', ``2'' or ``3'';
    \item
      forests are considered old if they are registered with age groups
      ``4'', or ``5'';
    \end{itemize}
  \item
    created codes are formatted as factors and then again as scalars, with 660 added.
  \end{itemize}
\end{itemize}

Once the landscape classification is done, diversity index is calculated for 25 ha
landscapes using the function \texttt{egvtools::landscape\_function}. To guard value coverage,
inverse distance weighted (power = 2) gap filling is incorporated; however,
there were no gaps to fill.

\begin{Shaded}
\begin{Highlighting}[]
\CommentTok{\# Libs {-}{-}{-}{-}}
\ControlFlowTok{if}\NormalTok{(}\SpecialCharTok{!}\FunctionTok{require}\NormalTok{(egvtools)) \{}\FunctionTok{install.packages}\NormalTok{(}\StringTok{"egvtools"}\NormalTok{); }\FunctionTok{require}\NormalTok{(egvtools)\}}
\ControlFlowTok{if}\NormalTok{(}\SpecialCharTok{!}\FunctionTok{require}\NormalTok{(tidyverse)) \{}\FunctionTok{install.packages}\NormalTok{(}\StringTok{"tidyverse"}\NormalTok{); }\FunctionTok{require}\NormalTok{(tidyverse)\}}
\ControlFlowTok{if}\NormalTok{(}\SpecialCharTok{!}\FunctionTok{require}\NormalTok{(sf)) \{}\FunctionTok{install.packages}\NormalTok{(}\StringTok{"sf"}\NormalTok{); }\FunctionTok{require}\NormalTok{(sf)\}}
\ControlFlowTok{if}\NormalTok{(}\SpecialCharTok{!}\FunctionTok{require}\NormalTok{(arrow)) \{}\FunctionTok{install.packages}\NormalTok{(}\StringTok{"arrow"}\NormalTok{); }\FunctionTok{require}\NormalTok{(arrow)\}}
\ControlFlowTok{if}\NormalTok{(}\SpecialCharTok{!}\FunctionTok{require}\NormalTok{(sfarrow)) \{}\FunctionTok{install.packages}\NormalTok{(}\StringTok{"sfarrow"}\NormalTok{); }\FunctionTok{require}\NormalTok{(sfarrow)\}}
\ControlFlowTok{if}\NormalTok{(}\SpecialCharTok{!}\FunctionTok{require}\NormalTok{(terra)) \{}\FunctionTok{install.packages}\NormalTok{(}\StringTok{"terra"}\NormalTok{); }\FunctionTok{require}\NormalTok{(terra)\}}
\ControlFlowTok{if}\NormalTok{(}\SpecialCharTok{!}\FunctionTok{require}\NormalTok{(readxl)) \{}\FunctionTok{install.packages}\NormalTok{(}\StringTok{"readxl"}\NormalTok{); }\FunctionTok{require}\NormalTok{(readxl)\}}

\CommentTok{\# templates {-}{-}{-}{-}}
\NormalTok{template\_t}\OtherTok{=}\FunctionTok{rast}\NormalTok{(}\StringTok{"./Templates/TemplateRasters/LV10m\_10km.tif"}\NormalTok{)}
\NormalTok{template\_r}\OtherTok{=}\FunctionTok{raster}\NormalTok{(template\_t)}


\CommentTok{\# overall diversity {-}{-}{-}{-}}

\DocumentationTok{\#\# Farmland broad {-}{-}{-}{-}}

\CommentTok{\# classification }
\NormalTok{culturecodes}\OtherTok{=}\FunctionTok{read\_excel}\NormalTok{(}\StringTok{"./Geodata/2024/LAD/KulturuKodi\_2024.xlsx"}\NormalTok{)}
\NormalTok{culturecodes}\SpecialCharTok{$}\NormalTok{kods}\OtherTok{=}\FunctionTok{as.character}\NormalTok{(culturecodes}\SpecialCharTok{$}\NormalTok{kods)}
\NormalTok{lad}\OtherTok{=}\NormalTok{sfarrow}\SpecialCharTok{::}\FunctionTok{st\_read\_parquet}\NormalTok{(}\StringTok{"./Geodata/2024/LAD/Lauki\_2024.parquet"}\NormalTok{)}
\NormalTok{lad2}\OtherTok{=}\NormalTok{lad }\SpecialCharTok{\%\textgreater{}\%} 
  \FunctionTok{left\_join}\NormalTok{(culturecodes, }\AttributeTok{by=}\FunctionTok{c}\NormalTok{(}\StringTok{"PRODUCT\_CODE"}\OtherTok{=}\StringTok{"kods"}\NormalTok{)) }\SpecialCharTok{\%\textgreater{}\%} 
  \FunctionTok{mutate}\NormalTok{(}\AttributeTok{numeric\_code=}\FunctionTok{as.numeric}\NormalTok{(}\FunctionTok{as.factor}\NormalTok{(SDM\_grupa\_sakums))}\SpecialCharTok{+}\DecValTok{350}\NormalTok{) }\SpecialCharTok{\%\textgreater{}\%} 
  \FunctionTok{filter}\NormalTok{(}\SpecialCharTok{!}\FunctionTok{is.na}\NormalTok{(numeric\_code))}
\FunctionTok{table}\NormalTok{(lad2}\SpecialCharTok{$}\NormalTok{numeric\_code,}\AttributeTok{useNA =} \StringTok{"always"}\NormalTok{)}

\CommentTok{\# input layer}
\FunctionTok{polygon2input}\NormalTok{(}\AttributeTok{vector\_data =}\NormalTok{ lad2,}
              \AttributeTok{template\_path =} \StringTok{"./Templates/TemplateRasters/LV10m\_10km.tif"}\NormalTok{,}
              \AttributeTok{out\_path =} \StringTok{"./RasterGrids\_10m/2024/"}\NormalTok{,}
              \AttributeTok{file\_name =} \StringTok{"Diversity\_FarmlandBroad\_only.tif"}\NormalTok{,}
              \AttributeTok{value\_field =} \StringTok{"numeric\_code"}\NormalTok{,}
              \AttributeTok{fun=}\StringTok{"first"}\NormalTok{,}
              \AttributeTok{prepare=}\ConstantTok{FALSE}\NormalTok{,}
              \AttributeTok{project\_mode =} \StringTok{"auto"}\NormalTok{)}
\CommentTok{\# cleaning}
\FunctionTok{rm}\NormalTok{(culturecodes)}
\FunctionTok{rm}\NormalTok{(lad)}
\FunctionTok{rm}\NormalTok{(lad2)}


\DocumentationTok{\#\# Forests broad {-}{-}{-}{-}}

\CommentTok{\# data}
\NormalTok{mvr}\OtherTok{=}\NormalTok{sfarrow}\SpecialCharTok{::}\FunctionTok{st\_read\_parquet}\NormalTok{(}\StringTok{"./Geodata/2024/MVR/nogabali\_2024janv.parquet"}\NormalTok{)}

\CommentTok{\# species groups}
\NormalTok{skujkoki}\OtherTok{=}\FunctionTok{c}\NormalTok{(}\StringTok{"1"}\NormalTok{,}\StringTok{"3"}\NormalTok{,}\StringTok{"13"}\NormalTok{,}\StringTok{"14"}\NormalTok{,}\StringTok{"15"}\NormalTok{,}\StringTok{"22"}\NormalTok{,}\StringTok{"23"}\NormalTok{,}\StringTok{"28"}\NormalTok{) }\CommentTok{\# 8}
\NormalTok{saurlapji}\OtherTok{=}\FunctionTok{c}\NormalTok{(}\StringTok{"4"}\NormalTok{,}\StringTok{"6"}\NormalTok{,}\StringTok{"8"}\NormalTok{,}\StringTok{"9"}\NormalTok{,}\StringTok{"19"}\NormalTok{,}\StringTok{"20"}\NormalTok{,}\StringTok{"21"}\NormalTok{,}\StringTok{"32"}\NormalTok{,}\StringTok{"35"}\NormalTok{,}\StringTok{"68"}\NormalTok{) }\CommentTok{\# 10}
\NormalTok{platlapji}\OtherTok{=}\FunctionTok{c}\NormalTok{(}\StringTok{"10"}\NormalTok{,}\StringTok{"11"}\NormalTok{,}\StringTok{"12"}\NormalTok{,}\StringTok{"16"}\NormalTok{,}\StringTok{"17"}\NormalTok{,}\StringTok{"18"}\NormalTok{,}\StringTok{"24"}\NormalTok{,}\StringTok{"25"}\NormalTok{,}\StringTok{"26"}\NormalTok{,}\StringTok{"27"}\NormalTok{,}\StringTok{"28"}\NormalTok{,}\StringTok{"29"}\NormalTok{,}\StringTok{"50"}\NormalTok{,}
      \StringTok{"61"}\NormalTok{,}\StringTok{"62"}\NormalTok{,}\StringTok{"63"}\NormalTok{,}\StringTok{"64"}\NormalTok{,}\StringTok{"65"}\NormalTok{,}\StringTok{"66"}\NormalTok{,}\StringTok{"67"}\NormalTok{,}\StringTok{"69"}\NormalTok{) }\CommentTok{\# 21}

\CommentTok{\# classification}
\NormalTok{mvr2}\OtherTok{=}\NormalTok{mvr }\SpecialCharTok{\%\textgreater{}\%} 
  \FunctionTok{mutate}\NormalTok{(}\AttributeTok{vol\_coniferous=}\FunctionTok{ifelse}\NormalTok{(s10 }\SpecialCharTok{\%in\%}\NormalTok{ coniferous,v10,}\DecValTok{0}\NormalTok{)}\SpecialCharTok{+}
           \FunctionTok{ifelse}\NormalTok{(s11 }\SpecialCharTok{\%in\%}\NormalTok{ coniferous,v11,}\DecValTok{0}\NormalTok{)}\SpecialCharTok{+}\FunctionTok{ifelse}\NormalTok{(s12 }\SpecialCharTok{\%in\%}\NormalTok{ coniferous,v12,}\DecValTok{0}\NormalTok{)}\SpecialCharTok{+}
           \FunctionTok{ifelse}\NormalTok{(s13 }\SpecialCharTok{\%in\%}\NormalTok{ coniferous,v13,}\DecValTok{0}\NormalTok{)}\SpecialCharTok{+}\FunctionTok{ifelse}\NormalTok{(s14 }\SpecialCharTok{\%in\%}\NormalTok{ coniferous,v14,}\DecValTok{0}\NormalTok{),}
         \AttributeTok{vol\_boreal=}\FunctionTok{ifelse}\NormalTok{(s10 }\SpecialCharTok{\%in\%}\NormalTok{ boreal\_deciduous,v10,}\DecValTok{0}\NormalTok{)}\SpecialCharTok{+}
           \FunctionTok{ifelse}\NormalTok{(s11 }\SpecialCharTok{\%in\%}\NormalTok{ boreal\_deciduous,v11,}\DecValTok{0}\NormalTok{)}\SpecialCharTok{+}\FunctionTok{ifelse}\NormalTok{(s12 }\SpecialCharTok{\%in\%}\NormalTok{ boreal\_deciduous,v12,}\DecValTok{0}\NormalTok{)}\SpecialCharTok{+}
           \FunctionTok{ifelse}\NormalTok{(s13 }\SpecialCharTok{\%in\%}\NormalTok{ boreal\_deciduous,v13,}\DecValTok{0}\NormalTok{)}\SpecialCharTok{+}\FunctionTok{ifelse}\NormalTok{(s14 }\SpecialCharTok{\%in\%}\NormalTok{ boreal\_deciduous,v14,}\DecValTok{0}\NormalTok{),}
         \AttributeTok{vol\_temperate=}\FunctionTok{ifelse}\NormalTok{(s10 }\SpecialCharTok{\%in\%}\NormalTok{ temperate\_deciduous,v10,}\DecValTok{0}\NormalTok{)}\SpecialCharTok{+}
           \FunctionTok{ifelse}\NormalTok{(s11 }\SpecialCharTok{\%in\%}\NormalTok{ temperate\_deciduous,v11,}\DecValTok{0}\NormalTok{)}\SpecialCharTok{+}\FunctionTok{ifelse}\NormalTok{(s12 }\SpecialCharTok{\%in\%}\NormalTok{ temperate\_deciduous,v12,}\DecValTok{0}\NormalTok{)}\SpecialCharTok{+}
           \FunctionTok{ifelse}\NormalTok{(s13 }\SpecialCharTok{\%in\%}\NormalTok{ temperate\_deciduous,v13,}\DecValTok{0}\NormalTok{)}\SpecialCharTok{+}\FunctionTok{ifelse}\NormalTok{(s14 }\SpecialCharTok{\%in\%}\NormalTok{ temperate\_deciduous,v14,}\DecValTok{0}\NormalTok{)) }\SpecialCharTok{\%\textgreater{}\%} 
  \FunctionTok{mutate}\NormalTok{(}\AttributeTok{vol\_total=}\NormalTok{vol\_coniferous}\SpecialCharTok{+}\NormalTok{vol\_boreal}\SpecialCharTok{+}\NormalTok{vol\_temperate) }\SpecialCharTok{\%\textgreater{}\%} 
  \FunctionTok{mutate}\NormalTok{(}\AttributeTok{forest\_type=}\FunctionTok{ifelse}\NormalTok{(vol\_coniferous}\SpecialCharTok{/}\NormalTok{vol\_total}\SpecialCharTok{\textgreater{}=}\FloatTok{0.75}\NormalTok{,}\StringTok{"coniferous"}\NormalTok{,}
                     \FunctionTok{ifelse}\NormalTok{(vol\_boreal}\SpecialCharTok{/}\NormalTok{vol\_total}\SpecialCharTok{\textgreater{}=}\FloatTok{0.75}\NormalTok{,}\StringTok{"boreal"}\NormalTok{,}
                            \FunctionTok{ifelse}\NormalTok{(vol\_temperate}\SpecialCharTok{/}\NormalTok{vol\_total}\SpecialCharTok{\textgreater{}}\FloatTok{0.5}\NormalTok{,}\StringTok{"temperate"}\NormalTok{,}
                                   \StringTok{"mixed"}\NormalTok{)))) }\SpecialCharTok{\%\textgreater{}\%} 
  \FunctionTok{mutate}\NormalTok{(}\AttributeTok{forest\_age=}\FunctionTok{ifelse}\NormalTok{(vgr}\SpecialCharTok{==}\StringTok{"1"}\SpecialCharTok{|}\NormalTok{vgr}\SpecialCharTok{==}\StringTok{"2"}\SpecialCharTok{|}\NormalTok{vgr}\SpecialCharTok{==}\StringTok{"3"}\NormalTok{,}\StringTok{"young"}\NormalTok{,}
                           \FunctionTok{ifelse}\NormalTok{(vgr}\SpecialCharTok{==}\StringTok{"4"}\SpecialCharTok{|}\NormalTok{vgr}\SpecialCharTok{==}\StringTok{"5"}\NormalTok{,}\StringTok{"old"}\NormalTok{,}\ConstantTok{NA}\NormalTok{))) }\SpecialCharTok{\%\textgreater{}\%} 
  \FunctionTok{filter}\NormalTok{(}\SpecialCharTok{!}\FunctionTok{is.na}\NormalTok{(forest\_type)) }\SpecialCharTok{\%\textgreater{}\%} 
  \FunctionTok{filter}\NormalTok{(}\SpecialCharTok{!}\FunctionTok{is.na}\NormalTok{(forest\_age)) }\SpecialCharTok{\%\textgreater{}\%} 
  \FunctionTok{mutate}\NormalTok{(}\AttributeTok{divbroad\_class=}\FunctionTok{paste0}\NormalTok{(forest\_type,}\StringTok{"\_"}\NormalTok{,forest\_age)) }\SpecialCharTok{\%\textgreater{}\%} 
  \FunctionTok{mutate}\NormalTok{(}\AttributeTok{divbroad\_numeric=}\FunctionTok{as.numeric}\NormalTok{(}\FunctionTok{as.factor}\NormalTok{(divbroad\_class))}\SpecialCharTok{+}\DecValTok{660}\NormalTok{) }\SpecialCharTok{\%\textgreater{}\%} 
  \FunctionTok{filter}\NormalTok{(}\SpecialCharTok{!}\FunctionTok{is.na}\NormalTok{(divbroad\_numeric))}

\CommentTok{\# input layer}
\FunctionTok{polygon2input}\NormalTok{(}\AttributeTok{vector\_data =}\NormalTok{ mvr2,}
              \AttributeTok{template\_path =} \StringTok{"./Templates/TemplateRasters/LV10m\_10km.tif"}\NormalTok{,}
              \AttributeTok{out\_path =} \StringTok{"./RasterGrids\_10m/2024/"}\NormalTok{,}
              \AttributeTok{file\_name =} \StringTok{"Diversity\_ForestBroad\_only.tif"}\NormalTok{,}
              \AttributeTok{value\_field =} \StringTok{"divbroad\_numeric"}\NormalTok{,}
              \AttributeTok{fun=}\StringTok{"first"}\NormalTok{,}
              \AttributeTok{prepare=}\ConstantTok{FALSE}\NormalTok{,}
              \AttributeTok{project\_mode =} \StringTok{"auto"}\NormalTok{,}
              \AttributeTok{overwrite =} \ConstantTok{TRUE}\NormalTok{)}
\CommentTok{\# cleaning}
\FunctionTok{rm}\NormalTok{(mvr)}
\FunctionTok{rm}\NormalTok{(mvr2)}

\DocumentationTok{\#\# overall classification {-}{-}{-}{-}}

\NormalTok{simple\_landscape}\OtherTok{=}\FunctionTok{rast}\NormalTok{(}\StringTok{"./RasterGrids\_10m/2024/Ainava\_vienk\_mask.tif"}\NormalTok{)}

\DocumentationTok{\#\# Covered classes for general diversity {-}{-}{-}{-}}

\NormalTok{farmland\_broad}\OtherTok{=}\FunctionTok{rast}\NormalTok{(}\StringTok{"./RasterGrids\_10m/2024/Diversity\_FarmlandBroad\_only.tif"}\NormalTok{)}
\NormalTok{forests\_broad}\OtherTok{=}\FunctionTok{rast}\NormalTok{(}\StringTok{"./RasterGrids\_10m/2024/Diversity\_ForestBroad\_only.tif"}\NormalTok{)}

\NormalTok{diversity\_classes}\OtherTok{=}\FunctionTok{cover}\NormalTok{(farmland\_broad,forests\_broad)}
\NormalTok{diversity\_classes2}\OtherTok{=}\FunctionTok{cover}\NormalTok{(diversity\_classes,simple\_landscape,}
                        \AttributeTok{filename=}\StringTok{"./RasterGrids\_10m/2024/Diversity\_GeneralLandscapeBroad.tif"}\NormalTok{,}
                        \AttributeTok{overwrite=}\ConstantTok{TRUE}\NormalTok{)}

\FunctionTok{rm}\NormalTok{(simple\_landscape)}
\FunctionTok{rm}\NormalTok{(farmland\_broad)}
\FunctionTok{rm}\NormalTok{(forests\_broad)}
\FunctionTok{rm}\NormalTok{(diversity\_classes)}
\FunctionTok{rm}\NormalTok{(diversity\_classes2)}

\DocumentationTok{\#\# Diversity index at 25ha {-}{-}{-}{-}{-}}


\NormalTok{res\_tbl }\OtherTok{\textless{}{-}} \FunctionTok{landscape\_function}\NormalTok{(}
  \AttributeTok{landscape      =} \StringTok{"./RasterGrids\_10m/2024/Diversity\_GeneralLandscapeBroad.tif"}\NormalTok{,}
  \AttributeTok{zones          =} \StringTok{"./Templates/TemplateGrids/tikls500\_sauzeme.parquet"}\NormalTok{,}
  \AttributeTok{id\_field       =} \StringTok{"rinda500"}\NormalTok{,}
  \AttributeTok{tile\_field     =} \StringTok{"tks50km"}\NormalTok{,}
  \AttributeTok{template       =} \StringTok{"./Templates/TemplateRasters/LV500m\_10km.tif"}\NormalTok{,}
  \AttributeTok{out\_dir        =} \StringTok{"./RasterGrids\_500m/2024/"}\NormalTok{,}
  \AttributeTok{out\_filename   =} \StringTok{"Diversity\_GeneralLandscape\_500x.tif"}\NormalTok{,}
  \AttributeTok{out\_layername  =} \StringTok{"Diversity\_GeneralLandscape\_500x"}\NormalTok{,}
  \AttributeTok{what           =} \StringTok{"lsm\_l\_shdi"}\NormalTok{,}
  \AttributeTok{rasterize\_engine =} \StringTok{"fasterize"}\NormalTok{,}
  \AttributeTok{n\_workers      =} \DecValTok{8}\NormalTok{,}
  \AttributeTok{future\_max\_size =} \DecValTok{3} \SpecialCharTok{*} \DecValTok{1024}\SpecialCharTok{\^{}}\DecValTok{3}\NormalTok{,}
  \AttributeTok{fill\_gaps      =} \ConstantTok{TRUE}\NormalTok{,}
  \AttributeTok{plot\_gaps      =} \ConstantTok{TRUE}\NormalTok{,}
  \AttributeTok{plot\_result    =} \ConstantTok{TRUE}
\NormalTok{)}
\FunctionTok{print}\NormalTok{(res\_tbl)}
\FunctionTok{plot}\NormalTok{(}\FunctionTok{rast}\NormalTok{(}\StringTok{"./RasterGrids\_500m/2024/Diversity\_GeneralLandscape\_500x.tif"}\NormalTok{))}
\FunctionTok{rm}\NormalTok{(res\_tbl)}
\end{Highlighting}
\end{Shaded}

\subsection{Forest diversity}\label{Ch05.04.02}

An input grid with a cell size of 10 m covers the entire territory of Latvia. It
contains the following values, in order of hierarchy:

\begin{itemize}
\item
  \hyperref[Ch04.01]{State Forest Service's Forest State Register} code, in which the
  code of the dominant tree species is multiplied by 1000 and the age group
  code is added. However, before rasterisation, geometries in which no code has
  been assigned or one of the code components is 0, are excluded;
\item
  forest diversity class values prepared in \hyperref[Ch05.04.01]{Overall landscape diversity};
\item
  forest classes from \hyperref[Ch05.03]{Landscape classification};
\item
  value 1 for all other cells located in the territory of Latvia.
\end{itemize}

Once the landscape classification is done, the Shannon's diversity index is calculated for 25 ha
landscapes using the function \texttt{egvtools::landscape\_function}. To ensure value coverage,
inverse distance weighted (power = 2) gap filling is incorporated; however,
there were no gaps to fill.

\begin{Shaded}
\begin{Highlighting}[]
\CommentTok{\# Libs {-}{-}{-}{-}}
\ControlFlowTok{if}\NormalTok{(}\SpecialCharTok{!}\FunctionTok{require}\NormalTok{(egvtools)) \{}\FunctionTok{install.packages}\NormalTok{(}\StringTok{"egvtools"}\NormalTok{); }\FunctionTok{require}\NormalTok{(egvtools)\}}
\ControlFlowTok{if}\NormalTok{(}\SpecialCharTok{!}\FunctionTok{require}\NormalTok{(tidyverse)) \{}\FunctionTok{install.packages}\NormalTok{(}\StringTok{"tidyverse"}\NormalTok{); }\FunctionTok{require}\NormalTok{(tidyverse)\}}
\ControlFlowTok{if}\NormalTok{(}\SpecialCharTok{!}\FunctionTok{require}\NormalTok{(sf)) \{}\FunctionTok{install.packages}\NormalTok{(}\StringTok{"sf"}\NormalTok{); }\FunctionTok{require}\NormalTok{(sf)\}}
\ControlFlowTok{if}\NormalTok{(}\SpecialCharTok{!}\FunctionTok{require}\NormalTok{(arrow)) \{}\FunctionTok{install.packages}\NormalTok{(}\StringTok{"arrow"}\NormalTok{); }\FunctionTok{require}\NormalTok{(arrow)\}}
\ControlFlowTok{if}\NormalTok{(}\SpecialCharTok{!}\FunctionTok{require}\NormalTok{(sfarrow)) \{}\FunctionTok{install.packages}\NormalTok{(}\StringTok{"sfarrow"}\NormalTok{); }\FunctionTok{require}\NormalTok{(sfarrow)\}}
\ControlFlowTok{if}\NormalTok{(}\SpecialCharTok{!}\FunctionTok{require}\NormalTok{(terra)) \{}\FunctionTok{install.packages}\NormalTok{(}\StringTok{"terra"}\NormalTok{); }\FunctionTok{require}\NormalTok{(terra)\}}
\ControlFlowTok{if}\NormalTok{(}\SpecialCharTok{!}\FunctionTok{require}\NormalTok{(readxl)) \{}\FunctionTok{install.packages}\NormalTok{(}\StringTok{"readxl"}\NormalTok{); }\FunctionTok{require}\NormalTok{(readxl)\}}

\CommentTok{\# templates {-}{-}{-}{-}}
\NormalTok{template\_t}\OtherTok{=}\FunctionTok{rast}\NormalTok{(}\StringTok{"./Templates/TemplateRasters/LV10m\_10km.tif"}\NormalTok{)}
\NormalTok{template\_r}\OtherTok{=}\FunctionTok{raster}\NormalTok{(template\_t)}

\CommentTok{\# forest diversity {-}{-}{-}{-}}

\DocumentationTok{\#\# forest broad {-}{-}{-}{-}}
\NormalTok{forest\_broad}\OtherTok{=}\FunctionTok{rast}\NormalTok{(}\StringTok{"./RasterGrids\_10m/2024/Diversity\_ForestBroad\_only.tif"}\NormalTok{)}


\DocumentationTok{\#\# forest codes {-}{-}{-}{-}}
\CommentTok{\# mezi}
\NormalTok{mvr}\OtherTok{=}\FunctionTok{st\_read\_parquet}\NormalTok{(}\StringTok{"./Geodata/2024/MVR/nogabali\_2024janv.parquet"}\NormalTok{)}

\NormalTok{mvr}\OtherTok{=}\NormalTok{mvr }\SpecialCharTok{\%\textgreater{}\%} 
  \FunctionTok{mutate}\NormalTok{(}\AttributeTok{kods1=}\FunctionTok{as.numeric}\NormalTok{(s10)}\SpecialCharTok{*}\DecValTok{1000}\NormalTok{,}
         \AttributeTok{kods2=}\FunctionTok{as.numeric}\NormalTok{(vgr),}
         \AttributeTok{kods=}\NormalTok{kods1}\SpecialCharTok{+}\NormalTok{kods2) }\SpecialCharTok{\%\textgreater{}\%} 
  \FunctionTok{filter}\NormalTok{(}\SpecialCharTok{!}\FunctionTok{is.na}\NormalTok{(kods)) }\SpecialCharTok{\%\textgreater{}\%} 
  \FunctionTok{filter}\NormalTok{(kods1}\SpecialCharTok{\textgreater{}}\DecValTok{0}\NormalTok{) }\SpecialCharTok{\%\textgreater{}\%} 
  \FunctionTok{filter}\NormalTok{(kods2}\SpecialCharTok{\textgreater{}}\DecValTok{0}\NormalTok{)}

\CommentTok{\# input layer}
\FunctionTok{polygon2input}\NormalTok{(}\AttributeTok{vector\_data =}\NormalTok{ mvr,}
              \AttributeTok{template\_path =} \StringTok{"./Templates/TemplateRasters/LV10m\_10km.tif"}\NormalTok{,}
              \AttributeTok{out\_path =} \StringTok{"./RasterGrids\_10m/2024/"}\NormalTok{,}
              \AttributeTok{file\_name =} \StringTok{"Diversity\_ForestCodes\_only.tif"}\NormalTok{,}
              \AttributeTok{value\_field =} \StringTok{"kods"}\NormalTok{,}
              \AttributeTok{fun=}\StringTok{"first"}\NormalTok{,}
              \AttributeTok{prepare=}\ConstantTok{FALSE}\NormalTok{,}
              \AttributeTok{project\_mode =} \StringTok{"auto"}\NormalTok{,}
              \AttributeTok{overwrite =} \ConstantTok{TRUE}\NormalTok{)}

\CommentTok{\# cleaning}
\FunctionTok{rm}\NormalTok{(mvr)}

\CommentTok{\# simple forests}
\NormalTok{simple\_forests}\OtherTok{=}\FunctionTok{rast}\NormalTok{(}\StringTok{"./RasterGrids\_10m/2024/SimpleLandscape\_class600\_meziem\_premask.tif"}\NormalTok{)}

\DocumentationTok{\#\# Covered classes for forest diversity {-}{-}{-}{-}}

\NormalTok{forest\_codes}\OtherTok{=}\FunctionTok{rast}\NormalTok{(}\StringTok{"./RasterGrids\_10m/2024/Diversity\_ForestCodes\_only.tif"}\NormalTok{)}
\FunctionTok{plot}\NormalTok{(forest\_codes)}
\NormalTok{forest\_covered}\OtherTok{=}\FunctionTok{cover}\NormalTok{(forest\_codes,forest\_broad)}
\NormalTok{forest\_covered}\OtherTok{=}\FunctionTok{cover}\NormalTok{(forest\_covered,simple\_forests)}
\FunctionTok{plot}\NormalTok{(forest\_covered)}

\NormalTok{forest\_covered2}\OtherTok{=}\FunctionTok{cover}\NormalTok{(forest\_covered,template\_t,}
                        \AttributeTok{filename=}\StringTok{"./RasterGrids\_10m/2024/Diversity\_ForestsDetailed.tif"}\NormalTok{,}
                        \AttributeTok{overwrite=}\ConstantTok{TRUE}\NormalTok{)}
\FunctionTok{plot}\NormalTok{(forest\_covered2)}

\CommentTok{\# cleaning}
\FunctionTok{rm}\NormalTok{(forest\_codes)}
\FunctionTok{rm}\NormalTok{(forest\_covered)}
\FunctionTok{rm}\NormalTok{(forest\_covered2)}
\FunctionTok{rm}\NormalTok{(forest\_broad)}
\FunctionTok{rm}\NormalTok{(simple\_forests)}



\DocumentationTok{\#\# Diversity index at 25ha {-}{-}{-}{-}{-}}

\NormalTok{res\_tbl }\OtherTok{\textless{}{-}} \FunctionTok{landscape\_function}\NormalTok{(}
  \AttributeTok{landscape      =} \StringTok{"./RasterGrids\_10m/2024/Diversity\_ForestsDetailed.tif"}\NormalTok{,}
  \AttributeTok{zones          =} \StringTok{"./Templates/TemplateGrids/tikls500\_sauzeme.parquet"}\NormalTok{,}
  \AttributeTok{id\_field       =} \StringTok{"rinda500"}\NormalTok{,}
  \AttributeTok{tile\_field     =} \StringTok{"tks50km"}\NormalTok{,}
  \AttributeTok{template       =} \StringTok{"./Templates/TemplateRasters/LV500m\_10km.tif"}\NormalTok{,}
  \AttributeTok{out\_dir        =} \StringTok{"./RasterGrids\_500m/2024/"}\NormalTok{,}
  \AttributeTok{out\_filename   =} \StringTok{"Diversity\_Forests\_500x.tif"}\NormalTok{,}
  \AttributeTok{out\_layername  =} \StringTok{"Diversity\_Forests\_500x"}\NormalTok{,}
  \AttributeTok{what           =} \StringTok{"lsm\_l\_shdi"}\NormalTok{,}
  \AttributeTok{rasterize\_engine =} \StringTok{"fasterize"}\NormalTok{,}
  \AttributeTok{n\_workers      =} \DecValTok{8}\NormalTok{,}
  \AttributeTok{future\_max\_size =} \DecValTok{3} \SpecialCharTok{*} \DecValTok{1024}\SpecialCharTok{\^{}}\DecValTok{3}\NormalTok{,}
  \AttributeTok{fill\_gaps      =} \ConstantTok{TRUE}\NormalTok{,}
  \AttributeTok{plot\_gaps      =} \ConstantTok{TRUE}\NormalTok{,}
  \AttributeTok{plot\_result    =} \ConstantTok{TRUE}
\NormalTok{)}
\FunctionTok{print}\NormalTok{(res\_tbl)}

\FunctionTok{plot}\NormalTok{(}\FunctionTok{rast}\NormalTok{(}\StringTok{"./RasterGrids\_500m/2024/Diversity\_Forests\_500x.tif"}\NormalTok{))}
\FunctionTok{rm}\NormalTok{(res\_tbl)}
\end{Highlighting}
\end{Shaded}

\subsection{Farmland diversity}\label{Ch05.04.03}

A grid with a cell size of 10 m covers the entire territory of Latvia. It
contains the following values, listed in order of hierarchy:

\begin{itemize}
\item
  \hyperref[Ch04.02]{Rural Support Service} crop codes with 1000 added;
\item
  farmland diversity class values prepared in \hyperref[Ch05.04.01]{Overall landscape diversity};
\item
  farmland classes from \hyperref[Ch05.03]{Landscape classification};
\item
  value 1 for all other cells located within the territory of Latvia.
\end{itemize}

Once the landscape classification is done, diversity index is calculated for 25 ha
landscapes with function \texttt{egvtools::landscape\_function}. To guard value coverage,
inverse distance weighted (power = 2) gap filling is incorporated; however,
there were no gaps to fill.

\begin{Shaded}
\begin{Highlighting}[]
\CommentTok{\# Libs {-}{-}{-}{-}}
\ControlFlowTok{if}\NormalTok{(}\SpecialCharTok{!}\FunctionTok{require}\NormalTok{(egvtools)) \{}\FunctionTok{install.packages}\NormalTok{(}\StringTok{"egvtools"}\NormalTok{); }\FunctionTok{require}\NormalTok{(egvtools)\}}
\ControlFlowTok{if}\NormalTok{(}\SpecialCharTok{!}\FunctionTok{require}\NormalTok{(tidyverse)) \{}\FunctionTok{install.packages}\NormalTok{(}\StringTok{"tidyverse"}\NormalTok{); }\FunctionTok{require}\NormalTok{(tidyverse)\}}
\ControlFlowTok{if}\NormalTok{(}\SpecialCharTok{!}\FunctionTok{require}\NormalTok{(sf)) \{}\FunctionTok{install.packages}\NormalTok{(}\StringTok{"sf"}\NormalTok{); }\FunctionTok{require}\NormalTok{(sf)\}}
\ControlFlowTok{if}\NormalTok{(}\SpecialCharTok{!}\FunctionTok{require}\NormalTok{(arrow)) \{}\FunctionTok{install.packages}\NormalTok{(}\StringTok{"arrow"}\NormalTok{); }\FunctionTok{require}\NormalTok{(arrow)\}}
\ControlFlowTok{if}\NormalTok{(}\SpecialCharTok{!}\FunctionTok{require}\NormalTok{(sfarrow)) \{}\FunctionTok{install.packages}\NormalTok{(}\StringTok{"sfarrow"}\NormalTok{); }\FunctionTok{require}\NormalTok{(sfarrow)\}}
\ControlFlowTok{if}\NormalTok{(}\SpecialCharTok{!}\FunctionTok{require}\NormalTok{(terra)) \{}\FunctionTok{install.packages}\NormalTok{(}\StringTok{"terra"}\NormalTok{); }\FunctionTok{require}\NormalTok{(terra)\}}
\ControlFlowTok{if}\NormalTok{(}\SpecialCharTok{!}\FunctionTok{require}\NormalTok{(readxl)) \{}\FunctionTok{install.packages}\NormalTok{(}\StringTok{"readxl"}\NormalTok{); }\FunctionTok{require}\NormalTok{(readxl)\}}

\CommentTok{\# templates {-}{-}{-}{-}}
\NormalTok{template\_t}\OtherTok{=}\FunctionTok{rast}\NormalTok{(}\StringTok{"./Templates/TemplateRasters/LV10m\_10km.tif"}\NormalTok{)}
\NormalTok{template\_r}\OtherTok{=}\FunctionTok{raster}\NormalTok{(template\_t)}



\CommentTok{\# farmland diversity {-}{-}{-}{-}{-}}


\DocumentationTok{\#\# Farmland broad {-}{-}{-}{-}}

\NormalTok{farmland\_broad}\OtherTok{=}\FunctionTok{rast}\NormalTok{(}\StringTok{"./RasterGrids\_10m/2024/Diversity\_FarmlandBroad\_only.tif"}\NormalTok{)}


\DocumentationTok{\#\# Farmland codes {-}{-}{-}{-}}

\NormalTok{lad}\OtherTok{=}\NormalTok{sfarrow}\SpecialCharTok{::}\FunctionTok{st\_read\_parquet}\NormalTok{(}\StringTok{"./Geodata/2024/LAD/Lauki\_2024.parquet"}\NormalTok{)}
\NormalTok{lad}\SpecialCharTok{$}\NormalTok{product\_code}\OtherTok{=}\FunctionTok{as.numeric}\NormalTok{(lad}\SpecialCharTok{$}\NormalTok{PRODUCT\_CODE)}\SpecialCharTok{+}\DecValTok{1000}

\CommentTok{\# input layer}
\FunctionTok{polygon2input}\NormalTok{(}\AttributeTok{vector\_data =}\NormalTok{ lad,}
              \AttributeTok{template\_path =} \StringTok{"./Templates/TemplateRasters/LV10m\_10km.tif"}\NormalTok{,}
              \AttributeTok{out\_path =} \StringTok{"./RasterGrids\_10m/2024/"}\NormalTok{,}
              \AttributeTok{file\_name =} \StringTok{"Diversity\_FarmlandCodes\_only.tif"}\NormalTok{,}
              \AttributeTok{value\_field =} \StringTok{"product\_code"}\NormalTok{,}
              \AttributeTok{fun=}\StringTok{"first"}\NormalTok{,}
              \AttributeTok{prepare=}\ConstantTok{FALSE}\NormalTok{,}
              \AttributeTok{project\_mode =} \StringTok{"auto"}\NormalTok{,}
              \AttributeTok{overwrite =} \ConstantTok{TRUE}\NormalTok{)}

\CommentTok{\# cleaning}
\FunctionTok{rm}\NormalTok{(lad)}


\CommentTok{\# simple landscapes input }
\NormalTok{simple\_farmland}\OtherTok{=}\FunctionTok{rast}\NormalTok{(}\StringTok{"./RasterGrids\_10m/2024/SimpleLandscape\_class300\_lauki\_premask.tif"}\NormalTok{)}

\DocumentationTok{\#\# Covered classes for farmland diversity {-}{-}{-}{-}}

\NormalTok{farmland\_codes}\OtherTok{=}\FunctionTok{rast}\NormalTok{(}\StringTok{"./RasterGrids\_10m/2024/Diversity\_FarmlandCodes\_only.tif"}\NormalTok{)}
\NormalTok{farmland\_covered}\OtherTok{=}\FunctionTok{cover}\NormalTok{(farmland\_codes,farmland\_broad)}
\NormalTok{farmland\_covered}\OtherTok{=}\FunctionTok{cover}\NormalTok{(farmland\_covered,simple\_farmland)}
\NormalTok{farmland\_covered2}\OtherTok{=}\FunctionTok{cover}\NormalTok{(farmland\_covered,template\_t,}
                        \AttributeTok{filename=}\StringTok{"./RasterGrids\_10m/2024/Diversity\_FarmlandDetailed.tif"}\NormalTok{,}
                        \AttributeTok{overwrite=}\ConstantTok{TRUE}\NormalTok{)}
\FunctionTok{plot}\NormalTok{(farmland\_covered2)}

\CommentTok{\# cleaning}
\FunctionTok{rm}\NormalTok{(farmland\_codes)}
\FunctionTok{rm}\NormalTok{(farmland\_covered)}
\FunctionTok{rm}\NormalTok{(farmland\_covered2)}
\FunctionTok{rm}\NormalTok{(simple\_farmland)}
\FunctionTok{rm}\NormalTok{(farmland\_broad)}


\DocumentationTok{\#\# Diversity index at 25ha {-}{-}{-}{-}{-}}

\NormalTok{res\_tbl }\OtherTok{\textless{}{-}} \FunctionTok{landscape\_function}\NormalTok{(}
  \AttributeTok{landscape      =} \StringTok{"./RasterGrids\_10m/2024/Diversity\_FarmlandDetailed.tif"}\NormalTok{,}
  \AttributeTok{zones          =} \StringTok{"./Templates/TemplateGrids/tikls500\_sauzeme.parquet"}\NormalTok{,}
  \AttributeTok{id\_field       =} \StringTok{"rinda500"}\NormalTok{,}
  \AttributeTok{tile\_field     =} \StringTok{"tks50km"}\NormalTok{,}
  \AttributeTok{template       =} \StringTok{"./Templates/TemplateRasters/LV500m\_10km.tif"}\NormalTok{,}
  \AttributeTok{out\_dir        =} \StringTok{"./RasterGrids\_500m/2024/"}\NormalTok{,}
  \AttributeTok{out\_filename   =} \StringTok{"Diversity\_Farmland\_500x.tif"}\NormalTok{,}
  \AttributeTok{out\_layername  =} \StringTok{"Diversity\_Farmland\_500x"}\NormalTok{,}
  \AttributeTok{what           =} \StringTok{"lsm\_l\_shdi"}\NormalTok{,}
  \AttributeTok{rasterize\_engine =} \StringTok{"fasterize"}\NormalTok{,}
  \AttributeTok{n\_workers      =} \DecValTok{8}\NormalTok{,}
  \AttributeTok{future\_max\_size =} \DecValTok{3} \SpecialCharTok{*} \DecValTok{1024}\SpecialCharTok{\^{}}\DecValTok{3}\NormalTok{,}
  \AttributeTok{fill\_gaps      =} \ConstantTok{TRUE}\NormalTok{,}
  \AttributeTok{plot\_gaps      =} \ConstantTok{TRUE}\NormalTok{,}
  \AttributeTok{plot\_result    =} \ConstantTok{TRUE}
\NormalTok{)}
\FunctionTok{print}\NormalTok{(res\_tbl)}

\FunctionTok{plot}\NormalTok{(}\FunctionTok{rast}\NormalTok{(}\StringTok{"./RasterGrids\_500m/2024/Diversity\_Farmland\_500x.tif"}\NormalTok{))}
\FunctionTok{rm}\NormalTok{(res\_tbl)}
\end{Highlighting}
\end{Shaded}

\chapter{Ecogeographical variables}\label{Ch06}

This section names and provides description (R code with its explanation in
procedure) of each of the 538 EGVs created.

Refer to the flowchart below (Fig. \ref{fig:flowchart}) for a better
understanding of how these varable relate. The names used in the figure correspond
to EGV layer names and follow naming convention: {[}group{]} \_ {[}specific name{]} \_ {[}scale{]},
where:

\begin{itemize}
\item
  group is a broader collection of EGVs describing the same phenomena or ecosystem,
  derived from the same source, etc.;
\item
  specific name briefly describes the landscape class and/or metrics used in
  the creation of the layer;
\item
  scale is one of: cell, 500, 1250, 3000, 10000 m around the centre of the
  EGV-cell. The resolution of each EGV is 1 ha; larger scales are summarised
  to this resolution.
\end{itemize}

\begin{figure}
\includegraphics[width=1\linewidth]{./Figures/EGV_FlowChartZ_17_11_25} \caption{Relationships of the created ecogeographical variables.}\label{fig:flowchart}
\end{figure}

For cover fraction and edge variables, we first calculated values at the EGV-cell
resolution and then used \{exactextract\} to summarise values from larger
scales. This package uses pixel area weights to calculate weighted summary
statistics, making the aggregation error negligible, particularly
at larger scales, but reduces computation time thousands up to even hundreds of
thousands times compared to input resolution (10 m). To further speed up the
procedures, we used ``sparse'' mode in the workflow \texttt{egvtools::radius\_function()}, thus
summarising zonal statistics every 300 m for 3000 m radius buffers and every
1000 m for 10000 m buffers, obtaining near linear reduction in time relative to
the number of zones (ninefold and 100 fold further computation time reduction),
while loosing less than 0.001 \% of variability overall.

We used a slightly different approach with diversity metrics. First, we calculated
Shanon's diversity index at 25 ha raster grid cells, as there is nearly no
variability of landscape classes at 1 ha grid cells. Next, we calculated
arithmetic mean as zonal statictics value (using the ``sparse'' mode with the workflow
\texttt{egvtools::radius\_function()}), but we did not create this EGV at the analysis
cells scale.

\section{Climate\_CHELSAv2.1-bio1\_cell}\label{ch06.001}

\textbf{filename:} \texttt{Climate\_CHELSAv2.1-bio1\_cell.tif}

\textbf{layername:} \texttt{egv\_001}

\textbf{English name:} Mean annual daily mean air temperature (°C) (CHELSA v2.1)
within the analysis cell (1 ha)

\textbf{Latvian name:} Gada vidējā ik dienas vidējā gaisa temperatūra (°C) (CHELSA v2.1) analīzes
šūnā (1 ha)

\textbf{Procedure:} Directly follows \hyperref[Ch04.11]{CHELSA v2.1}. EGV is prepared using
the workflow \texttt{egvtools::downscale2egv()} with inverse distance weighted (power =
2) gap filling and soft smoothing (power = 0.5) over 5 km radius around each cell.
Finally, the layer is standardised by subtracting the arithmetic mean and
dividing by the root mean squared error.

\begin{Shaded}
\begin{Highlighting}[]
\CommentTok{\# libs {-}{-}{-}{-}}
\ControlFlowTok{if}\NormalTok{(}\SpecialCharTok{!}\FunctionTok{require}\NormalTok{(egvtools)) \{remotes}\SpecialCharTok{::}\FunctionTok{install\_github}\NormalTok{(}\StringTok{"aavotins/egvtools"}\NormalTok{); }\FunctionTok{require}\NormalTok{(egvtools)\}}

\CommentTok{\# job {-}{-}{-}{-}}
\NormalTok{localname}\OtherTok{=}\StringTok{"Climate\_CHELSAv2.1{-}bio1\_cell.tif"}
\NormalTok{layername}\OtherTok{=}\StringTok{"egv\_001"}
\NormalTok{reading}\OtherTok{=}\StringTok{"./Geodata/2024/CHELSA/Climate\_CHELSAv2.1{-}bio1\_cell.tif"}

\NormalTok{df }\OtherTok{\textless{}{-}} \FunctionTok{downscale2egv}\NormalTok{(}
 \AttributeTok{template\_path =} \StringTok{"./Templates/TemplateRasters/LV100m\_10km.tif"}\NormalTok{,}
 \AttributeTok{grid\_path   =} \StringTok{"./Templates/TemplateGrids/tikls1km\_sauzeme.parquet"}\NormalTok{,}
 \AttributeTok{rawfile\_path =}\NormalTok{ reading,}
 \AttributeTok{out\_path   =} \StringTok{"./RasterGrids\_100m/2024/RAW/"}\NormalTok{,}
 \AttributeTok{file\_name   =}\NormalTok{ localname,}
 \AttributeTok{layer\_name  =}\NormalTok{ layername,}
 \AttributeTok{fill\_gaps   =} \ConstantTok{TRUE}\NormalTok{,}
 \AttributeTok{smooth    =} \ConstantTok{TRUE}\NormalTok{,}
 \AttributeTok{smooth\_radius\_km =} \DecValTok{5}\NormalTok{,}
 \AttributeTok{plot\_result  =} \ConstantTok{TRUE}\NormalTok{)}
\FunctionTok{print}\NormalTok{(df)}

\CommentTok{\# standardisation {-}{-}{-}{-}}
\ControlFlowTok{if}\NormalTok{(}\SpecialCharTok{!}\FunctionTok{require}\NormalTok{(terra)) \{}\FunctionTok{install.packages}\NormalTok{(}\StringTok{"terra"}\NormalTok{); }\FunctionTok{require}\NormalTok{(terra)\}}
\ControlFlowTok{if}\NormalTok{(}\SpecialCharTok{!}\FunctionTok{require}\NormalTok{(tidyverse)) \{}\FunctionTok{install.packages}\NormalTok{(}\StringTok{"tidyverse"}\NormalTok{); }\FunctionTok{require}\NormalTok{(tidyverse)\}}
\NormalTok{nosaukums}\OtherTok{=}\StringTok{"Climate\_CHELSAv2.1{-}bio1\_cell.tif"}
\NormalTok{ielasisanas\_cels}\OtherTok{=}\FunctionTok{paste0}\NormalTok{(}\StringTok{"./RasterGrids\_100m/2024/RAW/"}\NormalTok{,nosaukums)}
\NormalTok{saglabasanas\_cels}\OtherTok{=}\FunctionTok{paste0}\NormalTok{(}\StringTok{"./RasterGrids\_100m/2024/Scaled/"}\NormalTok{,nosaukums)}
\NormalTok{slanis}\OtherTok{=}\FunctionTok{rast}\NormalTok{(ielasisanas\_cels)}
\NormalTok{videjais}\OtherTok{=}\FunctionTok{global}\NormalTok{(slanis,}\AttributeTok{fun=}\StringTok{"mean"}\NormalTok{,}\AttributeTok{na.rm=}\ConstantTok{TRUE}\NormalTok{)}
\NormalTok{centrets}\OtherTok{=}\NormalTok{slanis}\SpecialCharTok{{-}}\NormalTok{videjais[,}\DecValTok{1}\NormalTok{]}
\NormalTok{standartnovirze}\OtherTok{=}\NormalTok{terra}\SpecialCharTok{::}\FunctionTok{global}\NormalTok{(centrets,}\AttributeTok{fun=}\StringTok{"rms"}\NormalTok{,}\AttributeTok{na.rm=}\ConstantTok{TRUE}\NormalTok{)}
\NormalTok{merogots}\OtherTok{=}\NormalTok{centrets}\SpecialCharTok{/}\NormalTok{standartnovirze[,}\DecValTok{1}\NormalTok{]}
\FunctionTok{writeRaster}\NormalTok{(merogots,}
      \AttributeTok{filename=}\NormalTok{saglabasanas\_cels,}
      \AttributeTok{overwrite=}\ConstantTok{TRUE}\NormalTok{)}
\end{Highlighting}
\end{Shaded}

\section{Climate\_CHELSAv2.1-bio10\_cell}\label{ch06.002}

\textbf{filename:} \texttt{Climate\_CHELSAv2.1-bio10\_cell.tif}

\textbf{layername:} \texttt{egv\_002}

\textbf{English name:} Mean daily mean air temperatures (°C) of the warmest quarter
(CHELSA v2.1) within the analysis cell (1 ha)

\textbf{Latvian name:} Gada siltākā ceturkšņa vidējā ik dienas vidējā gaisa temperatūra (°C) (CHELSA
v2.1) analīzes šūnā (1 ha)

\textbf{Procedure:} Directly follows \hyperref[Ch04.11]{CHELSA v2.1}. EGV is prepared using
the workflow \texttt{egvtools::downscale2egv()} with inverse distance weighted (power =
2) gap filling and soft smoothing (power = 0.5) over 5 km radius around each cell.
Finally, the layer is standardised by subtracting the arithmetic mean and
dividing by the root mean squared error.

\begin{Shaded}
\begin{Highlighting}[]
\CommentTok{\# libs {-}{-}{-}{-}}
\ControlFlowTok{if}\NormalTok{(}\SpecialCharTok{!}\FunctionTok{require}\NormalTok{(egvtools)) \{remotes}\SpecialCharTok{::}\FunctionTok{install\_github}\NormalTok{(}\StringTok{"aavotins/egvtools"}\NormalTok{); }\FunctionTok{require}\NormalTok{(egvtools)\}}

\CommentTok{\# job {-}{-}{-}{-}}

\NormalTok{localname}\OtherTok{=}\StringTok{"Climate\_CHELSAv2.1{-}bio10\_cell.tif"}
\NormalTok{layername}\OtherTok{=}\StringTok{"egv\_002"}
\NormalTok{reading}\OtherTok{=}\StringTok{"./Geodata/2024/CHELSA/Climate\_CHELSAv2.1{-}bio10\_cell.tif"}

\NormalTok{df }\OtherTok{\textless{}{-}} \FunctionTok{downscale2egv}\NormalTok{(}
 \AttributeTok{template\_path =} \StringTok{"./Templates/TemplateRasters/LV100m\_10km.tif"}\NormalTok{,}
 \AttributeTok{grid\_path   =} \StringTok{"./Templates/TemplateGrids/tikls1km\_sauzeme.parquet"}\NormalTok{,}
 \AttributeTok{rawfile\_path =}\NormalTok{ reading,}
 \AttributeTok{out\_path   =} \StringTok{"./RasterGrids\_100m/2024/RAW/"}\NormalTok{,}
 \AttributeTok{file\_name   =}\NormalTok{ localname,}
 \AttributeTok{layer\_name  =}\NormalTok{ layername,}
 \AttributeTok{fill\_gaps   =} \ConstantTok{TRUE}\NormalTok{,}
 \AttributeTok{smooth    =} \ConstantTok{TRUE}\NormalTok{,}
 \AttributeTok{smooth\_radius\_km =} \DecValTok{5}\NormalTok{,}
 \AttributeTok{plot\_result  =} \ConstantTok{TRUE}\NormalTok{)}
\FunctionTok{print}\NormalTok{(df)}

\CommentTok{\# standardisation {-}{-}{-}{-}}
\ControlFlowTok{if}\NormalTok{(}\SpecialCharTok{!}\FunctionTok{require}\NormalTok{(terra)) \{}\FunctionTok{install.packages}\NormalTok{(}\StringTok{"terra"}\NormalTok{); }\FunctionTok{require}\NormalTok{(terra)\}}
\ControlFlowTok{if}\NormalTok{(}\SpecialCharTok{!}\FunctionTok{require}\NormalTok{(tidyverse)) \{}\FunctionTok{install.packages}\NormalTok{(}\StringTok{"tidyverse"}\NormalTok{); }\FunctionTok{require}\NormalTok{(tidyverse)\}}

\NormalTok{nosaukums}\OtherTok{=}\StringTok{"Climate\_CHELSAv2.1{-}bio10\_cell.tif"}
\NormalTok{ielasisanas\_cels}\OtherTok{=}\FunctionTok{paste0}\NormalTok{(}\StringTok{"./RasterGrids\_100m/2024/RAW/"}\NormalTok{,nosaukums)}
\NormalTok{saglabasanas\_cels}\OtherTok{=}\FunctionTok{paste0}\NormalTok{(}\StringTok{"./RasterGrids\_100m/2024/Scaled/"}\NormalTok{,nosaukums)}
\NormalTok{slanis}\OtherTok{=}\FunctionTok{rast}\NormalTok{(ielasisanas\_cels)}
\NormalTok{videjais}\OtherTok{=}\FunctionTok{global}\NormalTok{(slanis,}\AttributeTok{fun=}\StringTok{"mean"}\NormalTok{,}\AttributeTok{na.rm=}\ConstantTok{TRUE}\NormalTok{)}
\NormalTok{centrets}\OtherTok{=}\NormalTok{slanis}\SpecialCharTok{{-}}\NormalTok{videjais[,}\DecValTok{1}\NormalTok{]}
\NormalTok{standartnovirze}\OtherTok{=}\NormalTok{terra}\SpecialCharTok{::}\FunctionTok{global}\NormalTok{(centrets,}\AttributeTok{fun=}\StringTok{"rms"}\NormalTok{,}\AttributeTok{na.rm=}\ConstantTok{TRUE}\NormalTok{)}
\NormalTok{merogots}\OtherTok{=}\NormalTok{centrets}\SpecialCharTok{/}\NormalTok{standartnovirze[,}\DecValTok{1}\NormalTok{]}
\FunctionTok{writeRaster}\NormalTok{(merogots,}
      \AttributeTok{filename=}\NormalTok{saglabasanas\_cels,}
      \AttributeTok{overwrite=}\ConstantTok{TRUE}\NormalTok{)}
\end{Highlighting}
\end{Shaded}

\section{Climate\_CHELSAv2.1-bio11\_cell}\label{ch06.003}

\textbf{filename:} \texttt{Climate\_CHELSAv2.1-bio11\_cell.tif}

\textbf{layername:} \texttt{egv\_003}

\textbf{English name:} Mean daily mean air temperatures (°C) of the coldest quarter
(CHELSA v2.1) within the analysis cell (1 ha)

\textbf{Latvian name:} Gada aukstākā ceturkšņa vidējā ik dienas vidējā gaisa temperatūra (°C) (CHELSA
v2.1) analīzes šūnā (1 ha)

\textbf{Procedure:} Directly follows \hyperref[Ch04.11]{CHELSA v2.1}. EGV is prepared using
the workflow \texttt{egvtools::downscale2egv()} with inverse distance weighted (power =
2) gap filling and soft smoothing (power = 0.5) over 5 km radius around each cell.
Finally, the layer is standardised by subtracting the arithmetic mean and
dividing by the root mean squared error.

\begin{Shaded}
\begin{Highlighting}[]
\CommentTok{\# libs {-}{-}{-}{-}}
\ControlFlowTok{if}\NormalTok{(}\SpecialCharTok{!}\FunctionTok{require}\NormalTok{(egvtools)) \{remotes}\SpecialCharTok{::}\FunctionTok{install\_github}\NormalTok{(}\StringTok{"aavotins/egvtools"}\NormalTok{); }\FunctionTok{require}\NormalTok{(egvtools)\}}

\CommentTok{\# job {-}{-}{-}{-}}

\NormalTok{localname}\OtherTok{=}\StringTok{"Climate\_CHELSAv2.1{-}bio11\_cell.tif"}
\NormalTok{layername}\OtherTok{=}\StringTok{"egv\_003"}
\NormalTok{reading}\OtherTok{=}\StringTok{"./Geodata/2024/CHELSA/Climate\_CHELSAv2.1{-}bio11\_cell.tif"}

\NormalTok{df }\OtherTok{\textless{}{-}} \FunctionTok{downscale2egv}\NormalTok{(}
 \AttributeTok{template\_path =} \StringTok{"./Templates/TemplateRasters/LV100m\_10km.tif"}\NormalTok{,}
 \AttributeTok{grid\_path   =} \StringTok{"./Templates/TemplateGrids/tikls1km\_sauzeme.parquet"}\NormalTok{,}
 \AttributeTok{rawfile\_path =}\NormalTok{ reading,}
 \AttributeTok{out\_path   =} \StringTok{"./RasterGrids\_100m/2024/RAW/"}\NormalTok{,}
 \AttributeTok{file\_name   =}\NormalTok{ localname,}
 \AttributeTok{layer\_name  =}\NormalTok{ layername,}
 \AttributeTok{fill\_gaps   =} \ConstantTok{TRUE}\NormalTok{,}
 \AttributeTok{smooth    =} \ConstantTok{TRUE}\NormalTok{,}
 \AttributeTok{smooth\_radius\_km =} \DecValTok{5}\NormalTok{,}
 \AttributeTok{plot\_result  =} \ConstantTok{TRUE}\NormalTok{)}
\FunctionTok{print}\NormalTok{(df)}

\CommentTok{\# standardisation {-}{-}{-}{-}}
\ControlFlowTok{if}\NormalTok{(}\SpecialCharTok{!}\FunctionTok{require}\NormalTok{(terra)) \{}\FunctionTok{install.packages}\NormalTok{(}\StringTok{"terra"}\NormalTok{); }\FunctionTok{require}\NormalTok{(terra)\}}
\ControlFlowTok{if}\NormalTok{(}\SpecialCharTok{!}\FunctionTok{require}\NormalTok{(tidyverse)) \{}\FunctionTok{install.packages}\NormalTok{(}\StringTok{"tidyverse"}\NormalTok{); }\FunctionTok{require}\NormalTok{(tidyverse)\}}

\NormalTok{nosaukums}\OtherTok{=}\StringTok{"Climate\_CHELSAv2.1{-}bio11\_cell.tif"}
\NormalTok{ielasisanas\_cels}\OtherTok{=}\FunctionTok{paste0}\NormalTok{(}\StringTok{"./RasterGrids\_100m/2024/RAW/"}\NormalTok{,nosaukums)}
\NormalTok{saglabasanas\_cels}\OtherTok{=}\FunctionTok{paste0}\NormalTok{(}\StringTok{"./RasterGrids\_100m/2024/Scaled/"}\NormalTok{,nosaukums)}
\NormalTok{slanis}\OtherTok{=}\FunctionTok{rast}\NormalTok{(ielasisanas\_cels)}
\NormalTok{videjais}\OtherTok{=}\FunctionTok{global}\NormalTok{(slanis,}\AttributeTok{fun=}\StringTok{"mean"}\NormalTok{,}\AttributeTok{na.rm=}\ConstantTok{TRUE}\NormalTok{)}
\NormalTok{centrets}\OtherTok{=}\NormalTok{slanis}\SpecialCharTok{{-}}\NormalTok{videjais[,}\DecValTok{1}\NormalTok{]}
\NormalTok{standartnovirze}\OtherTok{=}\NormalTok{terra}\SpecialCharTok{::}\FunctionTok{global}\NormalTok{(centrets,}\AttributeTok{fun=}\StringTok{"rms"}\NormalTok{,}\AttributeTok{na.rm=}\ConstantTok{TRUE}\NormalTok{)}
\NormalTok{merogots}\OtherTok{=}\NormalTok{centrets}\SpecialCharTok{/}\NormalTok{standartnovirze[,}\DecValTok{1}\NormalTok{]}
\FunctionTok{writeRaster}\NormalTok{(merogots,}
      \AttributeTok{filename=}\NormalTok{saglabasanas\_cels,}
      \AttributeTok{overwrite=}\ConstantTok{TRUE}\NormalTok{)}
\end{Highlighting}
\end{Shaded}

\section{Climate\_CHELSAv2.1-bio12\_cell}\label{ch06.004}

\textbf{filename:} \texttt{Climate\_CHELSAv2.1-bio12\_cell.tif}

\textbf{layername:} \texttt{egv\_004}

\textbf{English name:} Annual precipitation amount (kg m⁻² year⁻¹) (CHELSA v2.1)
within the analysis cell (1 ha)

\textbf{Latvian name:} Nokrišņu daudzums (kg m⁻² gadā) gadā (CHELSA v2.1) analīzes
šūnā (1 ha)

\textbf{Procedure:} Directly follows \hyperref[Ch04.11]{CHELSA v2.1}. EGV is prepared using
the workflow \texttt{egvtools::downscale2egv()} with inverse distance weighted (power =
2) gap filling and soft smoothing (power = 0.5) over 5 km radius around each cell.
Finally, the layer is standardised by subtracting the arithmetic mean and
dividing by the root mean squared error.

\begin{Shaded}
\begin{Highlighting}[]
\CommentTok{\# libs {-}{-}{-}{-}}
\ControlFlowTok{if}\NormalTok{(}\SpecialCharTok{!}\FunctionTok{require}\NormalTok{(egvtools)) \{remotes}\SpecialCharTok{::}\FunctionTok{install\_github}\NormalTok{(}\StringTok{"aavotins/egvtools"}\NormalTok{); }\FunctionTok{require}\NormalTok{(egvtools)\}}

\CommentTok{\# job {-}{-}{-}{-}}

\NormalTok{localname}\OtherTok{=}\StringTok{"Climate\_CHELSAv2.1{-}bio12\_cell.tif"}
\NormalTok{layername}\OtherTok{=}\StringTok{"egv\_004"}
\NormalTok{reading}\OtherTok{=}\StringTok{"./Geodata/2024/CHELSA/Climate\_CHELSAv2.1{-}bio12\_cell.tif"}

\NormalTok{df }\OtherTok{\textless{}{-}} \FunctionTok{downscale2egv}\NormalTok{(}
 \AttributeTok{template\_path =} \StringTok{"./Templates/TemplateRasters/LV100m\_10km.tif"}\NormalTok{,}
 \AttributeTok{grid\_path   =} \StringTok{"./Templates/TemplateGrids/tikls1km\_sauzeme.parquet"}\NormalTok{,}
 \AttributeTok{rawfile\_path =}\NormalTok{ reading,}
 \AttributeTok{out\_path   =} \StringTok{"./RasterGrids\_100m/2024/RAW/"}\NormalTok{,}
 \AttributeTok{file\_name   =}\NormalTok{ localname,}
 \AttributeTok{layer\_name  =}\NormalTok{ layername,}
 \AttributeTok{fill\_gaps   =} \ConstantTok{TRUE}\NormalTok{,}
 \AttributeTok{smooth    =} \ConstantTok{TRUE}\NormalTok{,}
 \AttributeTok{smooth\_radius\_km =} \DecValTok{5}\NormalTok{,}
 \AttributeTok{plot\_result  =} \ConstantTok{TRUE}\NormalTok{)}
\FunctionTok{print}\NormalTok{(df)}

\CommentTok{\# standardisation {-}{-}{-}{-}}
\ControlFlowTok{if}\NormalTok{(}\SpecialCharTok{!}\FunctionTok{require}\NormalTok{(terra)) \{}\FunctionTok{install.packages}\NormalTok{(}\StringTok{"terra"}\NormalTok{); }\FunctionTok{require}\NormalTok{(terra)\}}
\ControlFlowTok{if}\NormalTok{(}\SpecialCharTok{!}\FunctionTok{require}\NormalTok{(tidyverse)) \{}\FunctionTok{install.packages}\NormalTok{(}\StringTok{"tidyverse"}\NormalTok{); }\FunctionTok{require}\NormalTok{(tidyverse)\}}

\NormalTok{nosaukums}\OtherTok{=}\StringTok{"Climate\_CHELSAv2.1{-}bio12\_cell.tif"}
\NormalTok{ielasisanas\_cels}\OtherTok{=}\FunctionTok{paste0}\NormalTok{(}\StringTok{"./RasterGrids\_100m/2024/RAW/"}\NormalTok{,nosaukums)}
\NormalTok{saglabasanas\_cels}\OtherTok{=}\FunctionTok{paste0}\NormalTok{(}\StringTok{"./RasterGrids\_100m/2024/Scaled/"}\NormalTok{,nosaukums)}
\NormalTok{slanis}\OtherTok{=}\FunctionTok{rast}\NormalTok{(ielasisanas\_cels)}
\NormalTok{videjais}\OtherTok{=}\FunctionTok{global}\NormalTok{(slanis,}\AttributeTok{fun=}\StringTok{"mean"}\NormalTok{,}\AttributeTok{na.rm=}\ConstantTok{TRUE}\NormalTok{)}
\NormalTok{centrets}\OtherTok{=}\NormalTok{slanis}\SpecialCharTok{{-}}\NormalTok{videjais[,}\DecValTok{1}\NormalTok{]}
\NormalTok{standartnovirze}\OtherTok{=}\NormalTok{terra}\SpecialCharTok{::}\FunctionTok{global}\NormalTok{(centrets,}\AttributeTok{fun=}\StringTok{"rms"}\NormalTok{,}\AttributeTok{na.rm=}\ConstantTok{TRUE}\NormalTok{)}
\NormalTok{merogots}\OtherTok{=}\NormalTok{centrets}\SpecialCharTok{/}\NormalTok{standartnovirze[,}\DecValTok{1}\NormalTok{]}
\FunctionTok{writeRaster}\NormalTok{(merogots,}
      \AttributeTok{filename=}\NormalTok{saglabasanas\_cels,}
      \AttributeTok{overwrite=}\ConstantTok{TRUE}\NormalTok{)}
\end{Highlighting}
\end{Shaded}

\section{Climate\_CHELSAv2.1-bio13\_cell}\label{ch06.005}

\textbf{filename:} \texttt{Climate\_CHELSAv2.1-bio13\_cell.tif}

\textbf{layername:} \texttt{egv\_005}

\textbf{English name:} Precipitation amount (kg m⁻² month⁻¹) of the wettest month
(CHELSA v2.1) within the analysis cell (1 ha)

\textbf{Latvian name:} Slapjākā mēneša nokrišņu daudzums (kg m⁻² mēnesī) (CHELSA
v2.1) analīzes šūnā (1 ha)

\textbf{Procedure:} Directly follows \hyperref[Ch04.11]{CHELSA v2.1}. EGV is prepared using
the workflow \texttt{egvtools::downscale2egv()} with inverse distance weighted (power =
2) gap filling and soft smoothing (power = 0.5) over 5 km radius around each cell.
Finally, the layer is standardised by subtracting the arithmetic mean and
dividing by the root mean squared error.

\begin{Shaded}
\begin{Highlighting}[]
\CommentTok{\# libs {-}{-}{-}{-}}
\ControlFlowTok{if}\NormalTok{(}\SpecialCharTok{!}\FunctionTok{require}\NormalTok{(egvtools)) \{remotes}\SpecialCharTok{::}\FunctionTok{install\_github}\NormalTok{(}\StringTok{"aavotins/egvtools"}\NormalTok{); }\FunctionTok{require}\NormalTok{(egvtools)\}}

\CommentTok{\# job {-}{-}{-}{-}}

\NormalTok{localname}\OtherTok{=}\StringTok{"Climate\_CHELSAv2.1{-}bio13\_cell.tif"}
\NormalTok{layername}\OtherTok{=}\StringTok{"egv\_005"}
\NormalTok{reading}\OtherTok{=}\StringTok{"./Geodata/2024/CHELSA/Climate\_CHELSAv2.1{-}bio13\_cell.tif"}

\NormalTok{df }\OtherTok{\textless{}{-}} \FunctionTok{downscale2egv}\NormalTok{(}
 \AttributeTok{template\_path =} \StringTok{"./Templates/TemplateRasters/LV100m\_10km.tif"}\NormalTok{,}
 \AttributeTok{grid\_path   =} \StringTok{"./Templates/TemplateGrids/tikls1km\_sauzeme.parquet"}\NormalTok{,}
 \AttributeTok{rawfile\_path =}\NormalTok{ reading,}
 \AttributeTok{out\_path   =} \StringTok{"./RasterGrids\_100m/2024/RAW/"}\NormalTok{,}
 \AttributeTok{file\_name   =}\NormalTok{ localname,}
 \AttributeTok{layer\_name  =}\NormalTok{ layername,}
 \AttributeTok{fill\_gaps   =} \ConstantTok{TRUE}\NormalTok{,}
 \AttributeTok{smooth    =} \ConstantTok{TRUE}\NormalTok{,}
 \AttributeTok{smooth\_radius\_km =} \DecValTok{5}\NormalTok{,}
 \AttributeTok{plot\_result  =} \ConstantTok{TRUE}\NormalTok{)}
\FunctionTok{print}\NormalTok{(df)}

\CommentTok{\# standardisation {-}{-}{-}{-}}
\ControlFlowTok{if}\NormalTok{(}\SpecialCharTok{!}\FunctionTok{require}\NormalTok{(terra)) \{}\FunctionTok{install.packages}\NormalTok{(}\StringTok{"terra"}\NormalTok{); }\FunctionTok{require}\NormalTok{(terra)\}}
\ControlFlowTok{if}\NormalTok{(}\SpecialCharTok{!}\FunctionTok{require}\NormalTok{(tidyverse)) \{}\FunctionTok{install.packages}\NormalTok{(}\StringTok{"tidyverse"}\NormalTok{); }\FunctionTok{require}\NormalTok{(tidyverse)\}}

\NormalTok{nosaukums}\OtherTok{=}\StringTok{"Climate\_CHELSAv2.1{-}bio13\_cell.tif"}
\NormalTok{ielasisanas\_cels}\OtherTok{=}\FunctionTok{paste0}\NormalTok{(}\StringTok{"./RasterGrids\_100m/2024/RAW/"}\NormalTok{,nosaukums)}
\NormalTok{saglabasanas\_cels}\OtherTok{=}\FunctionTok{paste0}\NormalTok{(}\StringTok{"./RasterGrids\_100m/2024/Scaled/"}\NormalTok{,nosaukums)}
\NormalTok{slanis}\OtherTok{=}\FunctionTok{rast}\NormalTok{(ielasisanas\_cels)}
\NormalTok{videjais}\OtherTok{=}\FunctionTok{global}\NormalTok{(slanis,}\AttributeTok{fun=}\StringTok{"mean"}\NormalTok{,}\AttributeTok{na.rm=}\ConstantTok{TRUE}\NormalTok{)}
\NormalTok{centrets}\OtherTok{=}\NormalTok{slanis}\SpecialCharTok{{-}}\NormalTok{videjais[,}\DecValTok{1}\NormalTok{]}
\NormalTok{standartnovirze}\OtherTok{=}\NormalTok{terra}\SpecialCharTok{::}\FunctionTok{global}\NormalTok{(centrets,}\AttributeTok{fun=}\StringTok{"rms"}\NormalTok{,}\AttributeTok{na.rm=}\ConstantTok{TRUE}\NormalTok{)}
\NormalTok{merogots}\OtherTok{=}\NormalTok{centrets}\SpecialCharTok{/}\NormalTok{standartnovirze[,}\DecValTok{1}\NormalTok{]}
\FunctionTok{writeRaster}\NormalTok{(merogots,}
      \AttributeTok{filename=}\NormalTok{saglabasanas\_cels,}
      \AttributeTok{overwrite=}\ConstantTok{TRUE}\NormalTok{)}
\end{Highlighting}
\end{Shaded}

\section{Climate\_CHELSAv2.1-bio14\_cell}\label{ch06.006}

\textbf{filename:} \texttt{Climate\_CHELSAv2.1-bio14\_cell.tif}

\textbf{layername:} \texttt{egv\_006}

\textbf{English name:} Precipitation amount (kg m⁻² month⁻¹) of the driest month
(CHELSA v2.1) within the analysis cell (1 ha)

\textbf{Latvian name:} Sausākā mēneša nokrišņu daudzums (kg m⁻² mēnesī) (CHELSA v2.1)
analīzes šūnā (1 ha)

\textbf{Procedure:} Directly follows \hyperref[Ch04.11]{CHELSA v2.1}. EGV is prepared using
the workflow \texttt{egvtools::downscale2egv()} with inverse distance weighted (power =
2) gap filling and soft smoothing (power = 0.5) over 5 km radius around each cell.
Finally, the layer is standardised by subtracting the arithmetic mean and
dividing by the root mean squared error.

\begin{Shaded}
\begin{Highlighting}[]
\CommentTok{\# libs {-}{-}{-}{-}}
\ControlFlowTok{if}\NormalTok{(}\SpecialCharTok{!}\FunctionTok{require}\NormalTok{(egvtools)) \{remotes}\SpecialCharTok{::}\FunctionTok{install\_github}\NormalTok{(}\StringTok{"aavotins/egvtools"}\NormalTok{); }\FunctionTok{require}\NormalTok{(egvtools)\}}

\CommentTok{\# job {-}{-}{-}{-}}

\NormalTok{localname}\OtherTok{=}\StringTok{"Climate\_CHELSAv2.1{-}bio14\_cell.tif"}
\NormalTok{layername}\OtherTok{=}\StringTok{"egv\_006"}
\NormalTok{reading}\OtherTok{=}\StringTok{"./Geodata/2024/CHELSA/Climate\_CHELSAv2.1{-}bio14\_cell.tif"}

\NormalTok{df }\OtherTok{\textless{}{-}} \FunctionTok{downscale2egv}\NormalTok{(}
 \AttributeTok{template\_path =} \StringTok{"./Templates/TemplateRasters/LV100m\_10km.tif"}\NormalTok{,}
 \AttributeTok{grid\_path   =} \StringTok{"./Templates/TemplateGrids/tikls1km\_sauzeme.parquet"}\NormalTok{,}
 \AttributeTok{rawfile\_path =}\NormalTok{ reading,}
 \AttributeTok{out\_path   =} \StringTok{"./RasterGrids\_100m/2024/RAW/"}\NormalTok{,}
 \AttributeTok{file\_name   =}\NormalTok{ localname,}
 \AttributeTok{layer\_name  =}\NormalTok{ layername,}
 \AttributeTok{fill\_gaps   =} \ConstantTok{TRUE}\NormalTok{,}
 \AttributeTok{smooth    =} \ConstantTok{TRUE}\NormalTok{,}
 \AttributeTok{smooth\_radius\_km =} \DecValTok{5}\NormalTok{,}
 \AttributeTok{plot\_result  =} \ConstantTok{TRUE}\NormalTok{)}
\FunctionTok{print}\NormalTok{(df)}

\CommentTok{\# standardisation {-}{-}{-}{-}}
\ControlFlowTok{if}\NormalTok{(}\SpecialCharTok{!}\FunctionTok{require}\NormalTok{(terra)) \{}\FunctionTok{install.packages}\NormalTok{(}\StringTok{"terra"}\NormalTok{); }\FunctionTok{require}\NormalTok{(terra)\}}
\ControlFlowTok{if}\NormalTok{(}\SpecialCharTok{!}\FunctionTok{require}\NormalTok{(tidyverse)) \{}\FunctionTok{install.packages}\NormalTok{(}\StringTok{"tidyverse"}\NormalTok{); }\FunctionTok{require}\NormalTok{(tidyverse)\}}

\NormalTok{nosaukums}\OtherTok{=}\StringTok{"Climate\_CHELSAv2.1{-}bio14\_cell.tif"}
\NormalTok{ielasisanas\_cels}\OtherTok{=}\FunctionTok{paste0}\NormalTok{(}\StringTok{"./RasterGrids\_100m/2024/RAW/"}\NormalTok{,nosaukums)}
\NormalTok{saglabasanas\_cels}\OtherTok{=}\FunctionTok{paste0}\NormalTok{(}\StringTok{"./RasterGrids\_100m/2024/Scaled/"}\NormalTok{,nosaukums)}
\NormalTok{slanis}\OtherTok{=}\FunctionTok{rast}\NormalTok{(ielasisanas\_cels)}
\NormalTok{videjais}\OtherTok{=}\FunctionTok{global}\NormalTok{(slanis,}\AttributeTok{fun=}\StringTok{"mean"}\NormalTok{,}\AttributeTok{na.rm=}\ConstantTok{TRUE}\NormalTok{)}
\NormalTok{centrets}\OtherTok{=}\NormalTok{slanis}\SpecialCharTok{{-}}\NormalTok{videjais[,}\DecValTok{1}\NormalTok{]}
\NormalTok{standartnovirze}\OtherTok{=}\NormalTok{terra}\SpecialCharTok{::}\FunctionTok{global}\NormalTok{(centrets,}\AttributeTok{fun=}\StringTok{"rms"}\NormalTok{,}\AttributeTok{na.rm=}\ConstantTok{TRUE}\NormalTok{)}
\NormalTok{merogots}\OtherTok{=}\NormalTok{centrets}\SpecialCharTok{/}\NormalTok{standartnovirze[,}\DecValTok{1}\NormalTok{]}
\FunctionTok{writeRaster}\NormalTok{(merogots,}
      \AttributeTok{filename=}\NormalTok{saglabasanas\_cels,}
      \AttributeTok{overwrite=}\ConstantTok{TRUE}\NormalTok{)}
\end{Highlighting}
\end{Shaded}

\section{Climate\_CHELSAv2.1-bio15\_cell}\label{ch06.007}

\textbf{filename:} \texttt{Climate\_CHELSAv2.1-bio15\_cell.tif}

\textbf{layername:} \texttt{egv\_007}

\textbf{English name:} Precipitation seasonality (kg m⁻²) (CHELSA v2.1) within the
analysis cell (1 ha)

\textbf{Latvian name:} Nokrišņu sezonalitāte (kg m⁻²) (CHELSA v2.1) analīzes šūnā (1
ha)

\textbf{Procedure:} Directly follows \hyperref[Ch04.11]{CHELSA v2.1}. EGV is prepared using
the workflow \texttt{egvtools::downscale2egv()} with inverse distance weighted (power =
2) gap filling and soft smoothing (power = 0.5) over 5 km radius around each cell.
Finally, the layer is standardised by subtracting the arithmetic mean and
dividing by the root mean squared error.

\begin{Shaded}
\begin{Highlighting}[]
\CommentTok{\# libs {-}{-}{-}{-}}
\ControlFlowTok{if}\NormalTok{(}\SpecialCharTok{!}\FunctionTok{require}\NormalTok{(egvtools)) \{remotes}\SpecialCharTok{::}\FunctionTok{install\_github}\NormalTok{(}\StringTok{"aavotins/egvtools"}\NormalTok{); }\FunctionTok{require}\NormalTok{(egvtools)\}}

\CommentTok{\# job {-}{-}{-}{-}}

\NormalTok{localname}\OtherTok{=}\StringTok{"Climate\_CHELSAv2.1{-}bio15\_cell.tif"}
\NormalTok{layername}\OtherTok{=}\StringTok{"egv\_007"}
\NormalTok{reading}\OtherTok{=}\StringTok{"./Geodata/2024/CHELSA/Climate\_CHELSAv2.1{-}bio15\_cell.tif"}

\NormalTok{df }\OtherTok{\textless{}{-}} \FunctionTok{downscale2egv}\NormalTok{(}
 \AttributeTok{template\_path =} \StringTok{"./Templates/TemplateRasters/LV100m\_10km.tif"}\NormalTok{,}
 \AttributeTok{grid\_path   =} \StringTok{"./Templates/TemplateGrids/tikls1km\_sauzeme.parquet"}\NormalTok{,}
 \AttributeTok{rawfile\_path =}\NormalTok{ reading,}
 \AttributeTok{out\_path   =} \StringTok{"./RasterGrids\_100m/2024/RAW/"}\NormalTok{,}
 \AttributeTok{file\_name   =}\NormalTok{ localname,}
 \AttributeTok{layer\_name  =}\NormalTok{ layername,}
 \AttributeTok{fill\_gaps   =} \ConstantTok{TRUE}\NormalTok{,}
 \AttributeTok{smooth    =} \ConstantTok{TRUE}\NormalTok{,}
 \AttributeTok{smooth\_radius\_km =} \DecValTok{5}\NormalTok{,}
 \AttributeTok{plot\_result  =} \ConstantTok{TRUE}\NormalTok{)}
\FunctionTok{print}\NormalTok{(df)}

\CommentTok{\# standardisation {-}{-}{-}{-}}
\ControlFlowTok{if}\NormalTok{(}\SpecialCharTok{!}\FunctionTok{require}\NormalTok{(terra)) \{}\FunctionTok{install.packages}\NormalTok{(}\StringTok{"terra"}\NormalTok{); }\FunctionTok{require}\NormalTok{(terra)\}}
\ControlFlowTok{if}\NormalTok{(}\SpecialCharTok{!}\FunctionTok{require}\NormalTok{(tidyverse)) \{}\FunctionTok{install.packages}\NormalTok{(}\StringTok{"tidyverse"}\NormalTok{); }\FunctionTok{require}\NormalTok{(tidyverse)\}}

\NormalTok{nosaukums}\OtherTok{=}\StringTok{"Climate\_CHELSAv2.1{-}bio15\_cell.tif"}
\NormalTok{ielasisanas\_cels}\OtherTok{=}\FunctionTok{paste0}\NormalTok{(}\StringTok{"./RasterGrids\_100m/2024/RAW/"}\NormalTok{,nosaukums)}
\NormalTok{saglabasanas\_cels}\OtherTok{=}\FunctionTok{paste0}\NormalTok{(}\StringTok{"./RasterGrids\_100m/2024/Scaled/"}\NormalTok{,nosaukums)}
\NormalTok{slanis}\OtherTok{=}\FunctionTok{rast}\NormalTok{(ielasisanas\_cels)}
\NormalTok{videjais}\OtherTok{=}\FunctionTok{global}\NormalTok{(slanis,}\AttributeTok{fun=}\StringTok{"mean"}\NormalTok{,}\AttributeTok{na.rm=}\ConstantTok{TRUE}\NormalTok{)}
\NormalTok{centrets}\OtherTok{=}\NormalTok{slanis}\SpecialCharTok{{-}}\NormalTok{videjais[,}\DecValTok{1}\NormalTok{]}
\NormalTok{standartnovirze}\OtherTok{=}\NormalTok{terra}\SpecialCharTok{::}\FunctionTok{global}\NormalTok{(centrets,}\AttributeTok{fun=}\StringTok{"rms"}\NormalTok{,}\AttributeTok{na.rm=}\ConstantTok{TRUE}\NormalTok{)}
\NormalTok{merogots}\OtherTok{=}\NormalTok{centrets}\SpecialCharTok{/}\NormalTok{standartnovirze[,}\DecValTok{1}\NormalTok{]}
\FunctionTok{writeRaster}\NormalTok{(merogots,}
      \AttributeTok{filename=}\NormalTok{saglabasanas\_cels,}
      \AttributeTok{overwrite=}\ConstantTok{TRUE}\NormalTok{)}
\end{Highlighting}
\end{Shaded}

\section{Climate\_CHELSAv2.1-bio16\_cell}\label{ch06.008}

\textbf{filename:} \texttt{Climate\_CHELSAv2.1-bio16\_cell.tif}

\textbf{layername:} \texttt{egv\_008}

\textbf{English name:} Mean monthly precipitation amount (kg m⁻² month⁻¹) of the
wettest quarter (CHELSA v2.1) within the analysis cell (1 ha)

\textbf{Latvian name:} Slapjākā ceturkšņa vidējais nokrišņu daudzums mēnesī (kg m⁻²
mēnesī) (CHELSA v2.1) analīzes šūnā (1 ha)

\textbf{Procedure:} Directly follows \hyperref[Ch04.11]{CHELSA v2.1}. EGV is prepared using
the workflow \texttt{egvtools::downscale2egv()} with inverse distance weighted (power =
2) gap filling and soft smoothing (power = 0.5) over 5 km radius around each cell.
Finally, the layer is standardised by subtracting the arithmetic mean and
dividing by the root mean squared error.

\begin{Shaded}
\begin{Highlighting}[]
\CommentTok{\# libs {-}{-}{-}{-}}
\ControlFlowTok{if}\NormalTok{(}\SpecialCharTok{!}\FunctionTok{require}\NormalTok{(egvtools)) \{remotes}\SpecialCharTok{::}\FunctionTok{install\_github}\NormalTok{(}\StringTok{"aavotins/egvtools"}\NormalTok{); }\FunctionTok{require}\NormalTok{(egvtools)\}}

\CommentTok{\# job {-}{-}{-}{-}}

\NormalTok{localname}\OtherTok{=}\StringTok{"Climate\_CHELSAv2.1{-}bio16\_cell.tif"}
\NormalTok{layername}\OtherTok{=}\StringTok{"egv\_008"}
\NormalTok{reading}\OtherTok{=}\StringTok{"./Geodata/2024/CHELSA/Climate\_CHELSAv2.1{-}bio16\_cell.tif"}

\NormalTok{df }\OtherTok{\textless{}{-}} \FunctionTok{downscale2egv}\NormalTok{(}
 \AttributeTok{template\_path =} \StringTok{"./Templates/TemplateRasters/LV100m\_10km.tif"}\NormalTok{,}
 \AttributeTok{grid\_path   =} \StringTok{"./Templates/TemplateGrids/tikls1km\_sauzeme.parquet"}\NormalTok{,}
 \AttributeTok{rawfile\_path =}\NormalTok{ reading,}
 \AttributeTok{out\_path   =} \StringTok{"./RasterGrids\_100m/2024/RAW/"}\NormalTok{,}
 \AttributeTok{file\_name   =}\NormalTok{ localname,}
 \AttributeTok{layer\_name  =}\NormalTok{ layername,}
 \AttributeTok{fill\_gaps   =} \ConstantTok{TRUE}\NormalTok{,}
 \AttributeTok{smooth    =} \ConstantTok{TRUE}\NormalTok{,}
 \AttributeTok{smooth\_radius\_km =} \DecValTok{5}\NormalTok{,}
 \AttributeTok{plot\_result  =} \ConstantTok{TRUE}\NormalTok{)}
\FunctionTok{print}\NormalTok{(df)}

\CommentTok{\# standardisation {-}{-}{-}{-}}
\ControlFlowTok{if}\NormalTok{(}\SpecialCharTok{!}\FunctionTok{require}\NormalTok{(terra)) \{}\FunctionTok{install.packages}\NormalTok{(}\StringTok{"terra"}\NormalTok{); }\FunctionTok{require}\NormalTok{(terra)\}}
\ControlFlowTok{if}\NormalTok{(}\SpecialCharTok{!}\FunctionTok{require}\NormalTok{(tidyverse)) \{}\FunctionTok{install.packages}\NormalTok{(}\StringTok{"tidyverse"}\NormalTok{); }\FunctionTok{require}\NormalTok{(tidyverse)\}}

\NormalTok{nosaukums}\OtherTok{=}\StringTok{"Climate\_CHELSAv2.1{-}bio16\_cell.tif"}
\NormalTok{ielasisanas\_cels}\OtherTok{=}\FunctionTok{paste0}\NormalTok{(}\StringTok{"./RasterGrids\_100m/2024/RAW/"}\NormalTok{,nosaukums)}
\NormalTok{saglabasanas\_cels}\OtherTok{=}\FunctionTok{paste0}\NormalTok{(}\StringTok{"./RasterGrids\_100m/2024/Scaled/"}\NormalTok{,nosaukums)}
\NormalTok{slanis}\OtherTok{=}\FunctionTok{rast}\NormalTok{(ielasisanas\_cels)}
\NormalTok{videjais}\OtherTok{=}\FunctionTok{global}\NormalTok{(slanis,}\AttributeTok{fun=}\StringTok{"mean"}\NormalTok{,}\AttributeTok{na.rm=}\ConstantTok{TRUE}\NormalTok{)}
\NormalTok{centrets}\OtherTok{=}\NormalTok{slanis}\SpecialCharTok{{-}}\NormalTok{videjais[,}\DecValTok{1}\NormalTok{]}
\NormalTok{standartnovirze}\OtherTok{=}\NormalTok{terra}\SpecialCharTok{::}\FunctionTok{global}\NormalTok{(centrets,}\AttributeTok{fun=}\StringTok{"rms"}\NormalTok{,}\AttributeTok{na.rm=}\ConstantTok{TRUE}\NormalTok{)}
\NormalTok{merogots}\OtherTok{=}\NormalTok{centrets}\SpecialCharTok{/}\NormalTok{standartnovirze[,}\DecValTok{1}\NormalTok{]}
\FunctionTok{writeRaster}\NormalTok{(merogots,}
      \AttributeTok{filename=}\NormalTok{saglabasanas\_cels,}
      \AttributeTok{overwrite=}\ConstantTok{TRUE}\NormalTok{)}
\end{Highlighting}
\end{Shaded}

\section{Climate\_CHELSAv2.1-bio17\_cell}\label{ch06.009}

\textbf{filename:} \texttt{Climate\_CHELSAv2.1-bio17\_cell.tif}

\textbf{layername:} \texttt{egv\_009}

\textbf{English name:} Mean monthly precipitation amount (kg m⁻² month⁻¹) of the
driest quarter (CHELSA v2.1) within the analysis cell (1 ha)

\textbf{Latvian name:} Sausākā ceturkšņa vidējais nokrišņu daudzums mēnesī (kg m⁻²
mēnesī) (CHELSA v2.1) analīzes šūnā (1 ha)

\textbf{Procedure:} Directly follows \hyperref[Ch04.11]{CHELSA v2.1}. EGV is prepared using
the workflow \texttt{egvtools::downscale2egv()} with inverse distance weighted (power =
2) gap filling and soft smoothing (power = 0.5) over 5 km radius around each cell.
Finally, the layer is standardised by subtracting the arithmetic mean and
dividing by the root mean squared error.

\begin{Shaded}
\begin{Highlighting}[]
\CommentTok{\# libs {-}{-}{-}{-}}
\ControlFlowTok{if}\NormalTok{(}\SpecialCharTok{!}\FunctionTok{require}\NormalTok{(egvtools)) \{remotes}\SpecialCharTok{::}\FunctionTok{install\_github}\NormalTok{(}\StringTok{"aavotins/egvtools"}\NormalTok{); }\FunctionTok{require}\NormalTok{(egvtools)\}}

\CommentTok{\# job {-}{-}{-}{-}}

\NormalTok{localname}\OtherTok{=}\StringTok{"Climate\_CHELSAv2.1{-}bio17\_cell.tif"}
\NormalTok{layername}\OtherTok{=}\StringTok{"egv\_009"}
\NormalTok{reading}\OtherTok{=}\StringTok{"./Geodata/2024/CHELSA/Climate\_CHELSAv2.1{-}bio17\_cell.tif"}

\NormalTok{df }\OtherTok{\textless{}{-}} \FunctionTok{downscale2egv}\NormalTok{(}
 \AttributeTok{template\_path =} \StringTok{"./Templates/TemplateRasters/LV100m\_10km.tif"}\NormalTok{,}
 \AttributeTok{grid\_path   =} \StringTok{"./Templates/TemplateGrids/tikls1km\_sauzeme.parquet"}\NormalTok{,}
 \AttributeTok{rawfile\_path =}\NormalTok{ reading,}
 \AttributeTok{out\_path   =} \StringTok{"./RasterGrids\_100m/2024/RAW/"}\NormalTok{,}
 \AttributeTok{file\_name   =}\NormalTok{ localname,}
 \AttributeTok{layer\_name  =}\NormalTok{ layername,}
 \AttributeTok{fill\_gaps   =} \ConstantTok{TRUE}\NormalTok{,}
 \AttributeTok{smooth    =} \ConstantTok{TRUE}\NormalTok{,}
 \AttributeTok{smooth\_radius\_km =} \DecValTok{5}\NormalTok{,}
 \AttributeTok{plot\_result  =} \ConstantTok{TRUE}\NormalTok{)}
\FunctionTok{print}\NormalTok{(df)}

\CommentTok{\# standardisation {-}{-}{-}{-}}
\ControlFlowTok{if}\NormalTok{(}\SpecialCharTok{!}\FunctionTok{require}\NormalTok{(terra)) \{}\FunctionTok{install.packages}\NormalTok{(}\StringTok{"terra"}\NormalTok{); }\FunctionTok{require}\NormalTok{(terra)\}}
\ControlFlowTok{if}\NormalTok{(}\SpecialCharTok{!}\FunctionTok{require}\NormalTok{(tidyverse)) \{}\FunctionTok{install.packages}\NormalTok{(}\StringTok{"tidyverse"}\NormalTok{); }\FunctionTok{require}\NormalTok{(tidyverse)\}}

\NormalTok{nosaukums}\OtherTok{=}\StringTok{"Climate\_CHELSAv2.1{-}bio17\_cell.tif"}
\NormalTok{ielasisanas\_cels}\OtherTok{=}\FunctionTok{paste0}\NormalTok{(}\StringTok{"./RasterGrids\_100m/2024/RAW/"}\NormalTok{,nosaukums)}
\NormalTok{saglabasanas\_cels}\OtherTok{=}\FunctionTok{paste0}\NormalTok{(}\StringTok{"./RasterGrids\_100m/2024/Scaled/"}\NormalTok{,nosaukums)}
\NormalTok{slanis}\OtherTok{=}\FunctionTok{rast}\NormalTok{(ielasisanas\_cels)}
\NormalTok{videjais}\OtherTok{=}\FunctionTok{global}\NormalTok{(slanis,}\AttributeTok{fun=}\StringTok{"mean"}\NormalTok{,}\AttributeTok{na.rm=}\ConstantTok{TRUE}\NormalTok{)}
\NormalTok{centrets}\OtherTok{=}\NormalTok{slanis}\SpecialCharTok{{-}}\NormalTok{videjais[,}\DecValTok{1}\NormalTok{]}
\NormalTok{standartnovirze}\OtherTok{=}\NormalTok{terra}\SpecialCharTok{::}\FunctionTok{global}\NormalTok{(centrets,}\AttributeTok{fun=}\StringTok{"rms"}\NormalTok{,}\AttributeTok{na.rm=}\ConstantTok{TRUE}\NormalTok{)}
\NormalTok{merogots}\OtherTok{=}\NormalTok{centrets}\SpecialCharTok{/}\NormalTok{standartnovirze[,}\DecValTok{1}\NormalTok{]}
\FunctionTok{writeRaster}\NormalTok{(merogots,}
      \AttributeTok{filename=}\NormalTok{saglabasanas\_cels,}
      \AttributeTok{overwrite=}\ConstantTok{TRUE}\NormalTok{)}
\end{Highlighting}
\end{Shaded}

\section{Climate\_CHELSAv2.1-bio18\_cell}\label{ch06.010}

\textbf{filename:} \texttt{Climate\_CHELSAv2.1-bio18\_cell.tif}

\textbf{layername:} \texttt{egv\_010}

\textbf{English name:} Mean monthly precipitation amount (kg m⁻² month⁻¹) of the
warmest quarter (CHELSA v2.1) within the analysis cell (1 ha)

\textbf{Latvian name:} Siltākā ceturkšņa vidējais nokrišņu daudzums mēnesī (kg m⁻²
mēnesī) (CHELSA v2.1) analīzes šūnā (1 ha)

\textbf{Procedure:} Directly follows \hyperref[Ch04.11]{CHELSA v2.1}. EGV is prepared using
the workflow \texttt{egvtools::downscale2egv()} with inverse distance weighted (power =
2) gap filling and soft smoothing (power = 0.5) over 5 km radius around each cell.
Finally, the layer is standardised by subtracting the arithmetic mean and
dividing by the root mean squared error.

\begin{Shaded}
\begin{Highlighting}[]
\CommentTok{\# libs {-}{-}{-}{-}}
\ControlFlowTok{if}\NormalTok{(}\SpecialCharTok{!}\FunctionTok{require}\NormalTok{(egvtools)) \{remotes}\SpecialCharTok{::}\FunctionTok{install\_github}\NormalTok{(}\StringTok{"aavotins/egvtools"}\NormalTok{); }\FunctionTok{require}\NormalTok{(egvtools)\}}

\CommentTok{\# job {-}{-}{-}{-}}

\NormalTok{localname}\OtherTok{=}\StringTok{"Climate\_CHELSAv2.1{-}bio18\_cell.tif"}
\NormalTok{layername}\OtherTok{=}\StringTok{"egv\_010"}
\NormalTok{reading}\OtherTok{=}\StringTok{"./Geodata/2024/CHELSA/Climate\_CHELSAv2.1{-}bio18\_cell.tif"}

\NormalTok{df }\OtherTok{\textless{}{-}} \FunctionTok{downscale2egv}\NormalTok{(}
 \AttributeTok{template\_path =} \StringTok{"./Templates/TemplateRasters/LV100m\_10km.tif"}\NormalTok{,}
 \AttributeTok{grid\_path   =} \StringTok{"./Templates/TemplateGrids/tikls1km\_sauzeme.parquet"}\NormalTok{,}
 \AttributeTok{rawfile\_path =}\NormalTok{ reading,}
 \AttributeTok{out\_path   =} \StringTok{"./RasterGrids\_100m/2024/RAW/"}\NormalTok{,}
 \AttributeTok{file\_name   =}\NormalTok{ localname,}
 \AttributeTok{layer\_name  =}\NormalTok{ layername,}
 \AttributeTok{fill\_gaps   =} \ConstantTok{TRUE}\NormalTok{,}
 \AttributeTok{smooth    =} \ConstantTok{TRUE}\NormalTok{,}
 \AttributeTok{smooth\_radius\_km =} \DecValTok{5}\NormalTok{,}
 \AttributeTok{plot\_result  =} \ConstantTok{TRUE}\NormalTok{)}
\FunctionTok{print}\NormalTok{(df)}

\CommentTok{\# standardisation {-}{-}{-}{-}}
\ControlFlowTok{if}\NormalTok{(}\SpecialCharTok{!}\FunctionTok{require}\NormalTok{(terra)) \{}\FunctionTok{install.packages}\NormalTok{(}\StringTok{"terra"}\NormalTok{); }\FunctionTok{require}\NormalTok{(terra)\}}
\ControlFlowTok{if}\NormalTok{(}\SpecialCharTok{!}\FunctionTok{require}\NormalTok{(tidyverse)) \{}\FunctionTok{install.packages}\NormalTok{(}\StringTok{"tidyverse"}\NormalTok{); }\FunctionTok{require}\NormalTok{(tidyverse)\}}

\NormalTok{nosaukums}\OtherTok{=}\StringTok{"Climate\_CHELSAv2.1{-}bio18\_cell.tif"}
\NormalTok{ielasisanas\_cels}\OtherTok{=}\FunctionTok{paste0}\NormalTok{(}\StringTok{"./RasterGrids\_100m/2024/RAW/"}\NormalTok{,nosaukums)}
\NormalTok{saglabasanas\_cels}\OtherTok{=}\FunctionTok{paste0}\NormalTok{(}\StringTok{"./RasterGrids\_100m/2024/Scaled/"}\NormalTok{,nosaukums)}
\NormalTok{slanis}\OtherTok{=}\FunctionTok{rast}\NormalTok{(ielasisanas\_cels)}
\NormalTok{videjais}\OtherTok{=}\FunctionTok{global}\NormalTok{(slanis,}\AttributeTok{fun=}\StringTok{"mean"}\NormalTok{,}\AttributeTok{na.rm=}\ConstantTok{TRUE}\NormalTok{)}
\NormalTok{centrets}\OtherTok{=}\NormalTok{slanis}\SpecialCharTok{{-}}\NormalTok{videjais[,}\DecValTok{1}\NormalTok{]}
\NormalTok{standartnovirze}\OtherTok{=}\NormalTok{terra}\SpecialCharTok{::}\FunctionTok{global}\NormalTok{(centrets,}\AttributeTok{fun=}\StringTok{"rms"}\NormalTok{,}\AttributeTok{na.rm=}\ConstantTok{TRUE}\NormalTok{)}
\NormalTok{merogots}\OtherTok{=}\NormalTok{centrets}\SpecialCharTok{/}\NormalTok{standartnovirze[,}\DecValTok{1}\NormalTok{]}
\FunctionTok{writeRaster}\NormalTok{(merogots,}
      \AttributeTok{filename=}\NormalTok{saglabasanas\_cels,}
      \AttributeTok{overwrite=}\ConstantTok{TRUE}\NormalTok{)}
\end{Highlighting}
\end{Shaded}

\section{Climate\_CHELSAv2.1-bio19\_cell}\label{ch06.011}

\textbf{filename:} \texttt{Climate\_CHELSAv2.1-bio19\_cell.tif}

\textbf{layername:} \texttt{egv\_011}

\textbf{English name:} Mean monthly precipitation amount (kg m⁻² month⁻¹) of the
coldest quarter (CHELSA v2.1) within the analysis cell (1 ha)

\textbf{Latvian name:} Aukstākā ceturkšņa vidējais nokrišņu daudzums mēnesī (kg m⁻²
mēnesī) (CHELSA v2.1) analīzes šūnā (1 ha)

\textbf{Procedure:} Directly follows \hyperref[Ch04.11]{CHELSA v2.1}. EGV is prepared using
the workflow \texttt{egvtools::downscale2egv()} with inverse distance weighted (power =
2) gap filling and soft smoothing (power = 0.5) over 5 km radius around each cell.
Finally, the layer is standardised by subtracting the arithmetic mean and
dividing by the root mean squared error.

\begin{Shaded}
\begin{Highlighting}[]
\CommentTok{\# libs {-}{-}{-}{-}}
\ControlFlowTok{if}\NormalTok{(}\SpecialCharTok{!}\FunctionTok{require}\NormalTok{(egvtools)) \{remotes}\SpecialCharTok{::}\FunctionTok{install\_github}\NormalTok{(}\StringTok{"aavotins/egvtools"}\NormalTok{); }\FunctionTok{require}\NormalTok{(egvtools)\}}

\CommentTok{\# job {-}{-}{-}{-}}

\NormalTok{localname}\OtherTok{=}\StringTok{"Climate\_CHELSAv2.1{-}bio19\_cell.tif"}
\NormalTok{layername}\OtherTok{=}\StringTok{"egv\_011"}
\NormalTok{reading}\OtherTok{=}\StringTok{"./Geodata/2024/CHELSA/Climate\_CHELSAv2.1{-}bio19\_cell.tif"}

\NormalTok{df }\OtherTok{\textless{}{-}} \FunctionTok{downscale2egv}\NormalTok{(}
 \AttributeTok{template\_path =} \StringTok{"./Templates/TemplateRasters/LV100m\_10km.tif"}\NormalTok{,}
 \AttributeTok{grid\_path   =} \StringTok{"./Templates/TemplateGrids/tikls1km\_sauzeme.parquet"}\NormalTok{,}
 \AttributeTok{rawfile\_path =}\NormalTok{ reading,}
 \AttributeTok{out\_path   =} \StringTok{"./RasterGrids\_100m/2024/RAW/"}\NormalTok{,}
 \AttributeTok{file\_name   =}\NormalTok{ localname,}
 \AttributeTok{layer\_name  =}\NormalTok{ layername,}
 \AttributeTok{fill\_gaps   =} \ConstantTok{TRUE}\NormalTok{,}
 \AttributeTok{smooth    =} \ConstantTok{TRUE}\NormalTok{,}
 \AttributeTok{smooth\_radius\_km =} \DecValTok{5}\NormalTok{,}
 \AttributeTok{plot\_result  =} \ConstantTok{TRUE}\NormalTok{)}
\FunctionTok{print}\NormalTok{(df)}

\CommentTok{\# standardisation {-}{-}{-}{-}}
\ControlFlowTok{if}\NormalTok{(}\SpecialCharTok{!}\FunctionTok{require}\NormalTok{(terra)) \{}\FunctionTok{install.packages}\NormalTok{(}\StringTok{"terra"}\NormalTok{); }\FunctionTok{require}\NormalTok{(terra)\}}
\ControlFlowTok{if}\NormalTok{(}\SpecialCharTok{!}\FunctionTok{require}\NormalTok{(tidyverse)) \{}\FunctionTok{install.packages}\NormalTok{(}\StringTok{"tidyverse"}\NormalTok{); }\FunctionTok{require}\NormalTok{(tidyverse)\}}

\NormalTok{nosaukums}\OtherTok{=}\StringTok{"Climate\_CHELSAv2.1{-}bio19\_cell.tif"}
\NormalTok{ielasisanas\_cels}\OtherTok{=}\FunctionTok{paste0}\NormalTok{(}\StringTok{"./RasterGrids\_100m/2024/RAW/"}\NormalTok{,nosaukums)}
\NormalTok{saglabasanas\_cels}\OtherTok{=}\FunctionTok{paste0}\NormalTok{(}\StringTok{"./RasterGrids\_100m/2024/Scaled/"}\NormalTok{,nosaukums)}
\NormalTok{slanis}\OtherTok{=}\FunctionTok{rast}\NormalTok{(ielasisanas\_cels)}
\NormalTok{videjais}\OtherTok{=}\FunctionTok{global}\NormalTok{(slanis,}\AttributeTok{fun=}\StringTok{"mean"}\NormalTok{,}\AttributeTok{na.rm=}\ConstantTok{TRUE}\NormalTok{)}
\NormalTok{centrets}\OtherTok{=}\NormalTok{slanis}\SpecialCharTok{{-}}\NormalTok{videjais[,}\DecValTok{1}\NormalTok{]}
\NormalTok{standartnovirze}\OtherTok{=}\NormalTok{terra}\SpecialCharTok{::}\FunctionTok{global}\NormalTok{(centrets,}\AttributeTok{fun=}\StringTok{"rms"}\NormalTok{,}\AttributeTok{na.rm=}\ConstantTok{TRUE}\NormalTok{)}
\NormalTok{merogots}\OtherTok{=}\NormalTok{centrets}\SpecialCharTok{/}\NormalTok{standartnovirze[,}\DecValTok{1}\NormalTok{]}
\FunctionTok{writeRaster}\NormalTok{(merogots,}
      \AttributeTok{filename=}\NormalTok{saglabasanas\_cels,}
      \AttributeTok{overwrite=}\ConstantTok{TRUE}\NormalTok{)}
\end{Highlighting}
\end{Shaded}

\section{Climate\_CHELSAv2.1-bio2\_cell}\label{ch06.012}

\textbf{filename:} \texttt{Climate\_CHELSAv2.1-bio2\_cell.tif}

\textbf{layername:} \texttt{egv\_012}

\textbf{English name:} Mean diurnal air temperature range (°C) (CHELSA v2.1) within
the analysis cell (1 ha)

\textbf{Latvian name:} Vidējā diennakts gaisa temperatūru amplitūda (°C) (CHELSA v2.1) analīzes
šūnā (1 ha)

\textbf{Procedure:} Directly follows \hyperref[Ch04.11]{CHELSA v2.1}. EGV is prepared using
the workflow \texttt{egvtools::downscale2egv()} with inverse distance weighted (power =
2) gap filling and soft smoothing (power = 0.5) over 5 km radius around each cell.
Finally, the layer is standardised by subtracting the arithmetic mean and
dividing by the root mean squared error.

\begin{Shaded}
\begin{Highlighting}[]
\CommentTok{\# libs {-}{-}{-}{-}}
\ControlFlowTok{if}\NormalTok{(}\SpecialCharTok{!}\FunctionTok{require}\NormalTok{(egvtools)) \{remotes}\SpecialCharTok{::}\FunctionTok{install\_github}\NormalTok{(}\StringTok{"aavotins/egvtools"}\NormalTok{); }\FunctionTok{require}\NormalTok{(egvtools)\}}

\CommentTok{\# job {-}{-}{-}{-}}

\NormalTok{localname}\OtherTok{=}\StringTok{"Climate\_CHELSAv2.1{-}bio2\_cell.tif"}
\NormalTok{layername}\OtherTok{=}\StringTok{"egv\_012"}
\NormalTok{reading}\OtherTok{=}\StringTok{"./Geodata/2024/CHELSA/Climate\_CHELSAv2.1{-}bio2\_cell.tif"}

\NormalTok{df }\OtherTok{\textless{}{-}} \FunctionTok{downscale2egv}\NormalTok{(}
 \AttributeTok{template\_path =} \StringTok{"./Templates/TemplateRasters/LV100m\_10km.tif"}\NormalTok{,}
 \AttributeTok{grid\_path   =} \StringTok{"./Templates/TemplateGrids/tikls1km\_sauzeme.parquet"}\NormalTok{,}
 \AttributeTok{rawfile\_path =}\NormalTok{ reading,}
 \AttributeTok{out\_path   =} \StringTok{"./RasterGrids\_100m/2024/RAW/"}\NormalTok{,}
 \AttributeTok{file\_name   =}\NormalTok{ localname,}
 \AttributeTok{layer\_name  =}\NormalTok{ layername,}
 \AttributeTok{fill\_gaps   =} \ConstantTok{TRUE}\NormalTok{,}
 \AttributeTok{smooth    =} \ConstantTok{TRUE}\NormalTok{,}
 \AttributeTok{smooth\_radius\_km =} \DecValTok{5}\NormalTok{,}
 \AttributeTok{plot\_result  =} \ConstantTok{TRUE}\NormalTok{)}
\FunctionTok{print}\NormalTok{(df)}

\CommentTok{\# standardisation {-}{-}{-}{-}}
\ControlFlowTok{if}\NormalTok{(}\SpecialCharTok{!}\FunctionTok{require}\NormalTok{(terra)) \{}\FunctionTok{install.packages}\NormalTok{(}\StringTok{"terra"}\NormalTok{); }\FunctionTok{require}\NormalTok{(terra)\}}
\ControlFlowTok{if}\NormalTok{(}\SpecialCharTok{!}\FunctionTok{require}\NormalTok{(tidyverse)) \{}\FunctionTok{install.packages}\NormalTok{(}\StringTok{"tidyverse"}\NormalTok{); }\FunctionTok{require}\NormalTok{(tidyverse)\}}

\NormalTok{nosaukums}\OtherTok{=}\StringTok{"Climate\_CHELSAv2.1{-}bio2\_cell.tif"}
\NormalTok{ielasisanas\_cels}\OtherTok{=}\FunctionTok{paste0}\NormalTok{(}\StringTok{"./RasterGrids\_100m/2024/RAW/"}\NormalTok{,nosaukums)}
\NormalTok{saglabasanas\_cels}\OtherTok{=}\FunctionTok{paste0}\NormalTok{(}\StringTok{"./RasterGrids\_100m/2024/Scaled/"}\NormalTok{,nosaukums)}
\NormalTok{slanis}\OtherTok{=}\FunctionTok{rast}\NormalTok{(ielasisanas\_cels)}
\NormalTok{videjais}\OtherTok{=}\FunctionTok{global}\NormalTok{(slanis,}\AttributeTok{fun=}\StringTok{"mean"}\NormalTok{,}\AttributeTok{na.rm=}\ConstantTok{TRUE}\NormalTok{)}
\NormalTok{centrets}\OtherTok{=}\NormalTok{slanis}\SpecialCharTok{{-}}\NormalTok{videjais[,}\DecValTok{1}\NormalTok{]}
\NormalTok{standartnovirze}\OtherTok{=}\NormalTok{terra}\SpecialCharTok{::}\FunctionTok{global}\NormalTok{(centrets,}\AttributeTok{fun=}\StringTok{"rms"}\NormalTok{,}\AttributeTok{na.rm=}\ConstantTok{TRUE}\NormalTok{)}
\NormalTok{merogots}\OtherTok{=}\NormalTok{centrets}\SpecialCharTok{/}\NormalTok{standartnovirze[,}\DecValTok{1}\NormalTok{]}
\FunctionTok{writeRaster}\NormalTok{(merogots,}
      \AttributeTok{filename=}\NormalTok{saglabasanas\_cels,}
      \AttributeTok{overwrite=}\ConstantTok{TRUE}\NormalTok{)}
\end{Highlighting}
\end{Shaded}

\section{Climate\_CHELSAv2.1-bio3\_cell}\label{ch06.013}

\textbf{filename:} \texttt{Climate\_CHELSAv2.1-bio3\_cell.tif}

\textbf{layername:} \texttt{egv\_013}

\textbf{English name:} Isothermality (ratio of diurnal variation to annual variation
in air temperatures) (°C) (CHELSA v2.1) within the analysis cell (1 ha)

\textbf{Latvian name:} Izotermalitāte (attiecība starp diennakts un gada gaisa temperatūras
svārstībām) (°C) (CHELSA v2.1) analīzes šūnā (1 ha)

\textbf{Procedure:} Directly follows \hyperref[Ch04.11]{CHELSA v2.1}. EGV is prepared using
the workflow \texttt{egvtools::downscale2egv()} with inverse distance weighted (power =
2) gap filling and soft smoothing (power = 0.5) over 5 km radius around each cell.
Finally, the layer is standardised by subtracting the arithmetic mean and
dividing by the root mean squared error.

\begin{Shaded}
\begin{Highlighting}[]
\CommentTok{\# libs {-}{-}{-}{-}}
\ControlFlowTok{if}\NormalTok{(}\SpecialCharTok{!}\FunctionTok{require}\NormalTok{(egvtools)) \{remotes}\SpecialCharTok{::}\FunctionTok{install\_github}\NormalTok{(}\StringTok{"aavotins/egvtools"}\NormalTok{); }\FunctionTok{require}\NormalTok{(egvtools)\}}

\CommentTok{\# job {-}{-}{-}{-}}

\NormalTok{localname}\OtherTok{=}\StringTok{"Climate\_CHELSAv2.1{-}bio3\_cell.tif"}
\NormalTok{layername}\OtherTok{=}\StringTok{"egv\_013"}
\NormalTok{reading}\OtherTok{=}\StringTok{"./Geodata/2024/CHELSA/Climate\_CHELSAv2.1{-}bio3\_cell.tif"}

\NormalTok{df }\OtherTok{\textless{}{-}} \FunctionTok{downscale2egv}\NormalTok{(}
 \AttributeTok{template\_path =} \StringTok{"./Templates/TemplateRasters/LV100m\_10km.tif"}\NormalTok{,}
 \AttributeTok{grid\_path   =} \StringTok{"./Templates/TemplateGrids/tikls1km\_sauzeme.parquet"}\NormalTok{,}
 \AttributeTok{rawfile\_path =}\NormalTok{ reading,}
 \AttributeTok{out\_path   =} \StringTok{"./RasterGrids\_100m/2024/RAW/"}\NormalTok{,}
 \AttributeTok{file\_name   =}\NormalTok{ localname,}
 \AttributeTok{layer\_name  =}\NormalTok{ layername,}
 \AttributeTok{fill\_gaps   =} \ConstantTok{TRUE}\NormalTok{,}
 \AttributeTok{smooth    =} \ConstantTok{TRUE}\NormalTok{,}
 \AttributeTok{smooth\_radius\_km =} \DecValTok{5}\NormalTok{,}
 \AttributeTok{plot\_result  =} \ConstantTok{TRUE}\NormalTok{)}
\FunctionTok{print}\NormalTok{(df)}

\CommentTok{\# standardisation {-}{-}{-}{-}}
\ControlFlowTok{if}\NormalTok{(}\SpecialCharTok{!}\FunctionTok{require}\NormalTok{(terra)) \{}\FunctionTok{install.packages}\NormalTok{(}\StringTok{"terra"}\NormalTok{); }\FunctionTok{require}\NormalTok{(terra)\}}
\ControlFlowTok{if}\NormalTok{(}\SpecialCharTok{!}\FunctionTok{require}\NormalTok{(tidyverse)) \{}\FunctionTok{install.packages}\NormalTok{(}\StringTok{"tidyverse"}\NormalTok{); }\FunctionTok{require}\NormalTok{(tidyverse)\}}

\NormalTok{nosaukums}\OtherTok{=}\StringTok{"Climate\_CHELSAv2.1{-}bio3\_cell.tif"}
\NormalTok{ielasisanas\_cels}\OtherTok{=}\FunctionTok{paste0}\NormalTok{(}\StringTok{"./RasterGrids\_100m/2024/RAW/"}\NormalTok{,nosaukums)}
\NormalTok{saglabasanas\_cels}\OtherTok{=}\FunctionTok{paste0}\NormalTok{(}\StringTok{"./RasterGrids\_100m/2024/Scaled/"}\NormalTok{,nosaukums)}
\NormalTok{slanis}\OtherTok{=}\FunctionTok{rast}\NormalTok{(ielasisanas\_cels)}
\NormalTok{videjais}\OtherTok{=}\FunctionTok{global}\NormalTok{(slanis,}\AttributeTok{fun=}\StringTok{"mean"}\NormalTok{,}\AttributeTok{na.rm=}\ConstantTok{TRUE}\NormalTok{)}
\NormalTok{centrets}\OtherTok{=}\NormalTok{slanis}\SpecialCharTok{{-}}\NormalTok{videjais[,}\DecValTok{1}\NormalTok{]}
\NormalTok{standartnovirze}\OtherTok{=}\NormalTok{terra}\SpecialCharTok{::}\FunctionTok{global}\NormalTok{(centrets,}\AttributeTok{fun=}\StringTok{"rms"}\NormalTok{,}\AttributeTok{na.rm=}\ConstantTok{TRUE}\NormalTok{)}
\NormalTok{merogots}\OtherTok{=}\NormalTok{centrets}\SpecialCharTok{/}\NormalTok{standartnovirze[,}\DecValTok{1}\NormalTok{]}
\FunctionTok{writeRaster}\NormalTok{(merogots,}
      \AttributeTok{filename=}\NormalTok{saglabasanas\_cels,}
      \AttributeTok{overwrite=}\ConstantTok{TRUE}\NormalTok{)}
\end{Highlighting}
\end{Shaded}

\section{Climate\_CHELSAv2.1-bio4\_cell}\label{ch06.014}

\textbf{filename:} \texttt{Climate\_CHELSAv2.1-bio4\_cell.tif}

\textbf{layername:} \texttt{egv\_014}

\textbf{English name:} Temperature seasonality (standard deviation of the monthly
mean air temperatures) (°C/100) (CHELSA v2.1) within the analysis cell (1 ha)

\textbf{Latvian name:} Temperatūru sezonalitāte (mēneša vidējo gaisa temperatūru
standartnovirze) (°C/100) (CHELSA v2.1) analīzes šūnā (1 ha)

\textbf{Procedure:} Directly follows \hyperref[Ch04.11]{CHELSA v2.1}. EGV is prepared using
the workflow \texttt{egvtools::downscale2egv()} with inverse distance weighted (power =
2) gap filling and soft smoothing (power = 0.5) over 5 km radius around each cell.
Finally, the layer is standardised by subtracting the arithmetic mean and
dividing by the root mean squared error.

\begin{Shaded}
\begin{Highlighting}[]
\CommentTok{\# libs {-}{-}{-}{-}}
\ControlFlowTok{if}\NormalTok{(}\SpecialCharTok{!}\FunctionTok{require}\NormalTok{(egvtools)) \{remotes}\SpecialCharTok{::}\FunctionTok{install\_github}\NormalTok{(}\StringTok{"aavotins/egvtools"}\NormalTok{); }\FunctionTok{require}\NormalTok{(egvtools)\}}

\CommentTok{\# job {-}{-}{-}{-}}

\NormalTok{localname}\OtherTok{=}\StringTok{"Climate\_CHELSAv2.1{-}bio4\_cell.tif"}
\NormalTok{layername}\OtherTok{=}\StringTok{"egv\_014"}
\NormalTok{reading}\OtherTok{=}\StringTok{"./Geodata/2024/CHELSA/Climate\_CHELSAv2.1{-}bio4\_cell.tif"}

\NormalTok{df }\OtherTok{\textless{}{-}} \FunctionTok{downscale2egv}\NormalTok{(}
 \AttributeTok{template\_path =} \StringTok{"./Templates/TemplateRasters/LV100m\_10km.tif"}\NormalTok{,}
 \AttributeTok{grid\_path   =} \StringTok{"./Templates/TemplateGrids/tikls1km\_sauzeme.parquet"}\NormalTok{,}
 \AttributeTok{rawfile\_path =}\NormalTok{ reading,}
 \AttributeTok{out\_path   =} \StringTok{"./RasterGrids\_100m/2024/RAW/"}\NormalTok{,}
 \AttributeTok{file\_name   =}\NormalTok{ localname,}
 \AttributeTok{layer\_name  =}\NormalTok{ layername,}
 \AttributeTok{fill\_gaps   =} \ConstantTok{TRUE}\NormalTok{,}
 \AttributeTok{smooth    =} \ConstantTok{TRUE}\NormalTok{,}
 \AttributeTok{smooth\_radius\_km =} \DecValTok{5}\NormalTok{,}
 \AttributeTok{plot\_result  =} \ConstantTok{TRUE}\NormalTok{)}
\FunctionTok{print}\NormalTok{(df)}

\CommentTok{\# standardisation {-}{-}{-}{-}}
\ControlFlowTok{if}\NormalTok{(}\SpecialCharTok{!}\FunctionTok{require}\NormalTok{(terra)) \{}\FunctionTok{install.packages}\NormalTok{(}\StringTok{"terra"}\NormalTok{); }\FunctionTok{require}\NormalTok{(terra)\}}
\ControlFlowTok{if}\NormalTok{(}\SpecialCharTok{!}\FunctionTok{require}\NormalTok{(tidyverse)) \{}\FunctionTok{install.packages}\NormalTok{(}\StringTok{"tidyverse"}\NormalTok{); }\FunctionTok{require}\NormalTok{(tidyverse)\}}

\NormalTok{nosaukums}\OtherTok{=}\StringTok{"Climate\_CHELSAv2.1{-}bio4\_cell.tif"}
\NormalTok{ielasisanas\_cels}\OtherTok{=}\FunctionTok{paste0}\NormalTok{(}\StringTok{"./RasterGrids\_100m/2024/RAW/"}\NormalTok{,nosaukums)}
\NormalTok{saglabasanas\_cels}\OtherTok{=}\FunctionTok{paste0}\NormalTok{(}\StringTok{"./RasterGrids\_100m/2024/Scaled/"}\NormalTok{,nosaukums)}
\NormalTok{slanis}\OtherTok{=}\FunctionTok{rast}\NormalTok{(ielasisanas\_cels)}
\NormalTok{videjais}\OtherTok{=}\FunctionTok{global}\NormalTok{(slanis,}\AttributeTok{fun=}\StringTok{"mean"}\NormalTok{,}\AttributeTok{na.rm=}\ConstantTok{TRUE}\NormalTok{)}
\NormalTok{centrets}\OtherTok{=}\NormalTok{slanis}\SpecialCharTok{{-}}\NormalTok{videjais[,}\DecValTok{1}\NormalTok{]}
\NormalTok{standartnovirze}\OtherTok{=}\NormalTok{terra}\SpecialCharTok{::}\FunctionTok{global}\NormalTok{(centrets,}\AttributeTok{fun=}\StringTok{"rms"}\NormalTok{,}\AttributeTok{na.rm=}\ConstantTok{TRUE}\NormalTok{)}
\NormalTok{merogots}\OtherTok{=}\NormalTok{centrets}\SpecialCharTok{/}\NormalTok{standartnovirze[,}\DecValTok{1}\NormalTok{]}
\FunctionTok{writeRaster}\NormalTok{(merogots,}
      \AttributeTok{filename=}\NormalTok{saglabasanas\_cels,}
      \AttributeTok{overwrite=}\ConstantTok{TRUE}\NormalTok{)}
\end{Highlighting}
\end{Shaded}

\section{Climate\_CHELSAv2.1-bio5\_cell}\label{ch06.015}

\textbf{filename:} \texttt{Climate\_CHELSAv2.1-bio5\_cell.tif}

\textbf{layername:} \texttt{egv\_015}

\textbf{English name:} Mean daily maximum air temperature (°C) of the warmest month
(CHELSA v2.1) within the analysis cell (1 ha)

\textbf{Latvian name:} Siltākā mēneša vidējā ik dienas augstākā gaisa temperatūra (°C)
(CHELSA v2.1) analīzes šūnā (1 ha)

\textbf{Procedure:} Directly follows \hyperref[Ch04.11]{CHELSA v2.1}. EGV is prepared using
the workflow \texttt{egvtools::downscale2egv()} with inverse distance weighted (power =
2) gap filling and soft smoothing (power = 0.5) over 5 km radius around each cell.
Finally, the layer is standardised by subtracting the arithmetic mean and
dividing by the root mean squared error.

\begin{Shaded}
\begin{Highlighting}[]
\CommentTok{\# libs {-}{-}{-}{-}}
\ControlFlowTok{if}\NormalTok{(}\SpecialCharTok{!}\FunctionTok{require}\NormalTok{(egvtools)) \{remotes}\SpecialCharTok{::}\FunctionTok{install\_github}\NormalTok{(}\StringTok{"aavotins/egvtools"}\NormalTok{); }\FunctionTok{require}\NormalTok{(egvtools)\}}

\CommentTok{\# job {-}{-}{-}{-}}

\NormalTok{localname}\OtherTok{=}\StringTok{"Climate\_CHELSAv2.1{-}bio5\_cell.tif"}
\NormalTok{layername}\OtherTok{=}\StringTok{"egv\_015"}
\NormalTok{reading}\OtherTok{=}\StringTok{"./Geodata/2024/CHELSA/Climate\_CHELSAv2.1{-}bio5\_cell.tif"}

\NormalTok{df }\OtherTok{\textless{}{-}} \FunctionTok{downscale2egv}\NormalTok{(}
 \AttributeTok{template\_path =} \StringTok{"./Templates/TemplateRasters/LV100m\_10km.tif"}\NormalTok{,}
 \AttributeTok{grid\_path   =} \StringTok{"./Templates/TemplateGrids/tikls1km\_sauzeme.parquet"}\NormalTok{,}
 \AttributeTok{rawfile\_path =}\NormalTok{ reading,}
 \AttributeTok{out\_path   =} \StringTok{"./RasterGrids\_100m/2024/RAW/"}\NormalTok{,}
 \AttributeTok{file\_name   =}\NormalTok{ localname,}
 \AttributeTok{layer\_name  =}\NormalTok{ layername,}
 \AttributeTok{fill\_gaps   =} \ConstantTok{TRUE}\NormalTok{,}
 \AttributeTok{smooth    =} \ConstantTok{TRUE}\NormalTok{,}
 \AttributeTok{smooth\_radius\_km =} \DecValTok{5}\NormalTok{,}
 \AttributeTok{plot\_result  =} \ConstantTok{TRUE}\NormalTok{)}
\FunctionTok{print}\NormalTok{(df)}

\CommentTok{\# standardisation {-}{-}{-}{-}}
\ControlFlowTok{if}\NormalTok{(}\SpecialCharTok{!}\FunctionTok{require}\NormalTok{(terra)) \{}\FunctionTok{install.packages}\NormalTok{(}\StringTok{"terra"}\NormalTok{); }\FunctionTok{require}\NormalTok{(terra)\}}
\ControlFlowTok{if}\NormalTok{(}\SpecialCharTok{!}\FunctionTok{require}\NormalTok{(tidyverse)) \{}\FunctionTok{install.packages}\NormalTok{(}\StringTok{"tidyverse"}\NormalTok{); }\FunctionTok{require}\NormalTok{(tidyverse)\}}

\NormalTok{nosaukums}\OtherTok{=}\StringTok{"Climate\_CHELSAv2.1{-}bio5\_cell.tif"}
\NormalTok{ielasisanas\_cels}\OtherTok{=}\FunctionTok{paste0}\NormalTok{(}\StringTok{"./RasterGrids\_100m/2024/RAW/"}\NormalTok{,nosaukums)}
\NormalTok{saglabasanas\_cels}\OtherTok{=}\FunctionTok{paste0}\NormalTok{(}\StringTok{"./RasterGrids\_100m/2024/Scaled/"}\NormalTok{,nosaukums)}
\NormalTok{slanis}\OtherTok{=}\FunctionTok{rast}\NormalTok{(ielasisanas\_cels)}
\NormalTok{videjais}\OtherTok{=}\FunctionTok{global}\NormalTok{(slanis,}\AttributeTok{fun=}\StringTok{"mean"}\NormalTok{,}\AttributeTok{na.rm=}\ConstantTok{TRUE}\NormalTok{)}
\NormalTok{centrets}\OtherTok{=}\NormalTok{slanis}\SpecialCharTok{{-}}\NormalTok{videjais[,}\DecValTok{1}\NormalTok{]}
\NormalTok{standartnovirze}\OtherTok{=}\NormalTok{terra}\SpecialCharTok{::}\FunctionTok{global}\NormalTok{(centrets,}\AttributeTok{fun=}\StringTok{"rms"}\NormalTok{,}\AttributeTok{na.rm=}\ConstantTok{TRUE}\NormalTok{)}
\NormalTok{merogots}\OtherTok{=}\NormalTok{centrets}\SpecialCharTok{/}\NormalTok{standartnovirze[,}\DecValTok{1}\NormalTok{]}
\FunctionTok{writeRaster}\NormalTok{(merogots,}
      \AttributeTok{filename=}\NormalTok{saglabasanas\_cels,}
      \AttributeTok{overwrite=}\ConstantTok{TRUE}\NormalTok{)}
\end{Highlighting}
\end{Shaded}

\section{Climate\_CHELSAv2.1-bio6\_cell}\label{ch06.016}

\textbf{filename:} \texttt{Climate\_CHELSAv2.1-bio6\_cell.tif}

\textbf{layername:} \texttt{egv\_016}

\textbf{English name:} Mean daily minimum air temperature (°C) of the coldest month
(CHELSA v2.1) within the analysis cell (1 ha)

\textbf{Latvian name:} Aukstākā mēneša vidējā ik dienas zemākā gaisa temperatūra (°C)
(CHELSA v2.1) analīzes šūnā (1 ha)

\textbf{Procedure:} Directly follows \hyperref[Ch04.11]{CHELSA v2.1}. EGV is prepared using
the workflow \texttt{egvtools::downscale2egv()} with inverse distance weighted (power =
2) gap filling and soft smoothing (power = 0.5) over 5 km radius around each cell.
Finally, the layer is standardised by subtracting the arithmetic mean and
dividing by the root mean squared error.

\begin{Shaded}
\begin{Highlighting}[]
\CommentTok{\# libs {-}{-}{-}{-}}
\ControlFlowTok{if}\NormalTok{(}\SpecialCharTok{!}\FunctionTok{require}\NormalTok{(egvtools)) \{remotes}\SpecialCharTok{::}\FunctionTok{install\_github}\NormalTok{(}\StringTok{"aavotins/egvtools"}\NormalTok{); }\FunctionTok{require}\NormalTok{(egvtools)\}}

\CommentTok{\# job {-}{-}{-}{-}}

\NormalTok{localname}\OtherTok{=}\StringTok{"Climate\_CHELSAv2.1{-}bio6\_cell.tif"}
\NormalTok{layername}\OtherTok{=}\StringTok{"egv\_016"}
\NormalTok{reading}\OtherTok{=}\StringTok{"./Geodata/2024/CHELSA/Climate\_CHELSAv2.1{-}bio6\_cell.tif"}

\NormalTok{df }\OtherTok{\textless{}{-}} \FunctionTok{downscale2egv}\NormalTok{(}
 \AttributeTok{template\_path =} \StringTok{"./Templates/TemplateRasters/LV100m\_10km.tif"}\NormalTok{,}
 \AttributeTok{grid\_path   =} \StringTok{"./Templates/TemplateGrids/tikls1km\_sauzeme.parquet"}\NormalTok{,}
 \AttributeTok{rawfile\_path =}\NormalTok{ reading,}
 \AttributeTok{out\_path   =} \StringTok{"./RasterGrids\_100m/2024/RAW/"}\NormalTok{,}
 \AttributeTok{file\_name   =}\NormalTok{ localname,}
 \AttributeTok{layer\_name  =}\NormalTok{ layername,}
 \AttributeTok{fill\_gaps   =} \ConstantTok{TRUE}\NormalTok{,}
 \AttributeTok{smooth    =} \ConstantTok{TRUE}\NormalTok{,}
 \AttributeTok{smooth\_radius\_km =} \DecValTok{5}\NormalTok{,}
 \AttributeTok{plot\_result  =} \ConstantTok{TRUE}\NormalTok{)}
\FunctionTok{print}\NormalTok{(df)}

\CommentTok{\# standardisation {-}{-}{-}{-}}
\ControlFlowTok{if}\NormalTok{(}\SpecialCharTok{!}\FunctionTok{require}\NormalTok{(terra)) \{}\FunctionTok{install.packages}\NormalTok{(}\StringTok{"terra"}\NormalTok{); }\FunctionTok{require}\NormalTok{(terra)\}}
\ControlFlowTok{if}\NormalTok{(}\SpecialCharTok{!}\FunctionTok{require}\NormalTok{(tidyverse)) \{}\FunctionTok{install.packages}\NormalTok{(}\StringTok{"tidyverse"}\NormalTok{); }\FunctionTok{require}\NormalTok{(tidyverse)\}}

\NormalTok{nosaukums}\OtherTok{=}\StringTok{"Climate\_CHELSAv2.1{-}bio6\_cell.tif"}
\NormalTok{ielasisanas\_cels}\OtherTok{=}\FunctionTok{paste0}\NormalTok{(}\StringTok{"./RasterGrids\_100m/2024/RAW/"}\NormalTok{,nosaukums)}
\NormalTok{saglabasanas\_cels}\OtherTok{=}\FunctionTok{paste0}\NormalTok{(}\StringTok{"./RasterGrids\_100m/2024/Scaled/"}\NormalTok{,nosaukums)}
\NormalTok{slanis}\OtherTok{=}\FunctionTok{rast}\NormalTok{(ielasisanas\_cels)}
\NormalTok{videjais}\OtherTok{=}\FunctionTok{global}\NormalTok{(slanis,}\AttributeTok{fun=}\StringTok{"mean"}\NormalTok{,}\AttributeTok{na.rm=}\ConstantTok{TRUE}\NormalTok{)}
\NormalTok{centrets}\OtherTok{=}\NormalTok{slanis}\SpecialCharTok{{-}}\NormalTok{videjais[,}\DecValTok{1}\NormalTok{]}
\NormalTok{standartnovirze}\OtherTok{=}\NormalTok{terra}\SpecialCharTok{::}\FunctionTok{global}\NormalTok{(centrets,}\AttributeTok{fun=}\StringTok{"rms"}\NormalTok{,}\AttributeTok{na.rm=}\ConstantTok{TRUE}\NormalTok{)}
\NormalTok{merogots}\OtherTok{=}\NormalTok{centrets}\SpecialCharTok{/}\NormalTok{standartnovirze[,}\DecValTok{1}\NormalTok{]}
\FunctionTok{writeRaster}\NormalTok{(merogots,}
      \AttributeTok{filename=}\NormalTok{saglabasanas\_cels,}
      \AttributeTok{overwrite=}\ConstantTok{TRUE}\NormalTok{)}
\end{Highlighting}
\end{Shaded}

\section{Climate\_CHELSAv2.1-bio7\_cell}\label{ch06.017}

\textbf{filename:} \texttt{Climate\_CHELSAv2.1-bio7\_cell.tif}

\textbf{layername:} \texttt{egv\_017}

\textbf{English name:} Annual range of air temperature (°C) (CHELSA v2.1) within the
analysis cell (1 ha)

\textbf{Latvian name:} Gada gaisa temperatūru amplitūda (°C) (CHELSA v2.1) analīzes šūnā (1
ha)

\textbf{Procedure:} Directly follows \hyperref[Ch04.11]{CHELSA v2.1}. EGV is prepared using
the workflow \texttt{egvtools::downscale2egv()} with inverse distance weighted (power =
2) gap filling and soft smoothing (power = 0.5) over 5 km radius around each cell.
Finally, the layer is standardised by subtracting the arithmetic mean and
dividing by the root mean squared error.

\begin{Shaded}
\begin{Highlighting}[]
\CommentTok{\# libs {-}{-}{-}{-}}
\ControlFlowTok{if}\NormalTok{(}\SpecialCharTok{!}\FunctionTok{require}\NormalTok{(egvtools)) \{remotes}\SpecialCharTok{::}\FunctionTok{install\_github}\NormalTok{(}\StringTok{"aavotins/egvtools"}\NormalTok{); }\FunctionTok{require}\NormalTok{(egvtools)\}}

\CommentTok{\# job {-}{-}{-}{-}}

\NormalTok{localname}\OtherTok{=}\StringTok{"Climate\_CHELSAv2.1{-}bio7\_cell.tif"}
\NormalTok{layername}\OtherTok{=}\StringTok{"egv\_017"}
\NormalTok{reading}\OtherTok{=}\StringTok{"./Geodata/2024/CHELSA/Climate\_CHELSAv2.1{-}bio7\_cell.tif"}

\NormalTok{df }\OtherTok{\textless{}{-}} \FunctionTok{downscale2egv}\NormalTok{(}
 \AttributeTok{template\_path =} \StringTok{"./Templates/TemplateRasters/LV100m\_10km.tif"}\NormalTok{,}
 \AttributeTok{grid\_path   =} \StringTok{"./Templates/TemplateGrids/tikls1km\_sauzeme.parquet"}\NormalTok{,}
 \AttributeTok{rawfile\_path =}\NormalTok{ reading,}
 \AttributeTok{out\_path   =} \StringTok{"./RasterGrids\_100m/2024/RAW/"}\NormalTok{,}
 \AttributeTok{file\_name   =}\NormalTok{ localname,}
 \AttributeTok{layer\_name  =}\NormalTok{ layername,}
 \AttributeTok{fill\_gaps   =} \ConstantTok{TRUE}\NormalTok{,}
 \AttributeTok{smooth    =} \ConstantTok{TRUE}\NormalTok{,}
 \AttributeTok{smooth\_radius\_km =} \DecValTok{5}\NormalTok{,}
 \AttributeTok{plot\_result  =} \ConstantTok{TRUE}\NormalTok{)}
\FunctionTok{print}\NormalTok{(df)}

\CommentTok{\# standardisation {-}{-}{-}{-}}
\ControlFlowTok{if}\NormalTok{(}\SpecialCharTok{!}\FunctionTok{require}\NormalTok{(terra)) \{}\FunctionTok{install.packages}\NormalTok{(}\StringTok{"terra"}\NormalTok{); }\FunctionTok{require}\NormalTok{(terra)\}}
\ControlFlowTok{if}\NormalTok{(}\SpecialCharTok{!}\FunctionTok{require}\NormalTok{(tidyverse)) \{}\FunctionTok{install.packages}\NormalTok{(}\StringTok{"tidyverse"}\NormalTok{); }\FunctionTok{require}\NormalTok{(tidyverse)\}}

\NormalTok{nosaukums}\OtherTok{=}\StringTok{"Climate\_CHELSAv2.1{-}bio7\_cell.tif"}
\NormalTok{ielasisanas\_cels}\OtherTok{=}\FunctionTok{paste0}\NormalTok{(}\StringTok{"./RasterGrids\_100m/2024/RAW/"}\NormalTok{,nosaukums)}
\NormalTok{saglabasanas\_cels}\OtherTok{=}\FunctionTok{paste0}\NormalTok{(}\StringTok{"./RasterGrids\_100m/2024/Scaled/"}\NormalTok{,nosaukums)}
\NormalTok{slanis}\OtherTok{=}\FunctionTok{rast}\NormalTok{(ielasisanas\_cels)}
\NormalTok{videjais}\OtherTok{=}\FunctionTok{global}\NormalTok{(slanis,}\AttributeTok{fun=}\StringTok{"mean"}\NormalTok{,}\AttributeTok{na.rm=}\ConstantTok{TRUE}\NormalTok{)}
\NormalTok{centrets}\OtherTok{=}\NormalTok{slanis}\SpecialCharTok{{-}}\NormalTok{videjais[,}\DecValTok{1}\NormalTok{]}
\NormalTok{standartnovirze}\OtherTok{=}\NormalTok{terra}\SpecialCharTok{::}\FunctionTok{global}\NormalTok{(centrets,}\AttributeTok{fun=}\StringTok{"rms"}\NormalTok{,}\AttributeTok{na.rm=}\ConstantTok{TRUE}\NormalTok{)}
\NormalTok{merogots}\OtherTok{=}\NormalTok{centrets}\SpecialCharTok{/}\NormalTok{standartnovirze[,}\DecValTok{1}\NormalTok{]}
\FunctionTok{writeRaster}\NormalTok{(merogots,}
      \AttributeTok{filename=}\NormalTok{saglabasanas\_cels,}
      \AttributeTok{overwrite=}\ConstantTok{TRUE}\NormalTok{)}
\end{Highlighting}
\end{Shaded}

\section{Climate\_CHELSAv2.1-bio8\_cell}\label{ch06.018}

\textbf{filename:} \texttt{Climate\_CHELSAv2.1-bio8\_cell.tif}

\textbf{layername:} \texttt{egv\_018}

\textbf{English name:} Mean daily mean air temperatures (°C) of the wettest quarter
(CHELSA v2.1) within the analysis cell (1 ha)

\textbf{Latvian name:} Slapjākā ceturkšņa vidējā ik dienas vidējā gaisa temperatūra
(°C) (CHELSA v2.1) analīzes šūnā (1 ha)

\textbf{Procedure:} Directly follows \hyperref[Ch04.11]{CHELSA v2.1}. EGV is prepared using
the workflow \texttt{egvtools::downscale2egv()} with inverse distance weighted (power =
2) gap filling and soft smoothing (power = 0.5) over 5 km radius around each cell.
Finally, the layer is standardised by subtracting the arithmetic mean and
dividing by the root mean squared error.

\begin{Shaded}
\begin{Highlighting}[]
\CommentTok{\# libs {-}{-}{-}{-}}
\ControlFlowTok{if}\NormalTok{(}\SpecialCharTok{!}\FunctionTok{require}\NormalTok{(egvtools)) \{remotes}\SpecialCharTok{::}\FunctionTok{install\_github}\NormalTok{(}\StringTok{"aavotins/egvtools"}\NormalTok{); }\FunctionTok{require}\NormalTok{(egvtools)\}}

\CommentTok{\# job {-}{-}{-}{-}}

\NormalTok{localname}\OtherTok{=}\StringTok{"Climate\_CHELSAv2.1{-}bio8\_cell.tif"}
\NormalTok{layername}\OtherTok{=}\StringTok{"egv\_018"}
\NormalTok{reading}\OtherTok{=}\StringTok{"./Geodata/2024/CHELSA/Climate\_CHELSAv2.1{-}bio8\_cell.tif"}

\NormalTok{df }\OtherTok{\textless{}{-}} \FunctionTok{downscale2egv}\NormalTok{(}
 \AttributeTok{template\_path =} \StringTok{"./Templates/TemplateRasters/LV100m\_10km.tif"}\NormalTok{,}
 \AttributeTok{grid\_path   =} \StringTok{"./Templates/TemplateGrids/tikls1km\_sauzeme.parquet"}\NormalTok{,}
 \AttributeTok{rawfile\_path =}\NormalTok{ reading,}
 \AttributeTok{out\_path   =} \StringTok{"./RasterGrids\_100m/2024/RAW/"}\NormalTok{,}
 \AttributeTok{file\_name   =}\NormalTok{ localname,}
 \AttributeTok{layer\_name  =}\NormalTok{ layername,}
 \AttributeTok{fill\_gaps   =} \ConstantTok{TRUE}\NormalTok{,}
 \AttributeTok{smooth    =} \ConstantTok{TRUE}\NormalTok{,}
 \AttributeTok{smooth\_radius\_km =} \DecValTok{5}\NormalTok{,}
 \AttributeTok{plot\_result  =} \ConstantTok{TRUE}\NormalTok{)}
\FunctionTok{print}\NormalTok{(df)}

\CommentTok{\# standardisation {-}{-}{-}{-}}
\ControlFlowTok{if}\NormalTok{(}\SpecialCharTok{!}\FunctionTok{require}\NormalTok{(terra)) \{}\FunctionTok{install.packages}\NormalTok{(}\StringTok{"terra"}\NormalTok{); }\FunctionTok{require}\NormalTok{(terra)\}}
\ControlFlowTok{if}\NormalTok{(}\SpecialCharTok{!}\FunctionTok{require}\NormalTok{(tidyverse)) \{}\FunctionTok{install.packages}\NormalTok{(}\StringTok{"tidyverse"}\NormalTok{); }\FunctionTok{require}\NormalTok{(tidyverse)\}}

\NormalTok{nosaukums}\OtherTok{=}\StringTok{"Climate\_CHELSAv2.1{-}bio8\_cell.tif"}
\NormalTok{ielasisanas\_cels}\OtherTok{=}\FunctionTok{paste0}\NormalTok{(}\StringTok{"./RasterGrids\_100m/2024/RAW/"}\NormalTok{,nosaukums)}
\NormalTok{saglabasanas\_cels}\OtherTok{=}\FunctionTok{paste0}\NormalTok{(}\StringTok{"./RasterGrids\_100m/2024/Scaled/"}\NormalTok{,nosaukums)}
\NormalTok{slanis}\OtherTok{=}\FunctionTok{rast}\NormalTok{(ielasisanas\_cels)}
\NormalTok{videjais}\OtherTok{=}\FunctionTok{global}\NormalTok{(slanis,}\AttributeTok{fun=}\StringTok{"mean"}\NormalTok{,}\AttributeTok{na.rm=}\ConstantTok{TRUE}\NormalTok{)}
\NormalTok{centrets}\OtherTok{=}\NormalTok{slanis}\SpecialCharTok{{-}}\NormalTok{videjais[,}\DecValTok{1}\NormalTok{]}
\NormalTok{standartnovirze}\OtherTok{=}\NormalTok{terra}\SpecialCharTok{::}\FunctionTok{global}\NormalTok{(centrets,}\AttributeTok{fun=}\StringTok{"rms"}\NormalTok{,}\AttributeTok{na.rm=}\ConstantTok{TRUE}\NormalTok{)}
\NormalTok{merogots}\OtherTok{=}\NormalTok{centrets}\SpecialCharTok{/}\NormalTok{standartnovirze[,}\DecValTok{1}\NormalTok{]}
\FunctionTok{writeRaster}\NormalTok{(merogots,}
      \AttributeTok{filename=}\NormalTok{saglabasanas\_cels,}
      \AttributeTok{overwrite=}\ConstantTok{TRUE}\NormalTok{)}
\end{Highlighting}
\end{Shaded}

\section{Climate\_CHELSAv2.1-bio9\_cell}\label{ch06.019}

\textbf{filename:} \texttt{Climate\_CHELSAv2.1-bio9\_cell.tif}

\textbf{layername:} \texttt{egv\_019}

\textbf{English name:} Mean daily mean air temperatures (°C) of the driest quarter
(CHELSA v2.1) within the analysis cell (1 ha)

\textbf{Latvian name:} Sausākā ceturkšņa vidējā ik dienas vidējā gaisa temperatūra
(°C) (CHELSA v2.1) analīzes šūnā (1 ha)

\textbf{Procedure:} Directly follows \hyperref[Ch04.11]{CHELSA v2.1}. EGV is prepared using
the workflow \texttt{egvtools::downscale2egv()} with inverse distance weighted (power =
2) gap filling and soft smoothing (power = 0.5) over 5 km radius around each cell.
Finally, the layer is standardised by subtracting the arithmetic mean and
dividing by the root mean squared error.

\begin{Shaded}
\begin{Highlighting}[]
\CommentTok{\# libs {-}{-}{-}{-}}
\ControlFlowTok{if}\NormalTok{(}\SpecialCharTok{!}\FunctionTok{require}\NormalTok{(egvtools)) \{remotes}\SpecialCharTok{::}\FunctionTok{install\_github}\NormalTok{(}\StringTok{"aavotins/egvtools"}\NormalTok{); }\FunctionTok{require}\NormalTok{(egvtools)\}}

\CommentTok{\# job {-}{-}{-}{-}}

\NormalTok{localname}\OtherTok{=}\StringTok{"Climate\_CHELSAv2.1{-}bio9\_cell.tif"}
\NormalTok{layername}\OtherTok{=}\StringTok{"egv\_019"}
\NormalTok{reading}\OtherTok{=}\StringTok{"./Geodata/2024/CHELSA/Climate\_CHELSAv2.1{-}bio9\_cell.tif"}

\NormalTok{df }\OtherTok{\textless{}{-}} \FunctionTok{downscale2egv}\NormalTok{(}
 \AttributeTok{template\_path =} \StringTok{"./Templates/TemplateRasters/LV100m\_10km.tif"}\NormalTok{,}
 \AttributeTok{grid\_path   =} \StringTok{"./Templates/TemplateGrids/tikls1km\_sauzeme.parquet"}\NormalTok{,}
 \AttributeTok{rawfile\_path =}\NormalTok{ reading,}
 \AttributeTok{out\_path   =} \StringTok{"./RasterGrids\_100m/2024/RAW/"}\NormalTok{,}
 \AttributeTok{file\_name   =}\NormalTok{ localname,}
 \AttributeTok{layer\_name  =}\NormalTok{ layername,}
 \AttributeTok{fill\_gaps   =} \ConstantTok{TRUE}\NormalTok{,}
 \AttributeTok{smooth    =} \ConstantTok{TRUE}\NormalTok{,}
 \AttributeTok{smooth\_radius\_km =} \DecValTok{5}\NormalTok{,}
 \AttributeTok{plot\_result  =} \ConstantTok{TRUE}\NormalTok{)}
\FunctionTok{print}\NormalTok{(df)}

\CommentTok{\# standardisation {-}{-}{-}{-}}
\ControlFlowTok{if}\NormalTok{(}\SpecialCharTok{!}\FunctionTok{require}\NormalTok{(terra)) \{}\FunctionTok{install.packages}\NormalTok{(}\StringTok{"terra"}\NormalTok{); }\FunctionTok{require}\NormalTok{(terra)\}}
\ControlFlowTok{if}\NormalTok{(}\SpecialCharTok{!}\FunctionTok{require}\NormalTok{(tidyverse)) \{}\FunctionTok{install.packages}\NormalTok{(}\StringTok{"tidyverse"}\NormalTok{); }\FunctionTok{require}\NormalTok{(tidyverse)\}}

\NormalTok{nosaukums}\OtherTok{=}\StringTok{"Climate\_CHELSAv2.1{-}bio9\_cell.tif"}
\NormalTok{ielasisanas\_cels}\OtherTok{=}\FunctionTok{paste0}\NormalTok{(}\StringTok{"./RasterGrids\_100m/2024/RAW/"}\NormalTok{,nosaukums)}
\NormalTok{saglabasanas\_cels}\OtherTok{=}\FunctionTok{paste0}\NormalTok{(}\StringTok{"./RasterGrids\_100m/2024/Scaled/"}\NormalTok{,nosaukums)}
\NormalTok{slanis}\OtherTok{=}\FunctionTok{rast}\NormalTok{(ielasisanas\_cels)}
\NormalTok{videjais}\OtherTok{=}\FunctionTok{global}\NormalTok{(slanis,}\AttributeTok{fun=}\StringTok{"mean"}\NormalTok{,}\AttributeTok{na.rm=}\ConstantTok{TRUE}\NormalTok{)}
\NormalTok{centrets}\OtherTok{=}\NormalTok{slanis}\SpecialCharTok{{-}}\NormalTok{videjais[,}\DecValTok{1}\NormalTok{]}
\NormalTok{standartnovirze}\OtherTok{=}\NormalTok{terra}\SpecialCharTok{::}\FunctionTok{global}\NormalTok{(centrets,}\AttributeTok{fun=}\StringTok{"rms"}\NormalTok{,}\AttributeTok{na.rm=}\ConstantTok{TRUE}\NormalTok{)}
\NormalTok{merogots}\OtherTok{=}\NormalTok{centrets}\SpecialCharTok{/}\NormalTok{standartnovirze[,}\DecValTok{1}\NormalTok{]}
\FunctionTok{writeRaster}\NormalTok{(merogots,}
      \AttributeTok{filename=}\NormalTok{saglabasanas\_cels,}
      \AttributeTok{overwrite=}\ConstantTok{TRUE}\NormalTok{)}
\end{Highlighting}
\end{Shaded}

\section{Climate\_CHELSAv2.1-clt-max\_cell}\label{ch06.020}

\textbf{filename:} \texttt{Climate\_CHELSAv2.1-clt-max\_cell.tif}

\textbf{layername:} \texttt{egv\_020}

\textbf{English name:} Mean of monthly maximum cloud area fraction (\%) (CHELSA v2.1) within
the analysis cell (1 ha)

\textbf{Latvian name:} Mēneša maksimumu vidējais mākoņu segums (\%) (CHELSA v2.1)
analīzes šūnā (1 ha)

\textbf{Procedure:} Directly follows \hyperref[Ch04.11]{CHELSA v2.1}. EGV is prepared using
the workflow \texttt{egvtools::downscale2egv()} with inverse distance weighted (power =
2) gap filling and soft smoothing (power = 0.5) over 5 km radius around each cell.
Finally, the layer is standardised by subtracting the arithmetic mean and
dividing by the root mean squared error.

\begin{Shaded}
\begin{Highlighting}[]
\CommentTok{\# libs {-}{-}{-}{-}}
\ControlFlowTok{if}\NormalTok{(}\SpecialCharTok{!}\FunctionTok{require}\NormalTok{(egvtools)) \{remotes}\SpecialCharTok{::}\FunctionTok{install\_github}\NormalTok{(}\StringTok{"aavotins/egvtools"}\NormalTok{); }\FunctionTok{require}\NormalTok{(egvtools)\}}

\CommentTok{\# job {-}{-}{-}{-}}

\NormalTok{localname}\OtherTok{=}\StringTok{"Climate\_CHELSAv2.1{-}clt{-}max\_cell.tif"}
\NormalTok{layername}\OtherTok{=}\StringTok{"egv\_020"}
\NormalTok{reading}\OtherTok{=}\StringTok{"./Geodata/2024/CHELSA/Climate\_CHELSAv2.1{-}clt{-}max\_cell.tif"}

\NormalTok{df }\OtherTok{\textless{}{-}} \FunctionTok{downscale2egv}\NormalTok{(}
 \AttributeTok{template\_path =} \StringTok{"./Templates/TemplateRasters/LV100m\_10km.tif"}\NormalTok{,}
 \AttributeTok{grid\_path   =} \StringTok{"./Templates/TemplateGrids/tikls1km\_sauzeme.parquet"}\NormalTok{,}
 \AttributeTok{rawfile\_path =}\NormalTok{ reading,}
 \AttributeTok{out\_path   =} \StringTok{"./RasterGrids\_100m/2024/RAW/"}\NormalTok{,}
 \AttributeTok{file\_name   =}\NormalTok{ localname,}
 \AttributeTok{layer\_name  =}\NormalTok{ layername,}
 \AttributeTok{fill\_gaps   =} \ConstantTok{TRUE}\NormalTok{,}
 \AttributeTok{smooth    =} \ConstantTok{TRUE}\NormalTok{,}
 \AttributeTok{smooth\_radius\_km =} \DecValTok{5}\NormalTok{,}
 \AttributeTok{plot\_result  =} \ConstantTok{TRUE}\NormalTok{)}
\FunctionTok{print}\NormalTok{(df)}

\CommentTok{\# standardisation {-}{-}{-}{-}}
\ControlFlowTok{if}\NormalTok{(}\SpecialCharTok{!}\FunctionTok{require}\NormalTok{(terra)) \{}\FunctionTok{install.packages}\NormalTok{(}\StringTok{"terra"}\NormalTok{); }\FunctionTok{require}\NormalTok{(terra)\}}
\ControlFlowTok{if}\NormalTok{(}\SpecialCharTok{!}\FunctionTok{require}\NormalTok{(tidyverse)) \{}\FunctionTok{install.packages}\NormalTok{(}\StringTok{"tidyverse"}\NormalTok{); }\FunctionTok{require}\NormalTok{(tidyverse)\}}

\NormalTok{nosaukums}\OtherTok{=}\StringTok{"Climate\_CHELSAv2.1{-}clt{-}max\_cell.tif"}
\NormalTok{ielasisanas\_cels}\OtherTok{=}\FunctionTok{paste0}\NormalTok{(}\StringTok{"./RasterGrids\_100m/2024/RAW/"}\NormalTok{,nosaukums)}
\NormalTok{saglabasanas\_cels}\OtherTok{=}\FunctionTok{paste0}\NormalTok{(}\StringTok{"./RasterGrids\_100m/2024/Scaled/"}\NormalTok{,nosaukums)}
\NormalTok{slanis}\OtherTok{=}\FunctionTok{rast}\NormalTok{(ielasisanas\_cels)}
\NormalTok{videjais}\OtherTok{=}\FunctionTok{global}\NormalTok{(slanis,}\AttributeTok{fun=}\StringTok{"mean"}\NormalTok{,}\AttributeTok{na.rm=}\ConstantTok{TRUE}\NormalTok{)}
\NormalTok{centrets}\OtherTok{=}\NormalTok{slanis}\SpecialCharTok{{-}}\NormalTok{videjais[,}\DecValTok{1}\NormalTok{]}
\NormalTok{standartnovirze}\OtherTok{=}\NormalTok{terra}\SpecialCharTok{::}\FunctionTok{global}\NormalTok{(centrets,}\AttributeTok{fun=}\StringTok{"rms"}\NormalTok{,}\AttributeTok{na.rm=}\ConstantTok{TRUE}\NormalTok{)}
\NormalTok{merogots}\OtherTok{=}\NormalTok{centrets}\SpecialCharTok{/}\NormalTok{standartnovirze[,}\DecValTok{1}\NormalTok{]}
\FunctionTok{writeRaster}\NormalTok{(merogots,}
      \AttributeTok{filename=}\NormalTok{saglabasanas\_cels,}
      \AttributeTok{overwrite=}\ConstantTok{TRUE}\NormalTok{)}
\end{Highlighting}
\end{Shaded}

\section{Climate\_CHELSAv2.1-clt-mean\_cell}\label{ch06.021}

\textbf{filename:} \texttt{Climate\_CHELSAv2.1-clt-mean\_cell.tif}

\textbf{layername:} \texttt{egv\_021}

\textbf{English name:} Mean monthly mean cloud area fraction (\%) (CHELSA v2.1) within the
analysis cell (1 ha)

\textbf{Latvian name:} Vidējais ik mēneša vidējais mākoņu segums (\%) (CHELSA v2.1) analīzes šūnā (1 ha)

\textbf{Procedure:} Directly follows \hyperref[Ch04.11]{CHELSA v2.1}. EGV is prepared using
the workflow \texttt{egvtools::downscale2egv()} with inverse distance weighted (power =
2) gap filling and soft smoothing (power = 0.5) over 5 km radius around each cell.
Finally, the layer is standardised by subtracting the arithmetic mean and
dividing by the root mean squared error.

\begin{Shaded}
\begin{Highlighting}[]
\CommentTok{\# libs {-}{-}{-}{-}}
\ControlFlowTok{if}\NormalTok{(}\SpecialCharTok{!}\FunctionTok{require}\NormalTok{(egvtools)) \{remotes}\SpecialCharTok{::}\FunctionTok{install\_github}\NormalTok{(}\StringTok{"aavotins/egvtools"}\NormalTok{); }\FunctionTok{require}\NormalTok{(egvtools)\}}

\CommentTok{\# job {-}{-}{-}{-}}

\NormalTok{localname}\OtherTok{=}\StringTok{"Climate\_CHELSAv2.1{-}clt{-}mean\_cell.tif"}
\NormalTok{layername}\OtherTok{=}\StringTok{"egv\_021"}
\NormalTok{reading}\OtherTok{=}\StringTok{"./Geodata/2024/CHELSA/Climate\_CHELSAv2.1{-}clt{-}mean\_cell.tif"}

\NormalTok{df }\OtherTok{\textless{}{-}} \FunctionTok{downscale2egv}\NormalTok{(}
 \AttributeTok{template\_path =} \StringTok{"./Templates/TemplateRasters/LV100m\_10km.tif"}\NormalTok{,}
 \AttributeTok{grid\_path   =} \StringTok{"./Templates/TemplateGrids/tikls1km\_sauzeme.parquet"}\NormalTok{,}
 \AttributeTok{rawfile\_path =}\NormalTok{ reading,}
 \AttributeTok{out\_path   =} \StringTok{"./RasterGrids\_100m/2024/RAW/"}\NormalTok{,}
 \AttributeTok{file\_name   =}\NormalTok{ localname,}
 \AttributeTok{layer\_name  =}\NormalTok{ layername,}
 \AttributeTok{fill\_gaps   =} \ConstantTok{TRUE}\NormalTok{,}
 \AttributeTok{smooth    =} \ConstantTok{TRUE}\NormalTok{,}
 \AttributeTok{smooth\_radius\_km =} \DecValTok{5}\NormalTok{,}
 \AttributeTok{plot\_result  =} \ConstantTok{TRUE}\NormalTok{)}
\FunctionTok{print}\NormalTok{(df)}

\CommentTok{\# standardisation {-}{-}{-}{-}}
\ControlFlowTok{if}\NormalTok{(}\SpecialCharTok{!}\FunctionTok{require}\NormalTok{(terra)) \{}\FunctionTok{install.packages}\NormalTok{(}\StringTok{"terra"}\NormalTok{); }\FunctionTok{require}\NormalTok{(terra)\}}
\ControlFlowTok{if}\NormalTok{(}\SpecialCharTok{!}\FunctionTok{require}\NormalTok{(tidyverse)) \{}\FunctionTok{install.packages}\NormalTok{(}\StringTok{"tidyverse"}\NormalTok{); }\FunctionTok{require}\NormalTok{(tidyverse)\}}

\NormalTok{nosaukums}\OtherTok{=}\StringTok{"Climate\_CHELSAv2.1{-}clt{-}mean\_cell.tif"}
\NormalTok{ielasisanas\_cels}\OtherTok{=}\FunctionTok{paste0}\NormalTok{(}\StringTok{"./RasterGrids\_100m/2024/RAW/"}\NormalTok{,nosaukums)}
\NormalTok{saglabasanas\_cels}\OtherTok{=}\FunctionTok{paste0}\NormalTok{(}\StringTok{"./RasterGrids\_100m/2024/Scaled/"}\NormalTok{,nosaukums)}
\NormalTok{slanis}\OtherTok{=}\FunctionTok{rast}\NormalTok{(ielasisanas\_cels)}
\NormalTok{videjais}\OtherTok{=}\FunctionTok{global}\NormalTok{(slanis,}\AttributeTok{fun=}\StringTok{"mean"}\NormalTok{,}\AttributeTok{na.rm=}\ConstantTok{TRUE}\NormalTok{)}
\NormalTok{centrets}\OtherTok{=}\NormalTok{slanis}\SpecialCharTok{{-}}\NormalTok{videjais[,}\DecValTok{1}\NormalTok{]}
\NormalTok{standartnovirze}\OtherTok{=}\NormalTok{terra}\SpecialCharTok{::}\FunctionTok{global}\NormalTok{(centrets,}\AttributeTok{fun=}\StringTok{"rms"}\NormalTok{,}\AttributeTok{na.rm=}\ConstantTok{TRUE}\NormalTok{)}
\NormalTok{merogots}\OtherTok{=}\NormalTok{centrets}\SpecialCharTok{/}\NormalTok{standartnovirze[,}\DecValTok{1}\NormalTok{]}
\FunctionTok{writeRaster}\NormalTok{(merogots,}
      \AttributeTok{filename=}\NormalTok{saglabasanas\_cels,}
      \AttributeTok{overwrite=}\ConstantTok{TRUE}\NormalTok{)}
\end{Highlighting}
\end{Shaded}

\section{Climate\_CHELSAv2.1-clt-min\_cell}\label{ch06.022}

\textbf{filename:} \texttt{Climate\_CHELSAv2.1-clt-min\_cell.tif}

\textbf{layername:} \texttt{egv\_022}

\textbf{English name:} Mean of monthly minimum cloud area fraction (\%) (CHELSA v2.1) within
the analysis cell (1 ha)

\textbf{Latvian name:} Mēneša minimumu vidējais mākoņu segums (\%) (CHELSA v2.1)
analīzes šūnā (1 ha)

\textbf{Procedure:} Directly follows \hyperref[Ch04.11]{CHELSA v2.1}. EGV is prepared using
the workflow \texttt{egvtools::downscale2egv()} with inverse distance weighted (power =
2) gap filling and soft smoothing (power = 0.5) over 5 km radius around each cell.
Finally, the layer is standardised by subtracting the arithmetic mean and
dividing by the root mean squared error.

\begin{Shaded}
\begin{Highlighting}[]
\CommentTok{\# libs {-}{-}{-}{-}}
\ControlFlowTok{if}\NormalTok{(}\SpecialCharTok{!}\FunctionTok{require}\NormalTok{(egvtools)) \{remotes}\SpecialCharTok{::}\FunctionTok{install\_github}\NormalTok{(}\StringTok{"aavotins/egvtools"}\NormalTok{); }\FunctionTok{require}\NormalTok{(egvtools)\}}

\CommentTok{\# job {-}{-}{-}{-}}

\NormalTok{localname}\OtherTok{=}\StringTok{"Climate\_CHELSAv2.1{-}clt{-}min\_cell.tif"}
\NormalTok{layername}\OtherTok{=}\StringTok{"egv\_022"}
\NormalTok{reading}\OtherTok{=}\StringTok{"./Geodata/2024/CHELSA/Climate\_CHELSAv2.1{-}clt{-}min\_cell.tif"}

\NormalTok{df }\OtherTok{\textless{}{-}} \FunctionTok{downscale2egv}\NormalTok{(}
 \AttributeTok{template\_path =} \StringTok{"./Templates/TemplateRasters/LV100m\_10km.tif"}\NormalTok{,}
 \AttributeTok{grid\_path   =} \StringTok{"./Templates/TemplateGrids/tikls1km\_sauzeme.parquet"}\NormalTok{,}
 \AttributeTok{rawfile\_path =}\NormalTok{ reading,}
 \AttributeTok{out\_path   =} \StringTok{"./RasterGrids\_100m/2024/RAW/"}\NormalTok{,}
 \AttributeTok{file\_name   =}\NormalTok{ localname,}
 \AttributeTok{layer\_name  =}\NormalTok{ layername,}
 \AttributeTok{fill\_gaps   =} \ConstantTok{TRUE}\NormalTok{,}
 \AttributeTok{smooth    =} \ConstantTok{TRUE}\NormalTok{,}
 \AttributeTok{smooth\_radius\_km =} \DecValTok{5}\NormalTok{,}
 \AttributeTok{plot\_result  =} \ConstantTok{TRUE}\NormalTok{)}
\FunctionTok{print}\NormalTok{(df)}

\CommentTok{\# standardisation {-}{-}{-}{-}}
\ControlFlowTok{if}\NormalTok{(}\SpecialCharTok{!}\FunctionTok{require}\NormalTok{(terra)) \{}\FunctionTok{install.packages}\NormalTok{(}\StringTok{"terra"}\NormalTok{); }\FunctionTok{require}\NormalTok{(terra)\}}
\ControlFlowTok{if}\NormalTok{(}\SpecialCharTok{!}\FunctionTok{require}\NormalTok{(tidyverse)) \{}\FunctionTok{install.packages}\NormalTok{(}\StringTok{"tidyverse"}\NormalTok{); }\FunctionTok{require}\NormalTok{(tidyverse)\}}

\NormalTok{nosaukums}\OtherTok{=}\StringTok{"Climate\_CHELSAv2.1{-}clt{-}min\_cell.tif"}
\NormalTok{ielasisanas\_cels}\OtherTok{=}\FunctionTok{paste0}\NormalTok{(}\StringTok{"./RasterGrids\_100m/2024/RAW/"}\NormalTok{,nosaukums)}
\NormalTok{saglabasanas\_cels}\OtherTok{=}\FunctionTok{paste0}\NormalTok{(}\StringTok{"./RasterGrids\_100m/2024/Scaled/"}\NormalTok{,nosaukums)}
\NormalTok{slanis}\OtherTok{=}\FunctionTok{rast}\NormalTok{(ielasisanas\_cels)}
\NormalTok{videjais}\OtherTok{=}\FunctionTok{global}\NormalTok{(slanis,}\AttributeTok{fun=}\StringTok{"mean"}\NormalTok{,}\AttributeTok{na.rm=}\ConstantTok{TRUE}\NormalTok{)}
\NormalTok{centrets}\OtherTok{=}\NormalTok{slanis}\SpecialCharTok{{-}}\NormalTok{videjais[,}\DecValTok{1}\NormalTok{]}
\NormalTok{standartnovirze}\OtherTok{=}\NormalTok{terra}\SpecialCharTok{::}\FunctionTok{global}\NormalTok{(centrets,}\AttributeTok{fun=}\StringTok{"rms"}\NormalTok{,}\AttributeTok{na.rm=}\ConstantTok{TRUE}\NormalTok{)}
\NormalTok{merogots}\OtherTok{=}\NormalTok{centrets}\SpecialCharTok{/}\NormalTok{standartnovirze[,}\DecValTok{1}\NormalTok{]}
\FunctionTok{writeRaster}\NormalTok{(merogots,}
      \AttributeTok{filename=}\NormalTok{saglabasanas\_cels,}
      \AttributeTok{overwrite=}\ConstantTok{TRUE}\NormalTok{)}
\end{Highlighting}
\end{Shaded}

\section{Climate\_CHELSAv2.1-clt-range\_cell}\label{ch06.023}

\textbf{filename:} \texttt{Climate\_CHELSAv2.1-clt-range\_cell.tif}

\textbf{layername:} \texttt{egv\_023}

\textbf{English name:} Annual range of monthly cloud area fraction (\%) (CHELSA v2.1)
within the analysis cell (1 ha)

\textbf{Latvian name:} Gada mākoņu seguma amplitūda (\%) (CHELSA v2.1) analīzes šūnā
(1 ha)

\textbf{Procedure:} Directly follows \hyperref[Ch04.11]{CHELSA v2.1}. EGV is prepared using
the workflow \texttt{egvtools::downscale2egv()} with inverse distance weighted (power =
2) gap filling and soft smoothing (power = 0.5) over 5 km radius around each cell.
Finally, the layer is standardised by subtracting the arithmetic mean and
dividing by the root mean squared error.

\begin{Shaded}
\begin{Highlighting}[]
\CommentTok{\# libs {-}{-}{-}{-}}
\ControlFlowTok{if}\NormalTok{(}\SpecialCharTok{!}\FunctionTok{require}\NormalTok{(egvtools)) \{remotes}\SpecialCharTok{::}\FunctionTok{install\_github}\NormalTok{(}\StringTok{"aavotins/egvtools"}\NormalTok{); }\FunctionTok{require}\NormalTok{(egvtools)\}}

\CommentTok{\# job {-}{-}{-}{-}}

\NormalTok{localname}\OtherTok{=}\StringTok{"Climate\_CHELSAv2.1{-}clt{-}range\_cell.tif"}
\NormalTok{layername}\OtherTok{=}\StringTok{"egv\_023"}
\NormalTok{reading}\OtherTok{=}\StringTok{"./Geodata/2024/CHELSA/Climate\_CHELSAv2.1{-}clt{-}range\_cell.tif"}

\NormalTok{df }\OtherTok{\textless{}{-}} \FunctionTok{downscale2egv}\NormalTok{(}
 \AttributeTok{template\_path =} \StringTok{"./Templates/TemplateRasters/LV100m\_10km.tif"}\NormalTok{,}
 \AttributeTok{grid\_path   =} \StringTok{"./Templates/TemplateGrids/tikls1km\_sauzeme.parquet"}\NormalTok{,}
 \AttributeTok{rawfile\_path =}\NormalTok{ reading,}
 \AttributeTok{out\_path   =} \StringTok{"./RasterGrids\_100m/2024/RAW/"}\NormalTok{,}
 \AttributeTok{file\_name   =}\NormalTok{ localname,}
 \AttributeTok{layer\_name  =}\NormalTok{ layername,}
 \AttributeTok{fill\_gaps   =} \ConstantTok{TRUE}\NormalTok{,}
 \AttributeTok{smooth    =} \ConstantTok{TRUE}\NormalTok{,}
 \AttributeTok{smooth\_radius\_km =} \DecValTok{5}\NormalTok{,}
 \AttributeTok{plot\_result  =} \ConstantTok{TRUE}\NormalTok{)}
\FunctionTok{print}\NormalTok{(df)}

\CommentTok{\# standardisation {-}{-}{-}{-}}
\ControlFlowTok{if}\NormalTok{(}\SpecialCharTok{!}\FunctionTok{require}\NormalTok{(terra)) \{}\FunctionTok{install.packages}\NormalTok{(}\StringTok{"terra"}\NormalTok{); }\FunctionTok{require}\NormalTok{(terra)\}}
\ControlFlowTok{if}\NormalTok{(}\SpecialCharTok{!}\FunctionTok{require}\NormalTok{(tidyverse)) \{}\FunctionTok{install.packages}\NormalTok{(}\StringTok{"tidyverse"}\NormalTok{); }\FunctionTok{require}\NormalTok{(tidyverse)\}}

\NormalTok{nosaukums}\OtherTok{=}\StringTok{"Climate\_CHELSAv2.1{-}clt{-}range\_cell.tif"}
\NormalTok{ielasisanas\_cels}\OtherTok{=}\FunctionTok{paste0}\NormalTok{(}\StringTok{"./RasterGrids\_100m/2024/RAW/"}\NormalTok{,nosaukums)}
\NormalTok{saglabasanas\_cels}\OtherTok{=}\FunctionTok{paste0}\NormalTok{(}\StringTok{"./RasterGrids\_100m/2024/Scaled/"}\NormalTok{,nosaukums)}
\NormalTok{slanis}\OtherTok{=}\FunctionTok{rast}\NormalTok{(ielasisanas\_cels)}
\NormalTok{videjais}\OtherTok{=}\FunctionTok{global}\NormalTok{(slanis,}\AttributeTok{fun=}\StringTok{"mean"}\NormalTok{,}\AttributeTok{na.rm=}\ConstantTok{TRUE}\NormalTok{)}
\NormalTok{centrets}\OtherTok{=}\NormalTok{slanis}\SpecialCharTok{{-}}\NormalTok{videjais[,}\DecValTok{1}\NormalTok{]}
\NormalTok{standartnovirze}\OtherTok{=}\NormalTok{terra}\SpecialCharTok{::}\FunctionTok{global}\NormalTok{(centrets,}\AttributeTok{fun=}\StringTok{"rms"}\NormalTok{,}\AttributeTok{na.rm=}\ConstantTok{TRUE}\NormalTok{)}
\NormalTok{merogots}\OtherTok{=}\NormalTok{centrets}\SpecialCharTok{/}\NormalTok{standartnovirze[,}\DecValTok{1}\NormalTok{]}
\FunctionTok{writeRaster}\NormalTok{(merogots,}
      \AttributeTok{filename=}\NormalTok{saglabasanas\_cels,}
      \AttributeTok{overwrite=}\ConstantTok{TRUE}\NormalTok{)}
\end{Highlighting}
\end{Shaded}

\section{Climate\_CHELSAv2.1-cmi-max\_cell}\label{ch06.024}

\textbf{filename:} \texttt{Climate\_CHELSAv2.1-cmi-max\_cell.tif}

\textbf{layername:} \texttt{egv\_024}

\textbf{English name:} Mean of monthly maximum climate moisture index (kg m⁻² month⁻¹)
(CHELSA v2.1) within the analysis cell (1 ha)

\textbf{Latvian name:} Vidējais ik mēneša maksimālais klimata mitruma indekss (kg m⁻²
month⁻¹) (CHELSA v2.1) analīzes šūnā (1 ha)

\textbf{Procedure:} Directly follows \hyperref[Ch04.11]{CHELSA v2.1}. EGV is prepared using
the workflow \texttt{egvtools::downscale2egv()} with inverse distance weighted (power =
2) gap filling and soft smoothing (power = 0.5) over 5 km radius around each cell.
Finally, the layer is standardised by subtracting the arithmetic mean and
dividing by the root mean squared error.

\begin{Shaded}
\begin{Highlighting}[]
\CommentTok{\# libs {-}{-}{-}{-}}
\ControlFlowTok{if}\NormalTok{(}\SpecialCharTok{!}\FunctionTok{require}\NormalTok{(egvtools)) \{remotes}\SpecialCharTok{::}\FunctionTok{install\_github}\NormalTok{(}\StringTok{"aavotins/egvtools"}\NormalTok{); }\FunctionTok{require}\NormalTok{(egvtools)\}}

\CommentTok{\# job {-}{-}{-}{-}}

\NormalTok{localname}\OtherTok{=}\StringTok{"Climate\_CHELSAv2.1{-}cmi{-}max\_cell.tif"}
\NormalTok{layername}\OtherTok{=}\StringTok{"egv\_024"}
\NormalTok{reading}\OtherTok{=}\StringTok{"./Geodata/2024/CHELSA/Climate\_CHELSAv2.1{-}cmi{-}max\_cell.tif"}

\NormalTok{df }\OtherTok{\textless{}{-}} \FunctionTok{downscale2egv}\NormalTok{(}
 \AttributeTok{template\_path =} \StringTok{"./Templates/TemplateRasters/LV100m\_10km.tif"}\NormalTok{,}
 \AttributeTok{grid\_path   =} \StringTok{"./Templates/TemplateGrids/tikls1km\_sauzeme.parquet"}\NormalTok{,}
 \AttributeTok{rawfile\_path =}\NormalTok{ reading,}
 \AttributeTok{out\_path   =} \StringTok{"./RasterGrids\_100m/2024/RAW/"}\NormalTok{,}
 \AttributeTok{file\_name   =}\NormalTok{ localname,}
 \AttributeTok{layer\_name  =}\NormalTok{ layername,}
 \AttributeTok{fill\_gaps   =} \ConstantTok{TRUE}\NormalTok{,}
 \AttributeTok{smooth    =} \ConstantTok{TRUE}\NormalTok{,}
 \AttributeTok{smooth\_radius\_km =} \DecValTok{5}\NormalTok{,}
 \AttributeTok{plot\_result  =} \ConstantTok{TRUE}\NormalTok{)}
\FunctionTok{print}\NormalTok{(df)}

\CommentTok{\# standardisation {-}{-}{-}{-}}
\ControlFlowTok{if}\NormalTok{(}\SpecialCharTok{!}\FunctionTok{require}\NormalTok{(terra)) \{}\FunctionTok{install.packages}\NormalTok{(}\StringTok{"terra"}\NormalTok{); }\FunctionTok{require}\NormalTok{(terra)\}}
\ControlFlowTok{if}\NormalTok{(}\SpecialCharTok{!}\FunctionTok{require}\NormalTok{(tidyverse)) \{}\FunctionTok{install.packages}\NormalTok{(}\StringTok{"tidyverse"}\NormalTok{); }\FunctionTok{require}\NormalTok{(tidyverse)\}}

\NormalTok{nosaukums}\OtherTok{=}\StringTok{"Climate\_CHELSAv2.1{-}cmi{-}max\_cell.tif"}
\NormalTok{ielasisanas\_cels}\OtherTok{=}\FunctionTok{paste0}\NormalTok{(}\StringTok{"./RasterGrids\_100m/2024/RAW/"}\NormalTok{,nosaukums)}
\NormalTok{saglabasanas\_cels}\OtherTok{=}\FunctionTok{paste0}\NormalTok{(}\StringTok{"./RasterGrids\_100m/2024/Scaled/"}\NormalTok{,nosaukums)}
\NormalTok{slanis}\OtherTok{=}\FunctionTok{rast}\NormalTok{(ielasisanas\_cels)}
\NormalTok{videjais}\OtherTok{=}\FunctionTok{global}\NormalTok{(slanis,}\AttributeTok{fun=}\StringTok{"mean"}\NormalTok{,}\AttributeTok{na.rm=}\ConstantTok{TRUE}\NormalTok{)}
\NormalTok{centrets}\OtherTok{=}\NormalTok{slanis}\SpecialCharTok{{-}}\NormalTok{videjais[,}\DecValTok{1}\NormalTok{]}
\NormalTok{standartnovirze}\OtherTok{=}\NormalTok{terra}\SpecialCharTok{::}\FunctionTok{global}\NormalTok{(centrets,}\AttributeTok{fun=}\StringTok{"rms"}\NormalTok{,}\AttributeTok{na.rm=}\ConstantTok{TRUE}\NormalTok{)}
\NormalTok{merogots}\OtherTok{=}\NormalTok{centrets}\SpecialCharTok{/}\NormalTok{standartnovirze[,}\DecValTok{1}\NormalTok{]}
\FunctionTok{writeRaster}\NormalTok{(merogots,}
      \AttributeTok{filename=}\NormalTok{saglabasanas\_cels,}
      \AttributeTok{overwrite=}\ConstantTok{TRUE}\NormalTok{)}
\end{Highlighting}
\end{Shaded}

\section{Climate\_CHELSAv2.1-cmi-mean\_cell}\label{ch06.025}

\textbf{filename:} \texttt{Climate\_CHELSAv2.1-cmi-mean\_cell.tif}

\textbf{layername:} \texttt{egv\_025}

\textbf{English name:} Mean of monthly mean climate moisture index (kg m⁻² month⁻¹) (CHELSA
v2.1) within the analysis cell (1 ha)

\textbf{Latvian name:} Vidējais ik mēneša vidējais klimata mitruma indekss (kg m⁻² month⁻¹) (CHELSA
v2.1) analīzes šūnā (1 ha)

\textbf{Procedure:} Directly follows \hyperref[Ch04.11]{CHELSA v2.1}. EGV is prepared using
the workflow \texttt{egvtools::downscale2egv()} with inverse distance weighted (power =
2) gap filling and soft smoothing (power = 0.5) over 5 km radius around each cell.
Finally, the layer is standardised by subtracting the arithmetic mean and
dividing by the root mean squared error.

\begin{Shaded}
\begin{Highlighting}[]
\CommentTok{\# libs {-}{-}{-}{-}}
\ControlFlowTok{if}\NormalTok{(}\SpecialCharTok{!}\FunctionTok{require}\NormalTok{(egvtools)) \{remotes}\SpecialCharTok{::}\FunctionTok{install\_github}\NormalTok{(}\StringTok{"aavotins/egvtools"}\NormalTok{); }\FunctionTok{require}\NormalTok{(egvtools)\}}

\CommentTok{\# job {-}{-}{-}{-}}

\NormalTok{localname}\OtherTok{=}\StringTok{"Climate\_CHELSAv2.1{-}cmi{-}mean\_cell.tif"}
\NormalTok{layername}\OtherTok{=}\StringTok{"egv\_025"}
\NormalTok{reading}\OtherTok{=}\StringTok{"./Geodata/2024/CHELSA/Climate\_CHELSAv2.1{-}cmi{-}mean\_cell.tif"}

\NormalTok{df }\OtherTok{\textless{}{-}} \FunctionTok{downscale2egv}\NormalTok{(}
 \AttributeTok{template\_path =} \StringTok{"./Templates/TemplateRasters/LV100m\_10km.tif"}\NormalTok{,}
 \AttributeTok{grid\_path   =} \StringTok{"./Templates/TemplateGrids/tikls1km\_sauzeme.parquet"}\NormalTok{,}
 \AttributeTok{rawfile\_path =}\NormalTok{ reading,}
 \AttributeTok{out\_path   =} \StringTok{"./RasterGrids\_100m/2024/RAW/"}\NormalTok{,}
 \AttributeTok{file\_name   =}\NormalTok{ localname,}
 \AttributeTok{layer\_name  =}\NormalTok{ layername,}
 \AttributeTok{fill\_gaps   =} \ConstantTok{TRUE}\NormalTok{,}
 \AttributeTok{smooth    =} \ConstantTok{TRUE}\NormalTok{,}
 \AttributeTok{smooth\_radius\_km =} \DecValTok{5}\NormalTok{,}
 \AttributeTok{plot\_result  =} \ConstantTok{TRUE}\NormalTok{)}
\FunctionTok{print}\NormalTok{(df)}

\CommentTok{\# standardisation {-}{-}{-}{-}}
\ControlFlowTok{if}\NormalTok{(}\SpecialCharTok{!}\FunctionTok{require}\NormalTok{(terra)) \{}\FunctionTok{install.packages}\NormalTok{(}\StringTok{"terra"}\NormalTok{); }\FunctionTok{require}\NormalTok{(terra)\}}
\ControlFlowTok{if}\NormalTok{(}\SpecialCharTok{!}\FunctionTok{require}\NormalTok{(tidyverse)) \{}\FunctionTok{install.packages}\NormalTok{(}\StringTok{"tidyverse"}\NormalTok{); }\FunctionTok{require}\NormalTok{(tidyverse)\}}

\NormalTok{nosaukums}\OtherTok{=}\StringTok{"Climate\_CHELSAv2.1{-}cmi{-}mean\_cell.tif"}
\NormalTok{ielasisanas\_cels}\OtherTok{=}\FunctionTok{paste0}\NormalTok{(}\StringTok{"./RasterGrids\_100m/2024/RAW/"}\NormalTok{,nosaukums)}
\NormalTok{saglabasanas\_cels}\OtherTok{=}\FunctionTok{paste0}\NormalTok{(}\StringTok{"./RasterGrids\_100m/2024/Scaled/"}\NormalTok{,nosaukums)}
\NormalTok{slanis}\OtherTok{=}\FunctionTok{rast}\NormalTok{(ielasisanas\_cels)}
\NormalTok{videjais}\OtherTok{=}\FunctionTok{global}\NormalTok{(slanis,}\AttributeTok{fun=}\StringTok{"mean"}\NormalTok{,}\AttributeTok{na.rm=}\ConstantTok{TRUE}\NormalTok{)}
\NormalTok{centrets}\OtherTok{=}\NormalTok{slanis}\SpecialCharTok{{-}}\NormalTok{videjais[,}\DecValTok{1}\NormalTok{]}
\NormalTok{standartnovirze}\OtherTok{=}\NormalTok{terra}\SpecialCharTok{::}\FunctionTok{global}\NormalTok{(centrets,}\AttributeTok{fun=}\StringTok{"rms"}\NormalTok{,}\AttributeTok{na.rm=}\ConstantTok{TRUE}\NormalTok{)}
\NormalTok{merogots}\OtherTok{=}\NormalTok{centrets}\SpecialCharTok{/}\NormalTok{standartnovirze[,}\DecValTok{1}\NormalTok{]}
\FunctionTok{writeRaster}\NormalTok{(merogots,}
      \AttributeTok{filename=}\NormalTok{saglabasanas\_cels,}
      \AttributeTok{overwrite=}\ConstantTok{TRUE}\NormalTok{)}
\end{Highlighting}
\end{Shaded}

\section{Climate\_CHELSAv2.1-cmi-min\_cell}\label{ch06.026}

\textbf{filename:} \texttt{Climate\_CHELSAv2.1-cmi-min\_cell.tif}

\textbf{layername:} \texttt{egv\_026}

\textbf{English name:} Mean of monthly minimum climate moisture index (kg m⁻² month⁻¹)
(CHELSA v2.1) within the analysis cell (1 ha)

\textbf{Latvian name:} Vidējais ik mēneša minimālais klimata mitruma indekss (kg m⁻²
month⁻¹) (CHELSA v2.1) analīzes šūnā (1 ha)

\textbf{Procedure:} Directly follows \hyperref[Ch04.11]{CHELSA v2.1}. EGV is prepared using
the workflow \texttt{egvtools::downscale2egv()} with inverse distance weighted (power =
2) gap filling and soft smoothing (power = 0.5) over 5 km radius around each cell.
Finally, the layer is standardised by subtracting the arithmetic mean and
dividing by the root mean squared error.

\begin{Shaded}
\begin{Highlighting}[]
\CommentTok{\# libs {-}{-}{-}{-}}
\ControlFlowTok{if}\NormalTok{(}\SpecialCharTok{!}\FunctionTok{require}\NormalTok{(egvtools)) \{remotes}\SpecialCharTok{::}\FunctionTok{install\_github}\NormalTok{(}\StringTok{"aavotins/egvtools"}\NormalTok{); }\FunctionTok{require}\NormalTok{(egvtools)\}}

\CommentTok{\# job {-}{-}{-}{-}}

\NormalTok{localname}\OtherTok{=}\StringTok{"Climate\_CHELSAv2.1{-}cmi{-}min\_cell.tif"}
\NormalTok{layername}\OtherTok{=}\StringTok{"egv\_026"}
\NormalTok{reading}\OtherTok{=}\StringTok{"./Geodata/2024/CHELSA/Climate\_CHELSAv2.1{-}cmi{-}min\_cell.tif"}

\NormalTok{df }\OtherTok{\textless{}{-}} \FunctionTok{downscale2egv}\NormalTok{(}
 \AttributeTok{template\_path =} \StringTok{"./Templates/TemplateRasters/LV100m\_10km.tif"}\NormalTok{,}
 \AttributeTok{grid\_path   =} \StringTok{"./Templates/TemplateGrids/tikls1km\_sauzeme.parquet"}\NormalTok{,}
 \AttributeTok{rawfile\_path =}\NormalTok{ reading,}
 \AttributeTok{out\_path   =} \StringTok{"./RasterGrids\_100m/2024/RAW/"}\NormalTok{,}
 \AttributeTok{file\_name   =}\NormalTok{ localname,}
 \AttributeTok{layer\_name  =}\NormalTok{ layername,}
 \AttributeTok{fill\_gaps   =} \ConstantTok{TRUE}\NormalTok{,}
 \AttributeTok{smooth    =} \ConstantTok{TRUE}\NormalTok{,}
 \AttributeTok{smooth\_radius\_km =} \DecValTok{5}\NormalTok{,}
 \AttributeTok{plot\_result  =} \ConstantTok{TRUE}\NormalTok{)}
\FunctionTok{print}\NormalTok{(df)}

\CommentTok{\# standardisation {-}{-}{-}{-}}
\ControlFlowTok{if}\NormalTok{(}\SpecialCharTok{!}\FunctionTok{require}\NormalTok{(terra)) \{}\FunctionTok{install.packages}\NormalTok{(}\StringTok{"terra"}\NormalTok{); }\FunctionTok{require}\NormalTok{(terra)\}}
\ControlFlowTok{if}\NormalTok{(}\SpecialCharTok{!}\FunctionTok{require}\NormalTok{(tidyverse)) \{}\FunctionTok{install.packages}\NormalTok{(}\StringTok{"tidyverse"}\NormalTok{); }\FunctionTok{require}\NormalTok{(tidyverse)\}}

\NormalTok{nosaukums}\OtherTok{=}\StringTok{"Climate\_CHELSAv2.1{-}cmi{-}min\_cell.tif"}
\NormalTok{ielasisanas\_cels}\OtherTok{=}\FunctionTok{paste0}\NormalTok{(}\StringTok{"./RasterGrids\_100m/2024/RAW/"}\NormalTok{,nosaukums)}
\NormalTok{saglabasanas\_cels}\OtherTok{=}\FunctionTok{paste0}\NormalTok{(}\StringTok{"./RasterGrids\_100m/2024/Scaled/"}\NormalTok{,nosaukums)}
\NormalTok{slanis}\OtherTok{=}\FunctionTok{rast}\NormalTok{(ielasisanas\_cels)}
\NormalTok{videjais}\OtherTok{=}\FunctionTok{global}\NormalTok{(slanis,}\AttributeTok{fun=}\StringTok{"mean"}\NormalTok{,}\AttributeTok{na.rm=}\ConstantTok{TRUE}\NormalTok{)}
\NormalTok{centrets}\OtherTok{=}\NormalTok{slanis}\SpecialCharTok{{-}}\NormalTok{videjais[,}\DecValTok{1}\NormalTok{]}
\NormalTok{standartnovirze}\OtherTok{=}\NormalTok{terra}\SpecialCharTok{::}\FunctionTok{global}\NormalTok{(centrets,}\AttributeTok{fun=}\StringTok{"rms"}\NormalTok{,}\AttributeTok{na.rm=}\ConstantTok{TRUE}\NormalTok{)}
\NormalTok{merogots}\OtherTok{=}\NormalTok{centrets}\SpecialCharTok{/}\NormalTok{standartnovirze[,}\DecValTok{1}\NormalTok{]}
\FunctionTok{writeRaster}\NormalTok{(merogots,}
      \AttributeTok{filename=}\NormalTok{saglabasanas\_cels,}
      \AttributeTok{overwrite=}\ConstantTok{TRUE}\NormalTok{)}
\end{Highlighting}
\end{Shaded}

\section{Climate\_CHELSAv2.1-cmi-range\_cell}\label{ch06.027}

\textbf{filename:} \texttt{Climate\_CHELSAv2.1-cmi-range\_cell.tif}

\textbf{layername:} \texttt{egv\_027}

\textbf{English name:} Annual range of monthly climate moisture index (kg m⁻²
month⁻¹) (CHELSA v2.1) within the analysis cell (1 ha)

\textbf{Latvian name:} Gada klimata mitruma indeksa amplitūda (kg m⁻² month⁻¹)
(CHELSA v2.1) analīzes šūnā (1 ha)

\textbf{Procedure:} Directly follows \hyperref[Ch04.11]{CHELSA v2.1}. EGV is prepared using
the workflow \texttt{egvtools::downscale2egv()} with inverse distance weighted (power =
2) gap filling and soft smoothing (power = 0.5) over 5 km radius around each cell.
Finally, the layer is standardised by subtracting the arithmetic mean and
dividing by the root mean squared error.

\begin{Shaded}
\begin{Highlighting}[]
\CommentTok{\# libs {-}{-}{-}{-}}
\ControlFlowTok{if}\NormalTok{(}\SpecialCharTok{!}\FunctionTok{require}\NormalTok{(egvtools)) \{remotes}\SpecialCharTok{::}\FunctionTok{install\_github}\NormalTok{(}\StringTok{"aavotins/egvtools"}\NormalTok{); }\FunctionTok{require}\NormalTok{(egvtools)\}}

\CommentTok{\# job {-}{-}{-}{-}}

\NormalTok{localname}\OtherTok{=}\StringTok{"Climate\_CHELSAv2.1{-}cmi{-}range\_cell.tif"}
\NormalTok{layername}\OtherTok{=}\StringTok{"egv\_027"}
\NormalTok{reading}\OtherTok{=}\StringTok{"./Geodata/2024/CHELSA/Climate\_CHELSAv2.1{-}cmi{-}range\_cell.tif"}

\NormalTok{df }\OtherTok{\textless{}{-}} \FunctionTok{downscale2egv}\NormalTok{(}
 \AttributeTok{template\_path =} \StringTok{"./Templates/TemplateRasters/LV100m\_10km.tif"}\NormalTok{,}
 \AttributeTok{grid\_path   =} \StringTok{"./Templates/TemplateGrids/tikls1km\_sauzeme.parquet"}\NormalTok{,}
 \AttributeTok{rawfile\_path =}\NormalTok{ reading,}
 \AttributeTok{out\_path   =} \StringTok{"./RasterGrids\_100m/2024/RAW/"}\NormalTok{,}
 \AttributeTok{file\_name   =}\NormalTok{ localname,}
 \AttributeTok{layer\_name  =}\NormalTok{ layername,}
 \AttributeTok{fill\_gaps   =} \ConstantTok{TRUE}\NormalTok{,}
 \AttributeTok{smooth    =} \ConstantTok{TRUE}\NormalTok{,}
 \AttributeTok{smooth\_radius\_km =} \DecValTok{5}\NormalTok{,}
 \AttributeTok{plot\_result  =} \ConstantTok{TRUE}\NormalTok{)}
\FunctionTok{print}\NormalTok{(df)}

\CommentTok{\# standardisation {-}{-}{-}{-}}
\ControlFlowTok{if}\NormalTok{(}\SpecialCharTok{!}\FunctionTok{require}\NormalTok{(terra)) \{}\FunctionTok{install.packages}\NormalTok{(}\StringTok{"terra"}\NormalTok{); }\FunctionTok{require}\NormalTok{(terra)\}}
\ControlFlowTok{if}\NormalTok{(}\SpecialCharTok{!}\FunctionTok{require}\NormalTok{(tidyverse)) \{}\FunctionTok{install.packages}\NormalTok{(}\StringTok{"tidyverse"}\NormalTok{); }\FunctionTok{require}\NormalTok{(tidyverse)\}}

\NormalTok{nosaukums}\OtherTok{=}\StringTok{"Climate\_CHELSAv2.1{-}cmi{-}range\_cell.tif"}
\NormalTok{ielasisanas\_cels}\OtherTok{=}\FunctionTok{paste0}\NormalTok{(}\StringTok{"./RasterGrids\_100m/2024/RAW/"}\NormalTok{,nosaukums)}
\NormalTok{saglabasanas\_cels}\OtherTok{=}\FunctionTok{paste0}\NormalTok{(}\StringTok{"./RasterGrids\_100m/2024/Scaled/"}\NormalTok{,nosaukums)}
\NormalTok{slanis}\OtherTok{=}\FunctionTok{rast}\NormalTok{(ielasisanas\_cels)}
\NormalTok{videjais}\OtherTok{=}\FunctionTok{global}\NormalTok{(slanis,}\AttributeTok{fun=}\StringTok{"mean"}\NormalTok{,}\AttributeTok{na.rm=}\ConstantTok{TRUE}\NormalTok{)}
\NormalTok{centrets}\OtherTok{=}\NormalTok{slanis}\SpecialCharTok{{-}}\NormalTok{videjais[,}\DecValTok{1}\NormalTok{]}
\NormalTok{standartnovirze}\OtherTok{=}\NormalTok{terra}\SpecialCharTok{::}\FunctionTok{global}\NormalTok{(centrets,}\AttributeTok{fun=}\StringTok{"rms"}\NormalTok{,}\AttributeTok{na.rm=}\ConstantTok{TRUE}\NormalTok{)}
\NormalTok{merogots}\OtherTok{=}\NormalTok{centrets}\SpecialCharTok{/}\NormalTok{standartnovirze[,}\DecValTok{1}\NormalTok{]}
\FunctionTok{writeRaster}\NormalTok{(merogots,}
      \AttributeTok{filename=}\NormalTok{saglabasanas\_cels,}
      \AttributeTok{overwrite=}\ConstantTok{TRUE}\NormalTok{)}
\end{Highlighting}
\end{Shaded}

\section{Climate\_CHELSAv2.1-fcf\_cell}\label{ch06.028}

\textbf{filename:} \texttt{Climate\_CHELSAv2.1-fcf\_cell.tif}

\textbf{layername:} \texttt{egv\_028}

\textbf{English name:} Frost change frequency (number of events in which tmin or tmax
go above or below 0°C) (CHELSA v2.1) within the analysis cell (1 ha)

\textbf{Latvian name:} Sasalšanas gadījumu biežums (zemākā vai augstākā temperatūra
šķērso 0°C) (CHELSA v2.1) analīzes šūnā (1 ha)

\textbf{Procedure:} Directly follows \hyperref[Ch04.11]{CHELSA v2.1}. EGV is prepared using
the workflow \texttt{egvtools::downscale2egv()} with inverse distance weighted (power =
2) gap filling and soft smoothing (power = 0.5) over 5 km radius around each cell.
Finally, the layer is standardised by subtracting the arithmetic mean and
dividing by the root mean squared error.

\begin{Shaded}
\begin{Highlighting}[]
\CommentTok{\# libs {-}{-}{-}{-}}
\ControlFlowTok{if}\NormalTok{(}\SpecialCharTok{!}\FunctionTok{require}\NormalTok{(egvtools)) \{remotes}\SpecialCharTok{::}\FunctionTok{install\_github}\NormalTok{(}\StringTok{"aavotins/egvtools"}\NormalTok{); }\FunctionTok{require}\NormalTok{(egvtools)\}}

\CommentTok{\# job {-}{-}{-}{-}}

\NormalTok{localname}\OtherTok{=}\StringTok{"Climate\_CHELSAv2.1{-}fcf\_cell.tif"}
\NormalTok{layername}\OtherTok{=}\StringTok{"egv\_028"}
\NormalTok{reading}\OtherTok{=}\StringTok{"./Geodata/2024/CHELSA/Climate\_CHELSAv2.1{-}fcf\_cell.tif"}

\NormalTok{df }\OtherTok{\textless{}{-}} \FunctionTok{downscale2egv}\NormalTok{(}
 \AttributeTok{template\_path =} \StringTok{"./Templates/TemplateRasters/LV100m\_10km.tif"}\NormalTok{,}
 \AttributeTok{grid\_path   =} \StringTok{"./Templates/TemplateGrids/tikls1km\_sauzeme.parquet"}\NormalTok{,}
 \AttributeTok{rawfile\_path =}\NormalTok{ reading,}
 \AttributeTok{out\_path   =} \StringTok{"./RasterGrids\_100m/2024/RAW/"}\NormalTok{,}
 \AttributeTok{file\_name   =}\NormalTok{ localname,}
 \AttributeTok{layer\_name  =}\NormalTok{ layername,}
 \AttributeTok{fill\_gaps   =} \ConstantTok{TRUE}\NormalTok{,}
 \AttributeTok{smooth    =} \ConstantTok{TRUE}\NormalTok{,}
 \AttributeTok{smooth\_radius\_km =} \DecValTok{5}\NormalTok{,}
 \AttributeTok{plot\_result  =} \ConstantTok{TRUE}\NormalTok{)}
\FunctionTok{print}\NormalTok{(df)}

\CommentTok{\# standardisation {-}{-}{-}{-}}
\ControlFlowTok{if}\NormalTok{(}\SpecialCharTok{!}\FunctionTok{require}\NormalTok{(terra)) \{}\FunctionTok{install.packages}\NormalTok{(}\StringTok{"terra"}\NormalTok{); }\FunctionTok{require}\NormalTok{(terra)\}}
\ControlFlowTok{if}\NormalTok{(}\SpecialCharTok{!}\FunctionTok{require}\NormalTok{(tidyverse)) \{}\FunctionTok{install.packages}\NormalTok{(}\StringTok{"tidyverse"}\NormalTok{); }\FunctionTok{require}\NormalTok{(tidyverse)\}}

\NormalTok{nosaukums}\OtherTok{=}\StringTok{"Climate\_CHELSAv2.1{-}fcf\_cell.tif"}
\NormalTok{ielasisanas\_cels}\OtherTok{=}\FunctionTok{paste0}\NormalTok{(}\StringTok{"./RasterGrids\_100m/2024/RAW/"}\NormalTok{,nosaukums)}
\NormalTok{saglabasanas\_cels}\OtherTok{=}\FunctionTok{paste0}\NormalTok{(}\StringTok{"./RasterGrids\_100m/2024/Scaled/"}\NormalTok{,nosaukums)}
\NormalTok{slanis}\OtherTok{=}\FunctionTok{rast}\NormalTok{(ielasisanas\_cels)}
\NormalTok{videjais}\OtherTok{=}\FunctionTok{global}\NormalTok{(slanis,}\AttributeTok{fun=}\StringTok{"mean"}\NormalTok{,}\AttributeTok{na.rm=}\ConstantTok{TRUE}\NormalTok{)}
\NormalTok{centrets}\OtherTok{=}\NormalTok{slanis}\SpecialCharTok{{-}}\NormalTok{videjais[,}\DecValTok{1}\NormalTok{]}
\NormalTok{standartnovirze}\OtherTok{=}\NormalTok{terra}\SpecialCharTok{::}\FunctionTok{global}\NormalTok{(centrets,}\AttributeTok{fun=}\StringTok{"rms"}\NormalTok{,}\AttributeTok{na.rm=}\ConstantTok{TRUE}\NormalTok{)}
\NormalTok{merogots}\OtherTok{=}\NormalTok{centrets}\SpecialCharTok{/}\NormalTok{standartnovirze[,}\DecValTok{1}\NormalTok{]}
\FunctionTok{writeRaster}\NormalTok{(merogots,}
      \AttributeTok{filename=}\NormalTok{saglabasanas\_cels,}
      \AttributeTok{overwrite=}\ConstantTok{TRUE}\NormalTok{)}
\end{Highlighting}
\end{Shaded}

\section{Climate\_CHELSAv2.1-fgd\_cell}\label{ch06.029}

\textbf{filename:} \texttt{Climate\_CHELSAv2.1-fgd\_cell.tif}

\textbf{layername:} \texttt{egv\_029}

\textbf{English name:} First day of the growing season (TREELIM) (CHELSA v2.1) within
the analysis cell (1 ha)

\textbf{Latvian name:} Veģetācijas sezonas pirmā diena (TREELIM) (CHELSA v2.1)
analīzes šūnā (1 ha)

\textbf{Procedure:} Directly follows \hyperref[Ch04.11]{CHELSA v2.1}. EGV is prepared using
the workflow \texttt{egvtools::downscale2egv()} with inverse distance weighted (power =
2) gap filling and soft smoothing (power = 0.5) over 5 km radius around each cell.
Finally, the layer is standardised by subtracting the arithmetic mean and
dividing by the root mean squared error.

\begin{Shaded}
\begin{Highlighting}[]
\CommentTok{\# libs {-}{-}{-}{-}}
\ControlFlowTok{if}\NormalTok{(}\SpecialCharTok{!}\FunctionTok{require}\NormalTok{(egvtools)) \{remotes}\SpecialCharTok{::}\FunctionTok{install\_github}\NormalTok{(}\StringTok{"aavotins/egvtools"}\NormalTok{); }\FunctionTok{require}\NormalTok{(egvtools)\}}

\CommentTok{\# job {-}{-}{-}{-}}

\NormalTok{localname}\OtherTok{=}\StringTok{"Climate\_CHELSAv2.1{-}fgd\_cell.tif"}
\NormalTok{layername}\OtherTok{=}\StringTok{"egv\_029"}
\NormalTok{reading}\OtherTok{=}\StringTok{"./Geodata/2024/CHELSA/Climate\_CHELSAv2.1{-}fgd\_cell.tif"}

\NormalTok{df }\OtherTok{\textless{}{-}} \FunctionTok{downscale2egv}\NormalTok{(}
 \AttributeTok{template\_path =} \StringTok{"./Templates/TemplateRasters/LV100m\_10km.tif"}\NormalTok{,}
 \AttributeTok{grid\_path   =} \StringTok{"./Templates/TemplateGrids/tikls1km\_sauzeme.parquet"}\NormalTok{,}
 \AttributeTok{rawfile\_path =}\NormalTok{ reading,}
 \AttributeTok{out\_path   =} \StringTok{"./RasterGrids\_100m/2024/RAW/"}\NormalTok{,}
 \AttributeTok{file\_name   =}\NormalTok{ localname,}
 \AttributeTok{layer\_name  =}\NormalTok{ layername,}
 \AttributeTok{fill\_gaps   =} \ConstantTok{TRUE}\NormalTok{,}
 \AttributeTok{smooth    =} \ConstantTok{TRUE}\NormalTok{,}
 \AttributeTok{smooth\_radius\_km =} \DecValTok{5}\NormalTok{,}
 \AttributeTok{plot\_result  =} \ConstantTok{TRUE}\NormalTok{)}
\FunctionTok{print}\NormalTok{(df)}

\CommentTok{\# standardisation {-}{-}{-}{-}}
\ControlFlowTok{if}\NormalTok{(}\SpecialCharTok{!}\FunctionTok{require}\NormalTok{(terra)) \{}\FunctionTok{install.packages}\NormalTok{(}\StringTok{"terra"}\NormalTok{); }\FunctionTok{require}\NormalTok{(terra)\}}
\ControlFlowTok{if}\NormalTok{(}\SpecialCharTok{!}\FunctionTok{require}\NormalTok{(tidyverse)) \{}\FunctionTok{install.packages}\NormalTok{(}\StringTok{"tidyverse"}\NormalTok{); }\FunctionTok{require}\NormalTok{(tidyverse)\}}

\NormalTok{nosaukums}\OtherTok{=}\StringTok{"Climate\_CHELSAv2.1{-}fgd\_cell.tif"}
\NormalTok{ielasisanas\_cels}\OtherTok{=}\FunctionTok{paste0}\NormalTok{(}\StringTok{"./RasterGrids\_100m/2024/RAW/"}\NormalTok{,nosaukums)}
\NormalTok{saglabasanas\_cels}\OtherTok{=}\FunctionTok{paste0}\NormalTok{(}\StringTok{"./RasterGrids\_100m/2024/Scaled/"}\NormalTok{,nosaukums)}
\NormalTok{slanis}\OtherTok{=}\FunctionTok{rast}\NormalTok{(ielasisanas\_cels)}
\NormalTok{videjais}\OtherTok{=}\FunctionTok{global}\NormalTok{(slanis,}\AttributeTok{fun=}\StringTok{"mean"}\NormalTok{,}\AttributeTok{na.rm=}\ConstantTok{TRUE}\NormalTok{)}
\NormalTok{centrets}\OtherTok{=}\NormalTok{slanis}\SpecialCharTok{{-}}\NormalTok{videjais[,}\DecValTok{1}\NormalTok{]}
\NormalTok{standartnovirze}\OtherTok{=}\NormalTok{terra}\SpecialCharTok{::}\FunctionTok{global}\NormalTok{(centrets,}\AttributeTok{fun=}\StringTok{"rms"}\NormalTok{,}\AttributeTok{na.rm=}\ConstantTok{TRUE}\NormalTok{)}
\NormalTok{merogots}\OtherTok{=}\NormalTok{centrets}\SpecialCharTok{/}\NormalTok{standartnovirze[,}\DecValTok{1}\NormalTok{]}
\FunctionTok{writeRaster}\NormalTok{(merogots,}
      \AttributeTok{filename=}\NormalTok{saglabasanas\_cels,}
      \AttributeTok{overwrite=}\ConstantTok{TRUE}\NormalTok{)}
\end{Highlighting}
\end{Shaded}

\section{Climate\_CHELSAv2.1-gdd0\_cell}\label{ch06.030}

\textbf{filename:} \texttt{Climate\_CHELSAv2.1-gdd0\_cell.tif}

\textbf{layername:} \texttt{egv\_030}

\textbf{English name:} Growing degree days temerature sum above 0°C (CHELSA v2.1) within
the analysis cell (1 ha)

\textbf{Latvian name:} Aktīvo temperatūru summa no 0°C (CHELSA v2.1) analīzes šūnā (1
ha)

\textbf{Procedure:} Directly follows \hyperref[Ch04.11]{CHELSA v2.1}. EGV is prepared using
the workflow \texttt{egvtools::downscale2egv()} with inverse distance weighted (power =
2) gap filling and soft smoothing (power = 0.5) over 5 km radius around each cell.
Finally, the layer is standardised by subtracting the arithmetic mean and
dividing by the root mean squared error.

\begin{Shaded}
\begin{Highlighting}[]
\CommentTok{\# libs {-}{-}{-}{-}}
\ControlFlowTok{if}\NormalTok{(}\SpecialCharTok{!}\FunctionTok{require}\NormalTok{(egvtools)) \{remotes}\SpecialCharTok{::}\FunctionTok{install\_github}\NormalTok{(}\StringTok{"aavotins/egvtools"}\NormalTok{); }\FunctionTok{require}\NormalTok{(egvtools)\}}

\CommentTok{\# job {-}{-}{-}{-}}

\NormalTok{localname}\OtherTok{=}\StringTok{"Climate\_CHELSAv2.1{-}gdd0\_cell.tif"}
\NormalTok{layername}\OtherTok{=}\StringTok{"egv\_030"}
\NormalTok{reading}\OtherTok{=}\StringTok{"./Geodata/2024/CHELSA/Climate\_CHELSAv2.1{-}gdd0\_cell.tif"}

\NormalTok{df }\OtherTok{\textless{}{-}} \FunctionTok{downscale2egv}\NormalTok{(}
 \AttributeTok{template\_path =} \StringTok{"./Templates/TemplateRasters/LV100m\_10km.tif"}\NormalTok{,}
 \AttributeTok{grid\_path   =} \StringTok{"./Templates/TemplateGrids/tikls1km\_sauzeme.parquet"}\NormalTok{,}
 \AttributeTok{rawfile\_path =}\NormalTok{ reading,}
 \AttributeTok{out\_path   =} \StringTok{"./RasterGrids\_100m/2024/RAW/"}\NormalTok{,}
 \AttributeTok{file\_name   =}\NormalTok{ localname,}
 \AttributeTok{layer\_name  =}\NormalTok{ layername,}
 \AttributeTok{fill\_gaps   =} \ConstantTok{TRUE}\NormalTok{,}
 \AttributeTok{smooth    =} \ConstantTok{TRUE}\NormalTok{,}
 \AttributeTok{smooth\_radius\_km =} \DecValTok{5}\NormalTok{,}
 \AttributeTok{plot\_result  =} \ConstantTok{TRUE}\NormalTok{)}
\FunctionTok{print}\NormalTok{(df)}

\CommentTok{\# standardisation {-}{-}{-}{-}}
\ControlFlowTok{if}\NormalTok{(}\SpecialCharTok{!}\FunctionTok{require}\NormalTok{(terra)) \{}\FunctionTok{install.packages}\NormalTok{(}\StringTok{"terra"}\NormalTok{); }\FunctionTok{require}\NormalTok{(terra)\}}
\ControlFlowTok{if}\NormalTok{(}\SpecialCharTok{!}\FunctionTok{require}\NormalTok{(tidyverse)) \{}\FunctionTok{install.packages}\NormalTok{(}\StringTok{"tidyverse"}\NormalTok{); }\FunctionTok{require}\NormalTok{(tidyverse)\}}

\NormalTok{nosaukums}\OtherTok{=}\StringTok{"Climate\_CHELSAv2.1{-}gdd0\_cell.tif"}
\NormalTok{ielasisanas\_cels}\OtherTok{=}\FunctionTok{paste0}\NormalTok{(}\StringTok{"./RasterGrids\_100m/2024/RAW/"}\NormalTok{,nosaukums)}
\NormalTok{saglabasanas\_cels}\OtherTok{=}\FunctionTok{paste0}\NormalTok{(}\StringTok{"./RasterGrids\_100m/2024/Scaled/"}\NormalTok{,nosaukums)}
\NormalTok{slanis}\OtherTok{=}\FunctionTok{rast}\NormalTok{(ielasisanas\_cels)}
\NormalTok{videjais}\OtherTok{=}\FunctionTok{global}\NormalTok{(slanis,}\AttributeTok{fun=}\StringTok{"mean"}\NormalTok{,}\AttributeTok{na.rm=}\ConstantTok{TRUE}\NormalTok{)}
\NormalTok{centrets}\OtherTok{=}\NormalTok{slanis}\SpecialCharTok{{-}}\NormalTok{videjais[,}\DecValTok{1}\NormalTok{]}
\NormalTok{standartnovirze}\OtherTok{=}\NormalTok{terra}\SpecialCharTok{::}\FunctionTok{global}\NormalTok{(centrets,}\AttributeTok{fun=}\StringTok{"rms"}\NormalTok{,}\AttributeTok{na.rm=}\ConstantTok{TRUE}\NormalTok{)}
\NormalTok{merogots}\OtherTok{=}\NormalTok{centrets}\SpecialCharTok{/}\NormalTok{standartnovirze[,}\DecValTok{1}\NormalTok{]}
\FunctionTok{writeRaster}\NormalTok{(merogots,}
      \AttributeTok{filename=}\NormalTok{saglabasanas\_cels,}
      \AttributeTok{overwrite=}\ConstantTok{TRUE}\NormalTok{)}
\end{Highlighting}
\end{Shaded}

\section{Climate\_CHELSAv2.1-gdd10\_cell}\label{ch06.031}

\textbf{filename:} \texttt{Climate\_CHELSAv2.1-gdd10\_cell.tif}

\textbf{layername:} \texttt{egv\_031}

\textbf{English name:} Growing degree days temerature sum above 10°C (CHELSA v2.1) within
the analysis cell (1 ha)

\textbf{Latvian name:} Aktīvo temperatūru summa no 10°C (CHELSA v2.1) analīzes šūnā
(1 ha)

\textbf{Procedure:} Directly follows \hyperref[Ch04.11]{CHELSA v2.1}. EGV is prepared using
the workflow \texttt{egvtools::downscale2egv()} with inverse distance weighted (power =
2) gap filling and soft smoothing (power = 0.5) over 5 km radius around each cell.
Finally, the layer is standardised by subtracting the arithmetic mean and
dividing by the root mean squared error.

\begin{Shaded}
\begin{Highlighting}[]
\CommentTok{\# libs {-}{-}{-}{-}}
\ControlFlowTok{if}\NormalTok{(}\SpecialCharTok{!}\FunctionTok{require}\NormalTok{(egvtools)) \{remotes}\SpecialCharTok{::}\FunctionTok{install\_github}\NormalTok{(}\StringTok{"aavotins/egvtools"}\NormalTok{); }\FunctionTok{require}\NormalTok{(egvtools)\}}

\CommentTok{\# job {-}{-}{-}{-}}

\NormalTok{localname}\OtherTok{=}\StringTok{"Climate\_CHELSAv2.1{-}gdd10\_cell.tif"}
\NormalTok{layername}\OtherTok{=}\StringTok{"egv\_031"}
\NormalTok{reading}\OtherTok{=}\StringTok{"./Geodata/2024/CHELSA/Climate\_CHELSAv2.1{-}gdd10\_cell.tif"}

\NormalTok{df }\OtherTok{\textless{}{-}} \FunctionTok{downscale2egv}\NormalTok{(}
 \AttributeTok{template\_path =} \StringTok{"./Templates/TemplateRasters/LV100m\_10km.tif"}\NormalTok{,}
 \AttributeTok{grid\_path   =} \StringTok{"./Templates/TemplateGrids/tikls1km\_sauzeme.parquet"}\NormalTok{,}
 \AttributeTok{rawfile\_path =}\NormalTok{ reading,}
 \AttributeTok{out\_path   =} \StringTok{"./RasterGrids\_100m/2024/RAW/"}\NormalTok{,}
 \AttributeTok{file\_name   =}\NormalTok{ localname,}
 \AttributeTok{layer\_name  =}\NormalTok{ layername,}
 \AttributeTok{fill\_gaps   =} \ConstantTok{TRUE}\NormalTok{,}
 \AttributeTok{smooth    =} \ConstantTok{TRUE}\NormalTok{,}
 \AttributeTok{smooth\_radius\_km =} \DecValTok{5}\NormalTok{,}
 \AttributeTok{plot\_result  =} \ConstantTok{TRUE}\NormalTok{)}
\FunctionTok{print}\NormalTok{(df)}

\CommentTok{\# standardisation {-}{-}{-}{-}}
\ControlFlowTok{if}\NormalTok{(}\SpecialCharTok{!}\FunctionTok{require}\NormalTok{(terra)) \{}\FunctionTok{install.packages}\NormalTok{(}\StringTok{"terra"}\NormalTok{); }\FunctionTok{require}\NormalTok{(terra)\}}
\ControlFlowTok{if}\NormalTok{(}\SpecialCharTok{!}\FunctionTok{require}\NormalTok{(tidyverse)) \{}\FunctionTok{install.packages}\NormalTok{(}\StringTok{"tidyverse"}\NormalTok{); }\FunctionTok{require}\NormalTok{(tidyverse)\}}

\NormalTok{nosaukums}\OtherTok{=}\StringTok{"Climate\_CHELSAv2.1{-}gdd10\_cell.tif"}
\NormalTok{ielasisanas\_cels}\OtherTok{=}\FunctionTok{paste0}\NormalTok{(}\StringTok{"./RasterGrids\_100m/2024/RAW/"}\NormalTok{,nosaukums)}
\NormalTok{saglabasanas\_cels}\OtherTok{=}\FunctionTok{paste0}\NormalTok{(}\StringTok{"./RasterGrids\_100m/2024/Scaled/"}\NormalTok{,nosaukums)}
\NormalTok{slanis}\OtherTok{=}\FunctionTok{rast}\NormalTok{(ielasisanas\_cels)}
\NormalTok{videjais}\OtherTok{=}\FunctionTok{global}\NormalTok{(slanis,}\AttributeTok{fun=}\StringTok{"mean"}\NormalTok{,}\AttributeTok{na.rm=}\ConstantTok{TRUE}\NormalTok{)}
\NormalTok{centrets}\OtherTok{=}\NormalTok{slanis}\SpecialCharTok{{-}}\NormalTok{videjais[,}\DecValTok{1}\NormalTok{]}
\NormalTok{standartnovirze}\OtherTok{=}\NormalTok{terra}\SpecialCharTok{::}\FunctionTok{global}\NormalTok{(centrets,}\AttributeTok{fun=}\StringTok{"rms"}\NormalTok{,}\AttributeTok{na.rm=}\ConstantTok{TRUE}\NormalTok{)}
\NormalTok{merogots}\OtherTok{=}\NormalTok{centrets}\SpecialCharTok{/}\NormalTok{standartnovirze[,}\DecValTok{1}\NormalTok{]}
\FunctionTok{writeRaster}\NormalTok{(merogots,}
      \AttributeTok{filename=}\NormalTok{saglabasanas\_cels,}
      \AttributeTok{overwrite=}\ConstantTok{TRUE}\NormalTok{)}
\end{Highlighting}
\end{Shaded}

\section{Climate\_CHELSAv2.1-gdd5\_cell}\label{ch06.032}

\textbf{filename:} \texttt{Climate\_CHELSAv2.1-gdd5\_cell.tif}

\textbf{layername:} \texttt{egv\_032}

\textbf{English name:} Growing degree days temerature sum above 5°C (CHELSA v2.1) within
the analysis cell (1 ha)

\textbf{Latvian name:} Aktīvo temperatūru summa no 5°C (CHELSA v2.1) analīzes šūnā (1
ha)

\textbf{Procedure:} Directly follows \hyperref[Ch04.11]{CHELSA v2.1}. EGV is prepared using
the workflow \texttt{egvtools::downscale2egv()} with inverse distance weighted (power =
2) gap filling and soft smoothing (power = 0.5) over 5 km radius around each cell.
Finally, the layer is standardised by subtracting the arithmetic mean and
dividing by the root mean squared error.

\begin{Shaded}
\begin{Highlighting}[]
\CommentTok{\# libs {-}{-}{-}{-}}
\ControlFlowTok{if}\NormalTok{(}\SpecialCharTok{!}\FunctionTok{require}\NormalTok{(egvtools)) \{remotes}\SpecialCharTok{::}\FunctionTok{install\_github}\NormalTok{(}\StringTok{"aavotins/egvtools"}\NormalTok{); }\FunctionTok{require}\NormalTok{(egvtools)\}}

\CommentTok{\# job {-}{-}{-}{-}}

\NormalTok{localname}\OtherTok{=}\StringTok{"Climate\_CHELSAv2.1{-}gdd5\_cell.tif"}
\NormalTok{layername}\OtherTok{=}\StringTok{"egv\_032"}
\NormalTok{reading}\OtherTok{=}\StringTok{"./Geodata/2024/CHELSA/Climate\_CHELSAv2.1{-}gdd5\_cell.tif"}

\NormalTok{df }\OtherTok{\textless{}{-}} \FunctionTok{downscale2egv}\NormalTok{(}
 \AttributeTok{template\_path =} \StringTok{"./Templates/TemplateRasters/LV100m\_10km.tif"}\NormalTok{,}
 \AttributeTok{grid\_path   =} \StringTok{"./Templates/TemplateGrids/tikls1km\_sauzeme.parquet"}\NormalTok{,}
 \AttributeTok{rawfile\_path =}\NormalTok{ reading,}
 \AttributeTok{out\_path   =} \StringTok{"./RasterGrids\_100m/2024/RAW/"}\NormalTok{,}
 \AttributeTok{file\_name   =}\NormalTok{ localname,}
 \AttributeTok{layer\_name  =}\NormalTok{ layername,}
 \AttributeTok{fill\_gaps   =} \ConstantTok{TRUE}\NormalTok{,}
 \AttributeTok{smooth    =} \ConstantTok{TRUE}\NormalTok{,}
 \AttributeTok{smooth\_radius\_km =} \DecValTok{5}\NormalTok{,}
 \AttributeTok{plot\_result  =} \ConstantTok{TRUE}\NormalTok{)}
\FunctionTok{print}\NormalTok{(df)}

\CommentTok{\# standardisation {-}{-}{-}{-}}
\ControlFlowTok{if}\NormalTok{(}\SpecialCharTok{!}\FunctionTok{require}\NormalTok{(terra)) \{}\FunctionTok{install.packages}\NormalTok{(}\StringTok{"terra"}\NormalTok{); }\FunctionTok{require}\NormalTok{(terra)\}}
\ControlFlowTok{if}\NormalTok{(}\SpecialCharTok{!}\FunctionTok{require}\NormalTok{(tidyverse)) \{}\FunctionTok{install.packages}\NormalTok{(}\StringTok{"tidyverse"}\NormalTok{); }\FunctionTok{require}\NormalTok{(tidyverse)\}}

\NormalTok{nosaukums}\OtherTok{=}\StringTok{"Climate\_CHELSAv2.1{-}gdd5\_cell.tif"}
\NormalTok{ielasisanas\_cels}\OtherTok{=}\FunctionTok{paste0}\NormalTok{(}\StringTok{"./RasterGrids\_100m/2024/RAW/"}\NormalTok{,nosaukums)}
\NormalTok{saglabasanas\_cels}\OtherTok{=}\FunctionTok{paste0}\NormalTok{(}\StringTok{"./RasterGrids\_100m/2024/Scaled/"}\NormalTok{,nosaukums)}
\NormalTok{slanis}\OtherTok{=}\FunctionTok{rast}\NormalTok{(ielasisanas\_cels)}
\NormalTok{videjais}\OtherTok{=}\FunctionTok{global}\NormalTok{(slanis,}\AttributeTok{fun=}\StringTok{"mean"}\NormalTok{,}\AttributeTok{na.rm=}\ConstantTok{TRUE}\NormalTok{)}
\NormalTok{centrets}\OtherTok{=}\NormalTok{slanis}\SpecialCharTok{{-}}\NormalTok{videjais[,}\DecValTok{1}\NormalTok{]}
\NormalTok{standartnovirze}\OtherTok{=}\NormalTok{terra}\SpecialCharTok{::}\FunctionTok{global}\NormalTok{(centrets,}\AttributeTok{fun=}\StringTok{"rms"}\NormalTok{,}\AttributeTok{na.rm=}\ConstantTok{TRUE}\NormalTok{)}
\NormalTok{merogots}\OtherTok{=}\NormalTok{centrets}\SpecialCharTok{/}\NormalTok{standartnovirze[,}\DecValTok{1}\NormalTok{]}
\FunctionTok{writeRaster}\NormalTok{(merogots,}
      \AttributeTok{filename=}\NormalTok{saglabasanas\_cels,}
      \AttributeTok{overwrite=}\ConstantTok{TRUE}\NormalTok{)}
\end{Highlighting}
\end{Shaded}

\section{Climate\_CHELSAv2.1-gddlgd0\_cell}\label{ch06.033}

\textbf{filename:} \texttt{Climate\_CHELSAv2.1-gddlgd0\_cell.tif}

\textbf{layername:} \texttt{egv\_033}

\textbf{English name:} Last growing degree day above 0°C (CHELSA v2.1) within the
analysis cell (1 ha)

\textbf{Latvian name:} Veģetācijas sezonas no 0°C pēdējā diena (CHELSA v2.1) analīzes
šūnā (1 ha)

\textbf{Procedure:} Directly follows \hyperref[Ch04.11]{CHELSA v2.1}. EGV is prepared using
the workflow \texttt{egvtools::downscale2egv()} with inverse distance weighted (power =
2) gap filling and soft smoothing (power = 0.5) over 5 km radius around each cell.
Finally, the layer is standardised by subtracting the arithmetic mean and
dividing by the root mean squared error.

\begin{Shaded}
\begin{Highlighting}[]
\CommentTok{\# libs {-}{-}{-}{-}}
\ControlFlowTok{if}\NormalTok{(}\SpecialCharTok{!}\FunctionTok{require}\NormalTok{(egvtools)) \{remotes}\SpecialCharTok{::}\FunctionTok{install\_github}\NormalTok{(}\StringTok{"aavotins/egvtools"}\NormalTok{); }\FunctionTok{require}\NormalTok{(egvtools)\}}

\CommentTok{\# job {-}{-}{-}{-}}

\NormalTok{localname}\OtherTok{=}\StringTok{"Climate\_CHELSAv2.1{-}gddlgd0\_cell.tif"}
\NormalTok{layername}\OtherTok{=}\StringTok{"egv\_033"}
\NormalTok{reading}\OtherTok{=}\StringTok{"./Geodata/2024/CHELSA/Climate\_CHELSAv2.1{-}gddlgd0\_cell.tif"}

\NormalTok{df }\OtherTok{\textless{}{-}} \FunctionTok{downscale2egv}\NormalTok{(}
 \AttributeTok{template\_path =} \StringTok{"./Templates/TemplateRasters/LV100m\_10km.tif"}\NormalTok{,}
 \AttributeTok{grid\_path   =} \StringTok{"./Templates/TemplateGrids/tikls1km\_sauzeme.parquet"}\NormalTok{,}
 \AttributeTok{rawfile\_path =}\NormalTok{ reading,}
 \AttributeTok{out\_path   =} \StringTok{"./RasterGrids\_100m/2024/RAW/"}\NormalTok{,}
 \AttributeTok{file\_name   =}\NormalTok{ localname,}
 \AttributeTok{layer\_name  =}\NormalTok{ layername,}
 \AttributeTok{fill\_gaps   =} \ConstantTok{TRUE}\NormalTok{,}
 \AttributeTok{smooth    =} \ConstantTok{TRUE}\NormalTok{,}
 \AttributeTok{smooth\_radius\_km =} \DecValTok{5}\NormalTok{,}
 \AttributeTok{plot\_result  =} \ConstantTok{TRUE}\NormalTok{)}
\FunctionTok{print}\NormalTok{(df)}

\CommentTok{\# standardisation {-}{-}{-}{-}}
\ControlFlowTok{if}\NormalTok{(}\SpecialCharTok{!}\FunctionTok{require}\NormalTok{(terra)) \{}\FunctionTok{install.packages}\NormalTok{(}\StringTok{"terra"}\NormalTok{); }\FunctionTok{require}\NormalTok{(terra)\}}
\ControlFlowTok{if}\NormalTok{(}\SpecialCharTok{!}\FunctionTok{require}\NormalTok{(tidyverse)) \{}\FunctionTok{install.packages}\NormalTok{(}\StringTok{"tidyverse"}\NormalTok{); }\FunctionTok{require}\NormalTok{(tidyverse)\}}

\NormalTok{nosaukums}\OtherTok{=}\StringTok{"Climate\_CHELSAv2.1{-}gddlgd0\_cell.tif"}
\NormalTok{ielasisanas\_cels}\OtherTok{=}\FunctionTok{paste0}\NormalTok{(}\StringTok{"./RasterGrids\_100m/2024/RAW/"}\NormalTok{,nosaukums)}
\NormalTok{saglabasanas\_cels}\OtherTok{=}\FunctionTok{paste0}\NormalTok{(}\StringTok{"./RasterGrids\_100m/2024/Scaled/"}\NormalTok{,nosaukums)}
\NormalTok{slanis}\OtherTok{=}\FunctionTok{rast}\NormalTok{(ielasisanas\_cels)}
\NormalTok{videjais}\OtherTok{=}\FunctionTok{global}\NormalTok{(slanis,}\AttributeTok{fun=}\StringTok{"mean"}\NormalTok{,}\AttributeTok{na.rm=}\ConstantTok{TRUE}\NormalTok{)}
\NormalTok{centrets}\OtherTok{=}\NormalTok{slanis}\SpecialCharTok{{-}}\NormalTok{videjais[,}\DecValTok{1}\NormalTok{]}
\NormalTok{standartnovirze}\OtherTok{=}\NormalTok{terra}\SpecialCharTok{::}\FunctionTok{global}\NormalTok{(centrets,}\AttributeTok{fun=}\StringTok{"rms"}\NormalTok{,}\AttributeTok{na.rm=}\ConstantTok{TRUE}\NormalTok{)}
\NormalTok{merogots}\OtherTok{=}\NormalTok{centrets}\SpecialCharTok{/}\NormalTok{standartnovirze[,}\DecValTok{1}\NormalTok{]}
\FunctionTok{writeRaster}\NormalTok{(merogots,}
      \AttributeTok{filename=}\NormalTok{saglabasanas\_cels,}
      \AttributeTok{overwrite=}\ConstantTok{TRUE}\NormalTok{)}
\end{Highlighting}
\end{Shaded}

\section{Climate\_CHELSAv2.1-gddlgd10\_cell}\label{ch06.034}

\textbf{filename:} \texttt{Climate\_CHELSAv2.1-gddlgd10\_cell.tif}

\textbf{layername:} \texttt{egv\_034}

\textbf{English name:} Last growing degree day above 10°C (CHELSA v2.1) within the
analysis cell (1 ha)

\textbf{Latvian name:} Veģetācijas sezonas no 10°C pēdējā diena (CHELSA v2.1)
analīzes šūnā (1 ha)

\textbf{Procedure:} Directly follows \hyperref[Ch04.11]{CHELSA v2.1}. EGV is prepared using
the workflow \texttt{egvtools::downscale2egv()} with inverse distance weighted (power =
2) gap filling and soft smoothing (power = 0.5) over 5 km radius around each cell.
Finally, the layer is standardised by subtracting the arithmetic mean and
dividing by the root mean squared error.

\begin{Shaded}
\begin{Highlighting}[]
\CommentTok{\# libs {-}{-}{-}{-}}
\ControlFlowTok{if}\NormalTok{(}\SpecialCharTok{!}\FunctionTok{require}\NormalTok{(egvtools)) \{remotes}\SpecialCharTok{::}\FunctionTok{install\_github}\NormalTok{(}\StringTok{"aavotins/egvtools"}\NormalTok{); }\FunctionTok{require}\NormalTok{(egvtools)\}}

\CommentTok{\# job {-}{-}{-}{-}}

\NormalTok{localname}\OtherTok{=}\StringTok{"Climate\_CHELSAv2.1{-}gddlgd10\_cell.tif"}
\NormalTok{layername}\OtherTok{=}\StringTok{"egv\_034"}
\NormalTok{reading}\OtherTok{=}\StringTok{"./Geodata/2024/CHELSA/Climate\_CHELSAv2.1{-}gddlgd10\_cell.tif"}

\NormalTok{df }\OtherTok{\textless{}{-}} \FunctionTok{downscale2egv}\NormalTok{(}
 \AttributeTok{template\_path =} \StringTok{"./Templates/TemplateRasters/LV100m\_10km.tif"}\NormalTok{,}
 \AttributeTok{grid\_path   =} \StringTok{"./Templates/TemplateGrids/tikls1km\_sauzeme.parquet"}\NormalTok{,}
 \AttributeTok{rawfile\_path =}\NormalTok{ reading,}
 \AttributeTok{out\_path   =} \StringTok{"./RasterGrids\_100m/2024/RAW/"}\NormalTok{,}
 \AttributeTok{file\_name   =}\NormalTok{ localname,}
 \AttributeTok{layer\_name  =}\NormalTok{ layername,}
 \AttributeTok{fill\_gaps   =} \ConstantTok{TRUE}\NormalTok{,}
 \AttributeTok{smooth    =} \ConstantTok{TRUE}\NormalTok{,}
 \AttributeTok{smooth\_radius\_km =} \DecValTok{5}\NormalTok{,}
 \AttributeTok{plot\_result  =} \ConstantTok{TRUE}\NormalTok{)}
\FunctionTok{print}\NormalTok{(df)}

\CommentTok{\# standardisation {-}{-}{-}{-}}
\ControlFlowTok{if}\NormalTok{(}\SpecialCharTok{!}\FunctionTok{require}\NormalTok{(terra)) \{}\FunctionTok{install.packages}\NormalTok{(}\StringTok{"terra"}\NormalTok{); }\FunctionTok{require}\NormalTok{(terra)\}}
\ControlFlowTok{if}\NormalTok{(}\SpecialCharTok{!}\FunctionTok{require}\NormalTok{(tidyverse)) \{}\FunctionTok{install.packages}\NormalTok{(}\StringTok{"tidyverse"}\NormalTok{); }\FunctionTok{require}\NormalTok{(tidyverse)\}}

\NormalTok{nosaukums}\OtherTok{=}\StringTok{"Climate\_CHELSAv2.1{-}gddlgd10\_cell.tif"}
\NormalTok{ielasisanas\_cels}\OtherTok{=}\FunctionTok{paste0}\NormalTok{(}\StringTok{"./RasterGrids\_100m/2024/RAW/"}\NormalTok{,nosaukums)}
\NormalTok{saglabasanas\_cels}\OtherTok{=}\FunctionTok{paste0}\NormalTok{(}\StringTok{"./RasterGrids\_100m/2024/Scaled/"}\NormalTok{,nosaukums)}
\NormalTok{slanis}\OtherTok{=}\FunctionTok{rast}\NormalTok{(ielasisanas\_cels)}
\NormalTok{videjais}\OtherTok{=}\FunctionTok{global}\NormalTok{(slanis,}\AttributeTok{fun=}\StringTok{"mean"}\NormalTok{,}\AttributeTok{na.rm=}\ConstantTok{TRUE}\NormalTok{)}
\NormalTok{centrets}\OtherTok{=}\NormalTok{slanis}\SpecialCharTok{{-}}\NormalTok{videjais[,}\DecValTok{1}\NormalTok{]}
\NormalTok{standartnovirze}\OtherTok{=}\NormalTok{terra}\SpecialCharTok{::}\FunctionTok{global}\NormalTok{(centrets,}\AttributeTok{fun=}\StringTok{"rms"}\NormalTok{,}\AttributeTok{na.rm=}\ConstantTok{TRUE}\NormalTok{)}
\NormalTok{merogots}\OtherTok{=}\NormalTok{centrets}\SpecialCharTok{/}\NormalTok{standartnovirze[,}\DecValTok{1}\NormalTok{]}
\FunctionTok{writeRaster}\NormalTok{(merogots,}
      \AttributeTok{filename=}\NormalTok{saglabasanas\_cels,}
      \AttributeTok{overwrite=}\ConstantTok{TRUE}\NormalTok{)}
\end{Highlighting}
\end{Shaded}

\section{Climate\_CHELSAv2.1-gddlgd5\_cell}\label{ch06.035}

\textbf{filename:} \texttt{Climate\_CHELSAv2.1-gddlgd5\_cell.tif}

\textbf{layername:} \texttt{egv\_035}

\textbf{English name:} Last growing degree day above 5°C (CHELSA v2.1) within the
analysis cell (1 ha)

\textbf{Latvian name:} Veģetācijas sezonas no 5°C pēdējā diena (CHELSA v2.1) analīzes
šūnā (1 ha)

\textbf{Procedure:} Directly follows \hyperref[Ch04.11]{CHELSA v2.1}. EGV is prepared using
the workflow \texttt{egvtools::downscale2egv()} with inverse distance weighted (power =
2) gap filling and soft smoothing (power = 0.5) over 5 km radius around each cell.
Finally, the layer is standardised by subtracting the arithmetic mean and
dividing by the root mean squared error.

\begin{Shaded}
\begin{Highlighting}[]
\CommentTok{\# libs {-}{-}{-}{-}}
\ControlFlowTok{if}\NormalTok{(}\SpecialCharTok{!}\FunctionTok{require}\NormalTok{(egvtools)) \{remotes}\SpecialCharTok{::}\FunctionTok{install\_github}\NormalTok{(}\StringTok{"aavotins/egvtools"}\NormalTok{); }\FunctionTok{require}\NormalTok{(egvtools)\}}

\CommentTok{\# job {-}{-}{-}{-}}

\NormalTok{localname}\OtherTok{=}\StringTok{"Climate\_CHELSAv2.1{-}gddlgd5\_cell.tif"}
\NormalTok{layername}\OtherTok{=}\StringTok{"egv\_035"}
\NormalTok{reading}\OtherTok{=}\StringTok{"./Geodata/2024/CHELSA/Climate\_CHELSAv2.1{-}gddlgd5\_cell.tif"}

\NormalTok{df }\OtherTok{\textless{}{-}} \FunctionTok{downscale2egv}\NormalTok{(}
 \AttributeTok{template\_path =} \StringTok{"./Templates/TemplateRasters/LV100m\_10km.tif"}\NormalTok{,}
 \AttributeTok{grid\_path   =} \StringTok{"./Templates/TemplateGrids/tikls1km\_sauzeme.parquet"}\NormalTok{,}
 \AttributeTok{rawfile\_path =}\NormalTok{ reading,}
 \AttributeTok{out\_path   =} \StringTok{"./RasterGrids\_100m/2024/RAW/"}\NormalTok{,}
 \AttributeTok{file\_name   =}\NormalTok{ localname,}
 \AttributeTok{layer\_name  =}\NormalTok{ layername,}
 \AttributeTok{fill\_gaps   =} \ConstantTok{TRUE}\NormalTok{,}
 \AttributeTok{smooth    =} \ConstantTok{TRUE}\NormalTok{,}
 \AttributeTok{smooth\_radius\_km =} \DecValTok{5}\NormalTok{,}
 \AttributeTok{plot\_result  =} \ConstantTok{TRUE}\NormalTok{)}
\FunctionTok{print}\NormalTok{(df)}

\CommentTok{\# standardisation {-}{-}{-}{-}}
\ControlFlowTok{if}\NormalTok{(}\SpecialCharTok{!}\FunctionTok{require}\NormalTok{(terra)) \{}\FunctionTok{install.packages}\NormalTok{(}\StringTok{"terra"}\NormalTok{); }\FunctionTok{require}\NormalTok{(terra)\}}
\ControlFlowTok{if}\NormalTok{(}\SpecialCharTok{!}\FunctionTok{require}\NormalTok{(tidyverse)) \{}\FunctionTok{install.packages}\NormalTok{(}\StringTok{"tidyverse"}\NormalTok{); }\FunctionTok{require}\NormalTok{(tidyverse)\}}

\NormalTok{nosaukums}\OtherTok{=}\StringTok{"Climate\_CHELSAv2.1{-}gddlgd5\_cell.tif"}
\NormalTok{ielasisanas\_cels}\OtherTok{=}\FunctionTok{paste0}\NormalTok{(}\StringTok{"./RasterGrids\_100m/2024/RAW/"}\NormalTok{,nosaukums)}
\NormalTok{saglabasanas\_cels}\OtherTok{=}\FunctionTok{paste0}\NormalTok{(}\StringTok{"./RasterGrids\_100m/2024/Scaled/"}\NormalTok{,nosaukums)}
\NormalTok{slanis}\OtherTok{=}\FunctionTok{rast}\NormalTok{(ielasisanas\_cels)}
\NormalTok{videjais}\OtherTok{=}\FunctionTok{global}\NormalTok{(slanis,}\AttributeTok{fun=}\StringTok{"mean"}\NormalTok{,}\AttributeTok{na.rm=}\ConstantTok{TRUE}\NormalTok{)}
\NormalTok{centrets}\OtherTok{=}\NormalTok{slanis}\SpecialCharTok{{-}}\NormalTok{videjais[,}\DecValTok{1}\NormalTok{]}
\NormalTok{standartnovirze}\OtherTok{=}\NormalTok{terra}\SpecialCharTok{::}\FunctionTok{global}\NormalTok{(centrets,}\AttributeTok{fun=}\StringTok{"rms"}\NormalTok{,}\AttributeTok{na.rm=}\ConstantTok{TRUE}\NormalTok{)}
\NormalTok{merogots}\OtherTok{=}\NormalTok{centrets}\SpecialCharTok{/}\NormalTok{standartnovirze[,}\DecValTok{1}\NormalTok{]}
\FunctionTok{writeRaster}\NormalTok{(merogots,}
      \AttributeTok{filename=}\NormalTok{saglabasanas\_cels,}
      \AttributeTok{overwrite=}\ConstantTok{TRUE}\NormalTok{)}
\end{Highlighting}
\end{Shaded}

\section{Climate\_CHELSAv2.1-gdgfgd0\_cell}\label{ch06.036}

\textbf{filename:} \texttt{Climate\_CHELSAv2.1-gdgfgd0\_cell.tif}

\textbf{layername:} \texttt{egv\_036}

\textbf{English name:} First growing degree day above 0°C (CHELSA v2.1) within the
analysis cell (1 ha)

\textbf{Latvian name:} Veģetācijas sezonas no 0°C pirmā diena (CHELSA v2.1) analīzes
šūnā (1 ha)

\textbf{Procedure:} Directly follows \hyperref[Ch04.11]{CHELSA v2.1}. EGV is prepared using
the workflow \texttt{egvtools::downscale2egv()} with inverse distance weighted (power =
2) gap filling and soft smoothing (power = 0.5) over 5 km radius around each cell.
Finally, the layer is standardised by subtracting the arithmetic mean and
dividing by the root mean squared error.

\begin{Shaded}
\begin{Highlighting}[]
\CommentTok{\# libs {-}{-}{-}{-}}
\ControlFlowTok{if}\NormalTok{(}\SpecialCharTok{!}\FunctionTok{require}\NormalTok{(egvtools)) \{remotes}\SpecialCharTok{::}\FunctionTok{install\_github}\NormalTok{(}\StringTok{"aavotins/egvtools"}\NormalTok{); }\FunctionTok{require}\NormalTok{(egvtools)\}}

\CommentTok{\# job {-}{-}{-}{-}}

\NormalTok{localname}\OtherTok{=}\StringTok{"Climate\_CHELSAv2.1{-}gdgfgd0\_cell.tif"}
\NormalTok{layername}\OtherTok{=}\StringTok{"egv\_036"}
\NormalTok{reading}\OtherTok{=}\StringTok{"./Geodata/2024/CHELSA/Climate\_CHELSAv2.1{-}gdgfgd0\_cell.tif"}

\NormalTok{df }\OtherTok{\textless{}{-}} \FunctionTok{downscale2egv}\NormalTok{(}
 \AttributeTok{template\_path =} \StringTok{"./Templates/TemplateRasters/LV100m\_10km.tif"}\NormalTok{,}
 \AttributeTok{grid\_path   =} \StringTok{"./Templates/TemplateGrids/tikls1km\_sauzeme.parquet"}\NormalTok{,}
 \AttributeTok{rawfile\_path =}\NormalTok{ reading,}
 \AttributeTok{out\_path   =} \StringTok{"./RasterGrids\_100m/2024/RAW/"}\NormalTok{,}
 \AttributeTok{file\_name   =}\NormalTok{ localname,}
 \AttributeTok{layer\_name  =}\NormalTok{ layername,}
 \AttributeTok{fill\_gaps   =} \ConstantTok{TRUE}\NormalTok{,}
 \AttributeTok{smooth    =} \ConstantTok{TRUE}\NormalTok{,}
 \AttributeTok{smooth\_radius\_km =} \DecValTok{5}\NormalTok{,}
 \AttributeTok{plot\_result  =} \ConstantTok{TRUE}\NormalTok{)}
\FunctionTok{print}\NormalTok{(df)}

\CommentTok{\# standardisation {-}{-}{-}{-}}
\ControlFlowTok{if}\NormalTok{(}\SpecialCharTok{!}\FunctionTok{require}\NormalTok{(terra)) \{}\FunctionTok{install.packages}\NormalTok{(}\StringTok{"terra"}\NormalTok{); }\FunctionTok{require}\NormalTok{(terra)\}}
\ControlFlowTok{if}\NormalTok{(}\SpecialCharTok{!}\FunctionTok{require}\NormalTok{(tidyverse)) \{}\FunctionTok{install.packages}\NormalTok{(}\StringTok{"tidyverse"}\NormalTok{); }\FunctionTok{require}\NormalTok{(tidyverse)\}}

\NormalTok{nosaukums}\OtherTok{=}\StringTok{"Climate\_CHELSAv2.1{-}gdgfgd0\_cell.tif"}
\NormalTok{ielasisanas\_cels}\OtherTok{=}\FunctionTok{paste0}\NormalTok{(}\StringTok{"./RasterGrids\_100m/2024/RAW/"}\NormalTok{,nosaukums)}
\NormalTok{saglabasanas\_cels}\OtherTok{=}\FunctionTok{paste0}\NormalTok{(}\StringTok{"./RasterGrids\_100m/2024/Scaled/"}\NormalTok{,nosaukums)}
\NormalTok{slanis}\OtherTok{=}\FunctionTok{rast}\NormalTok{(ielasisanas\_cels)}
\NormalTok{videjais}\OtherTok{=}\FunctionTok{global}\NormalTok{(slanis,}\AttributeTok{fun=}\StringTok{"mean"}\NormalTok{,}\AttributeTok{na.rm=}\ConstantTok{TRUE}\NormalTok{)}
\NormalTok{centrets}\OtherTok{=}\NormalTok{slanis}\SpecialCharTok{{-}}\NormalTok{videjais[,}\DecValTok{1}\NormalTok{]}
\NormalTok{standartnovirze}\OtherTok{=}\NormalTok{terra}\SpecialCharTok{::}\FunctionTok{global}\NormalTok{(centrets,}\AttributeTok{fun=}\StringTok{"rms"}\NormalTok{,}\AttributeTok{na.rm=}\ConstantTok{TRUE}\NormalTok{)}
\NormalTok{merogots}\OtherTok{=}\NormalTok{centrets}\SpecialCharTok{/}\NormalTok{standartnovirze[,}\DecValTok{1}\NormalTok{]}
\FunctionTok{writeRaster}\NormalTok{(merogots,}
      \AttributeTok{filename=}\NormalTok{saglabasanas\_cels,}
      \AttributeTok{overwrite=}\ConstantTok{TRUE}\NormalTok{)}
\end{Highlighting}
\end{Shaded}

\section{Climate\_CHELSAv2.1-gdgfgd10\_cell}\label{ch06.037}

\textbf{filename:} \texttt{Climate\_CHELSAv2.1-gdgfgd10\_cell.tif}

\textbf{layername:} \texttt{egv\_037}

\textbf{English name:} First growing degree day above 10°C (CHELSA v2.1) within the
analysis cell (1 ha)

\textbf{Latvian name:} Veģetācijas sezonas no 10°C pirmā diena (CHELSA v2.1) analīzes
šūnā (1 ha)

\textbf{Procedure:} Directly follows \hyperref[Ch04.11]{CHELSA v2.1}. EGV is prepared using
the workflow \texttt{egvtools::downscale2egv()} with inverse distance weighted (power =
2) gap filling and soft smoothing (power = 0.5) over 5 km radius around each cell.
Finally, the layer is standardised by subtracting the arithmetic mean and
dividing by the root mean squared error.

\begin{Shaded}
\begin{Highlighting}[]
\CommentTok{\# libs {-}{-}{-}{-}}
\ControlFlowTok{if}\NormalTok{(}\SpecialCharTok{!}\FunctionTok{require}\NormalTok{(egvtools)) \{remotes}\SpecialCharTok{::}\FunctionTok{install\_github}\NormalTok{(}\StringTok{"aavotins/egvtools"}\NormalTok{); }\FunctionTok{require}\NormalTok{(egvtools)\}}

\CommentTok{\# job {-}{-}{-}{-}}

\NormalTok{localname}\OtherTok{=}\StringTok{"Climate\_CHELSAv2.1{-}gdgfgd10\_cell.tif"}
\NormalTok{layername}\OtherTok{=}\StringTok{"egv\_037"}
\NormalTok{reading}\OtherTok{=}\StringTok{"./Geodata/2024/CHELSA/Climate\_CHELSAv2.1{-}gdgfgd10\_cell.tif"}

\NormalTok{df }\OtherTok{\textless{}{-}} \FunctionTok{downscale2egv}\NormalTok{(}
 \AttributeTok{template\_path =} \StringTok{"./Templates/TemplateRasters/LV100m\_10km.tif"}\NormalTok{,}
 \AttributeTok{grid\_path   =} \StringTok{"./Templates/TemplateGrids/tikls1km\_sauzeme.parquet"}\NormalTok{,}
 \AttributeTok{rawfile\_path =}\NormalTok{ reading,}
 \AttributeTok{out\_path   =} \StringTok{"./RasterGrids\_100m/2024/RAW/"}\NormalTok{,}
 \AttributeTok{file\_name   =}\NormalTok{ localname,}
 \AttributeTok{layer\_name  =}\NormalTok{ layername,}
 \AttributeTok{fill\_gaps   =} \ConstantTok{TRUE}\NormalTok{,}
 \AttributeTok{smooth    =} \ConstantTok{TRUE}\NormalTok{,}
 \AttributeTok{smooth\_radius\_km =} \DecValTok{5}\NormalTok{,}
 \AttributeTok{plot\_result  =} \ConstantTok{TRUE}\NormalTok{)}
\FunctionTok{print}\NormalTok{(df)}

\CommentTok{\# standardisation {-}{-}{-}{-}}
\ControlFlowTok{if}\NormalTok{(}\SpecialCharTok{!}\FunctionTok{require}\NormalTok{(terra)) \{}\FunctionTok{install.packages}\NormalTok{(}\StringTok{"terra"}\NormalTok{); }\FunctionTok{require}\NormalTok{(terra)\}}
\ControlFlowTok{if}\NormalTok{(}\SpecialCharTok{!}\FunctionTok{require}\NormalTok{(tidyverse)) \{}\FunctionTok{install.packages}\NormalTok{(}\StringTok{"tidyverse"}\NormalTok{); }\FunctionTok{require}\NormalTok{(tidyverse)\}}

\NormalTok{nosaukums}\OtherTok{=}\StringTok{"Climate\_CHELSAv2.1{-}gdgfgd10\_cell.tif"}
\NormalTok{ielasisanas\_cels}\OtherTok{=}\FunctionTok{paste0}\NormalTok{(}\StringTok{"./RasterGrids\_100m/2024/RAW/"}\NormalTok{,nosaukums)}
\NormalTok{saglabasanas\_cels}\OtherTok{=}\FunctionTok{paste0}\NormalTok{(}\StringTok{"./RasterGrids\_100m/2024/Scaled/"}\NormalTok{,nosaukums)}
\NormalTok{slanis}\OtherTok{=}\FunctionTok{rast}\NormalTok{(ielasisanas\_cels)}
\NormalTok{videjais}\OtherTok{=}\FunctionTok{global}\NormalTok{(slanis,}\AttributeTok{fun=}\StringTok{"mean"}\NormalTok{,}\AttributeTok{na.rm=}\ConstantTok{TRUE}\NormalTok{)}
\NormalTok{centrets}\OtherTok{=}\NormalTok{slanis}\SpecialCharTok{{-}}\NormalTok{videjais[,}\DecValTok{1}\NormalTok{]}
\NormalTok{standartnovirze}\OtherTok{=}\NormalTok{terra}\SpecialCharTok{::}\FunctionTok{global}\NormalTok{(centrets,}\AttributeTok{fun=}\StringTok{"rms"}\NormalTok{,}\AttributeTok{na.rm=}\ConstantTok{TRUE}\NormalTok{)}
\NormalTok{merogots}\OtherTok{=}\NormalTok{centrets}\SpecialCharTok{/}\NormalTok{standartnovirze[,}\DecValTok{1}\NormalTok{]}
\FunctionTok{writeRaster}\NormalTok{(merogots,}
      \AttributeTok{filename=}\NormalTok{saglabasanas\_cels,}
      \AttributeTok{overwrite=}\ConstantTok{TRUE}\NormalTok{)}
\end{Highlighting}
\end{Shaded}

\section{Climate\_CHELSAv2.1-gdgfgd5\_cell}\label{ch06.038}

\textbf{filename:} \texttt{Climate\_CHELSAv2.1-gdgfgd5\_cell.tif}

\textbf{layername:} \texttt{egv\_038}

\textbf{English name:} First growing degree day above 5°C (CHELSA v2.1) within the
analysis cell (1 ha)

\textbf{Latvian name:} Veģetācijas sezonas no 5°C pirmā diena (CHELSA v2.1) analīzes
šūnā (1 ha)

\textbf{Procedure:} Directly follows \hyperref[Ch04.11]{CHELSA v2.1}. EGV is prepared using
the workflow \texttt{egvtools::downscale2egv()} with inverse distance weighted (power =
2) gap filling and soft smoothing (power = 0.5) over 5 km radius around each cell.
Finally, the layer is standardised by subtracting the arithmetic mean and
dividing by the root mean squared error.

\begin{Shaded}
\begin{Highlighting}[]
\CommentTok{\# libs {-}{-}{-}{-}}
\ControlFlowTok{if}\NormalTok{(}\SpecialCharTok{!}\FunctionTok{require}\NormalTok{(egvtools)) \{remotes}\SpecialCharTok{::}\FunctionTok{install\_github}\NormalTok{(}\StringTok{"aavotins/egvtools"}\NormalTok{); }\FunctionTok{require}\NormalTok{(egvtools)\}}

\CommentTok{\# job {-}{-}{-}{-}}

\NormalTok{localname}\OtherTok{=}\StringTok{"Climate\_CHELSAv2.1{-}gdgfgd5\_cell.tif"}
\NormalTok{layername}\OtherTok{=}\StringTok{"egv\_038"}
\NormalTok{reading}\OtherTok{=}\StringTok{"./Geodata/2024/CHELSA/Climate\_CHELSAv2.1{-}gdgfgd5\_cell.tif"}

\NormalTok{df }\OtherTok{\textless{}{-}} \FunctionTok{downscale2egv}\NormalTok{(}
 \AttributeTok{template\_path =} \StringTok{"./Templates/TemplateRasters/LV100m\_10km.tif"}\NormalTok{,}
 \AttributeTok{grid\_path   =} \StringTok{"./Templates/TemplateGrids/tikls1km\_sauzeme.parquet"}\NormalTok{,}
 \AttributeTok{rawfile\_path =}\NormalTok{ reading,}
 \AttributeTok{out\_path   =} \StringTok{"./RasterGrids\_100m/2024/RAW/"}\NormalTok{,}
 \AttributeTok{file\_name   =}\NormalTok{ localname,}
 \AttributeTok{layer\_name  =}\NormalTok{ layername,}
 \AttributeTok{fill\_gaps   =} \ConstantTok{TRUE}\NormalTok{,}
 \AttributeTok{smooth    =} \ConstantTok{TRUE}\NormalTok{,}
 \AttributeTok{smooth\_radius\_km =} \DecValTok{5}\NormalTok{,}
 \AttributeTok{plot\_result  =} \ConstantTok{TRUE}\NormalTok{)}
\FunctionTok{print}\NormalTok{(df)}

\CommentTok{\# standardisation {-}{-}{-}{-}}
\ControlFlowTok{if}\NormalTok{(}\SpecialCharTok{!}\FunctionTok{require}\NormalTok{(terra)) \{}\FunctionTok{install.packages}\NormalTok{(}\StringTok{"terra"}\NormalTok{); }\FunctionTok{require}\NormalTok{(terra)\}}
\ControlFlowTok{if}\NormalTok{(}\SpecialCharTok{!}\FunctionTok{require}\NormalTok{(tidyverse)) \{}\FunctionTok{install.packages}\NormalTok{(}\StringTok{"tidyverse"}\NormalTok{); }\FunctionTok{require}\NormalTok{(tidyverse)\}}

\NormalTok{nosaukums}\OtherTok{=}\StringTok{"Climate\_CHELSAv2.1{-}gdgfgd5\_cell.tif"}
\NormalTok{ielasisanas\_cels}\OtherTok{=}\FunctionTok{paste0}\NormalTok{(}\StringTok{"./RasterGrids\_100m/2024/RAW/"}\NormalTok{,nosaukums)}
\NormalTok{saglabasanas\_cels}\OtherTok{=}\FunctionTok{paste0}\NormalTok{(}\StringTok{"./RasterGrids\_100m/2024/Scaled/"}\NormalTok{,nosaukums)}
\NormalTok{slanis}\OtherTok{=}\FunctionTok{rast}\NormalTok{(ielasisanas\_cels)}
\NormalTok{videjais}\OtherTok{=}\FunctionTok{global}\NormalTok{(slanis,}\AttributeTok{fun=}\StringTok{"mean"}\NormalTok{,}\AttributeTok{na.rm=}\ConstantTok{TRUE}\NormalTok{)}
\NormalTok{centrets}\OtherTok{=}\NormalTok{slanis}\SpecialCharTok{{-}}\NormalTok{videjais[,}\DecValTok{1}\NormalTok{]}
\NormalTok{standartnovirze}\OtherTok{=}\NormalTok{terra}\SpecialCharTok{::}\FunctionTok{global}\NormalTok{(centrets,}\AttributeTok{fun=}\StringTok{"rms"}\NormalTok{,}\AttributeTok{na.rm=}\ConstantTok{TRUE}\NormalTok{)}
\NormalTok{merogots}\OtherTok{=}\NormalTok{centrets}\SpecialCharTok{/}\NormalTok{standartnovirze[,}\DecValTok{1}\NormalTok{]}
\FunctionTok{writeRaster}\NormalTok{(merogots,}
      \AttributeTok{filename=}\NormalTok{saglabasanas\_cels,}
      \AttributeTok{overwrite=}\ConstantTok{TRUE}\NormalTok{)}
\end{Highlighting}
\end{Shaded}

\section{Climate\_CHELSAv2.1-gsl\_cell}\label{ch06.039}

\textbf{filename:} \texttt{Climate\_CHELSAv2.1-gsl\_cell.tif}

\textbf{layername:} \texttt{egv\_039}

\textbf{English name:} Length of the growing season (TREELIM) (CHELSA v2.1) within
the analysis cell (1 ha)

\textbf{Latvian name:} Veģetācijas sezonas garums (TREELIM) (CHELSA v2.1) analīzes
šūnā (1 ha)

\textbf{Procedure:} Directly follows \hyperref[Ch04.11]{CHELSA v2.1}. EGV is prepared using
the workflow \texttt{egvtools::downscale2egv()} with inverse distance weighted (power =
2) gap filling and soft smoothing (power = 0.5) over 5 km radius around each cell.
Finally, the layer is standardised by subtracting the arithmetic mean and
dividing by the root mean squared error.

\begin{Shaded}
\begin{Highlighting}[]
\CommentTok{\# libs {-}{-}{-}{-}}
\ControlFlowTok{if}\NormalTok{(}\SpecialCharTok{!}\FunctionTok{require}\NormalTok{(egvtools)) \{remotes}\SpecialCharTok{::}\FunctionTok{install\_github}\NormalTok{(}\StringTok{"aavotins/egvtools"}\NormalTok{); }\FunctionTok{require}\NormalTok{(egvtools)\}}

\CommentTok{\# job {-}{-}{-}{-}}

\NormalTok{localname}\OtherTok{=}\StringTok{"Climate\_CHELSAv2.1{-}gsl\_cell.tif"}
\NormalTok{layername}\OtherTok{=}\StringTok{"egv\_039"}
\NormalTok{reading}\OtherTok{=}\StringTok{"./Geodata/2024/CHELSA/Climate\_CHELSAv2.1{-}gsl\_cell.tif"}

\NormalTok{df }\OtherTok{\textless{}{-}} \FunctionTok{downscale2egv}\NormalTok{(}
 \AttributeTok{template\_path =} \StringTok{"./Templates/TemplateRasters/LV100m\_10km.tif"}\NormalTok{,}
 \AttributeTok{grid\_path   =} \StringTok{"./Templates/TemplateGrids/tikls1km\_sauzeme.parquet"}\NormalTok{,}
 \AttributeTok{rawfile\_path =}\NormalTok{ reading,}
 \AttributeTok{out\_path   =} \StringTok{"./RasterGrids\_100m/2024/RAW/"}\NormalTok{,}
 \AttributeTok{file\_name   =}\NormalTok{ localname,}
 \AttributeTok{layer\_name  =}\NormalTok{ layername,}
 \AttributeTok{fill\_gaps   =} \ConstantTok{TRUE}\NormalTok{,}
 \AttributeTok{smooth    =} \ConstantTok{TRUE}\NormalTok{,}
 \AttributeTok{smooth\_radius\_km =} \DecValTok{5}\NormalTok{,}
 \AttributeTok{plot\_result  =} \ConstantTok{TRUE}\NormalTok{)}
\FunctionTok{print}\NormalTok{(df)}

\CommentTok{\# standardisation {-}{-}{-}{-}}
\ControlFlowTok{if}\NormalTok{(}\SpecialCharTok{!}\FunctionTok{require}\NormalTok{(terra)) \{}\FunctionTok{install.packages}\NormalTok{(}\StringTok{"terra"}\NormalTok{); }\FunctionTok{require}\NormalTok{(terra)\}}
\ControlFlowTok{if}\NormalTok{(}\SpecialCharTok{!}\FunctionTok{require}\NormalTok{(tidyverse)) \{}\FunctionTok{install.packages}\NormalTok{(}\StringTok{"tidyverse"}\NormalTok{); }\FunctionTok{require}\NormalTok{(tidyverse)\}}

\NormalTok{nosaukums}\OtherTok{=}\StringTok{"Climate\_CHELSAv2.1{-}gsl\_cell.tif"}
\NormalTok{ielasisanas\_cels}\OtherTok{=}\FunctionTok{paste0}\NormalTok{(}\StringTok{"./RasterGrids\_100m/2024/RAW/"}\NormalTok{,nosaukums)}
\NormalTok{saglabasanas\_cels}\OtherTok{=}\FunctionTok{paste0}\NormalTok{(}\StringTok{"./RasterGrids\_100m/2024/Scaled/"}\NormalTok{,nosaukums)}
\NormalTok{slanis}\OtherTok{=}\FunctionTok{rast}\NormalTok{(ielasisanas\_cels)}
\NormalTok{videjais}\OtherTok{=}\FunctionTok{global}\NormalTok{(slanis,}\AttributeTok{fun=}\StringTok{"mean"}\NormalTok{,}\AttributeTok{na.rm=}\ConstantTok{TRUE}\NormalTok{)}
\NormalTok{centrets}\OtherTok{=}\NormalTok{slanis}\SpecialCharTok{{-}}\NormalTok{videjais[,}\DecValTok{1}\NormalTok{]}
\NormalTok{standartnovirze}\OtherTok{=}\NormalTok{terra}\SpecialCharTok{::}\FunctionTok{global}\NormalTok{(centrets,}\AttributeTok{fun=}\StringTok{"rms"}\NormalTok{,}\AttributeTok{na.rm=}\ConstantTok{TRUE}\NormalTok{)}
\NormalTok{merogots}\OtherTok{=}\NormalTok{centrets}\SpecialCharTok{/}\NormalTok{standartnovirze[,}\DecValTok{1}\NormalTok{]}
\FunctionTok{writeRaster}\NormalTok{(merogots,}
      \AttributeTok{filename=}\NormalTok{saglabasanas\_cels,}
      \AttributeTok{overwrite=}\ConstantTok{TRUE}\NormalTok{)}
\end{Highlighting}
\end{Shaded}

\section{Climate\_CHELSAv2.1-gsp\_cell}\label{ch06.040}

\textbf{filename:} \texttt{Climate\_CHELSAv2.1-gsp\_cell.tif}

\textbf{layername:} \texttt{egv\_040}

\textbf{English name:} Accumulated precipitation amount (kg m⁻² year⁻¹) on growing
season days (TREELIM) (CHELSA v2.1) within the analysis cell (1 ha)

\textbf{Latvian name:} Veģetācijas sezonā (TREELIM) uzkrātais nokrišņu daudzums (kg
m⁻² year⁻¹) (CHELSA v2.1) analīzes šūnā (1 ha)

\textbf{Procedure:} Directly follows \hyperref[Ch04.11]{CHELSA v2.1}. EGV is prepared using
the workflow \texttt{egvtools::downscale2egv()} with inverse distance weighted (power =
2) gap filling and soft smoothing (power = 0.5) over 5 km radius around each cell.
Finally, the layer is standardised by subtracting the arithmetic mean and
dividing by the root mean squared error.

\begin{Shaded}
\begin{Highlighting}[]
\CommentTok{\# libs {-}{-}{-}{-}}
\ControlFlowTok{if}\NormalTok{(}\SpecialCharTok{!}\FunctionTok{require}\NormalTok{(egvtools)) \{remotes}\SpecialCharTok{::}\FunctionTok{install\_github}\NormalTok{(}\StringTok{"aavotins/egvtools"}\NormalTok{); }\FunctionTok{require}\NormalTok{(egvtools)\}}

\CommentTok{\# job {-}{-}{-}{-}}

\NormalTok{localname}\OtherTok{=}\StringTok{"Climate\_CHELSAv2.1{-}gsp\_cell.tif"}
\NormalTok{layername}\OtherTok{=}\StringTok{"egv\_040"}
\NormalTok{reading}\OtherTok{=}\StringTok{"./Geodata/2024/CHELSA/Climate\_CHELSAv2.1{-}gsp\_cell.tif"}

\NormalTok{df }\OtherTok{\textless{}{-}} \FunctionTok{downscale2egv}\NormalTok{(}
 \AttributeTok{template\_path =} \StringTok{"./Templates/TemplateRasters/LV100m\_10km.tif"}\NormalTok{,}
 \AttributeTok{grid\_path   =} \StringTok{"./Templates/TemplateGrids/tikls1km\_sauzeme.parquet"}\NormalTok{,}
 \AttributeTok{rawfile\_path =}\NormalTok{ reading,}
 \AttributeTok{out\_path   =} \StringTok{"./RasterGrids\_100m/2024/RAW/"}\NormalTok{,}
 \AttributeTok{file\_name   =}\NormalTok{ localname,}
 \AttributeTok{layer\_name  =}\NormalTok{ layername,}
 \AttributeTok{fill\_gaps   =} \ConstantTok{TRUE}\NormalTok{,}
 \AttributeTok{smooth    =} \ConstantTok{TRUE}\NormalTok{,}
 \AttributeTok{smooth\_radius\_km =} \DecValTok{5}\NormalTok{,}
 \AttributeTok{plot\_result  =} \ConstantTok{TRUE}\NormalTok{)}
\FunctionTok{print}\NormalTok{(df)}

\CommentTok{\# standardisation {-}{-}{-}{-}}
\ControlFlowTok{if}\NormalTok{(}\SpecialCharTok{!}\FunctionTok{require}\NormalTok{(terra)) \{}\FunctionTok{install.packages}\NormalTok{(}\StringTok{"terra"}\NormalTok{); }\FunctionTok{require}\NormalTok{(terra)\}}
\ControlFlowTok{if}\NormalTok{(}\SpecialCharTok{!}\FunctionTok{require}\NormalTok{(tidyverse)) \{}\FunctionTok{install.packages}\NormalTok{(}\StringTok{"tidyverse"}\NormalTok{); }\FunctionTok{require}\NormalTok{(tidyverse)\}}

\NormalTok{nosaukums}\OtherTok{=}\StringTok{"Climate\_CHELSAv2.1{-}gsp\_cell.tif"}
\NormalTok{ielasisanas\_cels}\OtherTok{=}\FunctionTok{paste0}\NormalTok{(}\StringTok{"./RasterGrids\_100m/2024/RAW/"}\NormalTok{,nosaukums)}
\NormalTok{saglabasanas\_cels}\OtherTok{=}\FunctionTok{paste0}\NormalTok{(}\StringTok{"./RasterGrids\_100m/2024/Scaled/"}\NormalTok{,nosaukums)}
\NormalTok{slanis}\OtherTok{=}\FunctionTok{rast}\NormalTok{(ielasisanas\_cels)}
\NormalTok{videjais}\OtherTok{=}\FunctionTok{global}\NormalTok{(slanis,}\AttributeTok{fun=}\StringTok{"mean"}\NormalTok{,}\AttributeTok{na.rm=}\ConstantTok{TRUE}\NormalTok{)}
\NormalTok{centrets}\OtherTok{=}\NormalTok{slanis}\SpecialCharTok{{-}}\NormalTok{videjais[,}\DecValTok{1}\NormalTok{]}
\NormalTok{standartnovirze}\OtherTok{=}\NormalTok{terra}\SpecialCharTok{::}\FunctionTok{global}\NormalTok{(centrets,}\AttributeTok{fun=}\StringTok{"rms"}\NormalTok{,}\AttributeTok{na.rm=}\ConstantTok{TRUE}\NormalTok{)}
\NormalTok{merogots}\OtherTok{=}\NormalTok{centrets}\SpecialCharTok{/}\NormalTok{standartnovirze[,}\DecValTok{1}\NormalTok{]}
\FunctionTok{writeRaster}\NormalTok{(merogots,}
      \AttributeTok{filename=}\NormalTok{saglabasanas\_cels,}
      \AttributeTok{overwrite=}\ConstantTok{TRUE}\NormalTok{)}
\end{Highlighting}
\end{Shaded}

\section{Climate\_CHELSAv2.1-gst\_cell}\label{ch06.041}

\textbf{filename:} \texttt{Climate\_CHELSAv2.1-gst\_cell.tif}

\textbf{layername:} \texttt{egv\_041}

\textbf{English name:} Mean daily mean air temperature of the growing season (TREELIM) (CHELSA v2.1)
within the analysis cell (1 ha)

\textbf{Latvian name:} Vidējā ik dienas vidējā gaisa temperatūra (°C) veģetācijas sezonā
(TREELIM) (CHELSA v2.1) analīzes šūnā (1 ha)

\textbf{Procedure:} Directly follows \hyperref[Ch04.11]{CHELSA v2.1}. EGV is prepared using
the workflow \texttt{egvtools::downscale2egv()} with inverse distance weighted (power =
2) gap filling and soft smoothing (power = 0.5) over 5 km radius around each cell.
Finally, the layer is standardised by subtracting the arithmetic mean and
dividing by the root mean squared error.

\begin{Shaded}
\begin{Highlighting}[]
\CommentTok{\# libs {-}{-}{-}{-}}
\ControlFlowTok{if}\NormalTok{(}\SpecialCharTok{!}\FunctionTok{require}\NormalTok{(egvtools)) \{remotes}\SpecialCharTok{::}\FunctionTok{install\_github}\NormalTok{(}\StringTok{"aavotins/egvtools"}\NormalTok{); }\FunctionTok{require}\NormalTok{(egvtools)\}}

\CommentTok{\# job {-}{-}{-}{-}}

\NormalTok{localname}\OtherTok{=}\StringTok{"Climate\_CHELSAv2.1{-}gst\_cell.tif"}
\NormalTok{layername}\OtherTok{=}\StringTok{"egv\_041"}
\NormalTok{reading}\OtherTok{=}\StringTok{"./Geodata/2024/CHELSA/Climate\_CHELSAv2.1{-}gst\_cell.tif"}

\NormalTok{df }\OtherTok{\textless{}{-}} \FunctionTok{downscale2egv}\NormalTok{(}
 \AttributeTok{template\_path =} \StringTok{"./Templates/TemplateRasters/LV100m\_10km.tif"}\NormalTok{,}
 \AttributeTok{grid\_path   =} \StringTok{"./Templates/TemplateGrids/tikls1km\_sauzeme.parquet"}\NormalTok{,}
 \AttributeTok{rawfile\_path =}\NormalTok{ reading,}
 \AttributeTok{out\_path   =} \StringTok{"./RasterGrids\_100m/2024/RAW/"}\NormalTok{,}
 \AttributeTok{file\_name   =}\NormalTok{ localname,}
 \AttributeTok{layer\_name  =}\NormalTok{ layername,}
 \AttributeTok{fill\_gaps   =} \ConstantTok{TRUE}\NormalTok{,}
 \AttributeTok{smooth    =} \ConstantTok{TRUE}\NormalTok{,}
 \AttributeTok{smooth\_radius\_km =} \DecValTok{5}\NormalTok{,}
 \AttributeTok{plot\_result  =} \ConstantTok{TRUE}\NormalTok{)}
\FunctionTok{print}\NormalTok{(df)}

\CommentTok{\# standardisation {-}{-}{-}{-}}
\ControlFlowTok{if}\NormalTok{(}\SpecialCharTok{!}\FunctionTok{require}\NormalTok{(terra)) \{}\FunctionTok{install.packages}\NormalTok{(}\StringTok{"terra"}\NormalTok{); }\FunctionTok{require}\NormalTok{(terra)\}}
\ControlFlowTok{if}\NormalTok{(}\SpecialCharTok{!}\FunctionTok{require}\NormalTok{(tidyverse)) \{}\FunctionTok{install.packages}\NormalTok{(}\StringTok{"tidyverse"}\NormalTok{); }\FunctionTok{require}\NormalTok{(tidyverse)\}}

\NormalTok{nosaukums}\OtherTok{=}\StringTok{"Climate\_CHELSAv2.1{-}gst\_cell.tif"}
\NormalTok{ielasisanas\_cels}\OtherTok{=}\FunctionTok{paste0}\NormalTok{(}\StringTok{"./RasterGrids\_100m/2024/RAW/"}\NormalTok{,nosaukums)}
\NormalTok{saglabasanas\_cels}\OtherTok{=}\FunctionTok{paste0}\NormalTok{(}\StringTok{"./RasterGrids\_100m/2024/Scaled/"}\NormalTok{,nosaukums)}
\NormalTok{slanis}\OtherTok{=}\FunctionTok{rast}\NormalTok{(ielasisanas\_cels)}
\NormalTok{videjais}\OtherTok{=}\FunctionTok{global}\NormalTok{(slanis,}\AttributeTok{fun=}\StringTok{"mean"}\NormalTok{,}\AttributeTok{na.rm=}\ConstantTok{TRUE}\NormalTok{)}
\NormalTok{centrets}\OtherTok{=}\NormalTok{slanis}\SpecialCharTok{{-}}\NormalTok{videjais[,}\DecValTok{1}\NormalTok{]}
\NormalTok{standartnovirze}\OtherTok{=}\NormalTok{terra}\SpecialCharTok{::}\FunctionTok{global}\NormalTok{(centrets,}\AttributeTok{fun=}\StringTok{"rms"}\NormalTok{,}\AttributeTok{na.rm=}\ConstantTok{TRUE}\NormalTok{)}
\NormalTok{merogots}\OtherTok{=}\NormalTok{centrets}\SpecialCharTok{/}\NormalTok{standartnovirze[,}\DecValTok{1}\NormalTok{]}
\FunctionTok{writeRaster}\NormalTok{(merogots,}
      \AttributeTok{filename=}\NormalTok{saglabasanas\_cels,}
      \AttributeTok{overwrite=}\ConstantTok{TRUE}\NormalTok{)}
\end{Highlighting}
\end{Shaded}

\section{Climate\_CHELSAv2.1-hurs-max\_cell}\label{ch06.042}

\textbf{filename:} \texttt{Climate\_CHELSAv2.1-hurs-max\_cell.tif}

\textbf{layername:} \texttt{egv\_042}

\textbf{English name:} Mean of monthly maximum near-surface relative humidity (\%) (CHELSA
v2.1) within the analysis cell (1 ha)

\textbf{Latvian name:} Vidējais ik mēneša maksimālais piezemes slāņa gaisa mitrums (\%) (CHELSA v2.1)
analīzes šūnā (1 ha)

\textbf{Procedure:} Directly follows \hyperref[Ch04.11]{CHELSA v2.1}. EGV is prepared using
the workflow \texttt{egvtools::downscale2egv()} with inverse distance weighted (power =
2) gap filling and soft smoothing (power = 0.5) over 5 km radius around each cell.
Finally, the layer is standardised by subtracting the arithmetic mean and
dividing by the root mean squared error.

\begin{Shaded}
\begin{Highlighting}[]
\CommentTok{\# libs {-}{-}{-}{-}}
\ControlFlowTok{if}\NormalTok{(}\SpecialCharTok{!}\FunctionTok{require}\NormalTok{(egvtools)) \{remotes}\SpecialCharTok{::}\FunctionTok{install\_github}\NormalTok{(}\StringTok{"aavotins/egvtools"}\NormalTok{); }\FunctionTok{require}\NormalTok{(egvtools)\}}

\CommentTok{\# job {-}{-}{-}{-}}

\NormalTok{localname}\OtherTok{=}\StringTok{"Climate\_CHELSAv2.1{-}hurs{-}max\_cell.tif"}
\NormalTok{layername}\OtherTok{=}\StringTok{"egv\_042"}
\NormalTok{reading}\OtherTok{=}\StringTok{"./Geodata/2024/CHELSA/Climate\_CHELSAv2.1{-}hurs{-}max\_cell.tif"}

\NormalTok{df }\OtherTok{\textless{}{-}} \FunctionTok{downscale2egv}\NormalTok{(}
 \AttributeTok{template\_path =} \StringTok{"./Templates/TemplateRasters/LV100m\_10km.tif"}\NormalTok{,}
 \AttributeTok{grid\_path   =} \StringTok{"./Templates/TemplateGrids/tikls1km\_sauzeme.parquet"}\NormalTok{,}
 \AttributeTok{rawfile\_path =}\NormalTok{ reading,}
 \AttributeTok{out\_path   =} \StringTok{"./RasterGrids\_100m/2024/RAW/"}\NormalTok{,}
 \AttributeTok{file\_name   =}\NormalTok{ localname,}
 \AttributeTok{layer\_name  =}\NormalTok{ layername,}
 \AttributeTok{fill\_gaps   =} \ConstantTok{TRUE}\NormalTok{,}
 \AttributeTok{smooth    =} \ConstantTok{TRUE}\NormalTok{,}
 \AttributeTok{smooth\_radius\_km =} \DecValTok{5}\NormalTok{,}
 \AttributeTok{plot\_result  =} \ConstantTok{TRUE}\NormalTok{)}
\FunctionTok{print}\NormalTok{(df)}

\CommentTok{\# standardisation {-}{-}{-}{-}}
\ControlFlowTok{if}\NormalTok{(}\SpecialCharTok{!}\FunctionTok{require}\NormalTok{(terra)) \{}\FunctionTok{install.packages}\NormalTok{(}\StringTok{"terra"}\NormalTok{); }\FunctionTok{require}\NormalTok{(terra)\}}
\ControlFlowTok{if}\NormalTok{(}\SpecialCharTok{!}\FunctionTok{require}\NormalTok{(tidyverse)) \{}\FunctionTok{install.packages}\NormalTok{(}\StringTok{"tidyverse"}\NormalTok{); }\FunctionTok{require}\NormalTok{(tidyverse)\}}

\NormalTok{nosaukums}\OtherTok{=}\StringTok{"Climate\_CHELSAv2.1{-}hurs{-}max\_cell.tif"}
\NormalTok{ielasisanas\_cels}\OtherTok{=}\FunctionTok{paste0}\NormalTok{(}\StringTok{"./RasterGrids\_100m/2024/RAW/"}\NormalTok{,nosaukums)}
\NormalTok{saglabasanas\_cels}\OtherTok{=}\FunctionTok{paste0}\NormalTok{(}\StringTok{"./RasterGrids\_100m/2024/Scaled/"}\NormalTok{,nosaukums)}
\NormalTok{slanis}\OtherTok{=}\FunctionTok{rast}\NormalTok{(ielasisanas\_cels)}
\NormalTok{videjais}\OtherTok{=}\FunctionTok{global}\NormalTok{(slanis,}\AttributeTok{fun=}\StringTok{"mean"}\NormalTok{,}\AttributeTok{na.rm=}\ConstantTok{TRUE}\NormalTok{)}
\NormalTok{centrets}\OtherTok{=}\NormalTok{slanis}\SpecialCharTok{{-}}\NormalTok{videjais[,}\DecValTok{1}\NormalTok{]}
\NormalTok{standartnovirze}\OtherTok{=}\NormalTok{terra}\SpecialCharTok{::}\FunctionTok{global}\NormalTok{(centrets,}\AttributeTok{fun=}\StringTok{"rms"}\NormalTok{,}\AttributeTok{na.rm=}\ConstantTok{TRUE}\NormalTok{)}
\NormalTok{merogots}\OtherTok{=}\NormalTok{centrets}\SpecialCharTok{/}\NormalTok{standartnovirze[,}\DecValTok{1}\NormalTok{]}
\FunctionTok{writeRaster}\NormalTok{(merogots,}
      \AttributeTok{filename=}\NormalTok{saglabasanas\_cels,}
      \AttributeTok{overwrite=}\ConstantTok{TRUE}\NormalTok{)}
\end{Highlighting}
\end{Shaded}

\section{Climate\_CHELSAv2.1-hurs-mean\_cell}\label{ch06.043}

\textbf{filename:} \texttt{Climate\_CHELSAv2.1-hurs-mean\_cell.tif}

\textbf{layername:} \texttt{egv\_043}

\textbf{English name:} Mean of monthly mean near-surface relative humidity (\%) (CHELSA v2.1)
within the analysis cell (1 ha)

\textbf{Latvian name:} Vidējais ik mēneša vidējais piezemes slāņa gaisa mitrums (\%) (CHELSA v2.1) analīzes
šūnā (1 ha)

\textbf{Procedure:} Directly follows \hyperref[Ch04.11]{CHELSA v2.1}. EGV is prepared using
the workflow \texttt{egvtools::downscale2egv()} with inverse distance weighted (power =
2) gap filling and soft smoothing (power = 0.5) over 5 km radius around each cell.
Finally, the layer is standardised by subtracting the arithmetic mean and
dividing by the root mean squared error.

\begin{Shaded}
\begin{Highlighting}[]
\CommentTok{\# libs {-}{-}{-}{-}}
\ControlFlowTok{if}\NormalTok{(}\SpecialCharTok{!}\FunctionTok{require}\NormalTok{(egvtools)) \{remotes}\SpecialCharTok{::}\FunctionTok{install\_github}\NormalTok{(}\StringTok{"aavotins/egvtools"}\NormalTok{); }\FunctionTok{require}\NormalTok{(egvtools)\}}

\CommentTok{\# job {-}{-}{-}{-}}

\NormalTok{localname}\OtherTok{=}\StringTok{"Climate\_CHELSAv2.1{-}hurs{-}mean\_cell.tif"}
\NormalTok{layername}\OtherTok{=}\StringTok{"egv\_043"}
\NormalTok{reading}\OtherTok{=}\StringTok{"./Geodata/2024/CHELSA/Climate\_CHELSAv2.1{-}hurs{-}mean\_cell.tif"}

\NormalTok{df }\OtherTok{\textless{}{-}} \FunctionTok{downscale2egv}\NormalTok{(}
 \AttributeTok{template\_path =} \StringTok{"./Templates/TemplateRasters/LV100m\_10km.tif"}\NormalTok{,}
 \AttributeTok{grid\_path   =} \StringTok{"./Templates/TemplateGrids/tikls1km\_sauzeme.parquet"}\NormalTok{,}
 \AttributeTok{rawfile\_path =}\NormalTok{ reading,}
 \AttributeTok{out\_path   =} \StringTok{"./RasterGrids\_100m/2024/RAW/"}\NormalTok{,}
 \AttributeTok{file\_name   =}\NormalTok{ localname,}
 \AttributeTok{layer\_name  =}\NormalTok{ layername,}
 \AttributeTok{fill\_gaps   =} \ConstantTok{TRUE}\NormalTok{,}
 \AttributeTok{smooth    =} \ConstantTok{TRUE}\NormalTok{,}
 \AttributeTok{smooth\_radius\_km =} \DecValTok{5}\NormalTok{,}
 \AttributeTok{plot\_result  =} \ConstantTok{TRUE}\NormalTok{)}
\FunctionTok{print}\NormalTok{(df)}

\CommentTok{\# standardisation {-}{-}{-}{-}}
\ControlFlowTok{if}\NormalTok{(}\SpecialCharTok{!}\FunctionTok{require}\NormalTok{(terra)) \{}\FunctionTok{install.packages}\NormalTok{(}\StringTok{"terra"}\NormalTok{); }\FunctionTok{require}\NormalTok{(terra)\}}
\ControlFlowTok{if}\NormalTok{(}\SpecialCharTok{!}\FunctionTok{require}\NormalTok{(tidyverse)) \{}\FunctionTok{install.packages}\NormalTok{(}\StringTok{"tidyverse"}\NormalTok{); }\FunctionTok{require}\NormalTok{(tidyverse)\}}

\NormalTok{nosaukums}\OtherTok{=}\StringTok{"Climate\_CHELSAv2.1{-}hurs{-}mean\_cell.tif"}
\NormalTok{ielasisanas\_cels}\OtherTok{=}\FunctionTok{paste0}\NormalTok{(}\StringTok{"./RasterGrids\_100m/2024/RAW/"}\NormalTok{,nosaukums)}
\NormalTok{saglabasanas\_cels}\OtherTok{=}\FunctionTok{paste0}\NormalTok{(}\StringTok{"./RasterGrids\_100m/2024/Scaled/"}\NormalTok{,nosaukums)}
\NormalTok{slanis}\OtherTok{=}\FunctionTok{rast}\NormalTok{(ielasisanas\_cels)}
\NormalTok{videjais}\OtherTok{=}\FunctionTok{global}\NormalTok{(slanis,}\AttributeTok{fun=}\StringTok{"mean"}\NormalTok{,}\AttributeTok{na.rm=}\ConstantTok{TRUE}\NormalTok{)}
\NormalTok{centrets}\OtherTok{=}\NormalTok{slanis}\SpecialCharTok{{-}}\NormalTok{videjais[,}\DecValTok{1}\NormalTok{]}
\NormalTok{standartnovirze}\OtherTok{=}\NormalTok{terra}\SpecialCharTok{::}\FunctionTok{global}\NormalTok{(centrets,}\AttributeTok{fun=}\StringTok{"rms"}\NormalTok{,}\AttributeTok{na.rm=}\ConstantTok{TRUE}\NormalTok{)}
\NormalTok{merogots}\OtherTok{=}\NormalTok{centrets}\SpecialCharTok{/}\NormalTok{standartnovirze[,}\DecValTok{1}\NormalTok{]}
\FunctionTok{writeRaster}\NormalTok{(merogots,}
      \AttributeTok{filename=}\NormalTok{saglabasanas\_cels,}
      \AttributeTok{overwrite=}\ConstantTok{TRUE}\NormalTok{)}
\end{Highlighting}
\end{Shaded}

\section{Climate\_CHELSAv2.1-hurs-min\_cell}\label{ch06.044}

\textbf{filename:} \texttt{Climate\_CHELSAv2.1-hurs-min\_cell.tif}

\textbf{layername:} \texttt{egv\_044}

\textbf{English name:} Mean of monthly minimum near-surface relative humidity (\%) (CHELSA
v2.1) within the analysis cell (1 ha)

\textbf{Latvian name:} Vidējais ik mēneša minimālais gaisa mitrums (\%) (CHELSA v2.1)
analīzes šūnā (1 ha)

\textbf{Procedure:} Directly follows \hyperref[Ch04.11]{CHELSA v2.1}. EGV is prepared using
the workflow \texttt{egvtools::downscale2egv()} with inverse distance weighted (power =
2) gap filling and soft smoothing (power = 0.5) over 5 km radius around each cell.
Finally, the layer is standardised by subtracting the arithmetic mean and
dividing by the root mean squared error.

\begin{Shaded}
\begin{Highlighting}[]
\CommentTok{\# libs {-}{-}{-}{-}}
\ControlFlowTok{if}\NormalTok{(}\SpecialCharTok{!}\FunctionTok{require}\NormalTok{(egvtools)) \{remotes}\SpecialCharTok{::}\FunctionTok{install\_github}\NormalTok{(}\StringTok{"aavotins/egvtools"}\NormalTok{); }\FunctionTok{require}\NormalTok{(egvtools)\}}

\CommentTok{\# job {-}{-}{-}{-}}

\NormalTok{localname}\OtherTok{=}\StringTok{"Climate\_CHELSAv2.1{-}hurs{-}min\_cell.tif"}
\NormalTok{layername}\OtherTok{=}\StringTok{"egv\_044"}
\NormalTok{reading}\OtherTok{=}\StringTok{"./Geodata/2024/CHELSA/Climate\_CHELSAv2.1{-}hurs{-}min\_cell.tif"}

\NormalTok{df }\OtherTok{\textless{}{-}} \FunctionTok{downscale2egv}\NormalTok{(}
 \AttributeTok{template\_path =} \StringTok{"./Templates/TemplateRasters/LV100m\_10km.tif"}\NormalTok{,}
 \AttributeTok{grid\_path   =} \StringTok{"./Templates/TemplateGrids/tikls1km\_sauzeme.parquet"}\NormalTok{,}
 \AttributeTok{rawfile\_path =}\NormalTok{ reading,}
 \AttributeTok{out\_path   =} \StringTok{"./RasterGrids\_100m/2024/RAW/"}\NormalTok{,}
 \AttributeTok{file\_name   =}\NormalTok{ localname,}
 \AttributeTok{layer\_name  =}\NormalTok{ layername,}
 \AttributeTok{fill\_gaps   =} \ConstantTok{TRUE}\NormalTok{,}
 \AttributeTok{smooth    =} \ConstantTok{TRUE}\NormalTok{,}
 \AttributeTok{smooth\_radius\_km =} \DecValTok{5}\NormalTok{,}
 \AttributeTok{plot\_result  =} \ConstantTok{TRUE}\NormalTok{)}
\FunctionTok{print}\NormalTok{(df)}

\CommentTok{\# standardisation {-}{-}{-}{-}}
\ControlFlowTok{if}\NormalTok{(}\SpecialCharTok{!}\FunctionTok{require}\NormalTok{(terra)) \{}\FunctionTok{install.packages}\NormalTok{(}\StringTok{"terra"}\NormalTok{); }\FunctionTok{require}\NormalTok{(terra)\}}
\ControlFlowTok{if}\NormalTok{(}\SpecialCharTok{!}\FunctionTok{require}\NormalTok{(tidyverse)) \{}\FunctionTok{install.packages}\NormalTok{(}\StringTok{"tidyverse"}\NormalTok{); }\FunctionTok{require}\NormalTok{(tidyverse)\}}

\NormalTok{nosaukums}\OtherTok{=}\StringTok{"Climate\_CHELSAv2.1{-}hurs{-}min\_cell.tif"}
\NormalTok{ielasisanas\_cels}\OtherTok{=}\FunctionTok{paste0}\NormalTok{(}\StringTok{"./RasterGrids\_100m/2024/RAW/"}\NormalTok{,nosaukums)}
\NormalTok{saglabasanas\_cels}\OtherTok{=}\FunctionTok{paste0}\NormalTok{(}\StringTok{"./RasterGrids\_100m/2024/Scaled/"}\NormalTok{,nosaukums)}
\NormalTok{slanis}\OtherTok{=}\FunctionTok{rast}\NormalTok{(ielasisanas\_cels)}
\NormalTok{videjais}\OtherTok{=}\FunctionTok{global}\NormalTok{(slanis,}\AttributeTok{fun=}\StringTok{"mean"}\NormalTok{,}\AttributeTok{na.rm=}\ConstantTok{TRUE}\NormalTok{)}
\NormalTok{centrets}\OtherTok{=}\NormalTok{slanis}\SpecialCharTok{{-}}\NormalTok{videjais[,}\DecValTok{1}\NormalTok{]}
\NormalTok{standartnovirze}\OtherTok{=}\NormalTok{terra}\SpecialCharTok{::}\FunctionTok{global}\NormalTok{(centrets,}\AttributeTok{fun=}\StringTok{"rms"}\NormalTok{,}\AttributeTok{na.rm=}\ConstantTok{TRUE}\NormalTok{)}
\NormalTok{merogots}\OtherTok{=}\NormalTok{centrets}\SpecialCharTok{/}\NormalTok{standartnovirze[,}\DecValTok{1}\NormalTok{]}
\FunctionTok{writeRaster}\NormalTok{(merogots,}
      \AttributeTok{filename=}\NormalTok{saglabasanas\_cels,}
      \AttributeTok{overwrite=}\ConstantTok{TRUE}\NormalTok{)}
\end{Highlighting}
\end{Shaded}

\section{Climate\_CHELSAv2.1-hurs-range\_cell}\label{ch06.045}

\textbf{filename:} \texttt{Climate\_CHELSAv2.1-hurs-range\_cell.tif}

\textbf{layername:} \texttt{egv\_045}

\textbf{English name:} Annual range of monthly near-surface relative humidity (\%)
(CHELSA v2.1) within the analysis cell (1 ha)

\textbf{Latvian name:} Gada gaisa mitruma piezemes slānī amplitūda (\%) (CHELSA v2.1) analīzes šūnā
(1 ha)

\textbf{Procedure:} Directly follows \hyperref[Ch04.11]{CHELSA v2.1}. EGV is prepared using
the workflow \texttt{egvtools::downscale2egv()} with inverse distance weighted (power =
2) gap filling and soft smoothing (power = 0.5) over 5 km radius around each cell.
Finally, the layer is standardised by subtracting the arithmetic mean and
dividing by the root mean squared error.

\begin{Shaded}
\begin{Highlighting}[]
\CommentTok{\# libs {-}{-}{-}{-}}
\ControlFlowTok{if}\NormalTok{(}\SpecialCharTok{!}\FunctionTok{require}\NormalTok{(egvtools)) \{remotes}\SpecialCharTok{::}\FunctionTok{install\_github}\NormalTok{(}\StringTok{"aavotins/egvtools"}\NormalTok{); }\FunctionTok{require}\NormalTok{(egvtools)\}}

\CommentTok{\# job {-}{-}{-}{-}}

\NormalTok{localname}\OtherTok{=}\StringTok{"Climate\_CHELSAv2.1{-}hurs{-}range\_cell.tif"}
\NormalTok{layername}\OtherTok{=}\StringTok{"egv\_045"}
\NormalTok{reading}\OtherTok{=}\StringTok{"./Geodata/2024/CHELSA/Climate\_CHELSAv2.1{-}hurs{-}range\_cell.tif"}

\NormalTok{df }\OtherTok{\textless{}{-}} \FunctionTok{downscale2egv}\NormalTok{(}
 \AttributeTok{template\_path =} \StringTok{"./Templates/TemplateRasters/LV100m\_10km.tif"}\NormalTok{,}
 \AttributeTok{grid\_path   =} \StringTok{"./Templates/TemplateGrids/tikls1km\_sauzeme.parquet"}\NormalTok{,}
 \AttributeTok{rawfile\_path =}\NormalTok{ reading,}
 \AttributeTok{out\_path   =} \StringTok{"./RasterGrids\_100m/2024/RAW/"}\NormalTok{,}
 \AttributeTok{file\_name   =}\NormalTok{ localname,}
 \AttributeTok{layer\_name  =}\NormalTok{ layername,}
 \AttributeTok{fill\_gaps   =} \ConstantTok{TRUE}\NormalTok{,}
 \AttributeTok{smooth    =} \ConstantTok{TRUE}\NormalTok{,}
 \AttributeTok{smooth\_radius\_km =} \DecValTok{5}\NormalTok{,}
 \AttributeTok{plot\_result  =} \ConstantTok{TRUE}\NormalTok{)}
\FunctionTok{print}\NormalTok{(df)}

\CommentTok{\# standardisation {-}{-}{-}{-}}
\ControlFlowTok{if}\NormalTok{(}\SpecialCharTok{!}\FunctionTok{require}\NormalTok{(terra)) \{}\FunctionTok{install.packages}\NormalTok{(}\StringTok{"terra"}\NormalTok{); }\FunctionTok{require}\NormalTok{(terra)\}}
\ControlFlowTok{if}\NormalTok{(}\SpecialCharTok{!}\FunctionTok{require}\NormalTok{(tidyverse)) \{}\FunctionTok{install.packages}\NormalTok{(}\StringTok{"tidyverse"}\NormalTok{); }\FunctionTok{require}\NormalTok{(tidyverse)\}}

\NormalTok{nosaukums}\OtherTok{=}\StringTok{"Climate\_CHELSAv2.1{-}hurs{-}range\_cell.tif"}
\NormalTok{ielasisanas\_cels}\OtherTok{=}\FunctionTok{paste0}\NormalTok{(}\StringTok{"./RasterGrids\_100m/2024/RAW/"}\NormalTok{,nosaukums)}
\NormalTok{saglabasanas\_cels}\OtherTok{=}\FunctionTok{paste0}\NormalTok{(}\StringTok{"./RasterGrids\_100m/2024/Scaled/"}\NormalTok{,nosaukums)}
\NormalTok{slanis}\OtherTok{=}\FunctionTok{rast}\NormalTok{(ielasisanas\_cels)}
\NormalTok{videjais}\OtherTok{=}\FunctionTok{global}\NormalTok{(slanis,}\AttributeTok{fun=}\StringTok{"mean"}\NormalTok{,}\AttributeTok{na.rm=}\ConstantTok{TRUE}\NormalTok{)}
\NormalTok{centrets}\OtherTok{=}\NormalTok{slanis}\SpecialCharTok{{-}}\NormalTok{videjais[,}\DecValTok{1}\NormalTok{]}
\NormalTok{standartnovirze}\OtherTok{=}\NormalTok{terra}\SpecialCharTok{::}\FunctionTok{global}\NormalTok{(centrets,}\AttributeTok{fun=}\StringTok{"rms"}\NormalTok{,}\AttributeTok{na.rm=}\ConstantTok{TRUE}\NormalTok{)}
\NormalTok{merogots}\OtherTok{=}\NormalTok{centrets}\SpecialCharTok{/}\NormalTok{standartnovirze[,}\DecValTok{1}\NormalTok{]}
\FunctionTok{writeRaster}\NormalTok{(merogots,}
      \AttributeTok{filename=}\NormalTok{saglabasanas\_cels,}
      \AttributeTok{overwrite=}\ConstantTok{TRUE}\NormalTok{)}
\end{Highlighting}
\end{Shaded}

\section{Climate\_CHELSAv2.1-lgd\_cell}\label{ch06.046}

\textbf{filename:} \texttt{Climate\_CHELSAv2.1-lgd\_cell.tif}

\textbf{layername:} \texttt{egv\_046}

\textbf{English name:} Last day of the growing season (TREELIM) (CHELSA v2.1) within
the analysis cell (1 ha)

\textbf{Latvian name:} Pēdējā veģetācijas sezonas diena (TREELIM) (CHELSA v2.1)
analīzes šūnā (1 ha)

\textbf{Procedure:} Directly follows \hyperref[Ch04.11]{CHELSA v2.1}. EGV is prepared using
the workflow \texttt{egvtools::downscale2egv()} with inverse distance weighted (power =
2) gap filling and soft smoothing (power = 0.5) over 5 km radius around each cell.
Finally, the layer is standardised by subtracting the arithmetic mean and
dividing by the root mean squared error.

\begin{Shaded}
\begin{Highlighting}[]
\CommentTok{\# libs {-}{-}{-}{-}}
\ControlFlowTok{if}\NormalTok{(}\SpecialCharTok{!}\FunctionTok{require}\NormalTok{(egvtools)) \{remotes}\SpecialCharTok{::}\FunctionTok{install\_github}\NormalTok{(}\StringTok{"aavotins/egvtools"}\NormalTok{); }\FunctionTok{require}\NormalTok{(egvtools)\}}

\CommentTok{\# job {-}{-}{-}{-}}

\NormalTok{localname}\OtherTok{=}\StringTok{"Climate\_CHELSAv2.1{-}lgd\_cell.tif"}
\NormalTok{layername}\OtherTok{=}\StringTok{"egv\_046"}
\NormalTok{reading}\OtherTok{=}\StringTok{"./Geodata/2024/CHELSA/Climate\_CHELSAv2.1{-}lgd\_cell.tif"}

\NormalTok{df }\OtherTok{\textless{}{-}} \FunctionTok{downscale2egv}\NormalTok{(}
 \AttributeTok{template\_path =} \StringTok{"./Templates/TemplateRasters/LV100m\_10km.tif"}\NormalTok{,}
 \AttributeTok{grid\_path   =} \StringTok{"./Templates/TemplateGrids/tikls1km\_sauzeme.parquet"}\NormalTok{,}
 \AttributeTok{rawfile\_path =}\NormalTok{ reading,}
 \AttributeTok{out\_path   =} \StringTok{"./RasterGrids\_100m/2024/RAW/"}\NormalTok{,}
 \AttributeTok{file\_name   =}\NormalTok{ localname,}
 \AttributeTok{layer\_name  =}\NormalTok{ layername,}
 \AttributeTok{fill\_gaps   =} \ConstantTok{TRUE}\NormalTok{,}
 \AttributeTok{smooth    =} \ConstantTok{TRUE}\NormalTok{,}
 \AttributeTok{smooth\_radius\_km =} \DecValTok{5}\NormalTok{,}
 \AttributeTok{plot\_result  =} \ConstantTok{TRUE}\NormalTok{)}
\FunctionTok{print}\NormalTok{(df)}

\CommentTok{\# standardisation {-}{-}{-}{-}}
\ControlFlowTok{if}\NormalTok{(}\SpecialCharTok{!}\FunctionTok{require}\NormalTok{(terra)) \{}\FunctionTok{install.packages}\NormalTok{(}\StringTok{"terra"}\NormalTok{); }\FunctionTok{require}\NormalTok{(terra)\}}
\ControlFlowTok{if}\NormalTok{(}\SpecialCharTok{!}\FunctionTok{require}\NormalTok{(tidyverse)) \{}\FunctionTok{install.packages}\NormalTok{(}\StringTok{"tidyverse"}\NormalTok{); }\FunctionTok{require}\NormalTok{(tidyverse)\}}

\NormalTok{nosaukums}\OtherTok{=}\StringTok{"Climate\_CHELSAv2.1{-}lgd\_cell.tif"}
\NormalTok{ielasisanas\_cels}\OtherTok{=}\FunctionTok{paste0}\NormalTok{(}\StringTok{"./RasterGrids\_100m/2024/RAW/"}\NormalTok{,nosaukums)}
\NormalTok{saglabasanas\_cels}\OtherTok{=}\FunctionTok{paste0}\NormalTok{(}\StringTok{"./RasterGrids\_100m/2024/Scaled/"}\NormalTok{,nosaukums)}
\NormalTok{slanis}\OtherTok{=}\FunctionTok{rast}\NormalTok{(ielasisanas\_cels)}
\NormalTok{videjais}\OtherTok{=}\FunctionTok{global}\NormalTok{(slanis,}\AttributeTok{fun=}\StringTok{"mean"}\NormalTok{,}\AttributeTok{na.rm=}\ConstantTok{TRUE}\NormalTok{)}
\NormalTok{centrets}\OtherTok{=}\NormalTok{slanis}\SpecialCharTok{{-}}\NormalTok{videjais[,}\DecValTok{1}\NormalTok{]}
\NormalTok{standartnovirze}\OtherTok{=}\NormalTok{terra}\SpecialCharTok{::}\FunctionTok{global}\NormalTok{(centrets,}\AttributeTok{fun=}\StringTok{"rms"}\NormalTok{,}\AttributeTok{na.rm=}\ConstantTok{TRUE}\NormalTok{)}
\NormalTok{merogots}\OtherTok{=}\NormalTok{centrets}\SpecialCharTok{/}\NormalTok{standartnovirze[,}\DecValTok{1}\NormalTok{]}
\FunctionTok{writeRaster}\NormalTok{(merogots,}
      \AttributeTok{filename=}\NormalTok{saglabasanas\_cels,}
      \AttributeTok{overwrite=}\ConstantTok{TRUE}\NormalTok{)}
\end{Highlighting}
\end{Shaded}

\section{Climate\_CHELSAv2.1-ngd0\_cell}\label{ch06.047}

\textbf{filename:} \texttt{Climate\_CHELSAv2.1-ngd0\_cell.tif}

\textbf{layername:} \texttt{egv\_047}

\textbf{English name:} Number of days in which air temperature at 2 m is \textgreater{} 0°C (CHELSA
v2.1) within the analysis cell (1 ha)

\textbf{Latvian name:} Dienu skaits, kurā gaisa temperatūra 2 m augstumā pārsniedz
0°C (CHELSA v2.1) analīzes šūnā (1 ha)

\textbf{Procedure:} Directly follows \hyperref[Ch04.11]{CHELSA v2.1}. EGV is prepared using
the workflow \texttt{egvtools::downscale2egv()} with inverse distance weighted (power =
2) gap filling and soft smoothing (power = 0.5) over 5 km radius around each cell.
Finally, the layer is standardised by subtracting the arithmetic mean and
dividing by the root mean squared error.

\begin{Shaded}
\begin{Highlighting}[]
\CommentTok{\# libs {-}{-}{-}{-}}
\ControlFlowTok{if}\NormalTok{(}\SpecialCharTok{!}\FunctionTok{require}\NormalTok{(egvtools)) \{remotes}\SpecialCharTok{::}\FunctionTok{install\_github}\NormalTok{(}\StringTok{"aavotins/egvtools"}\NormalTok{); }\FunctionTok{require}\NormalTok{(egvtools)\}}

\CommentTok{\# job {-}{-}{-}{-}}

\NormalTok{localname}\OtherTok{=}\StringTok{"Climate\_CHELSAv2.1{-}ngd0\_cell.tif"}
\NormalTok{layername}\OtherTok{=}\StringTok{"egv\_047"}
\NormalTok{reading}\OtherTok{=}\StringTok{"./Geodata/2024/CHELSA/Climate\_CHELSAv2.1{-}ngd0\_cell.tif"}

\NormalTok{df }\OtherTok{\textless{}{-}} \FunctionTok{downscale2egv}\NormalTok{(}
 \AttributeTok{template\_path =} \StringTok{"./Templates/TemplateRasters/LV100m\_10km.tif"}\NormalTok{,}
 \AttributeTok{grid\_path   =} \StringTok{"./Templates/TemplateGrids/tikls1km\_sauzeme.parquet"}\NormalTok{,}
 \AttributeTok{rawfile\_path =}\NormalTok{ reading,}
 \AttributeTok{out\_path   =} \StringTok{"./RasterGrids\_100m/2024/RAW/"}\NormalTok{,}
 \AttributeTok{file\_name   =}\NormalTok{ localname,}
 \AttributeTok{layer\_name  =}\NormalTok{ layername,}
 \AttributeTok{fill\_gaps   =} \ConstantTok{TRUE}\NormalTok{,}
 \AttributeTok{smooth    =} \ConstantTok{TRUE}\NormalTok{,}
 \AttributeTok{smooth\_radius\_km =} \DecValTok{5}\NormalTok{,}
 \AttributeTok{plot\_result  =} \ConstantTok{TRUE}\NormalTok{)}
\FunctionTok{print}\NormalTok{(df)}

\CommentTok{\# standardisation {-}{-}{-}{-}}
\ControlFlowTok{if}\NormalTok{(}\SpecialCharTok{!}\FunctionTok{require}\NormalTok{(terra)) \{}\FunctionTok{install.packages}\NormalTok{(}\StringTok{"terra"}\NormalTok{); }\FunctionTok{require}\NormalTok{(terra)\}}
\ControlFlowTok{if}\NormalTok{(}\SpecialCharTok{!}\FunctionTok{require}\NormalTok{(tidyverse)) \{}\FunctionTok{install.packages}\NormalTok{(}\StringTok{"tidyverse"}\NormalTok{); }\FunctionTok{require}\NormalTok{(tidyverse)\}}

\NormalTok{nosaukums}\OtherTok{=}\StringTok{"Climate\_CHELSAv2.1{-}ngd0\_cell.tif"}
\NormalTok{ielasisanas\_cels}\OtherTok{=}\FunctionTok{paste0}\NormalTok{(}\StringTok{"./RasterGrids\_100m/2024/RAW/"}\NormalTok{,nosaukums)}
\NormalTok{saglabasanas\_cels}\OtherTok{=}\FunctionTok{paste0}\NormalTok{(}\StringTok{"./RasterGrids\_100m/2024/Scaled/"}\NormalTok{,nosaukums)}
\NormalTok{slanis}\OtherTok{=}\FunctionTok{rast}\NormalTok{(ielasisanas\_cels)}
\NormalTok{videjais}\OtherTok{=}\FunctionTok{global}\NormalTok{(slanis,}\AttributeTok{fun=}\StringTok{"mean"}\NormalTok{,}\AttributeTok{na.rm=}\ConstantTok{TRUE}\NormalTok{)}
\NormalTok{centrets}\OtherTok{=}\NormalTok{slanis}\SpecialCharTok{{-}}\NormalTok{videjais[,}\DecValTok{1}\NormalTok{]}
\NormalTok{standartnovirze}\OtherTok{=}\NormalTok{terra}\SpecialCharTok{::}\FunctionTok{global}\NormalTok{(centrets,}\AttributeTok{fun=}\StringTok{"rms"}\NormalTok{,}\AttributeTok{na.rm=}\ConstantTok{TRUE}\NormalTok{)}
\NormalTok{merogots}\OtherTok{=}\NormalTok{centrets}\SpecialCharTok{/}\NormalTok{standartnovirze[,}\DecValTok{1}\NormalTok{]}
\FunctionTok{writeRaster}\NormalTok{(merogots,}
      \AttributeTok{filename=}\NormalTok{saglabasanas\_cels,}
      \AttributeTok{overwrite=}\ConstantTok{TRUE}\NormalTok{)}
\end{Highlighting}
\end{Shaded}

\section{Climate\_CHELSAv2.1-ngd10\_cell}\label{ch06.048}

\textbf{filename:} \texttt{Climate\_CHELSAv2.1-ngd10\_cell.tif}

\textbf{layername:} \texttt{egv\_048}

\textbf{English name:} Number of days in which air temperature at 2 m is \textgreater{} 10°C (CHELSA
v2.1) within the analysis cell (1 ha)

\textbf{Latvian name:} Dienu skaits, kurā gaisa temperatūra 2 m augstumā pārsniedz
10°C (CHELSA v2.1) analīzes šūnā (1 ha)

\textbf{Procedure:} Directly follows \hyperref[Ch04.11]{CHELSA v2.1}. EGV is prepared using
the workflow \texttt{egvtools::downscale2egv()} with inverse distance weighted (power =
2) gap filling and soft smoothing (power = 0.5) over 5 km radius around each cell.
Finally, the layer is standardised by subtracting the arithmetic mean and
dividing by the root mean squared error.

\begin{Shaded}
\begin{Highlighting}[]
\CommentTok{\# libs {-}{-}{-}{-}}
\ControlFlowTok{if}\NormalTok{(}\SpecialCharTok{!}\FunctionTok{require}\NormalTok{(egvtools)) \{remotes}\SpecialCharTok{::}\FunctionTok{install\_github}\NormalTok{(}\StringTok{"aavotins/egvtools"}\NormalTok{); }\FunctionTok{require}\NormalTok{(egvtools)\}}

\CommentTok{\# job {-}{-}{-}{-}}

\NormalTok{localname}\OtherTok{=}\StringTok{"Climate\_CHELSAv2.1{-}ngd10\_cell.tif"}
\NormalTok{layername}\OtherTok{=}\StringTok{"egv\_048"}
\NormalTok{reading}\OtherTok{=}\StringTok{"./Geodata/2024/CHELSA/Climate\_CHELSAv2.1{-}ngd10\_cell.tif"}

\NormalTok{df }\OtherTok{\textless{}{-}} \FunctionTok{downscale2egv}\NormalTok{(}
 \AttributeTok{template\_path =} \StringTok{"./Templates/TemplateRasters/LV100m\_10km.tif"}\NormalTok{,}
 \AttributeTok{grid\_path   =} \StringTok{"./Templates/TemplateGrids/tikls1km\_sauzeme.parquet"}\NormalTok{,}
 \AttributeTok{rawfile\_path =}\NormalTok{ reading,}
 \AttributeTok{out\_path   =} \StringTok{"./RasterGrids\_100m/2024/RAW/"}\NormalTok{,}
 \AttributeTok{file\_name   =}\NormalTok{ localname,}
 \AttributeTok{layer\_name  =}\NormalTok{ layername,}
 \AttributeTok{fill\_gaps   =} \ConstantTok{TRUE}\NormalTok{,}
 \AttributeTok{smooth    =} \ConstantTok{TRUE}\NormalTok{,}
 \AttributeTok{smooth\_radius\_km =} \DecValTok{5}\NormalTok{,}
 \AttributeTok{plot\_result  =} \ConstantTok{TRUE}\NormalTok{)}
\FunctionTok{print}\NormalTok{(df)}

\CommentTok{\# standardisation {-}{-}{-}{-}}
\ControlFlowTok{if}\NormalTok{(}\SpecialCharTok{!}\FunctionTok{require}\NormalTok{(terra)) \{}\FunctionTok{install.packages}\NormalTok{(}\StringTok{"terra"}\NormalTok{); }\FunctionTok{require}\NormalTok{(terra)\}}
\ControlFlowTok{if}\NormalTok{(}\SpecialCharTok{!}\FunctionTok{require}\NormalTok{(tidyverse)) \{}\FunctionTok{install.packages}\NormalTok{(}\StringTok{"tidyverse"}\NormalTok{); }\FunctionTok{require}\NormalTok{(tidyverse)\}}

\NormalTok{nosaukums}\OtherTok{=}\StringTok{"Climate\_CHELSAv2.1{-}ngd10\_cell.tif"}
\NormalTok{ielasisanas\_cels}\OtherTok{=}\FunctionTok{paste0}\NormalTok{(}\StringTok{"./RasterGrids\_100m/2024/RAW/"}\NormalTok{,nosaukums)}
\NormalTok{saglabasanas\_cels}\OtherTok{=}\FunctionTok{paste0}\NormalTok{(}\StringTok{"./RasterGrids\_100m/2024/Scaled/"}\NormalTok{,nosaukums)}
\NormalTok{slanis}\OtherTok{=}\FunctionTok{rast}\NormalTok{(ielasisanas\_cels)}
\NormalTok{videjais}\OtherTok{=}\FunctionTok{global}\NormalTok{(slanis,}\AttributeTok{fun=}\StringTok{"mean"}\NormalTok{,}\AttributeTok{na.rm=}\ConstantTok{TRUE}\NormalTok{)}
\NormalTok{centrets}\OtherTok{=}\NormalTok{slanis}\SpecialCharTok{{-}}\NormalTok{videjais[,}\DecValTok{1}\NormalTok{]}
\NormalTok{standartnovirze}\OtherTok{=}\NormalTok{terra}\SpecialCharTok{::}\FunctionTok{global}\NormalTok{(centrets,}\AttributeTok{fun=}\StringTok{"rms"}\NormalTok{,}\AttributeTok{na.rm=}\ConstantTok{TRUE}\NormalTok{)}
\NormalTok{merogots}\OtherTok{=}\NormalTok{centrets}\SpecialCharTok{/}\NormalTok{standartnovirze[,}\DecValTok{1}\NormalTok{]}
\FunctionTok{writeRaster}\NormalTok{(merogots,}
      \AttributeTok{filename=}\NormalTok{saglabasanas\_cels,}
      \AttributeTok{overwrite=}\ConstantTok{TRUE}\NormalTok{)}
\end{Highlighting}
\end{Shaded}

\section{Climate\_CHELSAv2.1-ngd5\_cell}\label{ch06.049}

\textbf{filename:} \texttt{Climate\_CHELSAv2.1-ngd5\_cell.tif}

\textbf{layername:} \texttt{egv\_049}

\textbf{English name:} Number of days in which air temperature at 2 m is \textgreater{} 5°C (CHELSA
v2.1) within the analysis cell (1 ha)

\textbf{Latvian name:} Dienu skaits, kurā gaisa temperatūra 2 m augstumā pārsniedz
5°C (CHELSA v2.1) analīzes šūnā (1 ha)

\textbf{Procedure:} Directly follows \hyperref[Ch04.11]{CHELSA v2.1}. EGV is prepared using
the workflow \texttt{egvtools::downscale2egv()} with inverse distance weighted (power =
2) gap filling and soft smoothing (power = 0.5) over 5 km radius around each cell.
Finally, the layer is standardised by subtracting the arithmetic mean and
dividing by the root mean squared error.

\begin{Shaded}
\begin{Highlighting}[]
\CommentTok{\# libs {-}{-}{-}{-}}
\ControlFlowTok{if}\NormalTok{(}\SpecialCharTok{!}\FunctionTok{require}\NormalTok{(egvtools)) \{remotes}\SpecialCharTok{::}\FunctionTok{install\_github}\NormalTok{(}\StringTok{"aavotins/egvtools"}\NormalTok{); }\FunctionTok{require}\NormalTok{(egvtools)\}}

\CommentTok{\# job {-}{-}{-}{-}}

\NormalTok{localname}\OtherTok{=}\StringTok{"Climate\_CHELSAv2.1{-}ngd5\_cell.tif"}
\NormalTok{layername}\OtherTok{=}\StringTok{"egv\_049"}
\NormalTok{reading}\OtherTok{=}\StringTok{"./Geodata/2024/CHELSA/Climate\_CHELSAv2.1{-}ngd5\_cell.tif"}

\NormalTok{df }\OtherTok{\textless{}{-}} \FunctionTok{downscale2egv}\NormalTok{(}
 \AttributeTok{template\_path =} \StringTok{"./Templates/TemplateRasters/LV100m\_10km.tif"}\NormalTok{,}
 \AttributeTok{grid\_path   =} \StringTok{"./Templates/TemplateGrids/tikls1km\_sauzeme.parquet"}\NormalTok{,}
 \AttributeTok{rawfile\_path =}\NormalTok{ reading,}
 \AttributeTok{out\_path   =} \StringTok{"./RasterGrids\_100m/2024/RAW/"}\NormalTok{,}
 \AttributeTok{file\_name   =}\NormalTok{ localname,}
 \AttributeTok{layer\_name  =}\NormalTok{ layername,}
 \AttributeTok{fill\_gaps   =} \ConstantTok{TRUE}\NormalTok{,}
 \AttributeTok{smooth    =} \ConstantTok{TRUE}\NormalTok{,}
 \AttributeTok{smooth\_radius\_km =} \DecValTok{5}\NormalTok{,}
 \AttributeTok{plot\_result  =} \ConstantTok{TRUE}\NormalTok{)}
\FunctionTok{print}\NormalTok{(df)}

\CommentTok{\# standardisation {-}{-}{-}{-}}
\ControlFlowTok{if}\NormalTok{(}\SpecialCharTok{!}\FunctionTok{require}\NormalTok{(terra)) \{}\FunctionTok{install.packages}\NormalTok{(}\StringTok{"terra"}\NormalTok{); }\FunctionTok{require}\NormalTok{(terra)\}}
\ControlFlowTok{if}\NormalTok{(}\SpecialCharTok{!}\FunctionTok{require}\NormalTok{(tidyverse)) \{}\FunctionTok{install.packages}\NormalTok{(}\StringTok{"tidyverse"}\NormalTok{); }\FunctionTok{require}\NormalTok{(tidyverse)\}}

\NormalTok{nosaukums}\OtherTok{=}\StringTok{"Climate\_CHELSAv2.1{-}ngd5\_cell.tif"}
\NormalTok{ielasisanas\_cels}\OtherTok{=}\FunctionTok{paste0}\NormalTok{(}\StringTok{"./RasterGrids\_100m/2024/RAW/"}\NormalTok{,nosaukums)}
\NormalTok{saglabasanas\_cels}\OtherTok{=}\FunctionTok{paste0}\NormalTok{(}\StringTok{"./RasterGrids\_100m/2024/Scaled/"}\NormalTok{,nosaukums)}
\NormalTok{slanis}\OtherTok{=}\FunctionTok{rast}\NormalTok{(ielasisanas\_cels)}
\NormalTok{videjais}\OtherTok{=}\FunctionTok{global}\NormalTok{(slanis,}\AttributeTok{fun=}\StringTok{"mean"}\NormalTok{,}\AttributeTok{na.rm=}\ConstantTok{TRUE}\NormalTok{)}
\NormalTok{centrets}\OtherTok{=}\NormalTok{slanis}\SpecialCharTok{{-}}\NormalTok{videjais[,}\DecValTok{1}\NormalTok{]}
\NormalTok{standartnovirze}\OtherTok{=}\NormalTok{terra}\SpecialCharTok{::}\FunctionTok{global}\NormalTok{(centrets,}\AttributeTok{fun=}\StringTok{"rms"}\NormalTok{,}\AttributeTok{na.rm=}\ConstantTok{TRUE}\NormalTok{)}
\NormalTok{merogots}\OtherTok{=}\NormalTok{centrets}\SpecialCharTok{/}\NormalTok{standartnovirze[,}\DecValTok{1}\NormalTok{]}
\FunctionTok{writeRaster}\NormalTok{(merogots,}
      \AttributeTok{filename=}\NormalTok{saglabasanas\_cels,}
      \AttributeTok{overwrite=}\ConstantTok{TRUE}\NormalTok{)}
\end{Highlighting}
\end{Shaded}

\section{Climate\_CHELSAv2.1-npp\_cell}\label{ch06.050}

\textbf{filename:} \texttt{Climate\_CHELSAv2.1-npp\_cell.tif}

\textbf{layername:} \texttt{egv\_050}

\textbf{English name:} Net primary productivity (g C m⁻² year⁻¹) (CHELSA v2.1) within
the analysis cell (1 ha)

\textbf{Latvian name:} Neto primārā produkcija (g C m⁻² year⁻¹) (CHELSA v2.1)
analīzes šūnā (1 ha)

\textbf{Procedure:} Directly follows \hyperref[Ch04.11]{CHELSA v2.1}. EGV is prepared using
the workflow \texttt{egvtools::downscale2egv()} with inverse distance weighted (power =
2) gap filling and soft smoothing (power = 0.5) over 5 km radius around each cell.
Finally, the layer is standardised by subtracting the arithmetic mean and
dividing by the root mean squared error.

\begin{Shaded}
\begin{Highlighting}[]
\CommentTok{\# libs {-}{-}{-}{-}}
\ControlFlowTok{if}\NormalTok{(}\SpecialCharTok{!}\FunctionTok{require}\NormalTok{(egvtools)) \{remotes}\SpecialCharTok{::}\FunctionTok{install\_github}\NormalTok{(}\StringTok{"aavotins/egvtools"}\NormalTok{); }\FunctionTok{require}\NormalTok{(egvtools)\}}

\CommentTok{\# job {-}{-}{-}{-}}

\NormalTok{localname}\OtherTok{=}\StringTok{"Climate\_CHELSAv2.1{-}npp\_cell.tif"}
\NormalTok{layername}\OtherTok{=}\StringTok{"egv\_050"}
\NormalTok{reading}\OtherTok{=}\StringTok{"./Geodata/2024/CHELSA/Climate\_CHELSAv2.1{-}npp\_cell.tif"}

\NormalTok{df }\OtherTok{\textless{}{-}} \FunctionTok{downscale2egv}\NormalTok{(}
 \AttributeTok{template\_path =} \StringTok{"./Templates/TemplateRasters/LV100m\_10km.tif"}\NormalTok{,}
 \AttributeTok{grid\_path   =} \StringTok{"./Templates/TemplateGrids/tikls1km\_sauzeme.parquet"}\NormalTok{,}
 \AttributeTok{rawfile\_path =}\NormalTok{ reading,}
 \AttributeTok{out\_path   =} \StringTok{"./RasterGrids\_100m/2024/RAW/"}\NormalTok{,}
 \AttributeTok{file\_name   =}\NormalTok{ localname,}
 \AttributeTok{layer\_name  =}\NormalTok{ layername,}
 \AttributeTok{fill\_gaps   =} \ConstantTok{TRUE}\NormalTok{,}
 \AttributeTok{smooth    =} \ConstantTok{TRUE}\NormalTok{,}
 \AttributeTok{smooth\_radius\_km =} \DecValTok{5}\NormalTok{,}
 \AttributeTok{plot\_result  =} \ConstantTok{TRUE}\NormalTok{)}
\FunctionTok{print}\NormalTok{(df)}

\CommentTok{\# standardisation {-}{-}{-}{-}}
\ControlFlowTok{if}\NormalTok{(}\SpecialCharTok{!}\FunctionTok{require}\NormalTok{(terra)) \{}\FunctionTok{install.packages}\NormalTok{(}\StringTok{"terra"}\NormalTok{); }\FunctionTok{require}\NormalTok{(terra)\}}
\ControlFlowTok{if}\NormalTok{(}\SpecialCharTok{!}\FunctionTok{require}\NormalTok{(tidyverse)) \{}\FunctionTok{install.packages}\NormalTok{(}\StringTok{"tidyverse"}\NormalTok{); }\FunctionTok{require}\NormalTok{(tidyverse)\}}

\NormalTok{nosaukums}\OtherTok{=}\StringTok{"Climate\_CHELSAv2.1{-}npp\_cell.tif"}
\NormalTok{ielasisanas\_cels}\OtherTok{=}\FunctionTok{paste0}\NormalTok{(}\StringTok{"./RasterGrids\_100m/2024/RAW/"}\NormalTok{,nosaukums)}
\NormalTok{saglabasanas\_cels}\OtherTok{=}\FunctionTok{paste0}\NormalTok{(}\StringTok{"./RasterGrids\_100m/2024/Scaled/"}\NormalTok{,nosaukums)}
\NormalTok{slanis}\OtherTok{=}\FunctionTok{rast}\NormalTok{(ielasisanas\_cels)}
\NormalTok{videjais}\OtherTok{=}\FunctionTok{global}\NormalTok{(slanis,}\AttributeTok{fun=}\StringTok{"mean"}\NormalTok{,}\AttributeTok{na.rm=}\ConstantTok{TRUE}\NormalTok{)}
\NormalTok{centrets}\OtherTok{=}\NormalTok{slanis}\SpecialCharTok{{-}}\NormalTok{videjais[,}\DecValTok{1}\NormalTok{]}
\NormalTok{standartnovirze}\OtherTok{=}\NormalTok{terra}\SpecialCharTok{::}\FunctionTok{global}\NormalTok{(centrets,}\AttributeTok{fun=}\StringTok{"rms"}\NormalTok{,}\AttributeTok{na.rm=}\ConstantTok{TRUE}\NormalTok{)}
\NormalTok{merogots}\OtherTok{=}\NormalTok{centrets}\SpecialCharTok{/}\NormalTok{standartnovirze[,}\DecValTok{1}\NormalTok{]}
\FunctionTok{writeRaster}\NormalTok{(merogots,}
      \AttributeTok{filename=}\NormalTok{saglabasanas\_cels,}
      \AttributeTok{overwrite=}\ConstantTok{TRUE}\NormalTok{)}
\end{Highlighting}
\end{Shaded}

\section{Climate\_CHELSAv2.1-pet-penman-max\_cell}\label{ch06.051}

\textbf{filename:} \texttt{Climate\_CHELSAv2.1-pet-penman-max\_cell.tif}

\textbf{layername:} \texttt{egv\_051}

\textbf{English name:} Mean of monthly maximum potential evapotranspiration (kg m⁻² month⁻¹)
(CHELSA v2.1) within the analysis cell (1 ha)

\textbf{Latvian name:} Vidējā ik mēneša maksimālā potenciālā evapotranspirācija (kg m⁻²
month⁻¹) (CHELSA v2.1) analīzes šūnā (1 ha)

\textbf{Procedure:} Directly follows \hyperref[Ch04.11]{CHELSA v2.1}. EGV is prepared using
the workflow \texttt{egvtools::downscale2egv()} with inverse distance weighted (power =
2) gap filling and soft smoothing (power = 0.5) over 5 km radius around each cell.
Finally, the layer is standardised by subtracting the arithmetic mean and
dividing by the root mean squared error.

\begin{Shaded}
\begin{Highlighting}[]
\CommentTok{\# libs {-}{-}{-}{-}}
\ControlFlowTok{if}\NormalTok{(}\SpecialCharTok{!}\FunctionTok{require}\NormalTok{(egvtools)) \{remotes}\SpecialCharTok{::}\FunctionTok{install\_github}\NormalTok{(}\StringTok{"aavotins/egvtools"}\NormalTok{); }\FunctionTok{require}\NormalTok{(egvtools)\}}

\CommentTok{\# job {-}{-}{-}{-}}

\NormalTok{localname}\OtherTok{=}\StringTok{"Climate\_CHELSAv2.1{-}pet{-}penman{-}max\_cell.tif"}
\NormalTok{layername}\OtherTok{=}\StringTok{"egv\_051"}
\NormalTok{reading}\OtherTok{=}\StringTok{"./Geodata/2024/CHELSA/Climate\_CHELSAv2.1{-}pet{-}penman{-}max\_cell.tif"}

\NormalTok{df }\OtherTok{\textless{}{-}} \FunctionTok{downscale2egv}\NormalTok{(}
 \AttributeTok{template\_path =} \StringTok{"./Templates/TemplateRasters/LV100m\_10km.tif"}\NormalTok{,}
 \AttributeTok{grid\_path   =} \StringTok{"./Templates/TemplateGrids/tikls1km\_sauzeme.parquet"}\NormalTok{,}
 \AttributeTok{rawfile\_path =}\NormalTok{ reading,}
 \AttributeTok{out\_path   =} \StringTok{"./RasterGrids\_100m/2024/RAW/"}\NormalTok{,}
 \AttributeTok{file\_name   =}\NormalTok{ localname,}
 \AttributeTok{layer\_name  =}\NormalTok{ layername,}
 \AttributeTok{fill\_gaps   =} \ConstantTok{TRUE}\NormalTok{,}
 \AttributeTok{smooth    =} \ConstantTok{TRUE}\NormalTok{,}
 \AttributeTok{smooth\_radius\_km =} \DecValTok{5}\NormalTok{,}
 \AttributeTok{plot\_result  =} \ConstantTok{TRUE}\NormalTok{)}
\FunctionTok{print}\NormalTok{(df)}

\CommentTok{\# standardisation {-}{-}{-}{-}}
\ControlFlowTok{if}\NormalTok{(}\SpecialCharTok{!}\FunctionTok{require}\NormalTok{(terra)) \{}\FunctionTok{install.packages}\NormalTok{(}\StringTok{"terra"}\NormalTok{); }\FunctionTok{require}\NormalTok{(terra)\}}
\ControlFlowTok{if}\NormalTok{(}\SpecialCharTok{!}\FunctionTok{require}\NormalTok{(tidyverse)) \{}\FunctionTok{install.packages}\NormalTok{(}\StringTok{"tidyverse"}\NormalTok{); }\FunctionTok{require}\NormalTok{(tidyverse)\}}

\NormalTok{nosaukums}\OtherTok{=}\StringTok{"Climate\_CHELSAv2.1{-}pet{-}penman{-}max\_cell.tif"}
\NormalTok{ielasisanas\_cels}\OtherTok{=}\FunctionTok{paste0}\NormalTok{(}\StringTok{"./RasterGrids\_100m/2024/RAW/"}\NormalTok{,nosaukums)}
\NormalTok{saglabasanas\_cels}\OtherTok{=}\FunctionTok{paste0}\NormalTok{(}\StringTok{"./RasterGrids\_100m/2024/Scaled/"}\NormalTok{,nosaukums)}
\NormalTok{slanis}\OtherTok{=}\FunctionTok{rast}\NormalTok{(ielasisanas\_cels)}
\NormalTok{videjais}\OtherTok{=}\FunctionTok{global}\NormalTok{(slanis,}\AttributeTok{fun=}\StringTok{"mean"}\NormalTok{,}\AttributeTok{na.rm=}\ConstantTok{TRUE}\NormalTok{)}
\NormalTok{centrets}\OtherTok{=}\NormalTok{slanis}\SpecialCharTok{{-}}\NormalTok{videjais[,}\DecValTok{1}\NormalTok{]}
\NormalTok{standartnovirze}\OtherTok{=}\NormalTok{terra}\SpecialCharTok{::}\FunctionTok{global}\NormalTok{(centrets,}\AttributeTok{fun=}\StringTok{"rms"}\NormalTok{,}\AttributeTok{na.rm=}\ConstantTok{TRUE}\NormalTok{)}
\NormalTok{merogots}\OtherTok{=}\NormalTok{centrets}\SpecialCharTok{/}\NormalTok{standartnovirze[,}\DecValTok{1}\NormalTok{]}
\FunctionTok{writeRaster}\NormalTok{(merogots,}
      \AttributeTok{filename=}\NormalTok{saglabasanas\_cels,}
      \AttributeTok{overwrite=}\ConstantTok{TRUE}\NormalTok{)}
\end{Highlighting}
\end{Shaded}

\section{Climate\_CHELSAv2.1-pet-penman-mean\_cell}\label{ch06.052}

\textbf{filename:} \texttt{Climate\_CHELSAv2.1-pet-penman-mean\_cell.tif}

\textbf{layername:} \texttt{egv\_052}

\textbf{English name:} Mean of monthly mean potential evapotranspiration (kg m⁻² month⁻¹)
(CHELSA v2.1) within the analysis cell (1 ha)

\textbf{Latvian name:} Vidējā ik mēneša vidējā potenciālā evapotranspirācija (kg m⁻² month⁻¹)
(CHELSA v2.1) analīzes šūnā (1 ha)

\textbf{Procedure:} Directly follows \hyperref[Ch04.11]{CHELSA v2.1}. EGV is prepared using
the workflow \texttt{egvtools::downscale2egv()} with inverse distance weighted (power =
2) gap filling and soft smoothing (power = 0.5) over 5 km radius around each cell.
Finally, the layer is standardised by subtracting the arithmetic mean and
dividing by the root mean squared error.

\begin{Shaded}
\begin{Highlighting}[]
\CommentTok{\# libs {-}{-}{-}{-}}
\ControlFlowTok{if}\NormalTok{(}\SpecialCharTok{!}\FunctionTok{require}\NormalTok{(egvtools)) \{remotes}\SpecialCharTok{::}\FunctionTok{install\_github}\NormalTok{(}\StringTok{"aavotins/egvtools"}\NormalTok{); }\FunctionTok{require}\NormalTok{(egvtools)\}}

\CommentTok{\# job {-}{-}{-}{-}}

\NormalTok{localname}\OtherTok{=}\StringTok{"Climate\_CHELSAv2.1{-}pet{-}penman{-}mean\_cell.tif"}
\NormalTok{layername}\OtherTok{=}\StringTok{"egv\_052"}
\NormalTok{reading}\OtherTok{=}\StringTok{"./Geodata/2024/CHELSA/Climate\_CHELSAv2.1{-}pet{-}penman{-}mean\_cell.tif"}

\NormalTok{df }\OtherTok{\textless{}{-}} \FunctionTok{downscale2egv}\NormalTok{(}
 \AttributeTok{template\_path =} \StringTok{"./Templates/TemplateRasters/LV100m\_10km.tif"}\NormalTok{,}
 \AttributeTok{grid\_path   =} \StringTok{"./Templates/TemplateGrids/tikls1km\_sauzeme.parquet"}\NormalTok{,}
 \AttributeTok{rawfile\_path =}\NormalTok{ reading,}
 \AttributeTok{out\_path   =} \StringTok{"./RasterGrids\_100m/2024/RAW/"}\NormalTok{,}
 \AttributeTok{file\_name   =}\NormalTok{ localname,}
 \AttributeTok{layer\_name  =}\NormalTok{ layername,}
 \AttributeTok{fill\_gaps   =} \ConstantTok{TRUE}\NormalTok{,}
 \AttributeTok{smooth    =} \ConstantTok{TRUE}\NormalTok{,}
 \AttributeTok{smooth\_radius\_km =} \DecValTok{5}\NormalTok{,}
 \AttributeTok{plot\_result  =} \ConstantTok{TRUE}\NormalTok{)}
\FunctionTok{print}\NormalTok{(df)}

\CommentTok{\# standardisation {-}{-}{-}{-}}
\ControlFlowTok{if}\NormalTok{(}\SpecialCharTok{!}\FunctionTok{require}\NormalTok{(terra)) \{}\FunctionTok{install.packages}\NormalTok{(}\StringTok{"terra"}\NormalTok{); }\FunctionTok{require}\NormalTok{(terra)\}}
\ControlFlowTok{if}\NormalTok{(}\SpecialCharTok{!}\FunctionTok{require}\NormalTok{(tidyverse)) \{}\FunctionTok{install.packages}\NormalTok{(}\StringTok{"tidyverse"}\NormalTok{); }\FunctionTok{require}\NormalTok{(tidyverse)\}}

\NormalTok{nosaukums}\OtherTok{=}\StringTok{"Climate\_CHELSAv2.1{-}pet{-}penman{-}mean\_cell.tif"}
\NormalTok{ielasisanas\_cels}\OtherTok{=}\FunctionTok{paste0}\NormalTok{(}\StringTok{"./RasterGrids\_100m/2024/RAW/"}\NormalTok{,nosaukums)}
\NormalTok{saglabasanas\_cels}\OtherTok{=}\FunctionTok{paste0}\NormalTok{(}\StringTok{"./RasterGrids\_100m/2024/Scaled/"}\NormalTok{,nosaukums)}
\NormalTok{slanis}\OtherTok{=}\FunctionTok{rast}\NormalTok{(ielasisanas\_cels)}
\NormalTok{videjais}\OtherTok{=}\FunctionTok{global}\NormalTok{(slanis,}\AttributeTok{fun=}\StringTok{"mean"}\NormalTok{,}\AttributeTok{na.rm=}\ConstantTok{TRUE}\NormalTok{)}
\NormalTok{centrets}\OtherTok{=}\NormalTok{slanis}\SpecialCharTok{{-}}\NormalTok{videjais[,}\DecValTok{1}\NormalTok{]}
\NormalTok{standartnovirze}\OtherTok{=}\NormalTok{terra}\SpecialCharTok{::}\FunctionTok{global}\NormalTok{(centrets,}\AttributeTok{fun=}\StringTok{"rms"}\NormalTok{,}\AttributeTok{na.rm=}\ConstantTok{TRUE}\NormalTok{)}
\NormalTok{merogots}\OtherTok{=}\NormalTok{centrets}\SpecialCharTok{/}\NormalTok{standartnovirze[,}\DecValTok{1}\NormalTok{]}
\FunctionTok{writeRaster}\NormalTok{(merogots,}
      \AttributeTok{filename=}\NormalTok{saglabasanas\_cels,}
      \AttributeTok{overwrite=}\ConstantTok{TRUE}\NormalTok{)}
\end{Highlighting}
\end{Shaded}

\section{Climate\_CHELSAv2.1-pet-penman-min\_cell}\label{ch06.053}

\textbf{filename:} \texttt{Climate\_CHELSAv2.1-pet-penman-min\_cell.tif}

\textbf{layername:} \texttt{egv\_053}

\textbf{English name:} Mean of monthly minimum potential evapotranspiration (kg m⁻² month⁻¹)
(CHELSA v2.1) within the analysis cell (1 ha)

\textbf{Latvian name:} Vidējā ik mēneša minimālā potenciālā evapotranspirācija (kg m⁻²
month⁻¹) (CHELSA v2.1) analīzes šūnā (1 ha)

\textbf{Procedure:} Directly follows \hyperref[Ch04.11]{CHELSA v2.1}. EGV is prepared using
the workflow \texttt{egvtools::downscale2egv()} with inverse distance weighted (power =
2) gap filling and soft smoothing (power = 0.5) over 5 km radius around each cell.
Finally, the layer is standardised by subtracting the arithmetic mean and
dividing by the root mean squared error.

\begin{Shaded}
\begin{Highlighting}[]
\CommentTok{\# libs {-}{-}{-}{-}}
\ControlFlowTok{if}\NormalTok{(}\SpecialCharTok{!}\FunctionTok{require}\NormalTok{(egvtools)) \{remotes}\SpecialCharTok{::}\FunctionTok{install\_github}\NormalTok{(}\StringTok{"aavotins/egvtools"}\NormalTok{); }\FunctionTok{require}\NormalTok{(egvtools)\}}

\CommentTok{\# job {-}{-}{-}{-}}

\NormalTok{localname}\OtherTok{=}\StringTok{"Climate\_CHELSAv2.1{-}pet{-}penman{-}min\_cell.tif"}
\NormalTok{layername}\OtherTok{=}\StringTok{"egv\_053"}
\NormalTok{reading}\OtherTok{=}\StringTok{"./Geodata/2024/CHELSA/Climate\_CHELSAv2.1{-}pet{-}penman{-}min\_cell.tif"}

\NormalTok{df }\OtherTok{\textless{}{-}} \FunctionTok{downscale2egv}\NormalTok{(}
 \AttributeTok{template\_path =} \StringTok{"./Templates/TemplateRasters/LV100m\_10km.tif"}\NormalTok{,}
 \AttributeTok{grid\_path   =} \StringTok{"./Templates/TemplateGrids/tikls1km\_sauzeme.parquet"}\NormalTok{,}
 \AttributeTok{rawfile\_path =}\NormalTok{ reading,}
 \AttributeTok{out\_path   =} \StringTok{"./RasterGrids\_100m/2024/RAW/"}\NormalTok{,}
 \AttributeTok{file\_name   =}\NormalTok{ localname,}
 \AttributeTok{layer\_name  =}\NormalTok{ layername,}
 \AttributeTok{fill\_gaps   =} \ConstantTok{TRUE}\NormalTok{,}
 \AttributeTok{smooth    =} \ConstantTok{TRUE}\NormalTok{,}
 \AttributeTok{smooth\_radius\_km =} \DecValTok{5}\NormalTok{,}
 \AttributeTok{plot\_result  =} \ConstantTok{TRUE}\NormalTok{)}
\FunctionTok{print}\NormalTok{(df)}

\CommentTok{\# standardisation {-}{-}{-}{-}}
\ControlFlowTok{if}\NormalTok{(}\SpecialCharTok{!}\FunctionTok{require}\NormalTok{(terra)) \{}\FunctionTok{install.packages}\NormalTok{(}\StringTok{"terra"}\NormalTok{); }\FunctionTok{require}\NormalTok{(terra)\}}
\ControlFlowTok{if}\NormalTok{(}\SpecialCharTok{!}\FunctionTok{require}\NormalTok{(tidyverse)) \{}\FunctionTok{install.packages}\NormalTok{(}\StringTok{"tidyverse"}\NormalTok{); }\FunctionTok{require}\NormalTok{(tidyverse)\}}

\NormalTok{nosaukums}\OtherTok{=}\StringTok{"Climate\_CHELSAv2.1{-}pet{-}penman{-}min\_cell.tif"}
\NormalTok{ielasisanas\_cels}\OtherTok{=}\FunctionTok{paste0}\NormalTok{(}\StringTok{"./RasterGrids\_100m/2024/RAW/"}\NormalTok{,nosaukums)}
\NormalTok{saglabasanas\_cels}\OtherTok{=}\FunctionTok{paste0}\NormalTok{(}\StringTok{"./RasterGrids\_100m/2024/Scaled/"}\NormalTok{,nosaukums)}
\NormalTok{slanis}\OtherTok{=}\FunctionTok{rast}\NormalTok{(ielasisanas\_cels)}
\NormalTok{videjais}\OtherTok{=}\FunctionTok{global}\NormalTok{(slanis,}\AttributeTok{fun=}\StringTok{"mean"}\NormalTok{,}\AttributeTok{na.rm=}\ConstantTok{TRUE}\NormalTok{)}
\NormalTok{centrets}\OtherTok{=}\NormalTok{slanis}\SpecialCharTok{{-}}\NormalTok{videjais[,}\DecValTok{1}\NormalTok{]}
\NormalTok{standartnovirze}\OtherTok{=}\NormalTok{terra}\SpecialCharTok{::}\FunctionTok{global}\NormalTok{(centrets,}\AttributeTok{fun=}\StringTok{"rms"}\NormalTok{,}\AttributeTok{na.rm=}\ConstantTok{TRUE}\NormalTok{)}
\NormalTok{merogots}\OtherTok{=}\NormalTok{centrets}\SpecialCharTok{/}\NormalTok{standartnovirze[,}\DecValTok{1}\NormalTok{]}
\FunctionTok{writeRaster}\NormalTok{(merogots,}
      \AttributeTok{filename=}\NormalTok{saglabasanas\_cels,}
      \AttributeTok{overwrite=}\ConstantTok{TRUE}\NormalTok{)}
\end{Highlighting}
\end{Shaded}

\section{Climate\_CHELSAv2.1-pet-penman-range\_cell}\label{ch06.054}

\textbf{filename:} \texttt{Climate\_CHELSAv2.1-pet-penman-range\_cell.tif}

\textbf{layername:} \texttt{egv\_054}

\textbf{English name:} Annual range of monthly potential evapotranspiration (kg m⁻²
month⁻¹) (CHELSA v2.1) within the analysis cell (1 ha)

\textbf{Latvian name:} Gada potenciālās evapotranspirācijas amplitūda (kg m⁻² month⁻¹)
(CHELSA v2.1) analīzes šūnā (1 ha)

\textbf{Procedure:} Directly follows \hyperref[Ch04.11]{CHELSA v2.1}. EGV is prepared using
the workflow \texttt{egvtools::downscale2egv()} with inverse distance weighted (power =
2) gap filling and soft smoothing (power = 0.5) over 5 km radius around each cell.
Finally, the layer is standardised by subtracting the arithmetic mean and
dividing by the root mean squared error.

\begin{Shaded}
\begin{Highlighting}[]
\CommentTok{\# libs {-}{-}{-}{-}}
\ControlFlowTok{if}\NormalTok{(}\SpecialCharTok{!}\FunctionTok{require}\NormalTok{(egvtools)) \{remotes}\SpecialCharTok{::}\FunctionTok{install\_github}\NormalTok{(}\StringTok{"aavotins/egvtools"}\NormalTok{); }\FunctionTok{require}\NormalTok{(egvtools)\}}

\CommentTok{\# job {-}{-}{-}{-}}

\NormalTok{localname}\OtherTok{=}\StringTok{"Climate\_CHELSAv2.1{-}pet{-}penman{-}range\_cell.tif"}
\NormalTok{layername}\OtherTok{=}\StringTok{"egv\_054"}
\NormalTok{reading}\OtherTok{=}\StringTok{"./Geodata/2024/CHELSA/Climate\_CHELSAv2.1{-}pet{-}penman{-}range\_cell.tif"}

\NormalTok{df }\OtherTok{\textless{}{-}} \FunctionTok{downscale2egv}\NormalTok{(}
 \AttributeTok{template\_path =} \StringTok{"./Templates/TemplateRasters/LV100m\_10km.tif"}\NormalTok{,}
 \AttributeTok{grid\_path   =} \StringTok{"./Templates/TemplateGrids/tikls1km\_sauzeme.parquet"}\NormalTok{,}
 \AttributeTok{rawfile\_path =}\NormalTok{ reading,}
 \AttributeTok{out\_path   =} \StringTok{"./RasterGrids\_100m/2024/RAW/"}\NormalTok{,}
 \AttributeTok{file\_name   =}\NormalTok{ localname,}
 \AttributeTok{layer\_name  =}\NormalTok{ layername,}
 \AttributeTok{fill\_gaps   =} \ConstantTok{TRUE}\NormalTok{,}
 \AttributeTok{smooth    =} \ConstantTok{TRUE}\NormalTok{,}
 \AttributeTok{smooth\_radius\_km =} \DecValTok{5}\NormalTok{,}
 \AttributeTok{plot\_result  =} \ConstantTok{TRUE}\NormalTok{)}
\FunctionTok{print}\NormalTok{(df)}

\CommentTok{\# standardisation {-}{-}{-}{-}}
\ControlFlowTok{if}\NormalTok{(}\SpecialCharTok{!}\FunctionTok{require}\NormalTok{(terra)) \{}\FunctionTok{install.packages}\NormalTok{(}\StringTok{"terra"}\NormalTok{); }\FunctionTok{require}\NormalTok{(terra)\}}
\ControlFlowTok{if}\NormalTok{(}\SpecialCharTok{!}\FunctionTok{require}\NormalTok{(tidyverse)) \{}\FunctionTok{install.packages}\NormalTok{(}\StringTok{"tidyverse"}\NormalTok{); }\FunctionTok{require}\NormalTok{(tidyverse)\}}

\NormalTok{nosaukums}\OtherTok{=}\StringTok{"Climate\_CHELSAv2.1{-}pet{-}penman{-}range\_cell.tif"}
\NormalTok{ielasisanas\_cels}\OtherTok{=}\FunctionTok{paste0}\NormalTok{(}\StringTok{"./RasterGrids\_100m/2024/RAW/"}\NormalTok{,nosaukums)}
\NormalTok{saglabasanas\_cels}\OtherTok{=}\FunctionTok{paste0}\NormalTok{(}\StringTok{"./RasterGrids\_100m/2024/Scaled/"}\NormalTok{,nosaukums)}
\NormalTok{slanis}\OtherTok{=}\FunctionTok{rast}\NormalTok{(ielasisanas\_cels)}
\NormalTok{videjais}\OtherTok{=}\FunctionTok{global}\NormalTok{(slanis,}\AttributeTok{fun=}\StringTok{"mean"}\NormalTok{,}\AttributeTok{na.rm=}\ConstantTok{TRUE}\NormalTok{)}
\NormalTok{centrets}\OtherTok{=}\NormalTok{slanis}\SpecialCharTok{{-}}\NormalTok{videjais[,}\DecValTok{1}\NormalTok{]}
\NormalTok{standartnovirze}\OtherTok{=}\NormalTok{terra}\SpecialCharTok{::}\FunctionTok{global}\NormalTok{(centrets,}\AttributeTok{fun=}\StringTok{"rms"}\NormalTok{,}\AttributeTok{na.rm=}\ConstantTok{TRUE}\NormalTok{)}
\NormalTok{merogots}\OtherTok{=}\NormalTok{centrets}\SpecialCharTok{/}\NormalTok{standartnovirze[,}\DecValTok{1}\NormalTok{]}
\FunctionTok{writeRaster}\NormalTok{(merogots,}
      \AttributeTok{filename=}\NormalTok{saglabasanas\_cels,}
      \AttributeTok{overwrite=}\ConstantTok{TRUE}\NormalTok{)}
\end{Highlighting}
\end{Shaded}

\section{Climate\_CHELSAv2.1-rsds-max\_cell}\label{ch06.055}

\textbf{filename:} \texttt{Climate\_CHELSAv2.1-rsds-max\_cell.tif}

\textbf{layername:} \texttt{egv\_055}

\textbf{English name:} Mean of monthly maximum surface downwelling shortwave flux in air (MJ
m⁻² d⁻¹) (CHELSA v2.1) within the analysis cell (1 ha)

\textbf{Latvian name:} Vidējā ik mēneša maksimālā Zemes virsmu sasniedzošā saules
radiācija (MJ m⁻² d⁻¹) (CHELSA v2.1) analīzes šūnā (1 ha)

\textbf{Procedure:} Directly follows \hyperref[Ch04.11]{CHELSA v2.1}. EGV is prepared using
the workflow \texttt{egvtools::downscale2egv()} with inverse distance weighted (power =
2) gap filling and soft smoothing (power = 0.5) over 5 km radius around each cell.
Finally, the layer is standardised by subtracting the arithmetic mean and
dividing by the root mean squared error.

\begin{Shaded}
\begin{Highlighting}[]
\CommentTok{\# libs {-}{-}{-}{-}}
\ControlFlowTok{if}\NormalTok{(}\SpecialCharTok{!}\FunctionTok{require}\NormalTok{(egvtools)) \{remotes}\SpecialCharTok{::}\FunctionTok{install\_github}\NormalTok{(}\StringTok{"aavotins/egvtools"}\NormalTok{); }\FunctionTok{require}\NormalTok{(egvtools)\}}

\CommentTok{\# job {-}{-}{-}{-}}

\NormalTok{localname}\OtherTok{=}\StringTok{"Climate\_CHELSAv2.1{-}rsds{-}max\_cell.tif"}
\NormalTok{layername}\OtherTok{=}\StringTok{"egv\_055"}
\NormalTok{reading}\OtherTok{=}\StringTok{"./Geodata/2024/CHELSA/Climate\_CHELSAv2.1{-}rsds{-}max\_cell.tif"}

\NormalTok{df }\OtherTok{\textless{}{-}} \FunctionTok{downscale2egv}\NormalTok{(}
 \AttributeTok{template\_path =} \StringTok{"./Templates/TemplateRasters/LV100m\_10km.tif"}\NormalTok{,}
 \AttributeTok{grid\_path   =} \StringTok{"./Templates/TemplateGrids/tikls1km\_sauzeme.parquet"}\NormalTok{,}
 \AttributeTok{rawfile\_path =}\NormalTok{ reading,}
 \AttributeTok{out\_path   =} \StringTok{"./RasterGrids\_100m/2024/RAW/"}\NormalTok{,}
 \AttributeTok{file\_name   =}\NormalTok{ localname,}
 \AttributeTok{layer\_name  =}\NormalTok{ layername,}
 \AttributeTok{fill\_gaps   =} \ConstantTok{TRUE}\NormalTok{,}
 \AttributeTok{smooth    =} \ConstantTok{TRUE}\NormalTok{,}
 \AttributeTok{smooth\_radius\_km =} \DecValTok{5}\NormalTok{,}
 \AttributeTok{plot\_result  =} \ConstantTok{TRUE}\NormalTok{)}
\FunctionTok{print}\NormalTok{(df)}

\CommentTok{\# standardisation {-}{-}{-}{-}}
\ControlFlowTok{if}\NormalTok{(}\SpecialCharTok{!}\FunctionTok{require}\NormalTok{(terra)) \{}\FunctionTok{install.packages}\NormalTok{(}\StringTok{"terra"}\NormalTok{); }\FunctionTok{require}\NormalTok{(terra)\}}
\ControlFlowTok{if}\NormalTok{(}\SpecialCharTok{!}\FunctionTok{require}\NormalTok{(tidyverse)) \{}\FunctionTok{install.packages}\NormalTok{(}\StringTok{"tidyverse"}\NormalTok{); }\FunctionTok{require}\NormalTok{(tidyverse)\}}

\NormalTok{nosaukums}\OtherTok{=}\StringTok{"Climate\_CHELSAv2.1{-}rsds{-}max\_cell.tif"}
\NormalTok{ielasisanas\_cels}\OtherTok{=}\FunctionTok{paste0}\NormalTok{(}\StringTok{"./RasterGrids\_100m/2024/RAW/"}\NormalTok{,nosaukums)}
\NormalTok{saglabasanas\_cels}\OtherTok{=}\FunctionTok{paste0}\NormalTok{(}\StringTok{"./RasterGrids\_100m/2024/Scaled/"}\NormalTok{,nosaukums)}
\NormalTok{slanis}\OtherTok{=}\FunctionTok{rast}\NormalTok{(ielasisanas\_cels)}
\NormalTok{videjais}\OtherTok{=}\FunctionTok{global}\NormalTok{(slanis,}\AttributeTok{fun=}\StringTok{"mean"}\NormalTok{,}\AttributeTok{na.rm=}\ConstantTok{TRUE}\NormalTok{)}
\NormalTok{centrets}\OtherTok{=}\NormalTok{slanis}\SpecialCharTok{{-}}\NormalTok{videjais[,}\DecValTok{1}\NormalTok{]}
\NormalTok{standartnovirze}\OtherTok{=}\NormalTok{terra}\SpecialCharTok{::}\FunctionTok{global}\NormalTok{(centrets,}\AttributeTok{fun=}\StringTok{"rms"}\NormalTok{,}\AttributeTok{na.rm=}\ConstantTok{TRUE}\NormalTok{)}
\NormalTok{merogots}\OtherTok{=}\NormalTok{centrets}\SpecialCharTok{/}\NormalTok{standartnovirze[,}\DecValTok{1}\NormalTok{]}
\FunctionTok{writeRaster}\NormalTok{(merogots,}
      \AttributeTok{filename=}\NormalTok{saglabasanas\_cels,}
      \AttributeTok{overwrite=}\ConstantTok{TRUE}\NormalTok{)}
\end{Highlighting}
\end{Shaded}

\section{Climate\_CHELSAv2.1-rsds-mean\_cell}\label{ch06.056}

\textbf{filename:} \texttt{Climate\_CHELSAv2.1-rsds-mean\_cell.tif}

\textbf{layername:} \texttt{egv\_056}

\textbf{English name:} Mean of monthly mean surface downwelling shortwave flux in air (MJ m⁻²
d⁻¹) (CHELSA v2.1) within the analysis cell (1 ha)

\textbf{Latvian name:} Vidējā ik mēneša vidējā Zemes virsmu sasniedzošā saules radiācija (MJ m⁻² d⁻¹)
(CHELSA v2.1) analīzes šūnā (1 ha)

\textbf{Procedure:} Directly follows \hyperref[Ch04.11]{CHELSA v2.1}. EGV is prepared using
the workflow \texttt{egvtools::downscale2egv()} with inverse distance weighted (power =
2) gap filling and soft smoothing (power = 0.5) over 5 km radius around each cell.
Finally, the layer is standardised by subtracting the arithmetic mean and
dividing by the root mean squared error.

\begin{Shaded}
\begin{Highlighting}[]
\CommentTok{\# libs {-}{-}{-}{-}}
\ControlFlowTok{if}\NormalTok{(}\SpecialCharTok{!}\FunctionTok{require}\NormalTok{(egvtools)) \{remotes}\SpecialCharTok{::}\FunctionTok{install\_github}\NormalTok{(}\StringTok{"aavotins/egvtools"}\NormalTok{); }\FunctionTok{require}\NormalTok{(egvtools)\}}

\CommentTok{\# job {-}{-}{-}{-}}

\NormalTok{localname}\OtherTok{=}\StringTok{"Climate\_CHELSAv2.1{-}rsds{-}mean\_cell.tif"}
\NormalTok{layername}\OtherTok{=}\StringTok{"egv\_056"}
\NormalTok{reading}\OtherTok{=}\StringTok{"./Geodata/2024/CHELSA/Climate\_CHELSAv2.1{-}rsds{-}mean\_cell.tif"}

\NormalTok{df }\OtherTok{\textless{}{-}} \FunctionTok{downscale2egv}\NormalTok{(}
 \AttributeTok{template\_path =} \StringTok{"./Templates/TemplateRasters/LV100m\_10km.tif"}\NormalTok{,}
 \AttributeTok{grid\_path   =} \StringTok{"./Templates/TemplateGrids/tikls1km\_sauzeme.parquet"}\NormalTok{,}
 \AttributeTok{rawfile\_path =}\NormalTok{ reading,}
 \AttributeTok{out\_path   =} \StringTok{"./RasterGrids\_100m/2024/RAW/"}\NormalTok{,}
 \AttributeTok{file\_name   =}\NormalTok{ localname,}
 \AttributeTok{layer\_name  =}\NormalTok{ layername,}
 \AttributeTok{fill\_gaps   =} \ConstantTok{TRUE}\NormalTok{,}
 \AttributeTok{smooth    =} \ConstantTok{TRUE}\NormalTok{,}
 \AttributeTok{smooth\_radius\_km =} \DecValTok{5}\NormalTok{,}
 \AttributeTok{plot\_result  =} \ConstantTok{TRUE}\NormalTok{)}
\FunctionTok{print}\NormalTok{(df)}

\CommentTok{\# standardisation {-}{-}{-}{-}}
\ControlFlowTok{if}\NormalTok{(}\SpecialCharTok{!}\FunctionTok{require}\NormalTok{(terra)) \{}\FunctionTok{install.packages}\NormalTok{(}\StringTok{"terra"}\NormalTok{); }\FunctionTok{require}\NormalTok{(terra)\}}
\ControlFlowTok{if}\NormalTok{(}\SpecialCharTok{!}\FunctionTok{require}\NormalTok{(tidyverse)) \{}\FunctionTok{install.packages}\NormalTok{(}\StringTok{"tidyverse"}\NormalTok{); }\FunctionTok{require}\NormalTok{(tidyverse)\}}

\NormalTok{nosaukums}\OtherTok{=}\StringTok{"Climate\_CHELSAv2.1{-}rsds{-}mean\_cell.tif"}
\NormalTok{ielasisanas\_cels}\OtherTok{=}\FunctionTok{paste0}\NormalTok{(}\StringTok{"./RasterGrids\_100m/2024/RAW/"}\NormalTok{,nosaukums)}
\NormalTok{saglabasanas\_cels}\OtherTok{=}\FunctionTok{paste0}\NormalTok{(}\StringTok{"./RasterGrids\_100m/2024/Scaled/"}\NormalTok{,nosaukums)}
\NormalTok{slanis}\OtherTok{=}\FunctionTok{rast}\NormalTok{(ielasisanas\_cels)}
\NormalTok{videjais}\OtherTok{=}\FunctionTok{global}\NormalTok{(slanis,}\AttributeTok{fun=}\StringTok{"mean"}\NormalTok{,}\AttributeTok{na.rm=}\ConstantTok{TRUE}\NormalTok{)}
\NormalTok{centrets}\OtherTok{=}\NormalTok{slanis}\SpecialCharTok{{-}}\NormalTok{videjais[,}\DecValTok{1}\NormalTok{]}
\NormalTok{standartnovirze}\OtherTok{=}\NormalTok{terra}\SpecialCharTok{::}\FunctionTok{global}\NormalTok{(centrets,}\AttributeTok{fun=}\StringTok{"rms"}\NormalTok{,}\AttributeTok{na.rm=}\ConstantTok{TRUE}\NormalTok{)}
\NormalTok{merogots}\OtherTok{=}\NormalTok{centrets}\SpecialCharTok{/}\NormalTok{standartnovirze[,}\DecValTok{1}\NormalTok{]}
\FunctionTok{writeRaster}\NormalTok{(merogots,}
      \AttributeTok{filename=}\NormalTok{saglabasanas\_cels,}
      \AttributeTok{overwrite=}\ConstantTok{TRUE}\NormalTok{)}
\end{Highlighting}
\end{Shaded}

\section{Climate\_CHELSAv2.1-rsds-min\_cell}\label{ch06.057}

\textbf{filename:} \texttt{Climate\_CHELSAv2.1-rsds-min\_cell.tif}

\textbf{layername:} \texttt{egv\_057}

\textbf{English name:} Mean of monthly minimum surface shortwave flux in air (MJ m⁻² d⁻¹)
(CHELSA v2.1) within the analysis cell (1 ha)

\textbf{Latvian name:} Vidējā ik mēneša minimālā Zemes virsmu sasniedzošā saules
radiācija (MJ m⁻² d⁻¹) (CHELSA v2.1) analīzes šūnā (1 ha)

\textbf{Procedure:} Directly follows \hyperref[Ch04.11]{CHELSA v2.1}. EGV is prepared using
the workflow \texttt{egvtools::downscale2egv()} with inverse distance weighted (power =
2) gap filling and soft smoothing (power = 0.5) over 5 km radius around each cell.
Finally, the layer is standardised by subtracting the arithmetic mean and
dividing by the root mean squared error.

\begin{Shaded}
\begin{Highlighting}[]
\CommentTok{\# libs {-}{-}{-}{-}}
\ControlFlowTok{if}\NormalTok{(}\SpecialCharTok{!}\FunctionTok{require}\NormalTok{(egvtools)) \{remotes}\SpecialCharTok{::}\FunctionTok{install\_github}\NormalTok{(}\StringTok{"aavotins/egvtools"}\NormalTok{); }\FunctionTok{require}\NormalTok{(egvtools)\}}

\CommentTok{\# job {-}{-}{-}{-}}

\NormalTok{localname}\OtherTok{=}\StringTok{"Climate\_CHELSAv2.1{-}rsds{-}min\_cell.tif"}
\NormalTok{layername}\OtherTok{=}\StringTok{"egv\_057"}
\NormalTok{reading}\OtherTok{=}\StringTok{"./Geodata/2024/CHELSA/Climate\_CHELSAv2.1{-}rsds{-}min\_cell.tif"}

\NormalTok{df }\OtherTok{\textless{}{-}} \FunctionTok{downscale2egv}\NormalTok{(}
 \AttributeTok{template\_path =} \StringTok{"./Templates/TemplateRasters/LV100m\_10km.tif"}\NormalTok{,}
 \AttributeTok{grid\_path   =} \StringTok{"./Templates/TemplateGrids/tikls1km\_sauzeme.parquet"}\NormalTok{,}
 \AttributeTok{rawfile\_path =}\NormalTok{ reading,}
 \AttributeTok{out\_path   =} \StringTok{"./RasterGrids\_100m/2024/RAW/"}\NormalTok{,}
 \AttributeTok{file\_name   =}\NormalTok{ localname,}
 \AttributeTok{layer\_name  =}\NormalTok{ layername,}
 \AttributeTok{fill\_gaps   =} \ConstantTok{TRUE}\NormalTok{,}
 \AttributeTok{smooth    =} \ConstantTok{TRUE}\NormalTok{,}
 \AttributeTok{smooth\_radius\_km =} \DecValTok{5}\NormalTok{,}
 \AttributeTok{plot\_result  =} \ConstantTok{TRUE}\NormalTok{)}
\FunctionTok{print}\NormalTok{(df)}

\CommentTok{\# standardisation {-}{-}{-}{-}}
\ControlFlowTok{if}\NormalTok{(}\SpecialCharTok{!}\FunctionTok{require}\NormalTok{(terra)) \{}\FunctionTok{install.packages}\NormalTok{(}\StringTok{"terra"}\NormalTok{); }\FunctionTok{require}\NormalTok{(terra)\}}
\ControlFlowTok{if}\NormalTok{(}\SpecialCharTok{!}\FunctionTok{require}\NormalTok{(tidyverse)) \{}\FunctionTok{install.packages}\NormalTok{(}\StringTok{"tidyverse"}\NormalTok{); }\FunctionTok{require}\NormalTok{(tidyverse)\}}

\NormalTok{nosaukums}\OtherTok{=}\StringTok{"Climate\_CHELSAv2.1{-}rsds{-}min\_cell.tif"}
\NormalTok{ielasisanas\_cels}\OtherTok{=}\FunctionTok{paste0}\NormalTok{(}\StringTok{"./RasterGrids\_100m/2024/RAW/"}\NormalTok{,nosaukums)}
\NormalTok{saglabasanas\_cels}\OtherTok{=}\FunctionTok{paste0}\NormalTok{(}\StringTok{"./RasterGrids\_100m/2024/Scaled/"}\NormalTok{,nosaukums)}
\NormalTok{slanis}\OtherTok{=}\FunctionTok{rast}\NormalTok{(ielasisanas\_cels)}
\NormalTok{videjais}\OtherTok{=}\FunctionTok{global}\NormalTok{(slanis,}\AttributeTok{fun=}\StringTok{"mean"}\NormalTok{,}\AttributeTok{na.rm=}\ConstantTok{TRUE}\NormalTok{)}
\NormalTok{centrets}\OtherTok{=}\NormalTok{slanis}\SpecialCharTok{{-}}\NormalTok{videjais[,}\DecValTok{1}\NormalTok{]}
\NormalTok{standartnovirze}\OtherTok{=}\NormalTok{terra}\SpecialCharTok{::}\FunctionTok{global}\NormalTok{(centrets,}\AttributeTok{fun=}\StringTok{"rms"}\NormalTok{,}\AttributeTok{na.rm=}\ConstantTok{TRUE}\NormalTok{)}
\NormalTok{merogots}\OtherTok{=}\NormalTok{centrets}\SpecialCharTok{/}\NormalTok{standartnovirze[,}\DecValTok{1}\NormalTok{]}
\FunctionTok{writeRaster}\NormalTok{(merogots,}
      \AttributeTok{filename=}\NormalTok{saglabasanas\_cels,}
      \AttributeTok{overwrite=}\ConstantTok{TRUE}\NormalTok{)}
\end{Highlighting}
\end{Shaded}

\section{Climate\_CHELSAv2.1-rsds-range\_cell}\label{ch06.058}

\textbf{filename:} \texttt{Climate\_CHELSAv2.1-rsds-range\_cell.tif}

\textbf{layername:} \texttt{egv\_058}

\textbf{English name:} Annual range of daily mean surface downwelling shortwave flux in
air (MJ m⁻² d⁻¹) (CHELSA v2.1) within the analysis cell (1 ha)

\textbf{Latvian name:} Gada amplitūda ik dienas vidējai Zemes virsmu sasniedzošajai saules radiācijai
(MJ m⁻² d⁻¹) (CHELSA v2.1) analīzes šūnā (1 ha)

\textbf{Procedure:} Directly follows \hyperref[Ch04.11]{CHELSA v2.1}. EGV is prepared using
the workflow \texttt{egvtools::downscale2egv()} with inverse distance weighted (power =
2) gap filling and soft smoothing (power = 0.5) over 5 km radius around each cell.
Finally, the layer is standardised by subtracting the arithmetic mean and
dividing by the root mean squared error.

\begin{Shaded}
\begin{Highlighting}[]
\CommentTok{\# libs {-}{-}{-}{-}}
\ControlFlowTok{if}\NormalTok{(}\SpecialCharTok{!}\FunctionTok{require}\NormalTok{(egvtools)) \{remotes}\SpecialCharTok{::}\FunctionTok{install\_github}\NormalTok{(}\StringTok{"aavotins/egvtools"}\NormalTok{); }\FunctionTok{require}\NormalTok{(egvtools)\}}

\CommentTok{\# job {-}{-}{-}{-}}

\NormalTok{localname}\OtherTok{=}\StringTok{"Climate\_CHELSAv2.1{-}rsds{-}range\_cell.tif"}
\NormalTok{layername}\OtherTok{=}\StringTok{"egv\_058"}
\NormalTok{reading}\OtherTok{=}\StringTok{"./Geodata/2024/CHELSA/Climate\_CHELSAv2.1{-}rsds{-}range\_cell.tif"}

\NormalTok{df }\OtherTok{\textless{}{-}} \FunctionTok{downscale2egv}\NormalTok{(}
 \AttributeTok{template\_path =} \StringTok{"./Templates/TemplateRasters/LV100m\_10km.tif"}\NormalTok{,}
 \AttributeTok{grid\_path   =} \StringTok{"./Templates/TemplateGrids/tikls1km\_sauzeme.parquet"}\NormalTok{,}
 \AttributeTok{rawfile\_path =}\NormalTok{ reading,}
 \AttributeTok{out\_path   =} \StringTok{"./RasterGrids\_100m/2024/RAW/"}\NormalTok{,}
 \AttributeTok{file\_name   =}\NormalTok{ localname,}
 \AttributeTok{layer\_name  =}\NormalTok{ layername,}
 \AttributeTok{fill\_gaps   =} \ConstantTok{TRUE}\NormalTok{,}
 \AttributeTok{smooth    =} \ConstantTok{TRUE}\NormalTok{,}
 \AttributeTok{smooth\_radius\_km =} \DecValTok{5}\NormalTok{,}
 \AttributeTok{plot\_result  =} \ConstantTok{TRUE}\NormalTok{)}
\FunctionTok{print}\NormalTok{(df)}

\CommentTok{\# standardisation {-}{-}{-}{-}}
\ControlFlowTok{if}\NormalTok{(}\SpecialCharTok{!}\FunctionTok{require}\NormalTok{(terra)) \{}\FunctionTok{install.packages}\NormalTok{(}\StringTok{"terra"}\NormalTok{); }\FunctionTok{require}\NormalTok{(terra)\}}
\ControlFlowTok{if}\NormalTok{(}\SpecialCharTok{!}\FunctionTok{require}\NormalTok{(tidyverse)) \{}\FunctionTok{install.packages}\NormalTok{(}\StringTok{"tidyverse"}\NormalTok{); }\FunctionTok{require}\NormalTok{(tidyverse)\}}

\NormalTok{nosaukums}\OtherTok{=}\StringTok{"Climate\_CHELSAv2.1{-}rsds{-}range\_cell.tif"}
\NormalTok{ielasisanas\_cels}\OtherTok{=}\FunctionTok{paste0}\NormalTok{(}\StringTok{"./RasterGrids\_100m/2024/RAW/"}\NormalTok{,nosaukums)}
\NormalTok{saglabasanas\_cels}\OtherTok{=}\FunctionTok{paste0}\NormalTok{(}\StringTok{"./RasterGrids\_100m/2024/Scaled/"}\NormalTok{,nosaukums)}
\NormalTok{slanis}\OtherTok{=}\FunctionTok{rast}\NormalTok{(ielasisanas\_cels)}
\NormalTok{videjais}\OtherTok{=}\FunctionTok{global}\NormalTok{(slanis,}\AttributeTok{fun=}\StringTok{"mean"}\NormalTok{,}\AttributeTok{na.rm=}\ConstantTok{TRUE}\NormalTok{)}
\NormalTok{centrets}\OtherTok{=}\NormalTok{slanis}\SpecialCharTok{{-}}\NormalTok{videjais[,}\DecValTok{1}\NormalTok{]}
\NormalTok{standartnovirze}\OtherTok{=}\NormalTok{terra}\SpecialCharTok{::}\FunctionTok{global}\NormalTok{(centrets,}\AttributeTok{fun=}\StringTok{"rms"}\NormalTok{,}\AttributeTok{na.rm=}\ConstantTok{TRUE}\NormalTok{)}
\NormalTok{merogots}\OtherTok{=}\NormalTok{centrets}\SpecialCharTok{/}\NormalTok{standartnovirze[,}\DecValTok{1}\NormalTok{]}
\FunctionTok{writeRaster}\NormalTok{(merogots,}
      \AttributeTok{filename=}\NormalTok{saglabasanas\_cels,}
      \AttributeTok{overwrite=}\ConstantTok{TRUE}\NormalTok{)}
\end{Highlighting}
\end{Shaded}

\section{Climate\_CHELSAv2.1-scd\_cell}\label{ch06.059}

\textbf{filename:} \texttt{Climate\_CHELSAv2.1-scd\_cell.tif}

\textbf{layername:} \texttt{egv\_059}

\textbf{English name:} Number of days with snow cover (TREELIM) (CHELSA v2.1) within
the analysis cell (1 ha)

\textbf{Latvian name:} Dienu ar sniega segu skaits (TREELIM) (CHELSA v2.1) analīzes
šūnā (1 ha)

\textbf{Procedure:} Directly follows \hyperref[Ch04.11]{CHELSA v2.1}. EGV is prepared using
the workflow \texttt{egvtools::downscale2egv()} with inverse distance weighted (power =
2) gap filling and soft smoothing (power = 0.5) over 5 km radius around each cell.
Finally, the layer is standardised by subtracting the arithmetic mean and
dividing by the root mean squared error.

\begin{Shaded}
\begin{Highlighting}[]
\CommentTok{\# libs {-}{-}{-}{-}}
\ControlFlowTok{if}\NormalTok{(}\SpecialCharTok{!}\FunctionTok{require}\NormalTok{(egvtools)) \{remotes}\SpecialCharTok{::}\FunctionTok{install\_github}\NormalTok{(}\StringTok{"aavotins/egvtools"}\NormalTok{); }\FunctionTok{require}\NormalTok{(egvtools)\}}

\CommentTok{\# job {-}{-}{-}{-}}

\NormalTok{localname}\OtherTok{=}\StringTok{"Climate\_CHELSAv2.1{-}scd\_cell.tif"}
\NormalTok{layername}\OtherTok{=}\StringTok{"egv\_059"}
\NormalTok{reading}\OtherTok{=}\StringTok{"./Geodata/2024/CHELSA/Climate\_CHELSAv2.1{-}scd\_cell.tif"}

\NormalTok{df }\OtherTok{\textless{}{-}} \FunctionTok{downscale2egv}\NormalTok{(}
 \AttributeTok{template\_path =} \StringTok{"./Templates/TemplateRasters/LV100m\_10km.tif"}\NormalTok{,}
 \AttributeTok{grid\_path   =} \StringTok{"./Templates/TemplateGrids/tikls1km\_sauzeme.parquet"}\NormalTok{,}
 \AttributeTok{rawfile\_path =}\NormalTok{ reading,}
 \AttributeTok{out\_path   =} \StringTok{"./RasterGrids\_100m/2024/RAW/"}\NormalTok{,}
 \AttributeTok{file\_name   =}\NormalTok{ localname,}
 \AttributeTok{layer\_name  =}\NormalTok{ layername,}
 \AttributeTok{fill\_gaps   =} \ConstantTok{TRUE}\NormalTok{,}
 \AttributeTok{smooth    =} \ConstantTok{TRUE}\NormalTok{,}
 \AttributeTok{smooth\_radius\_km =} \DecValTok{5}\NormalTok{,}
 \AttributeTok{plot\_result  =} \ConstantTok{TRUE}\NormalTok{)}
\FunctionTok{print}\NormalTok{(df)}

\CommentTok{\# standardisation {-}{-}{-}{-}}
\ControlFlowTok{if}\NormalTok{(}\SpecialCharTok{!}\FunctionTok{require}\NormalTok{(terra)) \{}\FunctionTok{install.packages}\NormalTok{(}\StringTok{"terra"}\NormalTok{); }\FunctionTok{require}\NormalTok{(terra)\}}
\ControlFlowTok{if}\NormalTok{(}\SpecialCharTok{!}\FunctionTok{require}\NormalTok{(tidyverse)) \{}\FunctionTok{install.packages}\NormalTok{(}\StringTok{"tidyverse"}\NormalTok{); }\FunctionTok{require}\NormalTok{(tidyverse)\}}

\NormalTok{nosaukums}\OtherTok{=}\StringTok{"Climate\_CHELSAv2.1{-}scd\_cell.tif"}
\NormalTok{ielasisanas\_cels}\OtherTok{=}\FunctionTok{paste0}\NormalTok{(}\StringTok{"./RasterGrids\_100m/2024/RAW/"}\NormalTok{,nosaukums)}
\NormalTok{saglabasanas\_cels}\OtherTok{=}\FunctionTok{paste0}\NormalTok{(}\StringTok{"./RasterGrids\_100m/2024/Scaled/"}\NormalTok{,nosaukums)}
\NormalTok{slanis}\OtherTok{=}\FunctionTok{rast}\NormalTok{(ielasisanas\_cels)}
\NormalTok{videjais}\OtherTok{=}\FunctionTok{global}\NormalTok{(slanis,}\AttributeTok{fun=}\StringTok{"mean"}\NormalTok{,}\AttributeTok{na.rm=}\ConstantTok{TRUE}\NormalTok{)}
\NormalTok{centrets}\OtherTok{=}\NormalTok{slanis}\SpecialCharTok{{-}}\NormalTok{videjais[,}\DecValTok{1}\NormalTok{]}
\NormalTok{standartnovirze}\OtherTok{=}\NormalTok{terra}\SpecialCharTok{::}\FunctionTok{global}\NormalTok{(centrets,}\AttributeTok{fun=}\StringTok{"rms"}\NormalTok{,}\AttributeTok{na.rm=}\ConstantTok{TRUE}\NormalTok{)}
\NormalTok{merogots}\OtherTok{=}\NormalTok{centrets}\SpecialCharTok{/}\NormalTok{standartnovirze[,}\DecValTok{1}\NormalTok{]}
\FunctionTok{writeRaster}\NormalTok{(merogots,}
      \AttributeTok{filename=}\NormalTok{saglabasanas\_cels,}
      \AttributeTok{overwrite=}\ConstantTok{TRUE}\NormalTok{)}
\end{Highlighting}
\end{Shaded}

\section{Climate\_CHELSAv2.1-sfcWind-max\_cell}\label{ch06.060}

\textbf{filename:} \texttt{Climate\_CHELSAv2.1-sfcWind-max\_cell.tif}

\textbf{layername:} \texttt{egv\_060}

\textbf{English name:} Mean of monthly maximum near-surface wind speed (m s⁻¹) (CHELSA v2.1)
within the analysis cell (1 ha)

\textbf{Latvian name:} Vidējais ik mēneša maksimālais piezemes slāņa vēja ātrums (m s⁻¹)
(CHELSA v2.1) analīzes šūnā (1 ha)

\textbf{Procedure:} Directly follows \hyperref[Ch04.11]{CHELSA v2.1}. EGV is prepared using
the workflow \texttt{egvtools::downscale2egv()} with inverse distance weighted (power =
2) gap filling and soft smoothing (power = 0.5) over 5 km radius around each cell.
Finally, the layer is standardised by subtracting the arithmetic mean and
dividing by the root mean squared error.

\begin{Shaded}
\begin{Highlighting}[]
\CommentTok{\# libs {-}{-}{-}{-}}
\ControlFlowTok{if}\NormalTok{(}\SpecialCharTok{!}\FunctionTok{require}\NormalTok{(egvtools)) \{remotes}\SpecialCharTok{::}\FunctionTok{install\_github}\NormalTok{(}\StringTok{"aavotins/egvtools"}\NormalTok{); }\FunctionTok{require}\NormalTok{(egvtools)\}}

\CommentTok{\# job {-}{-}{-}{-}}

\NormalTok{localname}\OtherTok{=}\StringTok{"Climate\_CHELSAv2.1{-}sfcWind{-}max\_cell.tif"}
\NormalTok{layername}\OtherTok{=}\StringTok{"egv\_060"}
\NormalTok{reading}\OtherTok{=}\StringTok{"./Geodata/2024/CHELSA/Climate\_CHELSAv2.1{-}sfcWind{-}max\_cell.tif"}

\NormalTok{df }\OtherTok{\textless{}{-}} \FunctionTok{downscale2egv}\NormalTok{(}
 \AttributeTok{template\_path =} \StringTok{"./Templates/TemplateRasters/LV100m\_10km.tif"}\NormalTok{,}
 \AttributeTok{grid\_path   =} \StringTok{"./Templates/TemplateGrids/tikls1km\_sauzeme.parquet"}\NormalTok{,}
 \AttributeTok{rawfile\_path =}\NormalTok{ reading,}
 \AttributeTok{out\_path   =} \StringTok{"./RasterGrids\_100m/2024/RAW/"}\NormalTok{,}
 \AttributeTok{file\_name   =}\NormalTok{ localname,}
 \AttributeTok{layer\_name  =}\NormalTok{ layername,}
 \AttributeTok{fill\_gaps   =} \ConstantTok{TRUE}\NormalTok{,}
 \AttributeTok{smooth    =} \ConstantTok{TRUE}\NormalTok{,}
 \AttributeTok{smooth\_radius\_km =} \DecValTok{5}\NormalTok{,}
 \AttributeTok{plot\_result  =} \ConstantTok{TRUE}\NormalTok{)}
\FunctionTok{print}\NormalTok{(df)}

\CommentTok{\# standardisation {-}{-}{-}{-}}
\ControlFlowTok{if}\NormalTok{(}\SpecialCharTok{!}\FunctionTok{require}\NormalTok{(terra)) \{}\FunctionTok{install.packages}\NormalTok{(}\StringTok{"terra"}\NormalTok{); }\FunctionTok{require}\NormalTok{(terra)\}}
\ControlFlowTok{if}\NormalTok{(}\SpecialCharTok{!}\FunctionTok{require}\NormalTok{(tidyverse)) \{}\FunctionTok{install.packages}\NormalTok{(}\StringTok{"tidyverse"}\NormalTok{); }\FunctionTok{require}\NormalTok{(tidyverse)\}}

\NormalTok{nosaukums}\OtherTok{=}\StringTok{"Climate\_CHELSAv2.1{-}sfcWind{-}max\_cell.tif"}
\NormalTok{ielasisanas\_cels}\OtherTok{=}\FunctionTok{paste0}\NormalTok{(}\StringTok{"./RasterGrids\_100m/2024/RAW/"}\NormalTok{,nosaukums)}
\NormalTok{saglabasanas\_cels}\OtherTok{=}\FunctionTok{paste0}\NormalTok{(}\StringTok{"./RasterGrids\_100m/2024/Scaled/"}\NormalTok{,nosaukums)}
\NormalTok{slanis}\OtherTok{=}\FunctionTok{rast}\NormalTok{(ielasisanas\_cels)}
\NormalTok{videjais}\OtherTok{=}\FunctionTok{global}\NormalTok{(slanis,}\AttributeTok{fun=}\StringTok{"mean"}\NormalTok{,}\AttributeTok{na.rm=}\ConstantTok{TRUE}\NormalTok{)}
\NormalTok{centrets}\OtherTok{=}\NormalTok{slanis}\SpecialCharTok{{-}}\NormalTok{videjais[,}\DecValTok{1}\NormalTok{]}
\NormalTok{standartnovirze}\OtherTok{=}\NormalTok{terra}\SpecialCharTok{::}\FunctionTok{global}\NormalTok{(centrets,}\AttributeTok{fun=}\StringTok{"rms"}\NormalTok{,}\AttributeTok{na.rm=}\ConstantTok{TRUE}\NormalTok{)}
\NormalTok{merogots}\OtherTok{=}\NormalTok{centrets}\SpecialCharTok{/}\NormalTok{standartnovirze[,}\DecValTok{1}\NormalTok{]}
\FunctionTok{writeRaster}\NormalTok{(merogots,}
      \AttributeTok{filename=}\NormalTok{saglabasanas\_cels,}
      \AttributeTok{overwrite=}\ConstantTok{TRUE}\NormalTok{)}
\end{Highlighting}
\end{Shaded}

\section{Climate\_CHELSAv2.1-sfcWind-mean\_cell}\label{ch06.061}

\textbf{filename:} \texttt{Climate\_CHELSAv2.1-sfcWind-mean\_cell.tif}

\textbf{layername:} \texttt{egv\_061}

\textbf{English name:} Mean of monthly mean near-surface wind speed (m s⁻¹) (CHELSA v2.1)
within the analysis cell (1 ha)

\textbf{Latvian name:} Vidējais ik mēneša vidējais piezemes slāņa vēja ātrums (m s⁻¹) (CHELSA v2.1)
analīzes šūnā (1 ha)

\textbf{Procedure:} Directly follows \hyperref[Ch04.11]{CHELSA v2.1}. EGV is prepared using
the workflow \texttt{egvtools::downscale2egv()} with inverse distance weighted (power =
2) gap filling and soft smoothing (power = 0.5) over 5 km radius around each cell.
Finally, the layer is standardised by subtracting the arithmetic mean and
dividing by the root mean squared error.

\begin{Shaded}
\begin{Highlighting}[]
\CommentTok{\# libs {-}{-}{-}{-}}
\ControlFlowTok{if}\NormalTok{(}\SpecialCharTok{!}\FunctionTok{require}\NormalTok{(egvtools)) \{remotes}\SpecialCharTok{::}\FunctionTok{install\_github}\NormalTok{(}\StringTok{"aavotins/egvtools"}\NormalTok{); }\FunctionTok{require}\NormalTok{(egvtools)\}}

\CommentTok{\# job {-}{-}{-}{-}}

\NormalTok{localname}\OtherTok{=}\StringTok{"Climate\_CHELSAv2.1{-}sfcWind{-}mean\_cell.tif"}
\NormalTok{layername}\OtherTok{=}\StringTok{"egv\_061"}
\NormalTok{reading}\OtherTok{=}\StringTok{"./Geodata/2024/CHELSA/Climate\_CHELSAv2.1{-}sfcWind{-}mean\_cell.tif"}

\NormalTok{df }\OtherTok{\textless{}{-}} \FunctionTok{downscale2egv}\NormalTok{(}
 \AttributeTok{template\_path =} \StringTok{"./Templates/TemplateRasters/LV100m\_10km.tif"}\NormalTok{,}
 \AttributeTok{grid\_path   =} \StringTok{"./Templates/TemplateGrids/tikls1km\_sauzeme.parquet"}\NormalTok{,}
 \AttributeTok{rawfile\_path =}\NormalTok{ reading,}
 \AttributeTok{out\_path   =} \StringTok{"./RasterGrids\_100m/2024/RAW/"}\NormalTok{,}
 \AttributeTok{file\_name   =}\NormalTok{ localname,}
 \AttributeTok{layer\_name  =}\NormalTok{ layername,}
 \AttributeTok{fill\_gaps   =} \ConstantTok{TRUE}\NormalTok{,}
 \AttributeTok{smooth    =} \ConstantTok{TRUE}\NormalTok{,}
 \AttributeTok{smooth\_radius\_km =} \DecValTok{5}\NormalTok{,}
 \AttributeTok{plot\_result  =} \ConstantTok{TRUE}\NormalTok{)}
\FunctionTok{print}\NormalTok{(df)}

\CommentTok{\# standardisation {-}{-}{-}{-}}
\ControlFlowTok{if}\NormalTok{(}\SpecialCharTok{!}\FunctionTok{require}\NormalTok{(terra)) \{}\FunctionTok{install.packages}\NormalTok{(}\StringTok{"terra"}\NormalTok{); }\FunctionTok{require}\NormalTok{(terra)\}}
\ControlFlowTok{if}\NormalTok{(}\SpecialCharTok{!}\FunctionTok{require}\NormalTok{(tidyverse)) \{}\FunctionTok{install.packages}\NormalTok{(}\StringTok{"tidyverse"}\NormalTok{); }\FunctionTok{require}\NormalTok{(tidyverse)\}}

\NormalTok{nosaukums}\OtherTok{=}\StringTok{"Climate\_CHELSAv2.1{-}sfcWind{-}mean\_cell.tif"}
\NormalTok{ielasisanas\_cels}\OtherTok{=}\FunctionTok{paste0}\NormalTok{(}\StringTok{"./RasterGrids\_100m/2024/RAW/"}\NormalTok{,nosaukums)}
\NormalTok{saglabasanas\_cels}\OtherTok{=}\FunctionTok{paste0}\NormalTok{(}\StringTok{"./RasterGrids\_100m/2024/Scaled/"}\NormalTok{,nosaukums)}
\NormalTok{slanis}\OtherTok{=}\FunctionTok{rast}\NormalTok{(ielasisanas\_cels)}
\NormalTok{videjais}\OtherTok{=}\FunctionTok{global}\NormalTok{(slanis,}\AttributeTok{fun=}\StringTok{"mean"}\NormalTok{,}\AttributeTok{na.rm=}\ConstantTok{TRUE}\NormalTok{)}
\NormalTok{centrets}\OtherTok{=}\NormalTok{slanis}\SpecialCharTok{{-}}\NormalTok{videjais[,}\DecValTok{1}\NormalTok{]}
\NormalTok{standartnovirze}\OtherTok{=}\NormalTok{terra}\SpecialCharTok{::}\FunctionTok{global}\NormalTok{(centrets,}\AttributeTok{fun=}\StringTok{"rms"}\NormalTok{,}\AttributeTok{na.rm=}\ConstantTok{TRUE}\NormalTok{)}
\NormalTok{merogots}\OtherTok{=}\NormalTok{centrets}\SpecialCharTok{/}\NormalTok{standartnovirze[,}\DecValTok{1}\NormalTok{]}
\FunctionTok{writeRaster}\NormalTok{(merogots,}
      \AttributeTok{filename=}\NormalTok{saglabasanas\_cels,}
      \AttributeTok{overwrite=}\ConstantTok{TRUE}\NormalTok{)}
\end{Highlighting}
\end{Shaded}

\section{Climate\_CHELSAv2.1-sfcWind-min\_cell}\label{ch06.062}

\textbf{filename:} \texttt{Climate\_CHELSAv2.1-sfcWind-min\_cell.tif}

\textbf{layername:} \texttt{egv\_062}

\textbf{English name:} Mean of monthly minimum near-surface wind speed (m s⁻¹) (CHELSA v2.1)
within the analysis cell (1 ha)

\textbf{Latvian name:} Vidējais ik mēneša minimālais piezemes slāņa vēja ātrums (m s⁻¹)
(CHELSA v2.1) analīzes šūnā (1 ha)

\textbf{Procedure:} Directly follows \hyperref[Ch04.11]{CHELSA v2.1}. EGV is prepared using
the workflow \texttt{egvtools::downscale2egv()} with inverse distance weighted (power =
2) gap filling and soft smoothing (power = 0.5) over 5 km radius around each cell.
Finally, the layer is standardised by subtracting the arithmetic mean and
dividing by the root mean squared error.

\begin{Shaded}
\begin{Highlighting}[]
\CommentTok{\# libs {-}{-}{-}{-}}
\ControlFlowTok{if}\NormalTok{(}\SpecialCharTok{!}\FunctionTok{require}\NormalTok{(egvtools)) \{remotes}\SpecialCharTok{::}\FunctionTok{install\_github}\NormalTok{(}\StringTok{"aavotins/egvtools"}\NormalTok{); }\FunctionTok{require}\NormalTok{(egvtools)\}}

\CommentTok{\# job {-}{-}{-}{-}}

\NormalTok{localname}\OtherTok{=}\StringTok{"Climate\_CHELSAv2.1{-}sfcWind{-}min\_cell.tif"}
\NormalTok{layername}\OtherTok{=}\StringTok{"egv\_062"}
\NormalTok{reading}\OtherTok{=}\StringTok{"./Geodata/2024/CHELSA/Climate\_CHELSAv2.1{-}sfcWind{-}min\_cell.tif"}

\NormalTok{df }\OtherTok{\textless{}{-}} \FunctionTok{downscale2egv}\NormalTok{(}
 \AttributeTok{template\_path =} \StringTok{"./Templates/TemplateRasters/LV100m\_10km.tif"}\NormalTok{,}
 \AttributeTok{grid\_path   =} \StringTok{"./Templates/TemplateGrids/tikls1km\_sauzeme.parquet"}\NormalTok{,}
 \AttributeTok{rawfile\_path =}\NormalTok{ reading,}
 \AttributeTok{out\_path   =} \StringTok{"./RasterGrids\_100m/2024/RAW/"}\NormalTok{,}
 \AttributeTok{file\_name   =}\NormalTok{ localname,}
 \AttributeTok{layer\_name  =}\NormalTok{ layername,}
 \AttributeTok{fill\_gaps   =} \ConstantTok{TRUE}\NormalTok{,}
 \AttributeTok{smooth    =} \ConstantTok{TRUE}\NormalTok{,}
 \AttributeTok{smooth\_radius\_km =} \DecValTok{5}\NormalTok{,}
 \AttributeTok{plot\_result  =} \ConstantTok{TRUE}\NormalTok{)}
\FunctionTok{print}\NormalTok{(df)}

\CommentTok{\# standardisation {-}{-}{-}{-}}
\ControlFlowTok{if}\NormalTok{(}\SpecialCharTok{!}\FunctionTok{require}\NormalTok{(terra)) \{}\FunctionTok{install.packages}\NormalTok{(}\StringTok{"terra"}\NormalTok{); }\FunctionTok{require}\NormalTok{(terra)\}}
\ControlFlowTok{if}\NormalTok{(}\SpecialCharTok{!}\FunctionTok{require}\NormalTok{(tidyverse)) \{}\FunctionTok{install.packages}\NormalTok{(}\StringTok{"tidyverse"}\NormalTok{); }\FunctionTok{require}\NormalTok{(tidyverse)\}}

\NormalTok{nosaukums}\OtherTok{=}\StringTok{"Climate\_CHELSAv2.1{-}sfcWind{-}min\_cell.tif"}
\NormalTok{ielasisanas\_cels}\OtherTok{=}\FunctionTok{paste0}\NormalTok{(}\StringTok{"./RasterGrids\_100m/2024/RAW/"}\NormalTok{,nosaukums)}
\NormalTok{saglabasanas\_cels}\OtherTok{=}\FunctionTok{paste0}\NormalTok{(}\StringTok{"./RasterGrids\_100m/2024/Scaled/"}\NormalTok{,nosaukums)}
\NormalTok{slanis}\OtherTok{=}\FunctionTok{rast}\NormalTok{(ielasisanas\_cels)}
\NormalTok{videjais}\OtherTok{=}\FunctionTok{global}\NormalTok{(slanis,}\AttributeTok{fun=}\StringTok{"mean"}\NormalTok{,}\AttributeTok{na.rm=}\ConstantTok{TRUE}\NormalTok{)}
\NormalTok{centrets}\OtherTok{=}\NormalTok{slanis}\SpecialCharTok{{-}}\NormalTok{videjais[,}\DecValTok{1}\NormalTok{]}
\NormalTok{standartnovirze}\OtherTok{=}\NormalTok{terra}\SpecialCharTok{::}\FunctionTok{global}\NormalTok{(centrets,}\AttributeTok{fun=}\StringTok{"rms"}\NormalTok{,}\AttributeTok{na.rm=}\ConstantTok{TRUE}\NormalTok{)}
\NormalTok{merogots}\OtherTok{=}\NormalTok{centrets}\SpecialCharTok{/}\NormalTok{standartnovirze[,}\DecValTok{1}\NormalTok{]}
\FunctionTok{writeRaster}\NormalTok{(merogots,}
      \AttributeTok{filename=}\NormalTok{saglabasanas\_cels,}
      \AttributeTok{overwrite=}\ConstantTok{TRUE}\NormalTok{)}
\end{Highlighting}
\end{Shaded}

\section{Climate\_CHELSAv2.1-sfcWind-range\_cell}\label{ch06.063}

\textbf{filename:} \texttt{Climate\_CHELSAv2.1-sfcWind-range\_cell.tif}

\textbf{layername:} \texttt{egv\_063}

\textbf{English name:} Annual range of monthly mean near-surface wind speed (m s⁻¹)
(CHELSA v2.1) within the analysis cell (1 ha)

\textbf{Latvian name:} Gada amplitūda ik mēneša vidējam piezemes slāņa vēja ātrumam (m s⁻¹)
(CHELSA v2.1) analīzes šūnā (1 ha)

\textbf{Procedure:} Directly follows \hyperref[Ch04.11]{CHELSA v2.1}. EGV is prepared using
the workflow \texttt{egvtools::downscale2egv()} with inverse distance weighted (power =
2) gap filling and soft smoothing (power = 0.5) over 5 km radius around each cell.
Finally, the layer is standardised by subtracting the arithmetic mean and
dividing by the root mean squared error.

\begin{Shaded}
\begin{Highlighting}[]
\CommentTok{\# libs {-}{-}{-}{-}}
\ControlFlowTok{if}\NormalTok{(}\SpecialCharTok{!}\FunctionTok{require}\NormalTok{(egvtools)) \{remotes}\SpecialCharTok{::}\FunctionTok{install\_github}\NormalTok{(}\StringTok{"aavotins/egvtools"}\NormalTok{); }\FunctionTok{require}\NormalTok{(egvtools)\}}

\CommentTok{\# job {-}{-}{-}{-}}

\NormalTok{localname}\OtherTok{=}\StringTok{"Climate\_CHELSAv2.1{-}sfcWind{-}range\_cell.tif"}
\NormalTok{layername}\OtherTok{=}\StringTok{"egv\_063"}
\NormalTok{reading}\OtherTok{=}\StringTok{"./Geodata/2024/CHELSA/Climate\_CHELSAv2.1{-}sfcWind{-}range\_cell.tif"}

\NormalTok{df }\OtherTok{\textless{}{-}} \FunctionTok{downscale2egv}\NormalTok{(}
 \AttributeTok{template\_path =} \StringTok{"./Templates/TemplateRasters/LV100m\_10km.tif"}\NormalTok{,}
 \AttributeTok{grid\_path   =} \StringTok{"./Templates/TemplateGrids/tikls1km\_sauzeme.parquet"}\NormalTok{,}
 \AttributeTok{rawfile\_path =}\NormalTok{ reading,}
 \AttributeTok{out\_path   =} \StringTok{"./RasterGrids\_100m/2024/RAW/"}\NormalTok{,}
 \AttributeTok{file\_name   =}\NormalTok{ localname,}
 \AttributeTok{layer\_name  =}\NormalTok{ layername,}
 \AttributeTok{fill\_gaps   =} \ConstantTok{TRUE}\NormalTok{,}
 \AttributeTok{smooth    =} \ConstantTok{TRUE}\NormalTok{,}
 \AttributeTok{smooth\_radius\_km =} \DecValTok{5}\NormalTok{,}
 \AttributeTok{plot\_result  =} \ConstantTok{TRUE}\NormalTok{)}
\FunctionTok{print}\NormalTok{(df)}

\CommentTok{\# standardisation {-}{-}{-}{-}}
\ControlFlowTok{if}\NormalTok{(}\SpecialCharTok{!}\FunctionTok{require}\NormalTok{(terra)) \{}\FunctionTok{install.packages}\NormalTok{(}\StringTok{"terra"}\NormalTok{); }\FunctionTok{require}\NormalTok{(terra)\}}
\ControlFlowTok{if}\NormalTok{(}\SpecialCharTok{!}\FunctionTok{require}\NormalTok{(tidyverse)) \{}\FunctionTok{install.packages}\NormalTok{(}\StringTok{"tidyverse"}\NormalTok{); }\FunctionTok{require}\NormalTok{(tidyverse)\}}

\NormalTok{nosaukums}\OtherTok{=}\StringTok{"Climate\_CHELSAv2.1{-}sfcWind{-}range\_cell.tif"}
\NormalTok{ielasisanas\_cels}\OtherTok{=}\FunctionTok{paste0}\NormalTok{(}\StringTok{"./RasterGrids\_100m/2024/RAW/"}\NormalTok{,nosaukums)}
\NormalTok{saglabasanas\_cels}\OtherTok{=}\FunctionTok{paste0}\NormalTok{(}\StringTok{"./RasterGrids\_100m/2024/Scaled/"}\NormalTok{,nosaukums)}
\NormalTok{slanis}\OtherTok{=}\FunctionTok{rast}\NormalTok{(ielasisanas\_cels)}
\NormalTok{videjais}\OtherTok{=}\FunctionTok{global}\NormalTok{(slanis,}\AttributeTok{fun=}\StringTok{"mean"}\NormalTok{,}\AttributeTok{na.rm=}\ConstantTok{TRUE}\NormalTok{)}
\NormalTok{centrets}\OtherTok{=}\NormalTok{slanis}\SpecialCharTok{{-}}\NormalTok{videjais[,}\DecValTok{1}\NormalTok{]}
\NormalTok{standartnovirze}\OtherTok{=}\NormalTok{terra}\SpecialCharTok{::}\FunctionTok{global}\NormalTok{(centrets,}\AttributeTok{fun=}\StringTok{"rms"}\NormalTok{,}\AttributeTok{na.rm=}\ConstantTok{TRUE}\NormalTok{)}
\NormalTok{merogots}\OtherTok{=}\NormalTok{centrets}\SpecialCharTok{/}\NormalTok{standartnovirze[,}\DecValTok{1}\NormalTok{]}
\FunctionTok{writeRaster}\NormalTok{(merogots,}
      \AttributeTok{filename=}\NormalTok{saglabasanas\_cels,}
      \AttributeTok{overwrite=}\ConstantTok{TRUE}\NormalTok{)}
\end{Highlighting}
\end{Shaded}

\section{Climate\_CHELSAv2.1-swb\_cell}\label{ch06.064}

\textbf{filename:} \texttt{Climate\_CHELSAv2.1-swb\_cell.tif}

\textbf{layername:} \texttt{egv\_064}

\textbf{English name:} Site water balance (kg m⁻² year⁻¹) (CHELSA v2.1) within the
analysis cell (1 ha)

\textbf{Latvian name:} Ūdens bilance (kg m⁻² year⁻¹) (CHELSA v2.1) analīzes šūnā (1
ha)

\textbf{Procedure:} Directly follows \hyperref[Ch04.11]{CHELSA v2.1}. EGV is prepared using
the workflow \texttt{egvtools::downscale2egv()} with inverse distance weighted (power =
2) gap filling and soft smoothing (power = 0.5) over 5 km radius around each cell.
Finally, the layer is standardised by subtracting the arithmetic mean and
dividing by the root mean squared error.

\begin{Shaded}
\begin{Highlighting}[]
\CommentTok{\# libs {-}{-}{-}{-}}
\ControlFlowTok{if}\NormalTok{(}\SpecialCharTok{!}\FunctionTok{require}\NormalTok{(egvtools)) \{remotes}\SpecialCharTok{::}\FunctionTok{install\_github}\NormalTok{(}\StringTok{"aavotins/egvtools"}\NormalTok{); }\FunctionTok{require}\NormalTok{(egvtools)\}}

\CommentTok{\# job {-}{-}{-}{-}}

\NormalTok{localname}\OtherTok{=}\StringTok{"Climate\_CHELSAv2.1{-}swb\_cell.tif"}
\NormalTok{layername}\OtherTok{=}\StringTok{"egv\_064"}
\NormalTok{reading}\OtherTok{=}\StringTok{"./Geodata/2024/CHELSA/Climate\_CHELSAv2.1{-}swb\_cell.tif"}

\NormalTok{df }\OtherTok{\textless{}{-}} \FunctionTok{downscale2egv}\NormalTok{(}
 \AttributeTok{template\_path =} \StringTok{"./Templates/TemplateRasters/LV100m\_10km.tif"}\NormalTok{,}
 \AttributeTok{grid\_path   =} \StringTok{"./Templates/TemplateGrids/tikls1km\_sauzeme.parquet"}\NormalTok{,}
 \AttributeTok{rawfile\_path =}\NormalTok{ reading,}
 \AttributeTok{out\_path   =} \StringTok{"./RasterGrids\_100m/2024/RAW/"}\NormalTok{,}
 \AttributeTok{file\_name   =}\NormalTok{ localname,}
 \AttributeTok{layer\_name  =}\NormalTok{ layername,}
 \AttributeTok{fill\_gaps   =} \ConstantTok{TRUE}\NormalTok{,}
 \AttributeTok{smooth    =} \ConstantTok{TRUE}\NormalTok{,}
 \AttributeTok{smooth\_radius\_km =} \DecValTok{5}\NormalTok{,}
 \AttributeTok{plot\_result  =} \ConstantTok{TRUE}\NormalTok{)}
\FunctionTok{print}\NormalTok{(df)}

\CommentTok{\# standardisation {-}{-}{-}{-}}
\ControlFlowTok{if}\NormalTok{(}\SpecialCharTok{!}\FunctionTok{require}\NormalTok{(terra)) \{}\FunctionTok{install.packages}\NormalTok{(}\StringTok{"terra"}\NormalTok{); }\FunctionTok{require}\NormalTok{(terra)\}}
\ControlFlowTok{if}\NormalTok{(}\SpecialCharTok{!}\FunctionTok{require}\NormalTok{(tidyverse)) \{}\FunctionTok{install.packages}\NormalTok{(}\StringTok{"tidyverse"}\NormalTok{); }\FunctionTok{require}\NormalTok{(tidyverse)\}}

\NormalTok{nosaukums}\OtherTok{=}\StringTok{"Climate\_CHELSAv2.1{-}swb\_cell.tif"}
\NormalTok{ielasisanas\_cels}\OtherTok{=}\FunctionTok{paste0}\NormalTok{(}\StringTok{"./RasterGrids\_100m/2024/RAW/"}\NormalTok{,nosaukums)}
\NormalTok{saglabasanas\_cels}\OtherTok{=}\FunctionTok{paste0}\NormalTok{(}\StringTok{"./RasterGrids\_100m/2024/Scaled/"}\NormalTok{,nosaukums)}
\NormalTok{slanis}\OtherTok{=}\FunctionTok{rast}\NormalTok{(ielasisanas\_cels)}
\NormalTok{videjais}\OtherTok{=}\FunctionTok{global}\NormalTok{(slanis,}\AttributeTok{fun=}\StringTok{"mean"}\NormalTok{,}\AttributeTok{na.rm=}\ConstantTok{TRUE}\NormalTok{)}
\NormalTok{centrets}\OtherTok{=}\NormalTok{slanis}\SpecialCharTok{{-}}\NormalTok{videjais[,}\DecValTok{1}\NormalTok{]}
\NormalTok{standartnovirze}\OtherTok{=}\NormalTok{terra}\SpecialCharTok{::}\FunctionTok{global}\NormalTok{(centrets,}\AttributeTok{fun=}\StringTok{"rms"}\NormalTok{,}\AttributeTok{na.rm=}\ConstantTok{TRUE}\NormalTok{)}
\NormalTok{merogots}\OtherTok{=}\NormalTok{centrets}\SpecialCharTok{/}\NormalTok{standartnovirze[,}\DecValTok{1}\NormalTok{]}
\FunctionTok{writeRaster}\NormalTok{(merogots,}
      \AttributeTok{filename=}\NormalTok{saglabasanas\_cels,}
      \AttributeTok{overwrite=}\ConstantTok{TRUE}\NormalTok{)}
\end{Highlighting}
\end{Shaded}

\section{Climate\_CHELSAv2.1-swe\_cell}\label{ch06.065}

\textbf{filename:} \texttt{Climate\_CHELSAv2.1-swe\_cell.tif}

\textbf{layername:} \texttt{egv\_065}

\textbf{English name:} Snow water equivalent (kg m⁻² year⁻¹) (CHELSA v2.1) within the
analysis cell (1 ha)

\textbf{Latvian name:} Ūdens ekvivalents sniegā (kg m⁻² year⁻¹) (CHELSA v2.1)
analīzes šūnā (1 ha)

\textbf{Procedure:} Directly follows \hyperref[Ch04.11]{CHELSA v2.1}. EGV is prepared using
the workflow \texttt{egvtools::downscale2egv()} with inverse distance weighted (power =
2) gap filling and soft smoothing (power = 0.5) over 5 km radius around each cell.
Finally, the layer is standardised by subtracting the arithmetic mean and
dividing by the root mean squared error.

\begin{Shaded}
\begin{Highlighting}[]
\CommentTok{\# libs {-}{-}{-}{-}}
\ControlFlowTok{if}\NormalTok{(}\SpecialCharTok{!}\FunctionTok{require}\NormalTok{(egvtools)) \{remotes}\SpecialCharTok{::}\FunctionTok{install\_github}\NormalTok{(}\StringTok{"aavotins/egvtools"}\NormalTok{); }\FunctionTok{require}\NormalTok{(egvtools)\}}

\CommentTok{\# job {-}{-}{-}{-}}

\NormalTok{localname}\OtherTok{=}\StringTok{"Climate\_CHELSAv2.1{-}swe\_cell.tif"}
\NormalTok{layername}\OtherTok{=}\StringTok{"egv\_065"}
\NormalTok{reading}\OtherTok{=}\StringTok{"./Geodata/2024/CHELSA/Climate\_CHELSAv2.1{-}swe\_cell.tif"}

\NormalTok{df }\OtherTok{\textless{}{-}} \FunctionTok{downscale2egv}\NormalTok{(}
 \AttributeTok{template\_path =} \StringTok{"./Templates/TemplateRasters/LV100m\_10km.tif"}\NormalTok{,}
 \AttributeTok{grid\_path   =} \StringTok{"./Templates/TemplateGrids/tikls1km\_sauzeme.parquet"}\NormalTok{,}
 \AttributeTok{rawfile\_path =}\NormalTok{ reading,}
 \AttributeTok{out\_path   =} \StringTok{"./RasterGrids\_100m/2024/RAW/"}\NormalTok{,}
 \AttributeTok{file\_name   =}\NormalTok{ localname,}
 \AttributeTok{layer\_name  =}\NormalTok{ layername,}
 \AttributeTok{fill\_gaps   =} \ConstantTok{TRUE}\NormalTok{,}
 \AttributeTok{smooth    =} \ConstantTok{TRUE}\NormalTok{,}
 \AttributeTok{smooth\_radius\_km =} \DecValTok{5}\NormalTok{,}
 \AttributeTok{plot\_result  =} \ConstantTok{TRUE}\NormalTok{)}
\FunctionTok{print}\NormalTok{(df)}

\CommentTok{\# standardisation {-}{-}{-}{-}}
\ControlFlowTok{if}\NormalTok{(}\SpecialCharTok{!}\FunctionTok{require}\NormalTok{(terra)) \{}\FunctionTok{install.packages}\NormalTok{(}\StringTok{"terra"}\NormalTok{); }\FunctionTok{require}\NormalTok{(terra)\}}
\ControlFlowTok{if}\NormalTok{(}\SpecialCharTok{!}\FunctionTok{require}\NormalTok{(tidyverse)) \{}\FunctionTok{install.packages}\NormalTok{(}\StringTok{"tidyverse"}\NormalTok{); }\FunctionTok{require}\NormalTok{(tidyverse)\}}

\NormalTok{nosaukums}\OtherTok{=}\StringTok{"Climate\_CHELSAv2.1{-}swe\_cell.tif"}
\NormalTok{ielasisanas\_cels}\OtherTok{=}\FunctionTok{paste0}\NormalTok{(}\StringTok{"./RasterGrids\_100m/2024/RAW/"}\NormalTok{,nosaukums)}
\NormalTok{saglabasanas\_cels}\OtherTok{=}\FunctionTok{paste0}\NormalTok{(}\StringTok{"./RasterGrids\_100m/2024/Scaled/"}\NormalTok{,nosaukums)}
\NormalTok{slanis}\OtherTok{=}\FunctionTok{rast}\NormalTok{(ielasisanas\_cels)}
\NormalTok{videjais}\OtherTok{=}\FunctionTok{global}\NormalTok{(slanis,}\AttributeTok{fun=}\StringTok{"mean"}\NormalTok{,}\AttributeTok{na.rm=}\ConstantTok{TRUE}\NormalTok{)}
\NormalTok{centrets}\OtherTok{=}\NormalTok{slanis}\SpecialCharTok{{-}}\NormalTok{videjais[,}\DecValTok{1}\NormalTok{]}
\NormalTok{standartnovirze}\OtherTok{=}\NormalTok{terra}\SpecialCharTok{::}\FunctionTok{global}\NormalTok{(centrets,}\AttributeTok{fun=}\StringTok{"rms"}\NormalTok{,}\AttributeTok{na.rm=}\ConstantTok{TRUE}\NormalTok{)}
\NormalTok{merogots}\OtherTok{=}\NormalTok{centrets}\SpecialCharTok{/}\NormalTok{standartnovirze[,}\DecValTok{1}\NormalTok{]}
\FunctionTok{writeRaster}\NormalTok{(merogots,}
      \AttributeTok{filename=}\NormalTok{saglabasanas\_cels,}
      \AttributeTok{overwrite=}\ConstantTok{TRUE}\NormalTok{)}
\end{Highlighting}
\end{Shaded}

\section{Climate\_CHELSAv2.1-vpd-max\_cell}\label{ch06.066}

\textbf{filename:} \texttt{Climate\_CHELSAv2.1-vpd-max\_cell.tif}

\textbf{layername:} \texttt{egv\_066}

\textbf{English name:} Mean of monthly maximum vapor pressure deficit (Pa) (CHELSA v2.1)
within the analysis cell (1 ha)

\textbf{Latvian name:} Vidējais ik mēneša maksimālais iztvaikošanas spiediena deficīts
(Pa) (CHELSA v2.1) analīzes šūnā (1 ha)

\textbf{Procedure:} Directly follows \hyperref[Ch04.11]{CHELSA v2.1}. EGV is prepared using
the workflow \texttt{egvtools::downscale2egv()} with inverse distance weighted (power =
2) gap filling and soft smoothing (power = 0.5) over 5 km radius around each cell.
Finally, the layer is standardised by subtracting the arithmetic mean and
dividing by the root mean squared error.

\begin{Shaded}
\begin{Highlighting}[]
\CommentTok{\# libs {-}{-}{-}{-}}
\ControlFlowTok{if}\NormalTok{(}\SpecialCharTok{!}\FunctionTok{require}\NormalTok{(egvtools)) \{remotes}\SpecialCharTok{::}\FunctionTok{install\_github}\NormalTok{(}\StringTok{"aavotins/egvtools"}\NormalTok{); }\FunctionTok{require}\NormalTok{(egvtools)\}}

\CommentTok{\# job {-}{-}{-}{-}}

\NormalTok{localname}\OtherTok{=}\StringTok{"Climate\_CHELSAv2.1{-}vpd{-}max\_cell.tif"}
\NormalTok{layername}\OtherTok{=}\StringTok{"egv\_066"}
\NormalTok{reading}\OtherTok{=}\StringTok{"./Geodata/2024/CHELSA/Climate\_CHELSAv2.1{-}vpd{-}max\_cell.tif"}

\NormalTok{df }\OtherTok{\textless{}{-}} \FunctionTok{downscale2egv}\NormalTok{(}
 \AttributeTok{template\_path =} \StringTok{"./Templates/TemplateRasters/LV100m\_10km.tif"}\NormalTok{,}
 \AttributeTok{grid\_path   =} \StringTok{"./Templates/TemplateGrids/tikls1km\_sauzeme.parquet"}\NormalTok{,}
 \AttributeTok{rawfile\_path =}\NormalTok{ reading,}
 \AttributeTok{out\_path   =} \StringTok{"./RasterGrids\_100m/2024/RAW/"}\NormalTok{,}
 \AttributeTok{file\_name   =}\NormalTok{ localname,}
 \AttributeTok{layer\_name  =}\NormalTok{ layername,}
 \AttributeTok{fill\_gaps   =} \ConstantTok{TRUE}\NormalTok{,}
 \AttributeTok{smooth    =} \ConstantTok{TRUE}\NormalTok{,}
 \AttributeTok{smooth\_radius\_km =} \DecValTok{5}\NormalTok{,}
 \AttributeTok{plot\_result  =} \ConstantTok{TRUE}\NormalTok{)}
\FunctionTok{print}\NormalTok{(df)}

\CommentTok{\# standardisation {-}{-}{-}{-}}
\ControlFlowTok{if}\NormalTok{(}\SpecialCharTok{!}\FunctionTok{require}\NormalTok{(terra)) \{}\FunctionTok{install.packages}\NormalTok{(}\StringTok{"terra"}\NormalTok{); }\FunctionTok{require}\NormalTok{(terra)\}}
\ControlFlowTok{if}\NormalTok{(}\SpecialCharTok{!}\FunctionTok{require}\NormalTok{(tidyverse)) \{}\FunctionTok{install.packages}\NormalTok{(}\StringTok{"tidyverse"}\NormalTok{); }\FunctionTok{require}\NormalTok{(tidyverse)\}}

\NormalTok{nosaukums}\OtherTok{=}\StringTok{"Climate\_CHELSAv2.1{-}vpd{-}max\_cell.tif"}
\NormalTok{ielasisanas\_cels}\OtherTok{=}\FunctionTok{paste0}\NormalTok{(}\StringTok{"./RasterGrids\_100m/2024/RAW/"}\NormalTok{,nosaukums)}
\NormalTok{saglabasanas\_cels}\OtherTok{=}\FunctionTok{paste0}\NormalTok{(}\StringTok{"./RasterGrids\_100m/2024/Scaled/"}\NormalTok{,nosaukums)}
\NormalTok{slanis}\OtherTok{=}\FunctionTok{rast}\NormalTok{(ielasisanas\_cels)}
\NormalTok{videjais}\OtherTok{=}\FunctionTok{global}\NormalTok{(slanis,}\AttributeTok{fun=}\StringTok{"mean"}\NormalTok{,}\AttributeTok{na.rm=}\ConstantTok{TRUE}\NormalTok{)}
\NormalTok{centrets}\OtherTok{=}\NormalTok{slanis}\SpecialCharTok{{-}}\NormalTok{videjais[,}\DecValTok{1}\NormalTok{]}
\NormalTok{standartnovirze}\OtherTok{=}\NormalTok{terra}\SpecialCharTok{::}\FunctionTok{global}\NormalTok{(centrets,}\AttributeTok{fun=}\StringTok{"rms"}\NormalTok{,}\AttributeTok{na.rm=}\ConstantTok{TRUE}\NormalTok{)}
\NormalTok{merogots}\OtherTok{=}\NormalTok{centrets}\SpecialCharTok{/}\NormalTok{standartnovirze[,}\DecValTok{1}\NormalTok{]}
\FunctionTok{writeRaster}\NormalTok{(merogots,}
      \AttributeTok{filename=}\NormalTok{saglabasanas\_cels,}
      \AttributeTok{overwrite=}\ConstantTok{TRUE}\NormalTok{)}
\end{Highlighting}
\end{Shaded}

\section{Climate\_CHELSAv2.1-vpd-mean\_cell}\label{ch06.067}

\textbf{filename:} \texttt{Climate\_CHELSAv2.1-vpd-mean\_cell.tif}

\textbf{layername:} \texttt{egv\_067}

\textbf{English name:} Mean of monthly mean vapor pressure deficit (Pa) (CHELSA v2.1) within
the analysis cell (1 ha)

\textbf{Latvian name:} Vidējais ik mēneša vidējais iztvaikošanas spiediena deficīts (Pa) (CHELSA v2.1)
analīzes šūnā (1 ha)

\textbf{Procedure:} Directly follows \hyperref[Ch04.11]{CHELSA v2.1}. EGV is prepared using
the workflow \texttt{egvtools::downscale2egv()} with inverse distance weighted (power =
2) gap filling and soft smoothing (power = 0.5) over 5 km radius around each cell.
Finally, the layer is standardised by subtracting the arithmetic mean and
dividing by the root mean squared error.

\begin{Shaded}
\begin{Highlighting}[]
\CommentTok{\# libs {-}{-}{-}{-}}
\ControlFlowTok{if}\NormalTok{(}\SpecialCharTok{!}\FunctionTok{require}\NormalTok{(egvtools)) \{remotes}\SpecialCharTok{::}\FunctionTok{install\_github}\NormalTok{(}\StringTok{"aavotins/egvtools"}\NormalTok{); }\FunctionTok{require}\NormalTok{(egvtools)\}}

\CommentTok{\# job {-}{-}{-}{-}}

\NormalTok{localname}\OtherTok{=}\StringTok{"Climate\_CHELSAv2.1{-}vpd{-}mean\_cell.tif"}
\NormalTok{layername}\OtherTok{=}\StringTok{"egv\_067"}
\NormalTok{reading}\OtherTok{=}\StringTok{"./Geodata/2024/CHELSA/Climate\_CHELSAv2.1{-}vpd{-}mean\_cell.tif"}

\NormalTok{df }\OtherTok{\textless{}{-}} \FunctionTok{downscale2egv}\NormalTok{(}
 \AttributeTok{template\_path =} \StringTok{"./Templates/TemplateRasters/LV100m\_10km.tif"}\NormalTok{,}
 \AttributeTok{grid\_path   =} \StringTok{"./Templates/TemplateGrids/tikls1km\_sauzeme.parquet"}\NormalTok{,}
 \AttributeTok{rawfile\_path =}\NormalTok{ reading,}
 \AttributeTok{out\_path   =} \StringTok{"./RasterGrids\_100m/2024/RAW/"}\NormalTok{,}
 \AttributeTok{file\_name   =}\NormalTok{ localname,}
 \AttributeTok{layer\_name  =}\NormalTok{ layername,}
 \AttributeTok{fill\_gaps   =} \ConstantTok{TRUE}\NormalTok{,}
 \AttributeTok{smooth    =} \ConstantTok{TRUE}\NormalTok{,}
 \AttributeTok{smooth\_radius\_km =} \DecValTok{5}\NormalTok{,}
 \AttributeTok{plot\_result  =} \ConstantTok{TRUE}\NormalTok{)}
\FunctionTok{print}\NormalTok{(df)}

\CommentTok{\# standardisation {-}{-}{-}{-}}
\ControlFlowTok{if}\NormalTok{(}\SpecialCharTok{!}\FunctionTok{require}\NormalTok{(terra)) \{}\FunctionTok{install.packages}\NormalTok{(}\StringTok{"terra"}\NormalTok{); }\FunctionTok{require}\NormalTok{(terra)\}}
\ControlFlowTok{if}\NormalTok{(}\SpecialCharTok{!}\FunctionTok{require}\NormalTok{(tidyverse)) \{}\FunctionTok{install.packages}\NormalTok{(}\StringTok{"tidyverse"}\NormalTok{); }\FunctionTok{require}\NormalTok{(tidyverse)\}}

\NormalTok{nosaukums}\OtherTok{=}\StringTok{"Climate\_CHELSAv2.1{-}vpd{-}mean\_cell.tif"}
\NormalTok{ielasisanas\_cels}\OtherTok{=}\FunctionTok{paste0}\NormalTok{(}\StringTok{"./RasterGrids\_100m/2024/RAW/"}\NormalTok{,nosaukums)}
\NormalTok{saglabasanas\_cels}\OtherTok{=}\FunctionTok{paste0}\NormalTok{(}\StringTok{"./RasterGrids\_100m/2024/Scaled/"}\NormalTok{,nosaukums)}
\NormalTok{slanis}\OtherTok{=}\FunctionTok{rast}\NormalTok{(ielasisanas\_cels)}
\NormalTok{videjais}\OtherTok{=}\FunctionTok{global}\NormalTok{(slanis,}\AttributeTok{fun=}\StringTok{"mean"}\NormalTok{,}\AttributeTok{na.rm=}\ConstantTok{TRUE}\NormalTok{)}
\NormalTok{centrets}\OtherTok{=}\NormalTok{slanis}\SpecialCharTok{{-}}\NormalTok{videjais[,}\DecValTok{1}\NormalTok{]}
\NormalTok{standartnovirze}\OtherTok{=}\NormalTok{terra}\SpecialCharTok{::}\FunctionTok{global}\NormalTok{(centrets,}\AttributeTok{fun=}\StringTok{"rms"}\NormalTok{,}\AttributeTok{na.rm=}\ConstantTok{TRUE}\NormalTok{)}
\NormalTok{merogots}\OtherTok{=}\NormalTok{centrets}\SpecialCharTok{/}\NormalTok{standartnovirze[,}\DecValTok{1}\NormalTok{]}
\FunctionTok{writeRaster}\NormalTok{(merogots,}
      \AttributeTok{filename=}\NormalTok{saglabasanas\_cels,}
      \AttributeTok{overwrite=}\ConstantTok{TRUE}\NormalTok{)}
\end{Highlighting}
\end{Shaded}

\section{Climate\_CHELSAv2.1-vpd-min\_cell}\label{ch06.068}

\textbf{filename:} \texttt{Climate\_CHELSAv2.1-vpd-min\_cell.tif}

\textbf{layername:} \texttt{egv\_068}

\textbf{English name:} Mean of monthly minimum vapor pressure deficit (Pa) (CHELSA v2.1)
within the analysis cell (1 ha)

\textbf{Latvian name:} Vidējais ik mēneša minimālais iztvaikošanas spiediena deficīts
(Pa) (CHELSA v2.1) analīzes šūnā (1 ha)

\textbf{Procedure:} Directly follows \hyperref[Ch04.11]{CHELSA v2.1}. EGV is prepared using
the workflow \texttt{egvtools::downscale2egv()} with inverse distance weighted (power =
2) gap filling and soft smoothing (power = 0.5) over 5 km radius around each cell.
Finally, the layer is standardised by subtracting the arithmetic mean and
dividing by the root mean squared error.

\begin{Shaded}
\begin{Highlighting}[]
\CommentTok{\# libs {-}{-}{-}{-}}
\ControlFlowTok{if}\NormalTok{(}\SpecialCharTok{!}\FunctionTok{require}\NormalTok{(egvtools)) \{remotes}\SpecialCharTok{::}\FunctionTok{install\_github}\NormalTok{(}\StringTok{"aavotins/egvtools"}\NormalTok{); }\FunctionTok{require}\NormalTok{(egvtools)\}}

\CommentTok{\# job {-}{-}{-}{-}}

\NormalTok{localname}\OtherTok{=}\StringTok{"Climate\_CHELSAv2.1{-}vpd{-}min\_cell.tif"}
\NormalTok{layername}\OtherTok{=}\StringTok{"egv\_068"}
\NormalTok{reading}\OtherTok{=}\StringTok{"./Geodata/2024/CHELSA/Climate\_CHELSAv2.1{-}vpd{-}min\_cell.tif"}

\NormalTok{df }\OtherTok{\textless{}{-}} \FunctionTok{downscale2egv}\NormalTok{(}
 \AttributeTok{template\_path =} \StringTok{"./Templates/TemplateRasters/LV100m\_10km.tif"}\NormalTok{,}
 \AttributeTok{grid\_path   =} \StringTok{"./Templates/TemplateGrids/tikls1km\_sauzeme.parquet"}\NormalTok{,}
 \AttributeTok{rawfile\_path =}\NormalTok{ reading,}
 \AttributeTok{out\_path   =} \StringTok{"./RasterGrids\_100m/2024/RAW/"}\NormalTok{,}
 \AttributeTok{file\_name   =}\NormalTok{ localname,}
 \AttributeTok{layer\_name  =}\NormalTok{ layername,}
 \AttributeTok{fill\_gaps   =} \ConstantTok{TRUE}\NormalTok{,}
 \AttributeTok{smooth    =} \ConstantTok{TRUE}\NormalTok{,}
 \AttributeTok{smooth\_radius\_km =} \DecValTok{5}\NormalTok{,}
 \AttributeTok{plot\_result  =} \ConstantTok{TRUE}\NormalTok{)}
\FunctionTok{print}\NormalTok{(df)}

\CommentTok{\# standardisation {-}{-}{-}{-}}
\ControlFlowTok{if}\NormalTok{(}\SpecialCharTok{!}\FunctionTok{require}\NormalTok{(terra)) \{}\FunctionTok{install.packages}\NormalTok{(}\StringTok{"terra"}\NormalTok{); }\FunctionTok{require}\NormalTok{(terra)\}}
\ControlFlowTok{if}\NormalTok{(}\SpecialCharTok{!}\FunctionTok{require}\NormalTok{(tidyverse)) \{}\FunctionTok{install.packages}\NormalTok{(}\StringTok{"tidyverse"}\NormalTok{); }\FunctionTok{require}\NormalTok{(tidyverse)\}}

\NormalTok{nosaukums}\OtherTok{=}\StringTok{"Climate\_CHELSAv2.1{-}vpd{-}min\_cell.tif"}
\NormalTok{ielasisanas\_cels}\OtherTok{=}\FunctionTok{paste0}\NormalTok{(}\StringTok{"./RasterGrids\_100m/2024/RAW/"}\NormalTok{,nosaukums)}
\NormalTok{saglabasanas\_cels}\OtherTok{=}\FunctionTok{paste0}\NormalTok{(}\StringTok{"./RasterGrids\_100m/2024/Scaled/"}\NormalTok{,nosaukums)}
\NormalTok{slanis}\OtherTok{=}\FunctionTok{rast}\NormalTok{(ielasisanas\_cels)}
\NormalTok{videjais}\OtherTok{=}\FunctionTok{global}\NormalTok{(slanis,}\AttributeTok{fun=}\StringTok{"mean"}\NormalTok{,}\AttributeTok{na.rm=}\ConstantTok{TRUE}\NormalTok{)}
\NormalTok{centrets}\OtherTok{=}\NormalTok{slanis}\SpecialCharTok{{-}}\NormalTok{videjais[,}\DecValTok{1}\NormalTok{]}
\NormalTok{standartnovirze}\OtherTok{=}\NormalTok{terra}\SpecialCharTok{::}\FunctionTok{global}\NormalTok{(centrets,}\AttributeTok{fun=}\StringTok{"rms"}\NormalTok{,}\AttributeTok{na.rm=}\ConstantTok{TRUE}\NormalTok{)}
\NormalTok{merogots}\OtherTok{=}\NormalTok{centrets}\SpecialCharTok{/}\NormalTok{standartnovirze[,}\DecValTok{1}\NormalTok{]}
\FunctionTok{writeRaster}\NormalTok{(merogots,}
      \AttributeTok{filename=}\NormalTok{saglabasanas\_cels,}
      \AttributeTok{overwrite=}\ConstantTok{TRUE}\NormalTok{)}
\end{Highlighting}
\end{Shaded}

\section{Climate\_CHELSAv2.1-vpd-range\_cell}\label{ch06.069}

\textbf{filename:} \texttt{Climate\_CHELSAv2.1-vpd-range\_cell.tif}

\textbf{layername:} \texttt{egv\_069}

\textbf{English name:} Annual range of monthly mean vapor pressure deficit (Pa) (CHELSA
v2.1) within the analysis cell (1 ha)

\textbf{Latvian name:} Gada iztvaikošanas spiediena deficīta ik mēneša vidējo amplitūda (Pa) (CHELSA
v2.1) analīzes šūnā (1 ha)

\textbf{Procedure:} Directly follows \hyperref[Ch04.11]{CHELSA v2.1}. EGV is prepared using
the workflow \texttt{egvtools::downscale2egv()} with inverse distance weighted (power =
2) gap filling and soft smoothing (power = 0.5) over 5 km radius around each cell.
Finally, the layer is standardised by subtracting the arithmetic mean and
dividing by the root mean squared error.

\begin{Shaded}
\begin{Highlighting}[]
\CommentTok{\# libs {-}{-}{-}{-}}
\ControlFlowTok{if}\NormalTok{(}\SpecialCharTok{!}\FunctionTok{require}\NormalTok{(egvtools)) \{remotes}\SpecialCharTok{::}\FunctionTok{install\_github}\NormalTok{(}\StringTok{"aavotins/egvtools"}\NormalTok{); }\FunctionTok{require}\NormalTok{(egvtools)\}}

\CommentTok{\# job {-}{-}{-}{-}}

\NormalTok{localname}\OtherTok{=}\StringTok{"Climate\_CHELSAv2.1{-}vpd{-}range\_cell.tif"}
\NormalTok{layername}\OtherTok{=}\StringTok{"egv\_069"}
\NormalTok{reading}\OtherTok{=}\StringTok{"./Geodata/2024/CHELSA/Climate\_CHELSAv2.1{-}vpd{-}range\_cell.tif"}

\NormalTok{df }\OtherTok{\textless{}{-}} \FunctionTok{downscale2egv}\NormalTok{(}
 \AttributeTok{template\_path =} \StringTok{"./Templates/TemplateRasters/LV100m\_10km.tif"}\NormalTok{,}
 \AttributeTok{grid\_path   =} \StringTok{"./Templates/TemplateGrids/tikls1km\_sauzeme.parquet"}\NormalTok{,}
 \AttributeTok{rawfile\_path =}\NormalTok{ reading,}
 \AttributeTok{out\_path   =} \StringTok{"./RasterGrids\_100m/2024/RAW/"}\NormalTok{,}
 \AttributeTok{file\_name   =}\NormalTok{ localname,}
 \AttributeTok{layer\_name  =}\NormalTok{ layername,}
 \AttributeTok{fill\_gaps   =} \ConstantTok{TRUE}\NormalTok{,}
 \AttributeTok{smooth    =} \ConstantTok{TRUE}\NormalTok{,}
 \AttributeTok{smooth\_radius\_km =} \DecValTok{5}\NormalTok{,}
 \AttributeTok{plot\_result  =} \ConstantTok{TRUE}\NormalTok{)}
\FunctionTok{print}\NormalTok{(df)}

\CommentTok{\# standardisation {-}{-}{-}{-}}
\ControlFlowTok{if}\NormalTok{(}\SpecialCharTok{!}\FunctionTok{require}\NormalTok{(terra)) \{}\FunctionTok{install.packages}\NormalTok{(}\StringTok{"terra"}\NormalTok{); }\FunctionTok{require}\NormalTok{(terra)\}}
\ControlFlowTok{if}\NormalTok{(}\SpecialCharTok{!}\FunctionTok{require}\NormalTok{(tidyverse)) \{}\FunctionTok{install.packages}\NormalTok{(}\StringTok{"tidyverse"}\NormalTok{); }\FunctionTok{require}\NormalTok{(tidyverse)\}}

\NormalTok{nosaukums}\OtherTok{=}\StringTok{"Climate\_CHELSAv2.1{-}vpd{-}range\_cell.tif"}
\NormalTok{ielasisanas\_cels}\OtherTok{=}\FunctionTok{paste0}\NormalTok{(}\StringTok{"./RasterGrids\_100m/2024/RAW/"}\NormalTok{,nosaukums)}
\NormalTok{saglabasanas\_cels}\OtherTok{=}\FunctionTok{paste0}\NormalTok{(}\StringTok{"./RasterGrids\_100m/2024/Scaled/"}\NormalTok{,nosaukums)}
\NormalTok{slanis}\OtherTok{=}\FunctionTok{rast}\NormalTok{(ielasisanas\_cels)}
\NormalTok{videjais}\OtherTok{=}\FunctionTok{global}\NormalTok{(slanis,}\AttributeTok{fun=}\StringTok{"mean"}\NormalTok{,}\AttributeTok{na.rm=}\ConstantTok{TRUE}\NormalTok{)}
\NormalTok{centrets}\OtherTok{=}\NormalTok{slanis}\SpecialCharTok{{-}}\NormalTok{videjais[,}\DecValTok{1}\NormalTok{]}
\NormalTok{standartnovirze}\OtherTok{=}\NormalTok{terra}\SpecialCharTok{::}\FunctionTok{global}\NormalTok{(centrets,}\AttributeTok{fun=}\StringTok{"rms"}\NormalTok{,}\AttributeTok{na.rm=}\ConstantTok{TRUE}\NormalTok{)}
\NormalTok{merogots}\OtherTok{=}\NormalTok{centrets}\SpecialCharTok{/}\NormalTok{standartnovirze[,}\DecValTok{1}\NormalTok{]}
\FunctionTok{writeRaster}\NormalTok{(merogots,}
      \AttributeTok{filename=}\NormalTok{saglabasanas\_cels,}
      \AttributeTok{overwrite=}\ConstantTok{TRUE}\NormalTok{)}
\end{Highlighting}
\end{Shaded}

\section{HydroClim\_01-max\_cell}\label{ch06.070}

\textbf{filename:} \texttt{HydroClim\_01-max\_cell.tif}

\textbf{layername:} \texttt{egv\_070}

\textbf{English name:} Maximum per subcatchment upstream mean annual air temperature
(°C) (HydroClim) within the analysis cell (1 ha)

\textbf{Latvian name:} Sateces apakšbaseina maksimālā vidējā gaisa temperatūra
augštecē (°C) (HydroClim) analīzes šūnā (1 ha)

\textbf{Procedure:} Information from the \hyperref[Ch04.12]{HydroClim
data} - including both basin and raster layers - is used. First, basin CRS is transformed to EPSG:3059. Then,
zonal statistics (per basin) using a layer specific summary function (max) are
calculated (\texttt{exactextractr::exact\_extract()}), and the the results are rasterised with the workflow
\texttt{egvtools::polygon2input()}. Once rasterised to input data, EGV is created using the workflow
\texttt{egvtools::input2egv()}. To prevent from gaps at the edges, inverse distance
weighted (power = 2) gap filling is implemented. To save disk space,
the intermediate input layer is unlinked. Finally, the layer is standardised by
subtracting the arithmetic mean and dividing by the root mean squared error.

\begin{Shaded}
\begin{Highlighting}[]
\CommentTok{\# libs {-}{-}{-}{-}}
\ControlFlowTok{if}\NormalTok{(}\SpecialCharTok{!}\FunctionTok{require}\NormalTok{(egvtools)) \{remotes}\SpecialCharTok{::}\FunctionTok{install\_github}\NormalTok{(}\StringTok{"aavotins/egvtools"}\NormalTok{); }\FunctionTok{require}\NormalTok{(egvtools)\}}
\ControlFlowTok{if}\NormalTok{(}\SpecialCharTok{!}\FunctionTok{require}\NormalTok{(terra)) \{}\FunctionTok{install.packages}\NormalTok{(}\StringTok{"terra"}\NormalTok{); }\FunctionTok{require}\NormalTok{(terra)\}}
\ControlFlowTok{if}\NormalTok{(}\SpecialCharTok{!}\FunctionTok{require}\NormalTok{(tidyverse)) \{}\FunctionTok{install.packages}\NormalTok{(}\StringTok{"tidyverse"}\NormalTok{); }\FunctionTok{require}\NormalTok{(tidyverse)\}}
\ControlFlowTok{if}\NormalTok{(}\SpecialCharTok{!}\FunctionTok{require}\NormalTok{(sf)) \{}\FunctionTok{install.packages}\NormalTok{(}\StringTok{"sf"}\NormalTok{); }\FunctionTok{require}\NormalTok{(sf)\}}
\ControlFlowTok{if}\NormalTok{(}\SpecialCharTok{!}\FunctionTok{require}\NormalTok{(sfarrow)) \{}\FunctionTok{install.packages}\NormalTok{(}\StringTok{"sfarrow"}\NormalTok{); }\FunctionTok{require}\NormalTok{(sfarrow)\}}
\ControlFlowTok{if}\NormalTok{(}\SpecialCharTok{!}\FunctionTok{require}\NormalTok{(exactextractr)) \{}\FunctionTok{install.packages}\NormalTok{(}\StringTok{"exactextractr"}\NormalTok{); }\FunctionTok{require}\NormalTok{(exactextractr)\}}

\CommentTok{\# basins {-}{-}{-}{-}}
\NormalTok{level12}\OtherTok{=}\FunctionTok{st\_read}\NormalTok{(}\StringTok{"./Geodata/2024/HydroClim/hybas\_lake\_eu\_lev01{-}12\_v1c/hybas\_lake\_eu\_lev12\_v1c.shp"}\NormalTok{)}
\NormalTok{grid\_1km}\OtherTok{=}\NormalTok{sfarrow}\SpecialCharTok{::}\FunctionTok{st\_read\_parquet}\NormalTok{(}\StringTok{"./Templates/TemplateGrids/tikls1km\_sauzeme.parquet"}\NormalTok{)}
\NormalTok{grid\_1km}\OtherTok{=}\FunctionTok{st\_transform}\NormalTok{(grid\_1km,}\AttributeTok{crs=}\DecValTok{3059}\NormalTok{)}
\NormalTok{level12}\OtherTok{=}\FunctionTok{st\_transform}\NormalTok{(level12,}\AttributeTok{crs=}\DecValTok{3059}\NormalTok{)}
\NormalTok{level12}\OtherTok{=}\NormalTok{level12[grid\_1km,,]}

\NormalTok{level12}\OtherTok{=}\FunctionTok{st\_make\_valid}\NormalTok{(level12)}

\CommentTok{\# job {-}{-}{-}{-}}

\NormalTok{localname}\OtherTok{=}\StringTok{"HydroClim\_01{-}max\_cell.tif"}
\NormalTok{layername}\OtherTok{=}\StringTok{"egv\_070"}
\NormalTok{summary\_function}\OtherTok{=}\StringTok{"max"}
 
\NormalTok{slanis}\OtherTok{=}\FunctionTok{rast}\NormalTok{(}\FunctionTok{paste0}\NormalTok{(}\StringTok{"./Geodata/2024/HydroClim/"}\NormalTok{,localname))}
\NormalTok{level12}\SpecialCharTok{$}\NormalTok{Hydro\_values}\OtherTok{=}\FunctionTok{exact\_extract}\NormalTok{(slanis,level12,}\AttributeTok{fun=}\NormalTok{summary\_function)}
 
\FunctionTok{polygon2input}\NormalTok{(}\AttributeTok{vector\_data =}\NormalTok{ level12,}
       \AttributeTok{template\_path =} \StringTok{"./Templates/TemplateRasters/LV10m\_10km.tif"}\NormalTok{,}
       \AttributeTok{out\_path =} \StringTok{"./RasterGrids\_10m/2024/"}\NormalTok{,}
       \AttributeTok{file\_name =}\NormalTok{ localname,}
       \AttributeTok{value\_field =} \StringTok{"Hydro\_values"}\NormalTok{,}
       \AttributeTok{fun=}\StringTok{"first"}\NormalTok{,}
       \AttributeTok{value\_type =} \StringTok{"continuous"}\NormalTok{,}
       \AttributeTok{prepare=}\ConstantTok{FALSE}\NormalTok{,}
       \AttributeTok{project\_mode =} \StringTok{"auto"}\NormalTok{,}
       \AttributeTok{check\_na =} \ConstantTok{FALSE}\NormalTok{,}
       \AttributeTok{plot\_result=}\ConstantTok{FALSE}\NormalTok{,}
       \AttributeTok{plot\_gaps =} \ConstantTok{FALSE}\NormalTok{,}
       \AttributeTok{overwrite=}\ConstantTok{TRUE}\NormalTok{)}
 
\NormalTok{egvrez}\OtherTok{=}\FunctionTok{input2egv}\NormalTok{(}\AttributeTok{input=}\FunctionTok{paste0}\NormalTok{(}\StringTok{"./RasterGrids\_10m/2024/"}\NormalTok{,localname),}
         \AttributeTok{egv\_template=} \StringTok{"./Templates/TemplateRasters/LV100m\_10km.tif"}\NormalTok{,}
         \AttributeTok{summary\_function =} \StringTok{"average"}\NormalTok{,}
         \AttributeTok{missing\_job =} \StringTok{"FillOutput"}\NormalTok{,}
         \AttributeTok{input\_template =} \StringTok{"./Templates/TemplateRasters/LV10m\_10km.tif"}\NormalTok{,}
         \AttributeTok{outlocation =} \StringTok{"./RasterGrids\_100m/2024/RAW/"}\NormalTok{,}
         \AttributeTok{outfilename =}\NormalTok{ localname,}
         \AttributeTok{layername =}\NormalTok{ layername,}
         \AttributeTok{idw\_weight =} \DecValTok{2}\NormalTok{,}
         \AttributeTok{plot\_gaps =} \ConstantTok{FALSE}\NormalTok{,}\AttributeTok{plot\_final =} \ConstantTok{FALSE}\NormalTok{)}
\NormalTok{egvrez}
 
\FunctionTok{unlink}\NormalTok{(}\FunctionTok{paste0}\NormalTok{(}\StringTok{"./RasterGrids\_10m/2024/"}\NormalTok{,localname))}

\CommentTok{\# standardisation {-}{-}{-}{-}}
\ControlFlowTok{if}\NormalTok{(}\SpecialCharTok{!}\FunctionTok{require}\NormalTok{(terra)) \{}\FunctionTok{install.packages}\NormalTok{(}\StringTok{"terra"}\NormalTok{); }\FunctionTok{require}\NormalTok{(terra)\}}
\ControlFlowTok{if}\NormalTok{(}\SpecialCharTok{!}\FunctionTok{require}\NormalTok{(tidyverse)) \{}\FunctionTok{install.packages}\NormalTok{(}\StringTok{"tidyverse"}\NormalTok{); }\FunctionTok{require}\NormalTok{(tidyverse)\}}

\NormalTok{nosaukums}\OtherTok{=}\StringTok{"HydroClim\_01{-}max\_cell.tif"}
\NormalTok{ielasisanas\_cels}\OtherTok{=}\FunctionTok{paste0}\NormalTok{(}\StringTok{"./RasterGrids\_100m/2024/RAW/"}\NormalTok{,nosaukums)}
\NormalTok{saglabasanas\_cels}\OtherTok{=}\FunctionTok{paste0}\NormalTok{(}\StringTok{"./RasterGrids\_100m/2024/Scaled/"}\NormalTok{,nosaukums)}
\NormalTok{slanis}\OtherTok{=}\FunctionTok{rast}\NormalTok{(ielasisanas\_cels)}
\NormalTok{videjais}\OtherTok{=}\FunctionTok{global}\NormalTok{(slanis,}\AttributeTok{fun=}\StringTok{"mean"}\NormalTok{,}\AttributeTok{na.rm=}\ConstantTok{TRUE}\NormalTok{)}
\NormalTok{centrets}\OtherTok{=}\NormalTok{slanis}\SpecialCharTok{{-}}\NormalTok{videjais[,}\DecValTok{1}\NormalTok{]}
\NormalTok{standartnovirze}\OtherTok{=}\NormalTok{terra}\SpecialCharTok{::}\FunctionTok{global}\NormalTok{(centrets,}\AttributeTok{fun=}\StringTok{"rms"}\NormalTok{,}\AttributeTok{na.rm=}\ConstantTok{TRUE}\NormalTok{)}
\NormalTok{merogots}\OtherTok{=}\NormalTok{centrets}\SpecialCharTok{/}\NormalTok{standartnovirze[,}\DecValTok{1}\NormalTok{]}
\FunctionTok{writeRaster}\NormalTok{(merogots,}
      \AttributeTok{filename=}\NormalTok{saglabasanas\_cels,}
      \AttributeTok{overwrite=}\ConstantTok{TRUE}\NormalTok{)}
\end{Highlighting}
\end{Shaded}

\section{HydroClim\_02-max\_cell}\label{ch06.071}

\textbf{filename:} \texttt{HydroClim\_02-max\_cell.tif}

\textbf{layername:} \texttt{egv\_071}

\textbf{English name:} Maximum per subcatchment upstream mean diurnal air temperature
range (°C) (HydroClim) within the analysis cell (1 ha)

\textbf{Latvian name:} Sateces apakšbaseina maksimālā diennakts gaisa temperatūras
amplitūda augštecē (°C) (HydroClim) analīzes šūnā (1 ha)

\textbf{Procedure:} Information from the \hyperref[Ch04.12]{HydroClim
data} - including both basin and raster layers - is used. First, basin CRS is transformed to EPSG:3059. Then,
zonal statistics (per basin) using a layer specific summary function (max) are
calculated (\texttt{exactextractr::exact\_extract()}), and the the results are rasterised with the workflow
\texttt{egvtools::polygon2input()}. Once rasterised to input data, EGV is created using the workflow
\texttt{egvtools::input2egv()}. To prevent from gaps at the edges, inverse distance
weighted (power = 2) gap filling is implemented. To save disk space,
the intermediate input layer is unlinked. Finally, the layer is standardised by
subtracting the arithmetic mean and dividing by the root mean squared error.

\begin{Shaded}
\begin{Highlighting}[]
\CommentTok{\# libs {-}{-}{-}{-}}
\ControlFlowTok{if}\NormalTok{(}\SpecialCharTok{!}\FunctionTok{require}\NormalTok{(egvtools)) \{remotes}\SpecialCharTok{::}\FunctionTok{install\_github}\NormalTok{(}\StringTok{"aavotins/egvtools"}\NormalTok{); }\FunctionTok{require}\NormalTok{(egvtools)\}}
\ControlFlowTok{if}\NormalTok{(}\SpecialCharTok{!}\FunctionTok{require}\NormalTok{(terra)) \{}\FunctionTok{install.packages}\NormalTok{(}\StringTok{"terra"}\NormalTok{); }\FunctionTok{require}\NormalTok{(terra)\}}
\ControlFlowTok{if}\NormalTok{(}\SpecialCharTok{!}\FunctionTok{require}\NormalTok{(tidyverse)) \{}\FunctionTok{install.packages}\NormalTok{(}\StringTok{"tidyverse"}\NormalTok{); }\FunctionTok{require}\NormalTok{(tidyverse)\}}
\ControlFlowTok{if}\NormalTok{(}\SpecialCharTok{!}\FunctionTok{require}\NormalTok{(sf)) \{}\FunctionTok{install.packages}\NormalTok{(}\StringTok{"sf"}\NormalTok{); }\FunctionTok{require}\NormalTok{(sf)\}}
\ControlFlowTok{if}\NormalTok{(}\SpecialCharTok{!}\FunctionTok{require}\NormalTok{(sfarrow)) \{}\FunctionTok{install.packages}\NormalTok{(}\StringTok{"sfarrow"}\NormalTok{); }\FunctionTok{require}\NormalTok{(sfarrow)\}}
\ControlFlowTok{if}\NormalTok{(}\SpecialCharTok{!}\FunctionTok{require}\NormalTok{(exactextractr)) \{}\FunctionTok{install.packages}\NormalTok{(}\StringTok{"exactextractr"}\NormalTok{); }\FunctionTok{require}\NormalTok{(exactextractr)\}}

\CommentTok{\# basins {-}{-}{-}{-}}
\NormalTok{level12}\OtherTok{=}\FunctionTok{st\_read}\NormalTok{(}\StringTok{"./Geodata/2024/HydroClim/hybas\_lake\_eu\_lev01{-}12\_v1c/hybas\_lake\_eu\_lev12\_v1c.shp"}\NormalTok{)}
\NormalTok{grid\_1km}\OtherTok{=}\NormalTok{sfarrow}\SpecialCharTok{::}\FunctionTok{st\_read\_parquet}\NormalTok{(}\StringTok{"./Templates/TemplateGrids/tikls1km\_sauzeme.parquet"}\NormalTok{)}
\NormalTok{grid\_1km}\OtherTok{=}\FunctionTok{st\_transform}\NormalTok{(grid\_1km,}\AttributeTok{crs=}\DecValTok{3059}\NormalTok{)}
\NormalTok{level12}\OtherTok{=}\FunctionTok{st\_transform}\NormalTok{(level12,}\AttributeTok{crs=}\DecValTok{3059}\NormalTok{)}
\NormalTok{level12}\OtherTok{=}\NormalTok{level12[grid\_1km,,]}

\NormalTok{level12}\OtherTok{=}\FunctionTok{st\_make\_valid}\NormalTok{(level12)}

\CommentTok{\# job {-}{-}{-}{-}}

\NormalTok{localname}\OtherTok{=}\StringTok{"HydroClim\_02{-}max\_cell.tif"}
\NormalTok{layername}\OtherTok{=}\StringTok{"egv\_071"}
\NormalTok{summary\_function}\OtherTok{=}\StringTok{"max"}
 
\NormalTok{slanis}\OtherTok{=}\FunctionTok{rast}\NormalTok{(}\FunctionTok{paste0}\NormalTok{(}\StringTok{"./Geodata/2024/HydroClim/"}\NormalTok{,localname))}
\NormalTok{level12}\SpecialCharTok{$}\NormalTok{Hydro\_values}\OtherTok{=}\FunctionTok{exact\_extract}\NormalTok{(slanis,level12,}\AttributeTok{fun=}\NormalTok{summary\_function)}
 
\FunctionTok{polygon2input}\NormalTok{(}\AttributeTok{vector\_data =}\NormalTok{ level12,}
       \AttributeTok{template\_path =} \StringTok{"./Templates/TemplateRasters/LV10m\_10km.tif"}\NormalTok{,}
       \AttributeTok{out\_path =} \StringTok{"./RasterGrids\_10m/2024/"}\NormalTok{,}
       \AttributeTok{file\_name =}\NormalTok{ localname,}
       \AttributeTok{value\_field =} \StringTok{"Hydro\_values"}\NormalTok{,}
       \AttributeTok{fun=}\StringTok{"first"}\NormalTok{,}
       \AttributeTok{value\_type =} \StringTok{"continuous"}\NormalTok{,}
       \AttributeTok{prepare=}\ConstantTok{FALSE}\NormalTok{,}
       \AttributeTok{project\_mode =} \StringTok{"auto"}\NormalTok{,}
       \AttributeTok{check\_na =} \ConstantTok{FALSE}\NormalTok{,}
       \AttributeTok{plot\_result=}\ConstantTok{FALSE}\NormalTok{,}
       \AttributeTok{plot\_gaps =} \ConstantTok{FALSE}\NormalTok{,}
       \AttributeTok{overwrite=}\ConstantTok{TRUE}\NormalTok{)}
 
\NormalTok{egvrez}\OtherTok{=}\FunctionTok{input2egv}\NormalTok{(}\AttributeTok{input=}\FunctionTok{paste0}\NormalTok{(}\StringTok{"./RasterGrids\_10m/2024/"}\NormalTok{,localname),}
         \AttributeTok{egv\_template=} \StringTok{"./Templates/TemplateRasters/LV100m\_10km.tif"}\NormalTok{,}
         \AttributeTok{summary\_function =} \StringTok{"average"}\NormalTok{,}
         \AttributeTok{missing\_job =} \StringTok{"FillOutput"}\NormalTok{,}
         \AttributeTok{input\_template =} \StringTok{"./Templates/TemplateRasters/LV10m\_10km.tif"}\NormalTok{,}
         \AttributeTok{outlocation =} \StringTok{"./RasterGrids\_100m/2024/RAW/"}\NormalTok{,}
         \AttributeTok{outfilename =}\NormalTok{ localname,}
         \AttributeTok{layername =}\NormalTok{ layername,}
         \AttributeTok{idw\_weight =} \DecValTok{2}\NormalTok{,}
         \AttributeTok{plot\_gaps =} \ConstantTok{FALSE}\NormalTok{,}\AttributeTok{plot\_final =} \ConstantTok{FALSE}\NormalTok{)}
\NormalTok{egvrez}
 
\FunctionTok{unlink}\NormalTok{(}\FunctionTok{paste0}\NormalTok{(}\StringTok{"./RasterGrids\_10m/2024/"}\NormalTok{,localname))}

\CommentTok{\# standardisation {-}{-}{-}{-}}
\ControlFlowTok{if}\NormalTok{(}\SpecialCharTok{!}\FunctionTok{require}\NormalTok{(terra)) \{}\FunctionTok{install.packages}\NormalTok{(}\StringTok{"terra"}\NormalTok{); }\FunctionTok{require}\NormalTok{(terra)\}}
\ControlFlowTok{if}\NormalTok{(}\SpecialCharTok{!}\FunctionTok{require}\NormalTok{(tidyverse)) \{}\FunctionTok{install.packages}\NormalTok{(}\StringTok{"tidyverse"}\NormalTok{); }\FunctionTok{require}\NormalTok{(tidyverse)\}}

\NormalTok{nosaukums}\OtherTok{=}\StringTok{"HydroClim\_02{-}max\_cell.tif"}
\NormalTok{ielasisanas\_cels}\OtherTok{=}\FunctionTok{paste0}\NormalTok{(}\StringTok{"./RasterGrids\_100m/2024/RAW/"}\NormalTok{,nosaukums)}
\NormalTok{saglabasanas\_cels}\OtherTok{=}\FunctionTok{paste0}\NormalTok{(}\StringTok{"./RasterGrids\_100m/2024/Scaled/"}\NormalTok{,nosaukums)}
\NormalTok{slanis}\OtherTok{=}\FunctionTok{rast}\NormalTok{(ielasisanas\_cels)}
\NormalTok{videjais}\OtherTok{=}\FunctionTok{global}\NormalTok{(slanis,}\AttributeTok{fun=}\StringTok{"mean"}\NormalTok{,}\AttributeTok{na.rm=}\ConstantTok{TRUE}\NormalTok{)}
\NormalTok{centrets}\OtherTok{=}\NormalTok{slanis}\SpecialCharTok{{-}}\NormalTok{videjais[,}\DecValTok{1}\NormalTok{]}
\NormalTok{standartnovirze}\OtherTok{=}\NormalTok{terra}\SpecialCharTok{::}\FunctionTok{global}\NormalTok{(centrets,}\AttributeTok{fun=}\StringTok{"rms"}\NormalTok{,}\AttributeTok{na.rm=}\ConstantTok{TRUE}\NormalTok{)}
\NormalTok{merogots}\OtherTok{=}\NormalTok{centrets}\SpecialCharTok{/}\NormalTok{standartnovirze[,}\DecValTok{1}\NormalTok{]}
\FunctionTok{writeRaster}\NormalTok{(merogots,}
      \AttributeTok{filename=}\NormalTok{saglabasanas\_cels,}
      \AttributeTok{overwrite=}\ConstantTok{TRUE}\NormalTok{)}
\end{Highlighting}
\end{Shaded}

\section{HydroClim\_03-max\_cell}\label{ch06.072}

\textbf{filename:} \texttt{HydroClim\_03-max\_cell.tif}

\textbf{layername:} \texttt{egv\_072}

\textbf{English name:} Maximum per subcatchment upstream isothermality (ratio of
diurnal variation to annual variation in temperatures) (°C) (HydroClim) within
the analysis cell (1 ha)

\textbf{Latvian name:} Sateces apakšbaseina maksimālā izotermalitāte (diennakts un gada
temperatūru variabilitātes attiecība) augštecē (°C)
(HydroClim) analīzes šūnā (1 ha)

\textbf{Procedure:} Information from the \hyperref[Ch04.12]{HydroClim
data} - including both basin and raster layers - is used. First, basin CRS is transformed to EPSG:3059. Then,
zonal statistics (per basin) using a layer specific summary function (max) are
calculated (\texttt{exactextractr::exact\_extract()}), and the the results are rasterised with the workflow
\texttt{egvtools::polygon2input()}. Once rasterised to input data, EGV is created using the workflow
\texttt{egvtools::input2egv()}. To prevent from gaps at the edges, inverse distance
weighted (power = 2) gap filling is implemented. To save disk space,
the intermediate input layer is unlinked. Finally, the layer is standardised by
subtracting the arithmetic mean and dividing by the root mean squared error.

\begin{Shaded}
\begin{Highlighting}[]
\CommentTok{\# libs {-}{-}{-}{-}}
\ControlFlowTok{if}\NormalTok{(}\SpecialCharTok{!}\FunctionTok{require}\NormalTok{(egvtools)) \{remotes}\SpecialCharTok{::}\FunctionTok{install\_github}\NormalTok{(}\StringTok{"aavotins/egvtools"}\NormalTok{); }\FunctionTok{require}\NormalTok{(egvtools)\}}
\ControlFlowTok{if}\NormalTok{(}\SpecialCharTok{!}\FunctionTok{require}\NormalTok{(terra)) \{}\FunctionTok{install.packages}\NormalTok{(}\StringTok{"terra"}\NormalTok{); }\FunctionTok{require}\NormalTok{(terra)\}}
\ControlFlowTok{if}\NormalTok{(}\SpecialCharTok{!}\FunctionTok{require}\NormalTok{(tidyverse)) \{}\FunctionTok{install.packages}\NormalTok{(}\StringTok{"tidyverse"}\NormalTok{); }\FunctionTok{require}\NormalTok{(tidyverse)\}}
\ControlFlowTok{if}\NormalTok{(}\SpecialCharTok{!}\FunctionTok{require}\NormalTok{(sf)) \{}\FunctionTok{install.packages}\NormalTok{(}\StringTok{"sf"}\NormalTok{); }\FunctionTok{require}\NormalTok{(sf)\}}
\ControlFlowTok{if}\NormalTok{(}\SpecialCharTok{!}\FunctionTok{require}\NormalTok{(sfarrow)) \{}\FunctionTok{install.packages}\NormalTok{(}\StringTok{"sfarrow"}\NormalTok{); }\FunctionTok{require}\NormalTok{(sfarrow)\}}
\ControlFlowTok{if}\NormalTok{(}\SpecialCharTok{!}\FunctionTok{require}\NormalTok{(exactextractr)) \{}\FunctionTok{install.packages}\NormalTok{(}\StringTok{"exactextractr"}\NormalTok{); }\FunctionTok{require}\NormalTok{(exactextractr)\}}

\CommentTok{\# basins {-}{-}{-}{-}}
\NormalTok{level12}\OtherTok{=}\FunctionTok{st\_read}\NormalTok{(}\StringTok{"./Geodata/2024/HydroClim/hybas\_lake\_eu\_lev01{-}12\_v1c/hybas\_lake\_eu\_lev12\_v1c.shp"}\NormalTok{)}
\NormalTok{grid\_1km}\OtherTok{=}\NormalTok{sfarrow}\SpecialCharTok{::}\FunctionTok{st\_read\_parquet}\NormalTok{(}\StringTok{"./Templates/TemplateGrids/tikls1km\_sauzeme.parquet"}\NormalTok{)}
\NormalTok{grid\_1km}\OtherTok{=}\FunctionTok{st\_transform}\NormalTok{(grid\_1km,}\AttributeTok{crs=}\DecValTok{3059}\NormalTok{)}
\NormalTok{level12}\OtherTok{=}\FunctionTok{st\_transform}\NormalTok{(level12,}\AttributeTok{crs=}\DecValTok{3059}\NormalTok{)}
\NormalTok{level12}\OtherTok{=}\NormalTok{level12[grid\_1km,,]}

\NormalTok{level12}\OtherTok{=}\FunctionTok{st\_make\_valid}\NormalTok{(level12)}

\CommentTok{\# job {-}{-}{-}{-}}

\NormalTok{localname}\OtherTok{=}\StringTok{"HydroClim\_03{-}max\_cell.tif"}
\NormalTok{layername}\OtherTok{=}\StringTok{"egv\_072"}
\NormalTok{summary\_function}\OtherTok{=}\StringTok{"max"}
 
\NormalTok{slanis}\OtherTok{=}\FunctionTok{rast}\NormalTok{(}\FunctionTok{paste0}\NormalTok{(}\StringTok{"./Geodata/2024/HydroClim/"}\NormalTok{,localname))}
\NormalTok{level12}\SpecialCharTok{$}\NormalTok{Hydro\_values}\OtherTok{=}\FunctionTok{exact\_extract}\NormalTok{(slanis,level12,}\AttributeTok{fun=}\NormalTok{summary\_function)}
 
\FunctionTok{polygon2input}\NormalTok{(}\AttributeTok{vector\_data =}\NormalTok{ level12,}
       \AttributeTok{template\_path =} \StringTok{"./Templates/TemplateRasters/LV10m\_10km.tif"}\NormalTok{,}
       \AttributeTok{out\_path =} \StringTok{"./RasterGrids\_10m/2024/"}\NormalTok{,}
       \AttributeTok{file\_name =}\NormalTok{ localname,}
       \AttributeTok{value\_field =} \StringTok{"Hydro\_values"}\NormalTok{,}
       \AttributeTok{fun=}\StringTok{"first"}\NormalTok{,}
       \AttributeTok{value\_type =} \StringTok{"continuous"}\NormalTok{,}
       \AttributeTok{prepare=}\ConstantTok{FALSE}\NormalTok{,}
       \AttributeTok{project\_mode =} \StringTok{"auto"}\NormalTok{,}
       \AttributeTok{check\_na =} \ConstantTok{FALSE}\NormalTok{,}
       \AttributeTok{plot\_result=}\ConstantTok{FALSE}\NormalTok{,}
       \AttributeTok{plot\_gaps =} \ConstantTok{FALSE}\NormalTok{,}
       \AttributeTok{overwrite=}\ConstantTok{TRUE}\NormalTok{)}
 
\NormalTok{egvrez}\OtherTok{=}\FunctionTok{input2egv}\NormalTok{(}\AttributeTok{input=}\FunctionTok{paste0}\NormalTok{(}\StringTok{"./RasterGrids\_10m/2024/"}\NormalTok{,localname),}
         \AttributeTok{egv\_template=} \StringTok{"./Templates/TemplateRasters/LV100m\_10km.tif"}\NormalTok{,}
         \AttributeTok{summary\_function =} \StringTok{"average"}\NormalTok{,}
         \AttributeTok{missing\_job =} \StringTok{"FillOutput"}\NormalTok{,}
         \AttributeTok{input\_template =} \StringTok{"./Templates/TemplateRasters/LV10m\_10km.tif"}\NormalTok{,}
         \AttributeTok{outlocation =} \StringTok{"./RasterGrids\_100m/2024/RAW/"}\NormalTok{,}
         \AttributeTok{outfilename =}\NormalTok{ localname,}
         \AttributeTok{layername =}\NormalTok{ layername,}
         \AttributeTok{idw\_weight =} \DecValTok{2}\NormalTok{,}
         \AttributeTok{plot\_gaps =} \ConstantTok{FALSE}\NormalTok{,}\AttributeTok{plot\_final =} \ConstantTok{FALSE}\NormalTok{)}
\NormalTok{egvrez}
 
\FunctionTok{unlink}\NormalTok{(}\FunctionTok{paste0}\NormalTok{(}\StringTok{"./RasterGrids\_10m/2024/"}\NormalTok{,localname))}

\CommentTok{\# standardisation {-}{-}{-}{-}}
\ControlFlowTok{if}\NormalTok{(}\SpecialCharTok{!}\FunctionTok{require}\NormalTok{(terra)) \{}\FunctionTok{install.packages}\NormalTok{(}\StringTok{"terra"}\NormalTok{); }\FunctionTok{require}\NormalTok{(terra)\}}
\ControlFlowTok{if}\NormalTok{(}\SpecialCharTok{!}\FunctionTok{require}\NormalTok{(tidyverse)) \{}\FunctionTok{install.packages}\NormalTok{(}\StringTok{"tidyverse"}\NormalTok{); }\FunctionTok{require}\NormalTok{(tidyverse)\}}

\NormalTok{nosaukums}\OtherTok{=}\StringTok{"HydroClim\_03{-}max\_cell.tif"}
\NormalTok{ielasisanas\_cels}\OtherTok{=}\FunctionTok{paste0}\NormalTok{(}\StringTok{"./RasterGrids\_100m/2024/RAW/"}\NormalTok{,nosaukums)}
\NormalTok{saglabasanas\_cels}\OtherTok{=}\FunctionTok{paste0}\NormalTok{(}\StringTok{"./RasterGrids\_100m/2024/Scaled/"}\NormalTok{,nosaukums)}
\NormalTok{slanis}\OtherTok{=}\FunctionTok{rast}\NormalTok{(ielasisanas\_cels)}
\NormalTok{videjais}\OtherTok{=}\FunctionTok{global}\NormalTok{(slanis,}\AttributeTok{fun=}\StringTok{"mean"}\NormalTok{,}\AttributeTok{na.rm=}\ConstantTok{TRUE}\NormalTok{)}
\NormalTok{centrets}\OtherTok{=}\NormalTok{slanis}\SpecialCharTok{{-}}\NormalTok{videjais[,}\DecValTok{1}\NormalTok{]}
\NormalTok{standartnovirze}\OtherTok{=}\NormalTok{terra}\SpecialCharTok{::}\FunctionTok{global}\NormalTok{(centrets,}\AttributeTok{fun=}\StringTok{"rms"}\NormalTok{,}\AttributeTok{na.rm=}\ConstantTok{TRUE}\NormalTok{)}
\NormalTok{merogots}\OtherTok{=}\NormalTok{centrets}\SpecialCharTok{/}\NormalTok{standartnovirze[,}\DecValTok{1}\NormalTok{]}
\FunctionTok{writeRaster}\NormalTok{(merogots,}
      \AttributeTok{filename=}\NormalTok{saglabasanas\_cels,}
      \AttributeTok{overwrite=}\ConstantTok{TRUE}\NormalTok{)}
\end{Highlighting}
\end{Shaded}

\section{HydroClim\_04-max\_cell}\label{ch06.073}

\textbf{filename:} \texttt{HydroClim\_04-max\_cell.tif}

\textbf{layername:} \texttt{egv\_073}

\textbf{English name:} Maximum per subcatchment upstream temperature seasonality
(standard deviation of the monthly mean temperatures) (°C/100) (HydroClim)
within the analysis cell (1 ha)

\textbf{Latvian name:} Sateces apakšbaseina maksimālā temperatūras sezonalitāte (mēneša vidējo temperatūru standartnovirze)
augštecē (°C/100) (HydroClim) analīzes šūnā (1 ha)

\textbf{Procedure:} Information from the \hyperref[Ch04.12]{HydroClim
data} - including both basin and raster layers - is used. First, basin CRS is transformed to EPSG:3059. Then,
zonal statistics (per basin) using a layer specific summary function (max) are
calculated (\texttt{exactextractr::exact\_extract()}), and the the results are rasterised with the workflow
\texttt{egvtools::polygon2input()}. Once rasterised to input data, EGV is created using the workflow
\texttt{egvtools::input2egv()}. To prevent from gaps at the edges, inverse distance
weighted (power = 2) gap filling is implemented. To save disk space,
the intermediate input layer is unlinked. Finally, the layer is standardised by
subtracting the arithmetic mean and dividing by the root mean squared error.

\begin{Shaded}
\begin{Highlighting}[]
\CommentTok{\# libs {-}{-}{-}{-}}
\ControlFlowTok{if}\NormalTok{(}\SpecialCharTok{!}\FunctionTok{require}\NormalTok{(egvtools)) \{remotes}\SpecialCharTok{::}\FunctionTok{install\_github}\NormalTok{(}\StringTok{"aavotins/egvtools"}\NormalTok{); }\FunctionTok{require}\NormalTok{(egvtools)\}}
\ControlFlowTok{if}\NormalTok{(}\SpecialCharTok{!}\FunctionTok{require}\NormalTok{(terra)) \{}\FunctionTok{install.packages}\NormalTok{(}\StringTok{"terra"}\NormalTok{); }\FunctionTok{require}\NormalTok{(terra)\}}
\ControlFlowTok{if}\NormalTok{(}\SpecialCharTok{!}\FunctionTok{require}\NormalTok{(tidyverse)) \{}\FunctionTok{install.packages}\NormalTok{(}\StringTok{"tidyverse"}\NormalTok{); }\FunctionTok{require}\NormalTok{(tidyverse)\}}
\ControlFlowTok{if}\NormalTok{(}\SpecialCharTok{!}\FunctionTok{require}\NormalTok{(sf)) \{}\FunctionTok{install.packages}\NormalTok{(}\StringTok{"sf"}\NormalTok{); }\FunctionTok{require}\NormalTok{(sf)\}}
\ControlFlowTok{if}\NormalTok{(}\SpecialCharTok{!}\FunctionTok{require}\NormalTok{(sfarrow)) \{}\FunctionTok{install.packages}\NormalTok{(}\StringTok{"sfarrow"}\NormalTok{); }\FunctionTok{require}\NormalTok{(sfarrow)\}}
\ControlFlowTok{if}\NormalTok{(}\SpecialCharTok{!}\FunctionTok{require}\NormalTok{(exactextractr)) \{}\FunctionTok{install.packages}\NormalTok{(}\StringTok{"exactextractr"}\NormalTok{); }\FunctionTok{require}\NormalTok{(exactextractr)\}}

\CommentTok{\# basins {-}{-}{-}{-}}
\NormalTok{level12}\OtherTok{=}\FunctionTok{st\_read}\NormalTok{(}\StringTok{"./Geodata/2024/HydroClim/hybas\_lake\_eu\_lev01{-}12\_v1c/hybas\_lake\_eu\_lev12\_v1c.shp"}\NormalTok{)}
\NormalTok{grid\_1km}\OtherTok{=}\NormalTok{sfarrow}\SpecialCharTok{::}\FunctionTok{st\_read\_parquet}\NormalTok{(}\StringTok{"./Templates/TemplateGrids/tikls1km\_sauzeme.parquet"}\NormalTok{)}
\NormalTok{grid\_1km}\OtherTok{=}\FunctionTok{st\_transform}\NormalTok{(grid\_1km,}\AttributeTok{crs=}\DecValTok{3059}\NormalTok{)}
\NormalTok{level12}\OtherTok{=}\FunctionTok{st\_transform}\NormalTok{(level12,}\AttributeTok{crs=}\DecValTok{3059}\NormalTok{)}
\NormalTok{level12}\OtherTok{=}\NormalTok{level12[grid\_1km,,]}

\NormalTok{level12}\OtherTok{=}\FunctionTok{st\_make\_valid}\NormalTok{(level12)}

\CommentTok{\# job {-}{-}{-}{-}}

\NormalTok{localname}\OtherTok{=}\StringTok{"HydroClim\_04{-}max\_cell.tif"}
\NormalTok{layername}\OtherTok{=}\StringTok{"egv\_073"}
\NormalTok{summary\_function}\OtherTok{=}\StringTok{"max"}
 
\NormalTok{slanis}\OtherTok{=}\FunctionTok{rast}\NormalTok{(}\FunctionTok{paste0}\NormalTok{(}\StringTok{"./Geodata/2024/HydroClim/"}\NormalTok{,localname))}
\NormalTok{level12}\SpecialCharTok{$}\NormalTok{Hydro\_values}\OtherTok{=}\FunctionTok{exact\_extract}\NormalTok{(slanis,level12,}\AttributeTok{fun=}\NormalTok{summary\_function)}
 
\FunctionTok{polygon2input}\NormalTok{(}\AttributeTok{vector\_data =}\NormalTok{ level12,}
       \AttributeTok{template\_path =} \StringTok{"./Templates/TemplateRasters/LV10m\_10km.tif"}\NormalTok{,}
       \AttributeTok{out\_path =} \StringTok{"./RasterGrids\_10m/2024/"}\NormalTok{,}
       \AttributeTok{file\_name =}\NormalTok{ localname,}
       \AttributeTok{value\_field =} \StringTok{"Hydro\_values"}\NormalTok{,}
       \AttributeTok{fun=}\StringTok{"first"}\NormalTok{,}
       \AttributeTok{value\_type =} \StringTok{"continuous"}\NormalTok{,}
       \AttributeTok{prepare=}\ConstantTok{FALSE}\NormalTok{,}
       \AttributeTok{project\_mode =} \StringTok{"auto"}\NormalTok{,}
       \AttributeTok{check\_na =} \ConstantTok{FALSE}\NormalTok{,}
       \AttributeTok{plot\_result=}\ConstantTok{FALSE}\NormalTok{,}
       \AttributeTok{plot\_gaps =} \ConstantTok{FALSE}\NormalTok{,}
       \AttributeTok{overwrite=}\ConstantTok{TRUE}\NormalTok{)}
 
\NormalTok{egvrez}\OtherTok{=}\FunctionTok{input2egv}\NormalTok{(}\AttributeTok{input=}\FunctionTok{paste0}\NormalTok{(}\StringTok{"./RasterGrids\_10m/2024/"}\NormalTok{,localname),}
         \AttributeTok{egv\_template=} \StringTok{"./Templates/TemplateRasters/LV100m\_10km.tif"}\NormalTok{,}
         \AttributeTok{summary\_function =} \StringTok{"average"}\NormalTok{,}
         \AttributeTok{missing\_job =} \StringTok{"FillOutput"}\NormalTok{,}
         \AttributeTok{input\_template =} \StringTok{"./Templates/TemplateRasters/LV10m\_10km.tif"}\NormalTok{,}
         \AttributeTok{outlocation =} \StringTok{"./RasterGrids\_100m/2024/RAW/"}\NormalTok{,}
         \AttributeTok{outfilename =}\NormalTok{ localname,}
         \AttributeTok{layername =}\NormalTok{ layername,}
         \AttributeTok{idw\_weight =} \DecValTok{2}\NormalTok{,}
         \AttributeTok{plot\_gaps =} \ConstantTok{FALSE}\NormalTok{,}\AttributeTok{plot\_final =} \ConstantTok{FALSE}\NormalTok{)}
\NormalTok{egvrez}
 
\FunctionTok{unlink}\NormalTok{(}\FunctionTok{paste0}\NormalTok{(}\StringTok{"./RasterGrids\_10m/2024/"}\NormalTok{,localname))}

\CommentTok{\# standardisation {-}{-}{-}{-}}
\ControlFlowTok{if}\NormalTok{(}\SpecialCharTok{!}\FunctionTok{require}\NormalTok{(terra)) \{}\FunctionTok{install.packages}\NormalTok{(}\StringTok{"terra"}\NormalTok{); }\FunctionTok{require}\NormalTok{(terra)\}}
\ControlFlowTok{if}\NormalTok{(}\SpecialCharTok{!}\FunctionTok{require}\NormalTok{(tidyverse)) \{}\FunctionTok{install.packages}\NormalTok{(}\StringTok{"tidyverse"}\NormalTok{); }\FunctionTok{require}\NormalTok{(tidyverse)\}}

\NormalTok{nosaukums}\OtherTok{=}\StringTok{"HydroClim\_04{-}max\_cell.tif"}
\NormalTok{ielasisanas\_cels}\OtherTok{=}\FunctionTok{paste0}\NormalTok{(}\StringTok{"./RasterGrids\_100m/2024/RAW/"}\NormalTok{,nosaukums)}
\NormalTok{saglabasanas\_cels}\OtherTok{=}\FunctionTok{paste0}\NormalTok{(}\StringTok{"./RasterGrids\_100m/2024/Scaled/"}\NormalTok{,nosaukums)}
\NormalTok{slanis}\OtherTok{=}\FunctionTok{rast}\NormalTok{(ielasisanas\_cels)}
\NormalTok{videjais}\OtherTok{=}\FunctionTok{global}\NormalTok{(slanis,}\AttributeTok{fun=}\StringTok{"mean"}\NormalTok{,}\AttributeTok{na.rm=}\ConstantTok{TRUE}\NormalTok{)}
\NormalTok{centrets}\OtherTok{=}\NormalTok{slanis}\SpecialCharTok{{-}}\NormalTok{videjais[,}\DecValTok{1}\NormalTok{]}
\NormalTok{standartnovirze}\OtherTok{=}\NormalTok{terra}\SpecialCharTok{::}\FunctionTok{global}\NormalTok{(centrets,}\AttributeTok{fun=}\StringTok{"rms"}\NormalTok{,}\AttributeTok{na.rm=}\ConstantTok{TRUE}\NormalTok{)}
\NormalTok{merogots}\OtherTok{=}\NormalTok{centrets}\SpecialCharTok{/}\NormalTok{standartnovirze[,}\DecValTok{1}\NormalTok{]}
\FunctionTok{writeRaster}\NormalTok{(merogots,}
      \AttributeTok{filename=}\NormalTok{saglabasanas\_cels,}
      \AttributeTok{overwrite=}\ConstantTok{TRUE}\NormalTok{)}
\end{Highlighting}
\end{Shaded}

\section{HydroClim\_05-max\_cell}\label{ch06.074}

\textbf{filename:} \texttt{HydroClim\_05-max\_cell.tif}

\textbf{layername:} \texttt{egv\_074}

\textbf{English name:} Maximum per subcatchment upstream mean daily maximum air
temperature (°C) of the warmest month (HydroClim) within the analysis cell (1
ha)

\textbf{Latvian name:} Sateces apakšbaseina maksimālā augšteces dienas augstākā gaisa
temperatūra siltākajā mēnesī (°C) (HydroClim) analīzes šūnā (1 ha)

\textbf{Procedure:} Information from the \hyperref[Ch04.12]{HydroClim
data} - including both basin and raster layers - is used. First, basin CRS is transformed to EPSG:3059. Then,
zonal statistics (per basin) using a layer specific summary function (max) are
calculated (\texttt{exactextractr::exact\_extract()}), and the the results are rasterised with the workflow
\texttt{egvtools::polygon2input()}. Once rasterised to input data, EGV is created using the workflow
\texttt{egvtools::input2egv()}. To prevent from gaps at the edges, inverse distance
weighted (power = 2) gap filling is implemented. To save disk space,
the intermediate input layer is unlinked. Finally, the layer is standardised by
subtracting the arithmetic mean and dividing by the root mean squared error.

\begin{Shaded}
\begin{Highlighting}[]
\CommentTok{\# libs {-}{-}{-}{-}}
\ControlFlowTok{if}\NormalTok{(}\SpecialCharTok{!}\FunctionTok{require}\NormalTok{(egvtools)) \{remotes}\SpecialCharTok{::}\FunctionTok{install\_github}\NormalTok{(}\StringTok{"aavotins/egvtools"}\NormalTok{); }\FunctionTok{require}\NormalTok{(egvtools)\}}
\ControlFlowTok{if}\NormalTok{(}\SpecialCharTok{!}\FunctionTok{require}\NormalTok{(terra)) \{}\FunctionTok{install.packages}\NormalTok{(}\StringTok{"terra"}\NormalTok{); }\FunctionTok{require}\NormalTok{(terra)\}}
\ControlFlowTok{if}\NormalTok{(}\SpecialCharTok{!}\FunctionTok{require}\NormalTok{(tidyverse)) \{}\FunctionTok{install.packages}\NormalTok{(}\StringTok{"tidyverse"}\NormalTok{); }\FunctionTok{require}\NormalTok{(tidyverse)\}}
\ControlFlowTok{if}\NormalTok{(}\SpecialCharTok{!}\FunctionTok{require}\NormalTok{(sf)) \{}\FunctionTok{install.packages}\NormalTok{(}\StringTok{"sf"}\NormalTok{); }\FunctionTok{require}\NormalTok{(sf)\}}
\ControlFlowTok{if}\NormalTok{(}\SpecialCharTok{!}\FunctionTok{require}\NormalTok{(sfarrow)) \{}\FunctionTok{install.packages}\NormalTok{(}\StringTok{"sfarrow"}\NormalTok{); }\FunctionTok{require}\NormalTok{(sfarrow)\}}
\ControlFlowTok{if}\NormalTok{(}\SpecialCharTok{!}\FunctionTok{require}\NormalTok{(exactextractr)) \{}\FunctionTok{install.packages}\NormalTok{(}\StringTok{"exactextractr"}\NormalTok{); }\FunctionTok{require}\NormalTok{(exactextractr)\}}

\CommentTok{\# basins {-}{-}{-}{-}}
\NormalTok{level12}\OtherTok{=}\FunctionTok{st\_read}\NormalTok{(}\StringTok{"./Geodata/2024/HydroClim/hybas\_lake\_eu\_lev01{-}12\_v1c/hybas\_lake\_eu\_lev12\_v1c.shp"}\NormalTok{)}
\NormalTok{grid\_1km}\OtherTok{=}\NormalTok{sfarrow}\SpecialCharTok{::}\FunctionTok{st\_read\_parquet}\NormalTok{(}\StringTok{"./Templates/TemplateGrids/tikls1km\_sauzeme.parquet"}\NormalTok{)}
\NormalTok{grid\_1km}\OtherTok{=}\FunctionTok{st\_transform}\NormalTok{(grid\_1km,}\AttributeTok{crs=}\DecValTok{3059}\NormalTok{)}
\NormalTok{level12}\OtherTok{=}\FunctionTok{st\_transform}\NormalTok{(level12,}\AttributeTok{crs=}\DecValTok{3059}\NormalTok{)}
\NormalTok{level12}\OtherTok{=}\NormalTok{level12[grid\_1km,,]}

\NormalTok{level12}\OtherTok{=}\FunctionTok{st\_make\_valid}\NormalTok{(level12)}

\CommentTok{\# job {-}{-}{-}{-}}

\NormalTok{localname}\OtherTok{=}\StringTok{"HydroClim\_05{-}max\_cell.tif"}
\NormalTok{layername}\OtherTok{=}\StringTok{"egv\_074"}
\NormalTok{summary\_function}\OtherTok{=}\StringTok{"max"}
 
\NormalTok{slanis}\OtherTok{=}\FunctionTok{rast}\NormalTok{(}\FunctionTok{paste0}\NormalTok{(}\StringTok{"./Geodata/2024/HydroClim/"}\NormalTok{,localname))}
\NormalTok{level12}\SpecialCharTok{$}\NormalTok{Hydro\_values}\OtherTok{=}\FunctionTok{exact\_extract}\NormalTok{(slanis,level12,}\AttributeTok{fun=}\NormalTok{summary\_function)}
 
\FunctionTok{polygon2input}\NormalTok{(}\AttributeTok{vector\_data =}\NormalTok{ level12,}
       \AttributeTok{template\_path =} \StringTok{"./Templates/TemplateRasters/LV10m\_10km.tif"}\NormalTok{,}
       \AttributeTok{out\_path =} \StringTok{"./RasterGrids\_10m/2024/"}\NormalTok{,}
       \AttributeTok{file\_name =}\NormalTok{ localname,}
       \AttributeTok{value\_field =} \StringTok{"Hydro\_values"}\NormalTok{,}
       \AttributeTok{fun=}\StringTok{"first"}\NormalTok{,}
       \AttributeTok{value\_type =} \StringTok{"continuous"}\NormalTok{,}
       \AttributeTok{prepare=}\ConstantTok{FALSE}\NormalTok{,}
       \AttributeTok{project\_mode =} \StringTok{"auto"}\NormalTok{,}
       \AttributeTok{check\_na =} \ConstantTok{FALSE}\NormalTok{,}
       \AttributeTok{plot\_result=}\ConstantTok{FALSE}\NormalTok{,}
       \AttributeTok{plot\_gaps =} \ConstantTok{FALSE}\NormalTok{,}
       \AttributeTok{overwrite=}\ConstantTok{TRUE}\NormalTok{)}
 
\NormalTok{egvrez}\OtherTok{=}\FunctionTok{input2egv}\NormalTok{(}\AttributeTok{input=}\FunctionTok{paste0}\NormalTok{(}\StringTok{"./RasterGrids\_10m/2024/"}\NormalTok{,localname),}
         \AttributeTok{egv\_template=} \StringTok{"./Templates/TemplateRasters/LV100m\_10km.tif"}\NormalTok{,}
         \AttributeTok{summary\_function =} \StringTok{"average"}\NormalTok{,}
         \AttributeTok{missing\_job =} \StringTok{"FillOutput"}\NormalTok{,}
         \AttributeTok{input\_template =} \StringTok{"./Templates/TemplateRasters/LV10m\_10km.tif"}\NormalTok{,}
         \AttributeTok{outlocation =} \StringTok{"./RasterGrids\_100m/2024/RAW/"}\NormalTok{,}
         \AttributeTok{outfilename =}\NormalTok{ localname,}
         \AttributeTok{layername =}\NormalTok{ layername,}
         \AttributeTok{idw\_weight =} \DecValTok{2}\NormalTok{,}
         \AttributeTok{plot\_gaps =} \ConstantTok{FALSE}\NormalTok{,}\AttributeTok{plot\_final =} \ConstantTok{FALSE}\NormalTok{)}
\NormalTok{egvrez}
 
\FunctionTok{unlink}\NormalTok{(}\FunctionTok{paste0}\NormalTok{(}\StringTok{"./RasterGrids\_10m/2024/"}\NormalTok{,localname))}

\CommentTok{\# standardisation {-}{-}{-}{-}}
\ControlFlowTok{if}\NormalTok{(}\SpecialCharTok{!}\FunctionTok{require}\NormalTok{(terra)) \{}\FunctionTok{install.packages}\NormalTok{(}\StringTok{"terra"}\NormalTok{); }\FunctionTok{require}\NormalTok{(terra)\}}
\ControlFlowTok{if}\NormalTok{(}\SpecialCharTok{!}\FunctionTok{require}\NormalTok{(tidyverse)) \{}\FunctionTok{install.packages}\NormalTok{(}\StringTok{"tidyverse"}\NormalTok{); }\FunctionTok{require}\NormalTok{(tidyverse)\}}

\NormalTok{nosaukums}\OtherTok{=}\StringTok{"HydroClim\_05{-}max\_cell.tif"}
\NormalTok{ielasisanas\_cels}\OtherTok{=}\FunctionTok{paste0}\NormalTok{(}\StringTok{"./RasterGrids\_100m/2024/RAW/"}\NormalTok{,nosaukums)}
\NormalTok{saglabasanas\_cels}\OtherTok{=}\FunctionTok{paste0}\NormalTok{(}\StringTok{"./RasterGrids\_100m/2024/Scaled/"}\NormalTok{,nosaukums)}
\NormalTok{slanis}\OtherTok{=}\FunctionTok{rast}\NormalTok{(ielasisanas\_cels)}
\NormalTok{videjais}\OtherTok{=}\FunctionTok{global}\NormalTok{(slanis,}\AttributeTok{fun=}\StringTok{"mean"}\NormalTok{,}\AttributeTok{na.rm=}\ConstantTok{TRUE}\NormalTok{)}
\NormalTok{centrets}\OtherTok{=}\NormalTok{slanis}\SpecialCharTok{{-}}\NormalTok{videjais[,}\DecValTok{1}\NormalTok{]}
\NormalTok{standartnovirze}\OtherTok{=}\NormalTok{terra}\SpecialCharTok{::}\FunctionTok{global}\NormalTok{(centrets,}\AttributeTok{fun=}\StringTok{"rms"}\NormalTok{,}\AttributeTok{na.rm=}\ConstantTok{TRUE}\NormalTok{)}
\NormalTok{merogots}\OtherTok{=}\NormalTok{centrets}\SpecialCharTok{/}\NormalTok{standartnovirze[,}\DecValTok{1}\NormalTok{]}
\FunctionTok{writeRaster}\NormalTok{(merogots,}
      \AttributeTok{filename=}\NormalTok{saglabasanas\_cels,}
      \AttributeTok{overwrite=}\ConstantTok{TRUE}\NormalTok{)}
\end{Highlighting}
\end{Shaded}

\section{HydroClim\_06-min\_cell}\label{ch06.075}

\textbf{filename:} \texttt{HydroClim\_06-min\_cell.tif}

\textbf{layername:} \texttt{egv\_075}

\textbf{English name:} Minimum per subcatchment upstream mean daily minimum air
temperature (°C) of the coldest month (HydroClim) within the analysis cell (1
ha)

\textbf{Latvian name:} Sateces apakšbaseina minimālā augšteces dienas zemākā gaisa
temperatūra vēsākajā mēnesī (°C) (HydroClim) analīzes šūnā (1 ha)

\textbf{Procedure:} Information from the \hyperref[Ch04.12]{HydroClim
data} - including both basin and raster layers - is used. First, basin CRS is transformed to EPSG:3059. Then,
zonal statistics (per basin) using a layer specific summary function (min) are
calculated (\texttt{exactextractr::exact\_extract()}), and the the results are rasterised with the workflow
\texttt{egvtools::polygon2input()}. Once rasterised to input data, EGV is created using the workflow
\texttt{egvtools::input2egv()}. To prevent from gaps at the edges, inverse distance
weighted (power = 2) gap filling is implemented. To save disk space,
the intermediate input layer is unlinked. Finally, the layer is standardised by
subtracting the arithmetic mean and dividing by the root mean squared error.

\begin{Shaded}
\begin{Highlighting}[]
\CommentTok{\# libs {-}{-}{-}{-}}
\ControlFlowTok{if}\NormalTok{(}\SpecialCharTok{!}\FunctionTok{require}\NormalTok{(egvtools)) \{remotes}\SpecialCharTok{::}\FunctionTok{install\_github}\NormalTok{(}\StringTok{"aavotins/egvtools"}\NormalTok{); }\FunctionTok{require}\NormalTok{(egvtools)\}}
\ControlFlowTok{if}\NormalTok{(}\SpecialCharTok{!}\FunctionTok{require}\NormalTok{(terra)) \{}\FunctionTok{install.packages}\NormalTok{(}\StringTok{"terra"}\NormalTok{); }\FunctionTok{require}\NormalTok{(terra)\}}
\ControlFlowTok{if}\NormalTok{(}\SpecialCharTok{!}\FunctionTok{require}\NormalTok{(tidyverse)) \{}\FunctionTok{install.packages}\NormalTok{(}\StringTok{"tidyverse"}\NormalTok{); }\FunctionTok{require}\NormalTok{(tidyverse)\}}
\ControlFlowTok{if}\NormalTok{(}\SpecialCharTok{!}\FunctionTok{require}\NormalTok{(sf)) \{}\FunctionTok{install.packages}\NormalTok{(}\StringTok{"sf"}\NormalTok{); }\FunctionTok{require}\NormalTok{(sf)\}}
\ControlFlowTok{if}\NormalTok{(}\SpecialCharTok{!}\FunctionTok{require}\NormalTok{(sfarrow)) \{}\FunctionTok{install.packages}\NormalTok{(}\StringTok{"sfarrow"}\NormalTok{); }\FunctionTok{require}\NormalTok{(sfarrow)\}}
\ControlFlowTok{if}\NormalTok{(}\SpecialCharTok{!}\FunctionTok{require}\NormalTok{(exactextractr)) \{}\FunctionTok{install.packages}\NormalTok{(}\StringTok{"exactextractr"}\NormalTok{); }\FunctionTok{require}\NormalTok{(exactextractr)\}}

\CommentTok{\# basins {-}{-}{-}{-}}
\NormalTok{level12}\OtherTok{=}\FunctionTok{st\_read}\NormalTok{(}\StringTok{"./Geodata/2024/HydroClim/hybas\_lake\_eu\_lev01{-}12\_v1c/hybas\_lake\_eu\_lev12\_v1c.shp"}\NormalTok{)}
\NormalTok{grid\_1km}\OtherTok{=}\NormalTok{sfarrow}\SpecialCharTok{::}\FunctionTok{st\_read\_parquet}\NormalTok{(}\StringTok{"./Templates/TemplateGrids/tikls1km\_sauzeme.parquet"}\NormalTok{)}
\NormalTok{grid\_1km}\OtherTok{=}\FunctionTok{st\_transform}\NormalTok{(grid\_1km,}\AttributeTok{crs=}\DecValTok{3059}\NormalTok{)}
\NormalTok{level12}\OtherTok{=}\FunctionTok{st\_transform}\NormalTok{(level12,}\AttributeTok{crs=}\DecValTok{3059}\NormalTok{)}
\NormalTok{level12}\OtherTok{=}\NormalTok{level12[grid\_1km,,]}

\NormalTok{level12}\OtherTok{=}\FunctionTok{st\_make\_valid}\NormalTok{(level12)}

\CommentTok{\# job {-}{-}{-}{-}}

\NormalTok{localname}\OtherTok{=}\StringTok{"HydroClim\_06{-}min\_cell.tif"}
\NormalTok{layername}\OtherTok{=}\StringTok{"egv\_075"}
\NormalTok{summary\_function}\OtherTok{=}\StringTok{"min"}
 
\NormalTok{slanis}\OtherTok{=}\FunctionTok{rast}\NormalTok{(}\FunctionTok{paste0}\NormalTok{(}\StringTok{"./Geodata/2024/HydroClim/"}\NormalTok{,localname))}
\NormalTok{level12}\SpecialCharTok{$}\NormalTok{Hydro\_values}\OtherTok{=}\FunctionTok{exact\_extract}\NormalTok{(slanis,level12,}\AttributeTok{fun=}\NormalTok{summary\_function)}
 
\FunctionTok{polygon2input}\NormalTok{(}\AttributeTok{vector\_data =}\NormalTok{ level12,}
       \AttributeTok{template\_path =} \StringTok{"./Templates/TemplateRasters/LV10m\_10km.tif"}\NormalTok{,}
       \AttributeTok{out\_path =} \StringTok{"./RasterGrids\_10m/2024/"}\NormalTok{,}
       \AttributeTok{file\_name =}\NormalTok{ localname,}
       \AttributeTok{value\_field =} \StringTok{"Hydro\_values"}\NormalTok{,}
       \AttributeTok{fun=}\StringTok{"first"}\NormalTok{,}
       \AttributeTok{value\_type =} \StringTok{"continuous"}\NormalTok{,}
       \AttributeTok{prepare=}\ConstantTok{FALSE}\NormalTok{,}
       \AttributeTok{project\_mode =} \StringTok{"auto"}\NormalTok{,}
       \AttributeTok{check\_na =} \ConstantTok{FALSE}\NormalTok{,}
       \AttributeTok{plot\_result=}\ConstantTok{FALSE}\NormalTok{,}
       \AttributeTok{plot\_gaps =} \ConstantTok{FALSE}\NormalTok{,}
       \AttributeTok{overwrite=}\ConstantTok{TRUE}\NormalTok{)}
 
\NormalTok{egvrez}\OtherTok{=}\FunctionTok{input2egv}\NormalTok{(}\AttributeTok{input=}\FunctionTok{paste0}\NormalTok{(}\StringTok{"./RasterGrids\_10m/2024/"}\NormalTok{,localname),}
         \AttributeTok{egv\_template=} \StringTok{"./Templates/TemplateRasters/LV100m\_10km.tif"}\NormalTok{,}
         \AttributeTok{summary\_function =} \StringTok{"average"}\NormalTok{,}
         \AttributeTok{missing\_job =} \StringTok{"FillOutput"}\NormalTok{,}
         \AttributeTok{input\_template =} \StringTok{"./Templates/TemplateRasters/LV10m\_10km.tif"}\NormalTok{,}
         \AttributeTok{outlocation =} \StringTok{"./RasterGrids\_100m/2024/RAW/"}\NormalTok{,}
         \AttributeTok{outfilename =}\NormalTok{ localname,}
         \AttributeTok{layername =}\NormalTok{ layername,}
         \AttributeTok{idw\_weight =} \DecValTok{2}\NormalTok{,}
         \AttributeTok{plot\_gaps =} \ConstantTok{FALSE}\NormalTok{,}\AttributeTok{plot\_final =} \ConstantTok{FALSE}\NormalTok{)}
\NormalTok{egvrez}
 
\FunctionTok{unlink}\NormalTok{(}\FunctionTok{paste0}\NormalTok{(}\StringTok{"./RasterGrids\_10m/2024/"}\NormalTok{,localname))}

\CommentTok{\# standardisation {-}{-}{-}{-}}
\ControlFlowTok{if}\NormalTok{(}\SpecialCharTok{!}\FunctionTok{require}\NormalTok{(terra)) \{}\FunctionTok{install.packages}\NormalTok{(}\StringTok{"terra"}\NormalTok{); }\FunctionTok{require}\NormalTok{(terra)\}}
\ControlFlowTok{if}\NormalTok{(}\SpecialCharTok{!}\FunctionTok{require}\NormalTok{(tidyverse)) \{}\FunctionTok{install.packages}\NormalTok{(}\StringTok{"tidyverse"}\NormalTok{); }\FunctionTok{require}\NormalTok{(tidyverse)\}}

\NormalTok{nosaukums}\OtherTok{=}\StringTok{"HydroClim\_06{-}min\_cell.tif"}
\NormalTok{ielasisanas\_cels}\OtherTok{=}\FunctionTok{paste0}\NormalTok{(}\StringTok{"./RasterGrids\_100m/2024/RAW/"}\NormalTok{,nosaukums)}
\NormalTok{saglabasanas\_cels}\OtherTok{=}\FunctionTok{paste0}\NormalTok{(}\StringTok{"./RasterGrids\_100m/2024/Scaled/"}\NormalTok{,nosaukums)}
\NormalTok{slanis}\OtherTok{=}\FunctionTok{rast}\NormalTok{(ielasisanas\_cels)}
\NormalTok{videjais}\OtherTok{=}\FunctionTok{global}\NormalTok{(slanis,}\AttributeTok{fun=}\StringTok{"mean"}\NormalTok{,}\AttributeTok{na.rm=}\ConstantTok{TRUE}\NormalTok{)}
\NormalTok{centrets}\OtherTok{=}\NormalTok{slanis}\SpecialCharTok{{-}}\NormalTok{videjais[,}\DecValTok{1}\NormalTok{]}
\NormalTok{standartnovirze}\OtherTok{=}\NormalTok{terra}\SpecialCharTok{::}\FunctionTok{global}\NormalTok{(centrets,}\AttributeTok{fun=}\StringTok{"rms"}\NormalTok{,}\AttributeTok{na.rm=}\ConstantTok{TRUE}\NormalTok{)}
\NormalTok{merogots}\OtherTok{=}\NormalTok{centrets}\SpecialCharTok{/}\NormalTok{standartnovirze[,}\DecValTok{1}\NormalTok{]}
\FunctionTok{writeRaster}\NormalTok{(merogots,}
      \AttributeTok{filename=}\NormalTok{saglabasanas\_cels,}
      \AttributeTok{overwrite=}\ConstantTok{TRUE}\NormalTok{)}
\end{Highlighting}
\end{Shaded}

\section{HydroClim\_07-max\_cell}\label{ch06.076}

\textbf{filename:} \texttt{HydroClim\_07-max\_cell.tif}

\textbf{layername:} \texttt{egv\_076}

\textbf{English name:} Maximum per subcatchment upstream annual range of air
temperature (°C) (HydroClim) within the analysis cell (1 ha)

\textbf{Latvian name:} Sateces apakšbaseina maksimālā augšteces gada gaisa
temperatūru amplitūda (°C) (HydroClim) analīzes šūnā (1 ha)

\textbf{Procedure:} Information from the \hyperref[Ch04.12]{HydroClim
data} - including both basin and raster layers - is used. First, basin CRS is transformed to EPSG:3059. Then,
zonal statistics (per basin) using a layer specific summary function (max) are
calculated (\texttt{exactextractr::exact\_extract()}), and the the results are rasterised with the workflow
\texttt{egvtools::polygon2input()}. Once rasterised to input data, EGV is created using the workflow
\texttt{egvtools::input2egv()}. To prevent from gaps at the edges, inverse distance
weighted (power = 2) gap filling is implemented. To save disk space,
the intermediate input layer is unlinked. Finally, the layer is standardised by
subtracting the arithmetic mean and dividing by the root mean squared error.

\begin{Shaded}
\begin{Highlighting}[]
\CommentTok{\# libs {-}{-}{-}{-}}
\ControlFlowTok{if}\NormalTok{(}\SpecialCharTok{!}\FunctionTok{require}\NormalTok{(egvtools)) \{remotes}\SpecialCharTok{::}\FunctionTok{install\_github}\NormalTok{(}\StringTok{"aavotins/egvtools"}\NormalTok{); }\FunctionTok{require}\NormalTok{(egvtools)\}}
\ControlFlowTok{if}\NormalTok{(}\SpecialCharTok{!}\FunctionTok{require}\NormalTok{(terra)) \{}\FunctionTok{install.packages}\NormalTok{(}\StringTok{"terra"}\NormalTok{); }\FunctionTok{require}\NormalTok{(terra)\}}
\ControlFlowTok{if}\NormalTok{(}\SpecialCharTok{!}\FunctionTok{require}\NormalTok{(tidyverse)) \{}\FunctionTok{install.packages}\NormalTok{(}\StringTok{"tidyverse"}\NormalTok{); }\FunctionTok{require}\NormalTok{(tidyverse)\}}
\ControlFlowTok{if}\NormalTok{(}\SpecialCharTok{!}\FunctionTok{require}\NormalTok{(sf)) \{}\FunctionTok{install.packages}\NormalTok{(}\StringTok{"sf"}\NormalTok{); }\FunctionTok{require}\NormalTok{(sf)\}}
\ControlFlowTok{if}\NormalTok{(}\SpecialCharTok{!}\FunctionTok{require}\NormalTok{(sfarrow)) \{}\FunctionTok{install.packages}\NormalTok{(}\StringTok{"sfarrow"}\NormalTok{); }\FunctionTok{require}\NormalTok{(sfarrow)\}}
\ControlFlowTok{if}\NormalTok{(}\SpecialCharTok{!}\FunctionTok{require}\NormalTok{(exactextractr)) \{}\FunctionTok{install.packages}\NormalTok{(}\StringTok{"exactextractr"}\NormalTok{); }\FunctionTok{require}\NormalTok{(exactextractr)\}}

\CommentTok{\# basins {-}{-}{-}{-}}
\NormalTok{level12}\OtherTok{=}\FunctionTok{st\_read}\NormalTok{(}\StringTok{"./Geodata/2024/HydroClim/hybas\_lake\_eu\_lev01{-}12\_v1c/hybas\_lake\_eu\_lev12\_v1c.shp"}\NormalTok{)}
\NormalTok{grid\_1km}\OtherTok{=}\NormalTok{sfarrow}\SpecialCharTok{::}\FunctionTok{st\_read\_parquet}\NormalTok{(}\StringTok{"./Templates/TemplateGrids/tikls1km\_sauzeme.parquet"}\NormalTok{)}
\NormalTok{grid\_1km}\OtherTok{=}\FunctionTok{st\_transform}\NormalTok{(grid\_1km,}\AttributeTok{crs=}\DecValTok{3059}\NormalTok{)}
\NormalTok{level12}\OtherTok{=}\FunctionTok{st\_transform}\NormalTok{(level12,}\AttributeTok{crs=}\DecValTok{3059}\NormalTok{)}
\NormalTok{level12}\OtherTok{=}\NormalTok{level12[grid\_1km,,]}

\NormalTok{level12}\OtherTok{=}\FunctionTok{st\_make\_valid}\NormalTok{(level12)}

\CommentTok{\# job {-}{-}{-}{-}}

\NormalTok{localname}\OtherTok{=}\StringTok{"HydroClim\_07{-}max\_cell.tif"}
\NormalTok{layername}\OtherTok{=}\StringTok{"egv\_076"}
\NormalTok{summary\_function}\OtherTok{=}\StringTok{"max"}
 
\NormalTok{slanis}\OtherTok{=}\FunctionTok{rast}\NormalTok{(}\FunctionTok{paste0}\NormalTok{(}\StringTok{"./Geodata/2024/HydroClim/"}\NormalTok{,localname))}
\NormalTok{level12}\SpecialCharTok{$}\NormalTok{Hydro\_values}\OtherTok{=}\FunctionTok{exact\_extract}\NormalTok{(slanis,level12,}\AttributeTok{fun=}\NormalTok{summary\_function)}
 
\FunctionTok{polygon2input}\NormalTok{(}\AttributeTok{vector\_data =}\NormalTok{ level12,}
       \AttributeTok{template\_path =} \StringTok{"./Templates/TemplateRasters/LV10m\_10km.tif"}\NormalTok{,}
       \AttributeTok{out\_path =} \StringTok{"./RasterGrids\_10m/2024/"}\NormalTok{,}
       \AttributeTok{file\_name =}\NormalTok{ localname,}
       \AttributeTok{value\_field =} \StringTok{"Hydro\_values"}\NormalTok{,}
       \AttributeTok{fun=}\StringTok{"first"}\NormalTok{,}
       \AttributeTok{value\_type =} \StringTok{"continuous"}\NormalTok{,}
       \AttributeTok{prepare=}\ConstantTok{FALSE}\NormalTok{,}
       \AttributeTok{project\_mode =} \StringTok{"auto"}\NormalTok{,}
       \AttributeTok{check\_na =} \ConstantTok{FALSE}\NormalTok{,}
       \AttributeTok{plot\_result=}\ConstantTok{FALSE}\NormalTok{,}
       \AttributeTok{plot\_gaps =} \ConstantTok{FALSE}\NormalTok{,}
       \AttributeTok{overwrite=}\ConstantTok{TRUE}\NormalTok{)}
 
\NormalTok{egvrez}\OtherTok{=}\FunctionTok{input2egv}\NormalTok{(}\AttributeTok{input=}\FunctionTok{paste0}\NormalTok{(}\StringTok{"./RasterGrids\_10m/2024/"}\NormalTok{,localname),}
         \AttributeTok{egv\_template=} \StringTok{"./Templates/TemplateRasters/LV100m\_10km.tif"}\NormalTok{,}
         \AttributeTok{summary\_function =} \StringTok{"average"}\NormalTok{,}
         \AttributeTok{missing\_job =} \StringTok{"FillOutput"}\NormalTok{,}
         \AttributeTok{input\_template =} \StringTok{"./Templates/TemplateRasters/LV10m\_10km.tif"}\NormalTok{,}
         \AttributeTok{outlocation =} \StringTok{"./RasterGrids\_100m/2024/RAW/"}\NormalTok{,}
         \AttributeTok{outfilename =}\NormalTok{ localname,}
         \AttributeTok{layername =}\NormalTok{ layername,}
         \AttributeTok{idw\_weight =} \DecValTok{2}\NormalTok{,}
         \AttributeTok{plot\_gaps =} \ConstantTok{FALSE}\NormalTok{,}\AttributeTok{plot\_final =} \ConstantTok{FALSE}\NormalTok{)}
\NormalTok{egvrez}
 
\FunctionTok{unlink}\NormalTok{(}\FunctionTok{paste0}\NormalTok{(}\StringTok{"./RasterGrids\_10m/2024/"}\NormalTok{,localname))}

\CommentTok{\# standardisation {-}{-}{-}{-}}
\ControlFlowTok{if}\NormalTok{(}\SpecialCharTok{!}\FunctionTok{require}\NormalTok{(terra)) \{}\FunctionTok{install.packages}\NormalTok{(}\StringTok{"terra"}\NormalTok{); }\FunctionTok{require}\NormalTok{(terra)\}}
\ControlFlowTok{if}\NormalTok{(}\SpecialCharTok{!}\FunctionTok{require}\NormalTok{(tidyverse)) \{}\FunctionTok{install.packages}\NormalTok{(}\StringTok{"tidyverse"}\NormalTok{); }\FunctionTok{require}\NormalTok{(tidyverse)\}}

\NormalTok{nosaukums}\OtherTok{=}\StringTok{"HydroClim\_07{-}max\_cell.tif"}
\NormalTok{ielasisanas\_cels}\OtherTok{=}\FunctionTok{paste0}\NormalTok{(}\StringTok{"./RasterGrids\_100m/2024/RAW/"}\NormalTok{,nosaukums)}
\NormalTok{saglabasanas\_cels}\OtherTok{=}\FunctionTok{paste0}\NormalTok{(}\StringTok{"./RasterGrids\_100m/2024/Scaled/"}\NormalTok{,nosaukums)}
\NormalTok{slanis}\OtherTok{=}\FunctionTok{rast}\NormalTok{(ielasisanas\_cels)}
\NormalTok{videjais}\OtherTok{=}\FunctionTok{global}\NormalTok{(slanis,}\AttributeTok{fun=}\StringTok{"mean"}\NormalTok{,}\AttributeTok{na.rm=}\ConstantTok{TRUE}\NormalTok{)}
\NormalTok{centrets}\OtherTok{=}\NormalTok{slanis}\SpecialCharTok{{-}}\NormalTok{videjais[,}\DecValTok{1}\NormalTok{]}
\NormalTok{standartnovirze}\OtherTok{=}\NormalTok{terra}\SpecialCharTok{::}\FunctionTok{global}\NormalTok{(centrets,}\AttributeTok{fun=}\StringTok{"rms"}\NormalTok{,}\AttributeTok{na.rm=}\ConstantTok{TRUE}\NormalTok{)}
\NormalTok{merogots}\OtherTok{=}\NormalTok{centrets}\SpecialCharTok{/}\NormalTok{standartnovirze[,}\DecValTok{1}\NormalTok{]}
\FunctionTok{writeRaster}\NormalTok{(merogots,}
      \AttributeTok{filename=}\NormalTok{saglabasanas\_cels,}
      \AttributeTok{overwrite=}\ConstantTok{TRUE}\NormalTok{)}
\end{Highlighting}
\end{Shaded}

\section{HydroClim\_08-max\_cell}\label{ch06.077}

\textbf{filename:} \texttt{HydroClim\_08-max\_cell.tif}

\textbf{layername:} \texttt{egv\_077}

\textbf{English name:} Maximum per subcatchment upstream mean daily mean air
temperatures (°C) of the wettest quarter (HydroClim) within the analysis cell (1
ha)

\textbf{Latvian name:} Sateces apakšbaseina maksimālā augšteces dienas vidējā gaisa
temperatūra mitrākajā ceturksnī (°C) (HydroClim) analīzes šūnā (1 ha)

\textbf{Procedure:} Information from the \hyperref[Ch04.12]{HydroClim
data} - including both basin and raster layers - is used. First, basin CRS is transformed to EPSG:3059. Then,
zonal statistics (per basin) using a layer specific summary function (max) are
calculated (\texttt{exactextractr::exact\_extract()}), and the the results are rasterised with the workflow
\texttt{egvtools::polygon2input()}. Once rasterised to input data, EGV is created using the workflow
\texttt{egvtools::input2egv()}. To prevent from gaps at the edges, inverse distance
weighted (power = 2) gap filling is implemented. To save disk space,
the intermediate input layer is unlinked. Finally, the layer is standardised by
subtracting the arithmetic mean and dividing by the root mean squared error.

\begin{Shaded}
\begin{Highlighting}[]
\CommentTok{\# libs {-}{-}{-}{-}}
\ControlFlowTok{if}\NormalTok{(}\SpecialCharTok{!}\FunctionTok{require}\NormalTok{(egvtools)) \{remotes}\SpecialCharTok{::}\FunctionTok{install\_github}\NormalTok{(}\StringTok{"aavotins/egvtools"}\NormalTok{); }\FunctionTok{require}\NormalTok{(egvtools)\}}
\ControlFlowTok{if}\NormalTok{(}\SpecialCharTok{!}\FunctionTok{require}\NormalTok{(terra)) \{}\FunctionTok{install.packages}\NormalTok{(}\StringTok{"terra"}\NormalTok{); }\FunctionTok{require}\NormalTok{(terra)\}}
\ControlFlowTok{if}\NormalTok{(}\SpecialCharTok{!}\FunctionTok{require}\NormalTok{(tidyverse)) \{}\FunctionTok{install.packages}\NormalTok{(}\StringTok{"tidyverse"}\NormalTok{); }\FunctionTok{require}\NormalTok{(tidyverse)\}}
\ControlFlowTok{if}\NormalTok{(}\SpecialCharTok{!}\FunctionTok{require}\NormalTok{(sf)) \{}\FunctionTok{install.packages}\NormalTok{(}\StringTok{"sf"}\NormalTok{); }\FunctionTok{require}\NormalTok{(sf)\}}
\ControlFlowTok{if}\NormalTok{(}\SpecialCharTok{!}\FunctionTok{require}\NormalTok{(sfarrow)) \{}\FunctionTok{install.packages}\NormalTok{(}\StringTok{"sfarrow"}\NormalTok{); }\FunctionTok{require}\NormalTok{(sfarrow)\}}
\ControlFlowTok{if}\NormalTok{(}\SpecialCharTok{!}\FunctionTok{require}\NormalTok{(exactextractr)) \{}\FunctionTok{install.packages}\NormalTok{(}\StringTok{"exactextractr"}\NormalTok{); }\FunctionTok{require}\NormalTok{(exactextractr)\}}

\CommentTok{\# basins {-}{-}{-}{-}}
\NormalTok{level12}\OtherTok{=}\FunctionTok{st\_read}\NormalTok{(}\StringTok{"./Geodata/2024/HydroClim/hybas\_lake\_eu\_lev01{-}12\_v1c/hybas\_lake\_eu\_lev12\_v1c.shp"}\NormalTok{)}
\NormalTok{grid\_1km}\OtherTok{=}\NormalTok{sfarrow}\SpecialCharTok{::}\FunctionTok{st\_read\_parquet}\NormalTok{(}\StringTok{"./Templates/TemplateGrids/tikls1km\_sauzeme.parquet"}\NormalTok{)}
\NormalTok{grid\_1km}\OtherTok{=}\FunctionTok{st\_transform}\NormalTok{(grid\_1km,}\AttributeTok{crs=}\DecValTok{3059}\NormalTok{)}
\NormalTok{level12}\OtherTok{=}\FunctionTok{st\_transform}\NormalTok{(level12,}\AttributeTok{crs=}\DecValTok{3059}\NormalTok{)}
\NormalTok{level12}\OtherTok{=}\NormalTok{level12[grid\_1km,,]}

\NormalTok{level12}\OtherTok{=}\FunctionTok{st\_make\_valid}\NormalTok{(level12)}

\CommentTok{\# job {-}{-}{-}{-}}

\NormalTok{localname}\OtherTok{=}\StringTok{"HydroClim\_08{-}max\_cell.tif"}
\NormalTok{layername}\OtherTok{=}\StringTok{"egv\_077"}
\NormalTok{summary\_function}\OtherTok{=}\StringTok{"max"}
 
\NormalTok{slanis}\OtherTok{=}\FunctionTok{rast}\NormalTok{(}\FunctionTok{paste0}\NormalTok{(}\StringTok{"./Geodata/2024/HydroClim/"}\NormalTok{,localname))}
\NormalTok{level12}\SpecialCharTok{$}\NormalTok{Hydro\_values}\OtherTok{=}\FunctionTok{exact\_extract}\NormalTok{(slanis,level12,}\AttributeTok{fun=}\NormalTok{summary\_function)}
 
\FunctionTok{polygon2input}\NormalTok{(}\AttributeTok{vector\_data =}\NormalTok{ level12,}
       \AttributeTok{template\_path =} \StringTok{"./Templates/TemplateRasters/LV10m\_10km.tif"}\NormalTok{,}
       \AttributeTok{out\_path =} \StringTok{"./RasterGrids\_10m/2024/"}\NormalTok{,}
       \AttributeTok{file\_name =}\NormalTok{ localname,}
       \AttributeTok{value\_field =} \StringTok{"Hydro\_values"}\NormalTok{,}
       \AttributeTok{fun=}\StringTok{"first"}\NormalTok{,}
       \AttributeTok{value\_type =} \StringTok{"continuous"}\NormalTok{,}
       \AttributeTok{prepare=}\ConstantTok{FALSE}\NormalTok{,}
       \AttributeTok{project\_mode =} \StringTok{"auto"}\NormalTok{,}
       \AttributeTok{check\_na =} \ConstantTok{FALSE}\NormalTok{,}
       \AttributeTok{plot\_result=}\ConstantTok{FALSE}\NormalTok{,}
       \AttributeTok{plot\_gaps =} \ConstantTok{FALSE}\NormalTok{,}
       \AttributeTok{overwrite=}\ConstantTok{TRUE}\NormalTok{)}
 
\NormalTok{egvrez}\OtherTok{=}\FunctionTok{input2egv}\NormalTok{(}\AttributeTok{input=}\FunctionTok{paste0}\NormalTok{(}\StringTok{"./RasterGrids\_10m/2024/"}\NormalTok{,localname),}
         \AttributeTok{egv\_template=} \StringTok{"./Templates/TemplateRasters/LV100m\_10km.tif"}\NormalTok{,}
         \AttributeTok{summary\_function =} \StringTok{"average"}\NormalTok{,}
         \AttributeTok{missing\_job =} \StringTok{"FillOutput"}\NormalTok{,}
         \AttributeTok{input\_template =} \StringTok{"./Templates/TemplateRasters/LV10m\_10km.tif"}\NormalTok{,}
         \AttributeTok{outlocation =} \StringTok{"./RasterGrids\_100m/2024/RAW/"}\NormalTok{,}
         \AttributeTok{outfilename =}\NormalTok{ localname,}
         \AttributeTok{layername =}\NormalTok{ layername,}
         \AttributeTok{idw\_weight =} \DecValTok{2}\NormalTok{,}
         \AttributeTok{plot\_gaps =} \ConstantTok{FALSE}\NormalTok{,}\AttributeTok{plot\_final =} \ConstantTok{FALSE}\NormalTok{)}
\NormalTok{egvrez}
 
\FunctionTok{unlink}\NormalTok{(}\FunctionTok{paste0}\NormalTok{(}\StringTok{"./RasterGrids\_10m/2024/"}\NormalTok{,localname))}

\CommentTok{\# standardisation {-}{-}{-}{-}}
\ControlFlowTok{if}\NormalTok{(}\SpecialCharTok{!}\FunctionTok{require}\NormalTok{(terra)) \{}\FunctionTok{install.packages}\NormalTok{(}\StringTok{"terra"}\NormalTok{); }\FunctionTok{require}\NormalTok{(terra)\}}
\ControlFlowTok{if}\NormalTok{(}\SpecialCharTok{!}\FunctionTok{require}\NormalTok{(tidyverse)) \{}\FunctionTok{install.packages}\NormalTok{(}\StringTok{"tidyverse"}\NormalTok{); }\FunctionTok{require}\NormalTok{(tidyverse)\}}

\NormalTok{nosaukums}\OtherTok{=}\StringTok{"HydroClim\_08{-}max\_cell.tif"}
\NormalTok{ielasisanas\_cels}\OtherTok{=}\FunctionTok{paste0}\NormalTok{(}\StringTok{"./RasterGrids\_100m/2024/RAW/"}\NormalTok{,nosaukums)}
\NormalTok{saglabasanas\_cels}\OtherTok{=}\FunctionTok{paste0}\NormalTok{(}\StringTok{"./RasterGrids\_100m/2024/Scaled/"}\NormalTok{,nosaukums)}
\NormalTok{slanis}\OtherTok{=}\FunctionTok{rast}\NormalTok{(ielasisanas\_cels)}
\NormalTok{videjais}\OtherTok{=}\FunctionTok{global}\NormalTok{(slanis,}\AttributeTok{fun=}\StringTok{"mean"}\NormalTok{,}\AttributeTok{na.rm=}\ConstantTok{TRUE}\NormalTok{)}
\NormalTok{centrets}\OtherTok{=}\NormalTok{slanis}\SpecialCharTok{{-}}\NormalTok{videjais[,}\DecValTok{1}\NormalTok{]}
\NormalTok{standartnovirze}\OtherTok{=}\NormalTok{terra}\SpecialCharTok{::}\FunctionTok{global}\NormalTok{(centrets,}\AttributeTok{fun=}\StringTok{"rms"}\NormalTok{,}\AttributeTok{na.rm=}\ConstantTok{TRUE}\NormalTok{)}
\NormalTok{merogots}\OtherTok{=}\NormalTok{centrets}\SpecialCharTok{/}\NormalTok{standartnovirze[,}\DecValTok{1}\NormalTok{]}
\FunctionTok{writeRaster}\NormalTok{(merogots,}
      \AttributeTok{filename=}\NormalTok{saglabasanas\_cels,}
      \AttributeTok{overwrite=}\ConstantTok{TRUE}\NormalTok{)}
\end{Highlighting}
\end{Shaded}

\section{HydroClim\_09-min\_cell}\label{ch06.078}

\textbf{filename:} \texttt{HydroClim\_09-min\_cell.tif}

\textbf{layername:} \texttt{egv\_078}

\textbf{English name:} Minimum per subcatchment upstream mean daily mean air
temperatures (°C) of the driest quarter (HydroClim) within the analysis cell (1
ha)

\textbf{Latvian name:} Sateces apakšbaseina minimālā augšteces dienas vidējā gaisa
temperatūra sausākajā ceturksnī (°C) (HydroClim) analīzes šūnā (1 ha)

\textbf{Procedure:} Information from the \hyperref[Ch04.12]{HydroClim
data} - including both basin and raster layers - is used. First, basin CRS is transformed to EPSG:3059. Then,
zonal statistics (per basin) using a layer specific summary function (min) are
calculated (\texttt{exactextractr::exact\_extract()}), and the the results are rasterised with the workflow
\texttt{egvtools::polygon2input()}. Once rasterised to input data, EGV is created using the workflow
\texttt{egvtools::input2egv()}. To prevent from gaps at the edges, inverse distance
weighted (power = 2) gap filling is implemented. To save disk space,
the intermediate input layer is unlinked. Finally, the layer is standardised by
subtracting the arithmetic mean and dividing by the root mean squared error.

\begin{Shaded}
\begin{Highlighting}[]
\CommentTok{\# libs {-}{-}{-}{-}}
\ControlFlowTok{if}\NormalTok{(}\SpecialCharTok{!}\FunctionTok{require}\NormalTok{(egvtools)) \{remotes}\SpecialCharTok{::}\FunctionTok{install\_github}\NormalTok{(}\StringTok{"aavotins/egvtools"}\NormalTok{); }\FunctionTok{require}\NormalTok{(egvtools)\}}
\ControlFlowTok{if}\NormalTok{(}\SpecialCharTok{!}\FunctionTok{require}\NormalTok{(terra)) \{}\FunctionTok{install.packages}\NormalTok{(}\StringTok{"terra"}\NormalTok{); }\FunctionTok{require}\NormalTok{(terra)\}}
\ControlFlowTok{if}\NormalTok{(}\SpecialCharTok{!}\FunctionTok{require}\NormalTok{(tidyverse)) \{}\FunctionTok{install.packages}\NormalTok{(}\StringTok{"tidyverse"}\NormalTok{); }\FunctionTok{require}\NormalTok{(tidyverse)\}}
\ControlFlowTok{if}\NormalTok{(}\SpecialCharTok{!}\FunctionTok{require}\NormalTok{(sf)) \{}\FunctionTok{install.packages}\NormalTok{(}\StringTok{"sf"}\NormalTok{); }\FunctionTok{require}\NormalTok{(sf)\}}
\ControlFlowTok{if}\NormalTok{(}\SpecialCharTok{!}\FunctionTok{require}\NormalTok{(sfarrow)) \{}\FunctionTok{install.packages}\NormalTok{(}\StringTok{"sfarrow"}\NormalTok{); }\FunctionTok{require}\NormalTok{(sfarrow)\}}
\ControlFlowTok{if}\NormalTok{(}\SpecialCharTok{!}\FunctionTok{require}\NormalTok{(exactextractr)) \{}\FunctionTok{install.packages}\NormalTok{(}\StringTok{"exactextractr"}\NormalTok{); }\FunctionTok{require}\NormalTok{(exactextractr)\}}

\CommentTok{\# basins {-}{-}{-}{-}}
\NormalTok{level12}\OtherTok{=}\FunctionTok{st\_read}\NormalTok{(}\StringTok{"./Geodata/2024/HydroClim/hybas\_lake\_eu\_lev01{-}12\_v1c/hybas\_lake\_eu\_lev12\_v1c.shp"}\NormalTok{)}
\NormalTok{grid\_1km}\OtherTok{=}\NormalTok{sfarrow}\SpecialCharTok{::}\FunctionTok{st\_read\_parquet}\NormalTok{(}\StringTok{"./Templates/TemplateGrids/tikls1km\_sauzeme.parquet"}\NormalTok{)}
\NormalTok{grid\_1km}\OtherTok{=}\FunctionTok{st\_transform}\NormalTok{(grid\_1km,}\AttributeTok{crs=}\DecValTok{3059}\NormalTok{)}
\NormalTok{level12}\OtherTok{=}\FunctionTok{st\_transform}\NormalTok{(level12,}\AttributeTok{crs=}\DecValTok{3059}\NormalTok{)}
\NormalTok{level12}\OtherTok{=}\NormalTok{level12[grid\_1km,,]}

\NormalTok{level12}\OtherTok{=}\FunctionTok{st\_make\_valid}\NormalTok{(level12)}

\CommentTok{\# job {-}{-}{-}{-}}

\NormalTok{localname}\OtherTok{=}\StringTok{"HydroClim\_09{-}min\_cell.tif"}
\NormalTok{layername}\OtherTok{=}\StringTok{"egv\_078"}
\NormalTok{summary\_function}\OtherTok{=}\StringTok{"min"}
 
\NormalTok{slanis}\OtherTok{=}\FunctionTok{rast}\NormalTok{(}\FunctionTok{paste0}\NormalTok{(}\StringTok{"./Geodata/2024/HydroClim/"}\NormalTok{,localname))}
\NormalTok{level12}\SpecialCharTok{$}\NormalTok{Hydro\_values}\OtherTok{=}\FunctionTok{exact\_extract}\NormalTok{(slanis,level12,}\AttributeTok{fun=}\NormalTok{summary\_function)}
 
\FunctionTok{polygon2input}\NormalTok{(}\AttributeTok{vector\_data =}\NormalTok{ level12,}
       \AttributeTok{template\_path =} \StringTok{"./Templates/TemplateRasters/LV10m\_10km.tif"}\NormalTok{,}
       \AttributeTok{out\_path =} \StringTok{"./RasterGrids\_10m/2024/"}\NormalTok{,}
       \AttributeTok{file\_name =}\NormalTok{ localname,}
       \AttributeTok{value\_field =} \StringTok{"Hydro\_values"}\NormalTok{,}
       \AttributeTok{fun=}\StringTok{"first"}\NormalTok{,}
       \AttributeTok{value\_type =} \StringTok{"continuous"}\NormalTok{,}
       \AttributeTok{prepare=}\ConstantTok{FALSE}\NormalTok{,}
       \AttributeTok{project\_mode =} \StringTok{"auto"}\NormalTok{,}
       \AttributeTok{check\_na =} \ConstantTok{FALSE}\NormalTok{,}
       \AttributeTok{plot\_result=}\ConstantTok{FALSE}\NormalTok{,}
       \AttributeTok{plot\_gaps =} \ConstantTok{FALSE}\NormalTok{,}
       \AttributeTok{overwrite=}\ConstantTok{TRUE}\NormalTok{)}
 
\NormalTok{egvrez}\OtherTok{=}\FunctionTok{input2egv}\NormalTok{(}\AttributeTok{input=}\FunctionTok{paste0}\NormalTok{(}\StringTok{"./RasterGrids\_10m/2024/"}\NormalTok{,localname),}
         \AttributeTok{egv\_template=} \StringTok{"./Templates/TemplateRasters/LV100m\_10km.tif"}\NormalTok{,}
         \AttributeTok{summary\_function =} \StringTok{"average"}\NormalTok{,}
         \AttributeTok{missing\_job =} \StringTok{"FillOutput"}\NormalTok{,}
         \AttributeTok{input\_template =} \StringTok{"./Templates/TemplateRasters/LV10m\_10km.tif"}\NormalTok{,}
         \AttributeTok{outlocation =} \StringTok{"./RasterGrids\_100m/2024/RAW/"}\NormalTok{,}
         \AttributeTok{outfilename =}\NormalTok{ localname,}
         \AttributeTok{layername =}\NormalTok{ layername,}
         \AttributeTok{idw\_weight =} \DecValTok{2}\NormalTok{,}
         \AttributeTok{plot\_gaps =} \ConstantTok{FALSE}\NormalTok{,}\AttributeTok{plot\_final =} \ConstantTok{FALSE}\NormalTok{)}
\NormalTok{egvrez}
 
\FunctionTok{unlink}\NormalTok{(}\FunctionTok{paste0}\NormalTok{(}\StringTok{"./RasterGrids\_10m/2024/"}\NormalTok{,localname))}

\CommentTok{\# standardisation {-}{-}{-}{-}}
\ControlFlowTok{if}\NormalTok{(}\SpecialCharTok{!}\FunctionTok{require}\NormalTok{(terra)) \{}\FunctionTok{install.packages}\NormalTok{(}\StringTok{"terra"}\NormalTok{); }\FunctionTok{require}\NormalTok{(terra)\}}
\ControlFlowTok{if}\NormalTok{(}\SpecialCharTok{!}\FunctionTok{require}\NormalTok{(tidyverse)) \{}\FunctionTok{install.packages}\NormalTok{(}\StringTok{"tidyverse"}\NormalTok{); }\FunctionTok{require}\NormalTok{(tidyverse)\}}

\NormalTok{nosaukums}\OtherTok{=}\StringTok{"HydroClim\_09{-}min\_cell.tif"}
\NormalTok{ielasisanas\_cels}\OtherTok{=}\FunctionTok{paste0}\NormalTok{(}\StringTok{"./RasterGrids\_100m/2024/RAW/"}\NormalTok{,nosaukums)}
\NormalTok{saglabasanas\_cels}\OtherTok{=}\FunctionTok{paste0}\NormalTok{(}\StringTok{"./RasterGrids\_100m/2024/Scaled/"}\NormalTok{,nosaukums)}
\NormalTok{slanis}\OtherTok{=}\FunctionTok{rast}\NormalTok{(ielasisanas\_cels)}
\NormalTok{videjais}\OtherTok{=}\FunctionTok{global}\NormalTok{(slanis,}\AttributeTok{fun=}\StringTok{"mean"}\NormalTok{,}\AttributeTok{na.rm=}\ConstantTok{TRUE}\NormalTok{)}
\NormalTok{centrets}\OtherTok{=}\NormalTok{slanis}\SpecialCharTok{{-}}\NormalTok{videjais[,}\DecValTok{1}\NormalTok{]}
\NormalTok{standartnovirze}\OtherTok{=}\NormalTok{terra}\SpecialCharTok{::}\FunctionTok{global}\NormalTok{(centrets,}\AttributeTok{fun=}\StringTok{"rms"}\NormalTok{,}\AttributeTok{na.rm=}\ConstantTok{TRUE}\NormalTok{)}
\NormalTok{merogots}\OtherTok{=}\NormalTok{centrets}\SpecialCharTok{/}\NormalTok{standartnovirze[,}\DecValTok{1}\NormalTok{]}
\FunctionTok{writeRaster}\NormalTok{(merogots,}
      \AttributeTok{filename=}\NormalTok{saglabasanas\_cels,}
      \AttributeTok{overwrite=}\ConstantTok{TRUE}\NormalTok{)}
\end{Highlighting}
\end{Shaded}

\section{HydroClim\_10-max\_cell}\label{ch06.079}

\textbf{filename:} \texttt{HydroClim\_10-max\_cell.tif}

\textbf{layername:} \texttt{egv\_079}

\textbf{English name:} Maximum per subcatchment upstream mean daily mean air
temperatures (°C) of the warmest quarter (HydroClim) within the analysis cell (1
ha)

\textbf{Latvian name:} Sateces apakšbaseina maksimālā augšteces dienas vidējā gaisa
temperatūra siltākajā ceturksnī (°C) (HydroClim) analīzes šūnā (1 ha)

\textbf{Procedure:} Information from the \hyperref[Ch04.12]{HydroClim
data} - including both basin and raster layers - is used. First, basin CRS is transformed to EPSG:3059. Then,
zonal statistics (per basin) using a layer specific summary function (max) are
calculated (\texttt{exactextractr::exact\_extract()}), and the the results are rasterised with the workflow
\texttt{egvtools::polygon2input()}. Once rasterised to input data, EGV is created using the workflow
\texttt{egvtools::input2egv()}. To prevent from gaps at the edges, inverse distance
weighted (power = 2) gap filling is implemented. To save disk space,
the intermediate input layer is unlinked. Finally, the layer is standardised by
subtracting the arithmetic mean and dividing by the root mean squared error.

\begin{Shaded}
\begin{Highlighting}[]
\CommentTok{\# libs {-}{-}{-}{-}}
\ControlFlowTok{if}\NormalTok{(}\SpecialCharTok{!}\FunctionTok{require}\NormalTok{(egvtools)) \{remotes}\SpecialCharTok{::}\FunctionTok{install\_github}\NormalTok{(}\StringTok{"aavotins/egvtools"}\NormalTok{); }\FunctionTok{require}\NormalTok{(egvtools)\}}
\ControlFlowTok{if}\NormalTok{(}\SpecialCharTok{!}\FunctionTok{require}\NormalTok{(terra)) \{}\FunctionTok{install.packages}\NormalTok{(}\StringTok{"terra"}\NormalTok{); }\FunctionTok{require}\NormalTok{(terra)\}}
\ControlFlowTok{if}\NormalTok{(}\SpecialCharTok{!}\FunctionTok{require}\NormalTok{(tidyverse)) \{}\FunctionTok{install.packages}\NormalTok{(}\StringTok{"tidyverse"}\NormalTok{); }\FunctionTok{require}\NormalTok{(tidyverse)\}}
\ControlFlowTok{if}\NormalTok{(}\SpecialCharTok{!}\FunctionTok{require}\NormalTok{(sf)) \{}\FunctionTok{install.packages}\NormalTok{(}\StringTok{"sf"}\NormalTok{); }\FunctionTok{require}\NormalTok{(sf)\}}
\ControlFlowTok{if}\NormalTok{(}\SpecialCharTok{!}\FunctionTok{require}\NormalTok{(sfarrow)) \{}\FunctionTok{install.packages}\NormalTok{(}\StringTok{"sfarrow"}\NormalTok{); }\FunctionTok{require}\NormalTok{(sfarrow)\}}
\ControlFlowTok{if}\NormalTok{(}\SpecialCharTok{!}\FunctionTok{require}\NormalTok{(exactextractr)) \{}\FunctionTok{install.packages}\NormalTok{(}\StringTok{"exactextractr"}\NormalTok{); }\FunctionTok{require}\NormalTok{(exactextractr)\}}

\CommentTok{\# basins {-}{-}{-}{-}}
\NormalTok{level12}\OtherTok{=}\FunctionTok{st\_read}\NormalTok{(}\StringTok{"./Geodata/2024/HydroClim/hybas\_lake\_eu\_lev01{-}12\_v1c/hybas\_lake\_eu\_lev12\_v1c.shp"}\NormalTok{)}
\NormalTok{grid\_1km}\OtherTok{=}\NormalTok{sfarrow}\SpecialCharTok{::}\FunctionTok{st\_read\_parquet}\NormalTok{(}\StringTok{"./Templates/TemplateGrids/tikls1km\_sauzeme.parquet"}\NormalTok{)}
\NormalTok{grid\_1km}\OtherTok{=}\FunctionTok{st\_transform}\NormalTok{(grid\_1km,}\AttributeTok{crs=}\DecValTok{3059}\NormalTok{)}
\NormalTok{level12}\OtherTok{=}\FunctionTok{st\_transform}\NormalTok{(level12,}\AttributeTok{crs=}\DecValTok{3059}\NormalTok{)}
\NormalTok{level12}\OtherTok{=}\NormalTok{level12[grid\_1km,,]}

\NormalTok{level12}\OtherTok{=}\FunctionTok{st\_make\_valid}\NormalTok{(level12)}

\CommentTok{\# job {-}{-}{-}{-}}

\NormalTok{localname}\OtherTok{=}\StringTok{"HydroClim\_10{-}max\_cell.tif"}
\NormalTok{layername}\OtherTok{=}\StringTok{"egv\_079"}
\NormalTok{summary\_function}\OtherTok{=}\StringTok{"max"}
 
\NormalTok{slanis}\OtherTok{=}\FunctionTok{rast}\NormalTok{(}\FunctionTok{paste0}\NormalTok{(}\StringTok{"./Geodata/2024/HydroClim/"}\NormalTok{,localname))}
\NormalTok{level12}\SpecialCharTok{$}\NormalTok{Hydro\_values}\OtherTok{=}\FunctionTok{exact\_extract}\NormalTok{(slanis,level12,}\AttributeTok{fun=}\NormalTok{summary\_function)}
 
\FunctionTok{polygon2input}\NormalTok{(}\AttributeTok{vector\_data =}\NormalTok{ level12,}
       \AttributeTok{template\_path =} \StringTok{"./Templates/TemplateRasters/LV10m\_10km.tif"}\NormalTok{,}
       \AttributeTok{out\_path =} \StringTok{"./RasterGrids\_10m/2024/"}\NormalTok{,}
       \AttributeTok{file\_name =}\NormalTok{ localname,}
       \AttributeTok{value\_field =} \StringTok{"Hydro\_values"}\NormalTok{,}
       \AttributeTok{fun=}\StringTok{"first"}\NormalTok{,}
       \AttributeTok{value\_type =} \StringTok{"continuous"}\NormalTok{,}
       \AttributeTok{prepare=}\ConstantTok{FALSE}\NormalTok{,}
       \AttributeTok{project\_mode =} \StringTok{"auto"}\NormalTok{,}
       \AttributeTok{check\_na =} \ConstantTok{FALSE}\NormalTok{,}
       \AttributeTok{plot\_result=}\ConstantTok{FALSE}\NormalTok{,}
       \AttributeTok{plot\_gaps =} \ConstantTok{FALSE}\NormalTok{,}
       \AttributeTok{overwrite=}\ConstantTok{TRUE}\NormalTok{)}
 
\NormalTok{egvrez}\OtherTok{=}\FunctionTok{input2egv}\NormalTok{(}\AttributeTok{input=}\FunctionTok{paste0}\NormalTok{(}\StringTok{"./RasterGrids\_10m/2024/"}\NormalTok{,localname),}
         \AttributeTok{egv\_template=} \StringTok{"./Templates/TemplateRasters/LV100m\_10km.tif"}\NormalTok{,}
         \AttributeTok{summary\_function =} \StringTok{"average"}\NormalTok{,}
         \AttributeTok{missing\_job =} \StringTok{"FillOutput"}\NormalTok{,}
         \AttributeTok{input\_template =} \StringTok{"./Templates/TemplateRasters/LV10m\_10km.tif"}\NormalTok{,}
         \AttributeTok{outlocation =} \StringTok{"./RasterGrids\_100m/2024/RAW/"}\NormalTok{,}
         \AttributeTok{outfilename =}\NormalTok{ localname,}
         \AttributeTok{layername =}\NormalTok{ layername,}
         \AttributeTok{idw\_weight =} \DecValTok{2}\NormalTok{,}
         \AttributeTok{plot\_gaps =} \ConstantTok{FALSE}\NormalTok{,}\AttributeTok{plot\_final =} \ConstantTok{FALSE}\NormalTok{)}
\NormalTok{egvrez}
 
\FunctionTok{unlink}\NormalTok{(}\FunctionTok{paste0}\NormalTok{(}\StringTok{"./RasterGrids\_10m/2024/"}\NormalTok{,localname))}

\CommentTok{\# standardisation {-}{-}{-}{-}}
\ControlFlowTok{if}\NormalTok{(}\SpecialCharTok{!}\FunctionTok{require}\NormalTok{(terra)) \{}\FunctionTok{install.packages}\NormalTok{(}\StringTok{"terra"}\NormalTok{); }\FunctionTok{require}\NormalTok{(terra)\}}
\ControlFlowTok{if}\NormalTok{(}\SpecialCharTok{!}\FunctionTok{require}\NormalTok{(tidyverse)) \{}\FunctionTok{install.packages}\NormalTok{(}\StringTok{"tidyverse"}\NormalTok{); }\FunctionTok{require}\NormalTok{(tidyverse)\}}

\NormalTok{nosaukums}\OtherTok{=}\StringTok{"HydroClim\_10{-}max\_cell.tif"}
\NormalTok{ielasisanas\_cels}\OtherTok{=}\FunctionTok{paste0}\NormalTok{(}\StringTok{"./RasterGrids\_100m/2024/RAW/"}\NormalTok{,nosaukums)}
\NormalTok{saglabasanas\_cels}\OtherTok{=}\FunctionTok{paste0}\NormalTok{(}\StringTok{"./RasterGrids\_100m/2024/Scaled/"}\NormalTok{,nosaukums)}
\NormalTok{slanis}\OtherTok{=}\FunctionTok{rast}\NormalTok{(ielasisanas\_cels)}
\NormalTok{videjais}\OtherTok{=}\FunctionTok{global}\NormalTok{(slanis,}\AttributeTok{fun=}\StringTok{"mean"}\NormalTok{,}\AttributeTok{na.rm=}\ConstantTok{TRUE}\NormalTok{)}
\NormalTok{centrets}\OtherTok{=}\NormalTok{slanis}\SpecialCharTok{{-}}\NormalTok{videjais[,}\DecValTok{1}\NormalTok{]}
\NormalTok{standartnovirze}\OtherTok{=}\NormalTok{terra}\SpecialCharTok{::}\FunctionTok{global}\NormalTok{(centrets,}\AttributeTok{fun=}\StringTok{"rms"}\NormalTok{,}\AttributeTok{na.rm=}\ConstantTok{TRUE}\NormalTok{)}
\NormalTok{merogots}\OtherTok{=}\NormalTok{centrets}\SpecialCharTok{/}\NormalTok{standartnovirze[,}\DecValTok{1}\NormalTok{]}
\FunctionTok{writeRaster}\NormalTok{(merogots,}
      \AttributeTok{filename=}\NormalTok{saglabasanas\_cels,}
      \AttributeTok{overwrite=}\ConstantTok{TRUE}\NormalTok{)}
\end{Highlighting}
\end{Shaded}

\section{HydroClim\_11-min\_cell}\label{ch06.080}

\textbf{filename:} \texttt{HydroClim\_11-min\_cell.tif}

\textbf{layername:} \texttt{egv\_080}

\textbf{English name:} Minimum per subcatchment upstream mean daily mean air
temperatures (°C) of the coldest quarter (HydroClim) within the analysis cell (1
ha)

\textbf{Latvian name:} Sateces apakšbaseina minimālā augšteces dienas vidējā gaisa
temperatūra vēsākajā ceturksnī (°C) (HydroClim) analīzes šūnā (1 ha)

\textbf{Procedure:} Information from the \hyperref[Ch04.12]{HydroClim
data} - including both basin and raster layers - is used. First, basin CRS is transformed to EPSG:3059. Then,
zonal statistics (per basin) using a layer specific summary function (min) are
calculated (\texttt{exactextractr::exact\_extract()}), and the the results are rasterised with the workflow
\texttt{egvtools::polygon2input()}. Once rasterised to input data, EGV is created using the workflow
\texttt{egvtools::input2egv()}. To prevent from gaps at the edges, inverse distance
weighted (power = 2) gap filling is implemented. To save disk space,
the intermediate input layer is unlinked. Finally, the layer is standardised by
subtracting the arithmetic mean and dividing by the root mean squared error.

\begin{Shaded}
\begin{Highlighting}[]
\CommentTok{\# libs {-}{-}{-}{-}}
\ControlFlowTok{if}\NormalTok{(}\SpecialCharTok{!}\FunctionTok{require}\NormalTok{(egvtools)) \{remotes}\SpecialCharTok{::}\FunctionTok{install\_github}\NormalTok{(}\StringTok{"aavotins/egvtools"}\NormalTok{); }\FunctionTok{require}\NormalTok{(egvtools)\}}
\ControlFlowTok{if}\NormalTok{(}\SpecialCharTok{!}\FunctionTok{require}\NormalTok{(terra)) \{}\FunctionTok{install.packages}\NormalTok{(}\StringTok{"terra"}\NormalTok{); }\FunctionTok{require}\NormalTok{(terra)\}}
\ControlFlowTok{if}\NormalTok{(}\SpecialCharTok{!}\FunctionTok{require}\NormalTok{(tidyverse)) \{}\FunctionTok{install.packages}\NormalTok{(}\StringTok{"tidyverse"}\NormalTok{); }\FunctionTok{require}\NormalTok{(tidyverse)\}}
\ControlFlowTok{if}\NormalTok{(}\SpecialCharTok{!}\FunctionTok{require}\NormalTok{(sf)) \{}\FunctionTok{install.packages}\NormalTok{(}\StringTok{"sf"}\NormalTok{); }\FunctionTok{require}\NormalTok{(sf)\}}
\ControlFlowTok{if}\NormalTok{(}\SpecialCharTok{!}\FunctionTok{require}\NormalTok{(sfarrow)) \{}\FunctionTok{install.packages}\NormalTok{(}\StringTok{"sfarrow"}\NormalTok{); }\FunctionTok{require}\NormalTok{(sfarrow)\}}
\ControlFlowTok{if}\NormalTok{(}\SpecialCharTok{!}\FunctionTok{require}\NormalTok{(exactextractr)) \{}\FunctionTok{install.packages}\NormalTok{(}\StringTok{"exactextractr"}\NormalTok{); }\FunctionTok{require}\NormalTok{(exactextractr)\}}

\CommentTok{\# basins {-}{-}{-}{-}}
\NormalTok{level12}\OtherTok{=}\FunctionTok{st\_read}\NormalTok{(}\StringTok{"./Geodata/2024/HydroClim/hybas\_lake\_eu\_lev01{-}12\_v1c/hybas\_lake\_eu\_lev12\_v1c.shp"}\NormalTok{)}
\NormalTok{grid\_1km}\OtherTok{=}\NormalTok{sfarrow}\SpecialCharTok{::}\FunctionTok{st\_read\_parquet}\NormalTok{(}\StringTok{"./Templates/TemplateGrids/tikls1km\_sauzeme.parquet"}\NormalTok{)}
\NormalTok{grid\_1km}\OtherTok{=}\FunctionTok{st\_transform}\NormalTok{(grid\_1km,}\AttributeTok{crs=}\DecValTok{3059}\NormalTok{)}
\NormalTok{level12}\OtherTok{=}\FunctionTok{st\_transform}\NormalTok{(level12,}\AttributeTok{crs=}\DecValTok{3059}\NormalTok{)}
\NormalTok{level12}\OtherTok{=}\NormalTok{level12[grid\_1km,,]}

\NormalTok{level12}\OtherTok{=}\FunctionTok{st\_make\_valid}\NormalTok{(level12)}

\CommentTok{\# job {-}{-}{-}{-}}

\NormalTok{localname}\OtherTok{=}\StringTok{"HydroClim\_11{-}min\_cell.tif"}
\NormalTok{layername}\OtherTok{=}\StringTok{"egv\_080"}
\NormalTok{summary\_function}\OtherTok{=}\StringTok{"min"}
 
\NormalTok{slanis}\OtherTok{=}\FunctionTok{rast}\NormalTok{(}\FunctionTok{paste0}\NormalTok{(}\StringTok{"./Geodata/2024/HydroClim/"}\NormalTok{,localname))}
\NormalTok{level12}\SpecialCharTok{$}\NormalTok{Hydro\_values}\OtherTok{=}\FunctionTok{exact\_extract}\NormalTok{(slanis,level12,}\AttributeTok{fun=}\NormalTok{summary\_function)}
 
\FunctionTok{polygon2input}\NormalTok{(}\AttributeTok{vector\_data =}\NormalTok{ level12,}
       \AttributeTok{template\_path =} \StringTok{"./Templates/TemplateRasters/LV10m\_10km.tif"}\NormalTok{,}
       \AttributeTok{out\_path =} \StringTok{"./RasterGrids\_10m/2024/"}\NormalTok{,}
       \AttributeTok{file\_name =}\NormalTok{ localname,}
       \AttributeTok{value\_field =} \StringTok{"Hydro\_values"}\NormalTok{,}
       \AttributeTok{fun=}\StringTok{"first"}\NormalTok{,}
       \AttributeTok{value\_type =} \StringTok{"continuous"}\NormalTok{,}
       \AttributeTok{prepare=}\ConstantTok{FALSE}\NormalTok{,}
       \AttributeTok{project\_mode =} \StringTok{"auto"}\NormalTok{,}
       \AttributeTok{check\_na =} \ConstantTok{FALSE}\NormalTok{,}
       \AttributeTok{plot\_result=}\ConstantTok{FALSE}\NormalTok{,}
       \AttributeTok{plot\_gaps =} \ConstantTok{FALSE}\NormalTok{,}
       \AttributeTok{overwrite=}\ConstantTok{TRUE}\NormalTok{)}
 
\NormalTok{egvrez}\OtherTok{=}\FunctionTok{input2egv}\NormalTok{(}\AttributeTok{input=}\FunctionTok{paste0}\NormalTok{(}\StringTok{"./RasterGrids\_10m/2024/"}\NormalTok{,localname),}
         \AttributeTok{egv\_template=} \StringTok{"./Templates/TemplateRasters/LV100m\_10km.tif"}\NormalTok{,}
         \AttributeTok{summary\_function =} \StringTok{"average"}\NormalTok{,}
         \AttributeTok{missing\_job =} \StringTok{"FillOutput"}\NormalTok{,}
         \AttributeTok{input\_template =} \StringTok{"./Templates/TemplateRasters/LV10m\_10km.tif"}\NormalTok{,}
         \AttributeTok{outlocation =} \StringTok{"./RasterGrids\_100m/2024/RAW/"}\NormalTok{,}
         \AttributeTok{outfilename =}\NormalTok{ localname,}
         \AttributeTok{layername =}\NormalTok{ layername,}
         \AttributeTok{idw\_weight =} \DecValTok{2}\NormalTok{,}
         \AttributeTok{plot\_gaps =} \ConstantTok{FALSE}\NormalTok{,}\AttributeTok{plot\_final =} \ConstantTok{FALSE}\NormalTok{)}
\NormalTok{egvrez}
 
\FunctionTok{unlink}\NormalTok{(}\FunctionTok{paste0}\NormalTok{(}\StringTok{"./RasterGrids\_10m/2024/"}\NormalTok{,localname))}

\CommentTok{\# standardisation {-}{-}{-}{-}}
\ControlFlowTok{if}\NormalTok{(}\SpecialCharTok{!}\FunctionTok{require}\NormalTok{(terra)) \{}\FunctionTok{install.packages}\NormalTok{(}\StringTok{"terra"}\NormalTok{); }\FunctionTok{require}\NormalTok{(terra)\}}
\ControlFlowTok{if}\NormalTok{(}\SpecialCharTok{!}\FunctionTok{require}\NormalTok{(tidyverse)) \{}\FunctionTok{install.packages}\NormalTok{(}\StringTok{"tidyverse"}\NormalTok{); }\FunctionTok{require}\NormalTok{(tidyverse)\}}

\NormalTok{nosaukums}\OtherTok{=}\StringTok{"HydroClim\_11{-}min\_cell.tif"}
\NormalTok{ielasisanas\_cels}\OtherTok{=}\FunctionTok{paste0}\NormalTok{(}\StringTok{"./RasterGrids\_100m/2024/RAW/"}\NormalTok{,nosaukums)}
\NormalTok{saglabasanas\_cels}\OtherTok{=}\FunctionTok{paste0}\NormalTok{(}\StringTok{"./RasterGrids\_100m/2024/Scaled/"}\NormalTok{,nosaukums)}
\NormalTok{slanis}\OtherTok{=}\FunctionTok{rast}\NormalTok{(ielasisanas\_cels)}
\NormalTok{videjais}\OtherTok{=}\FunctionTok{global}\NormalTok{(slanis,}\AttributeTok{fun=}\StringTok{"mean"}\NormalTok{,}\AttributeTok{na.rm=}\ConstantTok{TRUE}\NormalTok{)}
\NormalTok{centrets}\OtherTok{=}\NormalTok{slanis}\SpecialCharTok{{-}}\NormalTok{videjais[,}\DecValTok{1}\NormalTok{]}
\NormalTok{standartnovirze}\OtherTok{=}\NormalTok{terra}\SpecialCharTok{::}\FunctionTok{global}\NormalTok{(centrets,}\AttributeTok{fun=}\StringTok{"rms"}\NormalTok{,}\AttributeTok{na.rm=}\ConstantTok{TRUE}\NormalTok{)}
\NormalTok{merogots}\OtherTok{=}\NormalTok{centrets}\SpecialCharTok{/}\NormalTok{standartnovirze[,}\DecValTok{1}\NormalTok{]}
\FunctionTok{writeRaster}\NormalTok{(merogots,}
      \AttributeTok{filename=}\NormalTok{saglabasanas\_cels,}
      \AttributeTok{overwrite=}\ConstantTok{TRUE}\NormalTok{)}
\end{Highlighting}
\end{Shaded}

\section{HydroClim\_12-max\_cell}\label{ch06.081}

\textbf{filename:} \texttt{HydroClim\_12-max\_cell.tif}

\textbf{layername:} \texttt{egv\_081}

\textbf{English name:} Maximum per subcatchment upstream annual precipitation amount
(kg m⁻² year⁻¹) (HydroClim) within the analysis cell (1 ha)

\textbf{Latvian name:} Sateces apakšbaseina maksimālais augšteces nokrišņu daudzums
gadā (kg m⁻² year⁻¹) (HydroClim) analīzes šūnā (1 ha)

\textbf{Procedure:} Information from the \hyperref[Ch04.12]{HydroClim
data} - including both basin and raster layers - is used. First, basin CRS is transformed to EPSG:3059. Then,
zonal statistics (per basin) using a layer specific summary function (max) are
calculated (\texttt{exactextractr::exact\_extract()}), and the the results are rasterised with the workflow
\texttt{egvtools::polygon2input()}. Once rasterised to input data, EGV is created using the workflow
\texttt{egvtools::input2egv()}. To prevent from gaps at the edges, inverse distance
weighted (power = 2) gap filling is implemented. To save disk space,
the intermediate input layer is unlinked. Finally, the layer is standardised by
subtracting the arithmetic mean and dividing by the root mean squared error.

\begin{Shaded}
\begin{Highlighting}[]
\CommentTok{\# libs {-}{-}{-}{-}}
\ControlFlowTok{if}\NormalTok{(}\SpecialCharTok{!}\FunctionTok{require}\NormalTok{(egvtools)) \{remotes}\SpecialCharTok{::}\FunctionTok{install\_github}\NormalTok{(}\StringTok{"aavotins/egvtools"}\NormalTok{); }\FunctionTok{require}\NormalTok{(egvtools)\}}
\ControlFlowTok{if}\NormalTok{(}\SpecialCharTok{!}\FunctionTok{require}\NormalTok{(terra)) \{}\FunctionTok{install.packages}\NormalTok{(}\StringTok{"terra"}\NormalTok{); }\FunctionTok{require}\NormalTok{(terra)\}}
\ControlFlowTok{if}\NormalTok{(}\SpecialCharTok{!}\FunctionTok{require}\NormalTok{(tidyverse)) \{}\FunctionTok{install.packages}\NormalTok{(}\StringTok{"tidyverse"}\NormalTok{); }\FunctionTok{require}\NormalTok{(tidyverse)\}}
\ControlFlowTok{if}\NormalTok{(}\SpecialCharTok{!}\FunctionTok{require}\NormalTok{(sf)) \{}\FunctionTok{install.packages}\NormalTok{(}\StringTok{"sf"}\NormalTok{); }\FunctionTok{require}\NormalTok{(sf)\}}
\ControlFlowTok{if}\NormalTok{(}\SpecialCharTok{!}\FunctionTok{require}\NormalTok{(sfarrow)) \{}\FunctionTok{install.packages}\NormalTok{(}\StringTok{"sfarrow"}\NormalTok{); }\FunctionTok{require}\NormalTok{(sfarrow)\}}
\ControlFlowTok{if}\NormalTok{(}\SpecialCharTok{!}\FunctionTok{require}\NormalTok{(exactextractr)) \{}\FunctionTok{install.packages}\NormalTok{(}\StringTok{"exactextractr"}\NormalTok{); }\FunctionTok{require}\NormalTok{(exactextractr)\}}

\CommentTok{\# basins {-}{-}{-}{-}}
\NormalTok{level12}\OtherTok{=}\FunctionTok{st\_read}\NormalTok{(}\StringTok{"./Geodata/2024/HydroClim/hybas\_lake\_eu\_lev01{-}12\_v1c/hybas\_lake\_eu\_lev12\_v1c.shp"}\NormalTok{)}
\NormalTok{grid\_1km}\OtherTok{=}\NormalTok{sfarrow}\SpecialCharTok{::}\FunctionTok{st\_read\_parquet}\NormalTok{(}\StringTok{"./Templates/TemplateGrids/tikls1km\_sauzeme.parquet"}\NormalTok{)}
\NormalTok{grid\_1km}\OtherTok{=}\FunctionTok{st\_transform}\NormalTok{(grid\_1km,}\AttributeTok{crs=}\DecValTok{3059}\NormalTok{)}
\NormalTok{level12}\OtherTok{=}\FunctionTok{st\_transform}\NormalTok{(level12,}\AttributeTok{crs=}\DecValTok{3059}\NormalTok{)}
\NormalTok{level12}\OtherTok{=}\NormalTok{level12[grid\_1km,,]}

\NormalTok{level12}\OtherTok{=}\FunctionTok{st\_make\_valid}\NormalTok{(level12)}

\CommentTok{\# job {-}{-}{-}{-}}

\NormalTok{localname}\OtherTok{=}\StringTok{"HydroClim\_12{-}max\_cell.tif"}
\NormalTok{layername}\OtherTok{=}\StringTok{"egv\_081"}
\NormalTok{summary\_function}\OtherTok{=}\StringTok{"max"}
 
\NormalTok{slanis}\OtherTok{=}\FunctionTok{rast}\NormalTok{(}\FunctionTok{paste0}\NormalTok{(}\StringTok{"./Geodata/2024/HydroClim/"}\NormalTok{,localname))}
\NormalTok{level12}\SpecialCharTok{$}\NormalTok{Hydro\_values}\OtherTok{=}\FunctionTok{exact\_extract}\NormalTok{(slanis,level12,}\AttributeTok{fun=}\NormalTok{summary\_function)}
 
\FunctionTok{polygon2input}\NormalTok{(}\AttributeTok{vector\_data =}\NormalTok{ level12,}
       \AttributeTok{template\_path =} \StringTok{"./Templates/TemplateRasters/LV10m\_10km.tif"}\NormalTok{,}
       \AttributeTok{out\_path =} \StringTok{"./RasterGrids\_10m/2024/"}\NormalTok{,}
       \AttributeTok{file\_name =}\NormalTok{ localname,}
       \AttributeTok{value\_field =} \StringTok{"Hydro\_values"}\NormalTok{,}
       \AttributeTok{fun=}\StringTok{"first"}\NormalTok{,}
       \AttributeTok{value\_type =} \StringTok{"continuous"}\NormalTok{,}
       \AttributeTok{prepare=}\ConstantTok{FALSE}\NormalTok{,}
       \AttributeTok{project\_mode =} \StringTok{"auto"}\NormalTok{,}
       \AttributeTok{check\_na =} \ConstantTok{FALSE}\NormalTok{,}
       \AttributeTok{plot\_result=}\ConstantTok{FALSE}\NormalTok{,}
       \AttributeTok{plot\_gaps =} \ConstantTok{FALSE}\NormalTok{,}
       \AttributeTok{overwrite=}\ConstantTok{TRUE}\NormalTok{)}
 
\NormalTok{egvrez}\OtherTok{=}\FunctionTok{input2egv}\NormalTok{(}\AttributeTok{input=}\FunctionTok{paste0}\NormalTok{(}\StringTok{"./RasterGrids\_10m/2024/"}\NormalTok{,localname),}
         \AttributeTok{egv\_template=} \StringTok{"./Templates/TemplateRasters/LV100m\_10km.tif"}\NormalTok{,}
         \AttributeTok{summary\_function =} \StringTok{"average"}\NormalTok{,}
         \AttributeTok{missing\_job =} \StringTok{"FillOutput"}\NormalTok{,}
         \AttributeTok{input\_template =} \StringTok{"./Templates/TemplateRasters/LV10m\_10km.tif"}\NormalTok{,}
         \AttributeTok{outlocation =} \StringTok{"./RasterGrids\_100m/2024/RAW/"}\NormalTok{,}
         \AttributeTok{outfilename =}\NormalTok{ localname,}
         \AttributeTok{layername =}\NormalTok{ layername,}
         \AttributeTok{idw\_weight =} \DecValTok{2}\NormalTok{,}
         \AttributeTok{plot\_gaps =} \ConstantTok{FALSE}\NormalTok{,}\AttributeTok{plot\_final =} \ConstantTok{FALSE}\NormalTok{)}
\NormalTok{egvrez}
 
\FunctionTok{unlink}\NormalTok{(}\FunctionTok{paste0}\NormalTok{(}\StringTok{"./RasterGrids\_10m/2024/"}\NormalTok{,localname))}

\CommentTok{\# standardisation {-}{-}{-}{-}}
\ControlFlowTok{if}\NormalTok{(}\SpecialCharTok{!}\FunctionTok{require}\NormalTok{(terra)) \{}\FunctionTok{install.packages}\NormalTok{(}\StringTok{"terra"}\NormalTok{); }\FunctionTok{require}\NormalTok{(terra)\}}
\ControlFlowTok{if}\NormalTok{(}\SpecialCharTok{!}\FunctionTok{require}\NormalTok{(tidyverse)) \{}\FunctionTok{install.packages}\NormalTok{(}\StringTok{"tidyverse"}\NormalTok{); }\FunctionTok{require}\NormalTok{(tidyverse)\}}

\NormalTok{nosaukums}\OtherTok{=}\StringTok{"HydroClim\_12{-}max\_cell.tif"}
\NormalTok{ielasisanas\_cels}\OtherTok{=}\FunctionTok{paste0}\NormalTok{(}\StringTok{"./RasterGrids\_100m/2024/RAW/"}\NormalTok{,nosaukums)}
\NormalTok{saglabasanas\_cels}\OtherTok{=}\FunctionTok{paste0}\NormalTok{(}\StringTok{"./RasterGrids\_100m/2024/Scaled/"}\NormalTok{,nosaukums)}
\NormalTok{slanis}\OtherTok{=}\FunctionTok{rast}\NormalTok{(ielasisanas\_cels)}
\NormalTok{videjais}\OtherTok{=}\FunctionTok{global}\NormalTok{(slanis,}\AttributeTok{fun=}\StringTok{"mean"}\NormalTok{,}\AttributeTok{na.rm=}\ConstantTok{TRUE}\NormalTok{)}
\NormalTok{centrets}\OtherTok{=}\NormalTok{slanis}\SpecialCharTok{{-}}\NormalTok{videjais[,}\DecValTok{1}\NormalTok{]}
\NormalTok{standartnovirze}\OtherTok{=}\NormalTok{terra}\SpecialCharTok{::}\FunctionTok{global}\NormalTok{(centrets,}\AttributeTok{fun=}\StringTok{"rms"}\NormalTok{,}\AttributeTok{na.rm=}\ConstantTok{TRUE}\NormalTok{)}
\NormalTok{merogots}\OtherTok{=}\NormalTok{centrets}\SpecialCharTok{/}\NormalTok{standartnovirze[,}\DecValTok{1}\NormalTok{]}
\FunctionTok{writeRaster}\NormalTok{(merogots,}
      \AttributeTok{filename=}\NormalTok{saglabasanas\_cels,}
      \AttributeTok{overwrite=}\ConstantTok{TRUE}\NormalTok{)}
\end{Highlighting}
\end{Shaded}

\section{HydroClim\_13-max\_cell}\label{ch06.082}

\textbf{filename:} \texttt{HydroClim\_13-max\_cell.tif}

\textbf{layername:} \texttt{egv\_082}

\textbf{English name:} Maximum per subcatchment upstream precipitation amount (kg m⁻²
year⁻¹) of the wettest month (HydroClim) within the analysis cell (1 ha)

\textbf{Latvian name:} Sateces apakšbaseina maksimālais augšteces nokrišņu daudzums
mitrākajā mēnesī (kg m⁻² year⁻¹) (HydroClim) analīzes šūnā (1 ha)

\textbf{Procedure:} Information from the \hyperref[Ch04.12]{HydroClim
data} - including both basin and raster layers - is used. First, basin CRS is transformed to EPSG:3059. Then,
zonal statistics (per basin) using a layer specific summary function (max) are
calculated (\texttt{exactextractr::exact\_extract()}), and the the results are rasterised with the workflow
\texttt{egvtools::polygon2input()}. Once rasterised to input data, EGV is created using the workflow
\texttt{egvtools::input2egv()}. To prevent from gaps at the edges, inverse distance
weighted (power = 2) gap filling is implemented. To save disk space,
the intermediate input layer is unlinked. Finally, the layer is standardised by
subtracting the arithmetic mean and dividing by the root mean squared error.

\begin{Shaded}
\begin{Highlighting}[]
\CommentTok{\# libs {-}{-}{-}{-}}
\ControlFlowTok{if}\NormalTok{(}\SpecialCharTok{!}\FunctionTok{require}\NormalTok{(egvtools)) \{remotes}\SpecialCharTok{::}\FunctionTok{install\_github}\NormalTok{(}\StringTok{"aavotins/egvtools"}\NormalTok{); }\FunctionTok{require}\NormalTok{(egvtools)\}}
\ControlFlowTok{if}\NormalTok{(}\SpecialCharTok{!}\FunctionTok{require}\NormalTok{(terra)) \{}\FunctionTok{install.packages}\NormalTok{(}\StringTok{"terra"}\NormalTok{); }\FunctionTok{require}\NormalTok{(terra)\}}
\ControlFlowTok{if}\NormalTok{(}\SpecialCharTok{!}\FunctionTok{require}\NormalTok{(tidyverse)) \{}\FunctionTok{install.packages}\NormalTok{(}\StringTok{"tidyverse"}\NormalTok{); }\FunctionTok{require}\NormalTok{(tidyverse)\}}
\ControlFlowTok{if}\NormalTok{(}\SpecialCharTok{!}\FunctionTok{require}\NormalTok{(sf)) \{}\FunctionTok{install.packages}\NormalTok{(}\StringTok{"sf"}\NormalTok{); }\FunctionTok{require}\NormalTok{(sf)\}}
\ControlFlowTok{if}\NormalTok{(}\SpecialCharTok{!}\FunctionTok{require}\NormalTok{(sfarrow)) \{}\FunctionTok{install.packages}\NormalTok{(}\StringTok{"sfarrow"}\NormalTok{); }\FunctionTok{require}\NormalTok{(sfarrow)\}}
\ControlFlowTok{if}\NormalTok{(}\SpecialCharTok{!}\FunctionTok{require}\NormalTok{(exactextractr)) \{}\FunctionTok{install.packages}\NormalTok{(}\StringTok{"exactextractr"}\NormalTok{); }\FunctionTok{require}\NormalTok{(exactextractr)\}}

\CommentTok{\# basins {-}{-}{-}{-}}
\NormalTok{level12}\OtherTok{=}\FunctionTok{st\_read}\NormalTok{(}\StringTok{"./Geodata/2024/HydroClim/hybas\_lake\_eu\_lev01{-}12\_v1c/hybas\_lake\_eu\_lev12\_v1c.shp"}\NormalTok{)}
\NormalTok{grid\_1km}\OtherTok{=}\NormalTok{sfarrow}\SpecialCharTok{::}\FunctionTok{st\_read\_parquet}\NormalTok{(}\StringTok{"./Templates/TemplateGrids/tikls1km\_sauzeme.parquet"}\NormalTok{)}
\NormalTok{grid\_1km}\OtherTok{=}\FunctionTok{st\_transform}\NormalTok{(grid\_1km,}\AttributeTok{crs=}\DecValTok{3059}\NormalTok{)}
\NormalTok{level12}\OtherTok{=}\FunctionTok{st\_transform}\NormalTok{(level12,}\AttributeTok{crs=}\DecValTok{3059}\NormalTok{)}
\NormalTok{level12}\OtherTok{=}\NormalTok{level12[grid\_1km,,]}

\NormalTok{level12}\OtherTok{=}\FunctionTok{st\_make\_valid}\NormalTok{(level12)}

\CommentTok{\# job {-}{-}{-}{-}}

\NormalTok{localname}\OtherTok{=}\StringTok{"HydroClim\_13{-}max\_cell.tif"}
\NormalTok{layername}\OtherTok{=}\StringTok{"egv\_082"}
\NormalTok{summary\_function}\OtherTok{=}\StringTok{"max"}
 
\NormalTok{slanis}\OtherTok{=}\FunctionTok{rast}\NormalTok{(}\FunctionTok{paste0}\NormalTok{(}\StringTok{"./Geodata/2024/HydroClim/"}\NormalTok{,localname))}
\NormalTok{level12}\SpecialCharTok{$}\NormalTok{Hydro\_values}\OtherTok{=}\FunctionTok{exact\_extract}\NormalTok{(slanis,level12,}\AttributeTok{fun=}\NormalTok{summary\_function)}
 
\FunctionTok{polygon2input}\NormalTok{(}\AttributeTok{vector\_data =}\NormalTok{ level12,}
       \AttributeTok{template\_path =} \StringTok{"./Templates/TemplateRasters/LV10m\_10km.tif"}\NormalTok{,}
       \AttributeTok{out\_path =} \StringTok{"./RasterGrids\_10m/2024/"}\NormalTok{,}
       \AttributeTok{file\_name =}\NormalTok{ localname,}
       \AttributeTok{value\_field =} \StringTok{"Hydro\_values"}\NormalTok{,}
       \AttributeTok{fun=}\StringTok{"first"}\NormalTok{,}
       \AttributeTok{value\_type =} \StringTok{"continuous"}\NormalTok{,}
       \AttributeTok{prepare=}\ConstantTok{FALSE}\NormalTok{,}
       \AttributeTok{project\_mode =} \StringTok{"auto"}\NormalTok{,}
       \AttributeTok{check\_na =} \ConstantTok{FALSE}\NormalTok{,}
       \AttributeTok{plot\_result=}\ConstantTok{FALSE}\NormalTok{,}
       \AttributeTok{plot\_gaps =} \ConstantTok{FALSE}\NormalTok{,}
       \AttributeTok{overwrite=}\ConstantTok{TRUE}\NormalTok{)}
 
\NormalTok{egvrez}\OtherTok{=}\FunctionTok{input2egv}\NormalTok{(}\AttributeTok{input=}\FunctionTok{paste0}\NormalTok{(}\StringTok{"./RasterGrids\_10m/2024/"}\NormalTok{,localname),}
         \AttributeTok{egv\_template=} \StringTok{"./Templates/TemplateRasters/LV100m\_10km.tif"}\NormalTok{,}
         \AttributeTok{summary\_function =} \StringTok{"average"}\NormalTok{,}
         \AttributeTok{missing\_job =} \StringTok{"FillOutput"}\NormalTok{,}
         \AttributeTok{input\_template =} \StringTok{"./Templates/TemplateRasters/LV10m\_10km.tif"}\NormalTok{,}
         \AttributeTok{outlocation =} \StringTok{"./RasterGrids\_100m/2024/RAW/"}\NormalTok{,}
         \AttributeTok{outfilename =}\NormalTok{ localname,}
         \AttributeTok{layername =}\NormalTok{ layername,}
         \AttributeTok{idw\_weight =} \DecValTok{2}\NormalTok{,}
         \AttributeTok{plot\_gaps =} \ConstantTok{FALSE}\NormalTok{,}\AttributeTok{plot\_final =} \ConstantTok{FALSE}\NormalTok{)}
\NormalTok{egvrez}
 
\FunctionTok{unlink}\NormalTok{(}\FunctionTok{paste0}\NormalTok{(}\StringTok{"./RasterGrids\_10m/2024/"}\NormalTok{,localname))}

\CommentTok{\# standardisation {-}{-}{-}{-}}
\ControlFlowTok{if}\NormalTok{(}\SpecialCharTok{!}\FunctionTok{require}\NormalTok{(terra)) \{}\FunctionTok{install.packages}\NormalTok{(}\StringTok{"terra"}\NormalTok{); }\FunctionTok{require}\NormalTok{(terra)\}}
\ControlFlowTok{if}\NormalTok{(}\SpecialCharTok{!}\FunctionTok{require}\NormalTok{(tidyverse)) \{}\FunctionTok{install.packages}\NormalTok{(}\StringTok{"tidyverse"}\NormalTok{); }\FunctionTok{require}\NormalTok{(tidyverse)\}}

\NormalTok{nosaukums}\OtherTok{=}\StringTok{"HydroClim\_13{-}max\_cell.tif"}
\NormalTok{ielasisanas\_cels}\OtherTok{=}\FunctionTok{paste0}\NormalTok{(}\StringTok{"./RasterGrids\_100m/2024/RAW/"}\NormalTok{,nosaukums)}
\NormalTok{saglabasanas\_cels}\OtherTok{=}\FunctionTok{paste0}\NormalTok{(}\StringTok{"./RasterGrids\_100m/2024/Scaled/"}\NormalTok{,nosaukums)}
\NormalTok{slanis}\OtherTok{=}\FunctionTok{rast}\NormalTok{(ielasisanas\_cels)}
\NormalTok{videjais}\OtherTok{=}\FunctionTok{global}\NormalTok{(slanis,}\AttributeTok{fun=}\StringTok{"mean"}\NormalTok{,}\AttributeTok{na.rm=}\ConstantTok{TRUE}\NormalTok{)}
\NormalTok{centrets}\OtherTok{=}\NormalTok{slanis}\SpecialCharTok{{-}}\NormalTok{videjais[,}\DecValTok{1}\NormalTok{]}
\NormalTok{standartnovirze}\OtherTok{=}\NormalTok{terra}\SpecialCharTok{::}\FunctionTok{global}\NormalTok{(centrets,}\AttributeTok{fun=}\StringTok{"rms"}\NormalTok{,}\AttributeTok{na.rm=}\ConstantTok{TRUE}\NormalTok{)}
\NormalTok{merogots}\OtherTok{=}\NormalTok{centrets}\SpecialCharTok{/}\NormalTok{standartnovirze[,}\DecValTok{1}\NormalTok{]}
\FunctionTok{writeRaster}\NormalTok{(merogots,}
      \AttributeTok{filename=}\NormalTok{saglabasanas\_cels,}
      \AttributeTok{overwrite=}\ConstantTok{TRUE}\NormalTok{)}
\end{Highlighting}
\end{Shaded}

\section{HydroClim\_14-max\_cell}\label{ch06.083}

\textbf{filename:} \texttt{HydroClim\_14-max\_cell.tif}

\textbf{layername:} \texttt{egv\_083}

\textbf{English name:} Maximum per subcatchment upstream precipitation amount (kg m⁻²
year⁻¹) of the driest month (HydroClim) within the analysis cell (1 ha)

\textbf{Latvian name:} Sateces apakšbaseina maksimālais augšteces nokrišņu daudzums
sausākajā mēnesī (kg m⁻² year⁻¹) (HydroClim) analīzes šūnā (1 ha)

\textbf{Procedure:} Information from the \hyperref[Ch04.12]{HydroClim
data} - including both basin and raster layers - is used. First, basin CRS is transformed to EPSG:3059. Then,
zonal statistics (per basin) using a layer specific summary function (max) are
calculated (\texttt{exactextractr::exact\_extract()}), and the the results are rasterised with the workflow
\texttt{egvtools::polygon2input()}. Once rasterised to input data, EGV is created using the workflow
\texttt{egvtools::input2egv()}. To prevent from gaps at the edges, inverse distance
weighted (power = 2) gap filling is implemented. To save disk space,
the intermediate input layer is unlinked. Finally, the layer is standardised by
subtracting the arithmetic mean and dividing by the root mean squared error.

\begin{Shaded}
\begin{Highlighting}[]
\CommentTok{\# libs {-}{-}{-}{-}}
\ControlFlowTok{if}\NormalTok{(}\SpecialCharTok{!}\FunctionTok{require}\NormalTok{(egvtools)) \{remotes}\SpecialCharTok{::}\FunctionTok{install\_github}\NormalTok{(}\StringTok{"aavotins/egvtools"}\NormalTok{); }\FunctionTok{require}\NormalTok{(egvtools)\}}
\ControlFlowTok{if}\NormalTok{(}\SpecialCharTok{!}\FunctionTok{require}\NormalTok{(terra)) \{}\FunctionTok{install.packages}\NormalTok{(}\StringTok{"terra"}\NormalTok{); }\FunctionTok{require}\NormalTok{(terra)\}}
\ControlFlowTok{if}\NormalTok{(}\SpecialCharTok{!}\FunctionTok{require}\NormalTok{(tidyverse)) \{}\FunctionTok{install.packages}\NormalTok{(}\StringTok{"tidyverse"}\NormalTok{); }\FunctionTok{require}\NormalTok{(tidyverse)\}}
\ControlFlowTok{if}\NormalTok{(}\SpecialCharTok{!}\FunctionTok{require}\NormalTok{(sf)) \{}\FunctionTok{install.packages}\NormalTok{(}\StringTok{"sf"}\NormalTok{); }\FunctionTok{require}\NormalTok{(sf)\}}
\ControlFlowTok{if}\NormalTok{(}\SpecialCharTok{!}\FunctionTok{require}\NormalTok{(sfarrow)) \{}\FunctionTok{install.packages}\NormalTok{(}\StringTok{"sfarrow"}\NormalTok{); }\FunctionTok{require}\NormalTok{(sfarrow)\}}
\ControlFlowTok{if}\NormalTok{(}\SpecialCharTok{!}\FunctionTok{require}\NormalTok{(exactextractr)) \{}\FunctionTok{install.packages}\NormalTok{(}\StringTok{"exactextractr"}\NormalTok{); }\FunctionTok{require}\NormalTok{(exactextractr)\}}

\CommentTok{\# basins {-}{-}{-}{-}}
\NormalTok{level12}\OtherTok{=}\FunctionTok{st\_read}\NormalTok{(}\StringTok{"./Geodata/2024/HydroClim/hybas\_lake\_eu\_lev01{-}12\_v1c/hybas\_lake\_eu\_lev12\_v1c.shp"}\NormalTok{)}
\NormalTok{grid\_1km}\OtherTok{=}\NormalTok{sfarrow}\SpecialCharTok{::}\FunctionTok{st\_read\_parquet}\NormalTok{(}\StringTok{"./Templates/TemplateGrids/tikls1km\_sauzeme.parquet"}\NormalTok{)}
\NormalTok{grid\_1km}\OtherTok{=}\FunctionTok{st\_transform}\NormalTok{(grid\_1km,}\AttributeTok{crs=}\DecValTok{3059}\NormalTok{)}
\NormalTok{level12}\OtherTok{=}\FunctionTok{st\_transform}\NormalTok{(level12,}\AttributeTok{crs=}\DecValTok{3059}\NormalTok{)}
\NormalTok{level12}\OtherTok{=}\NormalTok{level12[grid\_1km,,]}

\NormalTok{level12}\OtherTok{=}\FunctionTok{st\_make\_valid}\NormalTok{(level12)}

\CommentTok{\# job {-}{-}{-}{-}}

\NormalTok{localname}\OtherTok{=}\StringTok{"HydroClim\_14{-}max\_cell.tif"}
\NormalTok{layername}\OtherTok{=}\StringTok{"egv\_083"}
\NormalTok{summary\_function}\OtherTok{=}\StringTok{"max"}
 
\NormalTok{slanis}\OtherTok{=}\FunctionTok{rast}\NormalTok{(}\FunctionTok{paste0}\NormalTok{(}\StringTok{"./Geodata/2024/HydroClim/"}\NormalTok{,localname))}
\NormalTok{level12}\SpecialCharTok{$}\NormalTok{Hydro\_values}\OtherTok{=}\FunctionTok{exact\_extract}\NormalTok{(slanis,level12,}\AttributeTok{fun=}\NormalTok{summary\_function)}
 
\FunctionTok{polygon2input}\NormalTok{(}\AttributeTok{vector\_data =}\NormalTok{ level12,}
       \AttributeTok{template\_path =} \StringTok{"./Templates/TemplateRasters/LV10m\_10km.tif"}\NormalTok{,}
       \AttributeTok{out\_path =} \StringTok{"./RasterGrids\_10m/2024/"}\NormalTok{,}
       \AttributeTok{file\_name =}\NormalTok{ localname,}
       \AttributeTok{value\_field =} \StringTok{"Hydro\_values"}\NormalTok{,}
       \AttributeTok{fun=}\StringTok{"first"}\NormalTok{,}
       \AttributeTok{value\_type =} \StringTok{"continuous"}\NormalTok{,}
       \AttributeTok{prepare=}\ConstantTok{FALSE}\NormalTok{,}
       \AttributeTok{project\_mode =} \StringTok{"auto"}\NormalTok{,}
       \AttributeTok{check\_na =} \ConstantTok{FALSE}\NormalTok{,}
       \AttributeTok{plot\_result=}\ConstantTok{FALSE}\NormalTok{,}
       \AttributeTok{plot\_gaps =} \ConstantTok{FALSE}\NormalTok{,}
       \AttributeTok{overwrite=}\ConstantTok{TRUE}\NormalTok{)}
 
\NormalTok{egvrez}\OtherTok{=}\FunctionTok{input2egv}\NormalTok{(}\AttributeTok{input=}\FunctionTok{paste0}\NormalTok{(}\StringTok{"./RasterGrids\_10m/2024/"}\NormalTok{,localname),}
         \AttributeTok{egv\_template=} \StringTok{"./Templates/TemplateRasters/LV100m\_10km.tif"}\NormalTok{,}
         \AttributeTok{summary\_function =} \StringTok{"average"}\NormalTok{,}
         \AttributeTok{missing\_job =} \StringTok{"FillOutput"}\NormalTok{,}
         \AttributeTok{input\_template =} \StringTok{"./Templates/TemplateRasters/LV10m\_10km.tif"}\NormalTok{,}
         \AttributeTok{outlocation =} \StringTok{"./RasterGrids\_100m/2024/RAW/"}\NormalTok{,}
         \AttributeTok{outfilename =}\NormalTok{ localname,}
         \AttributeTok{layername =}\NormalTok{ layername,}
         \AttributeTok{idw\_weight =} \DecValTok{2}\NormalTok{,}
         \AttributeTok{plot\_gaps =} \ConstantTok{FALSE}\NormalTok{,}\AttributeTok{plot\_final =} \ConstantTok{FALSE}\NormalTok{)}
\NormalTok{egvrez}
 
\FunctionTok{unlink}\NormalTok{(}\FunctionTok{paste0}\NormalTok{(}\StringTok{"./RasterGrids\_10m/2024/"}\NormalTok{,localname))}

\CommentTok{\# standardisation {-}{-}{-}{-}}
\ControlFlowTok{if}\NormalTok{(}\SpecialCharTok{!}\FunctionTok{require}\NormalTok{(terra)) \{}\FunctionTok{install.packages}\NormalTok{(}\StringTok{"terra"}\NormalTok{); }\FunctionTok{require}\NormalTok{(terra)\}}
\ControlFlowTok{if}\NormalTok{(}\SpecialCharTok{!}\FunctionTok{require}\NormalTok{(tidyverse)) \{}\FunctionTok{install.packages}\NormalTok{(}\StringTok{"tidyverse"}\NormalTok{); }\FunctionTok{require}\NormalTok{(tidyverse)\}}

\NormalTok{nosaukums}\OtherTok{=}\StringTok{"HydroClim\_14{-}max\_cell.tif"}
\NormalTok{ielasisanas\_cels}\OtherTok{=}\FunctionTok{paste0}\NormalTok{(}\StringTok{"./RasterGrids\_100m/2024/RAW/"}\NormalTok{,nosaukums)}
\NormalTok{saglabasanas\_cels}\OtherTok{=}\FunctionTok{paste0}\NormalTok{(}\StringTok{"./RasterGrids\_100m/2024/Scaled/"}\NormalTok{,nosaukums)}
\NormalTok{slanis}\OtherTok{=}\FunctionTok{rast}\NormalTok{(ielasisanas\_cels)}
\NormalTok{videjais}\OtherTok{=}\FunctionTok{global}\NormalTok{(slanis,}\AttributeTok{fun=}\StringTok{"mean"}\NormalTok{,}\AttributeTok{na.rm=}\ConstantTok{TRUE}\NormalTok{)}
\NormalTok{centrets}\OtherTok{=}\NormalTok{slanis}\SpecialCharTok{{-}}\NormalTok{videjais[,}\DecValTok{1}\NormalTok{]}
\NormalTok{standartnovirze}\OtherTok{=}\NormalTok{terra}\SpecialCharTok{::}\FunctionTok{global}\NormalTok{(centrets,}\AttributeTok{fun=}\StringTok{"rms"}\NormalTok{,}\AttributeTok{na.rm=}\ConstantTok{TRUE}\NormalTok{)}
\NormalTok{merogots}\OtherTok{=}\NormalTok{centrets}\SpecialCharTok{/}\NormalTok{standartnovirze[,}\DecValTok{1}\NormalTok{]}
\FunctionTok{writeRaster}\NormalTok{(merogots,}
      \AttributeTok{filename=}\NormalTok{saglabasanas\_cels,}
      \AttributeTok{overwrite=}\ConstantTok{TRUE}\NormalTok{)}
\end{Highlighting}
\end{Shaded}

\section{HydroClim\_15-max\_cell}\label{ch06.084}

\textbf{filename:} \texttt{HydroClim\_15-max\_cell.tif}

\textbf{layername:} \texttt{egv\_084}

\textbf{English name:} Maximum per subcatchment upstream precipitation seasonality
(kg m⁻²) (HydroClim) within the analysis cell (1 ha)

\textbf{Latvian name:} Sateces apakšbaseina maksimālais augšteces nokrišņu daudzuma
sezonalitāte (kg m⁻²) (HydroClim) analīzes šūnā (1 ha)

\textbf{Procedure:} Information from the \hyperref[Ch04.12]{HydroClim
data} - including both basin and raster layers - is used. First, basin CRS is transformed to EPSG:3059. Then,
zonal statistics (per basin) using a layer specific summary function (max) are
calculated (\texttt{exactextractr::exact\_extract()}), and the the results are rasterised with the workflow
\texttt{egvtools::polygon2input()}. Once rasterised to input data, EGV is created using the workflow
\texttt{egvtools::input2egv()}. To prevent from gaps at the edges, inverse distance
weighted (power = 2) gap filling is implemented. To save disk space,
the intermediate input layer is unlinked. Finally, the layer is standardised by
subtracting the arithmetic mean and dividing by the root mean squared error.

\begin{Shaded}
\begin{Highlighting}[]
\CommentTok{\# libs {-}{-}{-}{-}}
\ControlFlowTok{if}\NormalTok{(}\SpecialCharTok{!}\FunctionTok{require}\NormalTok{(egvtools)) \{remotes}\SpecialCharTok{::}\FunctionTok{install\_github}\NormalTok{(}\StringTok{"aavotins/egvtools"}\NormalTok{); }\FunctionTok{require}\NormalTok{(egvtools)\}}
\ControlFlowTok{if}\NormalTok{(}\SpecialCharTok{!}\FunctionTok{require}\NormalTok{(terra)) \{}\FunctionTok{install.packages}\NormalTok{(}\StringTok{"terra"}\NormalTok{); }\FunctionTok{require}\NormalTok{(terra)\}}
\ControlFlowTok{if}\NormalTok{(}\SpecialCharTok{!}\FunctionTok{require}\NormalTok{(tidyverse)) \{}\FunctionTok{install.packages}\NormalTok{(}\StringTok{"tidyverse"}\NormalTok{); }\FunctionTok{require}\NormalTok{(tidyverse)\}}
\ControlFlowTok{if}\NormalTok{(}\SpecialCharTok{!}\FunctionTok{require}\NormalTok{(sf)) \{}\FunctionTok{install.packages}\NormalTok{(}\StringTok{"sf"}\NormalTok{); }\FunctionTok{require}\NormalTok{(sf)\}}
\ControlFlowTok{if}\NormalTok{(}\SpecialCharTok{!}\FunctionTok{require}\NormalTok{(sfarrow)) \{}\FunctionTok{install.packages}\NormalTok{(}\StringTok{"sfarrow"}\NormalTok{); }\FunctionTok{require}\NormalTok{(sfarrow)\}}
\ControlFlowTok{if}\NormalTok{(}\SpecialCharTok{!}\FunctionTok{require}\NormalTok{(exactextractr)) \{}\FunctionTok{install.packages}\NormalTok{(}\StringTok{"exactextractr"}\NormalTok{); }\FunctionTok{require}\NormalTok{(exactextractr)\}}

\CommentTok{\# basins {-}{-}{-}{-}}
\NormalTok{level12}\OtherTok{=}\FunctionTok{st\_read}\NormalTok{(}\StringTok{"./Geodata/2024/HydroClim/hybas\_lake\_eu\_lev01{-}12\_v1c/hybas\_lake\_eu\_lev12\_v1c.shp"}\NormalTok{)}
\NormalTok{grid\_1km}\OtherTok{=}\NormalTok{sfarrow}\SpecialCharTok{::}\FunctionTok{st\_read\_parquet}\NormalTok{(}\StringTok{"./Templates/TemplateGrids/tikls1km\_sauzeme.parquet"}\NormalTok{)}
\NormalTok{grid\_1km}\OtherTok{=}\FunctionTok{st\_transform}\NormalTok{(grid\_1km,}\AttributeTok{crs=}\DecValTok{3059}\NormalTok{)}
\NormalTok{level12}\OtherTok{=}\FunctionTok{st\_transform}\NormalTok{(level12,}\AttributeTok{crs=}\DecValTok{3059}\NormalTok{)}
\NormalTok{level12}\OtherTok{=}\NormalTok{level12[grid\_1km,,]}

\NormalTok{level12}\OtherTok{=}\FunctionTok{st\_make\_valid}\NormalTok{(level12)}

\CommentTok{\# job {-}{-}{-}{-}}

\NormalTok{localname}\OtherTok{=}\StringTok{"HydroClim\_15{-}max\_cell.tif"}
\NormalTok{layername}\OtherTok{=}\StringTok{"egv\_084"}
\NormalTok{summary\_function}\OtherTok{=}\StringTok{"max"}
 
\NormalTok{slanis}\OtherTok{=}\FunctionTok{rast}\NormalTok{(}\FunctionTok{paste0}\NormalTok{(}\StringTok{"./Geodata/2024/HydroClim/"}\NormalTok{,localname))}
\NormalTok{level12}\SpecialCharTok{$}\NormalTok{Hydro\_values}\OtherTok{=}\FunctionTok{exact\_extract}\NormalTok{(slanis,level12,}\AttributeTok{fun=}\NormalTok{summary\_function)}
 
\FunctionTok{polygon2input}\NormalTok{(}\AttributeTok{vector\_data =}\NormalTok{ level12,}
       \AttributeTok{template\_path =} \StringTok{"./Templates/TemplateRasters/LV10m\_10km.tif"}\NormalTok{,}
       \AttributeTok{out\_path =} \StringTok{"./RasterGrids\_10m/2024/"}\NormalTok{,}
       \AttributeTok{file\_name =}\NormalTok{ localname,}
       \AttributeTok{value\_field =} \StringTok{"Hydro\_values"}\NormalTok{,}
       \AttributeTok{fun=}\StringTok{"first"}\NormalTok{,}
       \AttributeTok{value\_type =} \StringTok{"continuous"}\NormalTok{,}
       \AttributeTok{prepare=}\ConstantTok{FALSE}\NormalTok{,}
       \AttributeTok{project\_mode =} \StringTok{"auto"}\NormalTok{,}
       \AttributeTok{check\_na =} \ConstantTok{FALSE}\NormalTok{,}
       \AttributeTok{plot\_result=}\ConstantTok{FALSE}\NormalTok{,}
       \AttributeTok{plot\_gaps =} \ConstantTok{FALSE}\NormalTok{,}
       \AttributeTok{overwrite=}\ConstantTok{TRUE}\NormalTok{)}
 
\NormalTok{egvrez}\OtherTok{=}\FunctionTok{input2egv}\NormalTok{(}\AttributeTok{input=}\FunctionTok{paste0}\NormalTok{(}\StringTok{"./RasterGrids\_10m/2024/"}\NormalTok{,localname),}
         \AttributeTok{egv\_template=} \StringTok{"./Templates/TemplateRasters/LV100m\_10km.tif"}\NormalTok{,}
         \AttributeTok{summary\_function =} \StringTok{"average"}\NormalTok{,}
         \AttributeTok{missing\_job =} \StringTok{"FillOutput"}\NormalTok{,}
         \AttributeTok{input\_template =} \StringTok{"./Templates/TemplateRasters/LV10m\_10km.tif"}\NormalTok{,}
         \AttributeTok{outlocation =} \StringTok{"./RasterGrids\_100m/2024/RAW/"}\NormalTok{,}
         \AttributeTok{outfilename =}\NormalTok{ localname,}
         \AttributeTok{layername =}\NormalTok{ layername,}
         \AttributeTok{idw\_weight =} \DecValTok{2}\NormalTok{,}
         \AttributeTok{plot\_gaps =} \ConstantTok{FALSE}\NormalTok{,}\AttributeTok{plot\_final =} \ConstantTok{FALSE}\NormalTok{)}
\NormalTok{egvrez}
 
\FunctionTok{unlink}\NormalTok{(}\FunctionTok{paste0}\NormalTok{(}\StringTok{"./RasterGrids\_10m/2024/"}\NormalTok{,localname))}

\CommentTok{\# standardisation {-}{-}{-}{-}}
\ControlFlowTok{if}\NormalTok{(}\SpecialCharTok{!}\FunctionTok{require}\NormalTok{(terra)) \{}\FunctionTok{install.packages}\NormalTok{(}\StringTok{"terra"}\NormalTok{); }\FunctionTok{require}\NormalTok{(terra)\}}
\ControlFlowTok{if}\NormalTok{(}\SpecialCharTok{!}\FunctionTok{require}\NormalTok{(tidyverse)) \{}\FunctionTok{install.packages}\NormalTok{(}\StringTok{"tidyverse"}\NormalTok{); }\FunctionTok{require}\NormalTok{(tidyverse)\}}

\NormalTok{nosaukums}\OtherTok{=}\StringTok{"HydroClim\_15{-}max\_cell.tif"}
\NormalTok{ielasisanas\_cels}\OtherTok{=}\FunctionTok{paste0}\NormalTok{(}\StringTok{"./RasterGrids\_100m/2024/RAW/"}\NormalTok{,nosaukums)}
\NormalTok{saglabasanas\_cels}\OtherTok{=}\FunctionTok{paste0}\NormalTok{(}\StringTok{"./RasterGrids\_100m/2024/Scaled/"}\NormalTok{,nosaukums)}
\NormalTok{slanis}\OtherTok{=}\FunctionTok{rast}\NormalTok{(ielasisanas\_cels)}
\NormalTok{videjais}\OtherTok{=}\FunctionTok{global}\NormalTok{(slanis,}\AttributeTok{fun=}\StringTok{"mean"}\NormalTok{,}\AttributeTok{na.rm=}\ConstantTok{TRUE}\NormalTok{)}
\NormalTok{centrets}\OtherTok{=}\NormalTok{slanis}\SpecialCharTok{{-}}\NormalTok{videjais[,}\DecValTok{1}\NormalTok{]}
\NormalTok{standartnovirze}\OtherTok{=}\NormalTok{terra}\SpecialCharTok{::}\FunctionTok{global}\NormalTok{(centrets,}\AttributeTok{fun=}\StringTok{"rms"}\NormalTok{,}\AttributeTok{na.rm=}\ConstantTok{TRUE}\NormalTok{)}
\NormalTok{merogots}\OtherTok{=}\NormalTok{centrets}\SpecialCharTok{/}\NormalTok{standartnovirze[,}\DecValTok{1}\NormalTok{]}
\FunctionTok{writeRaster}\NormalTok{(merogots,}
      \AttributeTok{filename=}\NormalTok{saglabasanas\_cels,}
      \AttributeTok{overwrite=}\ConstantTok{TRUE}\NormalTok{)}
\end{Highlighting}
\end{Shaded}

\section{HydroClim\_16-max\_cell}\label{ch06.085}

\textbf{filename:} \texttt{HydroClim\_16-max\_cell.tif}

\textbf{layername:} \texttt{egv\_085}

\textbf{English name:} Maximum per subcatchment upstream mean monthly precipitation
amount (kg m⁻² year⁻¹) of the wettest quarter (HydroClim) within the analysis
cell (1 ha)

\textbf{Latvian name:} Sateces apakšbaseina maksimālais augšteces mēneša vidējais nokrišņu daudzums
mitrākajā ceturksnī (kg m⁻² year⁻¹) (HydroClim) analīzes šūnā (1 ha)

\textbf{Procedure:} Information from the \hyperref[Ch04.12]{HydroClim
data} - including both basin and raster layers - is used. First, basin CRS is transformed to EPSG:3059. Then,
zonal statistics (per basin) using a layer specific summary function (max) are
calculated (\texttt{exactextractr::exact\_extract()}), and the the results are rasterised with the workflow
\texttt{egvtools::polygon2input()}. Once rasterised to input data, EGV is created using the workflow
\texttt{egvtools::input2egv()}. To prevent from gaps at the edges, inverse distance
weighted (power = 2) gap filling is implemented. To save disk space,
the intermediate input layer is unlinked. Finally, the layer is standardised by
subtracting the arithmetic mean and dividing by the root mean squared error.

\begin{Shaded}
\begin{Highlighting}[]
\CommentTok{\# libs {-}{-}{-}{-}}
\ControlFlowTok{if}\NormalTok{(}\SpecialCharTok{!}\FunctionTok{require}\NormalTok{(egvtools)) \{remotes}\SpecialCharTok{::}\FunctionTok{install\_github}\NormalTok{(}\StringTok{"aavotins/egvtools"}\NormalTok{); }\FunctionTok{require}\NormalTok{(egvtools)\}}
\ControlFlowTok{if}\NormalTok{(}\SpecialCharTok{!}\FunctionTok{require}\NormalTok{(terra)) \{}\FunctionTok{install.packages}\NormalTok{(}\StringTok{"terra"}\NormalTok{); }\FunctionTok{require}\NormalTok{(terra)\}}
\ControlFlowTok{if}\NormalTok{(}\SpecialCharTok{!}\FunctionTok{require}\NormalTok{(tidyverse)) \{}\FunctionTok{install.packages}\NormalTok{(}\StringTok{"tidyverse"}\NormalTok{); }\FunctionTok{require}\NormalTok{(tidyverse)\}}
\ControlFlowTok{if}\NormalTok{(}\SpecialCharTok{!}\FunctionTok{require}\NormalTok{(sf)) \{}\FunctionTok{install.packages}\NormalTok{(}\StringTok{"sf"}\NormalTok{); }\FunctionTok{require}\NormalTok{(sf)\}}
\ControlFlowTok{if}\NormalTok{(}\SpecialCharTok{!}\FunctionTok{require}\NormalTok{(sfarrow)) \{}\FunctionTok{install.packages}\NormalTok{(}\StringTok{"sfarrow"}\NormalTok{); }\FunctionTok{require}\NormalTok{(sfarrow)\}}
\ControlFlowTok{if}\NormalTok{(}\SpecialCharTok{!}\FunctionTok{require}\NormalTok{(exactextractr)) \{}\FunctionTok{install.packages}\NormalTok{(}\StringTok{"exactextractr"}\NormalTok{); }\FunctionTok{require}\NormalTok{(exactextractr)\}}

\CommentTok{\# basins {-}{-}{-}{-}}
\NormalTok{level12}\OtherTok{=}\FunctionTok{st\_read}\NormalTok{(}\StringTok{"./Geodata/2024/HydroClim/hybas\_lake\_eu\_lev01{-}12\_v1c/hybas\_lake\_eu\_lev12\_v1c.shp"}\NormalTok{)}
\NormalTok{grid\_1km}\OtherTok{=}\NormalTok{sfarrow}\SpecialCharTok{::}\FunctionTok{st\_read\_parquet}\NormalTok{(}\StringTok{"./Templates/TemplateGrids/tikls1km\_sauzeme.parquet"}\NormalTok{)}
\NormalTok{grid\_1km}\OtherTok{=}\FunctionTok{st\_transform}\NormalTok{(grid\_1km,}\AttributeTok{crs=}\DecValTok{3059}\NormalTok{)}
\NormalTok{level12}\OtherTok{=}\FunctionTok{st\_transform}\NormalTok{(level12,}\AttributeTok{crs=}\DecValTok{3059}\NormalTok{)}
\NormalTok{level12}\OtherTok{=}\NormalTok{level12[grid\_1km,,]}

\NormalTok{level12}\OtherTok{=}\FunctionTok{st\_make\_valid}\NormalTok{(level12)}

\CommentTok{\# job {-}{-}{-}{-}}

\NormalTok{localname}\OtherTok{=}\StringTok{"HydroClim\_16{-}max\_cell.tif"}
\NormalTok{layername}\OtherTok{=}\StringTok{"egv\_085"}
\NormalTok{summary\_function}\OtherTok{=}\StringTok{"max"}
 
\NormalTok{slanis}\OtherTok{=}\FunctionTok{rast}\NormalTok{(}\FunctionTok{paste0}\NormalTok{(}\StringTok{"./Geodata/2024/HydroClim/"}\NormalTok{,localname))}
\NormalTok{level12}\SpecialCharTok{$}\NormalTok{Hydro\_values}\OtherTok{=}\FunctionTok{exact\_extract}\NormalTok{(slanis,level12,}\AttributeTok{fun=}\NormalTok{summary\_function)}
 
\FunctionTok{polygon2input}\NormalTok{(}\AttributeTok{vector\_data =}\NormalTok{ level12,}
       \AttributeTok{template\_path =} \StringTok{"./Templates/TemplateRasters/LV10m\_10km.tif"}\NormalTok{,}
       \AttributeTok{out\_path =} \StringTok{"./RasterGrids\_10m/2024/"}\NormalTok{,}
       \AttributeTok{file\_name =}\NormalTok{ localname,}
       \AttributeTok{value\_field =} \StringTok{"Hydro\_values"}\NormalTok{,}
       \AttributeTok{fun=}\StringTok{"first"}\NormalTok{,}
       \AttributeTok{value\_type =} \StringTok{"continuous"}\NormalTok{,}
       \AttributeTok{prepare=}\ConstantTok{FALSE}\NormalTok{,}
       \AttributeTok{project\_mode =} \StringTok{"auto"}\NormalTok{,}
       \AttributeTok{check\_na =} \ConstantTok{FALSE}\NormalTok{,}
       \AttributeTok{plot\_result=}\ConstantTok{FALSE}\NormalTok{,}
       \AttributeTok{plot\_gaps =} \ConstantTok{FALSE}\NormalTok{,}
       \AttributeTok{overwrite=}\ConstantTok{TRUE}\NormalTok{)}
 
\NormalTok{egvrez}\OtherTok{=}\FunctionTok{input2egv}\NormalTok{(}\AttributeTok{input=}\FunctionTok{paste0}\NormalTok{(}\StringTok{"./RasterGrids\_10m/2024/"}\NormalTok{,localname),}
         \AttributeTok{egv\_template=} \StringTok{"./Templates/TemplateRasters/LV100m\_10km.tif"}\NormalTok{,}
         \AttributeTok{summary\_function =} \StringTok{"average"}\NormalTok{,}
         \AttributeTok{missing\_job =} \StringTok{"FillOutput"}\NormalTok{,}
         \AttributeTok{input\_template =} \StringTok{"./Templates/TemplateRasters/LV10m\_10km.tif"}\NormalTok{,}
         \AttributeTok{outlocation =} \StringTok{"./RasterGrids\_100m/2024/RAW/"}\NormalTok{,}
         \AttributeTok{outfilename =}\NormalTok{ localname,}
         \AttributeTok{layername =}\NormalTok{ layername,}
         \AttributeTok{idw\_weight =} \DecValTok{2}\NormalTok{,}
         \AttributeTok{plot\_gaps =} \ConstantTok{FALSE}\NormalTok{,}\AttributeTok{plot\_final =} \ConstantTok{FALSE}\NormalTok{)}
\NormalTok{egvrez}
 
\FunctionTok{unlink}\NormalTok{(}\FunctionTok{paste0}\NormalTok{(}\StringTok{"./RasterGrids\_10m/2024/"}\NormalTok{,localname))}

\CommentTok{\# standardisation {-}{-}{-}{-}}
\ControlFlowTok{if}\NormalTok{(}\SpecialCharTok{!}\FunctionTok{require}\NormalTok{(terra)) \{}\FunctionTok{install.packages}\NormalTok{(}\StringTok{"terra"}\NormalTok{); }\FunctionTok{require}\NormalTok{(terra)\}}
\ControlFlowTok{if}\NormalTok{(}\SpecialCharTok{!}\FunctionTok{require}\NormalTok{(tidyverse)) \{}\FunctionTok{install.packages}\NormalTok{(}\StringTok{"tidyverse"}\NormalTok{); }\FunctionTok{require}\NormalTok{(tidyverse)\}}

\NormalTok{nosaukums}\OtherTok{=}\StringTok{"HydroClim\_16{-}max\_cell.tif"}
\NormalTok{ielasisanas\_cels}\OtherTok{=}\FunctionTok{paste0}\NormalTok{(}\StringTok{"./RasterGrids\_100m/2024/RAW/"}\NormalTok{,nosaukums)}
\NormalTok{saglabasanas\_cels}\OtherTok{=}\FunctionTok{paste0}\NormalTok{(}\StringTok{"./RasterGrids\_100m/2024/Scaled/"}\NormalTok{,nosaukums)}
\NormalTok{slanis}\OtherTok{=}\FunctionTok{rast}\NormalTok{(ielasisanas\_cels)}
\NormalTok{videjais}\OtherTok{=}\FunctionTok{global}\NormalTok{(slanis,}\AttributeTok{fun=}\StringTok{"mean"}\NormalTok{,}\AttributeTok{na.rm=}\ConstantTok{TRUE}\NormalTok{)}
\NormalTok{centrets}\OtherTok{=}\NormalTok{slanis}\SpecialCharTok{{-}}\NormalTok{videjais[,}\DecValTok{1}\NormalTok{]}
\NormalTok{standartnovirze}\OtherTok{=}\NormalTok{terra}\SpecialCharTok{::}\FunctionTok{global}\NormalTok{(centrets,}\AttributeTok{fun=}\StringTok{"rms"}\NormalTok{,}\AttributeTok{na.rm=}\ConstantTok{TRUE}\NormalTok{)}
\NormalTok{merogots}\OtherTok{=}\NormalTok{centrets}\SpecialCharTok{/}\NormalTok{standartnovirze[,}\DecValTok{1}\NormalTok{]}
\FunctionTok{writeRaster}\NormalTok{(merogots,}
      \AttributeTok{filename=}\NormalTok{saglabasanas\_cels,}
      \AttributeTok{overwrite=}\ConstantTok{TRUE}\NormalTok{)}
\end{Highlighting}
\end{Shaded}

\section{HydroClim\_17-max\_cell}\label{ch06.086}

\textbf{filename:} \texttt{HydroClim\_17-max\_cell.tif}

\textbf{layername:} \texttt{egv\_086}

\textbf{English name:} Maximum per subcatchment upstream mean monthly precipitation
amount (kg m⁻² year⁻¹) of the driest quarter (HydroClim) within the analysis
cell (1 ha)

\textbf{Latvian name:} Sateces apakšbaseina maksimālais augšteces mēneša vidējais nokrišņu daudzums
sausākajā ceturksnī (kg m⁻² year⁻¹) (HydroClim) analīzes šūnā (1 ha)

\textbf{Procedure:} Information from the \hyperref[Ch04.12]{HydroClim
data} - including both basin and raster layers - is used. First, basin CRS is transformed to EPSG:3059. Then,
zonal statistics (per basin) using a layer specific summary function (max) are
calculated (\texttt{exactextractr::exact\_extract()}), and the the results are rasterised with the workflow
\texttt{egvtools::polygon2input()}. Once rasterised to input data, EGV is created using the workflow
\texttt{egvtools::input2egv()}. To prevent from gaps at the edges, inverse distance
weighted (power = 2) gap filling is implemented. To save disk space,
the intermediate input layer is unlinked. Finally, the layer is standardised by
subtracting the arithmetic mean and dividing by the root mean squared error.

\begin{Shaded}
\begin{Highlighting}[]
\CommentTok{\# libs {-}{-}{-}{-}}
\ControlFlowTok{if}\NormalTok{(}\SpecialCharTok{!}\FunctionTok{require}\NormalTok{(egvtools)) \{remotes}\SpecialCharTok{::}\FunctionTok{install\_github}\NormalTok{(}\StringTok{"aavotins/egvtools"}\NormalTok{); }\FunctionTok{require}\NormalTok{(egvtools)\}}
\ControlFlowTok{if}\NormalTok{(}\SpecialCharTok{!}\FunctionTok{require}\NormalTok{(terra)) \{}\FunctionTok{install.packages}\NormalTok{(}\StringTok{"terra"}\NormalTok{); }\FunctionTok{require}\NormalTok{(terra)\}}
\ControlFlowTok{if}\NormalTok{(}\SpecialCharTok{!}\FunctionTok{require}\NormalTok{(tidyverse)) \{}\FunctionTok{install.packages}\NormalTok{(}\StringTok{"tidyverse"}\NormalTok{); }\FunctionTok{require}\NormalTok{(tidyverse)\}}
\ControlFlowTok{if}\NormalTok{(}\SpecialCharTok{!}\FunctionTok{require}\NormalTok{(sf)) \{}\FunctionTok{install.packages}\NormalTok{(}\StringTok{"sf"}\NormalTok{); }\FunctionTok{require}\NormalTok{(sf)\}}
\ControlFlowTok{if}\NormalTok{(}\SpecialCharTok{!}\FunctionTok{require}\NormalTok{(sfarrow)) \{}\FunctionTok{install.packages}\NormalTok{(}\StringTok{"sfarrow"}\NormalTok{); }\FunctionTok{require}\NormalTok{(sfarrow)\}}
\ControlFlowTok{if}\NormalTok{(}\SpecialCharTok{!}\FunctionTok{require}\NormalTok{(exactextractr)) \{}\FunctionTok{install.packages}\NormalTok{(}\StringTok{"exactextractr"}\NormalTok{); }\FunctionTok{require}\NormalTok{(exactextractr)\}}

\CommentTok{\# basins {-}{-}{-}{-}}
\NormalTok{level12}\OtherTok{=}\FunctionTok{st\_read}\NormalTok{(}\StringTok{"./Geodata/2024/HydroClim/hybas\_lake\_eu\_lev01{-}12\_v1c/hybas\_lake\_eu\_lev12\_v1c.shp"}\NormalTok{)}
\NormalTok{grid\_1km}\OtherTok{=}\NormalTok{sfarrow}\SpecialCharTok{::}\FunctionTok{st\_read\_parquet}\NormalTok{(}\StringTok{"./Templates/TemplateGrids/tikls1km\_sauzeme.parquet"}\NormalTok{)}
\NormalTok{grid\_1km}\OtherTok{=}\FunctionTok{st\_transform}\NormalTok{(grid\_1km,}\AttributeTok{crs=}\DecValTok{3059}\NormalTok{)}
\NormalTok{level12}\OtherTok{=}\FunctionTok{st\_transform}\NormalTok{(level12,}\AttributeTok{crs=}\DecValTok{3059}\NormalTok{)}
\NormalTok{level12}\OtherTok{=}\NormalTok{level12[grid\_1km,,]}

\NormalTok{level12}\OtherTok{=}\FunctionTok{st\_make\_valid}\NormalTok{(level12)}

\CommentTok{\# job {-}{-}{-}{-}}

\NormalTok{localname}\OtherTok{=}\StringTok{"HydroClim\_17{-}max\_cell.tif"}
\NormalTok{layername}\OtherTok{=}\StringTok{"egv\_086"}
\NormalTok{summary\_function}\OtherTok{=}\StringTok{"max"}
 
\NormalTok{slanis}\OtherTok{=}\FunctionTok{rast}\NormalTok{(}\FunctionTok{paste0}\NormalTok{(}\StringTok{"./Geodata/2024/HydroClim/"}\NormalTok{,localname))}
\NormalTok{level12}\SpecialCharTok{$}\NormalTok{Hydro\_values}\OtherTok{=}\FunctionTok{exact\_extract}\NormalTok{(slanis,level12,}\AttributeTok{fun=}\NormalTok{summary\_function)}
 
\FunctionTok{polygon2input}\NormalTok{(}\AttributeTok{vector\_data =}\NormalTok{ level12,}
       \AttributeTok{template\_path =} \StringTok{"./Templates/TemplateRasters/LV10m\_10km.tif"}\NormalTok{,}
       \AttributeTok{out\_path =} \StringTok{"./RasterGrids\_10m/2024/"}\NormalTok{,}
       \AttributeTok{file\_name =}\NormalTok{ localname,}
       \AttributeTok{value\_field =} \StringTok{"Hydro\_values"}\NormalTok{,}
       \AttributeTok{fun=}\StringTok{"first"}\NormalTok{,}
       \AttributeTok{value\_type =} \StringTok{"continuous"}\NormalTok{,}
       \AttributeTok{prepare=}\ConstantTok{FALSE}\NormalTok{,}
       \AttributeTok{project\_mode =} \StringTok{"auto"}\NormalTok{,}
       \AttributeTok{check\_na =} \ConstantTok{FALSE}\NormalTok{,}
       \AttributeTok{plot\_result=}\ConstantTok{FALSE}\NormalTok{,}
       \AttributeTok{plot\_gaps =} \ConstantTok{FALSE}\NormalTok{,}
       \AttributeTok{overwrite=}\ConstantTok{TRUE}\NormalTok{)}
 
\NormalTok{egvrez}\OtherTok{=}\FunctionTok{input2egv}\NormalTok{(}\AttributeTok{input=}\FunctionTok{paste0}\NormalTok{(}\StringTok{"./RasterGrids\_10m/2024/"}\NormalTok{,localname),}
         \AttributeTok{egv\_template=} \StringTok{"./Templates/TemplateRasters/LV100m\_10km.tif"}\NormalTok{,}
         \AttributeTok{summary\_function =} \StringTok{"average"}\NormalTok{,}
         \AttributeTok{missing\_job =} \StringTok{"FillOutput"}\NormalTok{,}
         \AttributeTok{input\_template =} \StringTok{"./Templates/TemplateRasters/LV10m\_10km.tif"}\NormalTok{,}
         \AttributeTok{outlocation =} \StringTok{"./RasterGrids\_100m/2024/RAW/"}\NormalTok{,}
         \AttributeTok{outfilename =}\NormalTok{ localname,}
         \AttributeTok{layername =}\NormalTok{ layername,}
         \AttributeTok{idw\_weight =} \DecValTok{2}\NormalTok{,}
         \AttributeTok{plot\_gaps =} \ConstantTok{FALSE}\NormalTok{,}\AttributeTok{plot\_final =} \ConstantTok{FALSE}\NormalTok{)}
\NormalTok{egvrez}
 
\FunctionTok{unlink}\NormalTok{(}\FunctionTok{paste0}\NormalTok{(}\StringTok{"./RasterGrids\_10m/2024/"}\NormalTok{,localname))}

\CommentTok{\# standardisation {-}{-}{-}{-}}
\ControlFlowTok{if}\NormalTok{(}\SpecialCharTok{!}\FunctionTok{require}\NormalTok{(terra)) \{}\FunctionTok{install.packages}\NormalTok{(}\StringTok{"terra"}\NormalTok{); }\FunctionTok{require}\NormalTok{(terra)\}}
\ControlFlowTok{if}\NormalTok{(}\SpecialCharTok{!}\FunctionTok{require}\NormalTok{(tidyverse)) \{}\FunctionTok{install.packages}\NormalTok{(}\StringTok{"tidyverse"}\NormalTok{); }\FunctionTok{require}\NormalTok{(tidyverse)\}}

\NormalTok{nosaukums}\OtherTok{=}\StringTok{"HydroClim\_17{-}max\_cell.tif"}
\NormalTok{ielasisanas\_cels}\OtherTok{=}\FunctionTok{paste0}\NormalTok{(}\StringTok{"./RasterGrids\_100m/2024/RAW/"}\NormalTok{,nosaukums)}
\NormalTok{saglabasanas\_cels}\OtherTok{=}\FunctionTok{paste0}\NormalTok{(}\StringTok{"./RasterGrids\_100m/2024/Scaled/"}\NormalTok{,nosaukums)}
\NormalTok{slanis}\OtherTok{=}\FunctionTok{rast}\NormalTok{(ielasisanas\_cels)}
\NormalTok{videjais}\OtherTok{=}\FunctionTok{global}\NormalTok{(slanis,}\AttributeTok{fun=}\StringTok{"mean"}\NormalTok{,}\AttributeTok{na.rm=}\ConstantTok{TRUE}\NormalTok{)}
\NormalTok{centrets}\OtherTok{=}\NormalTok{slanis}\SpecialCharTok{{-}}\NormalTok{videjais[,}\DecValTok{1}\NormalTok{]}
\NormalTok{standartnovirze}\OtherTok{=}\NormalTok{terra}\SpecialCharTok{::}\FunctionTok{global}\NormalTok{(centrets,}\AttributeTok{fun=}\StringTok{"rms"}\NormalTok{,}\AttributeTok{na.rm=}\ConstantTok{TRUE}\NormalTok{)}
\NormalTok{merogots}\OtherTok{=}\NormalTok{centrets}\SpecialCharTok{/}\NormalTok{standartnovirze[,}\DecValTok{1}\NormalTok{]}
\FunctionTok{writeRaster}\NormalTok{(merogots,}
      \AttributeTok{filename=}\NormalTok{saglabasanas\_cels,}
      \AttributeTok{overwrite=}\ConstantTok{TRUE}\NormalTok{)}
\end{Highlighting}
\end{Shaded}

\section{HydroClim\_18-max\_cell}\label{ch06.087}

\textbf{filename:} \texttt{HydroClim\_18-max\_cell.tif}

\textbf{layername:} \texttt{egv\_087}

\textbf{English name:} Maximum per subcatchment upstream mean monthly precipitation
amount (kg m⁻² year⁻¹) of the warmest quarter (HydroClim) within the analysis
cell (1 ha)

\textbf{Latvian name:} Sateces apakšbaseina maksimālais augšteces mēneša vidējais nokrišņu daudzums
siltākajā ceturksnī (kg m⁻² year⁻¹) (HydroClim) analīzes šūnā (1 ha)

\textbf{Procedure:} Information from the \hyperref[Ch04.12]{HydroClim
data} - including both basin and raster layers - is used. First, basin CRS is transformed to EPSG:3059. Then,
zonal statistics (per basin) using a layer specific summary function (max) are
calculated (\texttt{exactextractr::exact\_extract()}), and the the results are rasterised with the workflow
\texttt{egvtools::polygon2input()}. Once rasterised to input data, EGV is created using the workflow
\texttt{egvtools::input2egv()}. To prevent from gaps at the edges, inverse distance
weighted (power = 2) gap filling is implemented. To save disk space,
the intermediate input layer is unlinked. Finally, the layer is standardised by
subtracting the arithmetic mean and dividing by the root mean squared error.

\begin{Shaded}
\begin{Highlighting}[]
\CommentTok{\# libs {-}{-}{-}{-}}
\ControlFlowTok{if}\NormalTok{(}\SpecialCharTok{!}\FunctionTok{require}\NormalTok{(egvtools)) \{remotes}\SpecialCharTok{::}\FunctionTok{install\_github}\NormalTok{(}\StringTok{"aavotins/egvtools"}\NormalTok{); }\FunctionTok{require}\NormalTok{(egvtools)\}}
\ControlFlowTok{if}\NormalTok{(}\SpecialCharTok{!}\FunctionTok{require}\NormalTok{(terra)) \{}\FunctionTok{install.packages}\NormalTok{(}\StringTok{"terra"}\NormalTok{); }\FunctionTok{require}\NormalTok{(terra)\}}
\ControlFlowTok{if}\NormalTok{(}\SpecialCharTok{!}\FunctionTok{require}\NormalTok{(tidyverse)) \{}\FunctionTok{install.packages}\NormalTok{(}\StringTok{"tidyverse"}\NormalTok{); }\FunctionTok{require}\NormalTok{(tidyverse)\}}
\ControlFlowTok{if}\NormalTok{(}\SpecialCharTok{!}\FunctionTok{require}\NormalTok{(sf)) \{}\FunctionTok{install.packages}\NormalTok{(}\StringTok{"sf"}\NormalTok{); }\FunctionTok{require}\NormalTok{(sf)\}}
\ControlFlowTok{if}\NormalTok{(}\SpecialCharTok{!}\FunctionTok{require}\NormalTok{(sfarrow)) \{}\FunctionTok{install.packages}\NormalTok{(}\StringTok{"sfarrow"}\NormalTok{); }\FunctionTok{require}\NormalTok{(sfarrow)\}}
\ControlFlowTok{if}\NormalTok{(}\SpecialCharTok{!}\FunctionTok{require}\NormalTok{(exactextractr)) \{}\FunctionTok{install.packages}\NormalTok{(}\StringTok{"exactextractr"}\NormalTok{); }\FunctionTok{require}\NormalTok{(exactextractr)\}}

\CommentTok{\# basins {-}{-}{-}{-}}
\NormalTok{level12}\OtherTok{=}\FunctionTok{st\_read}\NormalTok{(}\StringTok{"./Geodata/2024/HydroClim/hybas\_lake\_eu\_lev01{-}12\_v1c/hybas\_lake\_eu\_lev12\_v1c.shp"}\NormalTok{)}
\NormalTok{grid\_1km}\OtherTok{=}\NormalTok{sfarrow}\SpecialCharTok{::}\FunctionTok{st\_read\_parquet}\NormalTok{(}\StringTok{"./Templates/TemplateGrids/tikls1km\_sauzeme.parquet"}\NormalTok{)}
\NormalTok{grid\_1km}\OtherTok{=}\FunctionTok{st\_transform}\NormalTok{(grid\_1km,}\AttributeTok{crs=}\DecValTok{3059}\NormalTok{)}
\NormalTok{level12}\OtherTok{=}\FunctionTok{st\_transform}\NormalTok{(level12,}\AttributeTok{crs=}\DecValTok{3059}\NormalTok{)}
\NormalTok{level12}\OtherTok{=}\NormalTok{level12[grid\_1km,,]}

\NormalTok{level12}\OtherTok{=}\FunctionTok{st\_make\_valid}\NormalTok{(level12)}

\CommentTok{\# job {-}{-}{-}{-}}

\NormalTok{localname}\OtherTok{=}\StringTok{"HydroClim\_18{-}max\_cell.tif"}
\NormalTok{layername}\OtherTok{=}\StringTok{"egv\_087"}
\NormalTok{summary\_function}\OtherTok{=}\StringTok{"max"}
 
\NormalTok{slanis}\OtherTok{=}\FunctionTok{rast}\NormalTok{(}\FunctionTok{paste0}\NormalTok{(}\StringTok{"./Geodata/2024/HydroClim/"}\NormalTok{,localname))}
\NormalTok{level12}\SpecialCharTok{$}\NormalTok{Hydro\_values}\OtherTok{=}\FunctionTok{exact\_extract}\NormalTok{(slanis,level12,}\AttributeTok{fun=}\NormalTok{summary\_function)}
 
\FunctionTok{polygon2input}\NormalTok{(}\AttributeTok{vector\_data =}\NormalTok{ level12,}
       \AttributeTok{template\_path =} \StringTok{"./Templates/TemplateRasters/LV10m\_10km.tif"}\NormalTok{,}
       \AttributeTok{out\_path =} \StringTok{"./RasterGrids\_10m/2024/"}\NormalTok{,}
       \AttributeTok{file\_name =}\NormalTok{ localname,}
       \AttributeTok{value\_field =} \StringTok{"Hydro\_values"}\NormalTok{,}
       \AttributeTok{fun=}\StringTok{"first"}\NormalTok{,}
       \AttributeTok{value\_type =} \StringTok{"continuous"}\NormalTok{,}
       \AttributeTok{prepare=}\ConstantTok{FALSE}\NormalTok{,}
       \AttributeTok{project\_mode =} \StringTok{"auto"}\NormalTok{,}
       \AttributeTok{check\_na =} \ConstantTok{FALSE}\NormalTok{,}
       \AttributeTok{plot\_result=}\ConstantTok{FALSE}\NormalTok{,}
       \AttributeTok{plot\_gaps =} \ConstantTok{FALSE}\NormalTok{,}
       \AttributeTok{overwrite=}\ConstantTok{TRUE}\NormalTok{)}
 
\NormalTok{egvrez}\OtherTok{=}\FunctionTok{input2egv}\NormalTok{(}\AttributeTok{input=}\FunctionTok{paste0}\NormalTok{(}\StringTok{"./RasterGrids\_10m/2024/"}\NormalTok{,localname),}
         \AttributeTok{egv\_template=} \StringTok{"./Templates/TemplateRasters/LV100m\_10km.tif"}\NormalTok{,}
         \AttributeTok{summary\_function =} \StringTok{"average"}\NormalTok{,}
         \AttributeTok{missing\_job =} \StringTok{"FillOutput"}\NormalTok{,}
         \AttributeTok{input\_template =} \StringTok{"./Templates/TemplateRasters/LV10m\_10km.tif"}\NormalTok{,}
         \AttributeTok{outlocation =} \StringTok{"./RasterGrids\_100m/2024/RAW/"}\NormalTok{,}
         \AttributeTok{outfilename =}\NormalTok{ localname,}
         \AttributeTok{layername =}\NormalTok{ layername,}
         \AttributeTok{idw\_weight =} \DecValTok{2}\NormalTok{,}
         \AttributeTok{plot\_gaps =} \ConstantTok{FALSE}\NormalTok{,}\AttributeTok{plot\_final =} \ConstantTok{FALSE}\NormalTok{)}
\NormalTok{egvrez}
 
\FunctionTok{unlink}\NormalTok{(}\FunctionTok{paste0}\NormalTok{(}\StringTok{"./RasterGrids\_10m/2024/"}\NormalTok{,localname))}

\CommentTok{\# standardisation {-}{-}{-}{-}}
\ControlFlowTok{if}\NormalTok{(}\SpecialCharTok{!}\FunctionTok{require}\NormalTok{(terra)) \{}\FunctionTok{install.packages}\NormalTok{(}\StringTok{"terra"}\NormalTok{); }\FunctionTok{require}\NormalTok{(terra)\}}
\ControlFlowTok{if}\NormalTok{(}\SpecialCharTok{!}\FunctionTok{require}\NormalTok{(tidyverse)) \{}\FunctionTok{install.packages}\NormalTok{(}\StringTok{"tidyverse"}\NormalTok{); }\FunctionTok{require}\NormalTok{(tidyverse)\}}

\NormalTok{nosaukums}\OtherTok{=}\StringTok{"HydroClim\_18{-}max\_cell.tif"}
\NormalTok{ielasisanas\_cels}\OtherTok{=}\FunctionTok{paste0}\NormalTok{(}\StringTok{"./RasterGrids\_100m/2024/RAW/"}\NormalTok{,nosaukums)}
\NormalTok{saglabasanas\_cels}\OtherTok{=}\FunctionTok{paste0}\NormalTok{(}\StringTok{"./RasterGrids\_100m/2024/Scaled/"}\NormalTok{,nosaukums)}
\NormalTok{slanis}\OtherTok{=}\FunctionTok{rast}\NormalTok{(ielasisanas\_cels)}
\NormalTok{videjais}\OtherTok{=}\FunctionTok{global}\NormalTok{(slanis,}\AttributeTok{fun=}\StringTok{"mean"}\NormalTok{,}\AttributeTok{na.rm=}\ConstantTok{TRUE}\NormalTok{)}
\NormalTok{centrets}\OtherTok{=}\NormalTok{slanis}\SpecialCharTok{{-}}\NormalTok{videjais[,}\DecValTok{1}\NormalTok{]}
\NormalTok{standartnovirze}\OtherTok{=}\NormalTok{terra}\SpecialCharTok{::}\FunctionTok{global}\NormalTok{(centrets,}\AttributeTok{fun=}\StringTok{"rms"}\NormalTok{,}\AttributeTok{na.rm=}\ConstantTok{TRUE}\NormalTok{)}
\NormalTok{merogots}\OtherTok{=}\NormalTok{centrets}\SpecialCharTok{/}\NormalTok{standartnovirze[,}\DecValTok{1}\NormalTok{]}
\FunctionTok{writeRaster}\NormalTok{(merogots,}
      \AttributeTok{filename=}\NormalTok{saglabasanas\_cels,}
      \AttributeTok{overwrite=}\ConstantTok{TRUE}\NormalTok{)}
\end{Highlighting}
\end{Shaded}

\section{HydroClim\_19-max\_cell}\label{ch06.088}

\textbf{filename:} \texttt{HydroClim\_19-max\_cell.tif}

\textbf{layername:} \texttt{egv\_088}

\textbf{English name:} Maximum per subcatchment upstream mean monthly precipitation
amount (kg m⁻² year⁻¹) of the coldest quarter (HydroClim) within the analysis
cell (1 ha)

\textbf{Latvian name:} Sateces apakšbaseina maksimālais augšteces mēneša vidējais nokrišņu daudzums
vēsākajā ceturksnī (kg m⁻² year⁻¹) (HydroClim) analīzes šūnā (1 ha)

\textbf{Procedure:} Information from the \hyperref[Ch04.12]{HydroClim
data} - including both basin and raster layers - is used. First, basin CRS is transformed to EPSG:3059. Then,
zonal statistics (per basin) using a layer specific summary function (max) are
calculated (\texttt{exactextractr::exact\_extract()}), and the the results are rasterised with the workflow
\texttt{egvtools::polygon2input()}. Once rasterised to input data, EGV is created using the workflow
\texttt{egvtools::input2egv()}. To prevent from gaps at the edges, inverse distance
weighted (power = 2) gap filling is implemented. To save disk space,
the intermediate input layer is unlinked. Finally, the layer is standardised by
subtracting the arithmetic mean and dividing by the root mean squared error.

\begin{Shaded}
\begin{Highlighting}[]
\CommentTok{\# libs {-}{-}{-}{-}}
\ControlFlowTok{if}\NormalTok{(}\SpecialCharTok{!}\FunctionTok{require}\NormalTok{(egvtools)) \{remotes}\SpecialCharTok{::}\FunctionTok{install\_github}\NormalTok{(}\StringTok{"aavotins/egvtools"}\NormalTok{); }\FunctionTok{require}\NormalTok{(egvtools)\}}
\ControlFlowTok{if}\NormalTok{(}\SpecialCharTok{!}\FunctionTok{require}\NormalTok{(terra)) \{}\FunctionTok{install.packages}\NormalTok{(}\StringTok{"terra"}\NormalTok{); }\FunctionTok{require}\NormalTok{(terra)\}}
\ControlFlowTok{if}\NormalTok{(}\SpecialCharTok{!}\FunctionTok{require}\NormalTok{(tidyverse)) \{}\FunctionTok{install.packages}\NormalTok{(}\StringTok{"tidyverse"}\NormalTok{); }\FunctionTok{require}\NormalTok{(tidyverse)\}}
\ControlFlowTok{if}\NormalTok{(}\SpecialCharTok{!}\FunctionTok{require}\NormalTok{(sf)) \{}\FunctionTok{install.packages}\NormalTok{(}\StringTok{"sf"}\NormalTok{); }\FunctionTok{require}\NormalTok{(sf)\}}
\ControlFlowTok{if}\NormalTok{(}\SpecialCharTok{!}\FunctionTok{require}\NormalTok{(sfarrow)) \{}\FunctionTok{install.packages}\NormalTok{(}\StringTok{"sfarrow"}\NormalTok{); }\FunctionTok{require}\NormalTok{(sfarrow)\}}
\ControlFlowTok{if}\NormalTok{(}\SpecialCharTok{!}\FunctionTok{require}\NormalTok{(exactextractr)) \{}\FunctionTok{install.packages}\NormalTok{(}\StringTok{"exactextractr"}\NormalTok{); }\FunctionTok{require}\NormalTok{(exactextractr)\}}

\CommentTok{\# basins {-}{-}{-}{-}}
\NormalTok{level12}\OtherTok{=}\FunctionTok{st\_read}\NormalTok{(}\StringTok{"./Geodata/2024/HydroClim/hybas\_lake\_eu\_lev01{-}12\_v1c/hybas\_lake\_eu\_lev12\_v1c.shp"}\NormalTok{)}
\NormalTok{grid\_1km}\OtherTok{=}\NormalTok{sfarrow}\SpecialCharTok{::}\FunctionTok{st\_read\_parquet}\NormalTok{(}\StringTok{"./Templates/TemplateGrids/tikls1km\_sauzeme.parquet"}\NormalTok{)}
\NormalTok{grid\_1km}\OtherTok{=}\FunctionTok{st\_transform}\NormalTok{(grid\_1km,}\AttributeTok{crs=}\DecValTok{3059}\NormalTok{)}
\NormalTok{level12}\OtherTok{=}\FunctionTok{st\_transform}\NormalTok{(level12,}\AttributeTok{crs=}\DecValTok{3059}\NormalTok{)}
\NormalTok{level12}\OtherTok{=}\NormalTok{level12[grid\_1km,,]}

\NormalTok{level12}\OtherTok{=}\FunctionTok{st\_make\_valid}\NormalTok{(level12)}

\CommentTok{\# job {-}{-}{-}{-}}

\NormalTok{localname}\OtherTok{=}\StringTok{"HydroClim\_19{-}max\_cell.tif"}
\NormalTok{layername}\OtherTok{=}\StringTok{"egv\_088"}
\NormalTok{summary\_function}\OtherTok{=}\StringTok{"max"}
 
\NormalTok{slanis}\OtherTok{=}\FunctionTok{rast}\NormalTok{(}\FunctionTok{paste0}\NormalTok{(}\StringTok{"./Geodata/2024/HydroClim/"}\NormalTok{,localname))}
\NormalTok{level12}\SpecialCharTok{$}\NormalTok{Hydro\_values}\OtherTok{=}\FunctionTok{exact\_extract}\NormalTok{(slanis,level12,}\AttributeTok{fun=}\NormalTok{summary\_function)}
 
\FunctionTok{polygon2input}\NormalTok{(}\AttributeTok{vector\_data =}\NormalTok{ level12,}
       \AttributeTok{template\_path =} \StringTok{"./Templates/TemplateRasters/LV10m\_10km.tif"}\NormalTok{,}
       \AttributeTok{out\_path =} \StringTok{"./RasterGrids\_10m/2024/"}\NormalTok{,}
       \AttributeTok{file\_name =}\NormalTok{ localname,}
       \AttributeTok{value\_field =} \StringTok{"Hydro\_values"}\NormalTok{,}
       \AttributeTok{fun=}\StringTok{"first"}\NormalTok{,}
       \AttributeTok{value\_type =} \StringTok{"continuous"}\NormalTok{,}
       \AttributeTok{prepare=}\ConstantTok{FALSE}\NormalTok{,}
       \AttributeTok{project\_mode =} \StringTok{"auto"}\NormalTok{,}
       \AttributeTok{check\_na =} \ConstantTok{FALSE}\NormalTok{,}
       \AttributeTok{plot\_result=}\ConstantTok{FALSE}\NormalTok{,}
       \AttributeTok{plot\_gaps =} \ConstantTok{FALSE}\NormalTok{,}
       \AttributeTok{overwrite=}\ConstantTok{TRUE}\NormalTok{)}
 
\NormalTok{egvrez}\OtherTok{=}\FunctionTok{input2egv}\NormalTok{(}\AttributeTok{input=}\FunctionTok{paste0}\NormalTok{(}\StringTok{"./RasterGrids\_10m/2024/"}\NormalTok{,localname),}
         \AttributeTok{egv\_template=} \StringTok{"./Templates/TemplateRasters/LV100m\_10km.tif"}\NormalTok{,}
         \AttributeTok{summary\_function =} \StringTok{"average"}\NormalTok{,}
         \AttributeTok{missing\_job =} \StringTok{"FillOutput"}\NormalTok{,}
         \AttributeTok{input\_template =} \StringTok{"./Templates/TemplateRasters/LV10m\_10km.tif"}\NormalTok{,}
         \AttributeTok{outlocation =} \StringTok{"./RasterGrids\_100m/2024/RAW/"}\NormalTok{,}
         \AttributeTok{outfilename =}\NormalTok{ localname,}
         \AttributeTok{layername =}\NormalTok{ layername,}
         \AttributeTok{idw\_weight =} \DecValTok{2}\NormalTok{,}
         \AttributeTok{plot\_gaps =} \ConstantTok{FALSE}\NormalTok{,}\AttributeTok{plot\_final =} \ConstantTok{FALSE}\NormalTok{)}
\NormalTok{egvrez}
 
\FunctionTok{unlink}\NormalTok{(}\FunctionTok{paste0}\NormalTok{(}\StringTok{"./RasterGrids\_10m/2024/"}\NormalTok{,localname))}

\CommentTok{\# standardisation {-}{-}{-}{-}}
\ControlFlowTok{if}\NormalTok{(}\SpecialCharTok{!}\FunctionTok{require}\NormalTok{(terra)) \{}\FunctionTok{install.packages}\NormalTok{(}\StringTok{"terra"}\NormalTok{); }\FunctionTok{require}\NormalTok{(terra)\}}
\ControlFlowTok{if}\NormalTok{(}\SpecialCharTok{!}\FunctionTok{require}\NormalTok{(tidyverse)) \{}\FunctionTok{install.packages}\NormalTok{(}\StringTok{"tidyverse"}\NormalTok{); }\FunctionTok{require}\NormalTok{(tidyverse)\}}

\NormalTok{nosaukums}\OtherTok{=}\StringTok{"HydroClim\_19{-}max\_cell.tif"}
\NormalTok{ielasisanas\_cels}\OtherTok{=}\FunctionTok{paste0}\NormalTok{(}\StringTok{"./RasterGrids\_100m/2024/RAW/"}\NormalTok{,nosaukums)}
\NormalTok{saglabasanas\_cels}\OtherTok{=}\FunctionTok{paste0}\NormalTok{(}\StringTok{"./RasterGrids\_100m/2024/Scaled/"}\NormalTok{,nosaukums)}
\NormalTok{slanis}\OtherTok{=}\FunctionTok{rast}\NormalTok{(ielasisanas\_cels)}
\NormalTok{videjais}\OtherTok{=}\FunctionTok{global}\NormalTok{(slanis,}\AttributeTok{fun=}\StringTok{"mean"}\NormalTok{,}\AttributeTok{na.rm=}\ConstantTok{TRUE}\NormalTok{)}
\NormalTok{centrets}\OtherTok{=}\NormalTok{slanis}\SpecialCharTok{{-}}\NormalTok{videjais[,}\DecValTok{1}\NormalTok{]}
\NormalTok{standartnovirze}\OtherTok{=}\NormalTok{terra}\SpecialCharTok{::}\FunctionTok{global}\NormalTok{(centrets,}\AttributeTok{fun=}\StringTok{"rms"}\NormalTok{,}\AttributeTok{na.rm=}\ConstantTok{TRUE}\NormalTok{)}
\NormalTok{merogots}\OtherTok{=}\NormalTok{centrets}\SpecialCharTok{/}\NormalTok{standartnovirze[,}\DecValTok{1}\NormalTok{]}
\FunctionTok{writeRaster}\NormalTok{(merogots,}
      \AttributeTok{filename=}\NormalTok{saglabasanas\_cels,}
      \AttributeTok{overwrite=}\ConstantTok{TRUE}\NormalTok{)}
\end{Highlighting}
\end{Shaded}

\section{Distance\_Builtup\_cell}\label{ch06.089}

\textbf{filename:} \texttt{Distance\_Builtup\_cell.tif}

\textbf{layername:} \texttt{egv\_089}

\textbf{English name:} Distance to Built-Up features, average within the analysis
cell (1 ha)

\textbf{Latvian name:} Attālums līdz apbūvei, vidējais analīzes šūnā (1 ha)

\textbf{Procedure:} Derived from the \hyperref[Ch05.03]{Landscape classification}, with class 500
reclassified as 1 and others as 0. Processed using the workflow \texttt{egvtools::distance2egv()}. To
prevent potential data loss at edge cells, inverse distance weighted
(power = 2) gap filling is implemented. Finally, the layer is standardised
by subtracting the arithmetic mean and dividing by the root mean squared error.

\begin{Shaded}
\begin{Highlighting}[]
\CommentTok{\# libs {-}{-}{-}{-}}
\ControlFlowTok{if}\NormalTok{(}\SpecialCharTok{!}\FunctionTok{require}\NormalTok{(terra)) \{}\FunctionTok{install.packages}\NormalTok{(}\StringTok{"terra"}\NormalTok{); }\FunctionTok{require}\NormalTok{(terra)\}}
\ControlFlowTok{if}\NormalTok{(}\SpecialCharTok{!}\FunctionTok{require}\NormalTok{(egvtools)) \{remotes}\SpecialCharTok{::}\FunctionTok{install\_github}\NormalTok{(}\StringTok{"aavotins/egvtools"}\NormalTok{); }\FunctionTok{require}\NormalTok{(egvtools)\}}

\CommentTok{\# Distance\_Builtup\_cell.tif egv\_89 {-}{-}{-}{-}}
\NormalTok{simple\_landscape}\OtherTok{=}\FunctionTok{rast}\NormalTok{(}\StringTok{"./RasterGrids\_10m/2024/Ainava\_vienk\_mask.tif"}\NormalTok{)}
\NormalTok{builtup}\OtherTok{=}\FunctionTok{ifel}\NormalTok{(simple\_landscape}\SpecialCharTok{==}\DecValTok{500}\NormalTok{,}\DecValTok{1}\NormalTok{,}\DecValTok{0}\NormalTok{)}
\FunctionTok{plot}\NormalTok{(builtup)}
\NormalTok{distegv}\OtherTok{=}\FunctionTok{distance2egv}\NormalTok{(}\AttributeTok{input =}\NormalTok{ builtup,}
           \AttributeTok{template\_egv =}\NormalTok{ template100,}
           \AttributeTok{values\_as\_one =} \DecValTok{1}\NormalTok{,}
           \AttributeTok{fill\_gaps =} \ConstantTok{TRUE}\NormalTok{, }\AttributeTok{idw\_weight =} \DecValTok{2}\NormalTok{,}
           \AttributeTok{outlocation =} \StringTok{"RasterGrids\_100m/2024/RAW/"}\NormalTok{,}
           \AttributeTok{outfilename =} \StringTok{"Distance\_Builtup\_cell.tif"}\NormalTok{,}
           \AttributeTok{layername =} \StringTok{"egv\_089"}\NormalTok{)}
\NormalTok{distegv}
\FunctionTok{plot}\NormalTok{(}\FunctionTok{rast}\NormalTok{(}\StringTok{"RasterGrids\_100m/2024/RAW/Distance\_Builtup\_cell.tif"}\NormalTok{))}
\FunctionTok{rm}\NormalTok{(builtup)}
\FunctionTok{rm}\NormalTok{(distegv)}

\CommentTok{\# standardisation {-}{-}{-}{-}}
\ControlFlowTok{if}\NormalTok{(}\SpecialCharTok{!}\FunctionTok{require}\NormalTok{(terra)) \{}\FunctionTok{install.packages}\NormalTok{(}\StringTok{"terra"}\NormalTok{); }\FunctionTok{require}\NormalTok{(terra)\}}
\ControlFlowTok{if}\NormalTok{(}\SpecialCharTok{!}\FunctionTok{require}\NormalTok{(tidyverse)) \{}\FunctionTok{install.packages}\NormalTok{(}\StringTok{"tidyverse"}\NormalTok{); }\FunctionTok{require}\NormalTok{(tidyverse)\}}

\NormalTok{nosaukums}\OtherTok{=}\StringTok{"Distance\_Builtup\_cell.tif"}
\NormalTok{ielasisanas\_cels}\OtherTok{=}\FunctionTok{paste0}\NormalTok{(}\StringTok{"./RasterGrids\_100m/2024/RAW/"}\NormalTok{,nosaukums)}
\NormalTok{saglabasanas\_cels}\OtherTok{=}\FunctionTok{paste0}\NormalTok{(}\StringTok{"./RasterGrids\_100m/2024/Scaled/"}\NormalTok{,nosaukums)}
\NormalTok{slanis}\OtherTok{=}\FunctionTok{rast}\NormalTok{(ielasisanas\_cels)}
\NormalTok{videjais}\OtherTok{=}\FunctionTok{global}\NormalTok{(slanis,}\AttributeTok{fun=}\StringTok{"mean"}\NormalTok{,}\AttributeTok{na.rm=}\ConstantTok{TRUE}\NormalTok{)}
\NormalTok{centrets}\OtherTok{=}\NormalTok{slanis}\SpecialCharTok{{-}}\NormalTok{videjais[,}\DecValTok{1}\NormalTok{]}
\NormalTok{standartnovirze}\OtherTok{=}\NormalTok{terra}\SpecialCharTok{::}\FunctionTok{global}\NormalTok{(centrets,}\AttributeTok{fun=}\StringTok{"rms"}\NormalTok{,}\AttributeTok{na.rm=}\ConstantTok{TRUE}\NormalTok{)}
\NormalTok{merogots}\OtherTok{=}\NormalTok{centrets}\SpecialCharTok{/}\NormalTok{standartnovirze[,}\DecValTok{1}\NormalTok{]}
\FunctionTok{writeRaster}\NormalTok{(merogots,}
      \AttributeTok{filename=}\NormalTok{saglabasanas\_cels,}
      \AttributeTok{overwrite=}\ConstantTok{TRUE}\NormalTok{)}
\end{Highlighting}
\end{Shaded}

\section{Distance\_ForestInside\_cell}\label{ch06.090}

\textbf{filename:} \texttt{Distance\_ForestInside\_cell.tif}

\textbf{layername:} \texttt{egv\_090}

\textbf{English name:} Distance to Forest Edge Inside Forests, average within the
analysis cell (1 ha)

\textbf{Latvian name:} Attālums līdz meža malai tā iekšienē, vidējais analīzes šūnā
(1 ha)

\textbf{Procedure:} Derived from the \hyperref[Ch05.03]{Landscape classification}, with values in
a range from 630 to 700 reclassified as 0 and others as 1. Processed
using the workflow \texttt{egvtools::distance2egv()}. To
prevent potential data loss at edge cells, inverse distance weighted
(power = 2) gap filling is implemented. Finally, the layer is standardised
by subtracting the arithmetic mean and dividing by the root mean squared error.

\begin{Shaded}
\begin{Highlighting}[]
\CommentTok{\# libs {-}{-}{-}{-}}
\ControlFlowTok{if}\NormalTok{(}\SpecialCharTok{!}\FunctionTok{require}\NormalTok{(terra)) \{}\FunctionTok{install.packages}\NormalTok{(}\StringTok{"terra"}\NormalTok{); }\FunctionTok{require}\NormalTok{(terra)\}}
\ControlFlowTok{if}\NormalTok{(}\SpecialCharTok{!}\FunctionTok{require}\NormalTok{(egvtools)) \{remotes}\SpecialCharTok{::}\FunctionTok{install\_github}\NormalTok{(}\StringTok{"aavotins/egvtools"}\NormalTok{); }\FunctionTok{require}\NormalTok{(egvtools)\}}

\CommentTok{\# Distance\_ForestInside\_cell.tif    egv\_90 {-}{-}{-}{-}}
\NormalTok{simple\_landscape}\OtherTok{=}\FunctionTok{rast}\NormalTok{(}\StringTok{"./RasterGrids\_10m/2024/Ainava\_vienk\_mask.tif"}\NormalTok{)}
\NormalTok{trees\_inside}\OtherTok{=}\FunctionTok{ifel}\NormalTok{(simple\_landscape}\SpecialCharTok{\textgreater{}=}\DecValTok{630}\SpecialCharTok{\&}\NormalTok{simple\_landscape}\SpecialCharTok{\textless{}}\DecValTok{700}\NormalTok{,}\DecValTok{0}\NormalTok{,}\DecValTok{1}\NormalTok{)}
\FunctionTok{plot}\NormalTok{(trees\_inside)}
\NormalTok{distegv}\OtherTok{=}\FunctionTok{distance2egv}\NormalTok{(}\AttributeTok{input =}\NormalTok{ trees\_inside,}
           \AttributeTok{template\_egv =}\NormalTok{ template100,}
           \AttributeTok{values\_as\_one =} \DecValTok{1}\NormalTok{,}
           \AttributeTok{fill\_gaps =} \ConstantTok{TRUE}\NormalTok{, }\AttributeTok{idw\_weight =} \DecValTok{2}\NormalTok{,}
           \AttributeTok{outlocation =} \StringTok{"RasterGrids\_100m/2024/RAW/"}\NormalTok{,}
           \AttributeTok{outfilename =} \StringTok{"Distance\_ForestInside\_cell.tif"}\NormalTok{,}
           \AttributeTok{layername =} \StringTok{"egv\_090"}\NormalTok{)}
\NormalTok{distegv}
\FunctionTok{plot}\NormalTok{(}\FunctionTok{rast}\NormalTok{(}\StringTok{"RasterGrids\_100m/2024/RAW/Distance\_ForestInside\_cell.tif"}\NormalTok{))}
\FunctionTok{rm}\NormalTok{(trees\_inside)}
\FunctionTok{rm}\NormalTok{(distegv)}

\CommentTok{\# standardisation {-}{-}{-}{-}}
\ControlFlowTok{if}\NormalTok{(}\SpecialCharTok{!}\FunctionTok{require}\NormalTok{(terra)) \{}\FunctionTok{install.packages}\NormalTok{(}\StringTok{"terra"}\NormalTok{); }\FunctionTok{require}\NormalTok{(terra)\}}
\ControlFlowTok{if}\NormalTok{(}\SpecialCharTok{!}\FunctionTok{require}\NormalTok{(tidyverse)) \{}\FunctionTok{install.packages}\NormalTok{(}\StringTok{"tidyverse"}\NormalTok{); }\FunctionTok{require}\NormalTok{(tidyverse)\}}

\NormalTok{nosaukums}\OtherTok{=}\StringTok{"Distance\_ForestInside\_cell.tif"}
\NormalTok{ielasisanas\_cels}\OtherTok{=}\FunctionTok{paste0}\NormalTok{(}\StringTok{"./RasterGrids\_100m/2024/RAW/"}\NormalTok{,nosaukums)}
\NormalTok{saglabasanas\_cels}\OtherTok{=}\FunctionTok{paste0}\NormalTok{(}\StringTok{"./RasterGrids\_100m/2024/Scaled/"}\NormalTok{,nosaukums)}
\NormalTok{slanis}\OtherTok{=}\FunctionTok{rast}\NormalTok{(ielasisanas\_cels)}
\NormalTok{videjais}\OtherTok{=}\FunctionTok{global}\NormalTok{(slanis,}\AttributeTok{fun=}\StringTok{"mean"}\NormalTok{,}\AttributeTok{na.rm=}\ConstantTok{TRUE}\NormalTok{)}
\NormalTok{centrets}\OtherTok{=}\NormalTok{slanis}\SpecialCharTok{{-}}\NormalTok{videjais[,}\DecValTok{1}\NormalTok{]}
\NormalTok{standartnovirze}\OtherTok{=}\NormalTok{terra}\SpecialCharTok{::}\FunctionTok{global}\NormalTok{(centrets,}\AttributeTok{fun=}\StringTok{"rms"}\NormalTok{,}\AttributeTok{na.rm=}\ConstantTok{TRUE}\NormalTok{)}
\NormalTok{merogots}\OtherTok{=}\NormalTok{centrets}\SpecialCharTok{/}\NormalTok{standartnovirze[,}\DecValTok{1}\NormalTok{]}
\FunctionTok{writeRaster}\NormalTok{(merogots,}
      \AttributeTok{filename=}\NormalTok{saglabasanas\_cels,}
      \AttributeTok{overwrite=}\ConstantTok{TRUE}\NormalTok{)}
\end{Highlighting}
\end{Shaded}

\section{Distance\_GrasslandPermanent\_cell}\label{ch06.091}

\textbf{filename:} \texttt{Distance\_GrasslandPermanent\_cell.tif}

\textbf{layername:} \texttt{egv\_091}

\textbf{English name:} Distance to Permanent Grasslands, average within the analysis
cell (1 ha)

\textbf{Latvian name:} Attālums līdz ilggadīgiem zālājiem, vidējais analīzes šūnā (1
ha)

\textbf{Procedure:} Derived from the \hyperref[Ch04.02]{Rural Support Service's information on declared
fields} where \texttt{PRODUCT\_CODE=="710"} classified as 1 and the rest of the
country as 0. Processed using the workflow \texttt{egvtools::distance2egv()}. To
prevent potential data loss at edge cells, inverse distance weighted
(power = 2) gap filling is implemented. Finally, the layer is standardised
by subtracting the arithmetic mean and dividing by the root mean squared error.

\begin{Shaded}
\begin{Highlighting}[]
\CommentTok{\# libs {-}{-}{-}{-}}
\ControlFlowTok{if}\NormalTok{(}\SpecialCharTok{!}\FunctionTok{require}\NormalTok{(terra)) \{}\FunctionTok{install.packages}\NormalTok{(}\StringTok{"terra"}\NormalTok{); }\FunctionTok{require}\NormalTok{(terra)\}}
\ControlFlowTok{if}\NormalTok{(}\SpecialCharTok{!}\FunctionTok{require}\NormalTok{(sf)) \{}\FunctionTok{install.packages}\NormalTok{(}\StringTok{"sf"}\NormalTok{); }\FunctionTok{require}\NormalTok{(sf)\}}
\ControlFlowTok{if}\NormalTok{(}\SpecialCharTok{!}\FunctionTok{require}\NormalTok{(sfarrow)) \{}\FunctionTok{install.packages}\NormalTok{(}\StringTok{"sfarrow"}\NormalTok{); }\FunctionTok{require}\NormalTok{(sfarrow)\}}
\ControlFlowTok{if}\NormalTok{(}\SpecialCharTok{!}\FunctionTok{require}\NormalTok{(tidyverse)) \{}\FunctionTok{install.packages}\NormalTok{(}\StringTok{"tidyverse"}\NormalTok{); }\FunctionTok{require}\NormalTok{(tidyverse)\}}
\ControlFlowTok{if}\NormalTok{(}\SpecialCharTok{!}\FunctionTok{require}\NormalTok{(readxl)) \{}\FunctionTok{install.packages}\NormalTok{(}\StringTok{"readxl"}\NormalTok{); }\FunctionTok{require}\NormalTok{(readxl)\}}
\ControlFlowTok{if}\NormalTok{(}\SpecialCharTok{!}\FunctionTok{require}\NormalTok{(raster)) \{}\FunctionTok{install.packages}\NormalTok{(}\StringTok{"raster"}\NormalTok{); }\FunctionTok{require}\NormalTok{(raster)\}}
\ControlFlowTok{if}\NormalTok{(}\SpecialCharTok{!}\FunctionTok{require}\NormalTok{(fasterize)) \{}\FunctionTok{install.packages}\NormalTok{(}\StringTok{"fasterize"}\NormalTok{); }\FunctionTok{require}\NormalTok{(fasterize)\}}
\ControlFlowTok{if}\NormalTok{(}\SpecialCharTok{!}\FunctionTok{require}\NormalTok{(egvtools)) \{remotes}\SpecialCharTok{::}\FunctionTok{install\_github}\NormalTok{(}\StringTok{"aavotins/egvtools"}\NormalTok{); }\FunctionTok{require}\NormalTok{(egvtools)\}}

\CommentTok{\# templates {-}{-}{-}{-}}
\NormalTok{template10}\OtherTok{=}\FunctionTok{rast}\NormalTok{(}\StringTok{"./Templates/TemplateRasters/LV10m\_10km.tif"}\NormalTok{)}
\NormalTok{nulls10}\OtherTok{=}\FunctionTok{rast}\NormalTok{(}\StringTok{"./Templates/TemplateRasters/nulls\_LV10m\_10km.tif"}\NormalTok{)}

\NormalTok{rastra\_pamatne}\OtherTok{=}\FunctionTok{raster}\NormalTok{(template10)}

\CommentTok{\# Distance\_GrasslandPermanent\_cell.tif  egv\_91 {-}{-}{-}{-}}
\NormalTok{kodes}\OtherTok{=}\FunctionTok{read\_excel}\NormalTok{(}\StringTok{"./Geodata/2024/LAD/KulturuKodi\_2024.xlsx"}\NormalTok{)}
\NormalTok{lad}\OtherTok{=}\NormalTok{sfarrow}\SpecialCharTok{::}\FunctionTok{st\_read\_parquet}\NormalTok{(}\StringTok{"./Geodata/2024/LAD/Lauki\_2024.parquet"}\NormalTok{)}
\NormalTok{permgrass}\OtherTok{=}\NormalTok{lad }\SpecialCharTok{\%\textgreater{}\%} 
 \FunctionTok{filter}\NormalTok{(PRODUCT\_CODE}\SpecialCharTok{==}\StringTok{"710"}\NormalTok{) }\SpecialCharTok{\%\textgreater{}\%} 
 \FunctionTok{mutate}\NormalTok{(}\AttributeTok{yes=}\DecValTok{1}\NormalTok{)}
\NormalTok{permgrass\_r}\OtherTok{=}\FunctionTok{fasterize}\NormalTok{(permgrass,rastra\_pamatne,}\AttributeTok{field=}\StringTok{"yes"}\NormalTok{,}\AttributeTok{fun=}\StringTok{"first"}\NormalTok{)}
\NormalTok{permgrass\_t}\OtherTok{=}\FunctionTok{rast}\NormalTok{(permgrass\_r)}
\NormalTok{permgrass\_t2}\OtherTok{=}\FunctionTok{cover}\NormalTok{(permgrass\_t,nulls10)}
\FunctionTok{plot}\NormalTok{(permgrass\_t2)}
\NormalTok{distegv}\OtherTok{=}\FunctionTok{distance2egv}\NormalTok{(}\AttributeTok{input =}\NormalTok{ permgrass\_t2,}
           \AttributeTok{template\_egv =}\NormalTok{ template100,}
           \AttributeTok{values\_as\_one =} \DecValTok{1}\NormalTok{,}
           \AttributeTok{fill\_gaps =} \ConstantTok{TRUE}\NormalTok{, }\AttributeTok{idw\_weight =} \DecValTok{2}\NormalTok{,}
           \AttributeTok{outlocation =} \StringTok{"RasterGrids\_100m/2024/RAW/"}\NormalTok{,}
           \AttributeTok{outfilename =} \StringTok{"Distance\_GrasslandPermanent\_cell.tif"}\NormalTok{,}
           \AttributeTok{layername =} \StringTok{"egv\_091"}\NormalTok{)}
\NormalTok{distegv}
\FunctionTok{plot}\NormalTok{(}\FunctionTok{rast}\NormalTok{(}\StringTok{"RasterGrids\_100m/2024/RAW/Distance\_GrasslandPermanent\_cell.tif"}\NormalTok{))}
\FunctionTok{rm}\NormalTok{(distegv)}
\FunctionTok{rm}\NormalTok{(kodes)}
\FunctionTok{rm}\NormalTok{(lad)}
\FunctionTok{rm}\NormalTok{(permgrass)}
\FunctionTok{rm}\NormalTok{(permgrass\_r)}
\FunctionTok{rm}\NormalTok{(permgrass\_t)}
\FunctionTok{rm}\NormalTok{(permgrass\_t2)}

\CommentTok{\# standardisation {-}{-}{-}{-}}
\ControlFlowTok{if}\NormalTok{(}\SpecialCharTok{!}\FunctionTok{require}\NormalTok{(terra)) \{}\FunctionTok{install.packages}\NormalTok{(}\StringTok{"terra"}\NormalTok{); }\FunctionTok{require}\NormalTok{(terra)\}}
\ControlFlowTok{if}\NormalTok{(}\SpecialCharTok{!}\FunctionTok{require}\NormalTok{(tidyverse)) \{}\FunctionTok{install.packages}\NormalTok{(}\StringTok{"tidyverse"}\NormalTok{); }\FunctionTok{require}\NormalTok{(tidyverse)\}}

\NormalTok{nosaukums}\OtherTok{=}\StringTok{"Distance\_GrasslandPermanent\_cell.tif"}
\NormalTok{ielasisanas\_cels}\OtherTok{=}\FunctionTok{paste0}\NormalTok{(}\StringTok{"./RasterGrids\_100m/2024/RAW/"}\NormalTok{,nosaukums)}
\NormalTok{saglabasanas\_cels}\OtherTok{=}\FunctionTok{paste0}\NormalTok{(}\StringTok{"./RasterGrids\_100m/2024/Scaled/"}\NormalTok{,nosaukums)}
\NormalTok{slanis}\OtherTok{=}\FunctionTok{rast}\NormalTok{(ielasisanas\_cels)}
\NormalTok{videjais}\OtherTok{=}\FunctionTok{global}\NormalTok{(slanis,}\AttributeTok{fun=}\StringTok{"mean"}\NormalTok{,}\AttributeTok{na.rm=}\ConstantTok{TRUE}\NormalTok{)}
\NormalTok{centrets}\OtherTok{=}\NormalTok{slanis}\SpecialCharTok{{-}}\NormalTok{videjais[,}\DecValTok{1}\NormalTok{]}
\NormalTok{standartnovirze}\OtherTok{=}\NormalTok{terra}\SpecialCharTok{::}\FunctionTok{global}\NormalTok{(centrets,}\AttributeTok{fun=}\StringTok{"rms"}\NormalTok{,}\AttributeTok{na.rm=}\ConstantTok{TRUE}\NormalTok{)}
\NormalTok{merogots}\OtherTok{=}\NormalTok{centrets}\SpecialCharTok{/}\NormalTok{standartnovirze[,}\DecValTok{1}\NormalTok{]}
\FunctionTok{writeRaster}\NormalTok{(merogots,}
      \AttributeTok{filename=}\NormalTok{saglabasanas\_cels,}
      \AttributeTok{overwrite=}\ConstantTok{TRUE}\NormalTok{)}
\end{Highlighting}
\end{Shaded}

\section{Distance\_Landfill\_cell}\label{ch06.092}

\textbf{filename:} \texttt{Distance\_Landfill\_cell.tif}

\textbf{layername:} \texttt{egv\_092}

\textbf{English name:} Distance to Landfills, average within the analysis cell (1 ha)

\textbf{Latvian name:} Attālums līdz atkritumu poligoniem, vidējais analīzes šūnā (1
ha)

\textbf{Procedure:} Directly follows \hyperref[Ch04.14]{Waste and garbage disposal sites,
landfills}.

\begin{enumerate}
\def\labelenumi{\arabic{enumi}.}
\item
  From the \href{https://github.com/aavotins/HiQBioDiv_EGVs/blob/main/Data/Geodata/2024/GarbageWasteLandfills/Atkritumi.xlsx}{attachaed
  file},
  read sheet ``Poligoni'';
\item
  Create an \texttt{sf} object (EPSG:3059);
\item
  Rasterize and cover so that cells of interest are 1 and others are 0;
\item
  Create an EGV using the workflow \texttt{egvtools::distance2egv()}. Expect warning regarding
  nothing to do with aggregation. This occurs because \texttt{egvtools::distance2egv()}
  already operate at EGV template not the input template resolution. To prevent
  potential data loss at edge cells,
  inverse distance weighted (power = 2) gap filling is implemented.
  Finally, the layer is standardised by subtracting the arithmetic mean
  and dividing by the root mean squared error.
\end{enumerate}

\begin{Shaded}
\begin{Highlighting}[]
\CommentTok{\# libs {-}{-}{-}{-}}
\ControlFlowTok{if}\NormalTok{(}\SpecialCharTok{!}\FunctionTok{require}\NormalTok{(terra)) \{}\FunctionTok{install.packages}\NormalTok{(}\StringTok{"terra"}\NormalTok{); }\FunctionTok{require}\NormalTok{(terra)\}}
\ControlFlowTok{if}\NormalTok{(}\SpecialCharTok{!}\FunctionTok{require}\NormalTok{(sf)) \{}\FunctionTok{install.packages}\NormalTok{(}\StringTok{"sf"}\NormalTok{); }\FunctionTok{require}\NormalTok{(sf)\}}
\ControlFlowTok{if}\NormalTok{(}\SpecialCharTok{!}\FunctionTok{require}\NormalTok{(tidyverse)) \{}\FunctionTok{install.packages}\NormalTok{(}\StringTok{"tidyverse"}\NormalTok{); }\FunctionTok{require}\NormalTok{(tidyverse)\}}
\ControlFlowTok{if}\NormalTok{(}\SpecialCharTok{!}\FunctionTok{require}\NormalTok{(readxl)) \{}\FunctionTok{install.packages}\NormalTok{(}\StringTok{"readxl"}\NormalTok{); }\FunctionTok{require}\NormalTok{(readxl)\}}
\ControlFlowTok{if}\NormalTok{(}\SpecialCharTok{!}\FunctionTok{require}\NormalTok{(egvtools)) \{remotes}\SpecialCharTok{::}\FunctionTok{install\_github}\NormalTok{(}\StringTok{"aavotins/egvtools"}\NormalTok{); }\FunctionTok{require}\NormalTok{(egvtools)\}}

\CommentTok{\# templates {-}{-}{-}{-}}
\NormalTok{template100}\OtherTok{=}\FunctionTok{rast}\NormalTok{(}\StringTok{"./Templates/TemplateRasters/LV100m\_10km.tif"}\NormalTok{)}
\NormalTok{nulls100}\OtherTok{=}\FunctionTok{rast}\NormalTok{(}\StringTok{"./Templates/TemplateRasters/nulls\_LV100m\_10km.tif"}\NormalTok{)}

\CommentTok{\# Distance\_Landfill\_cell.tif egv\_92 {-}{-}{-}{-}}

\CommentTok{\# reading coordinates}
\NormalTok{landfills}\OtherTok{=}\FunctionTok{read\_excel}\NormalTok{(}\StringTok{"./Geodata/2024/GarbageWasteLandfills/Atkritumi.xlsx"}\NormalTok{,}\AttributeTok{sheet=}\StringTok{"Poligoni"}\NormalTok{)}
\CommentTok{\#sf object}
\NormalTok{landfills\_sf}\OtherTok{=}\FunctionTok{st\_as\_sf}\NormalTok{(landfills,}\AttributeTok{coords=}\FunctionTok{c}\NormalTok{(}\StringTok{"X"}\NormalTok{,}\StringTok{"Y"}\NormalTok{),}\AttributeTok{crs=}\DecValTok{3059}\NormalTok{)}
\CommentTok{\# rasterize}
\NormalTok{landfills\_rast}\OtherTok{=}\FunctionTok{rasterize}\NormalTok{(landfills\_sf,template100)}
\CommentTok{\# raster to 1=Cell of interest, 0=background}
\NormalTok{landfills\_bg}\OtherTok{=}\FunctionTok{cover}\NormalTok{(landfills\_rast,nulls100)}

\CommentTok{\# create an egv}
\NormalTok{distegv}\OtherTok{=}\FunctionTok{distance2egv}\NormalTok{(}\AttributeTok{input =}\NormalTok{ landfills\_bg,}
       \AttributeTok{template\_egv =}\NormalTok{ template100,}
       \AttributeTok{values\_as\_one =} \DecValTok{1}\NormalTok{,}
       \AttributeTok{fill\_gaps =} \ConstantTok{TRUE}\NormalTok{, }\AttributeTok{idw\_weight =} \DecValTok{2}\NormalTok{,}
       \AttributeTok{outlocation =} \StringTok{"RasterGrids\_100m/2024/RAW/"}\NormalTok{,}
       \AttributeTok{outfilename =} \StringTok{"Distance\_Landfill\_cell.tif"}\NormalTok{,}
       \AttributeTok{layername =} \StringTok{"egv\_092"}\NormalTok{)}
\NormalTok{distegv}

\CommentTok{\# standardisation {-}{-}{-}{-}}
\ControlFlowTok{if}\NormalTok{(}\SpecialCharTok{!}\FunctionTok{require}\NormalTok{(terra)) \{}\FunctionTok{install.packages}\NormalTok{(}\StringTok{"terra"}\NormalTok{); }\FunctionTok{require}\NormalTok{(terra)\}}
\ControlFlowTok{if}\NormalTok{(}\SpecialCharTok{!}\FunctionTok{require}\NormalTok{(tidyverse)) \{}\FunctionTok{install.packages}\NormalTok{(}\StringTok{"tidyverse"}\NormalTok{); }\FunctionTok{require}\NormalTok{(tidyverse)\}}

\NormalTok{nosaukums}\OtherTok{=}\StringTok{"Distance\_Landfill\_cell.tif"}
\NormalTok{ielasisanas\_cels}\OtherTok{=}\FunctionTok{paste0}\NormalTok{(}\StringTok{"./RasterGrids\_100m/2024/RAW/"}\NormalTok{,nosaukums)}
\NormalTok{saglabasanas\_cels}\OtherTok{=}\FunctionTok{paste0}\NormalTok{(}\StringTok{"./RasterGrids\_100m/2024/Scaled/"}\NormalTok{,nosaukums)}
\NormalTok{slanis}\OtherTok{=}\FunctionTok{rast}\NormalTok{(ielasisanas\_cels)}
\NormalTok{videjais}\OtherTok{=}\FunctionTok{global}\NormalTok{(slanis,}\AttributeTok{fun=}\StringTok{"mean"}\NormalTok{,}\AttributeTok{na.rm=}\ConstantTok{TRUE}\NormalTok{)}
\NormalTok{centrets}\OtherTok{=}\NormalTok{slanis}\SpecialCharTok{{-}}\NormalTok{videjais[,}\DecValTok{1}\NormalTok{]}
\NormalTok{standartnovirze}\OtherTok{=}\NormalTok{terra}\SpecialCharTok{::}\FunctionTok{global}\NormalTok{(centrets,}\AttributeTok{fun=}\StringTok{"rms"}\NormalTok{,}\AttributeTok{na.rm=}\ConstantTok{TRUE}\NormalTok{)}
\NormalTok{merogots}\OtherTok{=}\NormalTok{centrets}\SpecialCharTok{/}\NormalTok{standartnovirze[,}\DecValTok{1}\NormalTok{]}
\FunctionTok{writeRaster}\NormalTok{(merogots,}
      \AttributeTok{filename=}\NormalTok{saglabasanas\_cels,}
      \AttributeTok{overwrite=}\ConstantTok{TRUE}\NormalTok{)}
\end{Highlighting}
\end{Shaded}

\section{Distance\_Sea\_cell}\label{ch06.093}

\textbf{filename:} \texttt{Distance\_Sea\_cell.tif}

\textbf{layername:} \texttt{egv\_093}

\textbf{English name:} Distance to Sea, average within the analysis cell (1 ha)

\textbf{Latvian name:} Attālums līdz jūrai, vidējais analīzes šūnā (1 ha)

\textbf{Procedure:} Directly follows \hyperref[Ch04.16]{Latvian Exclusive Economic Zone
polygon}.

\begin{enumerate}
\def\labelenumi{\arabic{enumi}.}
\item
  Read layer as \texttt{sf} object (it already is EPSG:3059);
\item
  Rasterize and cover so that cells of interest are 1 and others are 0;
\item
  Create an egv with the workflow \texttt{egvtools::distance2egv()}. The \{fasterize\} package does not write
  CRS with \texttt{WKT} from the EPSG-string; therefore, it is better to use
  \texttt{project\_to\_template\_input=TRUE} and define input-template. However, the
  only difference is in how the CRS is stored, therefore this can ignored -
  distance will be calculated on the input CRS and only resulting layer will
  be projected to match EGV template (faster due to 10x aggregation of
  resolution). To protect against possible data loss at edge cells, inverse
  distance weighted (power = 2) gap filling is implemented. Finally, the layer
  is standardised by subtracting the arithmetic mean and dividing by the
  root mean squared error.
\end{enumerate}

\begin{Shaded}
\begin{Highlighting}[]
\CommentTok{\# libs {-}{-}{-}{-}}
\ControlFlowTok{if}\NormalTok{(}\SpecialCharTok{!}\FunctionTok{require}\NormalTok{(terra)) \{}\FunctionTok{install.packages}\NormalTok{(}\StringTok{"terra"}\NormalTok{); }\FunctionTok{require}\NormalTok{(terra)\}}
\ControlFlowTok{if}\NormalTok{(}\SpecialCharTok{!}\FunctionTok{require}\NormalTok{(sf)) \{}\FunctionTok{install.packages}\NormalTok{(}\StringTok{"sf"}\NormalTok{); }\FunctionTok{require}\NormalTok{(sf)\}}
\ControlFlowTok{if}\NormalTok{(}\SpecialCharTok{!}\FunctionTok{require}\NormalTok{(egvtools)) \{remotes}\SpecialCharTok{::}\FunctionTok{install\_github}\NormalTok{(}\StringTok{"aavotins/egvtools"}\NormalTok{); }\FunctionTok{require}\NormalTok{(egvtools)\}}
\ControlFlowTok{if}\NormalTok{(}\SpecialCharTok{!}\FunctionTok{require}\NormalTok{(raster)) \{}\FunctionTok{install.packages}\NormalTok{(}\StringTok{"raster"}\NormalTok{); }\FunctionTok{require}\NormalTok{(raster)\}}
\ControlFlowTok{if}\NormalTok{(}\SpecialCharTok{!}\FunctionTok{require}\NormalTok{(fasterize)) \{}\FunctionTok{install.packages}\NormalTok{(}\StringTok{"fasterize"}\NormalTok{); }\FunctionTok{require}\NormalTok{(fasterize)\}}


\CommentTok{\# templates {-}{-}{-}{-}}
\NormalTok{template10}\OtherTok{=}\FunctionTok{rast}\NormalTok{(}\StringTok{"./Templates/TemplateRasters/LV10m\_10km.tif"}\NormalTok{)}
\NormalTok{nulls10}\OtherTok{=}\FunctionTok{rast}\NormalTok{(}\StringTok{"./Templates/TemplateRasters/nulls\_LV10m\_10km.tif"}\NormalTok{)}
\NormalTok{rastrs10}\OtherTok{=}\NormalTok{raster}\SpecialCharTok{::}\FunctionTok{raster}\NormalTok{(template10)}


\CommentTok{\# Distance\_Sea\_cell.tif egv\_93 {-}{-}{-}{-}}

\CommentTok{\# sea layer, sf}
\NormalTok{sea}\OtherTok{=}\FunctionTok{st\_read}\NormalTok{(}\StringTok{"./Geodata/2024/LV\_EEZ/LV\_EEZ.shp"}\NormalTok{)}

\CommentTok{\# quick rasterisation}
\NormalTok{sea\_r}\OtherTok{=}\FunctionTok{fasterize}\NormalTok{(sea,rastrs10,}\AttributeTok{field=}\StringTok{"LV\_EEZ"}\NormalTok{)}
\NormalTok{sea\_rast}\OtherTok{=}\FunctionTok{rast}\NormalTok{(sea\_r)}

\CommentTok{\# raster to 1=Cell of interest, 0=background}
\NormalTok{sea\_bg}\OtherTok{=}\FunctionTok{cover}\NormalTok{(sea\_rast,nulls10)}

\CommentTok{\# create an egv}
\NormalTok{distegv}\OtherTok{=}\FunctionTok{distance2egv}\NormalTok{(}\AttributeTok{input =}\NormalTok{ sea\_bg,}
           \AttributeTok{template\_egv =} \StringTok{"./Templates/TemplateRasters/LV100m\_10km.tif"}\NormalTok{,}
           \AttributeTok{values\_as\_one =} \DecValTok{1}\NormalTok{,}
           \AttributeTok{project\_to\_template\_input=}\ConstantTok{TRUE}\NormalTok{, }\CommentTok{\# fasterize stores CRS differently}
           \AttributeTok{template\_input=}\NormalTok{template10,}
           \AttributeTok{fill\_gaps =} \ConstantTok{TRUE}\NormalTok{, }\AttributeTok{idw\_weight =} \DecValTok{2}\NormalTok{,}
           \AttributeTok{outlocation =} \StringTok{"RasterGrids\_100m/2024/RAW/"}\NormalTok{,}
           \AttributeTok{outfilename =} \StringTok{"Distance\_Sea\_cell.tif"}\NormalTok{,}
           \AttributeTok{layername =} \StringTok{"egv\_093"}\NormalTok{)}
\NormalTok{distegv}

\CommentTok{\# standardisation {-}{-}{-}{-}}
\ControlFlowTok{if}\NormalTok{(}\SpecialCharTok{!}\FunctionTok{require}\NormalTok{(terra)) \{}\FunctionTok{install.packages}\NormalTok{(}\StringTok{"terra"}\NormalTok{); }\FunctionTok{require}\NormalTok{(terra)\}}
\ControlFlowTok{if}\NormalTok{(}\SpecialCharTok{!}\FunctionTok{require}\NormalTok{(tidyverse)) \{}\FunctionTok{install.packages}\NormalTok{(}\StringTok{"tidyverse"}\NormalTok{); }\FunctionTok{require}\NormalTok{(tidyverse)\}}

\NormalTok{nosaukums}\OtherTok{=}\StringTok{"Distance\_Sea\_cell.tif"}
\NormalTok{ielasisanas\_cels}\OtherTok{=}\FunctionTok{paste0}\NormalTok{(}\StringTok{"./RasterGrids\_100m/2024/RAW/"}\NormalTok{,nosaukums)}
\NormalTok{saglabasanas\_cels}\OtherTok{=}\FunctionTok{paste0}\NormalTok{(}\StringTok{"./RasterGrids\_100m/2024/Scaled/"}\NormalTok{,nosaukums)}
\NormalTok{slanis}\OtherTok{=}\FunctionTok{rast}\NormalTok{(ielasisanas\_cels)}
\NormalTok{videjais}\OtherTok{=}\FunctionTok{global}\NormalTok{(slanis,}\AttributeTok{fun=}\StringTok{"mean"}\NormalTok{,}\AttributeTok{na.rm=}\ConstantTok{TRUE}\NormalTok{)}
\NormalTok{centrets}\OtherTok{=}\NormalTok{slanis}\SpecialCharTok{{-}}\NormalTok{videjais[,}\DecValTok{1}\NormalTok{]}
\NormalTok{standartnovirze}\OtherTok{=}\NormalTok{terra}\SpecialCharTok{::}\FunctionTok{global}\NormalTok{(centrets,}\AttributeTok{fun=}\StringTok{"rms"}\NormalTok{,}\AttributeTok{na.rm=}\ConstantTok{TRUE}\NormalTok{)}
\NormalTok{merogots}\OtherTok{=}\NormalTok{centrets}\SpecialCharTok{/}\NormalTok{standartnovirze[,}\DecValTok{1}\NormalTok{]}
\FunctionTok{writeRaster}\NormalTok{(merogots,}
      \AttributeTok{filename=}\NormalTok{saglabasanas\_cels,}
      \AttributeTok{overwrite=}\ConstantTok{TRUE}\NormalTok{)}
\end{Highlighting}
\end{Shaded}

\section{Distance\_Trees\_cell}\label{ch06.094}

\textbf{filename:} \texttt{Distance\_Trees\_cell.tif}

\textbf{layername:} \texttt{egv\_094}

\textbf{English name:} Distance to Trees, average within the analysis cell (1 ha)

\textbf{Latvian name:} Attālums līdz kokiem, vidējais analīzes šūnā (1 ha)

\textbf{Procedure:} Derived from the \hyperref[Ch05.03]{Landscape classification}, with values in
a range from 630 to 700 reclassified as 1 and all others as 0. Processed using the workflow
\texttt{egvtools::distance2egv()}. To prevent possible data loss at edge cells,
inverse distance weighted (power = 2) gap filling is implemented. Finally, the
layer is standardised by subtracting the arithmetic mean and dividing by the
root mean squared error.

\begin{Shaded}
\begin{Highlighting}[]
\CommentTok{\# libs {-}{-}{-}{-}}
\ControlFlowTok{if}\NormalTok{(}\SpecialCharTok{!}\FunctionTok{require}\NormalTok{(terra)) \{}\FunctionTok{install.packages}\NormalTok{(}\StringTok{"terra"}\NormalTok{); }\FunctionTok{require}\NormalTok{(terra)\}}
\ControlFlowTok{if}\NormalTok{(}\SpecialCharTok{!}\FunctionTok{require}\NormalTok{(egvtools)) \{remotes}\SpecialCharTok{::}\FunctionTok{install\_github}\NormalTok{(}\StringTok{"aavotins/egvtools"}\NormalTok{); }\FunctionTok{require}\NormalTok{(egvtools)\}}

\CommentTok{\# Distance\_Trees\_cell.tif   egv\_94 {-}{-}{-}{-}}
\NormalTok{simple\_landscape}\OtherTok{=}\FunctionTok{rast}\NormalTok{(}\StringTok{"./RasterGrids\_10m/2024/Ainava\_vienk\_mask.tif"}\NormalTok{)}
\NormalTok{trees}\OtherTok{=}\FunctionTok{ifel}\NormalTok{(simple\_landscape}\SpecialCharTok{\textgreater{}=}\DecValTok{630}\SpecialCharTok{\&}\NormalTok{simple\_landscape}\SpecialCharTok{\textless{}}\DecValTok{700}\NormalTok{,}\DecValTok{1}\NormalTok{,}\DecValTok{0}\NormalTok{)}
\FunctionTok{plot}\NormalTok{(trees)}
\NormalTok{distegv}\OtherTok{=}\FunctionTok{distance2egv}\NormalTok{(}\AttributeTok{input =}\NormalTok{ trees,}
           \AttributeTok{template\_egv =}\NormalTok{ template100,}
           \AttributeTok{values\_as\_one =} \DecValTok{1}\NormalTok{,}
           \AttributeTok{fill\_gaps =} \ConstantTok{TRUE}\NormalTok{, }\AttributeTok{idw\_weight =} \DecValTok{2}\NormalTok{,}
           \AttributeTok{outlocation =} \StringTok{"RasterGrids\_100m/2024/RAW/"}\NormalTok{,}
           \AttributeTok{outfilename =} \StringTok{"Distance\_Trees\_cell.tif"}\NormalTok{,}
           \AttributeTok{layername =} \StringTok{"egv\_094"}\NormalTok{)}
\NormalTok{distegv}
\FunctionTok{plot}\NormalTok{(}\FunctionTok{rast}\NormalTok{(}\StringTok{"RasterGrids\_100m/2024/RAW/Distance\_Trees\_cell.tif"}\NormalTok{))}
\FunctionTok{rm}\NormalTok{(trees)}
\FunctionTok{rm}\NormalTok{(distegv)}

\CommentTok{\# standardisation {-}{-}{-}{-}}
\ControlFlowTok{if}\NormalTok{(}\SpecialCharTok{!}\FunctionTok{require}\NormalTok{(terra)) \{}\FunctionTok{install.packages}\NormalTok{(}\StringTok{"terra"}\NormalTok{); }\FunctionTok{require}\NormalTok{(terra)\}}
\ControlFlowTok{if}\NormalTok{(}\SpecialCharTok{!}\FunctionTok{require}\NormalTok{(tidyverse)) \{}\FunctionTok{install.packages}\NormalTok{(}\StringTok{"tidyverse"}\NormalTok{); }\FunctionTok{require}\NormalTok{(tidyverse)\}}

\NormalTok{nosaukums}\OtherTok{=}\StringTok{"Distance\_Trees\_cell.tif"}
\NormalTok{ielasisanas\_cels}\OtherTok{=}\FunctionTok{paste0}\NormalTok{(}\StringTok{"./RasterGrids\_100m/2024/RAW/"}\NormalTok{,nosaukums)}
\NormalTok{saglabasanas\_cels}\OtherTok{=}\FunctionTok{paste0}\NormalTok{(}\StringTok{"./RasterGrids\_100m/2024/Scaled/"}\NormalTok{,nosaukums)}
\NormalTok{slanis}\OtherTok{=}\FunctionTok{rast}\NormalTok{(ielasisanas\_cels)}
\NormalTok{videjais}\OtherTok{=}\FunctionTok{global}\NormalTok{(slanis,}\AttributeTok{fun=}\StringTok{"mean"}\NormalTok{,}\AttributeTok{na.rm=}\ConstantTok{TRUE}\NormalTok{)}
\NormalTok{centrets}\OtherTok{=}\NormalTok{slanis}\SpecialCharTok{{-}}\NormalTok{videjais[,}\DecValTok{1}\NormalTok{]}
\NormalTok{standartnovirze}\OtherTok{=}\NormalTok{terra}\SpecialCharTok{::}\FunctionTok{global}\NormalTok{(centrets,}\AttributeTok{fun=}\StringTok{"rms"}\NormalTok{,}\AttributeTok{na.rm=}\ConstantTok{TRUE}\NormalTok{)}
\NormalTok{merogots}\OtherTok{=}\NormalTok{centrets}\SpecialCharTok{/}\NormalTok{standartnovirze[,}\DecValTok{1}\NormalTok{]}
\FunctionTok{writeRaster}\NormalTok{(merogots,}
      \AttributeTok{filename=}\NormalTok{saglabasanas\_cels,}
      \AttributeTok{overwrite=}\ConstantTok{TRUE}\NormalTok{)}
\end{Highlighting}
\end{Shaded}

\section{Distance\_Waste\_cell}\label{ch06.095}

\textbf{filename:} \texttt{Distance\_Waste\_cell.tif}

\textbf{layername:} \texttt{egv\_095}

\textbf{English name:} Distance to Waste disposal sites, average within the analysis
cell (1 ha)

\textbf{Latvian name:} Attālums līdz atkritumu šķirošanas un uzglabāšanas vietām,
vidējais analīzes šūnā (1 ha)

\textbf{Procedure:} Directly follows \hyperref[Ch04.14]{Waste and garbage disposal sites,
landfills}.

\begin{enumerate}
\def\labelenumi{\arabic{enumi}.}
\item
  From the \href{https://github.com/aavotins/HiQBioDiv_EGVs/blob/main/Data/Geodata/2024/GarbageWasteLandfills/Atkritumi.xlsx}{attachaed
  file},
  read sheet ``AtkritumuVietas'' and clean names;
\item
  Create an \texttt{sf} object (EPSG:3059);
\item
  Filter to non-deposit collection locations;
\item
  Rasterize and cover so that cells of interest are 1 and others are 0;
\item
  Create an EGV using the workflow \texttt{egvtools::distance2egv()}. Expect warning regarding
  nothing to do with aggregation. That is because \texttt{egvtools::distance2egv()}
  already operate at EGV template not the input template resolution. To
  prevent possible data loss at edge cells, inverse distance weighted
  (power = 2) gap filling is implemented.
\item
  Finally, the layer is standardised by subtracting the arithmetic mean and
  dividing by the root mean squared error.
\end{enumerate}

\begin{Shaded}
\begin{Highlighting}[]
\CommentTok{\# libs {-}{-}{-}{-}}
\ControlFlowTok{if}\NormalTok{(}\SpecialCharTok{!}\FunctionTok{require}\NormalTok{(terra)) \{}\FunctionTok{install.packages}\NormalTok{(}\StringTok{"terra"}\NormalTok{); }\FunctionTok{require}\NormalTok{(terra)\}}
\ControlFlowTok{if}\NormalTok{(}\SpecialCharTok{!}\FunctionTok{require}\NormalTok{(sf)) \{}\FunctionTok{install.packages}\NormalTok{(}\StringTok{"sf"}\NormalTok{); }\FunctionTok{require}\NormalTok{(sf)\}}
\ControlFlowTok{if}\NormalTok{(}\SpecialCharTok{!}\FunctionTok{require}\NormalTok{(tidyverse)) \{}\FunctionTok{install.packages}\NormalTok{(}\StringTok{"tidyverse"}\NormalTok{); }\FunctionTok{require}\NormalTok{(tidyverse)\}}
\ControlFlowTok{if}\NormalTok{(}\SpecialCharTok{!}\FunctionTok{require}\NormalTok{(readxl)) \{}\FunctionTok{install.packages}\NormalTok{(}\StringTok{"readxl"}\NormalTok{); }\FunctionTok{require}\NormalTok{(readxl)\}}
\ControlFlowTok{if}\NormalTok{(}\SpecialCharTok{!}\FunctionTok{require}\NormalTok{(egvtools)) \{remotes}\SpecialCharTok{::}\FunctionTok{install\_github}\NormalTok{(}\StringTok{"aavotins/egvtools"}\NormalTok{); }\FunctionTok{require}\NormalTok{(egvtools)\}}

\CommentTok{\# templates {-}{-}{-}{-}}
\NormalTok{template100}\OtherTok{=}\FunctionTok{rast}\NormalTok{(}\StringTok{"./Templates/TemplateRasters/LV100m\_10km.tif"}\NormalTok{)}
\NormalTok{nulls100}\OtherTok{=}\FunctionTok{rast}\NormalTok{(}\StringTok{"./Templates/TemplateRasters/nulls\_LV100m\_10km.tif"}\NormalTok{)}


\CommentTok{\# Distance\_Waste\_cell.tif egv\_95 {-}{-}{-}{-}}

\CommentTok{\# reading coordinates}
\NormalTok{waste}\OtherTok{=}\FunctionTok{read\_excel}\NormalTok{(}\StringTok{"./Geodata/2024/GarbageWasteLandfills/Atkritumi.xlsx"}\NormalTok{,}\AttributeTok{sheet=}\StringTok{"AtkritumuVietas"}\NormalTok{)}
\CommentTok{\# cleaning names}
\NormalTok{waste2}\OtherTok{=}\NormalTok{janitor}\SpecialCharTok{::}\FunctionTok{clean\_names}\NormalTok{(waste)}
\CommentTok{\#sf object}
\NormalTok{waste\_sf}\OtherTok{=}\FunctionTok{st\_as\_sf}\NormalTok{(waste2,}\AttributeTok{coords=}\FunctionTok{c}\NormalTok{(}\StringTok{"y\_koordinata\_lks92\_tm"}\NormalTok{,}\StringTok{"x\_koordinata\_lks92\_tm"}\NormalTok{),}\AttributeTok{crs=}\DecValTok{3059}\NormalTok{)}
\CommentTok{\# filtering to non{-}deposit}
\FunctionTok{table}\NormalTok{(waste\_sf}\SpecialCharTok{$}\NormalTok{pienemsanas\_vietas\_tips)}
\NormalTok{waste\_sf2}\OtherTok{=}\NormalTok{waste\_sf }\SpecialCharTok{\%\textgreater{}\%} 
 \FunctionTok{filter}\NormalTok{(}\SpecialCharTok{!}\FunctionTok{str\_detect}\NormalTok{(pienemsanas\_vietas\_tips,}\StringTok{"Depozīta"}\NormalTok{))}
\CommentTok{\# rasterize}
\NormalTok{waste\_rast}\OtherTok{=}\FunctionTok{rasterize}\NormalTok{(waste\_sf2,template100)}
\CommentTok{\# raster to 1=Cell of interest, 0=background}
\NormalTok{wastw\_bg}\OtherTok{=}\FunctionTok{cover}\NormalTok{(waste\_rast,nulls100)}

\CommentTok{\# create an egv}
\NormalTok{distegv}\OtherTok{=}\FunctionTok{distance2egv}\NormalTok{(}\AttributeTok{input =}\NormalTok{ wastw\_bg,}
           \AttributeTok{template\_egv =}\NormalTok{ template100,}
           \AttributeTok{values\_as\_one =} \DecValTok{1}\NormalTok{,}
           \AttributeTok{fill\_gaps =} \ConstantTok{TRUE}\NormalTok{, }\AttributeTok{idw\_weight =} \DecValTok{2}\NormalTok{,}
           \AttributeTok{outlocation =} \StringTok{"RasterGrids\_100m/2024/RAW/"}\NormalTok{,}
           \AttributeTok{outfilename =} \StringTok{"Distance\_Waste\_cell.tif"}\NormalTok{,}
           \AttributeTok{layername =} \StringTok{"egv\_095"}\NormalTok{)}
\NormalTok{distegv}

\CommentTok{\# standardisation {-}{-}{-}{-}}
\ControlFlowTok{if}\NormalTok{(}\SpecialCharTok{!}\FunctionTok{require}\NormalTok{(terra)) \{}\FunctionTok{install.packages}\NormalTok{(}\StringTok{"terra"}\NormalTok{); }\FunctionTok{require}\NormalTok{(terra)\}}
\ControlFlowTok{if}\NormalTok{(}\SpecialCharTok{!}\FunctionTok{require}\NormalTok{(tidyverse)) \{}\FunctionTok{install.packages}\NormalTok{(}\StringTok{"tidyverse"}\NormalTok{); }\FunctionTok{require}\NormalTok{(tidyverse)\}}

\NormalTok{nosaukums}\OtherTok{=}\StringTok{"Distance\_Waste\_cell.tif"}
\NormalTok{ielasisanas\_cels}\OtherTok{=}\FunctionTok{paste0}\NormalTok{(}\StringTok{"./RasterGrids\_100m/2024/RAW/"}\NormalTok{,nosaukums)}
\NormalTok{saglabasanas\_cels}\OtherTok{=}\FunctionTok{paste0}\NormalTok{(}\StringTok{"./RasterGrids\_100m/2024/Scaled/"}\NormalTok{,nosaukums)}
\NormalTok{slanis}\OtherTok{=}\FunctionTok{rast}\NormalTok{(ielasisanas\_cels)}
\NormalTok{videjais}\OtherTok{=}\FunctionTok{global}\NormalTok{(slanis,}\AttributeTok{fun=}\StringTok{"mean"}\NormalTok{,}\AttributeTok{na.rm=}\ConstantTok{TRUE}\NormalTok{)}
\NormalTok{centrets}\OtherTok{=}\NormalTok{slanis}\SpecialCharTok{{-}}\NormalTok{videjais[,}\DecValTok{1}\NormalTok{]}
\NormalTok{standartnovirze}\OtherTok{=}\NormalTok{terra}\SpecialCharTok{::}\FunctionTok{global}\NormalTok{(centrets,}\AttributeTok{fun=}\StringTok{"rms"}\NormalTok{,}\AttributeTok{na.rm=}\ConstantTok{TRUE}\NormalTok{)}
\NormalTok{merogots}\OtherTok{=}\NormalTok{centrets}\SpecialCharTok{/}\NormalTok{standartnovirze[,}\DecValTok{1}\NormalTok{]}
\FunctionTok{writeRaster}\NormalTok{(merogots,}
      \AttributeTok{filename=}\NormalTok{saglabasanas\_cels,}
      \AttributeTok{overwrite=}\ConstantTok{TRUE}\NormalTok{)}
\end{Highlighting}
\end{Shaded}

\section{Distance\_Water\_cell}\label{ch06.096}

\textbf{filename:} \texttt{Distance\_Water\_cell.tif}

\textbf{layername:} \texttt{egv\_096}

\textbf{English name:} Distance to Waterbodies, average within the analysis cell (1
ha)

\textbf{Latvian name:} Attālums līdz ūdenstilpēm, vidējais analīzes šūnā (1 ha)

\textbf{Procedure:} Derived from the \hyperref[Ch05.03]{Landscape classification}, with class 200
reclassified as 1 and all others as 0. Processed using the workflow \texttt{egvtools::distance2egv()}. To
prevent possible data loss at edge cells, inverse distance weighted
(power = 2) gap filling is implemented. Finally, the layer is standardised
by subtracting the arithmetic mean and dividing by the root mean squared error.

\begin{Shaded}
\begin{Highlighting}[]
\CommentTok{\# libs {-}{-}{-}{-}}
\ControlFlowTok{if}\NormalTok{(}\SpecialCharTok{!}\FunctionTok{require}\NormalTok{(terra)) \{}\FunctionTok{install.packages}\NormalTok{(}\StringTok{"terra"}\NormalTok{); }\FunctionTok{require}\NormalTok{(terra)\}}
\ControlFlowTok{if}\NormalTok{(}\SpecialCharTok{!}\FunctionTok{require}\NormalTok{(egvtools)) \{remotes}\SpecialCharTok{::}\FunctionTok{install\_github}\NormalTok{(}\StringTok{"aavotins/egvtools"}\NormalTok{); }\FunctionTok{require}\NormalTok{(egvtools)\}}

\CommentTok{\# Distance\_Water\_cell.tif   egv\_96 {-}{-}{-}{-}}
\NormalTok{simple\_landscape}\OtherTok{=}\FunctionTok{rast}\NormalTok{(}\StringTok{"./RasterGrids\_10m/2024/Ainava\_vienk\_mask.tif"}\NormalTok{)}
\NormalTok{water}\OtherTok{=}\FunctionTok{ifel}\NormalTok{(simple\_landscape}\SpecialCharTok{==}\DecValTok{200}\NormalTok{,}\DecValTok{1}\NormalTok{,}\DecValTok{0}\NormalTok{)}
\FunctionTok{plot}\NormalTok{(water)}
\NormalTok{distegv}\OtherTok{=}\FunctionTok{distance2egv}\NormalTok{(}\AttributeTok{input =}\NormalTok{ water,}
           \AttributeTok{template\_egv =}\NormalTok{ template100,}
           \AttributeTok{values\_as\_one =} \DecValTok{1}\NormalTok{,}
           \AttributeTok{fill\_gaps =} \ConstantTok{TRUE}\NormalTok{, }\AttributeTok{idw\_weight =} \DecValTok{2}\NormalTok{,}
           \AttributeTok{outlocation =} \StringTok{"RasterGrids\_100m/2024/RAW/"}\NormalTok{,}
           \AttributeTok{outfilename =} \StringTok{"Distance\_Water\_cell.tif"}\NormalTok{,}
           \AttributeTok{layername =} \StringTok{"egv\_096"}\NormalTok{)}
\NormalTok{distegv}
\FunctionTok{plot}\NormalTok{(}\FunctionTok{rast}\NormalTok{(}\StringTok{"RasterGrids\_100m/2024/RAW/Distance\_Water\_cell.tif"}\NormalTok{))}
\FunctionTok{rm}\NormalTok{(water)}
\FunctionTok{rm}\NormalTok{(distegv)}

\CommentTok{\# standardisation {-}{-}{-}{-}}
\ControlFlowTok{if}\NormalTok{(}\SpecialCharTok{!}\FunctionTok{require}\NormalTok{(terra)) \{}\FunctionTok{install.packages}\NormalTok{(}\StringTok{"terra"}\NormalTok{); }\FunctionTok{require}\NormalTok{(terra)\}}
\ControlFlowTok{if}\NormalTok{(}\SpecialCharTok{!}\FunctionTok{require}\NormalTok{(tidyverse)) \{}\FunctionTok{install.packages}\NormalTok{(}\StringTok{"tidyverse"}\NormalTok{); }\FunctionTok{require}\NormalTok{(tidyverse)\}}

\NormalTok{nosaukums}\OtherTok{=}\StringTok{"Distance\_Water\_cell.tif"}
\NormalTok{ielasisanas\_cels}\OtherTok{=}\FunctionTok{paste0}\NormalTok{(}\StringTok{"./RasterGrids\_100m/2024/RAW/"}\NormalTok{,nosaukums)}
\NormalTok{saglabasanas\_cels}\OtherTok{=}\FunctionTok{paste0}\NormalTok{(}\StringTok{"./RasterGrids\_100m/2024/Scaled/"}\NormalTok{,nosaukums)}
\NormalTok{slanis}\OtherTok{=}\FunctionTok{rast}\NormalTok{(ielasisanas\_cels)}
\NormalTok{videjais}\OtherTok{=}\FunctionTok{global}\NormalTok{(slanis,}\AttributeTok{fun=}\StringTok{"mean"}\NormalTok{,}\AttributeTok{na.rm=}\ConstantTok{TRUE}\NormalTok{)}
\NormalTok{centrets}\OtherTok{=}\NormalTok{slanis}\SpecialCharTok{{-}}\NormalTok{videjais[,}\DecValTok{1}\NormalTok{]}
\NormalTok{standartnovirze}\OtherTok{=}\NormalTok{terra}\SpecialCharTok{::}\FunctionTok{global}\NormalTok{(centrets,}\AttributeTok{fun=}\StringTok{"rms"}\NormalTok{,}\AttributeTok{na.rm=}\ConstantTok{TRUE}\NormalTok{)}
\NormalTok{merogots}\OtherTok{=}\NormalTok{centrets}\SpecialCharTok{/}\NormalTok{standartnovirze[,}\DecValTok{1}\NormalTok{]}
\FunctionTok{writeRaster}\NormalTok{(merogots,}
      \AttributeTok{filename=}\NormalTok{saglabasanas\_cels,}
      \AttributeTok{overwrite=}\ConstantTok{TRUE}\NormalTok{)}
\end{Highlighting}
\end{Shaded}

\section{Distance\_WaterInside\_cell}\label{ch06.097}

\textbf{filename:} \texttt{Distance\_WaterInside\_cell.tif}

\textbf{layername:} \texttt{egv\_097}

\textbf{English name:} Distance to Waterbody Edge Inside Waterbody, average within
the analysis cell (1 ha)

\textbf{Latvian name:} Attālums līdz ūdenstilpes malai tās iekšienē, vidējais
analīzes šūnā (1 ha)

\textbf{Procedure:} Derived from the \hyperref[Ch05.03]{Landscape classification}, with class 200
reclassified as 0 and all others as 1. Processed using the workflow \texttt{egvtools::distance2egv()}. To
prevent possible data loss at edge cells, inverse distance weighted
(power = 2) gap filling is implemented. Finally, the layer is standardised
by subtracting the arithmetic mean and dividing by the root mean squared error.

\begin{Shaded}
\begin{Highlighting}[]
\CommentTok{\# libs {-}{-}{-}{-}}
\ControlFlowTok{if}\NormalTok{(}\SpecialCharTok{!}\FunctionTok{require}\NormalTok{(terra)) \{}\FunctionTok{install.packages}\NormalTok{(}\StringTok{"terra"}\NormalTok{); }\FunctionTok{require}\NormalTok{(terra)\}}
\ControlFlowTok{if}\NormalTok{(}\SpecialCharTok{!}\FunctionTok{require}\NormalTok{(egvtools)) \{remotes}\SpecialCharTok{::}\FunctionTok{install\_github}\NormalTok{(}\StringTok{"aavotins/egvtools"}\NormalTok{); }\FunctionTok{require}\NormalTok{(egvtools)\}}

\CommentTok{\# Distance\_WaterInside\_cell.tif egv\_97 {-}{-}{-}{-}}
\NormalTok{simple\_landscape}\OtherTok{=}\FunctionTok{rast}\NormalTok{(}\StringTok{"./RasterGrids\_10m/2024/Ainava\_vienk\_mask.tif"}\NormalTok{)}
\NormalTok{water\_outside}\OtherTok{=}\FunctionTok{ifel}\NormalTok{(simple\_landscape}\SpecialCharTok{==}\DecValTok{200}\NormalTok{,}\DecValTok{0}\NormalTok{,}\DecValTok{1}\NormalTok{)}
\FunctionTok{plot}\NormalTok{(water\_outside)}
\NormalTok{distegv}\OtherTok{=}\FunctionTok{distance2egv}\NormalTok{(}\AttributeTok{input =}\NormalTok{ water\_outside,}
           \AttributeTok{template\_egv =}\NormalTok{ template100,}
           \AttributeTok{values\_as\_one =} \DecValTok{1}\NormalTok{,}
           \AttributeTok{fill\_gaps =} \ConstantTok{TRUE}\NormalTok{, }\AttributeTok{idw\_weight =} \DecValTok{2}\NormalTok{,}
           \AttributeTok{outlocation =} \StringTok{"RasterGrids\_100m/2024/RAW/"}\NormalTok{,}
           \AttributeTok{outfilename =} \StringTok{"Distance\_WaterInside\_cell.tif"}\NormalTok{,}
           \AttributeTok{layername =} \StringTok{"egv\_097"}\NormalTok{)}
\NormalTok{distegv}
\FunctionTok{plot}\NormalTok{(}\FunctionTok{rast}\NormalTok{(}\StringTok{"RasterGrids\_100m/2024/RAW/Distance\_WaterInside\_cell.tif"}\NormalTok{))}
\FunctionTok{rm}\NormalTok{(water\_outside)}
\FunctionTok{rm}\NormalTok{(distegv)}

\CommentTok{\# standardisation {-}{-}{-}{-}}
\ControlFlowTok{if}\NormalTok{(}\SpecialCharTok{!}\FunctionTok{require}\NormalTok{(terra)) \{}\FunctionTok{install.packages}\NormalTok{(}\StringTok{"terra"}\NormalTok{); }\FunctionTok{require}\NormalTok{(terra)\}}
\ControlFlowTok{if}\NormalTok{(}\SpecialCharTok{!}\FunctionTok{require}\NormalTok{(tidyverse)) \{}\FunctionTok{install.packages}\NormalTok{(}\StringTok{"tidyverse"}\NormalTok{); }\FunctionTok{require}\NormalTok{(tidyverse)\}}

\NormalTok{nosaukums}\OtherTok{=}\StringTok{"Distance\_WaterInside\_cell.tif"}
\NormalTok{ielasisanas\_cels}\OtherTok{=}\FunctionTok{paste0}\NormalTok{(}\StringTok{"./RasterGrids\_100m/2024/RAW/"}\NormalTok{,nosaukums)}
\NormalTok{saglabasanas\_cels}\OtherTok{=}\FunctionTok{paste0}\NormalTok{(}\StringTok{"./RasterGrids\_100m/2024/Scaled/"}\NormalTok{,nosaukums)}
\NormalTok{slanis}\OtherTok{=}\FunctionTok{rast}\NormalTok{(ielasisanas\_cels)}
\NormalTok{videjais}\OtherTok{=}\FunctionTok{global}\NormalTok{(slanis,}\AttributeTok{fun=}\StringTok{"mean"}\NormalTok{,}\AttributeTok{na.rm=}\ConstantTok{TRUE}\NormalTok{)}
\NormalTok{centrets}\OtherTok{=}\NormalTok{slanis}\SpecialCharTok{{-}}\NormalTok{videjais[,}\DecValTok{1}\NormalTok{]}
\NormalTok{standartnovirze}\OtherTok{=}\NormalTok{terra}\SpecialCharTok{::}\FunctionTok{global}\NormalTok{(centrets,}\AttributeTok{fun=}\StringTok{"rms"}\NormalTok{,}\AttributeTok{na.rm=}\ConstantTok{TRUE}\NormalTok{)}
\NormalTok{merogots}\OtherTok{=}\NormalTok{centrets}\SpecialCharTok{/}\NormalTok{standartnovirze[,}\DecValTok{1}\NormalTok{]}
\FunctionTok{writeRaster}\NormalTok{(merogots,}
      \AttributeTok{filename=}\NormalTok{saglabasanas\_cels,}
      \AttributeTok{overwrite=}\ConstantTok{TRUE}\NormalTok{)}
\end{Highlighting}
\end{Shaded}

\section{Diversity\_Farmland\_r500}\label{ch06.098}

\textbf{filename:} \texttt{Diversity\_Farmland\_r500.tif}

\textbf{layername:} \texttt{egv\_098}

\textbf{English name:} Average farmland class α-diversity of 500 m grid cells within
the 0.5 km landscape

\textbf{Latvian name:} Vidējā lauku ainavas klašu 500 m šūnu α-daudzveidība 0,5 km
ainavā

\textbf{Procedure:} Derived from the \hyperref[Ch05.04]{Landscape diversity}, more precisely
\hyperref[Ch05.04.01]{Farmland diversity}. The average value of 25 ha cells diversity index
values is calculated using the workflow \texttt{egvtools::radius\_function()}. To
prevent possible data loss at edge cells, inverse distance weighted
(power = 2) gap filling is implemented. File is written twice, to ensure
layername. Finally, the layer is standardised
by subtracting the arithmetic mean and dividing by the root mean squared error.

\begin{Shaded}
\begin{Highlighting}[]
\CommentTok{\# libs {-}{-}{-}{-}}
\ControlFlowTok{if}\NormalTok{(}\SpecialCharTok{!}\FunctionTok{require}\NormalTok{(egvtools)) \{remotes}\SpecialCharTok{::}\FunctionTok{install\_github}\NormalTok{(}\StringTok{"aavotins/egvtools"}\NormalTok{); }\FunctionTok{require}\NormalTok{(egvtools)\}}
\ControlFlowTok{if}\NormalTok{(}\SpecialCharTok{!}\FunctionTok{require}\NormalTok{(terra)) \{}\FunctionTok{install.packages}\NormalTok{(}\StringTok{"terra"}\NormalTok{); }\FunctionTok{require}\NormalTok{(terra)\}}

\CommentTok{\# templates {-}{-}{-}{-}}
\NormalTok{template100}\OtherTok{=}\FunctionTok{rast}\NormalTok{(}\StringTok{"./Templates/TemplateRasters/LV100m\_10km.tif"}\NormalTok{)}

\CommentTok{\# radii}
\FunctionTok{radius\_function}\NormalTok{(}
 \AttributeTok{kvadrati\_path =} \StringTok{"./Templates/TemplateGrids/tiles/"}\NormalTok{,}
 \AttributeTok{radii\_path   =} \StringTok{"./Templates/TemplateGridPoints/tiles/"}\NormalTok{,}
 \AttributeTok{tikls100\_path =} \StringTok{"./Templates/TemplateGrids/tikls100\_sauzeme.parquet"}\NormalTok{,}
 \AttributeTok{template\_path =} \StringTok{"./Templates/TemplateRasters/LV100m\_10km.tif"}\NormalTok{,}
 \AttributeTok{input\_layers  =} \FunctionTok{c}\NormalTok{(}\StringTok{"./RasterGrids\_500m/2024/Diversity\_Farmland\_500x.tif"}\NormalTok{),}
 \AttributeTok{layer\_prefixes =} \FunctionTok{c}\NormalTok{(}\StringTok{"Diversity\_Farmland"}\NormalTok{),}
 \AttributeTok{output\_dir   =} \StringTok{"./RasterGrids\_100m/2024/RAW/"}\NormalTok{,}
 \AttributeTok{n\_workers   =} \DecValTok{12}\NormalTok{,}
 \AttributeTok{radii     =} \FunctionTok{c}\NormalTok{(}\StringTok{"r500"}\NormalTok{),}
 \AttributeTok{radius\_mode  =} \StringTok{"sparse"}\NormalTok{,}
 \AttributeTok{extract\_fun  =} \StringTok{"mean"}\NormalTok{,}
 \AttributeTok{fill\_missing  =} \ConstantTok{TRUE}\NormalTok{,}
 \AttributeTok{IDW\_weight   =} \DecValTok{2}\NormalTok{,}
 \AttributeTok{future\_max\_size =} \DecValTok{5} \SpecialCharTok{*} \DecValTok{1024}\SpecialCharTok{\^{}}\DecValTok{3}\NormalTok{)}

\CommentTok{\# Diversity\_Farmland\_r500.tif   egv\_98}
\NormalTok{slanis}\OtherTok{=}\FunctionTok{rast}\NormalTok{(}\StringTok{"./RasterGrids\_100m/2024/RAW/Diversity\_Farmland\_r500.tif"}\NormalTok{)}
\FunctionTok{names}\NormalTok{(slanis)}\OtherTok{=}\StringTok{"egv\_098"}
\NormalTok{slanis2}\OtherTok{=}\FunctionTok{project}\NormalTok{(slanis,template100)}
\FunctionTok{writeRaster}\NormalTok{(slanis2,}
      \StringTok{"./RasterGrids\_100m/2024/RAW/Diversity\_Farmland\_r500.tif"}\NormalTok{,}
      \AttributeTok{overwrite=}\ConstantTok{TRUE}\NormalTok{)}

\CommentTok{\# standardisation {-}{-}{-}{-}}
\ControlFlowTok{if}\NormalTok{(}\SpecialCharTok{!}\FunctionTok{require}\NormalTok{(terra)) \{}\FunctionTok{install.packages}\NormalTok{(}\StringTok{"terra"}\NormalTok{); }\FunctionTok{require}\NormalTok{(terra)\}}
\ControlFlowTok{if}\NormalTok{(}\SpecialCharTok{!}\FunctionTok{require}\NormalTok{(tidyverse)) \{}\FunctionTok{install.packages}\NormalTok{(}\StringTok{"tidyverse"}\NormalTok{); }\FunctionTok{require}\NormalTok{(tidyverse)\}}

\NormalTok{nosaukums}\OtherTok{=}\StringTok{"Diversity\_Farmland\_r500.tif"}
\NormalTok{ielasisanas\_cels}\OtherTok{=}\FunctionTok{paste0}\NormalTok{(}\StringTok{"./RasterGrids\_100m/2024/RAW/"}\NormalTok{,nosaukums)}
\NormalTok{saglabasanas\_cels}\OtherTok{=}\FunctionTok{paste0}\NormalTok{(}\StringTok{"./RasterGrids\_100m/2024/Scaled/"}\NormalTok{,nosaukums)}
\NormalTok{slanis}\OtherTok{=}\FunctionTok{rast}\NormalTok{(ielasisanas\_cels)}
\NormalTok{videjais}\OtherTok{=}\FunctionTok{global}\NormalTok{(slanis,}\AttributeTok{fun=}\StringTok{"mean"}\NormalTok{,}\AttributeTok{na.rm=}\ConstantTok{TRUE}\NormalTok{)}
\NormalTok{centrets}\OtherTok{=}\NormalTok{slanis}\SpecialCharTok{{-}}\NormalTok{videjais[,}\DecValTok{1}\NormalTok{]}
\NormalTok{standartnovirze}\OtherTok{=}\NormalTok{terra}\SpecialCharTok{::}\FunctionTok{global}\NormalTok{(centrets,}\AttributeTok{fun=}\StringTok{"rms"}\NormalTok{,}\AttributeTok{na.rm=}\ConstantTok{TRUE}\NormalTok{)}
\NormalTok{merogots}\OtherTok{=}\NormalTok{centrets}\SpecialCharTok{/}\NormalTok{standartnovirze[,}\DecValTok{1}\NormalTok{]}
\FunctionTok{writeRaster}\NormalTok{(merogots,}
      \AttributeTok{filename=}\NormalTok{saglabasanas\_cels,}
      \AttributeTok{overwrite=}\ConstantTok{TRUE}\NormalTok{)}
\end{Highlighting}
\end{Shaded}

\section{Diversity\_Farmland\_r1250}\label{ch06.099}

\textbf{filename:} \texttt{Diversity\_Farmland\_r1250.tif}

\textbf{layername:} \texttt{egv\_099}

\textbf{English name:} Average farmland class α-diversity of 500 m grid cells within
the 1.25 km landscape

\textbf{Latvian name:} Vidējā lauku ainavas klašu 500 m šūnu α-daudzveidība 1,25 km
ainavā

\textbf{Procedure:} Derived from the \hyperref[Ch05.04]{Landscape diversity}, more precisely
\hyperref[Ch05.04.01]{Farmland diversity}. The average value of 25 ha cells diversity index
values is calculated using the workflow \texttt{egvtools::radius\_function()}. To
prevent possible data loss at edge cells, inverse distance weighted
(power = 2) gap filling is implemented. File is written twice, to ensure
layername. Finally, the layer is standardised
by subtracting the arithmetic mean and dividing by the root mean squared error.

\begin{Shaded}
\begin{Highlighting}[]
\CommentTok{\# libs {-}{-}{-}{-}}
\ControlFlowTok{if}\NormalTok{(}\SpecialCharTok{!}\FunctionTok{require}\NormalTok{(egvtools)) \{remotes}\SpecialCharTok{::}\FunctionTok{install\_github}\NormalTok{(}\StringTok{"aavotins/egvtools"}\NormalTok{); }\FunctionTok{require}\NormalTok{(egvtools)\}}
\ControlFlowTok{if}\NormalTok{(}\SpecialCharTok{!}\FunctionTok{require}\NormalTok{(terra)) \{}\FunctionTok{install.packages}\NormalTok{(}\StringTok{"terra"}\NormalTok{); }\FunctionTok{require}\NormalTok{(terra)\}}

\CommentTok{\# templates {-}{-}{-}{-}}
\NormalTok{template100}\OtherTok{=}\FunctionTok{rast}\NormalTok{(}\StringTok{"./Templates/TemplateRasters/LV100m\_10km.tif"}\NormalTok{)}

\CommentTok{\# radii}
\FunctionTok{radius\_function}\NormalTok{(}
 \AttributeTok{kvadrati\_path =} \StringTok{"./Templates/TemplateGrids/tiles/"}\NormalTok{,}
 \AttributeTok{radii\_path   =} \StringTok{"./Templates/TemplateGridPoints/tiles/"}\NormalTok{,}
 \AttributeTok{tikls100\_path =} \StringTok{"./Templates/TemplateGrids/tikls100\_sauzeme.parquet"}\NormalTok{,}
 \AttributeTok{template\_path =} \StringTok{"./Templates/TemplateRasters/LV100m\_10km.tif"}\NormalTok{,}
 \AttributeTok{input\_layers  =} \FunctionTok{c}\NormalTok{(}\StringTok{"./RasterGrids\_500m/2024/Diversity\_Farmland\_500x.tif"}\NormalTok{),}
 \AttributeTok{layer\_prefixes =} \FunctionTok{c}\NormalTok{(}\StringTok{"Diversity\_Farmland"}\NormalTok{),}
 \AttributeTok{output\_dir   =} \StringTok{"./RasterGrids\_100m/2024/RAW/"}\NormalTok{,}
 \AttributeTok{n\_workers   =} \DecValTok{12}\NormalTok{,}
 \AttributeTok{radii     =} \FunctionTok{c}\NormalTok{(}\StringTok{"r1250"}\NormalTok{),}
 \AttributeTok{radius\_mode  =} \StringTok{"sparse"}\NormalTok{,}
 \AttributeTok{extract\_fun  =} \StringTok{"mean"}\NormalTok{,}
 \AttributeTok{fill\_missing  =} \ConstantTok{TRUE}\NormalTok{,}
 \AttributeTok{IDW\_weight   =} \DecValTok{2}\NormalTok{,}
 \AttributeTok{future\_max\_size =} \DecValTok{5} \SpecialCharTok{*} \DecValTok{1024}\SpecialCharTok{\^{}}\DecValTok{3}\NormalTok{)}

\CommentTok{\# Diversity\_Farmland\_r1250.tif  egv\_99}
\NormalTok{slanis}\OtherTok{=}\FunctionTok{rast}\NormalTok{(}\StringTok{"./RasterGrids\_100m/2024/RAW/Diversity\_Farmland\_r1250.tif"}\NormalTok{)}
\FunctionTok{names}\NormalTok{(slanis)}\OtherTok{=}\StringTok{"egv\_099"}
\NormalTok{slanis2}\OtherTok{=}\FunctionTok{project}\NormalTok{(slanis,template100)}
\FunctionTok{writeRaster}\NormalTok{(slanis2,}
      \StringTok{"./RasterGrids\_100m/2024/RAW/Diversity\_Farmland\_r1250.tif"}\NormalTok{,}
      \AttributeTok{overwrite=}\ConstantTok{TRUE}\NormalTok{)}

\CommentTok{\# standardisation {-}{-}{-}{-}}
\ControlFlowTok{if}\NormalTok{(}\SpecialCharTok{!}\FunctionTok{require}\NormalTok{(terra)) \{}\FunctionTok{install.packages}\NormalTok{(}\StringTok{"terra"}\NormalTok{); }\FunctionTok{require}\NormalTok{(terra)\}}
\ControlFlowTok{if}\NormalTok{(}\SpecialCharTok{!}\FunctionTok{require}\NormalTok{(tidyverse)) \{}\FunctionTok{install.packages}\NormalTok{(}\StringTok{"tidyverse"}\NormalTok{); }\FunctionTok{require}\NormalTok{(tidyverse)\}}

\NormalTok{nosaukums}\OtherTok{=}\StringTok{"Diversity\_Farmland\_r1250.tif"}
\NormalTok{ielasisanas\_cels}\OtherTok{=}\FunctionTok{paste0}\NormalTok{(}\StringTok{"./RasterGrids\_100m/2024/RAW/"}\NormalTok{,nosaukums)}
\NormalTok{saglabasanas\_cels}\OtherTok{=}\FunctionTok{paste0}\NormalTok{(}\StringTok{"./RasterGrids\_100m/2024/Scaled/"}\NormalTok{,nosaukums)}
\NormalTok{slanis}\OtherTok{=}\FunctionTok{rast}\NormalTok{(ielasisanas\_cels)}
\NormalTok{videjais}\OtherTok{=}\FunctionTok{global}\NormalTok{(slanis,}\AttributeTok{fun=}\StringTok{"mean"}\NormalTok{,}\AttributeTok{na.rm=}\ConstantTok{TRUE}\NormalTok{)}
\NormalTok{centrets}\OtherTok{=}\NormalTok{slanis}\SpecialCharTok{{-}}\NormalTok{videjais[,}\DecValTok{1}\NormalTok{]}
\NormalTok{standartnovirze}\OtherTok{=}\NormalTok{terra}\SpecialCharTok{::}\FunctionTok{global}\NormalTok{(centrets,}\AttributeTok{fun=}\StringTok{"rms"}\NormalTok{,}\AttributeTok{na.rm=}\ConstantTok{TRUE}\NormalTok{)}
\NormalTok{merogots}\OtherTok{=}\NormalTok{centrets}\SpecialCharTok{/}\NormalTok{standartnovirze[,}\DecValTok{1}\NormalTok{]}
\FunctionTok{writeRaster}\NormalTok{(merogots,}
      \AttributeTok{filename=}\NormalTok{saglabasanas\_cels,}
      \AttributeTok{overwrite=}\ConstantTok{TRUE}\NormalTok{)}
\end{Highlighting}
\end{Shaded}

\section{Diversity\_Farmland\_r3000}\label{ch06.100}

\textbf{filename:} \texttt{Diversity\_Farmland\_r3000.tif}

\textbf{layername:} \texttt{egv\_100}

\textbf{English name:} Average farmland class α-diversity of 500 m grid cells within
the 3 km landscape

\textbf{Latvian name:} Vidējā lauku ainavas klašu 500 m šūnu α-daudzveidība 3 km
ainavā

\textbf{Procedure:} Derived from the \hyperref[Ch05.04]{Landscape diversity}, more precisely
\hyperref[Ch05.04.01]{Farmland diversity}. The average value of 25 ha cells diversity index
values is calculated using the workflow \texttt{egvtools::radius\_function()}. To
prevent possible data loss at edge cells, inverse distance weighted
(power = 2) gap filling is implemented. File is written twice, to ensure
layername. Finally, the layer is standardised
by subtracting the arithmetic mean and dividing by the root mean squared error.

\begin{Shaded}
\begin{Highlighting}[]
\CommentTok{\# libs {-}{-}{-}{-}}
\ControlFlowTok{if}\NormalTok{(}\SpecialCharTok{!}\FunctionTok{require}\NormalTok{(egvtools)) \{remotes}\SpecialCharTok{::}\FunctionTok{install\_github}\NormalTok{(}\StringTok{"aavotins/egvtools"}\NormalTok{); }\FunctionTok{require}\NormalTok{(egvtools)\}}
\ControlFlowTok{if}\NormalTok{(}\SpecialCharTok{!}\FunctionTok{require}\NormalTok{(terra)) \{}\FunctionTok{install.packages}\NormalTok{(}\StringTok{"terra"}\NormalTok{); }\FunctionTok{require}\NormalTok{(terra)\}}

\CommentTok{\# templates {-}{-}{-}{-}}
\NormalTok{template100}\OtherTok{=}\FunctionTok{rast}\NormalTok{(}\StringTok{"./Templates/TemplateRasters/LV100m\_10km.tif"}\NormalTok{)}

\CommentTok{\# radii}
\FunctionTok{radius\_function}\NormalTok{(}
 \AttributeTok{kvadrati\_path =} \StringTok{"./Templates/TemplateGrids/tiles/"}\NormalTok{,}
 \AttributeTok{radii\_path   =} \StringTok{"./Templates/TemplateGridPoints/tiles/"}\NormalTok{,}
 \AttributeTok{tikls100\_path =} \StringTok{"./Templates/TemplateGrids/tikls100\_sauzeme.parquet"}\NormalTok{,}
 \AttributeTok{template\_path =} \StringTok{"./Templates/TemplateRasters/LV100m\_10km.tif"}\NormalTok{,}
 \AttributeTok{input\_layers  =} \FunctionTok{c}\NormalTok{(}\StringTok{"./RasterGrids\_500m/2024/Diversity\_Farmland\_500x.tif"}\NormalTok{),}
 \AttributeTok{layer\_prefixes =} \FunctionTok{c}\NormalTok{(}\StringTok{"Diversity\_Farmland"}\NormalTok{),}
 \AttributeTok{output\_dir   =} \StringTok{"./RasterGrids\_100m/2024/RAW/"}\NormalTok{,}
 \AttributeTok{n\_workers   =} \DecValTok{12}\NormalTok{,}
 \AttributeTok{radii     =} \FunctionTok{c}\NormalTok{(}\StringTok{"r3000"}\NormalTok{),}
 \AttributeTok{radius\_mode  =} \StringTok{"sparse"}\NormalTok{,}
 \AttributeTok{extract\_fun  =} \StringTok{"mean"}\NormalTok{,}
 \AttributeTok{fill\_missing  =} \ConstantTok{TRUE}\NormalTok{,}
 \AttributeTok{IDW\_weight   =} \DecValTok{2}\NormalTok{,}
 \AttributeTok{future\_max\_size =} \DecValTok{5} \SpecialCharTok{*} \DecValTok{1024}\SpecialCharTok{\^{}}\DecValTok{3}\NormalTok{)}

\CommentTok{\# Diversity\_Farmland\_r3000.tif  egv\_100}
\NormalTok{slanis}\OtherTok{=}\FunctionTok{rast}\NormalTok{(}\StringTok{"./RasterGrids\_100m/2024/RAW/Diversity\_Farmland\_r3000.tif"}\NormalTok{)}
\FunctionTok{names}\NormalTok{(slanis)}\OtherTok{=}\StringTok{"egv\_100"}
\NormalTok{slanis2}\OtherTok{=}\FunctionTok{project}\NormalTok{(slanis,template100)}
\FunctionTok{writeRaster}\NormalTok{(slanis2,}
      \StringTok{"./RasterGrids\_100m/2024/RAW/Diversity\_Farmland\_r3000.tif"}\NormalTok{,}
      \AttributeTok{overwrite=}\ConstantTok{TRUE}\NormalTok{)}

\CommentTok{\# standardisation {-}{-}{-}{-}}
\ControlFlowTok{if}\NormalTok{(}\SpecialCharTok{!}\FunctionTok{require}\NormalTok{(terra)) \{}\FunctionTok{install.packages}\NormalTok{(}\StringTok{"terra"}\NormalTok{); }\FunctionTok{require}\NormalTok{(terra)\}}
\ControlFlowTok{if}\NormalTok{(}\SpecialCharTok{!}\FunctionTok{require}\NormalTok{(tidyverse)) \{}\FunctionTok{install.packages}\NormalTok{(}\StringTok{"tidyverse"}\NormalTok{); }\FunctionTok{require}\NormalTok{(tidyverse)\}}

\NormalTok{nosaukums}\OtherTok{=}\StringTok{"Diversity\_Farmland\_r3000.tif"}
\NormalTok{ielasisanas\_cels}\OtherTok{=}\FunctionTok{paste0}\NormalTok{(}\StringTok{"./RasterGrids\_100m/2024/RAW/"}\NormalTok{,nosaukums)}
\NormalTok{saglabasanas\_cels}\OtherTok{=}\FunctionTok{paste0}\NormalTok{(}\StringTok{"./RasterGrids\_100m/2024/Scaled/"}\NormalTok{,nosaukums)}
\NormalTok{slanis}\OtherTok{=}\FunctionTok{rast}\NormalTok{(ielasisanas\_cels)}
\NormalTok{videjais}\OtherTok{=}\FunctionTok{global}\NormalTok{(slanis,}\AttributeTok{fun=}\StringTok{"mean"}\NormalTok{,}\AttributeTok{na.rm=}\ConstantTok{TRUE}\NormalTok{)}
\NormalTok{centrets}\OtherTok{=}\NormalTok{slanis}\SpecialCharTok{{-}}\NormalTok{videjais[,}\DecValTok{1}\NormalTok{]}
\NormalTok{standartnovirze}\OtherTok{=}\NormalTok{terra}\SpecialCharTok{::}\FunctionTok{global}\NormalTok{(centrets,}\AttributeTok{fun=}\StringTok{"rms"}\NormalTok{,}\AttributeTok{na.rm=}\ConstantTok{TRUE}\NormalTok{)}
\NormalTok{merogots}\OtherTok{=}\NormalTok{centrets}\SpecialCharTok{/}\NormalTok{standartnovirze[,}\DecValTok{1}\NormalTok{]}
\FunctionTok{writeRaster}\NormalTok{(merogots,}
      \AttributeTok{filename=}\NormalTok{saglabasanas\_cels,}
      \AttributeTok{overwrite=}\ConstantTok{TRUE}\NormalTok{)}
\end{Highlighting}
\end{Shaded}

\section{Diversity\_Farmland\_r10000}\label{ch06.101}

\textbf{filename:} \texttt{Diversity\_Farmland\_r10000.tif}

\textbf{layername:} \texttt{egv\_101}

\textbf{English name:} Average farmland class α-diversity of 500 m grid cells within
the 10 km landscape

\textbf{Latvian name:} Vidējā lauku ainavas klašu 500 m šūnu α-daudzveidība 10 km
ainavā

\textbf{Procedure:} Derived from the \hyperref[Ch05.04]{Landscape diversity}, more precisely
\hyperref[Ch05.04.01]{Farmland diversity}. The average value of 25 ha cells diversity index
values is calculated using the workflow \texttt{egvtools::radius\_function()}. To
prevent possible data loss at edge cells, inverse distance weighted
(power = 2) gap filling is implemented. File is written twice, to ensure
layername. Finally, the layer is standardised
by subtracting the arithmetic mean and dividing by the root mean squared error.

\begin{Shaded}
\begin{Highlighting}[]
\CommentTok{\# libs {-}{-}{-}{-}}
\ControlFlowTok{if}\NormalTok{(}\SpecialCharTok{!}\FunctionTok{require}\NormalTok{(egvtools)) \{remotes}\SpecialCharTok{::}\FunctionTok{install\_github}\NormalTok{(}\StringTok{"aavotins/egvtools"}\NormalTok{); }\FunctionTok{require}\NormalTok{(egvtools)\}}
\ControlFlowTok{if}\NormalTok{(}\SpecialCharTok{!}\FunctionTok{require}\NormalTok{(terra)) \{}\FunctionTok{install.packages}\NormalTok{(}\StringTok{"terra"}\NormalTok{); }\FunctionTok{require}\NormalTok{(terra)\}}

\CommentTok{\# templates {-}{-}{-}{-}}
\NormalTok{template100}\OtherTok{=}\FunctionTok{rast}\NormalTok{(}\StringTok{"./Templates/TemplateRasters/LV100m\_10km.tif"}\NormalTok{)}

\CommentTok{\# radii}
\FunctionTok{radius\_function}\NormalTok{(}
 \AttributeTok{kvadrati\_path =} \StringTok{"./Templates/TemplateGrids/tiles/"}\NormalTok{,}
 \AttributeTok{radii\_path   =} \StringTok{"./Templates/TemplateGridPoints/tiles/"}\NormalTok{,}
 \AttributeTok{tikls100\_path =} \StringTok{"./Templates/TemplateGrids/tikls100\_sauzeme.parquet"}\NormalTok{,}
 \AttributeTok{template\_path =} \StringTok{"./Templates/TemplateRasters/LV100m\_10km.tif"}\NormalTok{,}
 \AttributeTok{input\_layers  =} \FunctionTok{c}\NormalTok{(}\StringTok{"./RasterGrids\_500m/2024/Diversity\_Farmland\_500x.tif"}\NormalTok{),}
 \AttributeTok{layer\_prefixes =} \FunctionTok{c}\NormalTok{(}\StringTok{"Diversity\_Farmland"}\NormalTok{),}
 \AttributeTok{output\_dir   =} \StringTok{"./RasterGrids\_100m/2024/RAW/"}\NormalTok{,}
 \AttributeTok{n\_workers   =} \DecValTok{12}\NormalTok{,}
 \AttributeTok{radii     =} \FunctionTok{c}\NormalTok{(}\StringTok{"r10000"}\NormalTok{),}
 \AttributeTok{radius\_mode  =} \StringTok{"sparse"}\NormalTok{,}
 \AttributeTok{extract\_fun  =} \StringTok{"mean"}\NormalTok{,}
 \AttributeTok{fill\_missing  =} \ConstantTok{TRUE}\NormalTok{,}
 \AttributeTok{IDW\_weight   =} \DecValTok{2}\NormalTok{,}
 \AttributeTok{future\_max\_size =} \DecValTok{5} \SpecialCharTok{*} \DecValTok{1024}\SpecialCharTok{\^{}}\DecValTok{3}\NormalTok{)}

\CommentTok{\# Diversity\_Farmland\_r10000.tif egv\_101}
\NormalTok{slanis}\OtherTok{=}\FunctionTok{rast}\NormalTok{(}\StringTok{"./RasterGrids\_100m/2024/RAW/Diversity\_Farmland\_r10000.tif"}\NormalTok{)}
\FunctionTok{names}\NormalTok{(slanis)}\OtherTok{=}\StringTok{"egv\_101"}
\NormalTok{slanis2}\OtherTok{=}\FunctionTok{project}\NormalTok{(slanis,template100)}
\FunctionTok{writeRaster}\NormalTok{(slanis2,}
      \StringTok{"./RasterGrids\_100m/2024/RAW/Diversity\_Farmland\_r10000.tif"}\NormalTok{,}
      \AttributeTok{overwrite=}\ConstantTok{TRUE}\NormalTok{)}

\CommentTok{\# standardisation {-}{-}{-}{-}}
\ControlFlowTok{if}\NormalTok{(}\SpecialCharTok{!}\FunctionTok{require}\NormalTok{(terra)) \{}\FunctionTok{install.packages}\NormalTok{(}\StringTok{"terra"}\NormalTok{); }\FunctionTok{require}\NormalTok{(terra)\}}
\ControlFlowTok{if}\NormalTok{(}\SpecialCharTok{!}\FunctionTok{require}\NormalTok{(tidyverse)) \{}\FunctionTok{install.packages}\NormalTok{(}\StringTok{"tidyverse"}\NormalTok{); }\FunctionTok{require}\NormalTok{(tidyverse)\}}

\NormalTok{nosaukums}\OtherTok{=}\StringTok{"Diversity\_Farmland\_r10000.tif"}
\NormalTok{ielasisanas\_cels}\OtherTok{=}\FunctionTok{paste0}\NormalTok{(}\StringTok{"./RasterGrids\_100m/2024/RAW/"}\NormalTok{,nosaukums)}
\NormalTok{saglabasanas\_cels}\OtherTok{=}\FunctionTok{paste0}\NormalTok{(}\StringTok{"./RasterGrids\_100m/2024/Scaled/"}\NormalTok{,nosaukums)}
\NormalTok{slanis}\OtherTok{=}\FunctionTok{rast}\NormalTok{(ielasisanas\_cels)}
\NormalTok{videjais}\OtherTok{=}\FunctionTok{global}\NormalTok{(slanis,}\AttributeTok{fun=}\StringTok{"mean"}\NormalTok{,}\AttributeTok{na.rm=}\ConstantTok{TRUE}\NormalTok{)}
\NormalTok{centrets}\OtherTok{=}\NormalTok{slanis}\SpecialCharTok{{-}}\NormalTok{videjais[,}\DecValTok{1}\NormalTok{]}
\NormalTok{standartnovirze}\OtherTok{=}\NormalTok{terra}\SpecialCharTok{::}\FunctionTok{global}\NormalTok{(centrets,}\AttributeTok{fun=}\StringTok{"rms"}\NormalTok{,}\AttributeTok{na.rm=}\ConstantTok{TRUE}\NormalTok{)}
\NormalTok{merogots}\OtherTok{=}\NormalTok{centrets}\SpecialCharTok{/}\NormalTok{standartnovirze[,}\DecValTok{1}\NormalTok{]}
\FunctionTok{writeRaster}\NormalTok{(merogots,}
      \AttributeTok{filename=}\NormalTok{saglabasanas\_cels,}
      \AttributeTok{overwrite=}\ConstantTok{TRUE}\NormalTok{)}
\end{Highlighting}
\end{Shaded}

\section{Diversity\_Forest\_r500}\label{ch06.102}

\textbf{filename:} \texttt{Diversity\_Forest\_r500.tif}

\textbf{layername:} \texttt{egv\_102}

\textbf{English name:} Average forest class α-diversity of 500 m grid cells within
the 0.5 km landscape

\textbf{Latvian name:} Vidējā mežu ainavas klašu 500 m šūnu α-daudzveidība 0,5 km
ainavā

\textbf{Procedure:} Derived from the \hyperref[Ch05.04]{Landscape diversity}, more precisely
\hyperref[Ch05.04.02]{Forest diversity}. The average value of 25 ha cells diversity index
values is calculated using the workflow \texttt{egvtools::radius\_function()}. To
prevent possible data loss at edge cells, inverse distance weighted
(power = 2) gap filling is implemented. File is written twice, to ensure
layername. Finally, the layer is standardised
by subtracting the arithmetic mean and dividing by the root mean squared error.

\begin{Shaded}
\begin{Highlighting}[]
\CommentTok{\# libs {-}{-}{-}{-}}
\ControlFlowTok{if}\NormalTok{(}\SpecialCharTok{!}\FunctionTok{require}\NormalTok{(egvtools)) \{remotes}\SpecialCharTok{::}\FunctionTok{install\_github}\NormalTok{(}\StringTok{"aavotins/egvtools"}\NormalTok{); }\FunctionTok{require}\NormalTok{(egvtools)\}}
\ControlFlowTok{if}\NormalTok{(}\SpecialCharTok{!}\FunctionTok{require}\NormalTok{(terra)) \{}\FunctionTok{install.packages}\NormalTok{(}\StringTok{"terra"}\NormalTok{); }\FunctionTok{require}\NormalTok{(terra)\}}

\CommentTok{\# templates {-}{-}{-}{-}}
\NormalTok{template100}\OtherTok{=}\FunctionTok{rast}\NormalTok{(}\StringTok{"./Templates/TemplateRasters/LV100m\_10km.tif"}\NormalTok{)}

\CommentTok{\# radii}
\FunctionTok{radius\_function}\NormalTok{(}
 \AttributeTok{kvadrati\_path =} \StringTok{"./Templates/TemplateGrids/tiles/"}\NormalTok{,}
 \AttributeTok{radii\_path   =} \StringTok{"./Templates/TemplateGridPoints/tiles/"}\NormalTok{,}
 \AttributeTok{tikls100\_path =} \StringTok{"./Templates/TemplateGrids/tikls100\_sauzeme.parquet"}\NormalTok{,}
 \AttributeTok{template\_path =} \StringTok{"./Templates/TemplateRasters/LV100m\_10km.tif"}\NormalTok{,}
 \AttributeTok{input\_layers  =} \FunctionTok{c}\NormalTok{(}\StringTok{"./RasterGrids\_500m/2024/Diversity\_Forests\_500x.tif"}\NormalTok{),}
 \AttributeTok{layer\_prefixes =} \FunctionTok{c}\NormalTok{(}\StringTok{"Diversity\_Forest"}\NormalTok{),}
 \AttributeTok{output\_dir   =} \StringTok{"./RasterGrids\_100m/2024/RAW/"}\NormalTok{,}
 \AttributeTok{n\_workers   =} \DecValTok{12}\NormalTok{,}
 \AttributeTok{radii     =} \FunctionTok{c}\NormalTok{(}\StringTok{"r500"}\NormalTok{),}
 \AttributeTok{radius\_mode  =} \StringTok{"sparse"}\NormalTok{,}
 \AttributeTok{extract\_fun  =} \StringTok{"mean"}\NormalTok{,}
 \AttributeTok{fill\_missing  =} \ConstantTok{TRUE}\NormalTok{,}
 \AttributeTok{IDW\_weight   =} \DecValTok{2}\NormalTok{,}
 \AttributeTok{future\_max\_size =} \DecValTok{5} \SpecialCharTok{*} \DecValTok{1024}\SpecialCharTok{\^{}}\DecValTok{3}\NormalTok{)}

\CommentTok{\# Diversity\_Forest\_r500.tif egv\_102}
\NormalTok{slanis}\OtherTok{=}\FunctionTok{rast}\NormalTok{(}\StringTok{"./RasterGrids\_100m/2024/RAW/Diversity\_Forest\_r500.tif"}\NormalTok{)}
\FunctionTok{names}\NormalTok{(slanis)}\OtherTok{=}\StringTok{"egv\_102"}
\NormalTok{slanis2}\OtherTok{=}\FunctionTok{project}\NormalTok{(slanis,template100)}
\FunctionTok{writeRaster}\NormalTok{(slanis2,}
      \StringTok{"./RasterGrids\_100m/2024/RAW/Diversity\_Forest\_r500.tif"}\NormalTok{,}
      \AttributeTok{overwrite=}\ConstantTok{TRUE}\NormalTok{)}

\CommentTok{\# standardisation {-}{-}{-}{-}}
\ControlFlowTok{if}\NormalTok{(}\SpecialCharTok{!}\FunctionTok{require}\NormalTok{(terra)) \{}\FunctionTok{install.packages}\NormalTok{(}\StringTok{"terra"}\NormalTok{); }\FunctionTok{require}\NormalTok{(terra)\}}
\ControlFlowTok{if}\NormalTok{(}\SpecialCharTok{!}\FunctionTok{require}\NormalTok{(tidyverse)) \{}\FunctionTok{install.packages}\NormalTok{(}\StringTok{"tidyverse"}\NormalTok{); }\FunctionTok{require}\NormalTok{(tidyverse)\}}

\NormalTok{nosaukums}\OtherTok{=}\StringTok{"Diversity\_Forest\_r500.tif"}
\NormalTok{ielasisanas\_cels}\OtherTok{=}\FunctionTok{paste0}\NormalTok{(}\StringTok{"./RasterGrids\_100m/2024/RAW/"}\NormalTok{,nosaukums)}
\NormalTok{saglabasanas\_cels}\OtherTok{=}\FunctionTok{paste0}\NormalTok{(}\StringTok{"./RasterGrids\_100m/2024/Scaled/"}\NormalTok{,nosaukums)}
\NormalTok{slanis}\OtherTok{=}\FunctionTok{rast}\NormalTok{(ielasisanas\_cels)}
\NormalTok{videjais}\OtherTok{=}\FunctionTok{global}\NormalTok{(slanis,}\AttributeTok{fun=}\StringTok{"mean"}\NormalTok{,}\AttributeTok{na.rm=}\ConstantTok{TRUE}\NormalTok{)}
\NormalTok{centrets}\OtherTok{=}\NormalTok{slanis}\SpecialCharTok{{-}}\NormalTok{videjais[,}\DecValTok{1}\NormalTok{]}
\NormalTok{standartnovirze}\OtherTok{=}\NormalTok{terra}\SpecialCharTok{::}\FunctionTok{global}\NormalTok{(centrets,}\AttributeTok{fun=}\StringTok{"rms"}\NormalTok{,}\AttributeTok{na.rm=}\ConstantTok{TRUE}\NormalTok{)}
\NormalTok{merogots}\OtherTok{=}\NormalTok{centrets}\SpecialCharTok{/}\NormalTok{standartnovirze[,}\DecValTok{1}\NormalTok{]}
\FunctionTok{writeRaster}\NormalTok{(merogots,}
      \AttributeTok{filename=}\NormalTok{saglabasanas\_cels,}
      \AttributeTok{overwrite=}\ConstantTok{TRUE}\NormalTok{)}
\end{Highlighting}
\end{Shaded}

\section{Diversity\_Forest\_r1250}\label{ch06.103}

\textbf{filename:} \texttt{Diversity\_Forest\_r1250.tif}

\textbf{layername:} \texttt{egv\_103}

\textbf{English name:} Average forest class α-diversity of 500 m grid cells within
the 1.25 km landscape

\textbf{Latvian name:} Vidējā mežu ainavas klašu 500 m šūnu α-daudzveidība 1,25 km
ainavā

\textbf{Procedure:} Derived from the \hyperref[Ch05.04]{Landscape diversity}, more precisely
\hyperref[Ch05.04.02]{Forest diversity}. The average value of 25 ha cells diversity index
values is calculated using the workflow \texttt{egvtools::radius\_function()}. To
prevent possible data loss at edge cells, inverse distance weighted
(power = 2) gap filling is implemented. File is written twice, to ensure
layername. Finally, the layer is standardised
by subtracting the arithmetic mean and dividing by the root mean squared error.

\begin{Shaded}
\begin{Highlighting}[]
\CommentTok{\# libs {-}{-}{-}{-}}
\ControlFlowTok{if}\NormalTok{(}\SpecialCharTok{!}\FunctionTok{require}\NormalTok{(egvtools)) \{remotes}\SpecialCharTok{::}\FunctionTok{install\_github}\NormalTok{(}\StringTok{"aavotins/egvtools"}\NormalTok{); }\FunctionTok{require}\NormalTok{(egvtools)\}}
\ControlFlowTok{if}\NormalTok{(}\SpecialCharTok{!}\FunctionTok{require}\NormalTok{(terra)) \{}\FunctionTok{install.packages}\NormalTok{(}\StringTok{"terra"}\NormalTok{); }\FunctionTok{require}\NormalTok{(terra)\}}

\CommentTok{\# templates {-}{-}{-}{-}}
\NormalTok{template100}\OtherTok{=}\FunctionTok{rast}\NormalTok{(}\StringTok{"./Templates/TemplateRasters/LV100m\_10km.tif"}\NormalTok{)}

\CommentTok{\# radii}
\FunctionTok{radius\_function}\NormalTok{(}
 \AttributeTok{kvadrati\_path =} \StringTok{"./Templates/TemplateGrids/tiles/"}\NormalTok{,}
 \AttributeTok{radii\_path   =} \StringTok{"./Templates/TemplateGridPoints/tiles/"}\NormalTok{,}
 \AttributeTok{tikls100\_path =} \StringTok{"./Templates/TemplateGrids/tikls100\_sauzeme.parquet"}\NormalTok{,}
 \AttributeTok{template\_path =} \StringTok{"./Templates/TemplateRasters/LV100m\_10km.tif"}\NormalTok{,}
 \AttributeTok{input\_layers  =} \FunctionTok{c}\NormalTok{(}\StringTok{"./RasterGrids\_500m/2024/Diversity\_Forests\_500x.tif"}\NormalTok{),}
 \AttributeTok{layer\_prefixes =} \FunctionTok{c}\NormalTok{(}\StringTok{"Diversity\_Forest"}\NormalTok{),}
 \AttributeTok{output\_dir   =} \StringTok{"./RasterGrids\_100m/2024/RAW/"}\NormalTok{,}
 \AttributeTok{n\_workers   =} \DecValTok{12}\NormalTok{,}
 \AttributeTok{radii     =} \FunctionTok{c}\NormalTok{(}\StringTok{"r1250"}\NormalTok{),}
 \AttributeTok{radius\_mode  =} \StringTok{"sparse"}\NormalTok{,}
 \AttributeTok{extract\_fun  =} \StringTok{"mean"}\NormalTok{,}
 \AttributeTok{fill\_missing  =} \ConstantTok{TRUE}\NormalTok{,}
 \AttributeTok{IDW\_weight   =} \DecValTok{2}\NormalTok{,}
 \AttributeTok{future\_max\_size =} \DecValTok{5} \SpecialCharTok{*} \DecValTok{1024}\SpecialCharTok{\^{}}\DecValTok{3}\NormalTok{)}

\CommentTok{\# Diversity\_Forest\_r1250.tif    egv\_103}
\NormalTok{slanis}\OtherTok{=}\FunctionTok{rast}\NormalTok{(}\StringTok{"./RasterGrids\_100m/2024/RAW/Diversity\_Forest\_r1250.tif"}\NormalTok{)}
\FunctionTok{names}\NormalTok{(slanis)}\OtherTok{=}\StringTok{"egv\_103"}
\NormalTok{slanis2}\OtherTok{=}\FunctionTok{project}\NormalTok{(slanis,template100)}
\FunctionTok{writeRaster}\NormalTok{(slanis2,}
      \StringTok{"./RasterGrids\_100m/2024/RAW/Diversity\_Forest\_r1250.tif"}\NormalTok{,}
      \AttributeTok{overwrite=}\ConstantTok{TRUE}\NormalTok{)}

\CommentTok{\# standardisation {-}{-}{-}{-}}
\ControlFlowTok{if}\NormalTok{(}\SpecialCharTok{!}\FunctionTok{require}\NormalTok{(terra)) \{}\FunctionTok{install.packages}\NormalTok{(}\StringTok{"terra"}\NormalTok{); }\FunctionTok{require}\NormalTok{(terra)\}}
\ControlFlowTok{if}\NormalTok{(}\SpecialCharTok{!}\FunctionTok{require}\NormalTok{(tidyverse)) \{}\FunctionTok{install.packages}\NormalTok{(}\StringTok{"tidyverse"}\NormalTok{); }\FunctionTok{require}\NormalTok{(tidyverse)\}}

\NormalTok{nosaukums}\OtherTok{=}\StringTok{"Diversity\_Forest\_r1250.tif"}
\NormalTok{ielasisanas\_cels}\OtherTok{=}\FunctionTok{paste0}\NormalTok{(}\StringTok{"./RasterGrids\_100m/2024/RAW/"}\NormalTok{,nosaukums)}
\NormalTok{saglabasanas\_cels}\OtherTok{=}\FunctionTok{paste0}\NormalTok{(}\StringTok{"./RasterGrids\_100m/2024/Scaled/"}\NormalTok{,nosaukums)}
\NormalTok{slanis}\OtherTok{=}\FunctionTok{rast}\NormalTok{(ielasisanas\_cels)}
\NormalTok{videjais}\OtherTok{=}\FunctionTok{global}\NormalTok{(slanis,}\AttributeTok{fun=}\StringTok{"mean"}\NormalTok{,}\AttributeTok{na.rm=}\ConstantTok{TRUE}\NormalTok{)}
\NormalTok{centrets}\OtherTok{=}\NormalTok{slanis}\SpecialCharTok{{-}}\NormalTok{videjais[,}\DecValTok{1}\NormalTok{]}
\NormalTok{standartnovirze}\OtherTok{=}\NormalTok{terra}\SpecialCharTok{::}\FunctionTok{global}\NormalTok{(centrets,}\AttributeTok{fun=}\StringTok{"rms"}\NormalTok{,}\AttributeTok{na.rm=}\ConstantTok{TRUE}\NormalTok{)}
\NormalTok{merogots}\OtherTok{=}\NormalTok{centrets}\SpecialCharTok{/}\NormalTok{standartnovirze[,}\DecValTok{1}\NormalTok{]}
\FunctionTok{writeRaster}\NormalTok{(merogots,}
      \AttributeTok{filename=}\NormalTok{saglabasanas\_cels,}
      \AttributeTok{overwrite=}\ConstantTok{TRUE}\NormalTok{)}
\end{Highlighting}
\end{Shaded}

\section{Diversity\_Forest\_r3000}\label{ch06.104}

\textbf{filename:} \texttt{Diversity\_Forest\_r3000.tif}

\textbf{layername:} \texttt{egv\_104}

\textbf{English name:} Average forest class α-diversity of 500 m grid cells within
the 3 km landscape

\textbf{Latvian name:} Vidējā mežu ainavas klašu 500 m šūnu α-daudzveidība 3 km
ainavā

\textbf{Procedure:} Derived from the \hyperref[Ch05.04]{Landscape diversity}, more precisely
\hyperref[Ch05.04.02]{Forest diversity}. The average value of 25 ha cells diversity index
values is calculated using the workflow \texttt{egvtools::radius\_function()}. To
prevent possible data loss at edge cells, inverse distance weighted
(power = 2) gap filling is implemented. File is written twice, to ensure
layername. Finally, the layer is standardised
by subtracting the arithmetic mean and dividing by the root mean squared error.

\begin{Shaded}
\begin{Highlighting}[]
\CommentTok{\# libs {-}{-}{-}{-}}
\ControlFlowTok{if}\NormalTok{(}\SpecialCharTok{!}\FunctionTok{require}\NormalTok{(egvtools)) \{remotes}\SpecialCharTok{::}\FunctionTok{install\_github}\NormalTok{(}\StringTok{"aavotins/egvtools"}\NormalTok{); }\FunctionTok{require}\NormalTok{(egvtools)\}}
\ControlFlowTok{if}\NormalTok{(}\SpecialCharTok{!}\FunctionTok{require}\NormalTok{(terra)) \{}\FunctionTok{install.packages}\NormalTok{(}\StringTok{"terra"}\NormalTok{); }\FunctionTok{require}\NormalTok{(terra)\}}

\CommentTok{\# templates {-}{-}{-}{-}}
\NormalTok{template100}\OtherTok{=}\FunctionTok{rast}\NormalTok{(}\StringTok{"./Templates/TemplateRasters/LV100m\_10km.tif"}\NormalTok{)}

\CommentTok{\# radii}
\FunctionTok{radius\_function}\NormalTok{(}
 \AttributeTok{kvadrati\_path =} \StringTok{"./Templates/TemplateGrids/tiles/"}\NormalTok{,}
 \AttributeTok{radii\_path   =} \StringTok{"./Templates/TemplateGridPoints/tiles/"}\NormalTok{,}
 \AttributeTok{tikls100\_path =} \StringTok{"./Templates/TemplateGrids/tikls100\_sauzeme.parquet"}\NormalTok{,}
 \AttributeTok{template\_path =} \StringTok{"./Templates/TemplateRasters/LV100m\_10km.tif"}\NormalTok{,}
 \AttributeTok{input\_layers  =} \FunctionTok{c}\NormalTok{(}\StringTok{"./RasterGrids\_500m/2024/Diversity\_Forests\_500x.tif"}\NormalTok{),}
 \AttributeTok{layer\_prefixes =} \FunctionTok{c}\NormalTok{(}\StringTok{"Diversity\_Forest"}\NormalTok{),}
 \AttributeTok{output\_dir   =} \StringTok{"./RasterGrids\_100m/2024/RAW/"}\NormalTok{,}
 \AttributeTok{n\_workers   =} \DecValTok{12}\NormalTok{,}
 \AttributeTok{radii     =} \FunctionTok{c}\NormalTok{(}\StringTok{"r3000"}\NormalTok{),}
 \AttributeTok{radius\_mode  =} \StringTok{"sparse"}\NormalTok{,}
 \AttributeTok{extract\_fun  =} \StringTok{"mean"}\NormalTok{,}
 \AttributeTok{fill\_missing  =} \ConstantTok{TRUE}\NormalTok{,}
 \AttributeTok{IDW\_weight   =} \DecValTok{2}\NormalTok{,}
 \AttributeTok{future\_max\_size =} \DecValTok{5} \SpecialCharTok{*} \DecValTok{1024}\SpecialCharTok{\^{}}\DecValTok{3}\NormalTok{)}

\CommentTok{\# Diversity\_Forest\_r3000.tif    egv\_104}
\NormalTok{slanis}\OtherTok{=}\FunctionTok{rast}\NormalTok{(}\StringTok{"./RasterGrids\_100m/2024/RAW/Diversity\_Forest\_r3000.tif"}\NormalTok{)}
\FunctionTok{names}\NormalTok{(slanis)}\OtherTok{=}\StringTok{"egv\_104"}
\NormalTok{slanis2}\OtherTok{=}\FunctionTok{project}\NormalTok{(slanis,template100)}
\FunctionTok{writeRaster}\NormalTok{(slanis2,}
      \StringTok{"./RasterGrids\_100m/2024/RAW/Diversity\_Forest\_r3000.tif"}\NormalTok{,}
      \AttributeTok{overwrite=}\ConstantTok{TRUE}\NormalTok{)}

\CommentTok{\# standardisation {-}{-}{-}{-}}
\ControlFlowTok{if}\NormalTok{(}\SpecialCharTok{!}\FunctionTok{require}\NormalTok{(terra)) \{}\FunctionTok{install.packages}\NormalTok{(}\StringTok{"terra"}\NormalTok{); }\FunctionTok{require}\NormalTok{(terra)\}}
\ControlFlowTok{if}\NormalTok{(}\SpecialCharTok{!}\FunctionTok{require}\NormalTok{(tidyverse)) \{}\FunctionTok{install.packages}\NormalTok{(}\StringTok{"tidyverse"}\NormalTok{); }\FunctionTok{require}\NormalTok{(tidyverse)\}}

\NormalTok{nosaukums}\OtherTok{=}\StringTok{"Diversity\_Forest\_r3000.tif"}
\NormalTok{ielasisanas\_cels}\OtherTok{=}\FunctionTok{paste0}\NormalTok{(}\StringTok{"./RasterGrids\_100m/2024/RAW/"}\NormalTok{,nosaukums)}
\NormalTok{saglabasanas\_cels}\OtherTok{=}\FunctionTok{paste0}\NormalTok{(}\StringTok{"./RasterGrids\_100m/2024/Scaled/"}\NormalTok{,nosaukums)}
\NormalTok{slanis}\OtherTok{=}\FunctionTok{rast}\NormalTok{(ielasisanas\_cels)}
\NormalTok{videjais}\OtherTok{=}\FunctionTok{global}\NormalTok{(slanis,}\AttributeTok{fun=}\StringTok{"mean"}\NormalTok{,}\AttributeTok{na.rm=}\ConstantTok{TRUE}\NormalTok{)}
\NormalTok{centrets}\OtherTok{=}\NormalTok{slanis}\SpecialCharTok{{-}}\NormalTok{videjais[,}\DecValTok{1}\NormalTok{]}
\NormalTok{standartnovirze}\OtherTok{=}\NormalTok{terra}\SpecialCharTok{::}\FunctionTok{global}\NormalTok{(centrets,}\AttributeTok{fun=}\StringTok{"rms"}\NormalTok{,}\AttributeTok{na.rm=}\ConstantTok{TRUE}\NormalTok{)}
\NormalTok{merogots}\OtherTok{=}\NormalTok{centrets}\SpecialCharTok{/}\NormalTok{standartnovirze[,}\DecValTok{1}\NormalTok{]}
\FunctionTok{writeRaster}\NormalTok{(merogots,}
      \AttributeTok{filename=}\NormalTok{saglabasanas\_cels,}
      \AttributeTok{overwrite=}\ConstantTok{TRUE}\NormalTok{)}
\end{Highlighting}
\end{Shaded}

\section{Diversity\_Forest\_r10000}\label{ch06.105}

\textbf{filename:} \texttt{Diversity\_Forest\_r10000.tif}

\textbf{layername:} \texttt{egv\_105}

\textbf{English name:} Average forest class α-diversity of 500 m grid cells within
the 10 km landscape

\textbf{Latvian name:} Vidējā mežu ainavas klašu 500 m šūnu α-daudzveidība 10 km
ainavā

\textbf{Procedure:} Derived from the \hyperref[Ch05.04]{Landscape diversity}, more precisely
\hyperref[Ch05.04.02]{Forest diversity}. The average value of 25 ha cells diversity index
values is calculated using the workflow \texttt{egvtools::radius\_function()}. To
prevent possible data loss at edge cells, inverse distance weighted
(power = 2) gap filling is implemented. File is written twice, to ensure
layername. Finally, the layer is standardised
by subtracting the arithmetic mean and dividing by the root mean squared error.

\begin{Shaded}
\begin{Highlighting}[]
\CommentTok{\# libs {-}{-}{-}{-}}
\ControlFlowTok{if}\NormalTok{(}\SpecialCharTok{!}\FunctionTok{require}\NormalTok{(egvtools)) \{remotes}\SpecialCharTok{::}\FunctionTok{install\_github}\NormalTok{(}\StringTok{"aavotins/egvtools"}\NormalTok{); }\FunctionTok{require}\NormalTok{(egvtools)\}}
\ControlFlowTok{if}\NormalTok{(}\SpecialCharTok{!}\FunctionTok{require}\NormalTok{(terra)) \{}\FunctionTok{install.packages}\NormalTok{(}\StringTok{"terra"}\NormalTok{); }\FunctionTok{require}\NormalTok{(terra)\}}

\CommentTok{\# templates {-}{-}{-}{-}}
\NormalTok{template100}\OtherTok{=}\FunctionTok{rast}\NormalTok{(}\StringTok{"./Templates/TemplateRasters/LV100m\_10km.tif"}\NormalTok{)}

\CommentTok{\# radii}
\FunctionTok{radius\_function}\NormalTok{(}
 \AttributeTok{kvadrati\_path =} \StringTok{"./Templates/TemplateGrids/tiles/"}\NormalTok{,}
 \AttributeTok{radii\_path   =} \StringTok{"./Templates/TemplateGridPoints/tiles/"}\NormalTok{,}
 \AttributeTok{tikls100\_path =} \StringTok{"./Templates/TemplateGrids/tikls100\_sauzeme.parquet"}\NormalTok{,}
 \AttributeTok{template\_path =} \StringTok{"./Templates/TemplateRasters/LV100m\_10km.tif"}\NormalTok{,}
 \AttributeTok{input\_layers  =} \FunctionTok{c}\NormalTok{(}\StringTok{"./RasterGrids\_500m/2024/Diversity\_Forests\_500x.tif"}\NormalTok{),}
 \AttributeTok{layer\_prefixes =} \FunctionTok{c}\NormalTok{(}\StringTok{"Diversity\_Forest"}\NormalTok{),}
 \AttributeTok{output\_dir   =} \StringTok{"./RasterGrids\_100m/2024/RAW/"}\NormalTok{,}
 \AttributeTok{n\_workers   =} \DecValTok{12}\NormalTok{,}
 \AttributeTok{radii     =} \FunctionTok{c}\NormalTok{(}\StringTok{"r10000"}\NormalTok{),}
 \AttributeTok{radius\_mode  =} \StringTok{"sparse"}\NormalTok{,}
 \AttributeTok{extract\_fun  =} \StringTok{"mean"}\NormalTok{,}
 \AttributeTok{fill\_missing  =} \ConstantTok{TRUE}\NormalTok{,}
 \AttributeTok{IDW\_weight   =} \DecValTok{2}\NormalTok{,}
 \AttributeTok{future\_max\_size =} \DecValTok{5} \SpecialCharTok{*} \DecValTok{1024}\SpecialCharTok{\^{}}\DecValTok{3}\NormalTok{)}

\CommentTok{\# Diversity\_Forest\_r10000.tif   egv\_105}
\NormalTok{slanis}\OtherTok{=}\FunctionTok{rast}\NormalTok{(}\StringTok{"./RasterGrids\_100m/2024/RAW/Diversity\_Forest\_r10000.tif"}\NormalTok{)}
\FunctionTok{names}\NormalTok{(slanis)}\OtherTok{=}\StringTok{"egv\_105"}
\NormalTok{slanis2}\OtherTok{=}\FunctionTok{project}\NormalTok{(slanis,template100)}
\FunctionTok{writeRaster}\NormalTok{(slanis2,}
      \StringTok{"./RasterGrids\_100m/2024/RAW/Diversity\_Forest\_r10000.tif"}\NormalTok{,}
      \AttributeTok{overwrite=}\ConstantTok{TRUE}\NormalTok{)}

\CommentTok{\# standardisation {-}{-}{-}{-}}
\ControlFlowTok{if}\NormalTok{(}\SpecialCharTok{!}\FunctionTok{require}\NormalTok{(terra)) \{}\FunctionTok{install.packages}\NormalTok{(}\StringTok{"terra"}\NormalTok{); }\FunctionTok{require}\NormalTok{(terra)\}}
\ControlFlowTok{if}\NormalTok{(}\SpecialCharTok{!}\FunctionTok{require}\NormalTok{(tidyverse)) \{}\FunctionTok{install.packages}\NormalTok{(}\StringTok{"tidyverse"}\NormalTok{); }\FunctionTok{require}\NormalTok{(tidyverse)\}}

\NormalTok{nosaukums}\OtherTok{=}\StringTok{"Diversity\_Forest\_r10000.tif"}
\NormalTok{ielasisanas\_cels}\OtherTok{=}\FunctionTok{paste0}\NormalTok{(}\StringTok{"./RasterGrids\_100m/2024/RAW/"}\NormalTok{,nosaukums)}
\NormalTok{saglabasanas\_cels}\OtherTok{=}\FunctionTok{paste0}\NormalTok{(}\StringTok{"./RasterGrids\_100m/2024/Scaled/"}\NormalTok{,nosaukums)}
\NormalTok{slanis}\OtherTok{=}\FunctionTok{rast}\NormalTok{(ielasisanas\_cels)}
\NormalTok{videjais}\OtherTok{=}\FunctionTok{global}\NormalTok{(slanis,}\AttributeTok{fun=}\StringTok{"mean"}\NormalTok{,}\AttributeTok{na.rm=}\ConstantTok{TRUE}\NormalTok{)}
\NormalTok{centrets}\OtherTok{=}\NormalTok{slanis}\SpecialCharTok{{-}}\NormalTok{videjais[,}\DecValTok{1}\NormalTok{]}
\NormalTok{standartnovirze}\OtherTok{=}\NormalTok{terra}\SpecialCharTok{::}\FunctionTok{global}\NormalTok{(centrets,}\AttributeTok{fun=}\StringTok{"rms"}\NormalTok{,}\AttributeTok{na.rm=}\ConstantTok{TRUE}\NormalTok{)}
\NormalTok{merogots}\OtherTok{=}\NormalTok{centrets}\SpecialCharTok{/}\NormalTok{standartnovirze[,}\DecValTok{1}\NormalTok{]}
\FunctionTok{writeRaster}\NormalTok{(merogots,}
      \AttributeTok{filename=}\NormalTok{saglabasanas\_cels,}
      \AttributeTok{overwrite=}\ConstantTok{TRUE}\NormalTok{)}
\end{Highlighting}
\end{Shaded}

\section{Diversity\_Total\_r500}\label{ch06.106}

\textbf{filename:} \texttt{Diversity\_Total\_r500.tif}

\textbf{layername:} \texttt{egv\_106}

\textbf{English name:} Average combined landscape α-diversity of 500 m grid cells
within the 0.5 km landscape

\textbf{Latvian name:} Vidējā visu ainavas klašu 500 m šūnu α-daudzveidība 0,5 km
ainavā

\textbf{Procedure:} Derived from the \hyperref[Ch05.04]{Landscape diversity}, more precisely
\hyperref[Ch05.04.01]{Overall landscape diversity}. The average value of 25 ha cells diversity index
values is calculated using the workflow \texttt{egvtools::radius\_function()}. To
prevent possible data loss at edge cells, inverse distance weighted
(power = 2) gap filling is implemented. File is written twice, to ensure
layername. Finally, the layer is standardised
by subtracting the arithmetic mean and dividing by the root mean squared error.

\begin{Shaded}
\begin{Highlighting}[]
\CommentTok{\# libs {-}{-}{-}{-}}
\ControlFlowTok{if}\NormalTok{(}\SpecialCharTok{!}\FunctionTok{require}\NormalTok{(egvtools)) \{remotes}\SpecialCharTok{::}\FunctionTok{install\_github}\NormalTok{(}\StringTok{"aavotins/egvtools"}\NormalTok{); }\FunctionTok{require}\NormalTok{(egvtools)\}}
\ControlFlowTok{if}\NormalTok{(}\SpecialCharTok{!}\FunctionTok{require}\NormalTok{(terra)) \{}\FunctionTok{install.packages}\NormalTok{(}\StringTok{"terra"}\NormalTok{); }\FunctionTok{require}\NormalTok{(terra)\}}

\CommentTok{\# templates {-}{-}{-}{-}}
\NormalTok{template100}\OtherTok{=}\FunctionTok{rast}\NormalTok{(}\StringTok{"./Templates/TemplateRasters/LV100m\_10km.tif"}\NormalTok{)}

\CommentTok{\# radii}
\FunctionTok{radius\_function}\NormalTok{(}
 \AttributeTok{kvadrati\_path =} \StringTok{"./Templates/TemplateGrids/tiles/"}\NormalTok{,}
 \AttributeTok{radii\_path   =} \StringTok{"./Templates/TemplateGridPoints/tiles/"}\NormalTok{,}
 \AttributeTok{tikls100\_path =} \StringTok{"./Templates/TemplateGrids/tikls100\_sauzeme.parquet"}\NormalTok{,}
 \AttributeTok{template\_path =} \StringTok{"./Templates/TemplateRasters/LV100m\_10km.tif"}\NormalTok{,}
 \AttributeTok{input\_layers  =} \FunctionTok{c}\NormalTok{(}\StringTok{"./RasterGrids\_500m/2024/Diversity\_GeneralLandscape\_500x.tif"}\NormalTok{),}
 \AttributeTok{layer\_prefixes =} \FunctionTok{c}\NormalTok{(}\StringTok{"Diversity\_Total"}\NormalTok{),}
 \AttributeTok{output\_dir   =} \StringTok{"./RasterGrids\_100m/2024/RAW/"}\NormalTok{,}
 \AttributeTok{n\_workers   =} \DecValTok{12}\NormalTok{,}
 \AttributeTok{radii     =} \FunctionTok{c}\NormalTok{(}\StringTok{"r500"}\NormalTok{),}
 \AttributeTok{radius\_mode  =} \StringTok{"sparse"}\NormalTok{,}
 \AttributeTok{extract\_fun  =} \StringTok{"mean"}\NormalTok{,}
 \AttributeTok{fill\_missing  =} \ConstantTok{TRUE}\NormalTok{,}
 \AttributeTok{IDW\_weight   =} \DecValTok{2}\NormalTok{,}
 \AttributeTok{future\_max\_size =} \DecValTok{5} \SpecialCharTok{*} \DecValTok{1024}\SpecialCharTok{\^{}}\DecValTok{3}\NormalTok{)}

\CommentTok{\# Diversity\_Total\_r500.tif  egv\_106}
\NormalTok{slanis}\OtherTok{=}\FunctionTok{rast}\NormalTok{(}\StringTok{"./RasterGrids\_100m/2024/RAW/Diversity\_Total\_r500.tif"}\NormalTok{)}
\FunctionTok{names}\NormalTok{(slanis)}\OtherTok{=}\StringTok{"egv\_106"}
\NormalTok{slanis2}\OtherTok{=}\FunctionTok{project}\NormalTok{(slanis,template100)}
\FunctionTok{writeRaster}\NormalTok{(slanis2,}
      \StringTok{"./RasterGrids\_100m/2024/RAW/Diversity\_Total\_r500.tif"}\NormalTok{,}
      \AttributeTok{overwrite=}\ConstantTok{TRUE}\NormalTok{)}

\CommentTok{\# standardisation {-}{-}{-}{-}}
\ControlFlowTok{if}\NormalTok{(}\SpecialCharTok{!}\FunctionTok{require}\NormalTok{(terra)) \{}\FunctionTok{install.packages}\NormalTok{(}\StringTok{"terra"}\NormalTok{); }\FunctionTok{require}\NormalTok{(terra)\}}
\ControlFlowTok{if}\NormalTok{(}\SpecialCharTok{!}\FunctionTok{require}\NormalTok{(tidyverse)) \{}\FunctionTok{install.packages}\NormalTok{(}\StringTok{"tidyverse"}\NormalTok{); }\FunctionTok{require}\NormalTok{(tidyverse)\}}

\NormalTok{nosaukums}\OtherTok{=}\StringTok{"Diversity\_Total\_r500.tif"}
\NormalTok{ielasisanas\_cels}\OtherTok{=}\FunctionTok{paste0}\NormalTok{(}\StringTok{"./RasterGrids\_100m/2024/RAW/"}\NormalTok{,nosaukums)}
\NormalTok{saglabasanas\_cels}\OtherTok{=}\FunctionTok{paste0}\NormalTok{(}\StringTok{"./RasterGrids\_100m/2024/Scaled/"}\NormalTok{,nosaukums)}
\NormalTok{slanis}\OtherTok{=}\FunctionTok{rast}\NormalTok{(ielasisanas\_cels)}
\NormalTok{videjais}\OtherTok{=}\FunctionTok{global}\NormalTok{(slanis,}\AttributeTok{fun=}\StringTok{"mean"}\NormalTok{,}\AttributeTok{na.rm=}\ConstantTok{TRUE}\NormalTok{)}
\NormalTok{centrets}\OtherTok{=}\NormalTok{slanis}\SpecialCharTok{{-}}\NormalTok{videjais[,}\DecValTok{1}\NormalTok{]}
\NormalTok{standartnovirze}\OtherTok{=}\NormalTok{terra}\SpecialCharTok{::}\FunctionTok{global}\NormalTok{(centrets,}\AttributeTok{fun=}\StringTok{"rms"}\NormalTok{,}\AttributeTok{na.rm=}\ConstantTok{TRUE}\NormalTok{)}
\NormalTok{merogots}\OtherTok{=}\NormalTok{centrets}\SpecialCharTok{/}\NormalTok{standartnovirze[,}\DecValTok{1}\NormalTok{]}
\FunctionTok{writeRaster}\NormalTok{(merogots,}
      \AttributeTok{filename=}\NormalTok{saglabasanas\_cels,}
      \AttributeTok{overwrite=}\ConstantTok{TRUE}\NormalTok{)}
\end{Highlighting}
\end{Shaded}

\section{Diversity\_Total\_r1250}\label{ch06.107}

\textbf{filename:} \texttt{Diversity\_Total\_r1250.tif}

\textbf{layername:} \texttt{egv\_107}

\textbf{English name:} Average combined landscape α-diversity of 500 m grid cells
within the 1.25 km landscape

\textbf{Latvian name:} Vidējā visu ainavas klašu 500 m šūnu α-daudzveidība 1,25 km
ainavā

\textbf{Procedure:} Derived from the \hyperref[Ch05.04]{Landscape diversity}, more precisely
\hyperref[Ch05.04.01]{Overall landscape diversity}. The average value of 25 ha cells diversity index
values is calculated using the workflow \texttt{egvtools::radius\_function()}. To
prevent possible data loss at edge cells, inverse distance weighted
(power = 2) gap filling is implemented. File is written twice, to ensure
layername. Finally, the layer is standardised
by subtracting the arithmetic mean and dividing by the root mean squared error.

\begin{Shaded}
\begin{Highlighting}[]
\CommentTok{\# libs {-}{-}{-}{-}}
\ControlFlowTok{if}\NormalTok{(}\SpecialCharTok{!}\FunctionTok{require}\NormalTok{(egvtools)) \{remotes}\SpecialCharTok{::}\FunctionTok{install\_github}\NormalTok{(}\StringTok{"aavotins/egvtools"}\NormalTok{); }\FunctionTok{require}\NormalTok{(egvtools)\}}
\ControlFlowTok{if}\NormalTok{(}\SpecialCharTok{!}\FunctionTok{require}\NormalTok{(terra)) \{}\FunctionTok{install.packages}\NormalTok{(}\StringTok{"terra"}\NormalTok{); }\FunctionTok{require}\NormalTok{(terra)\}}

\CommentTok{\# templates {-}{-}{-}{-}}
\NormalTok{template100}\OtherTok{=}\FunctionTok{rast}\NormalTok{(}\StringTok{"./Templates/TemplateRasters/LV100m\_10km.tif"}\NormalTok{)}

\CommentTok{\# radii}
\FunctionTok{radius\_function}\NormalTok{(}
 \AttributeTok{kvadrati\_path =} \StringTok{"./Templates/TemplateGrids/tiles/"}\NormalTok{,}
 \AttributeTok{radii\_path   =} \StringTok{"./Templates/TemplateGridPoints/tiles/"}\NormalTok{,}
 \AttributeTok{tikls100\_path =} \StringTok{"./Templates/TemplateGrids/tikls100\_sauzeme.parquet"}\NormalTok{,}
 \AttributeTok{template\_path =} \StringTok{"./Templates/TemplateRasters/LV100m\_10km.tif"}\NormalTok{,}
 \AttributeTok{input\_layers  =} \FunctionTok{c}\NormalTok{(}\StringTok{"./RasterGrids\_500m/2024/Diversity\_GeneralLandscape\_500x.tif"}\NormalTok{),}
 \AttributeTok{layer\_prefixes =} \FunctionTok{c}\NormalTok{(}\StringTok{"Diversity\_Total"}\NormalTok{),}
 \AttributeTok{output\_dir   =} \StringTok{"./RasterGrids\_100m/2024/RAW/"}\NormalTok{,}
 \AttributeTok{n\_workers   =} \DecValTok{12}\NormalTok{,}
 \AttributeTok{radii     =} \FunctionTok{c}\NormalTok{(}\StringTok{"r1250"}\NormalTok{),}
 \AttributeTok{radius\_mode  =} \StringTok{"sparse"}\NormalTok{,}
 \AttributeTok{extract\_fun  =} \StringTok{"mean"}\NormalTok{,}
 \AttributeTok{fill\_missing  =} \ConstantTok{TRUE}\NormalTok{,}
 \AttributeTok{IDW\_weight   =} \DecValTok{2}\NormalTok{,}
 \AttributeTok{future\_max\_size =} \DecValTok{5} \SpecialCharTok{*} \DecValTok{1024}\SpecialCharTok{\^{}}\DecValTok{3}\NormalTok{)}

\CommentTok{\# Diversity\_Total\_r1250.tif egv\_107}
\NormalTok{slanis}\OtherTok{=}\FunctionTok{rast}\NormalTok{(}\StringTok{"./RasterGrids\_100m/2024/RAW/Diversity\_Total\_r1250.tif"}\NormalTok{)}
\FunctionTok{names}\NormalTok{(slanis)}\OtherTok{=}\StringTok{"egv\_107"}
\NormalTok{slanis2}\OtherTok{=}\FunctionTok{project}\NormalTok{(slanis,template100)}
\FunctionTok{writeRaster}\NormalTok{(slanis2,}
      \StringTok{"./RasterGrids\_100m/2024/RAW/Diversity\_Total\_r1250.tif"}\NormalTok{,}
      \AttributeTok{overwrite=}\ConstantTok{TRUE}\NormalTok{)}

\CommentTok{\# standardisation {-}{-}{-}{-}}
\ControlFlowTok{if}\NormalTok{(}\SpecialCharTok{!}\FunctionTok{require}\NormalTok{(terra)) \{}\FunctionTok{install.packages}\NormalTok{(}\StringTok{"terra"}\NormalTok{); }\FunctionTok{require}\NormalTok{(terra)\}}
\ControlFlowTok{if}\NormalTok{(}\SpecialCharTok{!}\FunctionTok{require}\NormalTok{(tidyverse)) \{}\FunctionTok{install.packages}\NormalTok{(}\StringTok{"tidyverse"}\NormalTok{); }\FunctionTok{require}\NormalTok{(tidyverse)\}}

\NormalTok{nosaukums}\OtherTok{=}\StringTok{"Diversity\_Total\_r1250.tif"}
\NormalTok{ielasisanas\_cels}\OtherTok{=}\FunctionTok{paste0}\NormalTok{(}\StringTok{"./RasterGrids\_100m/2024/RAW/"}\NormalTok{,nosaukums)}
\NormalTok{saglabasanas\_cels}\OtherTok{=}\FunctionTok{paste0}\NormalTok{(}\StringTok{"./RasterGrids\_100m/2024/Scaled/"}\NormalTok{,nosaukums)}
\NormalTok{slanis}\OtherTok{=}\FunctionTok{rast}\NormalTok{(ielasisanas\_cels)}
\NormalTok{videjais}\OtherTok{=}\FunctionTok{global}\NormalTok{(slanis,}\AttributeTok{fun=}\StringTok{"mean"}\NormalTok{,}\AttributeTok{na.rm=}\ConstantTok{TRUE}\NormalTok{)}
\NormalTok{centrets}\OtherTok{=}\NormalTok{slanis}\SpecialCharTok{{-}}\NormalTok{videjais[,}\DecValTok{1}\NormalTok{]}
\NormalTok{standartnovirze}\OtherTok{=}\NormalTok{terra}\SpecialCharTok{::}\FunctionTok{global}\NormalTok{(centrets,}\AttributeTok{fun=}\StringTok{"rms"}\NormalTok{,}\AttributeTok{na.rm=}\ConstantTok{TRUE}\NormalTok{)}
\NormalTok{merogots}\OtherTok{=}\NormalTok{centrets}\SpecialCharTok{/}\NormalTok{standartnovirze[,}\DecValTok{1}\NormalTok{]}
\FunctionTok{writeRaster}\NormalTok{(merogots,}
      \AttributeTok{filename=}\NormalTok{saglabasanas\_cels,}
      \AttributeTok{overwrite=}\ConstantTok{TRUE}\NormalTok{)}
\end{Highlighting}
\end{Shaded}

\section{Diversity\_Total\_r3000}\label{ch06.108}

\textbf{filename:} \texttt{Diversity\_Total\_r3000.tif}

\textbf{layername:} \texttt{egv\_108}

\textbf{English name:} Average combined landscape α-diversity of 500 m grid cells
within the 3 km landscape

\textbf{Latvian name:} Vidējā visu ainavas klašu 500 m šūnu α-daudzveidība 3 km
ainavā

\textbf{Procedure:} Derived from the \hyperref[Ch05.04]{Landscape diversity}, more precisely
\hyperref[Ch05.04.01]{Overall landscape diversity}. The average value of 25 ha cells diversity index
values is calculated using the workflow \texttt{egvtools::radius\_function()}. To
prevent possible data loss at edge cells, inverse distance weighted
(power = 2) gap filling is implemented. File is written twice, to ensure
layername. Finally, the layer is standardised
by subtracting the arithmetic mean and dividing by the root mean squared error.

\begin{Shaded}
\begin{Highlighting}[]
\CommentTok{\# libs {-}{-}{-}{-}}
\ControlFlowTok{if}\NormalTok{(}\SpecialCharTok{!}\FunctionTok{require}\NormalTok{(egvtools)) \{remotes}\SpecialCharTok{::}\FunctionTok{install\_github}\NormalTok{(}\StringTok{"aavotins/egvtools"}\NormalTok{); }\FunctionTok{require}\NormalTok{(egvtools)\}}
\ControlFlowTok{if}\NormalTok{(}\SpecialCharTok{!}\FunctionTok{require}\NormalTok{(terra)) \{}\FunctionTok{install.packages}\NormalTok{(}\StringTok{"terra"}\NormalTok{); }\FunctionTok{require}\NormalTok{(terra)\}}

\CommentTok{\# templates {-}{-}{-}{-}}
\NormalTok{template100}\OtherTok{=}\FunctionTok{rast}\NormalTok{(}\StringTok{"./Templates/TemplateRasters/LV100m\_10km.tif"}\NormalTok{)}

\CommentTok{\# radii}
\FunctionTok{radius\_function}\NormalTok{(}
 \AttributeTok{kvadrati\_path =} \StringTok{"./Templates/TemplateGrids/tiles/"}\NormalTok{,}
 \AttributeTok{radii\_path   =} \StringTok{"./Templates/TemplateGridPoints/tiles/"}\NormalTok{,}
 \AttributeTok{tikls100\_path =} \StringTok{"./Templates/TemplateGrids/tikls100\_sauzeme.parquet"}\NormalTok{,}
 \AttributeTok{template\_path =} \StringTok{"./Templates/TemplateRasters/LV100m\_10km.tif"}\NormalTok{,}
 \AttributeTok{input\_layers  =} \FunctionTok{c}\NormalTok{(}\StringTok{"./RasterGrids\_500m/2024/Diversity\_GeneralLandscape\_500x.tif"}\NormalTok{),}
 \AttributeTok{layer\_prefixes =} \FunctionTok{c}\NormalTok{(}\StringTok{"Diversity\_Total"}\NormalTok{),}
 \AttributeTok{output\_dir   =} \StringTok{"./RasterGrids\_100m/2024/RAW/"}\NormalTok{,}
 \AttributeTok{n\_workers   =} \DecValTok{12}\NormalTok{,}
 \AttributeTok{radii     =} \FunctionTok{c}\NormalTok{(}\StringTok{"r3000"}\NormalTok{),}
 \AttributeTok{radius\_mode  =} \StringTok{"sparse"}\NormalTok{,}
 \AttributeTok{extract\_fun  =} \StringTok{"mean"}\NormalTok{,}
 \AttributeTok{fill\_missing  =} \ConstantTok{TRUE}\NormalTok{,}
 \AttributeTok{IDW\_weight   =} \DecValTok{2}\NormalTok{,}
 \AttributeTok{future\_max\_size =} \DecValTok{5} \SpecialCharTok{*} \DecValTok{1024}\SpecialCharTok{\^{}}\DecValTok{3}\NormalTok{)}

\CommentTok{\# Diversity\_Total\_r3000.tif egv\_108}
\NormalTok{slanis}\OtherTok{=}\FunctionTok{rast}\NormalTok{(}\StringTok{"./RasterGrids\_100m/2024/RAW/Diversity\_Total\_r3000.tif"}\NormalTok{)}
\FunctionTok{names}\NormalTok{(slanis)}\OtherTok{=}\StringTok{"egv\_108"}
\NormalTok{slanis2}\OtherTok{=}\FunctionTok{project}\NormalTok{(slanis,template100)}
\FunctionTok{writeRaster}\NormalTok{(slanis2,}
      \StringTok{"./RasterGrids\_100m/2024/RAW/Diversity\_Total\_r3000.tif"}\NormalTok{,}
      \AttributeTok{overwrite=}\ConstantTok{TRUE}\NormalTok{)}

\CommentTok{\# standardisation {-}{-}{-}{-}}
\ControlFlowTok{if}\NormalTok{(}\SpecialCharTok{!}\FunctionTok{require}\NormalTok{(terra)) \{}\FunctionTok{install.packages}\NormalTok{(}\StringTok{"terra"}\NormalTok{); }\FunctionTok{require}\NormalTok{(terra)\}}
\ControlFlowTok{if}\NormalTok{(}\SpecialCharTok{!}\FunctionTok{require}\NormalTok{(tidyverse)) \{}\FunctionTok{install.packages}\NormalTok{(}\StringTok{"tidyverse"}\NormalTok{); }\FunctionTok{require}\NormalTok{(tidyverse)\}}

\NormalTok{nosaukums}\OtherTok{=}\StringTok{"Diversity\_Total\_r3000.tif"}
\NormalTok{ielasisanas\_cels}\OtherTok{=}\FunctionTok{paste0}\NormalTok{(}\StringTok{"./RasterGrids\_100m/2024/RAW/"}\NormalTok{,nosaukums)}
\NormalTok{saglabasanas\_cels}\OtherTok{=}\FunctionTok{paste0}\NormalTok{(}\StringTok{"./RasterGrids\_100m/2024/Scaled/"}\NormalTok{,nosaukums)}
\NormalTok{slanis}\OtherTok{=}\FunctionTok{rast}\NormalTok{(ielasisanas\_cels)}
\NormalTok{videjais}\OtherTok{=}\FunctionTok{global}\NormalTok{(slanis,}\AttributeTok{fun=}\StringTok{"mean"}\NormalTok{,}\AttributeTok{na.rm=}\ConstantTok{TRUE}\NormalTok{)}
\NormalTok{centrets}\OtherTok{=}\NormalTok{slanis}\SpecialCharTok{{-}}\NormalTok{videjais[,}\DecValTok{1}\NormalTok{]}
\NormalTok{standartnovirze}\OtherTok{=}\NormalTok{terra}\SpecialCharTok{::}\FunctionTok{global}\NormalTok{(centrets,}\AttributeTok{fun=}\StringTok{"rms"}\NormalTok{,}\AttributeTok{na.rm=}\ConstantTok{TRUE}\NormalTok{)}
\NormalTok{merogots}\OtherTok{=}\NormalTok{centrets}\SpecialCharTok{/}\NormalTok{standartnovirze[,}\DecValTok{1}\NormalTok{]}
\FunctionTok{writeRaster}\NormalTok{(merogots,}
      \AttributeTok{filename=}\NormalTok{saglabasanas\_cels,}
      \AttributeTok{overwrite=}\ConstantTok{TRUE}\NormalTok{)}
\end{Highlighting}
\end{Shaded}

\section{Diversity\_Total\_r10000}\label{ch06.109}

\textbf{filename:} \texttt{Diversity\_Total\_r10000.tif}

\textbf{layername:} \texttt{egv\_109}

\textbf{English name:} Average combined landscape α-diversity of 500 m grid cells
within the 10 km landscape

\textbf{Latvian name:} Vidējā visu ainavas klašu 500 m šūnu α-daudzveidība 10 km
ainavā

\textbf{Procedure:} Derived from the \hyperref[Ch05.04]{Landscape diversity}, more precisely
\hyperref[Ch05.04.01]{Overall landscape diversity}. The average value of 25 ha cells diversity index
values is calculated using the workflow \texttt{egvtools::radius\_function()}. To
prevent possible data loss at edge cells, inverse distance weighted
(power = 2) gap filling is implemented. File is written twice, to ensure
layername. Finally, the layer is standardised
by subtracting the arithmetic mean and dividing by the root mean squared error.

\begin{Shaded}
\begin{Highlighting}[]
\CommentTok{\# libs {-}{-}{-}{-}}
\ControlFlowTok{if}\NormalTok{(}\SpecialCharTok{!}\FunctionTok{require}\NormalTok{(egvtools)) \{remotes}\SpecialCharTok{::}\FunctionTok{install\_github}\NormalTok{(}\StringTok{"aavotins/egvtools"}\NormalTok{); }\FunctionTok{require}\NormalTok{(egvtools)\}}
\ControlFlowTok{if}\NormalTok{(}\SpecialCharTok{!}\FunctionTok{require}\NormalTok{(terra)) \{}\FunctionTok{install.packages}\NormalTok{(}\StringTok{"terra"}\NormalTok{); }\FunctionTok{require}\NormalTok{(terra)\}}

\CommentTok{\# templates {-}{-}{-}{-}}
\NormalTok{template100}\OtherTok{=}\FunctionTok{rast}\NormalTok{(}\StringTok{"./Templates/TemplateRasters/LV100m\_10km.tif"}\NormalTok{)}

\CommentTok{\# radii}
\FunctionTok{radius\_function}\NormalTok{(}
 \AttributeTok{kvadrati\_path =} \StringTok{"./Templates/TemplateGrids/tiles/"}\NormalTok{,}
 \AttributeTok{radii\_path   =} \StringTok{"./Templates/TemplateGridPoints/tiles/"}\NormalTok{,}
 \AttributeTok{tikls100\_path =} \StringTok{"./Templates/TemplateGrids/tikls100\_sauzeme.parquet"}\NormalTok{,}
 \AttributeTok{template\_path =} \StringTok{"./Templates/TemplateRasters/LV100m\_10km.tif"}\NormalTok{,}
 \AttributeTok{input\_layers  =} \FunctionTok{c}\NormalTok{(}\StringTok{"./RasterGrids\_500m/2024/Diversity\_GeneralLandscape\_500x.tif"}\NormalTok{),}
 \AttributeTok{layer\_prefixes =} \FunctionTok{c}\NormalTok{(}\StringTok{"Diversity\_Total"}\NormalTok{),}
 \AttributeTok{output\_dir   =} \StringTok{"./RasterGrids\_100m/2024/RAW/"}\NormalTok{,}
 \AttributeTok{n\_workers   =} \DecValTok{12}\NormalTok{,}
 \AttributeTok{radii     =} \FunctionTok{c}\NormalTok{(}\StringTok{"r10000"}\NormalTok{),}
 \AttributeTok{radius\_mode  =} \StringTok{"sparse"}\NormalTok{,}
 \AttributeTok{extract\_fun  =} \StringTok{"mean"}\NormalTok{,}
 \AttributeTok{fill\_missing  =} \ConstantTok{TRUE}\NormalTok{,}
 \AttributeTok{IDW\_weight   =} \DecValTok{2}\NormalTok{,}
 \AttributeTok{future\_max\_size =} \DecValTok{5} \SpecialCharTok{*} \DecValTok{1024}\SpecialCharTok{\^{}}\DecValTok{3}\NormalTok{)}

\CommentTok{\# Diversity\_Total\_r10000.tif    egv\_109}
\NormalTok{slanis}\OtherTok{=}\FunctionTok{rast}\NormalTok{(}\StringTok{"./RasterGrids\_100m/2024/RAW/Diversity\_Total\_r10000.tif"}\NormalTok{)}
\FunctionTok{names}\NormalTok{(slanis)}\OtherTok{=}\StringTok{"egv\_109"}
\NormalTok{slanis2}\OtherTok{=}\FunctionTok{project}\NormalTok{(slanis,template100)}
\FunctionTok{writeRaster}\NormalTok{(slanis2,}
      \StringTok{"./RasterGrids\_100m/2024/RAW/Diversity\_Total\_r10000.tif"}\NormalTok{,}
      \AttributeTok{overwrite=}\ConstantTok{TRUE}\NormalTok{)}

\CommentTok{\# standardisation {-}{-}{-}{-}}
\ControlFlowTok{if}\NormalTok{(}\SpecialCharTok{!}\FunctionTok{require}\NormalTok{(terra)) \{}\FunctionTok{install.packages}\NormalTok{(}\StringTok{"terra"}\NormalTok{); }\FunctionTok{require}\NormalTok{(terra)\}}
\ControlFlowTok{if}\NormalTok{(}\SpecialCharTok{!}\FunctionTok{require}\NormalTok{(tidyverse)) \{}\FunctionTok{install.packages}\NormalTok{(}\StringTok{"tidyverse"}\NormalTok{); }\FunctionTok{require}\NormalTok{(tidyverse)\}}

\NormalTok{nosaukums}\OtherTok{=}\StringTok{"Diversity\_Total\_r10000.tif"}
\NormalTok{ielasisanas\_cels}\OtherTok{=}\FunctionTok{paste0}\NormalTok{(}\StringTok{"./RasterGrids\_100m/2024/RAW/"}\NormalTok{,nosaukums)}
\NormalTok{saglabasanas\_cels}\OtherTok{=}\FunctionTok{paste0}\NormalTok{(}\StringTok{"./RasterGrids\_100m/2024/Scaled/"}\NormalTok{,nosaukums)}
\NormalTok{slanis}\OtherTok{=}\FunctionTok{rast}\NormalTok{(ielasisanas\_cels)}
\NormalTok{videjais}\OtherTok{=}\FunctionTok{global}\NormalTok{(slanis,}\AttributeTok{fun=}\StringTok{"mean"}\NormalTok{,}\AttributeTok{na.rm=}\ConstantTok{TRUE}\NormalTok{)}
\NormalTok{centrets}\OtherTok{=}\NormalTok{slanis}\SpecialCharTok{{-}}\NormalTok{videjais[,}\DecValTok{1}\NormalTok{]}
\NormalTok{standartnovirze}\OtherTok{=}\NormalTok{terra}\SpecialCharTok{::}\FunctionTok{global}\NormalTok{(centrets,}\AttributeTok{fun=}\StringTok{"rms"}\NormalTok{,}\AttributeTok{na.rm=}\ConstantTok{TRUE}\NormalTok{)}
\NormalTok{merogots}\OtherTok{=}\NormalTok{centrets}\SpecialCharTok{/}\NormalTok{standartnovirze[,}\DecValTok{1}\NormalTok{]}
\FunctionTok{writeRaster}\NormalTok{(merogots,}
      \AttributeTok{filename=}\NormalTok{saglabasanas\_cels,}
      \AttributeTok{overwrite=}\ConstantTok{TRUE}\NormalTok{)}
\end{Highlighting}
\end{Shaded}

\section{Edges\_Bogs-Trees\_cell}\label{ch06.110}

\textbf{filename:} \texttt{Edges\_Bogs-Trees\_cell.tif}

\textbf{layername:} \texttt{egv\_110}

\textbf{English name:} Edge pixels of Bogs, Mires bordering with Trees within the
analysis cell (1 ha)

\textbf{Latvian name:} Purvu malu ar kokiem pikseļu skaits analīzes šūnā (1 ha)

\textbf{Procedure:} First, values from 620 to 700 from the \hyperref[Ch05.03]{Landscape
classification} are coded as 0, and all other values as NA. Then bog and
transitional mire layers from the \hyperref[Ch04.17]{EDI} are reclassified to presence-only
(value 1) and combined. Then, bog-and-mire layer (1 = presence) is covered over
tree layer (presence = 0) and written to file (matching the input). Then, with
the workflow \texttt{egvtools::landscape\_function()} total edge between the two classes
is calculated. During the calculation of the landscape metric, inverse distance weighted
(power = 2) gap filling on the output is applied to ensure no missing values
at the edges. Finally, the layer is standardised by subtracting the arithmetic
mean and dividing by the root mean squared error.

\begin{Shaded}
\begin{Highlighting}[]
\CommentTok{\# libs {-}{-}{-}{-}}
\ControlFlowTok{if}\NormalTok{(}\SpecialCharTok{!}\FunctionTok{require}\NormalTok{(terra)) \{}\FunctionTok{install.packages}\NormalTok{(}\StringTok{"terra"}\NormalTok{); }\FunctionTok{require}\NormalTok{(terra)\}}
\ControlFlowTok{if}\NormalTok{(}\SpecialCharTok{!}\FunctionTok{require}\NormalTok{(egvtools)) \{remotes}\SpecialCharTok{::}\FunctionTok{install\_github}\NormalTok{(}\StringTok{"aavotins/egvtools"}\NormalTok{); }\FunctionTok{require}\NormalTok{(egvtools)\}}

\ControlFlowTok{if}\NormalTok{(}\SpecialCharTok{!}\FunctionTok{require}\NormalTok{(sf)) \{}\FunctionTok{install.packages}\NormalTok{(}\StringTok{"sf"}\NormalTok{); }\FunctionTok{require}\NormalTok{(sf)\}}
\ControlFlowTok{if}\NormalTok{(}\SpecialCharTok{!}\FunctionTok{require}\NormalTok{(sfarrow)) \{}\FunctionTok{install.packages}\NormalTok{(}\StringTok{"sfarrow"}\NormalTok{); }\FunctionTok{require}\NormalTok{(sfarrow)\}}
\ControlFlowTok{if}\NormalTok{(}\SpecialCharTok{!}\FunctionTok{require}\NormalTok{(raster)) \{}\FunctionTok{install.packages}\NormalTok{(}\StringTok{"raster"}\NormalTok{); }\FunctionTok{require}\NormalTok{(raster)\}}
\ControlFlowTok{if}\NormalTok{(}\SpecialCharTok{!}\FunctionTok{require}\NormalTok{(fasterize)) \{}\FunctionTok{install.packages}\NormalTok{(}\StringTok{"fasterize"}\NormalTok{); }\FunctionTok{require}\NormalTok{(fasterize)\}}
\ControlFlowTok{if}\NormalTok{(}\SpecialCharTok{!}\FunctionTok{require}\NormalTok{(tidyverse)) \{}\FunctionTok{install.packages}\NormalTok{(}\StringTok{"tidyverse"}\NormalTok{); }\FunctionTok{require}\NormalTok{(tidyverse)\}}


\CommentTok{\# Templates {-}{-}{-}{-}{-}}
\NormalTok{template10}\OtherTok{=}\FunctionTok{rast}\NormalTok{(}\StringTok{"./Templates/TemplateRasters/LV10m\_10km.tif"}\NormalTok{)}
\NormalTok{nulls10}\OtherTok{=}\FunctionTok{rast}\NormalTok{(}\StringTok{"./Templates/TemplateRasters/nulls\_LV10m\_10km.tif"}\NormalTok{)}

\CommentTok{\# simple landscape {-}{-}{-}{-}}
\NormalTok{simple\_landscape}\OtherTok{=}\FunctionTok{rast}\NormalTok{(}\StringTok{"./RasterGrids\_10m/2024/Ainava\_vienk\_mask.tif"}\NormalTok{)}

\CommentTok{\# Edges\_Bogs{-}Trees\_input.tif {-}{-}{-}{-}}

\NormalTok{trees\_from620}\OtherTok{=}\FunctionTok{ifel}\NormalTok{(simple\_landscape}\SpecialCharTok{\textgreater{}=}\DecValTok{620} \SpecialCharTok{\&}\NormalTok{ simple\_landscape}\SpecialCharTok{\textless{}}\DecValTok{700}\NormalTok{,}\DecValTok{0}\NormalTok{,}\ConstantTok{NA}\NormalTok{)}
\FunctionTok{plot}\NormalTok{(trees\_from620)}

\NormalTok{bogs}\OtherTok{=}\FunctionTok{rast}\NormalTok{(}\StringTok{"./RasterGrids\_10m/2024/EDI\_BogsYN.tif"}\NormalTok{)}
\NormalTok{bogs}\OtherTok{=}\FunctionTok{subst}\NormalTok{(bogs,}\DecValTok{0}\NormalTok{,}\ConstantTok{NA}\NormalTok{)}
\FunctionTok{plot}\NormalTok{(bogs)}
\NormalTok{mires}\OtherTok{=}\FunctionTok{rast}\NormalTok{(}\StringTok{"./RasterGrids\_10m/2024/EDI\_TransitionalMiresYN.tif"}\NormalTok{)}
\NormalTok{mires}\OtherTok{=}\FunctionTok{subst}\NormalTok{(mires,}\DecValTok{0}\NormalTok{,}\ConstantTok{NA}\NormalTok{)}
\FunctionTok{plot}\NormalTok{(mires)}
\NormalTok{bogs\_mires}\OtherTok{=}\FunctionTok{cover}\NormalTok{(bogs,mires)}
\FunctionTok{plot}\NormalTok{(bogs\_mires)}

\NormalTok{bm\_trees}\OtherTok{=}\FunctionTok{cover}\NormalTok{(bogs\_mires,trees\_from620)}
\FunctionTok{plot}\NormalTok{(bm\_trees)}

\NormalTok{edge\_bm\_trees}\OtherTok{=}\FunctionTok{project}\NormalTok{(bm\_trees,template10,}
           \AttributeTok{filename=}\StringTok{"./RasterGrids\_10m/2024/Edges\_Bogs{-}Trees\_input.tif"}\NormalTok{,}
           \AttributeTok{overwrite=}\ConstantTok{TRUE}\NormalTok{)}

\FunctionTok{rm}\NormalTok{(edge\_bm\_trees)}
\FunctionTok{rm}\NormalTok{(bm\_trees)}

\CommentTok{\# Edges\_Bogs{-}Trees\_cell.tif egv\_110}

\FunctionTok{landscape\_function}\NormalTok{(}
 \AttributeTok{landscape   =} \StringTok{"./RasterGrids\_10m/2024/Edges\_Bogs{-}Trees\_input.tif"}\NormalTok{,}
 \AttributeTok{zones     =} \StringTok{"./Templates/TemplateGrids/tikls100\_sauzeme.parquet"}\NormalTok{,}
 \AttributeTok{id\_field    =} \StringTok{"id"}\NormalTok{,}
 \AttributeTok{tile\_field   =} \StringTok{"tks50km"}\NormalTok{,}
 \AttributeTok{template    =} \StringTok{"./Templates/TemplateRasters/LV100m\_10km.tif"}\NormalTok{,}
 \AttributeTok{out\_dir    =} \StringTok{"./RasterGrids\_100m/2024/RAW"}\NormalTok{,}
 \AttributeTok{out\_filename  =} \StringTok{"Edges\_Bogs{-}Trees\_cell.tif"}\NormalTok{,}
 \AttributeTok{out\_layername =} \StringTok{"egv\_110"}\NormalTok{,}
 \AttributeTok{what       =} \StringTok{"lsm\_l\_te"}\NormalTok{,}
 \AttributeTok{lm\_args     =} \FunctionTok{list}\NormalTok{(}\AttributeTok{count\_boundary =} \ConstantTok{FALSE}\NormalTok{),}
 \AttributeTok{rasterize\_engine =} \StringTok{"fasterize"}\NormalTok{,}
 \AttributeTok{n\_workers   =} \DecValTok{12}\NormalTok{,}
 \AttributeTok{future\_max\_size =} \DecValTok{20} \SpecialCharTok{*} \DecValTok{1024}\SpecialCharTok{\^{}}\DecValTok{3}\NormalTok{,}
 \AttributeTok{fill\_gaps   =} \ConstantTok{TRUE}\NormalTok{,}
 \AttributeTok{plot\_gaps   =} \ConstantTok{FALSE}\NormalTok{,}
 \AttributeTok{plot\_result  =} \ConstantTok{FALSE}
\NormalTok{)}

\CommentTok{\# standardisation {-}{-}{-}{-}}
\ControlFlowTok{if}\NormalTok{(}\SpecialCharTok{!}\FunctionTok{require}\NormalTok{(terra)) \{}\FunctionTok{install.packages}\NormalTok{(}\StringTok{"terra"}\NormalTok{); }\FunctionTok{require}\NormalTok{(terra)\}}
\ControlFlowTok{if}\NormalTok{(}\SpecialCharTok{!}\FunctionTok{require}\NormalTok{(tidyverse)) \{}\FunctionTok{install.packages}\NormalTok{(}\StringTok{"tidyverse"}\NormalTok{); }\FunctionTok{require}\NormalTok{(tidyverse)\}}

\NormalTok{nosaukums}\OtherTok{=}\StringTok{"Edges\_Bogs{-}Trees\_cell.tif"}
\NormalTok{ielasisanas\_cels}\OtherTok{=}\FunctionTok{paste0}\NormalTok{(}\StringTok{"./RasterGrids\_100m/2024/RAW/"}\NormalTok{,nosaukums)}
\NormalTok{saglabasanas\_cels}\OtherTok{=}\FunctionTok{paste0}\NormalTok{(}\StringTok{"./RasterGrids\_100m/2024/Scaled/"}\NormalTok{,nosaukums)}
\NormalTok{slanis}\OtherTok{=}\FunctionTok{rast}\NormalTok{(ielasisanas\_cels)}
\NormalTok{videjais}\OtherTok{=}\FunctionTok{global}\NormalTok{(slanis,}\AttributeTok{fun=}\StringTok{"mean"}\NormalTok{,}\AttributeTok{na.rm=}\ConstantTok{TRUE}\NormalTok{)}
\NormalTok{centrets}\OtherTok{=}\NormalTok{slanis}\SpecialCharTok{{-}}\NormalTok{videjais[,}\DecValTok{1}\NormalTok{]}
\NormalTok{standartnovirze}\OtherTok{=}\NormalTok{terra}\SpecialCharTok{::}\FunctionTok{global}\NormalTok{(centrets,}\AttributeTok{fun=}\StringTok{"rms"}\NormalTok{,}\AttributeTok{na.rm=}\ConstantTok{TRUE}\NormalTok{)}
\NormalTok{merogots}\OtherTok{=}\NormalTok{centrets}\SpecialCharTok{/}\NormalTok{standartnovirze[,}\DecValTok{1}\NormalTok{]}
\FunctionTok{writeRaster}\NormalTok{(merogots,}
      \AttributeTok{filename=}\NormalTok{saglabasanas\_cels,}
      \AttributeTok{overwrite=}\ConstantTok{TRUE}\NormalTok{)}
\end{Highlighting}
\end{Shaded}

\section{Edges\_Bogs-Trees\_r500}\label{ch06.111}

\textbf{filename:} \texttt{Edges\_Bogs-Trees\_r500.tif}

\textbf{layername:} \texttt{egv\_111}

\textbf{English name:} Edge pixels of Bogs, Mires bordering with Trees within the 0.5
km landscape

\textbf{Latvian name:} Purvu malu ar kokiem pikseļu skaits 0,5 km ainavā

\textbf{Procedure:} The total edge within a 500 m radius around the analysis grid cell is
calculated as the area-weighted sum of the \hyperref[ch06.110]{analysis cells} inside the
buffer, using the workflow \texttt{egvtools::radius\_function()}. During the calculation of the landscape metric,
inverse distance weighted (power = 2) gap filling on the output is applied
to ensure no missing values at the edges. Then the layer is rewritten to set
its name. Finally, the layer is standardised by subtracting the arithmetic
mean and dividing by the root mean squared error.

\begin{Shaded}
\begin{Highlighting}[]
\CommentTok{\# libs {-}{-}{-}{-}}
\ControlFlowTok{if}\NormalTok{(}\SpecialCharTok{!}\FunctionTok{require}\NormalTok{(terra)) \{}\FunctionTok{install.packages}\NormalTok{(}\StringTok{"terra"}\NormalTok{); }\FunctionTok{require}\NormalTok{(terra)\}}
\ControlFlowTok{if}\NormalTok{(}\SpecialCharTok{!}\FunctionTok{require}\NormalTok{(egvtools)) \{remotes}\SpecialCharTok{::}\FunctionTok{install\_github}\NormalTok{(}\StringTok{"aavotins/egvtools"}\NormalTok{); }\FunctionTok{require}\NormalTok{(egvtools)\}}


\CommentTok{\# Templates {-}{-}{-}{-}{-}}
\NormalTok{template100}\OtherTok{=}\FunctionTok{rast}\NormalTok{(}\StringTok{"./Templates/TemplateRasters/LV100m\_10km.tif"}\NormalTok{)}

\CommentTok{\# radii}
\FunctionTok{radius\_function}\NormalTok{(}
 \AttributeTok{kvadrati\_path =} \StringTok{"./Templates/TemplateGrids/tiles/"}\NormalTok{,}
 \AttributeTok{radii\_path   =} \StringTok{"./Templates/TemplateGridPoints/tiles/"}\NormalTok{,}
 \AttributeTok{tikls100\_path =} \StringTok{"./Templates/TemplateGrids/tikls100\_sauzeme.parquet"}\NormalTok{,}
 \AttributeTok{template\_path =} \StringTok{"./Templates/TemplateRasters/LV100m\_10km.tif"}\NormalTok{,}
 \AttributeTok{input\_layers  =} \FunctionTok{c}\NormalTok{(}\StringTok{"./RasterGrids\_100m/2024/RAW/Edges\_Bogs{-}Trees\_cell.tif"}\NormalTok{),}
 \AttributeTok{layer\_prefixes =} \FunctionTok{c}\NormalTok{(}\StringTok{"Edges\_Bogs{-}Trees"}\NormalTok{),}
 \AttributeTok{output\_dir   =} \StringTok{"./RasterGrids\_100m/2024/RAW/"}\NormalTok{,}
 \AttributeTok{n\_workers   =} \DecValTok{4}\NormalTok{,}
 \AttributeTok{radii     =} \FunctionTok{c}\NormalTok{(}\StringTok{"r500"}\NormalTok{),}
 \AttributeTok{radius\_mode  =} \StringTok{"sparse"}\NormalTok{,}
 \AttributeTok{extract\_fun  =} \StringTok{"sum"}\NormalTok{,}
 \AttributeTok{fill\_missing  =} \ConstantTok{TRUE}\NormalTok{,}
 \AttributeTok{IDW\_weight   =} \DecValTok{2}\NormalTok{,}
 \AttributeTok{future\_max\_size =} \DecValTok{20} \SpecialCharTok{*} \DecValTok{1024}\SpecialCharTok{\^{}}\DecValTok{3}\NormalTok{)}

\CommentTok{\# Edges\_Bogs{-}Trees\_r500.tif egv\_111}
\NormalTok{slanis}\OtherTok{=}\FunctionTok{rast}\NormalTok{(}\StringTok{"./RasterGrids\_100m/2024/RAW/Edges\_Bogs{-}Trees\_r500.tif"}\NormalTok{)}
\FunctionTok{names}\NormalTok{(slanis)}\OtherTok{=}\StringTok{"egv\_111"}
\NormalTok{slanis2}\OtherTok{=}\FunctionTok{project}\NormalTok{(slanis,template100)}
\FunctionTok{writeRaster}\NormalTok{(slanis2,}
      \StringTok{"./RasterGrids\_100m/2024/RAW/Edges\_Bogs{-}Trees\_r500.tif"}\NormalTok{,}
      \AttributeTok{overwrite=}\ConstantTok{TRUE}\NormalTok{)}

\CommentTok{\# standardisation {-}{-}{-}{-}}
\ControlFlowTok{if}\NormalTok{(}\SpecialCharTok{!}\FunctionTok{require}\NormalTok{(terra)) \{}\FunctionTok{install.packages}\NormalTok{(}\StringTok{"terra"}\NormalTok{); }\FunctionTok{require}\NormalTok{(terra)\}}
\ControlFlowTok{if}\NormalTok{(}\SpecialCharTok{!}\FunctionTok{require}\NormalTok{(tidyverse)) \{}\FunctionTok{install.packages}\NormalTok{(}\StringTok{"tidyverse"}\NormalTok{); }\FunctionTok{require}\NormalTok{(tidyverse)\}}

\NormalTok{nosaukums}\OtherTok{=}\StringTok{"Edges\_Bogs{-}Trees\_r500.tif"}
\NormalTok{ielasisanas\_cels}\OtherTok{=}\FunctionTok{paste0}\NormalTok{(}\StringTok{"./RasterGrids\_100m/2024/RAW/"}\NormalTok{,nosaukums)}
\NormalTok{saglabasanas\_cels}\OtherTok{=}\FunctionTok{paste0}\NormalTok{(}\StringTok{"./RasterGrids\_100m/2024/Scaled/"}\NormalTok{,nosaukums)}
\NormalTok{slanis}\OtherTok{=}\FunctionTok{rast}\NormalTok{(ielasisanas\_cels)}
\NormalTok{videjais}\OtherTok{=}\FunctionTok{global}\NormalTok{(slanis,}\AttributeTok{fun=}\StringTok{"mean"}\NormalTok{,}\AttributeTok{na.rm=}\ConstantTok{TRUE}\NormalTok{)}
\NormalTok{centrets}\OtherTok{=}\NormalTok{slanis}\SpecialCharTok{{-}}\NormalTok{videjais[,}\DecValTok{1}\NormalTok{]}
\NormalTok{standartnovirze}\OtherTok{=}\NormalTok{terra}\SpecialCharTok{::}\FunctionTok{global}\NormalTok{(centrets,}\AttributeTok{fun=}\StringTok{"rms"}\NormalTok{,}\AttributeTok{na.rm=}\ConstantTok{TRUE}\NormalTok{)}
\NormalTok{merogots}\OtherTok{=}\NormalTok{centrets}\SpecialCharTok{/}\NormalTok{standartnovirze[,}\DecValTok{1}\NormalTok{]}
\FunctionTok{writeRaster}\NormalTok{(merogots,}
      \AttributeTok{filename=}\NormalTok{saglabasanas\_cels,}
      \AttributeTok{overwrite=}\ConstantTok{TRUE}\NormalTok{)}
\end{Highlighting}
\end{Shaded}

\section{Edges\_Bogs-Trees\_r1250}\label{ch06.112}

\textbf{filename:} \texttt{Edges\_Bogs-Trees\_r1250.tif}

\textbf{layername:} \texttt{egv\_112}

\textbf{English name:} Edge pixels of Bogs, Mires bordering with Trees within the
1.25 km landscape

\textbf{Latvian name:} Purvu malu ar kokiem pikseļu skaits 1,25 km ainavā

\textbf{Procedure:} The total edge within a 1250 m radius around the analysis grid cell is
calculated as the area-weighted sum of the \hyperref[ch06.110]{analysis cells} inside the
buffer, using the workflow \texttt{egvtools::radius\_function()}. During the calculation of the landscape metric,
inverse distance weighted (power = 2) gap filling on the output is applied
to ensure no missing values at the edges. Then the layer is rewritten to set
its name. Finally, the layer is standardised by subtracting the arithmetic
mean and dividing by the root mean squared error.

\begin{Shaded}
\begin{Highlighting}[]
\CommentTok{\# libs {-}{-}{-}{-}}
\ControlFlowTok{if}\NormalTok{(}\SpecialCharTok{!}\FunctionTok{require}\NormalTok{(terra)) \{}\FunctionTok{install.packages}\NormalTok{(}\StringTok{"terra"}\NormalTok{); }\FunctionTok{require}\NormalTok{(terra)\}}
\ControlFlowTok{if}\NormalTok{(}\SpecialCharTok{!}\FunctionTok{require}\NormalTok{(egvtools)) \{remotes}\SpecialCharTok{::}\FunctionTok{install\_github}\NormalTok{(}\StringTok{"aavotins/egvtools"}\NormalTok{); }\FunctionTok{require}\NormalTok{(egvtools)\}}


\CommentTok{\# Templates {-}{-}{-}{-}{-}}
\NormalTok{template100}\OtherTok{=}\FunctionTok{rast}\NormalTok{(}\StringTok{"./Templates/TemplateRasters/LV100m\_10km.tif"}\NormalTok{)}

\CommentTok{\# radii}
\FunctionTok{radius\_function}\NormalTok{(}
 \AttributeTok{kvadrati\_path =} \StringTok{"./Templates/TemplateGrids/tiles/"}\NormalTok{,}
 \AttributeTok{radii\_path   =} \StringTok{"./Templates/TemplateGridPoints/tiles/"}\NormalTok{,}
 \AttributeTok{tikls100\_path =} \StringTok{"./Templates/TemplateGrids/tikls100\_sauzeme.parquet"}\NormalTok{,}
 \AttributeTok{template\_path =} \StringTok{"./Templates/TemplateRasters/LV100m\_10km.tif"}\NormalTok{,}
 \AttributeTok{input\_layers  =} \FunctionTok{c}\NormalTok{(}\StringTok{"./RasterGrids\_100m/2024/RAW/Edges\_Bogs{-}Trees\_cell.tif"}\NormalTok{),}
 \AttributeTok{layer\_prefixes =} \FunctionTok{c}\NormalTok{(}\StringTok{"Edges\_Bogs{-}Trees"}\NormalTok{),}
 \AttributeTok{output\_dir   =} \StringTok{"./RasterGrids\_100m/2024/RAW/"}\NormalTok{,}
 \AttributeTok{n\_workers   =} \DecValTok{4}\NormalTok{,}
 \AttributeTok{radii     =} \FunctionTok{c}\NormalTok{(}\StringTok{"r1250"}\NormalTok{),}
 \AttributeTok{radius\_mode  =} \StringTok{"sparse"}\NormalTok{,}
 \AttributeTok{extract\_fun  =} \StringTok{"sum"}\NormalTok{,}
 \AttributeTok{fill\_missing  =} \ConstantTok{TRUE}\NormalTok{,}
 \AttributeTok{IDW\_weight   =} \DecValTok{2}\NormalTok{,}
 \AttributeTok{future\_max\_size =} \DecValTok{20} \SpecialCharTok{*} \DecValTok{1024}\SpecialCharTok{\^{}}\DecValTok{3}\NormalTok{)}

\CommentTok{\# Edges\_Bogs{-}Trees\_r1250.tif    egv\_112}
\NormalTok{slanis}\OtherTok{=}\FunctionTok{rast}\NormalTok{(}\StringTok{"./RasterGrids\_100m/2024/RAW/Edges\_Bogs{-}Trees\_r1250.tif"}\NormalTok{)}
\FunctionTok{names}\NormalTok{(slanis)}\OtherTok{=}\StringTok{"egv\_112"}
\NormalTok{slanis2}\OtherTok{=}\FunctionTok{project}\NormalTok{(slanis,template100)}
\FunctionTok{writeRaster}\NormalTok{(slanis2,}
      \StringTok{"./RasterGrids\_100m/2024/RAW/Edges\_Bogs{-}Trees\_r1250.tif"}\NormalTok{,}
      \AttributeTok{overwrite=}\ConstantTok{TRUE}\NormalTok{)}

\CommentTok{\# standardisation {-}{-}{-}{-}}
\ControlFlowTok{if}\NormalTok{(}\SpecialCharTok{!}\FunctionTok{require}\NormalTok{(terra)) \{}\FunctionTok{install.packages}\NormalTok{(}\StringTok{"terra"}\NormalTok{); }\FunctionTok{require}\NormalTok{(terra)\}}
\ControlFlowTok{if}\NormalTok{(}\SpecialCharTok{!}\FunctionTok{require}\NormalTok{(tidyverse)) \{}\FunctionTok{install.packages}\NormalTok{(}\StringTok{"tidyverse"}\NormalTok{); }\FunctionTok{require}\NormalTok{(tidyverse)\}}

\NormalTok{nosaukums}\OtherTok{=}\StringTok{"Edges\_Bogs{-}Trees\_r1250.tif"}
\NormalTok{ielasisanas\_cels}\OtherTok{=}\FunctionTok{paste0}\NormalTok{(}\StringTok{"./RasterGrids\_100m/2024/RAW/"}\NormalTok{,nosaukums)}
\NormalTok{saglabasanas\_cels}\OtherTok{=}\FunctionTok{paste0}\NormalTok{(}\StringTok{"./RasterGrids\_100m/2024/Scaled/"}\NormalTok{,nosaukums)}
\NormalTok{slanis}\OtherTok{=}\FunctionTok{rast}\NormalTok{(ielasisanas\_cels)}
\NormalTok{videjais}\OtherTok{=}\FunctionTok{global}\NormalTok{(slanis,}\AttributeTok{fun=}\StringTok{"mean"}\NormalTok{,}\AttributeTok{na.rm=}\ConstantTok{TRUE}\NormalTok{)}
\NormalTok{centrets}\OtherTok{=}\NormalTok{slanis}\SpecialCharTok{{-}}\NormalTok{videjais[,}\DecValTok{1}\NormalTok{]}
\NormalTok{standartnovirze}\OtherTok{=}\NormalTok{terra}\SpecialCharTok{::}\FunctionTok{global}\NormalTok{(centrets,}\AttributeTok{fun=}\StringTok{"rms"}\NormalTok{,}\AttributeTok{na.rm=}\ConstantTok{TRUE}\NormalTok{)}
\NormalTok{merogots}\OtherTok{=}\NormalTok{centrets}\SpecialCharTok{/}\NormalTok{standartnovirze[,}\DecValTok{1}\NormalTok{]}
\FunctionTok{writeRaster}\NormalTok{(merogots,}
      \AttributeTok{filename=}\NormalTok{saglabasanas\_cels,}
      \AttributeTok{overwrite=}\ConstantTok{TRUE}\NormalTok{)}
\end{Highlighting}
\end{Shaded}

\section{Edges\_Bogs-Trees\_r3000}\label{ch06.113}

\textbf{filename:} \texttt{Edges\_Bogs-Trees\_r3000.tif}

\textbf{layername:} \texttt{egv\_113}

\textbf{English name:} Edge pixels of Bogs, Mires bordering with Trees within the 3
km landscape

\textbf{Latvian name:} Purvu malu ar kokiem pikseļu skaits 3 km ainavā

\textbf{Procedure:} Total edge within a 3000 m radius around the analysis grid cell is
calculated as the area-weighted sum of the \hyperref[ch06.110]{analysis cells} inside the
buffer, using the workflow \texttt{egvtools::radius\_function()}. During the calculation of the landscape metric,
inverse distance weighted (power = 2) gap filling on the output is applied
to ensure no missing values at the edges. Then the layer is rewritten to set
its name. Finally, the layer is standardised by subtracting the arithmetic
mean and dividing by the root mean squared error.

\begin{Shaded}
\begin{Highlighting}[]
\CommentTok{\# libs {-}{-}{-}{-}}
\ControlFlowTok{if}\NormalTok{(}\SpecialCharTok{!}\FunctionTok{require}\NormalTok{(terra)) \{}\FunctionTok{install.packages}\NormalTok{(}\StringTok{"terra"}\NormalTok{); }\FunctionTok{require}\NormalTok{(terra)\}}
\ControlFlowTok{if}\NormalTok{(}\SpecialCharTok{!}\FunctionTok{require}\NormalTok{(egvtools)) \{remotes}\SpecialCharTok{::}\FunctionTok{install\_github}\NormalTok{(}\StringTok{"aavotins/egvtools"}\NormalTok{); }\FunctionTok{require}\NormalTok{(egvtools)\}}


\CommentTok{\# Templates {-}{-}{-}{-}{-}}
\NormalTok{template100}\OtherTok{=}\FunctionTok{rast}\NormalTok{(}\StringTok{"./Templates/TemplateRasters/LV100m\_10km.tif"}\NormalTok{)}

\CommentTok{\# radii}
\FunctionTok{radius\_function}\NormalTok{(}
 \AttributeTok{kvadrati\_path =} \StringTok{"./Templates/TemplateGrids/tiles/"}\NormalTok{,}
 \AttributeTok{radii\_path   =} \StringTok{"./Templates/TemplateGridPoints/tiles/"}\NormalTok{,}
 \AttributeTok{tikls100\_path =} \StringTok{"./Templates/TemplateGrids/tikls100\_sauzeme.parquet"}\NormalTok{,}
 \AttributeTok{template\_path =} \StringTok{"./Templates/TemplateRasters/LV100m\_10km.tif"}\NormalTok{,}
 \AttributeTok{input\_layers  =} \FunctionTok{c}\NormalTok{(}\StringTok{"./RasterGrids\_100m/2024/RAW/Edges\_Bogs{-}Trees\_cell.tif"}\NormalTok{),}
 \AttributeTok{layer\_prefixes =} \FunctionTok{c}\NormalTok{(}\StringTok{"Edges\_Bogs{-}Trees"}\NormalTok{),}
 \AttributeTok{output\_dir   =} \StringTok{"./RasterGrids\_100m/2024/RAW/"}\NormalTok{,}
 \AttributeTok{n\_workers   =} \DecValTok{4}\NormalTok{,}
 \AttributeTok{radii     =} \FunctionTok{c}\NormalTok{(}\StringTok{"r3000"}\NormalTok{),}
 \AttributeTok{radius\_mode  =} \StringTok{"sparse"}\NormalTok{,}
 \AttributeTok{extract\_fun  =} \StringTok{"sum"}\NormalTok{,}
 \AttributeTok{fill\_missing  =} \ConstantTok{TRUE}\NormalTok{,}
 \AttributeTok{IDW\_weight   =} \DecValTok{2}\NormalTok{,}
 \AttributeTok{future\_max\_size =} \DecValTok{20} \SpecialCharTok{*} \DecValTok{1024}\SpecialCharTok{\^{}}\DecValTok{3}\NormalTok{)}

\CommentTok{\# Edges\_Bogs{-}Trees\_r3000.tif    egv\_113}
\NormalTok{slanis}\OtherTok{=}\FunctionTok{rast}\NormalTok{(}\StringTok{"./RasterGrids\_100m/2024/RAW/Edges\_Bogs{-}Trees\_r3000.tif"}\NormalTok{)}
\FunctionTok{names}\NormalTok{(slanis)}\OtherTok{=}\StringTok{"egv\_113"}
\NormalTok{slanis2}\OtherTok{=}\FunctionTok{project}\NormalTok{(slanis,template100)}
\FunctionTok{writeRaster}\NormalTok{(slanis2,}
      \StringTok{"./RasterGrids\_100m/2024/RAW/Edges\_Bogs{-}Trees\_r3000.tif"}\NormalTok{,}
      \AttributeTok{overwrite=}\ConstantTok{TRUE}\NormalTok{)}

\CommentTok{\# standardisation {-}{-}{-}{-}}
\ControlFlowTok{if}\NormalTok{(}\SpecialCharTok{!}\FunctionTok{require}\NormalTok{(terra)) \{}\FunctionTok{install.packages}\NormalTok{(}\StringTok{"terra"}\NormalTok{); }\FunctionTok{require}\NormalTok{(terra)\}}
\ControlFlowTok{if}\NormalTok{(}\SpecialCharTok{!}\FunctionTok{require}\NormalTok{(tidyverse)) \{}\FunctionTok{install.packages}\NormalTok{(}\StringTok{"tidyverse"}\NormalTok{); }\FunctionTok{require}\NormalTok{(tidyverse)\}}

\NormalTok{nosaukums}\OtherTok{=}\StringTok{"Edges\_Bogs{-}Trees\_r3000.tif"}
\NormalTok{ielasisanas\_cels}\OtherTok{=}\FunctionTok{paste0}\NormalTok{(}\StringTok{"./RasterGrids\_100m/2024/RAW/"}\NormalTok{,nosaukums)}
\NormalTok{saglabasanas\_cels}\OtherTok{=}\FunctionTok{paste0}\NormalTok{(}\StringTok{"./RasterGrids\_100m/2024/Scaled/"}\NormalTok{,nosaukums)}
\NormalTok{slanis}\OtherTok{=}\FunctionTok{rast}\NormalTok{(ielasisanas\_cels)}
\NormalTok{videjais}\OtherTok{=}\FunctionTok{global}\NormalTok{(slanis,}\AttributeTok{fun=}\StringTok{"mean"}\NormalTok{,}\AttributeTok{na.rm=}\ConstantTok{TRUE}\NormalTok{)}
\NormalTok{centrets}\OtherTok{=}\NormalTok{slanis}\SpecialCharTok{{-}}\NormalTok{videjais[,}\DecValTok{1}\NormalTok{]}
\NormalTok{standartnovirze}\OtherTok{=}\NormalTok{terra}\SpecialCharTok{::}\FunctionTok{global}\NormalTok{(centrets,}\AttributeTok{fun=}\StringTok{"rms"}\NormalTok{,}\AttributeTok{na.rm=}\ConstantTok{TRUE}\NormalTok{)}
\NormalTok{merogots}\OtherTok{=}\NormalTok{centrets}\SpecialCharTok{/}\NormalTok{standartnovirze[,}\DecValTok{1}\NormalTok{]}
\FunctionTok{writeRaster}\NormalTok{(merogots,}
      \AttributeTok{filename=}\NormalTok{saglabasanas\_cels,}
      \AttributeTok{overwrite=}\ConstantTok{TRUE}\NormalTok{)}
\end{Highlighting}
\end{Shaded}

\section{Edges\_Bogs-Trees\_r10000}\label{ch06.114}

\textbf{filename:} \texttt{Edges\_Bogs-Trees\_r10000.tif}

\textbf{layername:} \texttt{egv\_114}

\textbf{English name:} Edge pixels of Bogs, Mires bordering with Trees within the 10
km landscape

\textbf{Latvian name:} Purvu malu ar kokiem pikseļu skaits 10 km ainavā

\textbf{Procedure:} The total edge within a 10000 m radius around the analysis grid cell is
calculated as the area-weighted sum of the \hyperref[ch06.110]{analysis cells} inside the
buffer, using the workflow \texttt{egvtools::radius\_function()}. During the calculation of the landscape metric,
inverse distance weighted (power = 2) gap filling on the output is applied
to ensure no missing values at the edges. Then the layer is rewritten to set
its name. Finally, the layer is standardised by subtracting the arithmetic
mean and dividing by the root mean squared error.

\begin{Shaded}
\begin{Highlighting}[]
\CommentTok{\# libs {-}{-}{-}{-}}
\ControlFlowTok{if}\NormalTok{(}\SpecialCharTok{!}\FunctionTok{require}\NormalTok{(terra)) \{}\FunctionTok{install.packages}\NormalTok{(}\StringTok{"terra"}\NormalTok{); }\FunctionTok{require}\NormalTok{(terra)\}}
\ControlFlowTok{if}\NormalTok{(}\SpecialCharTok{!}\FunctionTok{require}\NormalTok{(egvtools)) \{remotes}\SpecialCharTok{::}\FunctionTok{install\_github}\NormalTok{(}\StringTok{"aavotins/egvtools"}\NormalTok{); }\FunctionTok{require}\NormalTok{(egvtools)\}}


\CommentTok{\# Templates {-}{-}{-}{-}{-}}
\NormalTok{template100}\OtherTok{=}\FunctionTok{rast}\NormalTok{(}\StringTok{"./Templates/TemplateRasters/LV100m\_10km.tif"}\NormalTok{)}

\CommentTok{\# radii}
\FunctionTok{radius\_function}\NormalTok{(}
 \AttributeTok{kvadrati\_path =} \StringTok{"./Templates/TemplateGrids/tiles/"}\NormalTok{,}
 \AttributeTok{radii\_path   =} \StringTok{"./Templates/TemplateGridPoints/tiles/"}\NormalTok{,}
 \AttributeTok{tikls100\_path =} \StringTok{"./Templates/TemplateGrids/tikls100\_sauzeme.parquet"}\NormalTok{,}
 \AttributeTok{template\_path =} \StringTok{"./Templates/TemplateRasters/LV100m\_10km.tif"}\NormalTok{,}
 \AttributeTok{input\_layers  =} \FunctionTok{c}\NormalTok{(}\StringTok{"./RasterGrids\_100m/2024/RAW/Edges\_Bogs{-}Trees\_cell.tif"}\NormalTok{),}
 \AttributeTok{layer\_prefixes =} \FunctionTok{c}\NormalTok{(}\StringTok{"Edges\_Bogs{-}Trees"}\NormalTok{),}
 \AttributeTok{output\_dir   =} \StringTok{"./RasterGrids\_100m/2024/RAW/"}\NormalTok{,}
 \AttributeTok{n\_workers   =} \DecValTok{4}\NormalTok{,}
 \AttributeTok{radii     =} \FunctionTok{c}\NormalTok{(}\StringTok{"r10000"}\NormalTok{),}
 \AttributeTok{radius\_mode  =} \StringTok{"sparse"}\NormalTok{,}
 \AttributeTok{extract\_fun  =} \StringTok{"sum"}\NormalTok{,}
 \AttributeTok{fill\_missing  =} \ConstantTok{TRUE}\NormalTok{,}
 \AttributeTok{IDW\_weight   =} \DecValTok{2}\NormalTok{,}
 \AttributeTok{future\_max\_size =} \DecValTok{20} \SpecialCharTok{*} \DecValTok{1024}\SpecialCharTok{\^{}}\DecValTok{3}\NormalTok{)}

\CommentTok{\# Edges\_Bogs{-}Trees\_r10000.tif   egv\_114}
\NormalTok{slanis}\OtherTok{=}\FunctionTok{rast}\NormalTok{(}\StringTok{"./RasterGrids\_100m/2024/RAW/Edges\_Bogs{-}Trees\_r10000.tif"}\NormalTok{)}
\FunctionTok{names}\NormalTok{(slanis)}\OtherTok{=}\StringTok{"egv\_114"}
\NormalTok{slanis2}\OtherTok{=}\FunctionTok{project}\NormalTok{(slanis,template100)}
\FunctionTok{writeRaster}\NormalTok{(slanis2,}
      \StringTok{"./RasterGrids\_100m/2024/RAW/Edges\_Bogs{-}Trees\_r10000.tif"}\NormalTok{,}
      \AttributeTok{overwrite=}\ConstantTok{TRUE}\NormalTok{)}

\CommentTok{\# standardisation {-}{-}{-}{-}}
\ControlFlowTok{if}\NormalTok{(}\SpecialCharTok{!}\FunctionTok{require}\NormalTok{(terra)) \{}\FunctionTok{install.packages}\NormalTok{(}\StringTok{"terra"}\NormalTok{); }\FunctionTok{require}\NormalTok{(terra)\}}
\ControlFlowTok{if}\NormalTok{(}\SpecialCharTok{!}\FunctionTok{require}\NormalTok{(tidyverse)) \{}\FunctionTok{install.packages}\NormalTok{(}\StringTok{"tidyverse"}\NormalTok{); }\FunctionTok{require}\NormalTok{(tidyverse)\}}

\NormalTok{nosaukums}\OtherTok{=}\StringTok{"Edges\_Bogs{-}Trees\_r10000.tif"}
\NormalTok{ielasisanas\_cels}\OtherTok{=}\FunctionTok{paste0}\NormalTok{(}\StringTok{"./RasterGrids\_100m/2024/RAW/"}\NormalTok{,nosaukums)}
\NormalTok{saglabasanas\_cels}\OtherTok{=}\FunctionTok{paste0}\NormalTok{(}\StringTok{"./RasterGrids\_100m/2024/Scaled/"}\NormalTok{,nosaukums)}
\NormalTok{slanis}\OtherTok{=}\FunctionTok{rast}\NormalTok{(ielasisanas\_cels)}
\NormalTok{videjais}\OtherTok{=}\FunctionTok{global}\NormalTok{(slanis,}\AttributeTok{fun=}\StringTok{"mean"}\NormalTok{,}\AttributeTok{na.rm=}\ConstantTok{TRUE}\NormalTok{)}
\NormalTok{centrets}\OtherTok{=}\NormalTok{slanis}\SpecialCharTok{{-}}\NormalTok{videjais[,}\DecValTok{1}\NormalTok{]}
\NormalTok{standartnovirze}\OtherTok{=}\NormalTok{terra}\SpecialCharTok{::}\FunctionTok{global}\NormalTok{(centrets,}\AttributeTok{fun=}\StringTok{"rms"}\NormalTok{,}\AttributeTok{na.rm=}\ConstantTok{TRUE}\NormalTok{)}
\NormalTok{merogots}\OtherTok{=}\NormalTok{centrets}\SpecialCharTok{/}\NormalTok{standartnovirze[,}\DecValTok{1}\NormalTok{]}
\FunctionTok{writeRaster}\NormalTok{(merogots,}
      \AttributeTok{filename=}\NormalTok{saglabasanas\_cels,}
      \AttributeTok{overwrite=}\ConstantTok{TRUE}\NormalTok{)}
\end{Highlighting}
\end{Shaded}

\section{Edges\_Bogs-Water\_cell}\label{ch06.115}

\textbf{filename:} \texttt{Edges\_Bogs-Water\_cell.tif}

\textbf{layername:} \texttt{egv\_115}

\textbf{English name:} Edge pixels of Bogs, Mires bordering with Water within the
analysis cell (1 ha)

\textbf{Latvian name:} Purvu malu ar ūdeni pikseļu skaits analīzes šūnā (1 ha)

\textbf{Procedure:} First, values 200 from the \hyperref[Ch05.03]{Landscape classification} are
coded as 0, and all other values as NA. Then bog and transitional mire layers from
\hyperref[Ch04.17]{EDI} are reclassified to presence-only (value 1) and combined. Then,
bog-and-mire layer (1 = presence) is covered over water layer (presence = 0) and
written to file (matching the input). Then, using the workflow
\texttt{egvtools::landscape\_function()} total edge between the two classes is
calculated. During the calculation of the landscape metric, inverse distance weighted
(power = 2) gap filling on the output is applied to ensure no missing values
at the edges. Finally, the layer is standardised by subtracting the arithmetic
mean and dividing by the root mean squared error.

\begin{Shaded}
\begin{Highlighting}[]
\CommentTok{\# libs {-}{-}{-}{-}}
\ControlFlowTok{if}\NormalTok{(}\SpecialCharTok{!}\FunctionTok{require}\NormalTok{(terra)) \{}\FunctionTok{install.packages}\NormalTok{(}\StringTok{"terra"}\NormalTok{); }\FunctionTok{require}\NormalTok{(terra)\}}
\ControlFlowTok{if}\NormalTok{(}\SpecialCharTok{!}\FunctionTok{require}\NormalTok{(egvtools)) \{remotes}\SpecialCharTok{::}\FunctionTok{install\_github}\NormalTok{(}\StringTok{"aavotins/egvtools"}\NormalTok{); }\FunctionTok{require}\NormalTok{(egvtools)\}}

\ControlFlowTok{if}\NormalTok{(}\SpecialCharTok{!}\FunctionTok{require}\NormalTok{(sf)) \{}\FunctionTok{install.packages}\NormalTok{(}\StringTok{"sf"}\NormalTok{); }\FunctionTok{require}\NormalTok{(sf)\}}
\ControlFlowTok{if}\NormalTok{(}\SpecialCharTok{!}\FunctionTok{require}\NormalTok{(sfarrow)) \{}\FunctionTok{install.packages}\NormalTok{(}\StringTok{"sfarrow"}\NormalTok{); }\FunctionTok{require}\NormalTok{(sfarrow)\}}
\ControlFlowTok{if}\NormalTok{(}\SpecialCharTok{!}\FunctionTok{require}\NormalTok{(raster)) \{}\FunctionTok{install.packages}\NormalTok{(}\StringTok{"raster"}\NormalTok{); }\FunctionTok{require}\NormalTok{(raster)\}}
\ControlFlowTok{if}\NormalTok{(}\SpecialCharTok{!}\FunctionTok{require}\NormalTok{(fasterize)) \{}\FunctionTok{install.packages}\NormalTok{(}\StringTok{"fasterize"}\NormalTok{); }\FunctionTok{require}\NormalTok{(fasterize)\}}
\ControlFlowTok{if}\NormalTok{(}\SpecialCharTok{!}\FunctionTok{require}\NormalTok{(tidyverse)) \{}\FunctionTok{install.packages}\NormalTok{(}\StringTok{"tidyverse"}\NormalTok{); }\FunctionTok{require}\NormalTok{(tidyverse)\}}


\CommentTok{\# Templates {-}{-}{-}{-}{-}}
\NormalTok{template10}\OtherTok{=}\FunctionTok{rast}\NormalTok{(}\StringTok{"./Templates/TemplateRasters/LV10m\_10km.tif"}\NormalTok{)}
\NormalTok{nulls10}\OtherTok{=}\FunctionTok{rast}\NormalTok{(}\StringTok{"./Templates/TemplateRasters/nulls\_LV10m\_10km.tif"}\NormalTok{)}

\CommentTok{\# simple landscape {-}{-}{-}{-}}
\NormalTok{simple\_landscape}\OtherTok{=}\FunctionTok{rast}\NormalTok{(}\StringTok{"./RasterGrids\_10m/2024/Ainava\_vienk\_mask.tif"}\NormalTok{)}

\CommentTok{\# Edges\_Bogs{-}Water\_input.tif {-}{-}{-}{-}}
\NormalTok{bogs}\OtherTok{=}\FunctionTok{rast}\NormalTok{(}\StringTok{"./RasterGrids\_10m/2024/EDI\_BogsYN.tif"}\NormalTok{)}
\NormalTok{bogs}\OtherTok{=}\FunctionTok{subst}\NormalTok{(bogs,}\DecValTok{0}\NormalTok{,}\ConstantTok{NA}\NormalTok{)}
\FunctionTok{plot}\NormalTok{(bogs)}
\NormalTok{mires}\OtherTok{=}\FunctionTok{rast}\NormalTok{(}\StringTok{"./RasterGrids\_10m/2024/EDI\_TransitionalMiresYN.tif"}\NormalTok{)}
\NormalTok{mires}\OtherTok{=}\FunctionTok{subst}\NormalTok{(mires,}\DecValTok{0}\NormalTok{,}\ConstantTok{NA}\NormalTok{)}
\FunctionTok{plot}\NormalTok{(mires)}
\NormalTok{bogs\_mires}\OtherTok{=}\FunctionTok{cover}\NormalTok{(bogs,mires)}
\FunctionTok{plot}\NormalTok{(bogs\_mires)}

\NormalTok{water}\OtherTok{=}\FunctionTok{ifel}\NormalTok{(simple\_landscape}\SpecialCharTok{==}\DecValTok{200}\NormalTok{,}\DecValTok{0}\NormalTok{,}\ConstantTok{NA}\NormalTok{)}
\FunctionTok{plot}\NormalTok{(water)}

\NormalTok{bm\_water}\OtherTok{=}\FunctionTok{cover}\NormalTok{(bogs\_mires,water)}
\FunctionTok{plot}\NormalTok{(bm\_water)}

\NormalTok{edge\_bm\_water}\OtherTok{=}\FunctionTok{project}\NormalTok{(bm\_water,template10,}
           \AttributeTok{filename=}\StringTok{"./RasterGrids\_10m/2024/Edges\_Bogs{-}Water\_input.tif"}\NormalTok{,}
           \AttributeTok{overwrite=}\ConstantTok{TRUE}\NormalTok{)}
\FunctionTok{rm}\NormalTok{(edge\_bm\_water)}
\FunctionTok{rm}\NormalTok{(bm\_water)}


\CommentTok{\# Edges\_Bogs{-}Water\_cell.tif egv\_115 {-}{-}{-}{-}}
\FunctionTok{landscape\_function}\NormalTok{(}
 \AttributeTok{landscape   =} \StringTok{"./RasterGrids\_10m/2024/Edges\_Bogs{-}Water\_input.tif"}\NormalTok{,}
 \AttributeTok{zones     =} \StringTok{"./Templates/TemplateGrids/tikls100\_sauzeme.parquet"}\NormalTok{,}
 \AttributeTok{id\_field    =} \StringTok{"id"}\NormalTok{,}
 \AttributeTok{tile\_field   =} \StringTok{"tks50km"}\NormalTok{,}
 \AttributeTok{template    =} \StringTok{"./Templates/TemplateRasters/LV100m\_10km.tif"}\NormalTok{,}
 \AttributeTok{out\_dir    =} \StringTok{"./RasterGrids\_100m/2024/RAW"}\NormalTok{,}
 \AttributeTok{out\_filename  =} \StringTok{"Edges\_Bogs{-}Water\_cell.tif"}\NormalTok{,}
 \AttributeTok{out\_layername =} \StringTok{"egv\_115"}\NormalTok{,}
 \AttributeTok{what       =} \StringTok{"lsm\_l\_te"}\NormalTok{,}
 \AttributeTok{lm\_args     =} \FunctionTok{list}\NormalTok{(}\AttributeTok{count\_boundary =} \ConstantTok{FALSE}\NormalTok{),}
 \AttributeTok{rasterize\_engine =} \StringTok{"fasterize"}\NormalTok{,}
 \AttributeTok{n\_workers   =} \DecValTok{12}\NormalTok{,}
 \AttributeTok{future\_max\_size =} \DecValTok{20} \SpecialCharTok{*} \DecValTok{1024}\SpecialCharTok{\^{}}\DecValTok{3}\NormalTok{,}
 \AttributeTok{fill\_gaps   =} \ConstantTok{TRUE}\NormalTok{,}
 \AttributeTok{plot\_gaps   =} \ConstantTok{FALSE}\NormalTok{,}
 \AttributeTok{plot\_result  =} \ConstantTok{FALSE}
\NormalTok{)}

\CommentTok{\# standardisation {-}{-}{-}{-}}
\ControlFlowTok{if}\NormalTok{(}\SpecialCharTok{!}\FunctionTok{require}\NormalTok{(terra)) \{}\FunctionTok{install.packages}\NormalTok{(}\StringTok{"terra"}\NormalTok{); }\FunctionTok{require}\NormalTok{(terra)\}}
\ControlFlowTok{if}\NormalTok{(}\SpecialCharTok{!}\FunctionTok{require}\NormalTok{(tidyverse)) \{}\FunctionTok{install.packages}\NormalTok{(}\StringTok{"tidyverse"}\NormalTok{); }\FunctionTok{require}\NormalTok{(tidyverse)\}}

\NormalTok{nosaukums}\OtherTok{=}\StringTok{"Edges\_Bogs{-}Water\_cell.tif"}
\NormalTok{ielasisanas\_cels}\OtherTok{=}\FunctionTok{paste0}\NormalTok{(}\StringTok{"./RasterGrids\_100m/2024/RAW/"}\NormalTok{,nosaukums)}
\NormalTok{saglabasanas\_cels}\OtherTok{=}\FunctionTok{paste0}\NormalTok{(}\StringTok{"./RasterGrids\_100m/2024/Scaled/"}\NormalTok{,nosaukums)}
\NormalTok{slanis}\OtherTok{=}\FunctionTok{rast}\NormalTok{(ielasisanas\_cels)}
\NormalTok{videjais}\OtherTok{=}\FunctionTok{global}\NormalTok{(slanis,}\AttributeTok{fun=}\StringTok{"mean"}\NormalTok{,}\AttributeTok{na.rm=}\ConstantTok{TRUE}\NormalTok{)}
\NormalTok{centrets}\OtherTok{=}\NormalTok{slanis}\SpecialCharTok{{-}}\NormalTok{videjais[,}\DecValTok{1}\NormalTok{]}
\NormalTok{standartnovirze}\OtherTok{=}\NormalTok{terra}\SpecialCharTok{::}\FunctionTok{global}\NormalTok{(centrets,}\AttributeTok{fun=}\StringTok{"rms"}\NormalTok{,}\AttributeTok{na.rm=}\ConstantTok{TRUE}\NormalTok{)}
\NormalTok{merogots}\OtherTok{=}\NormalTok{centrets}\SpecialCharTok{/}\NormalTok{standartnovirze[,}\DecValTok{1}\NormalTok{]}
\FunctionTok{writeRaster}\NormalTok{(merogots,}
      \AttributeTok{filename=}\NormalTok{saglabasanas\_cels,}
      \AttributeTok{overwrite=}\ConstantTok{TRUE}\NormalTok{)}
\end{Highlighting}
\end{Shaded}

\section{Edges\_Bogs-Water\_r500}\label{ch06.116}

\textbf{filename:} \texttt{Edges\_Bogs-Water\_r500.tif}

\textbf{layername:} \texttt{egv\_116}

\textbf{English name:} Edge pixels of Bogs, Mires bordering with Water within the 0.5
km landscape

\textbf{Latvian name:} Purvu malu ar ūdeni pikseļu skaits 0,5 km ainavā

\textbf{Procedure:} The total edge within a 500 m radius around the analysis grid cell is
calculated as the area-weighted sum of the \hyperref[ch06.115]{analysis cells} inside the
buffer, using the workflow \texttt{egvtools::radius\_function()}. During the calculation of the landscape metric,
inverse distance weighted (power = 2) gap filling on the output is applied
to ensure no missing values at the edges. Then the layer is rewritten to set
its name. Finally, the layer is standardised by subtracting the arithmetic
mean and dividing by the root mean squared error.

\begin{Shaded}
\begin{Highlighting}[]
\CommentTok{\# libs {-}{-}{-}{-}}
\ControlFlowTok{if}\NormalTok{(}\SpecialCharTok{!}\FunctionTok{require}\NormalTok{(terra)) \{}\FunctionTok{install.packages}\NormalTok{(}\StringTok{"terra"}\NormalTok{); }\FunctionTok{require}\NormalTok{(terra)\}}
\ControlFlowTok{if}\NormalTok{(}\SpecialCharTok{!}\FunctionTok{require}\NormalTok{(egvtools)) \{remotes}\SpecialCharTok{::}\FunctionTok{install\_github}\NormalTok{(}\StringTok{"aavotins/egvtools"}\NormalTok{); }\FunctionTok{require}\NormalTok{(egvtools)\}}


\CommentTok{\# Templates {-}{-}{-}{-}{-}}
\NormalTok{template100}\OtherTok{=}\FunctionTok{rast}\NormalTok{(}\StringTok{"./Templates/TemplateRasters/LV100m\_10km.tif"}\NormalTok{)}

\CommentTok{\# radii {-}{-}{-}{-}}
\FunctionTok{radius\_function}\NormalTok{(}
 \AttributeTok{kvadrati\_path =} \StringTok{"./Templates/TemplateGrids/tiles/"}\NormalTok{,}
 \AttributeTok{radii\_path   =} \StringTok{"./Templates/TemplateGridPoints/tiles/"}\NormalTok{,}
 \AttributeTok{tikls100\_path =} \StringTok{"./Templates/TemplateGrids/tikls100\_sauzeme.parquet"}\NormalTok{,}
 \AttributeTok{template\_path =} \StringTok{"./Templates/TemplateRasters/LV100m\_10km.tif"}\NormalTok{,}
 \AttributeTok{input\_layers  =} \FunctionTok{c}\NormalTok{(}\StringTok{"./RasterGrids\_100m/2024/RAW/Edges\_Bogs{-}Water\_cell.tif"}\NormalTok{),}
 \AttributeTok{layer\_prefixes =} \FunctionTok{c}\NormalTok{(}\StringTok{"Edges\_Bogs{-}Water"}\NormalTok{),}
 \AttributeTok{output\_dir   =} \StringTok{"./RasterGrids\_100m/2024/RAW/"}\NormalTok{,}
 \AttributeTok{n\_workers   =} \DecValTok{12}\NormalTok{,}
 \AttributeTok{radii     =} \FunctionTok{c}\NormalTok{(}\StringTok{"r500"}\NormalTok{),}
 \AttributeTok{radius\_mode  =} \StringTok{"sparse"}\NormalTok{,}
 \AttributeTok{extract\_fun  =} \StringTok{"sum"}\NormalTok{,}
 \AttributeTok{fill\_missing  =} \ConstantTok{TRUE}\NormalTok{,}
 \AttributeTok{IDW\_weight   =} \DecValTok{2}\NormalTok{,}
 \AttributeTok{future\_max\_size =} \DecValTok{20} \SpecialCharTok{*} \DecValTok{1024}\SpecialCharTok{\^{}}\DecValTok{3}\NormalTok{)}

\CommentTok{\# Edges\_Bogs{-}Water\_r500.tif egv\_116}
\NormalTok{slanis}\OtherTok{=}\FunctionTok{rast}\NormalTok{(}\StringTok{"./RasterGrids\_100m/2024/RAW/Edges\_Bogs{-}Water\_r500.tif"}\NormalTok{)}
\FunctionTok{names}\NormalTok{(slanis)}\OtherTok{=}\StringTok{"egv\_116"}
\NormalTok{slanis2}\OtherTok{=}\FunctionTok{project}\NormalTok{(slanis,template100)}
\FunctionTok{writeRaster}\NormalTok{(slanis2,}
      \StringTok{"./RasterGrids\_100m/2024/RAW/Edges\_Bogs{-}Water\_r500.tif"}\NormalTok{,}
      \AttributeTok{overwrite=}\ConstantTok{TRUE}\NormalTok{)}

\CommentTok{\# standardisation {-}{-}{-}{-}}
\ControlFlowTok{if}\NormalTok{(}\SpecialCharTok{!}\FunctionTok{require}\NormalTok{(terra)) \{}\FunctionTok{install.packages}\NormalTok{(}\StringTok{"terra"}\NormalTok{); }\FunctionTok{require}\NormalTok{(terra)\}}
\ControlFlowTok{if}\NormalTok{(}\SpecialCharTok{!}\FunctionTok{require}\NormalTok{(tidyverse)) \{}\FunctionTok{install.packages}\NormalTok{(}\StringTok{"tidyverse"}\NormalTok{); }\FunctionTok{require}\NormalTok{(tidyverse)\}}

\NormalTok{nosaukums}\OtherTok{=}\StringTok{"Edges\_Bogs{-}Water\_r500.tif"}
\NormalTok{ielasisanas\_cels}\OtherTok{=}\FunctionTok{paste0}\NormalTok{(}\StringTok{"./RasterGrids\_100m/2024/RAW/"}\NormalTok{,nosaukums)}
\NormalTok{saglabasanas\_cels}\OtherTok{=}\FunctionTok{paste0}\NormalTok{(}\StringTok{"./RasterGrids\_100m/2024/Scaled/"}\NormalTok{,nosaukums)}
\NormalTok{slanis}\OtherTok{=}\FunctionTok{rast}\NormalTok{(ielasisanas\_cels)}
\NormalTok{videjais}\OtherTok{=}\FunctionTok{global}\NormalTok{(slanis,}\AttributeTok{fun=}\StringTok{"mean"}\NormalTok{,}\AttributeTok{na.rm=}\ConstantTok{TRUE}\NormalTok{)}
\NormalTok{centrets}\OtherTok{=}\NormalTok{slanis}\SpecialCharTok{{-}}\NormalTok{videjais[,}\DecValTok{1}\NormalTok{]}
\NormalTok{standartnovirze}\OtherTok{=}\NormalTok{terra}\SpecialCharTok{::}\FunctionTok{global}\NormalTok{(centrets,}\AttributeTok{fun=}\StringTok{"rms"}\NormalTok{,}\AttributeTok{na.rm=}\ConstantTok{TRUE}\NormalTok{)}
\NormalTok{merogots}\OtherTok{=}\NormalTok{centrets}\SpecialCharTok{/}\NormalTok{standartnovirze[,}\DecValTok{1}\NormalTok{]}
\FunctionTok{writeRaster}\NormalTok{(merogots,}
      \AttributeTok{filename=}\NormalTok{saglabasanas\_cels,}
      \AttributeTok{overwrite=}\ConstantTok{TRUE}\NormalTok{)}
\end{Highlighting}
\end{Shaded}

\section{Edges\_Bogs-Water\_r1250}\label{ch06.117}

\textbf{filename:} \texttt{Edges\_Bogs-Water\_r1250.tif}

\textbf{layername:} \texttt{egv\_117}

\textbf{English name:} Edge pixels of Bogs, Mires bordering with Water within the
1.25 km landscape

\textbf{Latvian name:} Purvu malu ar ūdeni pikseļu skaits 1,25 km ainavā

\textbf{Procedure:} The total edge within a 1250 m radius around the analysis grid cell is
calculated as the area-weighted sum of the \hyperref[ch06.115]{analysis cells} inside the
buffer, using the workflow \texttt{egvtools::radius\_function()}. During the calculation of the landscape metric,
inverse distance weighted (power = 2) gap filling on the output is applied
to ensure no missing values at the edges. Then the layer is rewritten to set
its name. Finally, the layer is standardised by subtracting the arithmetic
mean and dividing by the root mean squared error.

\begin{Shaded}
\begin{Highlighting}[]
\CommentTok{\# libs {-}{-}{-}{-}}
\ControlFlowTok{if}\NormalTok{(}\SpecialCharTok{!}\FunctionTok{require}\NormalTok{(terra)) \{}\FunctionTok{install.packages}\NormalTok{(}\StringTok{"terra"}\NormalTok{); }\FunctionTok{require}\NormalTok{(terra)\}}
\ControlFlowTok{if}\NormalTok{(}\SpecialCharTok{!}\FunctionTok{require}\NormalTok{(egvtools)) \{remotes}\SpecialCharTok{::}\FunctionTok{install\_github}\NormalTok{(}\StringTok{"aavotins/egvtools"}\NormalTok{); }\FunctionTok{require}\NormalTok{(egvtools)\}}


\CommentTok{\# Templates {-}{-}{-}{-}{-}}
\NormalTok{template100}\OtherTok{=}\FunctionTok{rast}\NormalTok{(}\StringTok{"./Templates/TemplateRasters/LV100m\_10km.tif"}\NormalTok{)}

\CommentTok{\# radii {-}{-}{-}{-}}
\FunctionTok{radius\_function}\NormalTok{(}
 \AttributeTok{kvadrati\_path =} \StringTok{"./Templates/TemplateGrids/tiles/"}\NormalTok{,}
 \AttributeTok{radii\_path   =} \StringTok{"./Templates/TemplateGridPoints/tiles/"}\NormalTok{,}
 \AttributeTok{tikls100\_path =} \StringTok{"./Templates/TemplateGrids/tikls100\_sauzeme.parquet"}\NormalTok{,}
 \AttributeTok{template\_path =} \StringTok{"./Templates/TemplateRasters/LV100m\_10km.tif"}\NormalTok{,}
 \AttributeTok{input\_layers  =} \FunctionTok{c}\NormalTok{(}\StringTok{"./RasterGrids\_100m/2024/RAW/Edges\_Bogs{-}Water\_cell.tif"}\NormalTok{),}
 \AttributeTok{layer\_prefixes =} \FunctionTok{c}\NormalTok{(}\StringTok{"Edges\_Bogs{-}Water"}\NormalTok{),}
 \AttributeTok{output\_dir   =} \StringTok{"./RasterGrids\_100m/2024/RAW/"}\NormalTok{,}
 \AttributeTok{n\_workers   =} \DecValTok{12}\NormalTok{,}
 \AttributeTok{radii     =} \FunctionTok{c}\NormalTok{(}\StringTok{"r1250"}\NormalTok{),}
 \AttributeTok{radius\_mode  =} \StringTok{"sparse"}\NormalTok{,}
 \AttributeTok{extract\_fun  =} \StringTok{"sum"}\NormalTok{,}
 \AttributeTok{fill\_missing  =} \ConstantTok{TRUE}\NormalTok{,}
 \AttributeTok{IDW\_weight   =} \DecValTok{2}\NormalTok{,}
 \AttributeTok{future\_max\_size =} \DecValTok{20} \SpecialCharTok{*} \DecValTok{1024}\SpecialCharTok{\^{}}\DecValTok{3}\NormalTok{)}

\CommentTok{\# Edges\_Bogs{-}Water\_r1250.tif    egv\_117}
\NormalTok{slanis}\OtherTok{=}\FunctionTok{rast}\NormalTok{(}\StringTok{"./RasterGrids\_100m/2024/RAW/Edges\_Bogs{-}Water\_r1250.tif"}\NormalTok{)}
\FunctionTok{names}\NormalTok{(slanis)}\OtherTok{=}\StringTok{"egv\_117"}
\NormalTok{slanis2}\OtherTok{=}\FunctionTok{project}\NormalTok{(slanis,template100)}
\FunctionTok{writeRaster}\NormalTok{(slanis2,}
      \StringTok{"./RasterGrids\_100m/2024/RAW/Edges\_Bogs{-}Water\_r1250.tif"}\NormalTok{,}
      \AttributeTok{overwrite=}\ConstantTok{TRUE}\NormalTok{)}

\CommentTok{\# standardisation {-}{-}{-}{-}}
\ControlFlowTok{if}\NormalTok{(}\SpecialCharTok{!}\FunctionTok{require}\NormalTok{(terra)) \{}\FunctionTok{install.packages}\NormalTok{(}\StringTok{"terra"}\NormalTok{); }\FunctionTok{require}\NormalTok{(terra)\}}
\ControlFlowTok{if}\NormalTok{(}\SpecialCharTok{!}\FunctionTok{require}\NormalTok{(tidyverse)) \{}\FunctionTok{install.packages}\NormalTok{(}\StringTok{"tidyverse"}\NormalTok{); }\FunctionTok{require}\NormalTok{(tidyverse)\}}

\NormalTok{nosaukums}\OtherTok{=}\StringTok{"Edges\_Bogs{-}Water\_r1250.tif"}
\NormalTok{ielasisanas\_cels}\OtherTok{=}\FunctionTok{paste0}\NormalTok{(}\StringTok{"./RasterGrids\_100m/2024/RAW/"}\NormalTok{,nosaukums)}
\NormalTok{saglabasanas\_cels}\OtherTok{=}\FunctionTok{paste0}\NormalTok{(}\StringTok{"./RasterGrids\_100m/2024/Scaled/"}\NormalTok{,nosaukums)}
\NormalTok{slanis}\OtherTok{=}\FunctionTok{rast}\NormalTok{(ielasisanas\_cels)}
\NormalTok{videjais}\OtherTok{=}\FunctionTok{global}\NormalTok{(slanis,}\AttributeTok{fun=}\StringTok{"mean"}\NormalTok{,}\AttributeTok{na.rm=}\ConstantTok{TRUE}\NormalTok{)}
\NormalTok{centrets}\OtherTok{=}\NormalTok{slanis}\SpecialCharTok{{-}}\NormalTok{videjais[,}\DecValTok{1}\NormalTok{]}
\NormalTok{standartnovirze}\OtherTok{=}\NormalTok{terra}\SpecialCharTok{::}\FunctionTok{global}\NormalTok{(centrets,}\AttributeTok{fun=}\StringTok{"rms"}\NormalTok{,}\AttributeTok{na.rm=}\ConstantTok{TRUE}\NormalTok{)}
\NormalTok{merogots}\OtherTok{=}\NormalTok{centrets}\SpecialCharTok{/}\NormalTok{standartnovirze[,}\DecValTok{1}\NormalTok{]}
\FunctionTok{writeRaster}\NormalTok{(merogots,}
      \AttributeTok{filename=}\NormalTok{saglabasanas\_cels,}
      \AttributeTok{overwrite=}\ConstantTok{TRUE}\NormalTok{)}
\end{Highlighting}
\end{Shaded}

\section{Edges\_Bogs-Water\_r3000}\label{ch06.118}

\textbf{filename:} \texttt{Edges\_Bogs-Water\_r3000.tif}

\textbf{layername:} \texttt{egv\_118}

\textbf{English name:} Edge pixels of Bogs, Mires bordering with Water within the 3
km landscape

\textbf{Latvian name:} Purvu malu ar ūdeni pikseļu skaits 3 km ainavā

\textbf{Procedure:} The total edge within a 3000 m radius around the analysis grid cell is
calculated as the area-weighted sum of the \hyperref[ch06.115]{analysis cells} inside the
buffer, using the workflow \texttt{egvtools::radius\_function()}. During the calculation of the landscape metric,
inverse distance weighted (power = 2) gap filling on the output is applied
to ensure no missing values at the edges. Then the layer is rewritten to set
its name. Finally, the layer is standardised by subtracting the arithmetic
mean and dividing by the root mean squared error.

\begin{Shaded}
\begin{Highlighting}[]
\CommentTok{\# libs {-}{-}{-}{-}}
\ControlFlowTok{if}\NormalTok{(}\SpecialCharTok{!}\FunctionTok{require}\NormalTok{(terra)) \{}\FunctionTok{install.packages}\NormalTok{(}\StringTok{"terra"}\NormalTok{); }\FunctionTok{require}\NormalTok{(terra)\}}
\ControlFlowTok{if}\NormalTok{(}\SpecialCharTok{!}\FunctionTok{require}\NormalTok{(egvtools)) \{remotes}\SpecialCharTok{::}\FunctionTok{install\_github}\NormalTok{(}\StringTok{"aavotins/egvtools"}\NormalTok{); }\FunctionTok{require}\NormalTok{(egvtools)\}}


\CommentTok{\# Templates {-}{-}{-}{-}{-}}
\NormalTok{template100}\OtherTok{=}\FunctionTok{rast}\NormalTok{(}\StringTok{"./Templates/TemplateRasters/LV100m\_10km.tif"}\NormalTok{)}

\CommentTok{\# radii {-}{-}{-}{-}}
\FunctionTok{radius\_function}\NormalTok{(}
 \AttributeTok{kvadrati\_path =} \StringTok{"./Templates/TemplateGrids/tiles/"}\NormalTok{,}
 \AttributeTok{radii\_path   =} \StringTok{"./Templates/TemplateGridPoints/tiles/"}\NormalTok{,}
 \AttributeTok{tikls100\_path =} \StringTok{"./Templates/TemplateGrids/tikls100\_sauzeme.parquet"}\NormalTok{,}
 \AttributeTok{template\_path =} \StringTok{"./Templates/TemplateRasters/LV100m\_10km.tif"}\NormalTok{,}
 \AttributeTok{input\_layers  =} \FunctionTok{c}\NormalTok{(}\StringTok{"./RasterGrids\_100m/2024/RAW/Edges\_Bogs{-}Water\_cell.tif"}\NormalTok{),}
 \AttributeTok{layer\_prefixes =} \FunctionTok{c}\NormalTok{(}\StringTok{"Edges\_Bogs{-}Water"}\NormalTok{),}
 \AttributeTok{output\_dir   =} \StringTok{"./RasterGrids\_100m/2024/RAW/"}\NormalTok{,}
 \AttributeTok{n\_workers   =} \DecValTok{12}\NormalTok{,}
 \AttributeTok{radii     =} \FunctionTok{c}\NormalTok{(}\StringTok{"r3000"}\NormalTok{),}
 \AttributeTok{radius\_mode  =} \StringTok{"sparse"}\NormalTok{,}
 \AttributeTok{extract\_fun  =} \StringTok{"sum"}\NormalTok{,}
 \AttributeTok{fill\_missing  =} \ConstantTok{TRUE}\NormalTok{,}
 \AttributeTok{IDW\_weight   =} \DecValTok{2}\NormalTok{,}
 \AttributeTok{future\_max\_size =} \DecValTok{20} \SpecialCharTok{*} \DecValTok{1024}\SpecialCharTok{\^{}}\DecValTok{3}\NormalTok{)}

\CommentTok{\# Edges\_Bogs{-}Water\_r3000.tif    egv\_118}
\NormalTok{slanis}\OtherTok{=}\FunctionTok{rast}\NormalTok{(}\StringTok{"./RasterGrids\_100m/2024/RAW/Edges\_Bogs{-}Water\_r3000.tif"}\NormalTok{)}
\FunctionTok{names}\NormalTok{(slanis)}\OtherTok{=}\StringTok{"egv\_118"}
\NormalTok{slanis2}\OtherTok{=}\FunctionTok{project}\NormalTok{(slanis,template100)}
\FunctionTok{writeRaster}\NormalTok{(slanis2,}
      \StringTok{"./RasterGrids\_100m/2024/RAW/Edges\_Bogs{-}Water\_r3000.tif"}\NormalTok{,}
      \AttributeTok{overwrite=}\ConstantTok{TRUE}\NormalTok{)}

\CommentTok{\# standardisation {-}{-}{-}{-}}
\ControlFlowTok{if}\NormalTok{(}\SpecialCharTok{!}\FunctionTok{require}\NormalTok{(terra)) \{}\FunctionTok{install.packages}\NormalTok{(}\StringTok{"terra"}\NormalTok{); }\FunctionTok{require}\NormalTok{(terra)\}}
\ControlFlowTok{if}\NormalTok{(}\SpecialCharTok{!}\FunctionTok{require}\NormalTok{(tidyverse)) \{}\FunctionTok{install.packages}\NormalTok{(}\StringTok{"tidyverse"}\NormalTok{); }\FunctionTok{require}\NormalTok{(tidyverse)\}}

\NormalTok{nosaukums}\OtherTok{=}\StringTok{"Edges\_Bogs{-}Water\_r3000.tif"}
\NormalTok{ielasisanas\_cels}\OtherTok{=}\FunctionTok{paste0}\NormalTok{(}\StringTok{"./RasterGrids\_100m/2024/RAW/"}\NormalTok{,nosaukums)}
\NormalTok{saglabasanas\_cels}\OtherTok{=}\FunctionTok{paste0}\NormalTok{(}\StringTok{"./RasterGrids\_100m/2024/Scaled/"}\NormalTok{,nosaukums)}
\NormalTok{slanis}\OtherTok{=}\FunctionTok{rast}\NormalTok{(ielasisanas\_cels)}
\NormalTok{videjais}\OtherTok{=}\FunctionTok{global}\NormalTok{(slanis,}\AttributeTok{fun=}\StringTok{"mean"}\NormalTok{,}\AttributeTok{na.rm=}\ConstantTok{TRUE}\NormalTok{)}
\NormalTok{centrets}\OtherTok{=}\NormalTok{slanis}\SpecialCharTok{{-}}\NormalTok{videjais[,}\DecValTok{1}\NormalTok{]}
\NormalTok{standartnovirze}\OtherTok{=}\NormalTok{terra}\SpecialCharTok{::}\FunctionTok{global}\NormalTok{(centrets,}\AttributeTok{fun=}\StringTok{"rms"}\NormalTok{,}\AttributeTok{na.rm=}\ConstantTok{TRUE}\NormalTok{)}
\NormalTok{merogots}\OtherTok{=}\NormalTok{centrets}\SpecialCharTok{/}\NormalTok{standartnovirze[,}\DecValTok{1}\NormalTok{]}
\FunctionTok{writeRaster}\NormalTok{(merogots,}
      \AttributeTok{filename=}\NormalTok{saglabasanas\_cels,}
      \AttributeTok{overwrite=}\ConstantTok{TRUE}\NormalTok{)}
\end{Highlighting}
\end{Shaded}

\section{Edges\_Bogs-Water\_r10000}\label{ch06.119}

\textbf{filename:} \texttt{Edges\_Bogs-Water\_r10000.tif}

\textbf{layername:} \texttt{egv\_119}

\textbf{English name:} Edge pixels of Bogs, Mires bordering with Water within the 10
km landscape

\textbf{Latvian name:} Purvu malu ar ūdeni pikseļu skaits 10 km ainavā

\textbf{Procedure:} The total edge within a 10000 m radius around the analysis grid cell is
calculated as the area-weighted sum of the \hyperref[ch06.115]{analysis cells} inside the
buffer, using the workflow \texttt{egvtools::radius\_function()}. During the calculation of the landscape metric,
inverse distance weighted (power = 2) gap filling on the output is applied
to ensure no missing values at the edges. Then the layer is rewritten to set
its name. Finally, the layer is standardised by subtracting the arithmetic
mean and dividing by the root mean squared error.

\begin{Shaded}
\begin{Highlighting}[]
\CommentTok{\# libs {-}{-}{-}{-}}
\ControlFlowTok{if}\NormalTok{(}\SpecialCharTok{!}\FunctionTok{require}\NormalTok{(terra)) \{}\FunctionTok{install.packages}\NormalTok{(}\StringTok{"terra"}\NormalTok{); }\FunctionTok{require}\NormalTok{(terra)\}}
\ControlFlowTok{if}\NormalTok{(}\SpecialCharTok{!}\FunctionTok{require}\NormalTok{(egvtools)) \{remotes}\SpecialCharTok{::}\FunctionTok{install\_github}\NormalTok{(}\StringTok{"aavotins/egvtools"}\NormalTok{); }\FunctionTok{require}\NormalTok{(egvtools)\}}


\CommentTok{\# Templates {-}{-}{-}{-}{-}}
\NormalTok{template100}\OtherTok{=}\FunctionTok{rast}\NormalTok{(}\StringTok{"./Templates/TemplateRasters/LV100m\_10km.tif"}\NormalTok{)}

\CommentTok{\# radii {-}{-}{-}{-}}
\FunctionTok{radius\_function}\NormalTok{(}
 \AttributeTok{kvadrati\_path =} \StringTok{"./Templates/TemplateGrids/tiles/"}\NormalTok{,}
 \AttributeTok{radii\_path   =} \StringTok{"./Templates/TemplateGridPoints/tiles/"}\NormalTok{,}
 \AttributeTok{tikls100\_path =} \StringTok{"./Templates/TemplateGrids/tikls100\_sauzeme.parquet"}\NormalTok{,}
 \AttributeTok{template\_path =} \StringTok{"./Templates/TemplateRasters/LV100m\_10km.tif"}\NormalTok{,}
 \AttributeTok{input\_layers  =} \FunctionTok{c}\NormalTok{(}\StringTok{"./RasterGrids\_100m/2024/RAW/Edges\_Bogs{-}Water\_cell.tif"}\NormalTok{),}
 \AttributeTok{layer\_prefixes =} \FunctionTok{c}\NormalTok{(}\StringTok{"Edges\_Bogs{-}Water"}\NormalTok{),}
 \AttributeTok{output\_dir   =} \StringTok{"./RasterGrids\_100m/2024/RAW/"}\NormalTok{,}
 \AttributeTok{n\_workers   =} \DecValTok{12}\NormalTok{,}
 \AttributeTok{radii     =} \FunctionTok{c}\NormalTok{(}\StringTok{"r10000"}\NormalTok{),}
 \AttributeTok{radius\_mode  =} \StringTok{"sparse"}\NormalTok{,}
 \AttributeTok{extract\_fun  =} \StringTok{"sum"}\NormalTok{,}
 \AttributeTok{fill\_missing  =} \ConstantTok{TRUE}\NormalTok{,}
 \AttributeTok{IDW\_weight   =} \DecValTok{2}\NormalTok{,}
 \AttributeTok{future\_max\_size =} \DecValTok{20} \SpecialCharTok{*} \DecValTok{1024}\SpecialCharTok{\^{}}\DecValTok{3}\NormalTok{)}

\CommentTok{\# Edges\_Bogs{-}Water\_r10000.tif   egv\_119}
\NormalTok{slanis}\OtherTok{=}\FunctionTok{rast}\NormalTok{(}\StringTok{"./RasterGrids\_100m/2024/RAW/Edges\_Bogs{-}Water\_r10000.tif"}\NormalTok{)}
\FunctionTok{names}\NormalTok{(slanis)}\OtherTok{=}\StringTok{"egv\_119"}
\NormalTok{slanis2}\OtherTok{=}\FunctionTok{project}\NormalTok{(slanis,template100)}
\FunctionTok{writeRaster}\NormalTok{(slanis2,}
      \StringTok{"./RasterGrids\_100m/2024/RAW/Edges\_Bogs{-}Water\_r10000.tif"}\NormalTok{,}
      \AttributeTok{overwrite=}\ConstantTok{TRUE}\NormalTok{)}

\CommentTok{\# standardisation {-}{-}{-}{-}}
\ControlFlowTok{if}\NormalTok{(}\SpecialCharTok{!}\FunctionTok{require}\NormalTok{(terra)) \{}\FunctionTok{install.packages}\NormalTok{(}\StringTok{"terra"}\NormalTok{); }\FunctionTok{require}\NormalTok{(terra)\}}
\ControlFlowTok{if}\NormalTok{(}\SpecialCharTok{!}\FunctionTok{require}\NormalTok{(tidyverse)) \{}\FunctionTok{install.packages}\NormalTok{(}\StringTok{"tidyverse"}\NormalTok{); }\FunctionTok{require}\NormalTok{(tidyverse)\}}

\NormalTok{nosaukums}\OtherTok{=}\StringTok{"Edges\_Bogs{-}Water\_r10000.tif"}
\NormalTok{ielasisanas\_cels}\OtherTok{=}\FunctionTok{paste0}\NormalTok{(}\StringTok{"./RasterGrids\_100m/2024/RAW/"}\NormalTok{,nosaukums)}
\NormalTok{saglabasanas\_cels}\OtherTok{=}\FunctionTok{paste0}\NormalTok{(}\StringTok{"./RasterGrids\_100m/2024/Scaled/"}\NormalTok{,nosaukums)}
\NormalTok{slanis}\OtherTok{=}\FunctionTok{rast}\NormalTok{(ielasisanas\_cels)}
\NormalTok{videjais}\OtherTok{=}\FunctionTok{global}\NormalTok{(slanis,}\AttributeTok{fun=}\StringTok{"mean"}\NormalTok{,}\AttributeTok{na.rm=}\ConstantTok{TRUE}\NormalTok{)}
\NormalTok{centrets}\OtherTok{=}\NormalTok{slanis}\SpecialCharTok{{-}}\NormalTok{videjais[,}\DecValTok{1}\NormalTok{]}
\NormalTok{standartnovirze}\OtherTok{=}\NormalTok{terra}\SpecialCharTok{::}\FunctionTok{global}\NormalTok{(centrets,}\AttributeTok{fun=}\StringTok{"rms"}\NormalTok{,}\AttributeTok{na.rm=}\ConstantTok{TRUE}\NormalTok{)}
\NormalTok{merogots}\OtherTok{=}\NormalTok{centrets}\SpecialCharTok{/}\NormalTok{standartnovirze[,}\DecValTok{1}\NormalTok{]}
\FunctionTok{writeRaster}\NormalTok{(merogots,}
      \AttributeTok{filename=}\NormalTok{saglabasanas\_cels,}
      \AttributeTok{overwrite=}\ConstantTok{TRUE}\NormalTok{)}
\end{Highlighting}
\end{Shaded}

\section{Edges\_Farmland-Builtup\_cell}\label{ch06.120}

\textbf{filename:} \texttt{Edges\_Farmland-Builtup\_cell.tif}

\textbf{layername:} \texttt{egv\_120}

\textbf{English name:} Edge pixels of Farmland bordering with Built-Up areas within
the analysis cell (1 ha)

\textbf{Latvian name:} Lauksaimniecības zemju malu ar apbūvi pikseļu skaits analīzes šūnā (1
ha)

\textbf{Procedure:} First, values larger than 300 and smaller than 400 from
\hyperref[Ch05.03]{Landscape classification} are coded as 1, and all other values as NA.
Then values 500 from the \hyperref[Ch05.03]{Landscape classification} are coded as 0, and all
other values as NA. Then, the first layer (1 = presence) is covered over the
second layer (presence = 0) and written to file (matching the input). Next,
using the workflow \texttt{egvtools::landscape\_function()} total edge between the two
classes is calculated. During the calculation of the landscape metric, inverse distance
weighted (power = 2) gap filling on the output is applied to ensure no
missing values at the edges. Finally, the layer is standardised by
subtracting the arithmetic mean and dividing by the root mean squared error.

\begin{Shaded}
\begin{Highlighting}[]
\CommentTok{\# libs {-}{-}{-}{-}}
\ControlFlowTok{if}\NormalTok{(}\SpecialCharTok{!}\FunctionTok{require}\NormalTok{(terra)) \{}\FunctionTok{install.packages}\NormalTok{(}\StringTok{"terra"}\NormalTok{); }\FunctionTok{require}\NormalTok{(terra)\}}
\ControlFlowTok{if}\NormalTok{(}\SpecialCharTok{!}\FunctionTok{require}\NormalTok{(egvtools)) \{remotes}\SpecialCharTok{::}\FunctionTok{install\_github}\NormalTok{(}\StringTok{"aavotins/egvtools"}\NormalTok{); }\FunctionTok{require}\NormalTok{(egvtools)\}}

\ControlFlowTok{if}\NormalTok{(}\SpecialCharTok{!}\FunctionTok{require}\NormalTok{(sf)) \{}\FunctionTok{install.packages}\NormalTok{(}\StringTok{"sf"}\NormalTok{); }\FunctionTok{require}\NormalTok{(sf)\}}
\ControlFlowTok{if}\NormalTok{(}\SpecialCharTok{!}\FunctionTok{require}\NormalTok{(sfarrow)) \{}\FunctionTok{install.packages}\NormalTok{(}\StringTok{"sfarrow"}\NormalTok{); }\FunctionTok{require}\NormalTok{(sfarrow)\}}
\ControlFlowTok{if}\NormalTok{(}\SpecialCharTok{!}\FunctionTok{require}\NormalTok{(raster)) \{}\FunctionTok{install.packages}\NormalTok{(}\StringTok{"raster"}\NormalTok{); }\FunctionTok{require}\NormalTok{(raster)\}}
\ControlFlowTok{if}\NormalTok{(}\SpecialCharTok{!}\FunctionTok{require}\NormalTok{(fasterize)) \{}\FunctionTok{install.packages}\NormalTok{(}\StringTok{"fasterize"}\NormalTok{); }\FunctionTok{require}\NormalTok{(fasterize)\}}
\ControlFlowTok{if}\NormalTok{(}\SpecialCharTok{!}\FunctionTok{require}\NormalTok{(tidyverse)) \{}\FunctionTok{install.packages}\NormalTok{(}\StringTok{"tidyverse"}\NormalTok{); }\FunctionTok{require}\NormalTok{(tidyverse)\}}


\CommentTok{\# Templates {-}{-}{-}{-}{-}}
\NormalTok{template10}\OtherTok{=}\FunctionTok{rast}\NormalTok{(}\StringTok{"./Templates/TemplateRasters/LV10m\_10km.tif"}\NormalTok{)}
\NormalTok{nulls10}\OtherTok{=}\FunctionTok{rast}\NormalTok{(}\StringTok{"./Templates/TemplateRasters/nulls\_LV10m\_10km.tif"}\NormalTok{)}

\CommentTok{\# simple landscape {-}{-}{-}{-}}
\NormalTok{simple\_landscape}\OtherTok{=}\FunctionTok{rast}\NormalTok{(}\StringTok{"./RasterGrids\_10m/2024/Ainava\_vienk\_mask.tif"}\NormalTok{)}

\CommentTok{\# Edges\_Farmland{-}Builtup\_input.tif {-}{-}{-}{-}}
\NormalTok{farmland}\OtherTok{=}\FunctionTok{ifel}\NormalTok{(simple\_landscape}\SpecialCharTok{\textgreater{}}\DecValTok{300} \SpecialCharTok{\&}\NormalTok{ simple\_landscape}\SpecialCharTok{\textless{}}\DecValTok{400}\NormalTok{,}\DecValTok{1}\NormalTok{,}\ConstantTok{NA}\NormalTok{)}
\FunctionTok{plot}\NormalTok{(farmland)}

\NormalTok{builtup}\OtherTok{=}\FunctionTok{ifel}\NormalTok{(simple\_landscape}\SpecialCharTok{==}\DecValTok{500}\NormalTok{,}\DecValTok{0}\NormalTok{,}\ConstantTok{NA}\NormalTok{)}
\FunctionTok{plot}\NormalTok{(builtup)}

\NormalTok{farmland\_builtup}\OtherTok{=}\FunctionTok{cover}\NormalTok{(farmland,builtup)}
\FunctionTok{plot}\NormalTok{(farmland\_builtup)}

\NormalTok{edge\_farmland\_builtup}\OtherTok{=}\FunctionTok{project}\NormalTok{(farmland\_builtup,template10,}
           \AttributeTok{filename=}\StringTok{"./RasterGrids\_10m/2024/Edges\_Farmland{-}Builtup\_input.tif"}\NormalTok{,}
           \AttributeTok{overwrite=}\ConstantTok{TRUE}\NormalTok{)}
\FunctionTok{rm}\NormalTok{(edge\_farmland\_builtup)}
\FunctionTok{rm}\NormalTok{(farmland\_builtup)}


\CommentTok{\# Edges\_Farmland{-}Builtup\_cell.tif   egv\_120 {-}{-}{-}{-}}
\FunctionTok{landscape\_function}\NormalTok{(}
 \AttributeTok{landscape   =} \StringTok{"./RasterGrids\_10m/2024/Edges\_Farmland{-}Builtup\_input.tif"}\NormalTok{,}
 \AttributeTok{zones     =} \StringTok{"./Templates/TemplateGrids/tikls100\_sauzeme.parquet"}\NormalTok{,}
 \AttributeTok{id\_field    =} \StringTok{"id"}\NormalTok{,}
 \AttributeTok{tile\_field   =} \StringTok{"tks50km"}\NormalTok{,}
 \AttributeTok{template    =} \StringTok{"./Templates/TemplateRasters/LV100m\_10km.tif"}\NormalTok{,}
 \AttributeTok{out\_dir    =} \StringTok{"./RasterGrids\_100m/2024/RAW"}\NormalTok{,}
 \AttributeTok{out\_filename  =} \StringTok{"Edges\_Farmland{-}Builtup\_cell.tif"}\NormalTok{,}
 \AttributeTok{out\_layername =} \StringTok{"egv\_120"}\NormalTok{,}
 \AttributeTok{what       =} \StringTok{"lsm\_l\_te"}\NormalTok{,}
 \AttributeTok{lm\_args     =} \FunctionTok{list}\NormalTok{(}\AttributeTok{count\_boundary =} \ConstantTok{FALSE}\NormalTok{),}
 \AttributeTok{rasterize\_engine =} \StringTok{"fasterize"}\NormalTok{,}
 \AttributeTok{n\_workers   =} \DecValTok{12}\NormalTok{,}
 \AttributeTok{future\_max\_size =} \DecValTok{20} \SpecialCharTok{*} \DecValTok{1024}\SpecialCharTok{\^{}}\DecValTok{3}\NormalTok{,}
 \AttributeTok{fill\_gaps   =} \ConstantTok{TRUE}\NormalTok{,}
 \AttributeTok{plot\_gaps   =} \ConstantTok{FALSE}\NormalTok{,}
 \AttributeTok{plot\_result  =} \ConstantTok{FALSE}
\NormalTok{)}

\CommentTok{\# standardisation {-}{-}{-}{-}}
\ControlFlowTok{if}\NormalTok{(}\SpecialCharTok{!}\FunctionTok{require}\NormalTok{(terra)) \{}\FunctionTok{install.packages}\NormalTok{(}\StringTok{"terra"}\NormalTok{); }\FunctionTok{require}\NormalTok{(terra)\}}
\ControlFlowTok{if}\NormalTok{(}\SpecialCharTok{!}\FunctionTok{require}\NormalTok{(tidyverse)) \{}\FunctionTok{install.packages}\NormalTok{(}\StringTok{"tidyverse"}\NormalTok{); }\FunctionTok{require}\NormalTok{(tidyverse)\}}

\NormalTok{nosaukums}\OtherTok{=}\StringTok{"Edges\_Farmland{-}Builtup\_cell.tif"}
\NormalTok{ielasisanas\_cels}\OtherTok{=}\FunctionTok{paste0}\NormalTok{(}\StringTok{"./RasterGrids\_100m/2024/RAW/"}\NormalTok{,nosaukums)}
\NormalTok{saglabasanas\_cels}\OtherTok{=}\FunctionTok{paste0}\NormalTok{(}\StringTok{"./RasterGrids\_100m/2024/Scaled/"}\NormalTok{,nosaukums)}
\NormalTok{slanis}\OtherTok{=}\FunctionTok{rast}\NormalTok{(ielasisanas\_cels)}
\NormalTok{videjais}\OtherTok{=}\FunctionTok{global}\NormalTok{(slanis,}\AttributeTok{fun=}\StringTok{"mean"}\NormalTok{,}\AttributeTok{na.rm=}\ConstantTok{TRUE}\NormalTok{)}
\NormalTok{centrets}\OtherTok{=}\NormalTok{slanis}\SpecialCharTok{{-}}\NormalTok{videjais[,}\DecValTok{1}\NormalTok{]}
\NormalTok{standartnovirze}\OtherTok{=}\NormalTok{terra}\SpecialCharTok{::}\FunctionTok{global}\NormalTok{(centrets,}\AttributeTok{fun=}\StringTok{"rms"}\NormalTok{,}\AttributeTok{na.rm=}\ConstantTok{TRUE}\NormalTok{)}
\NormalTok{merogots}\OtherTok{=}\NormalTok{centrets}\SpecialCharTok{/}\NormalTok{standartnovirze[,}\DecValTok{1}\NormalTok{]}
\FunctionTok{writeRaster}\NormalTok{(merogots,}
      \AttributeTok{filename=}\NormalTok{saglabasanas\_cels,}
      \AttributeTok{overwrite=}\ConstantTok{TRUE}\NormalTok{)}
\end{Highlighting}
\end{Shaded}

\section{Edges\_Farmland-Builtup\_r500}\label{ch06.121}

\textbf{filename:} \texttt{Edges\_Farmland-Builtup\_r500.tif}

\textbf{layername:} \texttt{egv\_121}

\textbf{English name:} Edge pixels of Farmland bordering with Built-Up areas within
the 0.5 km landscape

\textbf{Latvian name:} Lauksaimniecības zemju malu ar apbūvi pikseļu skaits 0,5 km ainavā

\textbf{Procedure:} The total edge within a 500 m radius around the analysis grid cell is
calculated as the area-weighted sum of the \hyperref[ch06.120]{analysis cells} inside the
buffer, using the workflow \texttt{egvtools::radius\_function()}. During the calculation of the landscape metric,
inverse distance weighted (power = 2) gap filling on the output is applied
to ensure no missing values at the edges. Then the layer is rewritten to set
its name. Finally, the layer is standardised by subtracting the arithmetic
mean and dividing by the root mean squared error.

\begin{Shaded}
\begin{Highlighting}[]
\CommentTok{\# libs {-}{-}{-}{-}}
\ControlFlowTok{if}\NormalTok{(}\SpecialCharTok{!}\FunctionTok{require}\NormalTok{(terra)) \{}\FunctionTok{install.packages}\NormalTok{(}\StringTok{"terra"}\NormalTok{); }\FunctionTok{require}\NormalTok{(terra)\}}
\ControlFlowTok{if}\NormalTok{(}\SpecialCharTok{!}\FunctionTok{require}\NormalTok{(egvtools)) \{remotes}\SpecialCharTok{::}\FunctionTok{install\_github}\NormalTok{(}\StringTok{"aavotins/egvtools"}\NormalTok{); }\FunctionTok{require}\NormalTok{(egvtools)\}}


\CommentTok{\# Templates {-}{-}{-}{-}{-}}
\NormalTok{template100}\OtherTok{=}\FunctionTok{rast}\NormalTok{(}\StringTok{"./Templates/TemplateRasters/LV100m\_10km.tif"}\NormalTok{)}

\CommentTok{\# radii {-}{-}{-}{-}}
\FunctionTok{radius\_function}\NormalTok{(}
 \AttributeTok{kvadrati\_path =} \StringTok{"./Templates/TemplateGrids/tiles/"}\NormalTok{,}
 \AttributeTok{radii\_path   =} \StringTok{"./Templates/TemplateGridPoints/tiles/"}\NormalTok{,}
 \AttributeTok{tikls100\_path =} \StringTok{"./Templates/TemplateGrids/tikls100\_sauzeme.parquet"}\NormalTok{,}
 \AttributeTok{template\_path =} \StringTok{"./Templates/TemplateRasters/LV100m\_10km.tif"}\NormalTok{,}
 \AttributeTok{input\_layers  =} \FunctionTok{c}\NormalTok{(}\StringTok{"./RasterGrids\_100m/2024/RAW/Edges\_Farmland{-}Builtup\_cell.tif"}\NormalTok{),}
 \AttributeTok{layer\_prefixes =} \FunctionTok{c}\NormalTok{(}\StringTok{"Edges\_Farmland{-}Builtup"}\NormalTok{),}
 \AttributeTok{output\_dir   =} \StringTok{"./RasterGrids\_100m/2024/RAW/"}\NormalTok{,}
 \AttributeTok{n\_workers   =} \DecValTok{12}\NormalTok{,}
 \AttributeTok{radii     =} \FunctionTok{c}\NormalTok{(}\StringTok{"r500"}\NormalTok{),}
 \AttributeTok{radius\_mode  =} \StringTok{"sparse"}\NormalTok{,}
 \AttributeTok{extract\_fun  =} \StringTok{"sum"}\NormalTok{,}
 \AttributeTok{fill\_missing  =} \ConstantTok{TRUE}\NormalTok{,}
 \AttributeTok{IDW\_weight   =} \DecValTok{2}\NormalTok{,}
 \AttributeTok{future\_max\_size =} \DecValTok{20} \SpecialCharTok{*} \DecValTok{1024}\SpecialCharTok{\^{}}\DecValTok{3}\NormalTok{)}


\CommentTok{\# Edges\_Farmland{-}Builtup\_r500.tif   egv\_121 {-}{-}{-}{-}}
\NormalTok{slanis}\OtherTok{=}\FunctionTok{rast}\NormalTok{(}\StringTok{"./RasterGrids\_100m/2024/RAW/Edges\_Farmland{-}Builtup\_r500.tif"}\NormalTok{)}
\FunctionTok{names}\NormalTok{(slanis)}\OtherTok{=}\StringTok{"egv\_121"}
\NormalTok{slanis2}\OtherTok{=}\FunctionTok{project}\NormalTok{(slanis,template100)}
\FunctionTok{writeRaster}\NormalTok{(slanis2,}
      \StringTok{"./RasterGrids\_100m/2024/RAW/Edges\_Farmland{-}Builtup\_r500.tif"}\NormalTok{,}
      \AttributeTok{overwrite=}\ConstantTok{TRUE}\NormalTok{)}

\CommentTok{\# standardisation {-}{-}{-}{-}}
\ControlFlowTok{if}\NormalTok{(}\SpecialCharTok{!}\FunctionTok{require}\NormalTok{(terra)) \{}\FunctionTok{install.packages}\NormalTok{(}\StringTok{"terra"}\NormalTok{); }\FunctionTok{require}\NormalTok{(terra)\}}
\ControlFlowTok{if}\NormalTok{(}\SpecialCharTok{!}\FunctionTok{require}\NormalTok{(tidyverse)) \{}\FunctionTok{install.packages}\NormalTok{(}\StringTok{"tidyverse"}\NormalTok{); }\FunctionTok{require}\NormalTok{(tidyverse)\}}

\NormalTok{nosaukums}\OtherTok{=}\StringTok{"Edges\_Farmland{-}Builtup\_r500.tif"}
\NormalTok{ielasisanas\_cels}\OtherTok{=}\FunctionTok{paste0}\NormalTok{(}\StringTok{"./RasterGrids\_100m/2024/RAW/"}\NormalTok{,nosaukums)}
\NormalTok{saglabasanas\_cels}\OtherTok{=}\FunctionTok{paste0}\NormalTok{(}\StringTok{"./RasterGrids\_100m/2024/Scaled/"}\NormalTok{,nosaukums)}
\NormalTok{slanis}\OtherTok{=}\FunctionTok{rast}\NormalTok{(ielasisanas\_cels)}
\NormalTok{videjais}\OtherTok{=}\FunctionTok{global}\NormalTok{(slanis,}\AttributeTok{fun=}\StringTok{"mean"}\NormalTok{,}\AttributeTok{na.rm=}\ConstantTok{TRUE}\NormalTok{)}
\NormalTok{centrets}\OtherTok{=}\NormalTok{slanis}\SpecialCharTok{{-}}\NormalTok{videjais[,}\DecValTok{1}\NormalTok{]}
\NormalTok{standartnovirze}\OtherTok{=}\NormalTok{terra}\SpecialCharTok{::}\FunctionTok{global}\NormalTok{(centrets,}\AttributeTok{fun=}\StringTok{"rms"}\NormalTok{,}\AttributeTok{na.rm=}\ConstantTok{TRUE}\NormalTok{)}
\NormalTok{merogots}\OtherTok{=}\NormalTok{centrets}\SpecialCharTok{/}\NormalTok{standartnovirze[,}\DecValTok{1}\NormalTok{]}
\FunctionTok{writeRaster}\NormalTok{(merogots,}
      \AttributeTok{filename=}\NormalTok{saglabasanas\_cels,}
      \AttributeTok{overwrite=}\ConstantTok{TRUE}\NormalTok{)}
\end{Highlighting}
\end{Shaded}

\section{Edges\_Farmland-Builtup\_r1250}\label{ch06.122}

\textbf{filename:} \texttt{Edges\_Farmland-Builtup\_r1250.tif}

\textbf{layername:} \texttt{egv\_122}

\textbf{English name:} Edge pixels of Farmland bordering with Built-Up areas within
the 1.25 km landscape

\textbf{Latvian name:} Lauksaimniecības zemju malu ar apbūvi pikseļu skaits 1,25 km ainavā

\textbf{Procedure:} The total edge within a 1250 m radius around the analysis grid cell is
calculated as the area-weighted sum of the \hyperref[ch06.120]{analysis cells} inside the
buffer, using the workflow \texttt{egvtools::radius\_function()}. During the calculation of the landscape metric,
inverse distance weighted (power = 2) gap filling on the output is applied
to ensure no missing values at the edges. Then the layer is rewritten to set
its name. Finally, the layer is standardised by subtracting the arithmetic
mean and dividing by the root mean squared error.

\begin{Shaded}
\begin{Highlighting}[]
\CommentTok{\# libs {-}{-}{-}{-}}
\ControlFlowTok{if}\NormalTok{(}\SpecialCharTok{!}\FunctionTok{require}\NormalTok{(terra)) \{}\FunctionTok{install.packages}\NormalTok{(}\StringTok{"terra"}\NormalTok{); }\FunctionTok{require}\NormalTok{(terra)\}}
\ControlFlowTok{if}\NormalTok{(}\SpecialCharTok{!}\FunctionTok{require}\NormalTok{(egvtools)) \{remotes}\SpecialCharTok{::}\FunctionTok{install\_github}\NormalTok{(}\StringTok{"aavotins/egvtools"}\NormalTok{); }\FunctionTok{require}\NormalTok{(egvtools)\}}


\CommentTok{\# Templates {-}{-}{-}{-}{-}}
\NormalTok{template100}\OtherTok{=}\FunctionTok{rast}\NormalTok{(}\StringTok{"./Templates/TemplateRasters/LV100m\_10km.tif"}\NormalTok{)}

\CommentTok{\# radii {-}{-}{-}{-}}
\FunctionTok{radius\_function}\NormalTok{(}
 \AttributeTok{kvadrati\_path =} \StringTok{"./Templates/TemplateGrids/tiles/"}\NormalTok{,}
 \AttributeTok{radii\_path   =} \StringTok{"./Templates/TemplateGridPoints/tiles/"}\NormalTok{,}
 \AttributeTok{tikls100\_path =} \StringTok{"./Templates/TemplateGrids/tikls100\_sauzeme.parquet"}\NormalTok{,}
 \AttributeTok{template\_path =} \StringTok{"./Templates/TemplateRasters/LV100m\_10km.tif"}\NormalTok{,}
 \AttributeTok{input\_layers  =} \FunctionTok{c}\NormalTok{(}\StringTok{"./RasterGrids\_100m/2024/RAW/Edges\_Farmland{-}Builtup\_cell.tif"}\NormalTok{),}
 \AttributeTok{layer\_prefixes =} \FunctionTok{c}\NormalTok{(}\StringTok{"Edges\_Farmland{-}Builtup"}\NormalTok{),}
 \AttributeTok{output\_dir   =} \StringTok{"./RasterGrids\_100m/2024/RAW/"}\NormalTok{,}
 \AttributeTok{n\_workers   =} \DecValTok{12}\NormalTok{,}
 \AttributeTok{radii     =} \FunctionTok{c}\NormalTok{(}\StringTok{"r1250"}\NormalTok{),}
 \AttributeTok{radius\_mode  =} \StringTok{"sparse"}\NormalTok{,}
 \AttributeTok{extract\_fun  =} \StringTok{"sum"}\NormalTok{,}
 \AttributeTok{fill\_missing  =} \ConstantTok{TRUE}\NormalTok{,}
 \AttributeTok{IDW\_weight   =} \DecValTok{2}\NormalTok{,}
 \AttributeTok{future\_max\_size =} \DecValTok{20} \SpecialCharTok{*} \DecValTok{1024}\SpecialCharTok{\^{}}\DecValTok{3}\NormalTok{)}


\CommentTok{\# Edges\_Farmland{-}Builtup\_r1250.tif  egv\_122 {-}{-}{-}{-}}
\NormalTok{slanis}\OtherTok{=}\FunctionTok{rast}\NormalTok{(}\StringTok{"./RasterGrids\_100m/2024/RAW/Edges\_Farmland{-}Builtup\_r1250.tif"}\NormalTok{)}
\FunctionTok{names}\NormalTok{(slanis)}\OtherTok{=}\StringTok{"egv\_122"}
\NormalTok{slanis2}\OtherTok{=}\FunctionTok{project}\NormalTok{(slanis,template100)}
\FunctionTok{writeRaster}\NormalTok{(slanis2,}
      \StringTok{"./RasterGrids\_100m/2024/RAW/Edges\_Farmland{-}Builtup\_r1250.tif"}\NormalTok{,}
      \AttributeTok{overwrite=}\ConstantTok{TRUE}\NormalTok{)}

\CommentTok{\# standardisation {-}{-}{-}{-}}
\ControlFlowTok{if}\NormalTok{(}\SpecialCharTok{!}\FunctionTok{require}\NormalTok{(terra)) \{}\FunctionTok{install.packages}\NormalTok{(}\StringTok{"terra"}\NormalTok{); }\FunctionTok{require}\NormalTok{(terra)\}}
\ControlFlowTok{if}\NormalTok{(}\SpecialCharTok{!}\FunctionTok{require}\NormalTok{(tidyverse)) \{}\FunctionTok{install.packages}\NormalTok{(}\StringTok{"tidyverse"}\NormalTok{); }\FunctionTok{require}\NormalTok{(tidyverse)\}}

\NormalTok{nosaukums}\OtherTok{=}\StringTok{"Edges\_Farmland{-}Builtup\_r1250.tif"}
\NormalTok{ielasisanas\_cels}\OtherTok{=}\FunctionTok{paste0}\NormalTok{(}\StringTok{"./RasterGrids\_100m/2024/RAW/"}\NormalTok{,nosaukums)}
\NormalTok{saglabasanas\_cels}\OtherTok{=}\FunctionTok{paste0}\NormalTok{(}\StringTok{"./RasterGrids\_100m/2024/Scaled/"}\NormalTok{,nosaukums)}
\NormalTok{slanis}\OtherTok{=}\FunctionTok{rast}\NormalTok{(ielasisanas\_cels)}
\NormalTok{videjais}\OtherTok{=}\FunctionTok{global}\NormalTok{(slanis,}\AttributeTok{fun=}\StringTok{"mean"}\NormalTok{,}\AttributeTok{na.rm=}\ConstantTok{TRUE}\NormalTok{)}
\NormalTok{centrets}\OtherTok{=}\NormalTok{slanis}\SpecialCharTok{{-}}\NormalTok{videjais[,}\DecValTok{1}\NormalTok{]}
\NormalTok{standartnovirze}\OtherTok{=}\NormalTok{terra}\SpecialCharTok{::}\FunctionTok{global}\NormalTok{(centrets,}\AttributeTok{fun=}\StringTok{"rms"}\NormalTok{,}\AttributeTok{na.rm=}\ConstantTok{TRUE}\NormalTok{)}
\NormalTok{merogots}\OtherTok{=}\NormalTok{centrets}\SpecialCharTok{/}\NormalTok{standartnovirze[,}\DecValTok{1}\NormalTok{]}
\FunctionTok{writeRaster}\NormalTok{(merogots,}
      \AttributeTok{filename=}\NormalTok{saglabasanas\_cels,}
      \AttributeTok{overwrite=}\ConstantTok{TRUE}\NormalTok{)}
\end{Highlighting}
\end{Shaded}

\section{Edges\_Farmland-Builtup\_r3000}\label{ch06.123}

\textbf{filename:} \texttt{Edges\_Farmland-Builtup\_r3000.tif}

\textbf{layername:} \texttt{egv\_123}

\textbf{English name:} Edge pixels of Farmland bordering with Built-Up areas within
the 3 km landscape

\textbf{Latvian name:} Lauksaimniecības zemju malu ar apbūvi pikseļu skaits 3 km ainavā

\textbf{Procedure:} The total edge within a 3000 m radius around the analysis grid cell is
calculated as the area-weighted sum of the \hyperref[ch06.120]{analysis cells} inside the
buffer, using the workflow \texttt{egvtools::radius\_function()}. During the calculation of the landscape metric,
inverse distance weighted (power = 2) gap filling on the output is applied
to ensure no missing values at the edges. Then the layer is rewritten to set
its name. Finally, the layer is standardised by subtracting the arithmetic
mean and dividing by the root mean squared error.

\begin{Shaded}
\begin{Highlighting}[]
\CommentTok{\# libs {-}{-}{-}{-}}
\ControlFlowTok{if}\NormalTok{(}\SpecialCharTok{!}\FunctionTok{require}\NormalTok{(terra)) \{}\FunctionTok{install.packages}\NormalTok{(}\StringTok{"terra"}\NormalTok{); }\FunctionTok{require}\NormalTok{(terra)\}}
\ControlFlowTok{if}\NormalTok{(}\SpecialCharTok{!}\FunctionTok{require}\NormalTok{(egvtools)) \{remotes}\SpecialCharTok{::}\FunctionTok{install\_github}\NormalTok{(}\StringTok{"aavotins/egvtools"}\NormalTok{); }\FunctionTok{require}\NormalTok{(egvtools)\}}


\CommentTok{\# Templates {-}{-}{-}{-}{-}}
\NormalTok{template100}\OtherTok{=}\FunctionTok{rast}\NormalTok{(}\StringTok{"./Templates/TemplateRasters/LV100m\_10km.tif"}\NormalTok{)}

\CommentTok{\# radii {-}{-}{-}{-}}
\FunctionTok{radius\_function}\NormalTok{(}
 \AttributeTok{kvadrati\_path =} \StringTok{"./Templates/TemplateGrids/tiles/"}\NormalTok{,}
 \AttributeTok{radii\_path   =} \StringTok{"./Templates/TemplateGridPoints/tiles/"}\NormalTok{,}
 \AttributeTok{tikls100\_path =} \StringTok{"./Templates/TemplateGrids/tikls100\_sauzeme.parquet"}\NormalTok{,}
 \AttributeTok{template\_path =} \StringTok{"./Templates/TemplateRasters/LV100m\_10km.tif"}\NormalTok{,}
 \AttributeTok{input\_layers  =} \FunctionTok{c}\NormalTok{(}\StringTok{"./RasterGrids\_100m/2024/RAW/Edges\_Farmland{-}Builtup\_cell.tif"}\NormalTok{),}
 \AttributeTok{layer\_prefixes =} \FunctionTok{c}\NormalTok{(}\StringTok{"Edges\_Farmland{-}Builtup"}\NormalTok{),}
 \AttributeTok{output\_dir   =} \StringTok{"./RasterGrids\_100m/2024/RAW/"}\NormalTok{,}
 \AttributeTok{n\_workers   =} \DecValTok{12}\NormalTok{,}
 \AttributeTok{radii     =} \FunctionTok{c}\NormalTok{(}\StringTok{"r3000"}\NormalTok{),}
 \AttributeTok{radius\_mode  =} \StringTok{"sparse"}\NormalTok{,}
 \AttributeTok{extract\_fun  =} \StringTok{"sum"}\NormalTok{,}
 \AttributeTok{fill\_missing  =} \ConstantTok{TRUE}\NormalTok{,}
 \AttributeTok{IDW\_weight   =} \DecValTok{2}\NormalTok{,}
 \AttributeTok{future\_max\_size =} \DecValTok{20} \SpecialCharTok{*} \DecValTok{1024}\SpecialCharTok{\^{}}\DecValTok{3}\NormalTok{)}


\CommentTok{\# Edges\_Farmland{-}Builtup\_r3000.tif  egv\_123 {-}{-}{-}{-}}
\NormalTok{slanis}\OtherTok{=}\FunctionTok{rast}\NormalTok{(}\StringTok{"./RasterGrids\_100m/2024/RAW/Edges\_Farmland{-}Builtup\_r3000.tif"}\NormalTok{)}
\FunctionTok{names}\NormalTok{(slanis)}\OtherTok{=}\StringTok{"egv\_123"}
\NormalTok{slanis2}\OtherTok{=}\FunctionTok{project}\NormalTok{(slanis,template100)}
\FunctionTok{writeRaster}\NormalTok{(slanis2,}
      \StringTok{"./RasterGrids\_100m/2024/RAW/Edges\_Farmland{-}Builtup\_r3000.tif"}\NormalTok{,}
      \AttributeTok{overwrite=}\ConstantTok{TRUE}\NormalTok{)}

\CommentTok{\# standardisation {-}{-}{-}{-}}
\ControlFlowTok{if}\NormalTok{(}\SpecialCharTok{!}\FunctionTok{require}\NormalTok{(terra)) \{}\FunctionTok{install.packages}\NormalTok{(}\StringTok{"terra"}\NormalTok{); }\FunctionTok{require}\NormalTok{(terra)\}}
\ControlFlowTok{if}\NormalTok{(}\SpecialCharTok{!}\FunctionTok{require}\NormalTok{(tidyverse)) \{}\FunctionTok{install.packages}\NormalTok{(}\StringTok{"tidyverse"}\NormalTok{); }\FunctionTok{require}\NormalTok{(tidyverse)\}}

\NormalTok{nosaukums}\OtherTok{=}\StringTok{"Edges\_Farmland{-}Builtup\_r3000.tif"}
\NormalTok{ielasisanas\_cels}\OtherTok{=}\FunctionTok{paste0}\NormalTok{(}\StringTok{"./RasterGrids\_100m/2024/RAW/"}\NormalTok{,nosaukums)}
\NormalTok{saglabasanas\_cels}\OtherTok{=}\FunctionTok{paste0}\NormalTok{(}\StringTok{"./RasterGrids\_100m/2024/Scaled/"}\NormalTok{,nosaukums)}
\NormalTok{slanis}\OtherTok{=}\FunctionTok{rast}\NormalTok{(ielasisanas\_cels)}
\NormalTok{videjais}\OtherTok{=}\FunctionTok{global}\NormalTok{(slanis,}\AttributeTok{fun=}\StringTok{"mean"}\NormalTok{,}\AttributeTok{na.rm=}\ConstantTok{TRUE}\NormalTok{)}
\NormalTok{centrets}\OtherTok{=}\NormalTok{slanis}\SpecialCharTok{{-}}\NormalTok{videjais[,}\DecValTok{1}\NormalTok{]}
\NormalTok{standartnovirze}\OtherTok{=}\NormalTok{terra}\SpecialCharTok{::}\FunctionTok{global}\NormalTok{(centrets,}\AttributeTok{fun=}\StringTok{"rms"}\NormalTok{,}\AttributeTok{na.rm=}\ConstantTok{TRUE}\NormalTok{)}
\NormalTok{merogots}\OtherTok{=}\NormalTok{centrets}\SpecialCharTok{/}\NormalTok{standartnovirze[,}\DecValTok{1}\NormalTok{]}
\FunctionTok{writeRaster}\NormalTok{(merogots,}
      \AttributeTok{filename=}\NormalTok{saglabasanas\_cels,}
      \AttributeTok{overwrite=}\ConstantTok{TRUE}\NormalTok{)}
\end{Highlighting}
\end{Shaded}

\section{Edges\_Farmland-Builtup\_r10000}\label{ch06.124}

\textbf{filename:} \texttt{Edges\_Farmland-Builtup\_r10000.tif}

\textbf{layername:} \texttt{egv\_124}

\textbf{English name:} Edge pixels of Farmland bordering with Built-Up areas within
the 10 km landscape

\textbf{Latvian name:} Lauksaimniecības zemju malu ar apbūvi pikseļu skaits 10 km ainavā

\textbf{Procedure:} The total edge within a 10000 m radius around the analysis grid cell is
calculated as the area-weighted sum of the \hyperref[ch06.120]{analysis cells} inside the
buffer, using the workflow \texttt{egvtools::radius\_function()}. During the calculation of the landscape metric,
inverse distance weighted (power = 2) gap filling on the output is applied
to ensure no missing values at the edges. Then the layer is rewritten to set
its name. Finally, the layer is standardised by subtracting the arithmetic
mean and dividing by the root mean squared error.

\begin{Shaded}
\begin{Highlighting}[]
\CommentTok{\# libs {-}{-}{-}{-}}
\ControlFlowTok{if}\NormalTok{(}\SpecialCharTok{!}\FunctionTok{require}\NormalTok{(terra)) \{}\FunctionTok{install.packages}\NormalTok{(}\StringTok{"terra"}\NormalTok{); }\FunctionTok{require}\NormalTok{(terra)\}}
\ControlFlowTok{if}\NormalTok{(}\SpecialCharTok{!}\FunctionTok{require}\NormalTok{(egvtools)) \{remotes}\SpecialCharTok{::}\FunctionTok{install\_github}\NormalTok{(}\StringTok{"aavotins/egvtools"}\NormalTok{); }\FunctionTok{require}\NormalTok{(egvtools)\}}


\CommentTok{\# Templates {-}{-}{-}{-}{-}}
\NormalTok{template100}\OtherTok{=}\FunctionTok{rast}\NormalTok{(}\StringTok{"./Templates/TemplateRasters/LV100m\_10km.tif"}\NormalTok{)}

\CommentTok{\# radii {-}{-}{-}{-}}
\FunctionTok{radius\_function}\NormalTok{(}
 \AttributeTok{kvadrati\_path =} \StringTok{"./Templates/TemplateGrids/tiles/"}\NormalTok{,}
 \AttributeTok{radii\_path   =} \StringTok{"./Templates/TemplateGridPoints/tiles/"}\NormalTok{,}
 \AttributeTok{tikls100\_path =} \StringTok{"./Templates/TemplateGrids/tikls100\_sauzeme.parquet"}\NormalTok{,}
 \AttributeTok{template\_path =} \StringTok{"./Templates/TemplateRasters/LV100m\_10km.tif"}\NormalTok{,}
 \AttributeTok{input\_layers  =} \FunctionTok{c}\NormalTok{(}\StringTok{"./RasterGrids\_100m/2024/RAW/Edges\_Farmland{-}Builtup\_cell.tif"}\NormalTok{),}
 \AttributeTok{layer\_prefixes =} \FunctionTok{c}\NormalTok{(}\StringTok{"Edges\_Farmland{-}Builtup"}\NormalTok{),}
 \AttributeTok{output\_dir   =} \StringTok{"./RasterGrids\_100m/2024/RAW/"}\NormalTok{,}
 \AttributeTok{n\_workers   =} \DecValTok{12}\NormalTok{,}
 \AttributeTok{radii     =} \FunctionTok{c}\NormalTok{(}\StringTok{"r10000"}\NormalTok{),}
 \AttributeTok{radius\_mode  =} \StringTok{"sparse"}\NormalTok{,}
 \AttributeTok{extract\_fun  =} \StringTok{"sum"}\NormalTok{,}
 \AttributeTok{fill\_missing  =} \ConstantTok{TRUE}\NormalTok{,}
 \AttributeTok{IDW\_weight   =} \DecValTok{2}\NormalTok{,}
 \AttributeTok{future\_max\_size =} \DecValTok{20} \SpecialCharTok{*} \DecValTok{1024}\SpecialCharTok{\^{}}\DecValTok{3}\NormalTok{)}


\CommentTok{\# Edges\_Farmland{-}Builtup\_r10000.tif egv\_124 {-}{-}{-}{-}}
\NormalTok{slanis}\OtherTok{=}\FunctionTok{rast}\NormalTok{(}\StringTok{"./RasterGrids\_100m/2024/RAW/Edges\_Farmland{-}Builtup\_r10000.tif"}\NormalTok{)}
\FunctionTok{names}\NormalTok{(slanis)}\OtherTok{=}\StringTok{"egv\_124"}
\NormalTok{slanis2}\OtherTok{=}\FunctionTok{project}\NormalTok{(slanis,template100)}
\FunctionTok{writeRaster}\NormalTok{(slanis2,}
      \StringTok{"./RasterGrids\_100m/2024/RAW/Edges\_Farmland{-}Builtup\_r10000.tif"}\NormalTok{,}
      \AttributeTok{overwrite=}\ConstantTok{TRUE}\NormalTok{)}

\CommentTok{\# standardisation {-}{-}{-}{-}}
\ControlFlowTok{if}\NormalTok{(}\SpecialCharTok{!}\FunctionTok{require}\NormalTok{(terra)) \{}\FunctionTok{install.packages}\NormalTok{(}\StringTok{"terra"}\NormalTok{); }\FunctionTok{require}\NormalTok{(terra)\}}
\ControlFlowTok{if}\NormalTok{(}\SpecialCharTok{!}\FunctionTok{require}\NormalTok{(tidyverse)) \{}\FunctionTok{install.packages}\NormalTok{(}\StringTok{"tidyverse"}\NormalTok{); }\FunctionTok{require}\NormalTok{(tidyverse)\}}

\NormalTok{nosaukums}\OtherTok{=}\StringTok{"Edges\_Farmland{-}Builtup\_r10000.tif"}
\NormalTok{ielasisanas\_cels}\OtherTok{=}\FunctionTok{paste0}\NormalTok{(}\StringTok{"./RasterGrids\_100m/2024/RAW/"}\NormalTok{,nosaukums)}
\NormalTok{saglabasanas\_cels}\OtherTok{=}\FunctionTok{paste0}\NormalTok{(}\StringTok{"./RasterGrids\_100m/2024/Scaled/"}\NormalTok{,nosaukums)}
\NormalTok{slanis}\OtherTok{=}\FunctionTok{rast}\NormalTok{(ielasisanas\_cels)}
\NormalTok{videjais}\OtherTok{=}\FunctionTok{global}\NormalTok{(slanis,}\AttributeTok{fun=}\StringTok{"mean"}\NormalTok{,}\AttributeTok{na.rm=}\ConstantTok{TRUE}\NormalTok{)}
\NormalTok{centrets}\OtherTok{=}\NormalTok{slanis}\SpecialCharTok{{-}}\NormalTok{videjais[,}\DecValTok{1}\NormalTok{]}
\NormalTok{standartnovirze}\OtherTok{=}\NormalTok{terra}\SpecialCharTok{::}\FunctionTok{global}\NormalTok{(centrets,}\AttributeTok{fun=}\StringTok{"rms"}\NormalTok{,}\AttributeTok{na.rm=}\ConstantTok{TRUE}\NormalTok{)}
\NormalTok{merogots}\OtherTok{=}\NormalTok{centrets}\SpecialCharTok{/}\NormalTok{standartnovirze[,}\DecValTok{1}\NormalTok{]}
\FunctionTok{writeRaster}\NormalTok{(merogots,}
      \AttributeTok{filename=}\NormalTok{saglabasanas\_cels,}
      \AttributeTok{overwrite=}\ConstantTok{TRUE}\NormalTok{)}
\end{Highlighting}
\end{Shaded}

\section{Edges\_Trees-Builtup\_cell}\label{ch06.125}

\textbf{filename:} \texttt{Edges\_Trees-Builtup\_cell.tif}

\textbf{layername:} \texttt{egv\_125}

\textbf{English name:} Edge pixels of Trees bordering with Built-Up areas within the
analysis cell (1 ha)

\textbf{Latvian name:} Koku malu ar apbūvi pikseļu skaits analīzes šūnā (1 ha)

\textbf{Procedure:} First, values larger than 630 and smaller than 700 from
\hyperref[Ch05.03]{Landscape classification} are coded as 1, and other values as NA.
Then values 500 from the \hyperref[Ch05.03]{Landscape classification} are coded as 0, and
other values as NA. Then, the first layer (1 = presence) is covered over the
second layer (presence = 0) and written to file (matching the input). Next,
using the workflow \texttt{egvtools::landscape\_function()} total edge between the two
classes is calculated. During the calculation of the landscape metric, inverse distance
weighted (power = 2) gap filling on the output is applied to ensure no
missing values at the edges. Finally, the layer is standardised by
subtracting the arithmetic mean and dividing by the root mean squared error.

\begin{Shaded}
\begin{Highlighting}[]
\CommentTok{\# libs {-}{-}{-}{-}}
\ControlFlowTok{if}\NormalTok{(}\SpecialCharTok{!}\FunctionTok{require}\NormalTok{(terra)) \{}\FunctionTok{install.packages}\NormalTok{(}\StringTok{"terra"}\NormalTok{); }\FunctionTok{require}\NormalTok{(terra)\}}
\ControlFlowTok{if}\NormalTok{(}\SpecialCharTok{!}\FunctionTok{require}\NormalTok{(egvtools)) \{remotes}\SpecialCharTok{::}\FunctionTok{install\_github}\NormalTok{(}\StringTok{"aavotins/egvtools"}\NormalTok{); }\FunctionTok{require}\NormalTok{(egvtools)\}}

\ControlFlowTok{if}\NormalTok{(}\SpecialCharTok{!}\FunctionTok{require}\NormalTok{(sf)) \{}\FunctionTok{install.packages}\NormalTok{(}\StringTok{"sf"}\NormalTok{); }\FunctionTok{require}\NormalTok{(sf)\}}
\ControlFlowTok{if}\NormalTok{(}\SpecialCharTok{!}\FunctionTok{require}\NormalTok{(sfarrow)) \{}\FunctionTok{install.packages}\NormalTok{(}\StringTok{"sfarrow"}\NormalTok{); }\FunctionTok{require}\NormalTok{(sfarrow)\}}
\ControlFlowTok{if}\NormalTok{(}\SpecialCharTok{!}\FunctionTok{require}\NormalTok{(raster)) \{}\FunctionTok{install.packages}\NormalTok{(}\StringTok{"raster"}\NormalTok{); }\FunctionTok{require}\NormalTok{(raster)\}}
\ControlFlowTok{if}\NormalTok{(}\SpecialCharTok{!}\FunctionTok{require}\NormalTok{(fasterize)) \{}\FunctionTok{install.packages}\NormalTok{(}\StringTok{"fasterize"}\NormalTok{); }\FunctionTok{require}\NormalTok{(fasterize)\}}
\ControlFlowTok{if}\NormalTok{(}\SpecialCharTok{!}\FunctionTok{require}\NormalTok{(tidyverse)) \{}\FunctionTok{install.packages}\NormalTok{(}\StringTok{"tidyverse"}\NormalTok{); }\FunctionTok{require}\NormalTok{(tidyverse)\}}


\CommentTok{\# Templates {-}{-}{-}{-}{-}}
\NormalTok{template10}\OtherTok{=}\FunctionTok{rast}\NormalTok{(}\StringTok{"./Templates/TemplateRasters/LV10m\_10km.tif"}\NormalTok{)}
\NormalTok{nulls10}\OtherTok{=}\FunctionTok{rast}\NormalTok{(}\StringTok{"./Templates/TemplateRasters/nulls\_LV10m\_10km.tif"}\NormalTok{)}

\CommentTok{\# simple landscape {-}{-}{-}{-}}
\NormalTok{simple\_landscape}\OtherTok{=}\FunctionTok{rast}\NormalTok{(}\StringTok{"./RasterGrids\_10m/2024/Ainava\_vienk\_mask.tif"}\NormalTok{)}

\CommentTok{\# Edges\_Trees{-}Builtup\_input.tif {-}{-}{-}{-}}
\NormalTok{trees\_from630}\OtherTok{=}\FunctionTok{ifel}\NormalTok{(simple\_landscape}\SpecialCharTok{\textgreater{}=}\DecValTok{630} \SpecialCharTok{\&}\NormalTok{ simple\_landscape}\SpecialCharTok{\textless{}}\DecValTok{700}\NormalTok{,}\DecValTok{1}\NormalTok{,}\ConstantTok{NA}\NormalTok{)}
\FunctionTok{plot}\NormalTok{(trees\_from630)}

\NormalTok{builtup}\OtherTok{=}\FunctionTok{ifel}\NormalTok{(simple\_landscape}\SpecialCharTok{==}\DecValTok{500}\NormalTok{,}\DecValTok{0}\NormalTok{,}\ConstantTok{NA}\NormalTok{)}
\FunctionTok{plot}\NormalTok{(builtup)}

\NormalTok{trees630\_builtup}\OtherTok{=}\FunctionTok{cover}\NormalTok{(trees\_from630,builtup)}
\FunctionTok{plot}\NormalTok{(trees630\_builtup)}

\NormalTok{edge\_trees630\_builtup}\OtherTok{=}\FunctionTok{project}\NormalTok{(trees630\_builtup,template10,}
               \AttributeTok{filename=}\StringTok{"./RasterGrids\_10m/2024/Edges\_Trees{-}Builtup\_input.tif"}\NormalTok{,}
               \AttributeTok{overwrite=}\ConstantTok{TRUE}\NormalTok{)}
\FunctionTok{rm}\NormalTok{(edge\_trees630\_builtup)}
\FunctionTok{rm}\NormalTok{(trees630\_builtup)}


\CommentTok{\# Edges\_Trees{-}Builtup\_cell.tif  egv\_125 {-}{-}{-}{-}}
\FunctionTok{landscape\_function}\NormalTok{(}
 \AttributeTok{landscape   =} \StringTok{"./RasterGrids\_10m/2024/Edges\_Trees{-}Builtup\_input.tif"}\NormalTok{,}
 \AttributeTok{zones     =} \StringTok{"./Templates/TemplateGrids/tikls100\_sauzeme.parquet"}\NormalTok{,}
 \AttributeTok{id\_field    =} \StringTok{"id"}\NormalTok{,}
 \AttributeTok{tile\_field   =} \StringTok{"tks50km"}\NormalTok{,}
 \AttributeTok{template    =} \StringTok{"./Templates/TemplateRasters/LV100m\_10km.tif"}\NormalTok{,}
 \AttributeTok{out\_dir    =} \StringTok{"./RasterGrids\_100m/2024/RAW"}\NormalTok{,}
 \AttributeTok{out\_filename  =} \StringTok{"Edges\_Trees{-}Builtup\_cell.tif"}\NormalTok{,}
 \AttributeTok{out\_layername =} \StringTok{"egv\_125"}\NormalTok{,}
 \AttributeTok{what       =} \StringTok{"lsm\_l\_te"}\NormalTok{,}
 \AttributeTok{lm\_args     =} \FunctionTok{list}\NormalTok{(}\AttributeTok{count\_boundary =} \ConstantTok{FALSE}\NormalTok{),}
 \AttributeTok{rasterize\_engine =} \StringTok{"fasterize"}\NormalTok{,}
 \AttributeTok{n\_workers   =} \DecValTok{12}\NormalTok{,}
 \AttributeTok{future\_max\_size =} \DecValTok{20} \SpecialCharTok{*} \DecValTok{1024}\SpecialCharTok{\^{}}\DecValTok{3}\NormalTok{,}
 \AttributeTok{fill\_gaps   =} \ConstantTok{TRUE}\NormalTok{,}
 \AttributeTok{plot\_gaps   =} \ConstantTok{FALSE}\NormalTok{,}
 \AttributeTok{plot\_result  =} \ConstantTok{FALSE}
\NormalTok{)}

\CommentTok{\# standardisation {-}{-}{-}{-}}
\ControlFlowTok{if}\NormalTok{(}\SpecialCharTok{!}\FunctionTok{require}\NormalTok{(terra)) \{}\FunctionTok{install.packages}\NormalTok{(}\StringTok{"terra"}\NormalTok{); }\FunctionTok{require}\NormalTok{(terra)\}}
\ControlFlowTok{if}\NormalTok{(}\SpecialCharTok{!}\FunctionTok{require}\NormalTok{(tidyverse)) \{}\FunctionTok{install.packages}\NormalTok{(}\StringTok{"tidyverse"}\NormalTok{); }\FunctionTok{require}\NormalTok{(tidyverse)\}}

\NormalTok{nosaukums}\OtherTok{=}\StringTok{"Edges\_Trees{-}Builtup\_cell.tif"}
\NormalTok{ielasisanas\_cels}\OtherTok{=}\FunctionTok{paste0}\NormalTok{(}\StringTok{"./RasterGrids\_100m/2024/RAW/"}\NormalTok{,nosaukums)}
\NormalTok{saglabasanas\_cels}\OtherTok{=}\FunctionTok{paste0}\NormalTok{(}\StringTok{"./RasterGrids\_100m/2024/Scaled/"}\NormalTok{,nosaukums)}
\NormalTok{slanis}\OtherTok{=}\FunctionTok{rast}\NormalTok{(ielasisanas\_cels)}
\NormalTok{videjais}\OtherTok{=}\FunctionTok{global}\NormalTok{(slanis,}\AttributeTok{fun=}\StringTok{"mean"}\NormalTok{,}\AttributeTok{na.rm=}\ConstantTok{TRUE}\NormalTok{)}
\NormalTok{centrets}\OtherTok{=}\NormalTok{slanis}\SpecialCharTok{{-}}\NormalTok{videjais[,}\DecValTok{1}\NormalTok{]}
\NormalTok{standartnovirze}\OtherTok{=}\NormalTok{terra}\SpecialCharTok{::}\FunctionTok{global}\NormalTok{(centrets,}\AttributeTok{fun=}\StringTok{"rms"}\NormalTok{,}\AttributeTok{na.rm=}\ConstantTok{TRUE}\NormalTok{)}
\NormalTok{merogots}\OtherTok{=}\NormalTok{centrets}\SpecialCharTok{/}\NormalTok{standartnovirze[,}\DecValTok{1}\NormalTok{]}
\FunctionTok{writeRaster}\NormalTok{(merogots,}
      \AttributeTok{filename=}\NormalTok{saglabasanas\_cels,}
      \AttributeTok{overwrite=}\ConstantTok{TRUE}\NormalTok{)}
\end{Highlighting}
\end{Shaded}

\section{Edges\_Trees-Builtup\_r500}\label{ch06.126}

\textbf{filename:} \texttt{Edges\_Trees-Builtup\_r500.tif}

\textbf{layername:} \texttt{egv\_126}

\textbf{English name:} Edge pixels of Trees bordering with Built-Up areas within the
0.5 km landscape

\textbf{Latvian name:} Koku malu ar apbūvi pikseļu skaits 0,5 km ainavā

\textbf{Procedure:} The total edge within a 500 m radius around the analysis grid cell is
calculated as the area-weighted sum of the \hyperref[ch06.125]{analysis cells} inside the
buffer, using the workflow \texttt{egvtools::radius\_function()}. During the calculation of the landscape metric,
inverse distance weighted (power = 2) gap filling on the output is applied
to ensure no missing values at the edges. Then the layer is rewritten to set
its name. Finally, the layer is standardised by subtracting the arithmetic
mean and dividing by the root mean squared error.

\begin{Shaded}
\begin{Highlighting}[]
\CommentTok{\# libs {-}{-}{-}{-}}
\ControlFlowTok{if}\NormalTok{(}\SpecialCharTok{!}\FunctionTok{require}\NormalTok{(terra)) \{}\FunctionTok{install.packages}\NormalTok{(}\StringTok{"terra"}\NormalTok{); }\FunctionTok{require}\NormalTok{(terra)\}}
\ControlFlowTok{if}\NormalTok{(}\SpecialCharTok{!}\FunctionTok{require}\NormalTok{(egvtools)) \{remotes}\SpecialCharTok{::}\FunctionTok{install\_github}\NormalTok{(}\StringTok{"aavotins/egvtools"}\NormalTok{); }\FunctionTok{require}\NormalTok{(egvtools)\}}


\CommentTok{\# Templates {-}{-}{-}{-}{-}}
\NormalTok{template100}\OtherTok{=}\FunctionTok{rast}\NormalTok{(}\StringTok{"./Templates/TemplateRasters/LV100m\_10km.tif"}\NormalTok{)}

\CommentTok{\# radii {-}{-}{-}{-}}
\FunctionTok{radius\_function}\NormalTok{(}
 \AttributeTok{kvadrati\_path =} \StringTok{"./Templates/TemplateGrids/tiles/"}\NormalTok{,}
 \AttributeTok{radii\_path   =} \StringTok{"./Templates/TemplateGridPoints/tiles/"}\NormalTok{,}
 \AttributeTok{tikls100\_path =} \StringTok{"./Templates/TemplateGrids/tikls100\_sauzeme.parquet"}\NormalTok{,}
 \AttributeTok{template\_path =} \StringTok{"./Templates/TemplateRasters/LV100m\_10km.tif"}\NormalTok{,}
 \AttributeTok{input\_layers  =} \FunctionTok{c}\NormalTok{(}\StringTok{"./RasterGrids\_100m/2024/RAW/Edges\_Trees{-}Builtup\_cell.tif"}\NormalTok{),}
 \AttributeTok{layer\_prefixes =} \FunctionTok{c}\NormalTok{(}\StringTok{"Edges\_Trees{-}Builtup"}\NormalTok{),}
 \AttributeTok{output\_dir   =} \StringTok{"./RasterGrids\_100m/2024/RAW/"}\NormalTok{,}
 \AttributeTok{n\_workers   =} \DecValTok{12}\NormalTok{,}
 \AttributeTok{radii     =} \FunctionTok{c}\NormalTok{(}\StringTok{"r500"}\NormalTok{),}
 \AttributeTok{radius\_mode  =} \StringTok{"sparse"}\NormalTok{,}
 \AttributeTok{extract\_fun  =} \StringTok{"sum"}\NormalTok{,}
 \AttributeTok{fill\_missing  =} \ConstantTok{TRUE}\NormalTok{,}
 \AttributeTok{IDW\_weight   =} \DecValTok{2}\NormalTok{,}
 \AttributeTok{future\_max\_size =} \DecValTok{20} \SpecialCharTok{*} \DecValTok{1024}\SpecialCharTok{\^{}}\DecValTok{3}\NormalTok{)}


\CommentTok{\# Edges\_Trees{-}Builtup\_r500.tif  egv\_126 {-}{-}{-}{-}}
\NormalTok{slanis}\OtherTok{=}\FunctionTok{rast}\NormalTok{(}\StringTok{"./RasterGrids\_100m/2024/RAW/Edges\_Trees{-}Builtup\_r500.tif"}\NormalTok{)}
\FunctionTok{names}\NormalTok{(slanis)}\OtherTok{=}\StringTok{"egv\_126"}
\NormalTok{slanis2}\OtherTok{=}\FunctionTok{project}\NormalTok{(slanis,template100)}
\FunctionTok{writeRaster}\NormalTok{(slanis2,}
      \StringTok{"./RasterGrids\_100m/2024/RAW/Edges\_Trees{-}Builtup\_r500.tif"}\NormalTok{,}
      \AttributeTok{overwrite=}\ConstantTok{TRUE}\NormalTok{)}

\CommentTok{\# standardisation {-}{-}{-}{-}}
\ControlFlowTok{if}\NormalTok{(}\SpecialCharTok{!}\FunctionTok{require}\NormalTok{(terra)) \{}\FunctionTok{install.packages}\NormalTok{(}\StringTok{"terra"}\NormalTok{); }\FunctionTok{require}\NormalTok{(terra)\}}
\ControlFlowTok{if}\NormalTok{(}\SpecialCharTok{!}\FunctionTok{require}\NormalTok{(tidyverse)) \{}\FunctionTok{install.packages}\NormalTok{(}\StringTok{"tidyverse"}\NormalTok{); }\FunctionTok{require}\NormalTok{(tidyverse)\}}

\NormalTok{nosaukums}\OtherTok{=}\StringTok{"Edges\_Trees{-}Builtup\_r500.tif"}
\NormalTok{ielasisanas\_cels}\OtherTok{=}\FunctionTok{paste0}\NormalTok{(}\StringTok{"./RasterGrids\_100m/2024/RAW/"}\NormalTok{,nosaukums)}
\NormalTok{saglabasanas\_cels}\OtherTok{=}\FunctionTok{paste0}\NormalTok{(}\StringTok{"./RasterGrids\_100m/2024/Scaled/"}\NormalTok{,nosaukums)}
\NormalTok{slanis}\OtherTok{=}\FunctionTok{rast}\NormalTok{(ielasisanas\_cels)}
\NormalTok{videjais}\OtherTok{=}\FunctionTok{global}\NormalTok{(slanis,}\AttributeTok{fun=}\StringTok{"mean"}\NormalTok{,}\AttributeTok{na.rm=}\ConstantTok{TRUE}\NormalTok{)}
\NormalTok{centrets}\OtherTok{=}\NormalTok{slanis}\SpecialCharTok{{-}}\NormalTok{videjais[,}\DecValTok{1}\NormalTok{]}
\NormalTok{standartnovirze}\OtherTok{=}\NormalTok{terra}\SpecialCharTok{::}\FunctionTok{global}\NormalTok{(centrets,}\AttributeTok{fun=}\StringTok{"rms"}\NormalTok{,}\AttributeTok{na.rm=}\ConstantTok{TRUE}\NormalTok{)}
\NormalTok{merogots}\OtherTok{=}\NormalTok{centrets}\SpecialCharTok{/}\NormalTok{standartnovirze[,}\DecValTok{1}\NormalTok{]}
\FunctionTok{writeRaster}\NormalTok{(merogots,}
      \AttributeTok{filename=}\NormalTok{saglabasanas\_cels,}
      \AttributeTok{overwrite=}\ConstantTok{TRUE}\NormalTok{)}
\end{Highlighting}
\end{Shaded}

\section{Edges\_Trees-Builtup\_r1250}\label{ch06.127}

\textbf{filename:} \texttt{Edges\_Trees-Builtup\_r1250.tif}

\textbf{layername:} \texttt{egv\_127}

\textbf{English name:} Edge pixels of Trees bordering with Built-Up areas within the
1.25 km landscape

\textbf{Latvian name:} Koku malu ar apbūvi pikseļu skaits 1,25 km ainavā

\textbf{Procedure:} The total edge within a 1250 m radius around the analysis grid cell is
calculated as the area-weighted sum of the \hyperref[ch06.125]{analysis cells} inside the
buffer, using the workflow \texttt{egvtools::radius\_function()}. During the calculation of the landscape metric,
inverse distance weighted (power = 2) gap filling on the output is applied
to ensure no missing values at the edges. Then the layer is rewritten to set
its name. Finally, the layer is standardised by subtracting the arithmetic
mean and dividing by the root mean squared error.

\begin{Shaded}
\begin{Highlighting}[]
\CommentTok{\# libs {-}{-}{-}{-}}
\ControlFlowTok{if}\NormalTok{(}\SpecialCharTok{!}\FunctionTok{require}\NormalTok{(terra)) \{}\FunctionTok{install.packages}\NormalTok{(}\StringTok{"terra"}\NormalTok{); }\FunctionTok{require}\NormalTok{(terra)\}}
\ControlFlowTok{if}\NormalTok{(}\SpecialCharTok{!}\FunctionTok{require}\NormalTok{(egvtools)) \{remotes}\SpecialCharTok{::}\FunctionTok{install\_github}\NormalTok{(}\StringTok{"aavotins/egvtools"}\NormalTok{); }\FunctionTok{require}\NormalTok{(egvtools)\}}


\CommentTok{\# Templates {-}{-}{-}{-}{-}}
\NormalTok{template100}\OtherTok{=}\FunctionTok{rast}\NormalTok{(}\StringTok{"./Templates/TemplateRasters/LV100m\_10km.tif"}\NormalTok{)}

\CommentTok{\# radii {-}{-}{-}{-}}
\FunctionTok{radius\_function}\NormalTok{(}
 \AttributeTok{kvadrati\_path =} \StringTok{"./Templates/TemplateGrids/tiles/"}\NormalTok{,}
 \AttributeTok{radii\_path   =} \StringTok{"./Templates/TemplateGridPoints/tiles/"}\NormalTok{,}
 \AttributeTok{tikls100\_path =} \StringTok{"./Templates/TemplateGrids/tikls100\_sauzeme.parquet"}\NormalTok{,}
 \AttributeTok{template\_path =} \StringTok{"./Templates/TemplateRasters/LV100m\_10km.tif"}\NormalTok{,}
 \AttributeTok{input\_layers  =} \FunctionTok{c}\NormalTok{(}\StringTok{"./RasterGrids\_100m/2024/RAW/Edges\_Trees{-}Builtup\_cell.tif"}\NormalTok{),}
 \AttributeTok{layer\_prefixes =} \FunctionTok{c}\NormalTok{(}\StringTok{"Edges\_Trees{-}Builtup"}\NormalTok{),}
 \AttributeTok{output\_dir   =} \StringTok{"./RasterGrids\_100m/2024/RAW/"}\NormalTok{,}
 \AttributeTok{n\_workers   =} \DecValTok{12}\NormalTok{,}
 \AttributeTok{radii     =} \FunctionTok{c}\NormalTok{(}\StringTok{"r1250"}\NormalTok{),}
 \AttributeTok{radius\_mode  =} \StringTok{"sparse"}\NormalTok{,}
 \AttributeTok{extract\_fun  =} \StringTok{"sum"}\NormalTok{,}
 \AttributeTok{fill\_missing  =} \ConstantTok{TRUE}\NormalTok{,}
 \AttributeTok{IDW\_weight   =} \DecValTok{2}\NormalTok{,}
 \AttributeTok{future\_max\_size =} \DecValTok{20} \SpecialCharTok{*} \DecValTok{1024}\SpecialCharTok{\^{}}\DecValTok{3}\NormalTok{)}


\CommentTok{\# Edges\_Trees{-}Builtup\_r1250.tif egv\_127 {-}{-}{-}{-}}
\NormalTok{slanis}\OtherTok{=}\FunctionTok{rast}\NormalTok{(}\StringTok{"./RasterGrids\_100m/2024/RAW/Edges\_Trees{-}Builtup\_r1250.tif"}\NormalTok{)}
\FunctionTok{names}\NormalTok{(slanis)}\OtherTok{=}\StringTok{"egv\_127"}
\NormalTok{slanis2}\OtherTok{=}\FunctionTok{project}\NormalTok{(slanis,template100)}
\FunctionTok{writeRaster}\NormalTok{(slanis2,}
      \StringTok{"./RasterGrids\_100m/2024/RAW/Edges\_Trees{-}Builtup\_r1250.tif"}\NormalTok{,}
      \AttributeTok{overwrite=}\ConstantTok{TRUE}\NormalTok{)}

\CommentTok{\# standardisation {-}{-}{-}{-}}
\ControlFlowTok{if}\NormalTok{(}\SpecialCharTok{!}\FunctionTok{require}\NormalTok{(terra)) \{}\FunctionTok{install.packages}\NormalTok{(}\StringTok{"terra"}\NormalTok{); }\FunctionTok{require}\NormalTok{(terra)\}}
\ControlFlowTok{if}\NormalTok{(}\SpecialCharTok{!}\FunctionTok{require}\NormalTok{(tidyverse)) \{}\FunctionTok{install.packages}\NormalTok{(}\StringTok{"tidyverse"}\NormalTok{); }\FunctionTok{require}\NormalTok{(tidyverse)\}}

\NormalTok{nosaukums}\OtherTok{=}\StringTok{"Edges\_Trees{-}Builtup\_r1250.tif"}
\NormalTok{ielasisanas\_cels}\OtherTok{=}\FunctionTok{paste0}\NormalTok{(}\StringTok{"./RasterGrids\_100m/2024/RAW/"}\NormalTok{,nosaukums)}
\NormalTok{saglabasanas\_cels}\OtherTok{=}\FunctionTok{paste0}\NormalTok{(}\StringTok{"./RasterGrids\_100m/2024/Scaled/"}\NormalTok{,nosaukums)}
\NormalTok{slanis}\OtherTok{=}\FunctionTok{rast}\NormalTok{(ielasisanas\_cels)}
\NormalTok{videjais}\OtherTok{=}\FunctionTok{global}\NormalTok{(slanis,}\AttributeTok{fun=}\StringTok{"mean"}\NormalTok{,}\AttributeTok{na.rm=}\ConstantTok{TRUE}\NormalTok{)}
\NormalTok{centrets}\OtherTok{=}\NormalTok{slanis}\SpecialCharTok{{-}}\NormalTok{videjais[,}\DecValTok{1}\NormalTok{]}
\NormalTok{standartnovirze}\OtherTok{=}\NormalTok{terra}\SpecialCharTok{::}\FunctionTok{global}\NormalTok{(centrets,}\AttributeTok{fun=}\StringTok{"rms"}\NormalTok{,}\AttributeTok{na.rm=}\ConstantTok{TRUE}\NormalTok{)}
\NormalTok{merogots}\OtherTok{=}\NormalTok{centrets}\SpecialCharTok{/}\NormalTok{standartnovirze[,}\DecValTok{1}\NormalTok{]}
\FunctionTok{writeRaster}\NormalTok{(merogots,}
      \AttributeTok{filename=}\NormalTok{saglabasanas\_cels,}
      \AttributeTok{overwrite=}\ConstantTok{TRUE}\NormalTok{)}
\end{Highlighting}
\end{Shaded}

\section{Edges\_Trees-Builtup\_r3000}\label{ch06.128}

\textbf{filename:} \texttt{Edges\_Trees-Builtup\_r3000.tif}

\textbf{layername:} \texttt{egv\_128}

\textbf{English name:} Edge pixels of Trees bordering with Built-Up areas within the
3 km landscape

\textbf{Latvian name:} Koku malu ar apbūvi pikseļu skaits 3 km ainavā

\textbf{Procedure:} The total edge within a 3000 m radius around the analysis grid cell is
calculated as the area-weighted sum of the \hyperref[ch06.125]{analysis cells} inside the
buffer, using the workflow \texttt{egvtools::radius\_function()}. During the calculation of the landscape metric,
inverse distance weighted (power = 2) gap filling on the output is applied
to ensure no missing values at the edges. Then the layer is rewritten to set
its name. Finally, the layer is standardised by subtracting the arithmetic
mean and dividing by the root mean squared error.

\begin{Shaded}
\begin{Highlighting}[]
\CommentTok{\# libs {-}{-}{-}{-}}
\ControlFlowTok{if}\NormalTok{(}\SpecialCharTok{!}\FunctionTok{require}\NormalTok{(terra)) \{}\FunctionTok{install.packages}\NormalTok{(}\StringTok{"terra"}\NormalTok{); }\FunctionTok{require}\NormalTok{(terra)\}}
\ControlFlowTok{if}\NormalTok{(}\SpecialCharTok{!}\FunctionTok{require}\NormalTok{(egvtools)) \{remotes}\SpecialCharTok{::}\FunctionTok{install\_github}\NormalTok{(}\StringTok{"aavotins/egvtools"}\NormalTok{); }\FunctionTok{require}\NormalTok{(egvtools)\}}


\CommentTok{\# Templates {-}{-}{-}{-}{-}}
\NormalTok{template100}\OtherTok{=}\FunctionTok{rast}\NormalTok{(}\StringTok{"./Templates/TemplateRasters/LV100m\_10km.tif"}\NormalTok{)}

\CommentTok{\# radii {-}{-}{-}{-}}
\FunctionTok{radius\_function}\NormalTok{(}
 \AttributeTok{kvadrati\_path =} \StringTok{"./Templates/TemplateGrids/tiles/"}\NormalTok{,}
 \AttributeTok{radii\_path   =} \StringTok{"./Templates/TemplateGridPoints/tiles/"}\NormalTok{,}
 \AttributeTok{tikls100\_path =} \StringTok{"./Templates/TemplateGrids/tikls100\_sauzeme.parquet"}\NormalTok{,}
 \AttributeTok{template\_path =} \StringTok{"./Templates/TemplateRasters/LV100m\_10km.tif"}\NormalTok{,}
 \AttributeTok{input\_layers  =} \FunctionTok{c}\NormalTok{(}\StringTok{"./RasterGrids\_100m/2024/RAW/Edges\_Trees{-}Builtup\_cell.tif"}\NormalTok{),}
 \AttributeTok{layer\_prefixes =} \FunctionTok{c}\NormalTok{(}\StringTok{"Edges\_Trees{-}Builtup"}\NormalTok{),}
 \AttributeTok{output\_dir   =} \StringTok{"./RasterGrids\_100m/2024/RAW/"}\NormalTok{,}
 \AttributeTok{n\_workers   =} \DecValTok{12}\NormalTok{,}
 \AttributeTok{radii     =} \FunctionTok{c}\NormalTok{(}\StringTok{"r3000"}\NormalTok{),}
 \AttributeTok{radius\_mode  =} \StringTok{"sparse"}\NormalTok{,}
 \AttributeTok{extract\_fun  =} \StringTok{"sum"}\NormalTok{,}
 \AttributeTok{fill\_missing  =} \ConstantTok{TRUE}\NormalTok{,}
 \AttributeTok{IDW\_weight   =} \DecValTok{2}\NormalTok{,}
 \AttributeTok{future\_max\_size =} \DecValTok{20} \SpecialCharTok{*} \DecValTok{1024}\SpecialCharTok{\^{}}\DecValTok{3}\NormalTok{)}


\CommentTok{\# Edges\_Trees{-}Builtup\_r3000.tif egv\_128 {-}{-}{-}{-}}
\NormalTok{slanis}\OtherTok{=}\FunctionTok{rast}\NormalTok{(}\StringTok{"./RasterGrids\_100m/2024/RAW/Edges\_Trees{-}Builtup\_r3000.tif"}\NormalTok{)}
\FunctionTok{names}\NormalTok{(slanis)}\OtherTok{=}\StringTok{"egv\_128"}
\NormalTok{slanis2}\OtherTok{=}\FunctionTok{project}\NormalTok{(slanis,template100)}
\FunctionTok{writeRaster}\NormalTok{(slanis2,}
      \StringTok{"./RasterGrids\_100m/2024/RAW/Edges\_Trees{-}Builtup\_r3000.tif"}\NormalTok{,}
      \AttributeTok{overwrite=}\ConstantTok{TRUE}\NormalTok{)}

\CommentTok{\# standardisation {-}{-}{-}{-}}
\ControlFlowTok{if}\NormalTok{(}\SpecialCharTok{!}\FunctionTok{require}\NormalTok{(terra)) \{}\FunctionTok{install.packages}\NormalTok{(}\StringTok{"terra"}\NormalTok{); }\FunctionTok{require}\NormalTok{(terra)\}}
\ControlFlowTok{if}\NormalTok{(}\SpecialCharTok{!}\FunctionTok{require}\NormalTok{(tidyverse)) \{}\FunctionTok{install.packages}\NormalTok{(}\StringTok{"tidyverse"}\NormalTok{); }\FunctionTok{require}\NormalTok{(tidyverse)\}}

\NormalTok{nosaukums}\OtherTok{=}\StringTok{"Edges\_Trees{-}Builtup\_r3000.tif"}
\NormalTok{ielasisanas\_cels}\OtherTok{=}\FunctionTok{paste0}\NormalTok{(}\StringTok{"./RasterGrids\_100m/2024/RAW/"}\NormalTok{,nosaukums)}
\NormalTok{saglabasanas\_cels}\OtherTok{=}\FunctionTok{paste0}\NormalTok{(}\StringTok{"./RasterGrids\_100m/2024/Scaled/"}\NormalTok{,nosaukums)}
\NormalTok{slanis}\OtherTok{=}\FunctionTok{rast}\NormalTok{(ielasisanas\_cels)}
\NormalTok{videjais}\OtherTok{=}\FunctionTok{global}\NormalTok{(slanis,}\AttributeTok{fun=}\StringTok{"mean"}\NormalTok{,}\AttributeTok{na.rm=}\ConstantTok{TRUE}\NormalTok{)}
\NormalTok{centrets}\OtherTok{=}\NormalTok{slanis}\SpecialCharTok{{-}}\NormalTok{videjais[,}\DecValTok{1}\NormalTok{]}
\NormalTok{standartnovirze}\OtherTok{=}\NormalTok{terra}\SpecialCharTok{::}\FunctionTok{global}\NormalTok{(centrets,}\AttributeTok{fun=}\StringTok{"rms"}\NormalTok{,}\AttributeTok{na.rm=}\ConstantTok{TRUE}\NormalTok{)}
\NormalTok{merogots}\OtherTok{=}\NormalTok{centrets}\SpecialCharTok{/}\NormalTok{standartnovirze[,}\DecValTok{1}\NormalTok{]}
\FunctionTok{writeRaster}\NormalTok{(merogots,}
      \AttributeTok{filename=}\NormalTok{saglabasanas\_cels,}
      \AttributeTok{overwrite=}\ConstantTok{TRUE}\NormalTok{)}
\end{Highlighting}
\end{Shaded}

\section{Edges\_Trees-Builtup\_r10000}\label{ch06.129}

\textbf{filename:} \texttt{Edges\_Trees-Builtup\_r10000.tif}

\textbf{layername:} \texttt{egv\_129}

\textbf{English name:} Edge pixels of Trees bordering with Built-Up areas within the
10 km landscape

\textbf{Latvian name:} Koku malu ar apbūvi pikseļu skaits 10 km ainavā

\textbf{Procedure:} The total edge within a 10000 m radius around the analysis grid cell is
calculated as the area-weighted sum of the \hyperref[ch06.125]{analysis cells} inside the
buffer, using the workflow \texttt{egvtools::radius\_function()}. During the calculation of the landscape metric,
inverse distance weighted (power = 2) gap filling on the output is applied
to ensure no missing values at the edges. Then the layer is rewritten to set
its name. Finally, the layer is standardised by subtracting the arithmetic
mean and dividing by the root mean squared error.

\begin{Shaded}
\begin{Highlighting}[]
\CommentTok{\# libs {-}{-}{-}{-}}
\ControlFlowTok{if}\NormalTok{(}\SpecialCharTok{!}\FunctionTok{require}\NormalTok{(terra)) \{}\FunctionTok{install.packages}\NormalTok{(}\StringTok{"terra"}\NormalTok{); }\FunctionTok{require}\NormalTok{(terra)\}}
\ControlFlowTok{if}\NormalTok{(}\SpecialCharTok{!}\FunctionTok{require}\NormalTok{(egvtools)) \{remotes}\SpecialCharTok{::}\FunctionTok{install\_github}\NormalTok{(}\StringTok{"aavotins/egvtools"}\NormalTok{); }\FunctionTok{require}\NormalTok{(egvtools)\}}


\CommentTok{\# Templates {-}{-}{-}{-}{-}}
\NormalTok{template100}\OtherTok{=}\FunctionTok{rast}\NormalTok{(}\StringTok{"./Templates/TemplateRasters/LV100m\_10km.tif"}\NormalTok{)}

\CommentTok{\# radii {-}{-}{-}{-}}
\FunctionTok{radius\_function}\NormalTok{(}
 \AttributeTok{kvadrati\_path =} \StringTok{"./Templates/TemplateGrids/tiles/"}\NormalTok{,}
 \AttributeTok{radii\_path   =} \StringTok{"./Templates/TemplateGridPoints/tiles/"}\NormalTok{,}
 \AttributeTok{tikls100\_path =} \StringTok{"./Templates/TemplateGrids/tikls100\_sauzeme.parquet"}\NormalTok{,}
 \AttributeTok{template\_path =} \StringTok{"./Templates/TemplateRasters/LV100m\_10km.tif"}\NormalTok{,}
 \AttributeTok{input\_layers  =} \FunctionTok{c}\NormalTok{(}\StringTok{"./RasterGrids\_100m/2024/RAW/Edges\_Trees{-}Builtup\_cell.tif"}\NormalTok{),}
 \AttributeTok{layer\_prefixes =} \FunctionTok{c}\NormalTok{(}\StringTok{"Edges\_Trees{-}Builtup"}\NormalTok{),}
 \AttributeTok{output\_dir   =} \StringTok{"./RasterGrids\_100m/2024/RAW/"}\NormalTok{,}
 \AttributeTok{n\_workers   =} \DecValTok{12}\NormalTok{,}
 \AttributeTok{radii     =} \FunctionTok{c}\NormalTok{(}\StringTok{"r10000"}\NormalTok{),}
 \AttributeTok{radius\_mode  =} \StringTok{"sparse"}\NormalTok{,}
 \AttributeTok{extract\_fun  =} \StringTok{"sum"}\NormalTok{,}
 \AttributeTok{fill\_missing  =} \ConstantTok{TRUE}\NormalTok{,}
 \AttributeTok{IDW\_weight   =} \DecValTok{2}\NormalTok{,}
 \AttributeTok{future\_max\_size =} \DecValTok{20} \SpecialCharTok{*} \DecValTok{1024}\SpecialCharTok{\^{}}\DecValTok{3}\NormalTok{)}


\CommentTok{\# Edges\_Trees{-}Builtup\_r10000.tif    egv\_129 {-}{-}{-}{-}}
\NormalTok{slanis}\OtherTok{=}\FunctionTok{rast}\NormalTok{(}\StringTok{"./RasterGrids\_100m/2024/RAW/Edges\_Trees{-}Builtup\_r10000.tif"}\NormalTok{)}
\FunctionTok{names}\NormalTok{(slanis)}\OtherTok{=}\StringTok{"egv\_129"}
\NormalTok{slanis2}\OtherTok{=}\FunctionTok{project}\NormalTok{(slanis,template100)}
\FunctionTok{writeRaster}\NormalTok{(slanis2,}
      \StringTok{"./RasterGrids\_100m/2024/RAW/Edges\_Trees{-}Builtup\_r10000.tif"}\NormalTok{,}
      \AttributeTok{overwrite=}\ConstantTok{TRUE}\NormalTok{)}

\CommentTok{\# standardisation {-}{-}{-}{-}}
\ControlFlowTok{if}\NormalTok{(}\SpecialCharTok{!}\FunctionTok{require}\NormalTok{(terra)) \{}\FunctionTok{install.packages}\NormalTok{(}\StringTok{"terra"}\NormalTok{); }\FunctionTok{require}\NormalTok{(terra)\}}
\ControlFlowTok{if}\NormalTok{(}\SpecialCharTok{!}\FunctionTok{require}\NormalTok{(tidyverse)) \{}\FunctionTok{install.packages}\NormalTok{(}\StringTok{"tidyverse"}\NormalTok{); }\FunctionTok{require}\NormalTok{(tidyverse)\}}

\NormalTok{nosaukums}\OtherTok{=}\StringTok{"Edges\_Trees{-}Builtup\_r10000.tif"}
\NormalTok{ielasisanas\_cels}\OtherTok{=}\FunctionTok{paste0}\NormalTok{(}\StringTok{"./RasterGrids\_100m/2024/RAW/"}\NormalTok{,nosaukums)}
\NormalTok{saglabasanas\_cels}\OtherTok{=}\FunctionTok{paste0}\NormalTok{(}\StringTok{"./RasterGrids\_100m/2024/Scaled/"}\NormalTok{,nosaukums)}
\NormalTok{slanis}\OtherTok{=}\FunctionTok{rast}\NormalTok{(ielasisanas\_cels)}
\NormalTok{videjais}\OtherTok{=}\FunctionTok{global}\NormalTok{(slanis,}\AttributeTok{fun=}\StringTok{"mean"}\NormalTok{,}\AttributeTok{na.rm=}\ConstantTok{TRUE}\NormalTok{)}
\NormalTok{centrets}\OtherTok{=}\NormalTok{slanis}\SpecialCharTok{{-}}\NormalTok{videjais[,}\DecValTok{1}\NormalTok{]}
\NormalTok{standartnovirze}\OtherTok{=}\NormalTok{terra}\SpecialCharTok{::}\FunctionTok{global}\NormalTok{(centrets,}\AttributeTok{fun=}\StringTok{"rms"}\NormalTok{,}\AttributeTok{na.rm=}\ConstantTok{TRUE}\NormalTok{)}
\NormalTok{merogots}\OtherTok{=}\NormalTok{centrets}\SpecialCharTok{/}\NormalTok{standartnovirze[,}\DecValTok{1}\NormalTok{]}
\FunctionTok{writeRaster}\NormalTok{(merogots,}
      \AttributeTok{filename=}\NormalTok{saglabasanas\_cels,}
      \AttributeTok{overwrite=}\ConstantTok{TRUE}\NormalTok{)}
\end{Highlighting}
\end{Shaded}

\section{Edges\_CropsFallow\_cell}\label{ch06.130}

\textbf{filename:} \texttt{Edges\_CropsFallow\_cell.tif}

\textbf{layername:} \texttt{egv\_130}

\textbf{English name:} Edge pixels of Cropland, Fallow land within the analysis cell
(1 ha)

\textbf{Latvian name:} Aramzemju malu pikseļu skaits analīzes šūnā (1 ha)

\textbf{Procedure:} First, values larger than or equal to 310 and smaller than 325
from the \hyperref[Ch05.03]{Landscape classification} are coded as 1, and all other values as
NA. Then, the layer (1 = presence) is covered over the nulls layer (presence = 0)
and written to file (matching the input). Next, using the workflow
\texttt{egvtools::landscape\_function()} total edge between the two classes is
calculated. During the calculation of the landscape metric, inverse distance weighted
(power = 2) gap filling on the output is applied to ensure no missing values
at the edges. Finally, the layer is standardised by subtracting the arithmetic
mean and dividing by the root mean squared error.

\begin{Shaded}
\begin{Highlighting}[]
\CommentTok{\# libs {-}{-}{-}{-}}
\ControlFlowTok{if}\NormalTok{(}\SpecialCharTok{!}\FunctionTok{require}\NormalTok{(terra)) \{}\FunctionTok{install.packages}\NormalTok{(}\StringTok{"terra"}\NormalTok{); }\FunctionTok{require}\NormalTok{(terra)\}}
\ControlFlowTok{if}\NormalTok{(}\SpecialCharTok{!}\FunctionTok{require}\NormalTok{(egvtools)) \{remotes}\SpecialCharTok{::}\FunctionTok{install\_github}\NormalTok{(}\StringTok{"aavotins/egvtools"}\NormalTok{); }\FunctionTok{require}\NormalTok{(egvtools)\}}

\ControlFlowTok{if}\NormalTok{(}\SpecialCharTok{!}\FunctionTok{require}\NormalTok{(sf)) \{}\FunctionTok{install.packages}\NormalTok{(}\StringTok{"sf"}\NormalTok{); }\FunctionTok{require}\NormalTok{(sf)\}}
\ControlFlowTok{if}\NormalTok{(}\SpecialCharTok{!}\FunctionTok{require}\NormalTok{(sfarrow)) \{}\FunctionTok{install.packages}\NormalTok{(}\StringTok{"sfarrow"}\NormalTok{); }\FunctionTok{require}\NormalTok{(sfarrow)\}}
\ControlFlowTok{if}\NormalTok{(}\SpecialCharTok{!}\FunctionTok{require}\NormalTok{(raster)) \{}\FunctionTok{install.packages}\NormalTok{(}\StringTok{"raster"}\NormalTok{); }\FunctionTok{require}\NormalTok{(raster)\}}
\ControlFlowTok{if}\NormalTok{(}\SpecialCharTok{!}\FunctionTok{require}\NormalTok{(fasterize)) \{}\FunctionTok{install.packages}\NormalTok{(}\StringTok{"fasterize"}\NormalTok{); }\FunctionTok{require}\NormalTok{(fasterize)\}}
\ControlFlowTok{if}\NormalTok{(}\SpecialCharTok{!}\FunctionTok{require}\NormalTok{(tidyverse)) \{}\FunctionTok{install.packages}\NormalTok{(}\StringTok{"tidyverse"}\NormalTok{); }\FunctionTok{require}\NormalTok{(tidyverse)\}}


\CommentTok{\# Templates {-}{-}{-}{-}{-}}
\NormalTok{template10}\OtherTok{=}\FunctionTok{rast}\NormalTok{(}\StringTok{"./Templates/TemplateRasters/LV10m\_10km.tif"}\NormalTok{)}
\NormalTok{nulls10}\OtherTok{=}\FunctionTok{rast}\NormalTok{(}\StringTok{"./Templates/TemplateRasters/nulls\_LV10m\_10km.tif"}\NormalTok{)}

\CommentTok{\# simple landscape {-}{-}{-}{-}}
\NormalTok{simple\_landscape}\OtherTok{=}\FunctionTok{rast}\NormalTok{(}\StringTok{"./RasterGrids\_10m/2024/Ainava\_vienk\_mask.tif"}\NormalTok{)}

\CommentTok{\# Edges\_CropsFallow\_input.tif {-}{-}{-}{-}}
\NormalTok{cropsfallow}\OtherTok{=}\FunctionTok{ifel}\NormalTok{(simple\_landscape}\SpecialCharTok{\textgreater{}=}\DecValTok{310} \SpecialCharTok{\&}\NormalTok{ simple\_landscape}\SpecialCharTok{\textless{}}\DecValTok{325}\NormalTok{,}\DecValTok{1}\NormalTok{,}\ConstantTok{NA}\NormalTok{)}
\FunctionTok{plot}\NormalTok{(cropsfallow)}
\NormalTok{cropsfallow}\OtherTok{=}\FunctionTok{cover}\NormalTok{(cropsfallow,nulls10)}
\FunctionTok{plot}\NormalTok{(cropsfallow)}

\NormalTok{edge\_cropsfallow}\OtherTok{=}\FunctionTok{project}\NormalTok{(cropsfallow,template10,}
            \AttributeTok{filename=}\StringTok{"./RasterGrids\_10m/2024/Edges\_CropsFallow\_input.tif"}\NormalTok{,}
            \AttributeTok{overwrite=}\ConstantTok{TRUE}\NormalTok{)}
\FunctionTok{rm}\NormalTok{(edge\_cropsfallow)}


\CommentTok{\# Edges\_CropsFallow\_cell.tif    egv\_130 {-}{-}{-}{-}}
\FunctionTok{landscape\_function}\NormalTok{(}
 \AttributeTok{landscape   =} \StringTok{"./RasterGrids\_10m/2024/Edges\_CropsFallow\_input.tif"}\NormalTok{,}
 \AttributeTok{zones     =} \StringTok{"./Templates/TemplateGrids/tikls100\_sauzeme.parquet"}\NormalTok{,}
 \AttributeTok{id\_field    =} \StringTok{"id"}\NormalTok{,}
 \AttributeTok{tile\_field   =} \StringTok{"tks50km"}\NormalTok{,}
 \AttributeTok{template    =} \StringTok{"./Templates/TemplateRasters/LV100m\_10km.tif"}\NormalTok{,}
 \AttributeTok{out\_dir    =} \StringTok{"./RasterGrids\_100m/2024/RAW"}\NormalTok{,}
 \AttributeTok{out\_filename  =} \StringTok{"Edges\_CropsFallow\_cell.tif"}\NormalTok{,}
 \AttributeTok{out\_layername =} \StringTok{"egv\_130"}\NormalTok{,}
 \AttributeTok{what       =} \StringTok{"lsm\_l\_te"}\NormalTok{,}
 \AttributeTok{lm\_args     =} \FunctionTok{list}\NormalTok{(}\AttributeTok{count\_boundary =} \ConstantTok{FALSE}\NormalTok{),}
 \AttributeTok{rasterize\_engine =} \StringTok{"fasterize"}\NormalTok{,}
 \AttributeTok{n\_workers   =} \DecValTok{12}\NormalTok{,}
 \AttributeTok{future\_max\_size =} \DecValTok{20} \SpecialCharTok{*} \DecValTok{1024}\SpecialCharTok{\^{}}\DecValTok{3}\NormalTok{,}
 \AttributeTok{fill\_gaps   =} \ConstantTok{TRUE}\NormalTok{,}
 \AttributeTok{plot\_gaps   =} \ConstantTok{FALSE}\NormalTok{,}
 \AttributeTok{plot\_result  =} \ConstantTok{FALSE}
\NormalTok{)}

\CommentTok{\# standardisation {-}{-}{-}{-}}
\ControlFlowTok{if}\NormalTok{(}\SpecialCharTok{!}\FunctionTok{require}\NormalTok{(terra)) \{}\FunctionTok{install.packages}\NormalTok{(}\StringTok{"terra"}\NormalTok{); }\FunctionTok{require}\NormalTok{(terra)\}}
\ControlFlowTok{if}\NormalTok{(}\SpecialCharTok{!}\FunctionTok{require}\NormalTok{(tidyverse)) \{}\FunctionTok{install.packages}\NormalTok{(}\StringTok{"tidyverse"}\NormalTok{); }\FunctionTok{require}\NormalTok{(tidyverse)\}}

\NormalTok{nosaukums}\OtherTok{=}\StringTok{"Edges\_CropsFallow\_cell.tif"}
\NormalTok{ielasisanas\_cels}\OtherTok{=}\FunctionTok{paste0}\NormalTok{(}\StringTok{"./RasterGrids\_100m/2024/RAW/"}\NormalTok{,nosaukums)}
\NormalTok{saglabasanas\_cels}\OtherTok{=}\FunctionTok{paste0}\NormalTok{(}\StringTok{"./RasterGrids\_100m/2024/Scaled/"}\NormalTok{,nosaukums)}
\NormalTok{slanis}\OtherTok{=}\FunctionTok{rast}\NormalTok{(ielasisanas\_cels)}
\NormalTok{videjais}\OtherTok{=}\FunctionTok{global}\NormalTok{(slanis,}\AttributeTok{fun=}\StringTok{"mean"}\NormalTok{,}\AttributeTok{na.rm=}\ConstantTok{TRUE}\NormalTok{)}
\NormalTok{centrets}\OtherTok{=}\NormalTok{slanis}\SpecialCharTok{{-}}\NormalTok{videjais[,}\DecValTok{1}\NormalTok{]}
\NormalTok{standartnovirze}\OtherTok{=}\NormalTok{terra}\SpecialCharTok{::}\FunctionTok{global}\NormalTok{(centrets,}\AttributeTok{fun=}\StringTok{"rms"}\NormalTok{,}\AttributeTok{na.rm=}\ConstantTok{TRUE}\NormalTok{)}
\NormalTok{merogots}\OtherTok{=}\NormalTok{centrets}\SpecialCharTok{/}\NormalTok{standartnovirze[,}\DecValTok{1}\NormalTok{]}
\FunctionTok{writeRaster}\NormalTok{(merogots,}
      \AttributeTok{filename=}\NormalTok{saglabasanas\_cels,}
      \AttributeTok{overwrite=}\ConstantTok{TRUE}\NormalTok{)}
\end{Highlighting}
\end{Shaded}

\section{Edges\_CropsFallow\_r500}\label{ch06.131}

\textbf{filename:} \texttt{Edges\_CropsFallow\_r500.tif}

\textbf{layername:} \texttt{egv\_131}

\textbf{English name:} Edge pixels of Cropland, Fallow land within the 0.5 km
landscape

\textbf{Latvian name:} Aramzemju malu pikseļu skaits 0,5 km ainavā

\textbf{Procedure:} The total edge within a 500 m radius around the analysis grid cell is
calculated as the area-weighted sum of the \hyperref[ch06.130]{analysis cells} inside the
buffer, using the workflow \texttt{egvtools::radius\_function()}. During the calculation of the landscape metric,
inverse distance weighted (power = 2) gap filling on the output is applied
to ensure no missing values at the edges. Then the layer is rewritten to set
its name. Finally, the layer is standardised by subtracting the arithmetic
mean and dividing by the root mean squared error.

\begin{Shaded}
\begin{Highlighting}[]
\CommentTok{\# libs {-}{-}{-}{-}}
\ControlFlowTok{if}\NormalTok{(}\SpecialCharTok{!}\FunctionTok{require}\NormalTok{(terra)) \{}\FunctionTok{install.packages}\NormalTok{(}\StringTok{"terra"}\NormalTok{); }\FunctionTok{require}\NormalTok{(terra)\}}
\ControlFlowTok{if}\NormalTok{(}\SpecialCharTok{!}\FunctionTok{require}\NormalTok{(egvtools)) \{remotes}\SpecialCharTok{::}\FunctionTok{install\_github}\NormalTok{(}\StringTok{"aavotins/egvtools"}\NormalTok{); }\FunctionTok{require}\NormalTok{(egvtools)\}}


\CommentTok{\# Templates {-}{-}{-}{-}{-}}
\NormalTok{template100}\OtherTok{=}\FunctionTok{rast}\NormalTok{(}\StringTok{"./Templates/TemplateRasters/LV100m\_10km.tif"}\NormalTok{)}

\CommentTok{\# radii {-}{-}{-}{-}}
\FunctionTok{radius\_function}\NormalTok{(}
 \AttributeTok{kvadrati\_path =} \StringTok{"./Templates/TemplateGrids/tiles/"}\NormalTok{,}
 \AttributeTok{radii\_path   =} \StringTok{"./Templates/TemplateGridPoints/tiles/"}\NormalTok{,}
 \AttributeTok{tikls100\_path =} \StringTok{"./Templates/TemplateGrids/tikls100\_sauzeme.parquet"}\NormalTok{,}
 \AttributeTok{template\_path =} \StringTok{"./Templates/TemplateRasters/LV100m\_10km.tif"}\NormalTok{,}
 \AttributeTok{input\_layers  =} \FunctionTok{c}\NormalTok{(}\StringTok{"./RasterGrids\_100m/2024/RAW/Edges\_CropsFallow\_cell.tif"}\NormalTok{),}
 \AttributeTok{layer\_prefixes =} \FunctionTok{c}\NormalTok{(}\StringTok{"Edges\_CropsFallow"}\NormalTok{),}
 \AttributeTok{output\_dir   =} \StringTok{"./RasterGrids\_100m/2024/RAW/"}\NormalTok{,}
 \AttributeTok{n\_workers   =} \DecValTok{12}\NormalTok{,}
 \AttributeTok{radii     =} \FunctionTok{c}\NormalTok{(}\StringTok{"r500"}\NormalTok{),}
 \AttributeTok{radius\_mode  =} \StringTok{"sparse"}\NormalTok{,}
 \AttributeTok{extract\_fun  =} \StringTok{"sum"}\NormalTok{,}
 \AttributeTok{fill\_missing  =} \ConstantTok{TRUE}\NormalTok{,}
 \AttributeTok{IDW\_weight   =} \DecValTok{2}\NormalTok{,}
 \AttributeTok{future\_max\_size =} \DecValTok{20} \SpecialCharTok{*} \DecValTok{1024}\SpecialCharTok{\^{}}\DecValTok{3}\NormalTok{)}


\CommentTok{\# Edges\_CropsFallow\_r500.tif    egv\_131 {-}{-}{-}{-}}
\NormalTok{slanis}\OtherTok{=}\FunctionTok{rast}\NormalTok{(}\StringTok{"./RasterGrids\_100m/2024/RAW/Edges\_CropsFallow\_r500.tif"}\NormalTok{)}
\FunctionTok{names}\NormalTok{(slanis)}\OtherTok{=}\StringTok{"egv\_131"}
\NormalTok{slanis2}\OtherTok{=}\FunctionTok{project}\NormalTok{(slanis,template100)}
\FunctionTok{writeRaster}\NormalTok{(slanis2,}
      \StringTok{"./RasterGrids\_100m/2024/RAW/Edges\_CropsFallow\_r500.tif"}\NormalTok{,}
      \AttributeTok{overwrite=}\ConstantTok{TRUE}\NormalTok{)}

\CommentTok{\# standardisation {-}{-}{-}{-}}
\ControlFlowTok{if}\NormalTok{(}\SpecialCharTok{!}\FunctionTok{require}\NormalTok{(terra)) \{}\FunctionTok{install.packages}\NormalTok{(}\StringTok{"terra"}\NormalTok{); }\FunctionTok{require}\NormalTok{(terra)\}}
\ControlFlowTok{if}\NormalTok{(}\SpecialCharTok{!}\FunctionTok{require}\NormalTok{(tidyverse)) \{}\FunctionTok{install.packages}\NormalTok{(}\StringTok{"tidyverse"}\NormalTok{); }\FunctionTok{require}\NormalTok{(tidyverse)\}}

\NormalTok{nosaukums}\OtherTok{=}\StringTok{"Edges\_CropsFallow\_r500.tif"}
\NormalTok{ielasisanas\_cels}\OtherTok{=}\FunctionTok{paste0}\NormalTok{(}\StringTok{"./RasterGrids\_100m/2024/RAW/"}\NormalTok{,nosaukums)}
\NormalTok{saglabasanas\_cels}\OtherTok{=}\FunctionTok{paste0}\NormalTok{(}\StringTok{"./RasterGrids\_100m/2024/Scaled/"}\NormalTok{,nosaukums)}
\NormalTok{slanis}\OtherTok{=}\FunctionTok{rast}\NormalTok{(ielasisanas\_cels)}
\NormalTok{videjais}\OtherTok{=}\FunctionTok{global}\NormalTok{(slanis,}\AttributeTok{fun=}\StringTok{"mean"}\NormalTok{,}\AttributeTok{na.rm=}\ConstantTok{TRUE}\NormalTok{)}
\NormalTok{centrets}\OtherTok{=}\NormalTok{slanis}\SpecialCharTok{{-}}\NormalTok{videjais[,}\DecValTok{1}\NormalTok{]}
\NormalTok{standartnovirze}\OtherTok{=}\NormalTok{terra}\SpecialCharTok{::}\FunctionTok{global}\NormalTok{(centrets,}\AttributeTok{fun=}\StringTok{"rms"}\NormalTok{,}\AttributeTok{na.rm=}\ConstantTok{TRUE}\NormalTok{)}
\NormalTok{merogots}\OtherTok{=}\NormalTok{centrets}\SpecialCharTok{/}\NormalTok{standartnovirze[,}\DecValTok{1}\NormalTok{]}
\FunctionTok{writeRaster}\NormalTok{(merogots,}
      \AttributeTok{filename=}\NormalTok{saglabasanas\_cels,}
      \AttributeTok{overwrite=}\ConstantTok{TRUE}\NormalTok{)}
\end{Highlighting}
\end{Shaded}

\section{Edges\_CropsFallow\_r1250}\label{ch06.132}

\textbf{filename:} \texttt{Edges\_CropsFallow\_r1250.tif}

\textbf{layername:} \texttt{egv\_132}

\textbf{English name:} Edge pixels of Cropland, Fallow land within the 1.25 km
landscape

\textbf{Latvian name:} Aramzemju malu pikseļu skaits 1,25 km ainavā

\textbf{Procedure:} The total edge within a 1250 m radius around the analysis grid cell is
calculated as the area-weighted sum of the \hyperref[ch06.130]{analysis cells} inside the
buffer, using the workflow \texttt{egvtools::radius\_function()}. During the calculation of the landscape metric,
inverse distance weighted (power = 2) gap filling on the output is applied
to ensure no missing values at the edges. Then the layer is rewritten to set
its name. Finally, the layer is standardised by subtracting the arithmetic
mean and dividing by the root mean squared error.

\begin{Shaded}
\begin{Highlighting}[]
\CommentTok{\# libs {-}{-}{-}{-}}
\ControlFlowTok{if}\NormalTok{(}\SpecialCharTok{!}\FunctionTok{require}\NormalTok{(terra)) \{}\FunctionTok{install.packages}\NormalTok{(}\StringTok{"terra"}\NormalTok{); }\FunctionTok{require}\NormalTok{(terra)\}}
\ControlFlowTok{if}\NormalTok{(}\SpecialCharTok{!}\FunctionTok{require}\NormalTok{(egvtools)) \{remotes}\SpecialCharTok{::}\FunctionTok{install\_github}\NormalTok{(}\StringTok{"aavotins/egvtools"}\NormalTok{); }\FunctionTok{require}\NormalTok{(egvtools)\}}


\CommentTok{\# Templates {-}{-}{-}{-}{-}}
\NormalTok{template100}\OtherTok{=}\FunctionTok{rast}\NormalTok{(}\StringTok{"./Templates/TemplateRasters/LV100m\_10km.tif"}\NormalTok{)}

\CommentTok{\# radii {-}{-}{-}{-}}
\FunctionTok{radius\_function}\NormalTok{(}
 \AttributeTok{kvadrati\_path =} \StringTok{"./Templates/TemplateGrids/tiles/"}\NormalTok{,}
 \AttributeTok{radii\_path   =} \StringTok{"./Templates/TemplateGridPoints/tiles/"}\NormalTok{,}
 \AttributeTok{tikls100\_path =} \StringTok{"./Templates/TemplateGrids/tikls100\_sauzeme.parquet"}\NormalTok{,}
 \AttributeTok{template\_path =} \StringTok{"./Templates/TemplateRasters/LV100m\_10km.tif"}\NormalTok{,}
 \AttributeTok{input\_layers  =} \FunctionTok{c}\NormalTok{(}\StringTok{"./RasterGrids\_100m/2024/RAW/Edges\_CropsFallow\_cell.tif"}\NormalTok{),}
 \AttributeTok{layer\_prefixes =} \FunctionTok{c}\NormalTok{(}\StringTok{"Edges\_CropsFallow"}\NormalTok{),}
 \AttributeTok{output\_dir   =} \StringTok{"./RasterGrids\_100m/2024/RAW/"}\NormalTok{,}
 \AttributeTok{n\_workers   =} \DecValTok{12}\NormalTok{,}
 \AttributeTok{radii     =} \FunctionTok{c}\NormalTok{(}\StringTok{"r1250"}\NormalTok{),}
 \AttributeTok{radius\_mode  =} \StringTok{"sparse"}\NormalTok{,}
 \AttributeTok{extract\_fun  =} \StringTok{"sum"}\NormalTok{,}
 \AttributeTok{fill\_missing  =} \ConstantTok{TRUE}\NormalTok{,}
 \AttributeTok{IDW\_weight   =} \DecValTok{2}\NormalTok{,}
 \AttributeTok{future\_max\_size =} \DecValTok{20} \SpecialCharTok{*} \DecValTok{1024}\SpecialCharTok{\^{}}\DecValTok{3}\NormalTok{)}


\CommentTok{\# Edges\_CropsFallow\_r1250.tif   egv\_132 {-}{-}{-}{-}}
\NormalTok{slanis}\OtherTok{=}\FunctionTok{rast}\NormalTok{(}\StringTok{"./RasterGrids\_100m/2024/RAW/Edges\_CropsFallow\_r1250.tif"}\NormalTok{)}
\FunctionTok{names}\NormalTok{(slanis)}\OtherTok{=}\StringTok{"egv\_132"}
\NormalTok{slanis2}\OtherTok{=}\FunctionTok{project}\NormalTok{(slanis,template100)}
\FunctionTok{writeRaster}\NormalTok{(slanis2,}
      \StringTok{"./RasterGrids\_100m/2024/RAW/Edges\_CropsFallow\_r1250.tif"}\NormalTok{,}
      \AttributeTok{overwrite=}\ConstantTok{TRUE}\NormalTok{)}

\CommentTok{\# standardisation {-}{-}{-}{-}}
\ControlFlowTok{if}\NormalTok{(}\SpecialCharTok{!}\FunctionTok{require}\NormalTok{(terra)) \{}\FunctionTok{install.packages}\NormalTok{(}\StringTok{"terra"}\NormalTok{); }\FunctionTok{require}\NormalTok{(terra)\}}
\ControlFlowTok{if}\NormalTok{(}\SpecialCharTok{!}\FunctionTok{require}\NormalTok{(tidyverse)) \{}\FunctionTok{install.packages}\NormalTok{(}\StringTok{"tidyverse"}\NormalTok{); }\FunctionTok{require}\NormalTok{(tidyverse)\}}

\NormalTok{nosaukums}\OtherTok{=}\StringTok{"Edges\_CropsFallow\_r1250.tif"}
\NormalTok{ielasisanas\_cels}\OtherTok{=}\FunctionTok{paste0}\NormalTok{(}\StringTok{"./RasterGrids\_100m/2024/RAW/"}\NormalTok{,nosaukums)}
\NormalTok{saglabasanas\_cels}\OtherTok{=}\FunctionTok{paste0}\NormalTok{(}\StringTok{"./RasterGrids\_100m/2024/Scaled/"}\NormalTok{,nosaukums)}
\NormalTok{slanis}\OtherTok{=}\FunctionTok{rast}\NormalTok{(ielasisanas\_cels)}
\NormalTok{videjais}\OtherTok{=}\FunctionTok{global}\NormalTok{(slanis,}\AttributeTok{fun=}\StringTok{"mean"}\NormalTok{,}\AttributeTok{na.rm=}\ConstantTok{TRUE}\NormalTok{)}
\NormalTok{centrets}\OtherTok{=}\NormalTok{slanis}\SpecialCharTok{{-}}\NormalTok{videjais[,}\DecValTok{1}\NormalTok{]}
\NormalTok{standartnovirze}\OtherTok{=}\NormalTok{terra}\SpecialCharTok{::}\FunctionTok{global}\NormalTok{(centrets,}\AttributeTok{fun=}\StringTok{"rms"}\NormalTok{,}\AttributeTok{na.rm=}\ConstantTok{TRUE}\NormalTok{)}
\NormalTok{merogots}\OtherTok{=}\NormalTok{centrets}\SpecialCharTok{/}\NormalTok{standartnovirze[,}\DecValTok{1}\NormalTok{]}
\FunctionTok{writeRaster}\NormalTok{(merogots,}
      \AttributeTok{filename=}\NormalTok{saglabasanas\_cels,}
      \AttributeTok{overwrite=}\ConstantTok{TRUE}\NormalTok{)}
\end{Highlighting}
\end{Shaded}

\section{Edges\_CropsFallow\_r3000}\label{ch06.133}

\textbf{filename:} \texttt{Edges\_CropsFallow\_r3000.tif}

\textbf{layername:} \texttt{egv\_133}

\textbf{English name:} Edge pixels of Cropland, Fallow land within the 3 km landscape

\textbf{Latvian name:} Aramzemju malu pikseļu skaits 3 km ainavā

\textbf{Procedure:} The total edge within a 3000 m radius around the analysis grid cell is
calculated as the area-weighted sum of the \hyperref[ch06.130]{analysis cells} inside the
buffer, using the workflow \texttt{egvtools::radius\_function()}. During the calculation of the landscape metric,
inverse distance weighted (power = 2) gap filling on the output is applied
to ensure no missing values at the edges. Then the layer is rewritten to set
its name. Finally, the layer is standardised by subtracting the arithmetic
mean and dividing by the root mean squared error.

\begin{Shaded}
\begin{Highlighting}[]
\CommentTok{\# libs {-}{-}{-}{-}}
\ControlFlowTok{if}\NormalTok{(}\SpecialCharTok{!}\FunctionTok{require}\NormalTok{(terra)) \{}\FunctionTok{install.packages}\NormalTok{(}\StringTok{"terra"}\NormalTok{); }\FunctionTok{require}\NormalTok{(terra)\}}
\ControlFlowTok{if}\NormalTok{(}\SpecialCharTok{!}\FunctionTok{require}\NormalTok{(egvtools)) \{remotes}\SpecialCharTok{::}\FunctionTok{install\_github}\NormalTok{(}\StringTok{"aavotins/egvtools"}\NormalTok{); }\FunctionTok{require}\NormalTok{(egvtools)\}}


\CommentTok{\# Templates {-}{-}{-}{-}{-}}
\NormalTok{template100}\OtherTok{=}\FunctionTok{rast}\NormalTok{(}\StringTok{"./Templates/TemplateRasters/LV100m\_10km.tif"}\NormalTok{)}

\CommentTok{\# radii {-}{-}{-}{-}}
\FunctionTok{radius\_function}\NormalTok{(}
 \AttributeTok{kvadrati\_path =} \StringTok{"./Templates/TemplateGrids/tiles/"}\NormalTok{,}
 \AttributeTok{radii\_path   =} \StringTok{"./Templates/TemplateGridPoints/tiles/"}\NormalTok{,}
 \AttributeTok{tikls100\_path =} \StringTok{"./Templates/TemplateGrids/tikls100\_sauzeme.parquet"}\NormalTok{,}
 \AttributeTok{template\_path =} \StringTok{"./Templates/TemplateRasters/LV100m\_10km.tif"}\NormalTok{,}
 \AttributeTok{input\_layers  =} \FunctionTok{c}\NormalTok{(}\StringTok{"./RasterGrids\_100m/2024/RAW/Edges\_CropsFallow\_cell.tif"}\NormalTok{),}
 \AttributeTok{layer\_prefixes =} \FunctionTok{c}\NormalTok{(}\StringTok{"Edges\_CropsFallow"}\NormalTok{),}
 \AttributeTok{output\_dir   =} \StringTok{"./RasterGrids\_100m/2024/RAW/"}\NormalTok{,}
 \AttributeTok{n\_workers   =} \DecValTok{12}\NormalTok{,}
 \AttributeTok{radii     =} \FunctionTok{c}\NormalTok{(}\StringTok{"r3000"}\NormalTok{),}
 \AttributeTok{radius\_mode  =} \StringTok{"sparse"}\NormalTok{,}
 \AttributeTok{extract\_fun  =} \StringTok{"sum"}\NormalTok{,}
 \AttributeTok{fill\_missing  =} \ConstantTok{TRUE}\NormalTok{,}
 \AttributeTok{IDW\_weight   =} \DecValTok{2}\NormalTok{,}
 \AttributeTok{future\_max\_size =} \DecValTok{20} \SpecialCharTok{*} \DecValTok{1024}\SpecialCharTok{\^{}}\DecValTok{3}\NormalTok{)}


\CommentTok{\# Edges\_CropsFallow\_r3000.tif   egv\_133 {-}{-}{-}{-}}
\NormalTok{slanis}\OtherTok{=}\FunctionTok{rast}\NormalTok{(}\StringTok{"./RasterGrids\_100m/2024/RAW/Edges\_CropsFallow\_r3000.tif"}\NormalTok{)}
\FunctionTok{names}\NormalTok{(slanis)}\OtherTok{=}\StringTok{"egv\_133"}
\NormalTok{slanis2}\OtherTok{=}\FunctionTok{project}\NormalTok{(slanis,template100)}
\FunctionTok{writeRaster}\NormalTok{(slanis2,}
      \StringTok{"./RasterGrids\_100m/2024/RAW/Edges\_CropsFallow\_r3000.tif"}\NormalTok{,}
      \AttributeTok{overwrite=}\ConstantTok{TRUE}\NormalTok{)}

\CommentTok{\# standardisation {-}{-}{-}{-}}
\ControlFlowTok{if}\NormalTok{(}\SpecialCharTok{!}\FunctionTok{require}\NormalTok{(terra)) \{}\FunctionTok{install.packages}\NormalTok{(}\StringTok{"terra"}\NormalTok{); }\FunctionTok{require}\NormalTok{(terra)\}}
\ControlFlowTok{if}\NormalTok{(}\SpecialCharTok{!}\FunctionTok{require}\NormalTok{(tidyverse)) \{}\FunctionTok{install.packages}\NormalTok{(}\StringTok{"tidyverse"}\NormalTok{); }\FunctionTok{require}\NormalTok{(tidyverse)\}}

\NormalTok{nosaukums}\OtherTok{=}\StringTok{"Edges\_CropsFallow\_r3000.tif"}
\NormalTok{ielasisanas\_cels}\OtherTok{=}\FunctionTok{paste0}\NormalTok{(}\StringTok{"./RasterGrids\_100m/2024/RAW/"}\NormalTok{,nosaukums)}
\NormalTok{saglabasanas\_cels}\OtherTok{=}\FunctionTok{paste0}\NormalTok{(}\StringTok{"./RasterGrids\_100m/2024/Scaled/"}\NormalTok{,nosaukums)}
\NormalTok{slanis}\OtherTok{=}\FunctionTok{rast}\NormalTok{(ielasisanas\_cels)}
\NormalTok{videjais}\OtherTok{=}\FunctionTok{global}\NormalTok{(slanis,}\AttributeTok{fun=}\StringTok{"mean"}\NormalTok{,}\AttributeTok{na.rm=}\ConstantTok{TRUE}\NormalTok{)}
\NormalTok{centrets}\OtherTok{=}\NormalTok{slanis}\SpecialCharTok{{-}}\NormalTok{videjais[,}\DecValTok{1}\NormalTok{]}
\NormalTok{standartnovirze}\OtherTok{=}\NormalTok{terra}\SpecialCharTok{::}\FunctionTok{global}\NormalTok{(centrets,}\AttributeTok{fun=}\StringTok{"rms"}\NormalTok{,}\AttributeTok{na.rm=}\ConstantTok{TRUE}\NormalTok{)}
\NormalTok{merogots}\OtherTok{=}\NormalTok{centrets}\SpecialCharTok{/}\NormalTok{standartnovirze[,}\DecValTok{1}\NormalTok{]}
\FunctionTok{writeRaster}\NormalTok{(merogots,}
      \AttributeTok{filename=}\NormalTok{saglabasanas\_cels,}
      \AttributeTok{overwrite=}\ConstantTok{TRUE}\NormalTok{)}
\end{Highlighting}
\end{Shaded}

\section{Edges\_CropsFallow\_r10000}\label{ch06.134}

\textbf{filename:} \texttt{Edges\_CropsFallow\_r10000.tif}

\textbf{layername:} \texttt{egv\_134}

\textbf{English name:} Edge pixels of Cropland, Fallow land within the 10 km
landscape

\textbf{Latvian name:} Aramzemju malu pikseļu skaits 10 km ainavā

\textbf{Procedure:} The total edge within a 10000 m radius around the analysis grid cell is
calculated as the area-weighted sum of the \hyperref[ch06.130]{analysis cells} inside the
buffer, using the workflow \texttt{egvtools::radius\_function()}. During the calculation of the landscape metric,
inverse distance weighted (power = 2) gap filling on the output is applied
to ensure no missing values at the edges. Then the layer is rewritten to set
its name. Finally, the layer is standardised by subtracting the arithmetic
mean and dividing by the root mean squared error.

\begin{Shaded}
\begin{Highlighting}[]
\CommentTok{\# libs {-}{-}{-}{-}}
\ControlFlowTok{if}\NormalTok{(}\SpecialCharTok{!}\FunctionTok{require}\NormalTok{(terra)) \{}\FunctionTok{install.packages}\NormalTok{(}\StringTok{"terra"}\NormalTok{); }\FunctionTok{require}\NormalTok{(terra)\}}
\ControlFlowTok{if}\NormalTok{(}\SpecialCharTok{!}\FunctionTok{require}\NormalTok{(egvtools)) \{remotes}\SpecialCharTok{::}\FunctionTok{install\_github}\NormalTok{(}\StringTok{"aavotins/egvtools"}\NormalTok{); }\FunctionTok{require}\NormalTok{(egvtools)\}}


\CommentTok{\# Templates {-}{-}{-}{-}{-}}
\NormalTok{template100}\OtherTok{=}\FunctionTok{rast}\NormalTok{(}\StringTok{"./Templates/TemplateRasters/LV100m\_10km.tif"}\NormalTok{)}

\CommentTok{\# radii {-}{-}{-}{-}}
\FunctionTok{radius\_function}\NormalTok{(}
 \AttributeTok{kvadrati\_path =} \StringTok{"./Templates/TemplateGrids/tiles/"}\NormalTok{,}
 \AttributeTok{radii\_path   =} \StringTok{"./Templates/TemplateGridPoints/tiles/"}\NormalTok{,}
 \AttributeTok{tikls100\_path =} \StringTok{"./Templates/TemplateGrids/tikls100\_sauzeme.parquet"}\NormalTok{,}
 \AttributeTok{template\_path =} \StringTok{"./Templates/TemplateRasters/LV100m\_10km.tif"}\NormalTok{,}
 \AttributeTok{input\_layers  =} \FunctionTok{c}\NormalTok{(}\StringTok{"./RasterGrids\_100m/2024/RAW/Edges\_CropsFallow\_cell.tif"}\NormalTok{),}
 \AttributeTok{layer\_prefixes =} \FunctionTok{c}\NormalTok{(}\StringTok{"Edges\_CropsFallow"}\NormalTok{),}
 \AttributeTok{output\_dir   =} \StringTok{"./RasterGrids\_100m/2024/RAW/"}\NormalTok{,}
 \AttributeTok{n\_workers   =} \DecValTok{12}\NormalTok{,}
 \AttributeTok{radii     =} \FunctionTok{c}\NormalTok{(}\StringTok{"r10000"}\NormalTok{),}
 \AttributeTok{radius\_mode  =} \StringTok{"sparse"}\NormalTok{,}
 \AttributeTok{extract\_fun  =} \StringTok{"sum"}\NormalTok{,}
 \AttributeTok{fill\_missing  =} \ConstantTok{TRUE}\NormalTok{,}
 \AttributeTok{IDW\_weight   =} \DecValTok{2}\NormalTok{,}
 \AttributeTok{future\_max\_size =} \DecValTok{20} \SpecialCharTok{*} \DecValTok{1024}\SpecialCharTok{\^{}}\DecValTok{3}\NormalTok{)}


\CommentTok{\# Edges\_CropsFallow\_r10000.tif  egv\_134 {-}{-}{-}{-}}
\NormalTok{slanis}\OtherTok{=}\FunctionTok{rast}\NormalTok{(}\StringTok{"./RasterGrids\_100m/2024/RAW/Edges\_CropsFallow\_r10000.tif"}\NormalTok{)}
\FunctionTok{names}\NormalTok{(slanis)}\OtherTok{=}\StringTok{"egv\_134"}
\NormalTok{slanis2}\OtherTok{=}\FunctionTok{project}\NormalTok{(slanis,template100)}
\FunctionTok{writeRaster}\NormalTok{(slanis2,}
      \StringTok{"./RasterGrids\_100m/2024/RAW/Edges\_CropsFallow\_r10000.tif"}\NormalTok{,}
      \AttributeTok{overwrite=}\ConstantTok{TRUE}\NormalTok{)}

\CommentTok{\# standardisation {-}{-}{-}{-}}
\ControlFlowTok{if}\NormalTok{(}\SpecialCharTok{!}\FunctionTok{require}\NormalTok{(terra)) \{}\FunctionTok{install.packages}\NormalTok{(}\StringTok{"terra"}\NormalTok{); }\FunctionTok{require}\NormalTok{(terra)\}}
\ControlFlowTok{if}\NormalTok{(}\SpecialCharTok{!}\FunctionTok{require}\NormalTok{(tidyverse)) \{}\FunctionTok{install.packages}\NormalTok{(}\StringTok{"tidyverse"}\NormalTok{); }\FunctionTok{require}\NormalTok{(tidyverse)\}}

\NormalTok{nosaukums}\OtherTok{=}\StringTok{"Edges\_CropsFallow\_r10000.tif"}
\NormalTok{ielasisanas\_cels}\OtherTok{=}\FunctionTok{paste0}\NormalTok{(}\StringTok{"./RasterGrids\_100m/2024/RAW/"}\NormalTok{,nosaukums)}
\NormalTok{saglabasanas\_cels}\OtherTok{=}\FunctionTok{paste0}\NormalTok{(}\StringTok{"./RasterGrids\_100m/2024/Scaled/"}\NormalTok{,nosaukums)}
\NormalTok{slanis}\OtherTok{=}\FunctionTok{rast}\NormalTok{(ielasisanas\_cels)}
\NormalTok{videjais}\OtherTok{=}\FunctionTok{global}\NormalTok{(slanis,}\AttributeTok{fun=}\StringTok{"mean"}\NormalTok{,}\AttributeTok{na.rm=}\ConstantTok{TRUE}\NormalTok{)}
\NormalTok{centrets}\OtherTok{=}\NormalTok{slanis}\SpecialCharTok{{-}}\NormalTok{videjais[,}\DecValTok{1}\NormalTok{]}
\NormalTok{standartnovirze}\OtherTok{=}\NormalTok{terra}\SpecialCharTok{::}\FunctionTok{global}\NormalTok{(centrets,}\AttributeTok{fun=}\StringTok{"rms"}\NormalTok{,}\AttributeTok{na.rm=}\ConstantTok{TRUE}\NormalTok{)}
\NormalTok{merogots}\OtherTok{=}\NormalTok{centrets}\SpecialCharTok{/}\NormalTok{standartnovirze[,}\DecValTok{1}\NormalTok{]}
\FunctionTok{writeRaster}\NormalTok{(merogots,}
      \AttributeTok{filename=}\NormalTok{saglabasanas\_cels,}
      \AttributeTok{overwrite=}\ConstantTok{TRUE}\NormalTok{)}
\end{Highlighting}
\end{Shaded}

\section{Edges\_FarmlandShrubs-Trees\_cell}\label{ch06.135}

\textbf{filename:} \texttt{Edges\_FarmlandShrubs-Trees\_cell.tif}

\textbf{layername:} \texttt{egv\_135}

\textbf{English name:} Edge pixels of Farmland, Clear-Cuts, Shrubs bordering with
Trees within the analysis cell (1 ha)

\textbf{Latvian name:} Lauksaimniecības zemju, izcirtumu, krūmu malu ar kokiem pikseļu skaits
analīzes šūnā (1 ha)

\textbf{Procedure:} First, values between 300 and 400 and between 600 and 630 from
\hyperref[Ch05.03]{Landscape classification} are coded as 0, and all other values as NA.
Then values larger than or equal to 630 but smaller than 700 from the \hyperref[Ch05.03]{Landscape
classification} are coded as 1, and all other values as NA. Then, the
first layer (0 = presence) is covered over the second layer (presence = 1) and
written to file (matching the input). Next, using the workflow
\texttt{egvtools::landscape\_function()} total edge between the two classes is
calculated. During the calculation of the landscape metric, inverse distance weighted
(power = 2) gap filling on the output is applied to ensure no missing values
at the edges. Finally, the layer is standardised by subtracting the arithmetic
mean and dividing by the root mean squared error.

\begin{Shaded}
\begin{Highlighting}[]
\CommentTok{\# libs {-}{-}{-}{-}}
\ControlFlowTok{if}\NormalTok{(}\SpecialCharTok{!}\FunctionTok{require}\NormalTok{(terra)) \{}\FunctionTok{install.packages}\NormalTok{(}\StringTok{"terra"}\NormalTok{); }\FunctionTok{require}\NormalTok{(terra)\}}
\ControlFlowTok{if}\NormalTok{(}\SpecialCharTok{!}\FunctionTok{require}\NormalTok{(egvtools)) \{remotes}\SpecialCharTok{::}\FunctionTok{install\_github}\NormalTok{(}\StringTok{"aavotins/egvtools"}\NormalTok{); }\FunctionTok{require}\NormalTok{(egvtools)\}}

\ControlFlowTok{if}\NormalTok{(}\SpecialCharTok{!}\FunctionTok{require}\NormalTok{(sf)) \{}\FunctionTok{install.packages}\NormalTok{(}\StringTok{"sf"}\NormalTok{); }\FunctionTok{require}\NormalTok{(sf)\}}
\ControlFlowTok{if}\NormalTok{(}\SpecialCharTok{!}\FunctionTok{require}\NormalTok{(sfarrow)) \{}\FunctionTok{install.packages}\NormalTok{(}\StringTok{"sfarrow"}\NormalTok{); }\FunctionTok{require}\NormalTok{(sfarrow)\}}
\ControlFlowTok{if}\NormalTok{(}\SpecialCharTok{!}\FunctionTok{require}\NormalTok{(raster)) \{}\FunctionTok{install.packages}\NormalTok{(}\StringTok{"raster"}\NormalTok{); }\FunctionTok{require}\NormalTok{(raster)\}}
\ControlFlowTok{if}\NormalTok{(}\SpecialCharTok{!}\FunctionTok{require}\NormalTok{(fasterize)) \{}\FunctionTok{install.packages}\NormalTok{(}\StringTok{"fasterize"}\NormalTok{); }\FunctionTok{require}\NormalTok{(fasterize)\}}
\ControlFlowTok{if}\NormalTok{(}\SpecialCharTok{!}\FunctionTok{require}\NormalTok{(tidyverse)) \{}\FunctionTok{install.packages}\NormalTok{(}\StringTok{"tidyverse"}\NormalTok{); }\FunctionTok{require}\NormalTok{(tidyverse)\}}


\CommentTok{\# Templates {-}{-}{-}{-}{-}}
\NormalTok{template10}\OtherTok{=}\FunctionTok{rast}\NormalTok{(}\StringTok{"./Templates/TemplateRasters/LV10m\_10km.tif"}\NormalTok{)}
\NormalTok{nulls10}\OtherTok{=}\FunctionTok{rast}\NormalTok{(}\StringTok{"./Templates/TemplateRasters/nulls\_LV10m\_10km.tif"}\NormalTok{)}

\CommentTok{\# simple landscape {-}{-}{-}{-}}
\NormalTok{simple\_landscape}\OtherTok{=}\FunctionTok{rast}\NormalTok{(}\StringTok{"./RasterGrids\_10m/2024/Ainava\_vienk\_mask.tif"}\NormalTok{)}

\CommentTok{\# Edges\_FarmlandShrubs{-}Trees\_input.tif {-}{-}{-}{-}}
\NormalTok{farmshrub}\OtherTok{=}\FunctionTok{ifel}\NormalTok{((simple\_landscape}\SpecialCharTok{\textgreater{}}\DecValTok{300} \SpecialCharTok{\&}\NormalTok{ simple\_landscape}\SpecialCharTok{\textless{}}\DecValTok{400}\NormalTok{)}\SpecialCharTok{|}
\NormalTok{         (simple\_landscape}\SpecialCharTok{\textgreater{}}\DecValTok{600} \SpecialCharTok{\&}\NormalTok{ simple\_landscape}\SpecialCharTok{\textless{}}\DecValTok{630}\NormalTok{),}\DecValTok{0}\NormalTok{,}\ConstantTok{NA}\NormalTok{)}

\NormalTok{trees\_from630}\OtherTok{=}\FunctionTok{ifel}\NormalTok{(simple\_landscape}\SpecialCharTok{\textgreater{}=}\DecValTok{630} \SpecialCharTok{\&}\NormalTok{ simple\_landscape}\SpecialCharTok{\textless{}}\DecValTok{700}\NormalTok{,}\DecValTok{1}\NormalTok{,}\ConstantTok{NA}\NormalTok{)}
\FunctionTok{plot}\NormalTok{(trees\_from630)}

\NormalTok{farmshrub\_trees630}\OtherTok{=}\FunctionTok{cover}\NormalTok{(farmshrub,trees\_from630)}
\FunctionTok{plot}\NormalTok{(farmshrub\_trees630)}

\NormalTok{edge\_farmshrub\_trees630}\OtherTok{=}\FunctionTok{project}\NormalTok{(farmshrub\_trees630,template10,}
               \AttributeTok{filename=}\StringTok{"./RasterGrids\_10m/2024/Edges\_FarmlandShrubs{-}Trees\_input.tif"}\NormalTok{,}
               \AttributeTok{overwrite=}\ConstantTok{TRUE}\NormalTok{)}
\FunctionTok{rm}\NormalTok{(edge\_farmshrub\_trees630)}
\FunctionTok{rm}\NormalTok{(farmshrub\_trees630)}


\CommentTok{\# Edges\_FarmlandShrubs{-}Trees\_cell.tif   egv\_135 {-}{-}{-}{-}}
\FunctionTok{landscape\_function}\NormalTok{(}
 \AttributeTok{landscape   =} \StringTok{"./RasterGrids\_10m/2024/Edges\_FarmlandShrubs{-}Trees\_input.tif"}\NormalTok{,}
 \AttributeTok{zones     =} \StringTok{"./Templates/TemplateGrids/tikls100\_sauzeme.parquet"}\NormalTok{,}
 \AttributeTok{id\_field    =} \StringTok{"id"}\NormalTok{,}
 \AttributeTok{tile\_field   =} \StringTok{"tks50km"}\NormalTok{,}
 \AttributeTok{template    =} \StringTok{"./Templates/TemplateRasters/LV100m\_10km.tif"}\NormalTok{,}
 \AttributeTok{out\_dir    =} \StringTok{"./RasterGrids\_100m/2024/RAW"}\NormalTok{,}
 \AttributeTok{out\_filename  =} \StringTok{"Edges\_FarmlandShrubs{-}Trees\_cell.tif"}\NormalTok{,}
 \AttributeTok{out\_layername =} \StringTok{"egv\_135"}\NormalTok{,}
 \AttributeTok{what       =} \StringTok{"lsm\_l\_te"}\NormalTok{,}
 \AttributeTok{lm\_args     =} \FunctionTok{list}\NormalTok{(}\AttributeTok{count\_boundary =} \ConstantTok{FALSE}\NormalTok{),}
 \AttributeTok{rasterize\_engine =} \StringTok{"fasterize"}\NormalTok{,}
 \AttributeTok{n\_workers   =} \DecValTok{12}\NormalTok{,}
 \AttributeTok{future\_max\_size =} \DecValTok{20} \SpecialCharTok{*} \DecValTok{1024}\SpecialCharTok{\^{}}\DecValTok{3}\NormalTok{,}
 \AttributeTok{fill\_gaps   =} \ConstantTok{TRUE}\NormalTok{,}
 \AttributeTok{plot\_gaps   =} \ConstantTok{FALSE}\NormalTok{,}
 \AttributeTok{plot\_result  =} \ConstantTok{FALSE}
\NormalTok{)}

\CommentTok{\# standardisation {-}{-}{-}{-}}
\ControlFlowTok{if}\NormalTok{(}\SpecialCharTok{!}\FunctionTok{require}\NormalTok{(terra)) \{}\FunctionTok{install.packages}\NormalTok{(}\StringTok{"terra"}\NormalTok{); }\FunctionTok{require}\NormalTok{(terra)\}}
\ControlFlowTok{if}\NormalTok{(}\SpecialCharTok{!}\FunctionTok{require}\NormalTok{(tidyverse)) \{}\FunctionTok{install.packages}\NormalTok{(}\StringTok{"tidyverse"}\NormalTok{); }\FunctionTok{require}\NormalTok{(tidyverse)\}}

\NormalTok{nosaukums}\OtherTok{=}\StringTok{"Edges\_FarmlandShrubs{-}Trees\_cell.tif"}
\NormalTok{ielasisanas\_cels}\OtherTok{=}\FunctionTok{paste0}\NormalTok{(}\StringTok{"./RasterGrids\_100m/2024/RAW/"}\NormalTok{,nosaukums)}
\NormalTok{saglabasanas\_cels}\OtherTok{=}\FunctionTok{paste0}\NormalTok{(}\StringTok{"./RasterGrids\_100m/2024/Scaled/"}\NormalTok{,nosaukums)}
\NormalTok{slanis}\OtherTok{=}\FunctionTok{rast}\NormalTok{(ielasisanas\_cels)}
\NormalTok{videjais}\OtherTok{=}\FunctionTok{global}\NormalTok{(slanis,}\AttributeTok{fun=}\StringTok{"mean"}\NormalTok{,}\AttributeTok{na.rm=}\ConstantTok{TRUE}\NormalTok{)}
\NormalTok{centrets}\OtherTok{=}\NormalTok{slanis}\SpecialCharTok{{-}}\NormalTok{videjais[,}\DecValTok{1}\NormalTok{]}
\NormalTok{standartnovirze}\OtherTok{=}\NormalTok{terra}\SpecialCharTok{::}\FunctionTok{global}\NormalTok{(centrets,}\AttributeTok{fun=}\StringTok{"rms"}\NormalTok{,}\AttributeTok{na.rm=}\ConstantTok{TRUE}\NormalTok{)}
\NormalTok{merogots}\OtherTok{=}\NormalTok{centrets}\SpecialCharTok{/}\NormalTok{standartnovirze[,}\DecValTok{1}\NormalTok{]}
\FunctionTok{writeRaster}\NormalTok{(merogots,}
      \AttributeTok{filename=}\NormalTok{saglabasanas\_cels,}
      \AttributeTok{overwrite=}\ConstantTok{TRUE}\NormalTok{)}
\end{Highlighting}
\end{Shaded}

\section{Edges\_FarmlandShrubs-Trees\_r500}\label{ch06.136}

\textbf{filename:} \texttt{Edges\_FarmlandShrubs-Trees\_r500.tif}

\textbf{layername:} \texttt{egv\_136}

\textbf{English name:} Edge pixels of Farmland, Clear-Cuts, Shrubs bordering with
Trees within the 0.5 km landscape

\textbf{Latvian name:} Lauksaimniecības zemju, izcirtumu, krūmu malu ar kokiem pikseļu skaits
0,5 km ainavā

\textbf{Procedure:} The total edge within a 500 m radius around the analysis grid cell is
calculated as the area-weighted sum of the \hyperref[ch06.135]{analysis cells} inside the
buffer, using the workflow \texttt{egvtools::radius\_function()}. During the calculation of the landscape metric,
inverse distance weighted (power = 2) gap filling on the output is applied
to ensure no missing values at the edges. Then the layer is rewritten to set
its name. Finally, the layer is standardised by subtracting the arithmetic
mean and dividing by the root mean squared error.

\begin{Shaded}
\begin{Highlighting}[]
\CommentTok{\# libs {-}{-}{-}{-}}
\ControlFlowTok{if}\NormalTok{(}\SpecialCharTok{!}\FunctionTok{require}\NormalTok{(terra)) \{}\FunctionTok{install.packages}\NormalTok{(}\StringTok{"terra"}\NormalTok{); }\FunctionTok{require}\NormalTok{(terra)\}}
\ControlFlowTok{if}\NormalTok{(}\SpecialCharTok{!}\FunctionTok{require}\NormalTok{(egvtools)) \{remotes}\SpecialCharTok{::}\FunctionTok{install\_github}\NormalTok{(}\StringTok{"aavotins/egvtools"}\NormalTok{); }\FunctionTok{require}\NormalTok{(egvtools)\}}


\CommentTok{\# Templates {-}{-}{-}{-}{-}}
\NormalTok{template100}\OtherTok{=}\FunctionTok{rast}\NormalTok{(}\StringTok{"./Templates/TemplateRasters/LV100m\_10km.tif"}\NormalTok{)}

\CommentTok{\# radii {-}{-}{-}{-}}
\FunctionTok{radius\_function}\NormalTok{(}
 \AttributeTok{kvadrati\_path =} \StringTok{"./Templates/TemplateGrids/tiles/"}\NormalTok{,}
 \AttributeTok{radii\_path   =} \StringTok{"./Templates/TemplateGridPoints/tiles/"}\NormalTok{,}
 \AttributeTok{tikls100\_path =} \StringTok{"./Templates/TemplateGrids/tikls100\_sauzeme.parquet"}\NormalTok{,}
 \AttributeTok{template\_path =} \StringTok{"./Templates/TemplateRasters/LV100m\_10km.tif"}\NormalTok{,}
 \AttributeTok{input\_layers  =} \FunctionTok{c}\NormalTok{(}\StringTok{"./RasterGrids\_100m/2024/RAW/Edges\_FarmlandShrubs{-}Trees\_cell.tif"}\NormalTok{),}
 \AttributeTok{layer\_prefixes =} \FunctionTok{c}\NormalTok{(}\StringTok{"Edges\_FarmlandShrubs{-}Trees"}\NormalTok{),}
 \AttributeTok{output\_dir   =} \StringTok{"./RasterGrids\_100m/2024/RAW/"}\NormalTok{,}
 \AttributeTok{n\_workers   =} \DecValTok{12}\NormalTok{,}
 \AttributeTok{radii     =} \FunctionTok{c}\NormalTok{(}\StringTok{"r500"}\NormalTok{),}
 \AttributeTok{radius\_mode  =} \StringTok{"sparse"}\NormalTok{,}
 \AttributeTok{extract\_fun  =} \StringTok{"sum"}\NormalTok{,}
 \AttributeTok{fill\_missing  =} \ConstantTok{TRUE}\NormalTok{,}
 \AttributeTok{IDW\_weight   =} \DecValTok{2}\NormalTok{,}
 \AttributeTok{future\_max\_size =} \DecValTok{20} \SpecialCharTok{*} \DecValTok{1024}\SpecialCharTok{\^{}}\DecValTok{3}\NormalTok{)}


\CommentTok{\# Edges\_FarmlandShrubs{-}Trees\_r500.tif   egv\_136 {-}{-}{-}{-}}
\NormalTok{slanis}\OtherTok{=}\FunctionTok{rast}\NormalTok{(}\StringTok{"./RasterGrids\_100m/2024/RAW/Edges\_FarmlandShrubs{-}Trees\_r500.tif"}\NormalTok{)}
\FunctionTok{names}\NormalTok{(slanis)}\OtherTok{=}\StringTok{"egv\_136"}
\NormalTok{slanis2}\OtherTok{=}\FunctionTok{project}\NormalTok{(slanis,template100)}
\FunctionTok{writeRaster}\NormalTok{(slanis2,}
      \StringTok{"./RasterGrids\_100m/2024/RAW/Edges\_FarmlandShrubs{-}Trees\_r500.tif"}\NormalTok{,}
      \AttributeTok{overwrite=}\ConstantTok{TRUE}\NormalTok{)}

\CommentTok{\# standardisation {-}{-}{-}{-}}
\ControlFlowTok{if}\NormalTok{(}\SpecialCharTok{!}\FunctionTok{require}\NormalTok{(terra)) \{}\FunctionTok{install.packages}\NormalTok{(}\StringTok{"terra"}\NormalTok{); }\FunctionTok{require}\NormalTok{(terra)\}}
\ControlFlowTok{if}\NormalTok{(}\SpecialCharTok{!}\FunctionTok{require}\NormalTok{(tidyverse)) \{}\FunctionTok{install.packages}\NormalTok{(}\StringTok{"tidyverse"}\NormalTok{); }\FunctionTok{require}\NormalTok{(tidyverse)\}}

\NormalTok{nosaukums}\OtherTok{=}\StringTok{"Edges\_FarmlandShrubs{-}Trees\_r500.tif"}
\NormalTok{ielasisanas\_cels}\OtherTok{=}\FunctionTok{paste0}\NormalTok{(}\StringTok{"./RasterGrids\_100m/2024/RAW/"}\NormalTok{,nosaukums)}
\NormalTok{saglabasanas\_cels}\OtherTok{=}\FunctionTok{paste0}\NormalTok{(}\StringTok{"./RasterGrids\_100m/2024/Scaled/"}\NormalTok{,nosaukums)}
\NormalTok{slanis}\OtherTok{=}\FunctionTok{rast}\NormalTok{(ielasisanas\_cels)}
\NormalTok{videjais}\OtherTok{=}\FunctionTok{global}\NormalTok{(slanis,}\AttributeTok{fun=}\StringTok{"mean"}\NormalTok{,}\AttributeTok{na.rm=}\ConstantTok{TRUE}\NormalTok{)}
\NormalTok{centrets}\OtherTok{=}\NormalTok{slanis}\SpecialCharTok{{-}}\NormalTok{videjais[,}\DecValTok{1}\NormalTok{]}
\NormalTok{standartnovirze}\OtherTok{=}\NormalTok{terra}\SpecialCharTok{::}\FunctionTok{global}\NormalTok{(centrets,}\AttributeTok{fun=}\StringTok{"rms"}\NormalTok{,}\AttributeTok{na.rm=}\ConstantTok{TRUE}\NormalTok{)}
\NormalTok{merogots}\OtherTok{=}\NormalTok{centrets}\SpecialCharTok{/}\NormalTok{standartnovirze[,}\DecValTok{1}\NormalTok{]}
\FunctionTok{writeRaster}\NormalTok{(merogots,}
      \AttributeTok{filename=}\NormalTok{saglabasanas\_cels,}
      \AttributeTok{overwrite=}\ConstantTok{TRUE}\NormalTok{)}
\end{Highlighting}
\end{Shaded}

\section{Edges\_FarmlandShrubs-Trees\_r1250}\label{ch06.137}

\textbf{filename:} \texttt{Edges\_FarmlandShrubs-Trees\_r1250.tif}

\textbf{layername:} \texttt{egv\_137}

\textbf{English name:} Edge pixels of Farmland, Clear-Cuts, Shrubs bordering with
Trees within the 1.25 km landscape

\textbf{Latvian name:} Lauksaimniecības zemju, izcirtumu, krūmu malu ar kokiem pikseļu skaits
1,25 km ainavā

\textbf{Procedure:} The total edge within a 1250 m radius around the analysis grid cell is
calculated as the area-weighted sum of the \hyperref[ch06.135]{analysis cells} inside the
buffer, using the workflow \texttt{egvtools::radius\_function()}. During the calculation of the landscape metric,
inverse distance weighted (power = 2) gap filling on the output is applied
to ensure no missing values at the edges. Then the layer is rewritten to set
its name. Finally, the layer is standardised by subtracting the arithmetic
mean and dividing by the root mean squared error.

\begin{Shaded}
\begin{Highlighting}[]
\CommentTok{\# libs {-}{-}{-}{-}}
\ControlFlowTok{if}\NormalTok{(}\SpecialCharTok{!}\FunctionTok{require}\NormalTok{(terra)) \{}\FunctionTok{install.packages}\NormalTok{(}\StringTok{"terra"}\NormalTok{); }\FunctionTok{require}\NormalTok{(terra)\}}
\ControlFlowTok{if}\NormalTok{(}\SpecialCharTok{!}\FunctionTok{require}\NormalTok{(egvtools)) \{remotes}\SpecialCharTok{::}\FunctionTok{install\_github}\NormalTok{(}\StringTok{"aavotins/egvtools"}\NormalTok{); }\FunctionTok{require}\NormalTok{(egvtools)\}}


\CommentTok{\# Templates {-}{-}{-}{-}{-}}
\NormalTok{template100}\OtherTok{=}\FunctionTok{rast}\NormalTok{(}\StringTok{"./Templates/TemplateRasters/LV100m\_10km.tif"}\NormalTok{)}

\CommentTok{\# radii {-}{-}{-}{-}}
\FunctionTok{radius\_function}\NormalTok{(}
 \AttributeTok{kvadrati\_path =} \StringTok{"./Templates/TemplateGrids/tiles/"}\NormalTok{,}
 \AttributeTok{radii\_path   =} \StringTok{"./Templates/TemplateGridPoints/tiles/"}\NormalTok{,}
 \AttributeTok{tikls100\_path =} \StringTok{"./Templates/TemplateGrids/tikls100\_sauzeme.parquet"}\NormalTok{,}
 \AttributeTok{template\_path =} \StringTok{"./Templates/TemplateRasters/LV100m\_10km.tif"}\NormalTok{,}
 \AttributeTok{input\_layers  =} \FunctionTok{c}\NormalTok{(}\StringTok{"./RasterGrids\_100m/2024/RAW/Edges\_FarmlandShrubs{-}Trees\_cell.tif"}\NormalTok{),}
 \AttributeTok{layer\_prefixes =} \FunctionTok{c}\NormalTok{(}\StringTok{"Edges\_FarmlandShrubs{-}Trees"}\NormalTok{),}
 \AttributeTok{output\_dir   =} \StringTok{"./RasterGrids\_100m/2024/RAW/"}\NormalTok{,}
 \AttributeTok{n\_workers   =} \DecValTok{12}\NormalTok{,}
 \AttributeTok{radii     =} \FunctionTok{c}\NormalTok{(}\StringTok{"r1250"}\NormalTok{),}
 \AttributeTok{radius\_mode  =} \StringTok{"sparse"}\NormalTok{,}
 \AttributeTok{extract\_fun  =} \StringTok{"sum"}\NormalTok{,}
 \AttributeTok{fill\_missing  =} \ConstantTok{TRUE}\NormalTok{,}
 \AttributeTok{IDW\_weight   =} \DecValTok{2}\NormalTok{,}
 \AttributeTok{future\_max\_size =} \DecValTok{20} \SpecialCharTok{*} \DecValTok{1024}\SpecialCharTok{\^{}}\DecValTok{3}\NormalTok{)}


\CommentTok{\# Edges\_FarmlandShrubs{-}Trees\_r1250.tif  egv\_137 {-}{-}{-}{-}}
\NormalTok{slanis}\OtherTok{=}\FunctionTok{rast}\NormalTok{(}\StringTok{"./RasterGrids\_100m/2024/RAW/Edges\_FarmlandShrubs{-}Trees\_r1250.tif"}\NormalTok{)}
\FunctionTok{names}\NormalTok{(slanis)}\OtherTok{=}\StringTok{"egv\_137"}
\NormalTok{slanis2}\OtherTok{=}\FunctionTok{project}\NormalTok{(slanis,template100)}
\FunctionTok{writeRaster}\NormalTok{(slanis2,}
      \StringTok{"./RasterGrids\_100m/2024/RAW/Edges\_FarmlandShrubs{-}Trees\_r1250.tif"}\NormalTok{,}
      \AttributeTok{overwrite=}\ConstantTok{TRUE}\NormalTok{)}

\CommentTok{\# standardisation {-}{-}{-}{-}}
\ControlFlowTok{if}\NormalTok{(}\SpecialCharTok{!}\FunctionTok{require}\NormalTok{(terra)) \{}\FunctionTok{install.packages}\NormalTok{(}\StringTok{"terra"}\NormalTok{); }\FunctionTok{require}\NormalTok{(terra)\}}
\ControlFlowTok{if}\NormalTok{(}\SpecialCharTok{!}\FunctionTok{require}\NormalTok{(tidyverse)) \{}\FunctionTok{install.packages}\NormalTok{(}\StringTok{"tidyverse"}\NormalTok{); }\FunctionTok{require}\NormalTok{(tidyverse)\}}

\NormalTok{nosaukums}\OtherTok{=}\StringTok{"Edges\_FarmlandShrubs{-}Trees\_r1250.tif"}
\NormalTok{ielasisanas\_cels}\OtherTok{=}\FunctionTok{paste0}\NormalTok{(}\StringTok{"./RasterGrids\_100m/2024/RAW/"}\NormalTok{,nosaukums)}
\NormalTok{saglabasanas\_cels}\OtherTok{=}\FunctionTok{paste0}\NormalTok{(}\StringTok{"./RasterGrids\_100m/2024/Scaled/"}\NormalTok{,nosaukums)}
\NormalTok{slanis}\OtherTok{=}\FunctionTok{rast}\NormalTok{(ielasisanas\_cels)}
\NormalTok{videjais}\OtherTok{=}\FunctionTok{global}\NormalTok{(slanis,}\AttributeTok{fun=}\StringTok{"mean"}\NormalTok{,}\AttributeTok{na.rm=}\ConstantTok{TRUE}\NormalTok{)}
\NormalTok{centrets}\OtherTok{=}\NormalTok{slanis}\SpecialCharTok{{-}}\NormalTok{videjais[,}\DecValTok{1}\NormalTok{]}
\NormalTok{standartnovirze}\OtherTok{=}\NormalTok{terra}\SpecialCharTok{::}\FunctionTok{global}\NormalTok{(centrets,}\AttributeTok{fun=}\StringTok{"rms"}\NormalTok{,}\AttributeTok{na.rm=}\ConstantTok{TRUE}\NormalTok{)}
\NormalTok{merogots}\OtherTok{=}\NormalTok{centrets}\SpecialCharTok{/}\NormalTok{standartnovirze[,}\DecValTok{1}\NormalTok{]}
\FunctionTok{writeRaster}\NormalTok{(merogots,}
      \AttributeTok{filename=}\NormalTok{saglabasanas\_cels,}
      \AttributeTok{overwrite=}\ConstantTok{TRUE}\NormalTok{)}
\end{Highlighting}
\end{Shaded}

\section{Edges\_FarmlandShrubs-Trees\_r3000}\label{ch06.138}

\textbf{filename:} \texttt{Edges\_FarmlandShrubs-Trees\_r3000.tif}

\textbf{layername:} \texttt{egv\_138}

\textbf{English name:} Edge pixels of Farmland, Clear-Cuts, Shrubs bordering with
Trees within the 3 km landscape

\textbf{Latvian name:} Lauksaimniecības zemju, izcirtumu, krūmu malu ar kokiem pikseļu skaits
3 km ainavā

\textbf{Procedure:} The total edge within a 3000 m radius around the analysis grid cell is
calculated as the area-weighted sum of the \hyperref[ch06.135]{analysis cells} inside the
buffer, using the workflow \texttt{egvtools::radius\_function()}. During the calculation of the landscape metric,
inverse distance weighted (power = 2) gap filling on the output is applied
to ensure no missing values at the edges. Then the layer is rewritten to set
its name. Finally, the layer is standardised by subtracting the arithmetic
mean and dividing by the root mean squared error.

\begin{Shaded}
\begin{Highlighting}[]
\CommentTok{\# libs {-}{-}{-}{-}}
\ControlFlowTok{if}\NormalTok{(}\SpecialCharTok{!}\FunctionTok{require}\NormalTok{(terra)) \{}\FunctionTok{install.packages}\NormalTok{(}\StringTok{"terra"}\NormalTok{); }\FunctionTok{require}\NormalTok{(terra)\}}
\ControlFlowTok{if}\NormalTok{(}\SpecialCharTok{!}\FunctionTok{require}\NormalTok{(egvtools)) \{remotes}\SpecialCharTok{::}\FunctionTok{install\_github}\NormalTok{(}\StringTok{"aavotins/egvtools"}\NormalTok{); }\FunctionTok{require}\NormalTok{(egvtools)\}}


\CommentTok{\# Templates {-}{-}{-}{-}{-}}
\NormalTok{template100}\OtherTok{=}\FunctionTok{rast}\NormalTok{(}\StringTok{"./Templates/TemplateRasters/LV100m\_10km.tif"}\NormalTok{)}

\CommentTok{\# radii {-}{-}{-}{-}}
\FunctionTok{radius\_function}\NormalTok{(}
 \AttributeTok{kvadrati\_path =} \StringTok{"./Templates/TemplateGrids/tiles/"}\NormalTok{,}
 \AttributeTok{radii\_path   =} \StringTok{"./Templates/TemplateGridPoints/tiles/"}\NormalTok{,}
 \AttributeTok{tikls100\_path =} \StringTok{"./Templates/TemplateGrids/tikls100\_sauzeme.parquet"}\NormalTok{,}
 \AttributeTok{template\_path =} \StringTok{"./Templates/TemplateRasters/LV100m\_10km.tif"}\NormalTok{,}
 \AttributeTok{input\_layers  =} \FunctionTok{c}\NormalTok{(}\StringTok{"./RasterGrids\_100m/2024/RAW/Edges\_FarmlandShrubs{-}Trees\_cell.tif"}\NormalTok{),}
 \AttributeTok{layer\_prefixes =} \FunctionTok{c}\NormalTok{(}\StringTok{"Edges\_FarmlandShrubs{-}Trees"}\NormalTok{),}
 \AttributeTok{output\_dir   =} \StringTok{"./RasterGrids\_100m/2024/RAW/"}\NormalTok{,}
 \AttributeTok{n\_workers   =} \DecValTok{12}\NormalTok{,}
 \AttributeTok{radii     =} \FunctionTok{c}\NormalTok{(}\StringTok{"r3000"}\NormalTok{),}
 \AttributeTok{radius\_mode  =} \StringTok{"sparse"}\NormalTok{,}
 \AttributeTok{extract\_fun  =} \StringTok{"sum"}\NormalTok{,}
 \AttributeTok{fill\_missing  =} \ConstantTok{TRUE}\NormalTok{,}
 \AttributeTok{IDW\_weight   =} \DecValTok{2}\NormalTok{,}
 \AttributeTok{future\_max\_size =} \DecValTok{20} \SpecialCharTok{*} \DecValTok{1024}\SpecialCharTok{\^{}}\DecValTok{3}\NormalTok{)}


\CommentTok{\# Edges\_FarmlandShrubs{-}Trees\_r3000.tif  egv\_138 {-}{-}{-}{-}}
\NormalTok{slanis}\OtherTok{=}\FunctionTok{rast}\NormalTok{(}\StringTok{"./RasterGrids\_100m/2024/RAW/Edges\_FarmlandShrubs{-}Trees\_r3000.tif"}\NormalTok{)}
\FunctionTok{names}\NormalTok{(slanis)}\OtherTok{=}\StringTok{"egv\_138"}
\NormalTok{slanis2}\OtherTok{=}\FunctionTok{project}\NormalTok{(slanis,template100)}
\FunctionTok{writeRaster}\NormalTok{(slanis2,}
      \StringTok{"./RasterGrids\_100m/2024/RAW/Edges\_FarmlandShrubs{-}Trees\_r3000.tif"}\NormalTok{,}
      \AttributeTok{overwrite=}\ConstantTok{TRUE}\NormalTok{)}

\CommentTok{\# standardisation {-}{-}{-}{-}}
\ControlFlowTok{if}\NormalTok{(}\SpecialCharTok{!}\FunctionTok{require}\NormalTok{(terra)) \{}\FunctionTok{install.packages}\NormalTok{(}\StringTok{"terra"}\NormalTok{); }\FunctionTok{require}\NormalTok{(terra)\}}
\ControlFlowTok{if}\NormalTok{(}\SpecialCharTok{!}\FunctionTok{require}\NormalTok{(tidyverse)) \{}\FunctionTok{install.packages}\NormalTok{(}\StringTok{"tidyverse"}\NormalTok{); }\FunctionTok{require}\NormalTok{(tidyverse)\}}

\NormalTok{nosaukums}\OtherTok{=}\StringTok{"Edges\_FarmlandShrubs{-}Trees\_r3000.tif"}
\NormalTok{ielasisanas\_cels}\OtherTok{=}\FunctionTok{paste0}\NormalTok{(}\StringTok{"./RasterGrids\_100m/2024/RAW/"}\NormalTok{,nosaukums)}
\NormalTok{saglabasanas\_cels}\OtherTok{=}\FunctionTok{paste0}\NormalTok{(}\StringTok{"./RasterGrids\_100m/2024/Scaled/"}\NormalTok{,nosaukums)}
\NormalTok{slanis}\OtherTok{=}\FunctionTok{rast}\NormalTok{(ielasisanas\_cels)}
\NormalTok{videjais}\OtherTok{=}\FunctionTok{global}\NormalTok{(slanis,}\AttributeTok{fun=}\StringTok{"mean"}\NormalTok{,}\AttributeTok{na.rm=}\ConstantTok{TRUE}\NormalTok{)}
\NormalTok{centrets}\OtherTok{=}\NormalTok{slanis}\SpecialCharTok{{-}}\NormalTok{videjais[,}\DecValTok{1}\NormalTok{]}
\NormalTok{standartnovirze}\OtherTok{=}\NormalTok{terra}\SpecialCharTok{::}\FunctionTok{global}\NormalTok{(centrets,}\AttributeTok{fun=}\StringTok{"rms"}\NormalTok{,}\AttributeTok{na.rm=}\ConstantTok{TRUE}\NormalTok{)}
\NormalTok{merogots}\OtherTok{=}\NormalTok{centrets}\SpecialCharTok{/}\NormalTok{standartnovirze[,}\DecValTok{1}\NormalTok{]}
\FunctionTok{writeRaster}\NormalTok{(merogots,}
      \AttributeTok{filename=}\NormalTok{saglabasanas\_cels,}
      \AttributeTok{overwrite=}\ConstantTok{TRUE}\NormalTok{)}
\end{Highlighting}
\end{Shaded}

\section{Edges\_FarmlandShrubs-Trees\_r10000}\label{ch06.139}

\textbf{filename:} \texttt{Edges\_FarmlandShrubs-Trees\_r10000.tif}

\textbf{layername:} \texttt{egv\_139}

\textbf{English name:} Edge pixels of Farmland, Clear-Cuts, Shrubs bordering with
Trees within the 10 km landscape

\textbf{Latvian name:} Lauksaimniecības zemju, izcirtumu, krūmu malu ar kokiem pikseļu skaits
10 km ainavā

\textbf{Procedure:} The total edge within a 10000 m radius around the analysis grid cell is
calculated as the area-weighted sum of the \hyperref[ch06.135]{analysis cells} inside the
buffer, using the workflow \texttt{egvtools::radius\_function()}. During the calculation of the landscape metric,
inverse distance weighted (power = 2) gap filling on the output is applied
to ensure no missing values at the edges. Then the layer is rewritten to set
its name. Finally, the layer is standardised by subtracting the arithmetic
mean and dividing by the root mean squared error.

\begin{Shaded}
\begin{Highlighting}[]
\CommentTok{\# libs {-}{-}{-}{-}}
\ControlFlowTok{if}\NormalTok{(}\SpecialCharTok{!}\FunctionTok{require}\NormalTok{(terra)) \{}\FunctionTok{install.packages}\NormalTok{(}\StringTok{"terra"}\NormalTok{); }\FunctionTok{require}\NormalTok{(terra)\}}
\ControlFlowTok{if}\NormalTok{(}\SpecialCharTok{!}\FunctionTok{require}\NormalTok{(egvtools)) \{remotes}\SpecialCharTok{::}\FunctionTok{install\_github}\NormalTok{(}\StringTok{"aavotins/egvtools"}\NormalTok{); }\FunctionTok{require}\NormalTok{(egvtools)\}}


\CommentTok{\# Templates {-}{-}{-}{-}{-}}
\NormalTok{template100}\OtherTok{=}\FunctionTok{rast}\NormalTok{(}\StringTok{"./Templates/TemplateRasters/LV100m\_10km.tif"}\NormalTok{)}

\CommentTok{\# radii {-}{-}{-}{-}}
\FunctionTok{radius\_function}\NormalTok{(}
 \AttributeTok{kvadrati\_path =} \StringTok{"./Templates/TemplateGrids/tiles/"}\NormalTok{,}
 \AttributeTok{radii\_path   =} \StringTok{"./Templates/TemplateGridPoints/tiles/"}\NormalTok{,}
 \AttributeTok{tikls100\_path =} \StringTok{"./Templates/TemplateGrids/tikls100\_sauzeme.parquet"}\NormalTok{,}
 \AttributeTok{template\_path =} \StringTok{"./Templates/TemplateRasters/LV100m\_10km.tif"}\NormalTok{,}
 \AttributeTok{input\_layers  =} \FunctionTok{c}\NormalTok{(}\StringTok{"./RasterGrids\_100m/2024/RAW/Edges\_FarmlandShrubs{-}Trees\_cell.tif"}\NormalTok{),}
 \AttributeTok{layer\_prefixes =} \FunctionTok{c}\NormalTok{(}\StringTok{"Edges\_FarmlandShrubs{-}Trees"}\NormalTok{),}
 \AttributeTok{output\_dir   =} \StringTok{"./RasterGrids\_100m/2024/RAW/"}\NormalTok{,}
 \AttributeTok{n\_workers   =} \DecValTok{12}\NormalTok{,}
 \AttributeTok{radii     =} \FunctionTok{c}\NormalTok{(}\StringTok{"r10000"}\NormalTok{),}
 \AttributeTok{radius\_mode  =} \StringTok{"sparse"}\NormalTok{,}
 \AttributeTok{extract\_fun  =} \StringTok{"sum"}\NormalTok{,}
 \AttributeTok{fill\_missing  =} \ConstantTok{TRUE}\NormalTok{,}
 \AttributeTok{IDW\_weight   =} \DecValTok{2}\NormalTok{,}
 \AttributeTok{future\_max\_size =} \DecValTok{20} \SpecialCharTok{*} \DecValTok{1024}\SpecialCharTok{\^{}}\DecValTok{3}\NormalTok{)}


\CommentTok{\# Edges\_FarmlandShrubs{-}Trees\_r10000.tif egv\_139 {-}{-}{-}{-}}
\NormalTok{slanis}\OtherTok{=}\FunctionTok{rast}\NormalTok{(}\StringTok{"./RasterGrids\_100m/2024/RAW/Edges\_FarmlandShrubs{-}Trees\_r10000.tif"}\NormalTok{)}
\FunctionTok{names}\NormalTok{(slanis)}\OtherTok{=}\StringTok{"egv\_139"}
\NormalTok{slanis2}\OtherTok{=}\FunctionTok{project}\NormalTok{(slanis,template100)}
\FunctionTok{writeRaster}\NormalTok{(slanis2,}
      \StringTok{"./RasterGrids\_100m/2024/RAW/Edges\_FarmlandShrubs{-}Trees\_r10000.tif"}\NormalTok{,}
      \AttributeTok{overwrite=}\ConstantTok{TRUE}\NormalTok{)}

\CommentTok{\# standardisation {-}{-}{-}{-}}
\ControlFlowTok{if}\NormalTok{(}\SpecialCharTok{!}\FunctionTok{require}\NormalTok{(terra)) \{}\FunctionTok{install.packages}\NormalTok{(}\StringTok{"terra"}\NormalTok{); }\FunctionTok{require}\NormalTok{(terra)\}}
\ControlFlowTok{if}\NormalTok{(}\SpecialCharTok{!}\FunctionTok{require}\NormalTok{(tidyverse)) \{}\FunctionTok{install.packages}\NormalTok{(}\StringTok{"tidyverse"}\NormalTok{); }\FunctionTok{require}\NormalTok{(tidyverse)\}}

\NormalTok{nosaukums}\OtherTok{=}\StringTok{"Edges\_FarmlandShrubs{-}Trees\_r10000.tif"}
\NormalTok{ielasisanas\_cels}\OtherTok{=}\FunctionTok{paste0}\NormalTok{(}\StringTok{"./RasterGrids\_100m/2024/RAW/"}\NormalTok{,nosaukums)}
\NormalTok{saglabasanas\_cels}\OtherTok{=}\FunctionTok{paste0}\NormalTok{(}\StringTok{"./RasterGrids\_100m/2024/Scaled/"}\NormalTok{,nosaukums)}
\NormalTok{slanis}\OtherTok{=}\FunctionTok{rast}\NormalTok{(ielasisanas\_cels)}
\NormalTok{videjais}\OtherTok{=}\FunctionTok{global}\NormalTok{(slanis,}\AttributeTok{fun=}\StringTok{"mean"}\NormalTok{,}\AttributeTok{na.rm=}\ConstantTok{TRUE}\NormalTok{)}
\NormalTok{centrets}\OtherTok{=}\NormalTok{slanis}\SpecialCharTok{{-}}\NormalTok{videjais[,}\DecValTok{1}\NormalTok{]}
\NormalTok{standartnovirze}\OtherTok{=}\NormalTok{terra}\SpecialCharTok{::}\FunctionTok{global}\NormalTok{(centrets,}\AttributeTok{fun=}\StringTok{"rms"}\NormalTok{,}\AttributeTok{na.rm=}\ConstantTok{TRUE}\NormalTok{)}
\NormalTok{merogots}\OtherTok{=}\NormalTok{centrets}\SpecialCharTok{/}\NormalTok{standartnovirze[,}\DecValTok{1}\NormalTok{]}
\FunctionTok{writeRaster}\NormalTok{(merogots,}
      \AttributeTok{filename=}\NormalTok{saglabasanas\_cels,}
      \AttributeTok{overwrite=}\ConstantTok{TRUE}\NormalTok{)}
\end{Highlighting}
\end{Shaded}

\section{Edges\_Grasslands\_cell}\label{ch06.140}

\textbf{filename:} \texttt{Edges\_Grasslands\_cell.tif}

\textbf{layername:} \texttt{egv\_140}

\textbf{English name:} Edge pixels of Grassland within the analysis cell (1 ha)

\textbf{Latvian name:} Zālāju malu pikseļu skaits analīzes šūnā (1 ha)

\textbf{Procedure:} First, values equal to 330 from the \hyperref[Ch05.03]{Landscape
classification} are coded as 1, and all other values as NA. Then, the
layer (1 = presence) is covered over the nulls layer (presence = 0) and written to
file (matching the input). Then, using the workflow
\texttt{egvtools::landscape\_function()} total edge between the two classes is
calculated. During the calculation of the landscape metric, inverse distance weighted
(power = 2) gap filling on the output is applied to ensure no missing values
at the edges. Finally, the layer is standardised by subtracting the arithmetic
mean and dividing by the root mean squared error.

\begin{Shaded}
\begin{Highlighting}[]
\CommentTok{\# libs {-}{-}{-}{-}}
\ControlFlowTok{if}\NormalTok{(}\SpecialCharTok{!}\FunctionTok{require}\NormalTok{(terra)) \{}\FunctionTok{install.packages}\NormalTok{(}\StringTok{"terra"}\NormalTok{); }\FunctionTok{require}\NormalTok{(terra)\}}
\ControlFlowTok{if}\NormalTok{(}\SpecialCharTok{!}\FunctionTok{require}\NormalTok{(egvtools)) \{remotes}\SpecialCharTok{::}\FunctionTok{install\_github}\NormalTok{(}\StringTok{"aavotins/egvtools"}\NormalTok{); }\FunctionTok{require}\NormalTok{(egvtools)\}}

\ControlFlowTok{if}\NormalTok{(}\SpecialCharTok{!}\FunctionTok{require}\NormalTok{(sf)) \{}\FunctionTok{install.packages}\NormalTok{(}\StringTok{"sf"}\NormalTok{); }\FunctionTok{require}\NormalTok{(sf)\}}
\ControlFlowTok{if}\NormalTok{(}\SpecialCharTok{!}\FunctionTok{require}\NormalTok{(sfarrow)) \{}\FunctionTok{install.packages}\NormalTok{(}\StringTok{"sfarrow"}\NormalTok{); }\FunctionTok{require}\NormalTok{(sfarrow)\}}
\ControlFlowTok{if}\NormalTok{(}\SpecialCharTok{!}\FunctionTok{require}\NormalTok{(raster)) \{}\FunctionTok{install.packages}\NormalTok{(}\StringTok{"raster"}\NormalTok{); }\FunctionTok{require}\NormalTok{(raster)\}}
\ControlFlowTok{if}\NormalTok{(}\SpecialCharTok{!}\FunctionTok{require}\NormalTok{(fasterize)) \{}\FunctionTok{install.packages}\NormalTok{(}\StringTok{"fasterize"}\NormalTok{); }\FunctionTok{require}\NormalTok{(fasterize)\}}
\ControlFlowTok{if}\NormalTok{(}\SpecialCharTok{!}\FunctionTok{require}\NormalTok{(tidyverse)) \{}\FunctionTok{install.packages}\NormalTok{(}\StringTok{"tidyverse"}\NormalTok{); }\FunctionTok{require}\NormalTok{(tidyverse)\}}


\CommentTok{\# Templates {-}{-}{-}{-}{-}}
\NormalTok{template10}\OtherTok{=}\FunctionTok{rast}\NormalTok{(}\StringTok{"./Templates/TemplateRasters/LV10m\_10km.tif"}\NormalTok{)}
\NormalTok{nulls10}\OtherTok{=}\FunctionTok{rast}\NormalTok{(}\StringTok{"./Templates/TemplateRasters/nulls\_LV10m\_10km.tif"}\NormalTok{)}

\CommentTok{\# simple landscape {-}{-}{-}{-}}
\NormalTok{simple\_landscape}\OtherTok{=}\FunctionTok{rast}\NormalTok{(}\StringTok{"./RasterGrids\_10m/2024/Ainava\_vienk\_mask.tif"}\NormalTok{)}

\CommentTok{\# Edges\_Grasslands\_input.tif {-}{-}{-}{-}}
\NormalTok{grassland}\OtherTok{=}\FunctionTok{ifel}\NormalTok{(simple\_landscape}\SpecialCharTok{==}\DecValTok{330}\NormalTok{,}\DecValTok{1}\NormalTok{,}\ConstantTok{NA}\NormalTok{)}
\FunctionTok{plot}\NormalTok{(grassland)}
\NormalTok{grassland}\OtherTok{=}\FunctionTok{cover}\NormalTok{(grassland,nulls10)}
\FunctionTok{plot}\NormalTok{(grassland)}

\NormalTok{edge\_grassland}\OtherTok{=}\FunctionTok{project}\NormalTok{(grassland,template10,}
              \AttributeTok{filename=}\StringTok{"./RasterGrids\_10m/2024/Edges\_Grasslands\_input.tif"}\NormalTok{,}
              \AttributeTok{overwrite=}\ConstantTok{TRUE}\NormalTok{)}
\FunctionTok{rm}\NormalTok{(edge\_grassland)}


\CommentTok{\# Edges\_Grasslands\_cell.tif egv\_140 {-}{-}{-}{-}}
\FunctionTok{landscape\_function}\NormalTok{(}
 \AttributeTok{landscape   =} \StringTok{"./RasterGrids\_10m/2024/Edges\_Grasslands\_input.tif"}\NormalTok{,}
 \AttributeTok{zones     =} \StringTok{"./Templates/TemplateGrids/tikls100\_sauzeme.parquet"}\NormalTok{,}
 \AttributeTok{id\_field    =} \StringTok{"id"}\NormalTok{,}
 \AttributeTok{tile\_field   =} \StringTok{"tks50km"}\NormalTok{,}
 \AttributeTok{template    =} \StringTok{"./Templates/TemplateRasters/LV100m\_10km.tif"}\NormalTok{,}
 \AttributeTok{out\_dir    =} \StringTok{"./RasterGrids\_100m/2024/RAW"}\NormalTok{,}
 \AttributeTok{out\_filename  =} \StringTok{"Edges\_Grasslands\_cell.tif"}\NormalTok{,}
 \AttributeTok{out\_layername =} \StringTok{"egv\_140"}\NormalTok{,}
 \AttributeTok{what       =} \StringTok{"lsm\_l\_te"}\NormalTok{,}
 \AttributeTok{lm\_args     =} \FunctionTok{list}\NormalTok{(}\AttributeTok{count\_boundary =} \ConstantTok{FALSE}\NormalTok{),}
 \AttributeTok{rasterize\_engine =} \StringTok{"fasterize"}\NormalTok{,}
 \AttributeTok{n\_workers   =} \DecValTok{12}\NormalTok{,}
 \AttributeTok{future\_max\_size =} \DecValTok{20} \SpecialCharTok{*} \DecValTok{1024}\SpecialCharTok{\^{}}\DecValTok{3}\NormalTok{,}
 \AttributeTok{fill\_gaps   =} \ConstantTok{TRUE}\NormalTok{,}
 \AttributeTok{plot\_gaps   =} \ConstantTok{FALSE}\NormalTok{,}
 \AttributeTok{plot\_result  =} \ConstantTok{FALSE}
\NormalTok{)}

\CommentTok{\# standardisation {-}{-}{-}{-}}
\ControlFlowTok{if}\NormalTok{(}\SpecialCharTok{!}\FunctionTok{require}\NormalTok{(terra)) \{}\FunctionTok{install.packages}\NormalTok{(}\StringTok{"terra"}\NormalTok{); }\FunctionTok{require}\NormalTok{(terra)\}}
\ControlFlowTok{if}\NormalTok{(}\SpecialCharTok{!}\FunctionTok{require}\NormalTok{(tidyverse)) \{}\FunctionTok{install.packages}\NormalTok{(}\StringTok{"tidyverse"}\NormalTok{); }\FunctionTok{require}\NormalTok{(tidyverse)\}}

\NormalTok{nosaukums}\OtherTok{=}\StringTok{"Edges\_Grasslands\_cell.tif"}
\NormalTok{ielasisanas\_cels}\OtherTok{=}\FunctionTok{paste0}\NormalTok{(}\StringTok{"./RasterGrids\_100m/2024/RAW/"}\NormalTok{,nosaukums)}
\NormalTok{saglabasanas\_cels}\OtherTok{=}\FunctionTok{paste0}\NormalTok{(}\StringTok{"./RasterGrids\_100m/2024/Scaled/"}\NormalTok{,nosaukums)}
\NormalTok{slanis}\OtherTok{=}\FunctionTok{rast}\NormalTok{(ielasisanas\_cels)}
\NormalTok{videjais}\OtherTok{=}\FunctionTok{global}\NormalTok{(slanis,}\AttributeTok{fun=}\StringTok{"mean"}\NormalTok{,}\AttributeTok{na.rm=}\ConstantTok{TRUE}\NormalTok{)}
\NormalTok{centrets}\OtherTok{=}\NormalTok{slanis}\SpecialCharTok{{-}}\NormalTok{videjais[,}\DecValTok{1}\NormalTok{]}
\NormalTok{standartnovirze}\OtherTok{=}\NormalTok{terra}\SpecialCharTok{::}\FunctionTok{global}\NormalTok{(centrets,}\AttributeTok{fun=}\StringTok{"rms"}\NormalTok{,}\AttributeTok{na.rm=}\ConstantTok{TRUE}\NormalTok{)}
\NormalTok{merogots}\OtherTok{=}\NormalTok{centrets}\SpecialCharTok{/}\NormalTok{standartnovirze[,}\DecValTok{1}\NormalTok{]}
\FunctionTok{writeRaster}\NormalTok{(merogots,}
      \AttributeTok{filename=}\NormalTok{saglabasanas\_cels,}
      \AttributeTok{overwrite=}\ConstantTok{TRUE}\NormalTok{)}
\end{Highlighting}
\end{Shaded}

\section{Edges\_Grasslands\_r500}\label{ch06.141}

\textbf{filename:} \texttt{Edges\_Grasslands\_r500.tif}

\textbf{layername:} \texttt{egv\_141}

\textbf{English name:} Edge pixels of Grassland within the 0.5 km landscape

\textbf{Latvian name:} Zālāju malu pikseļu skaits 0,5 km ainavā

\textbf{Procedure:} The total edge within a 500 m radius around the analysis grid cell is
calculated as the area-weighted sum of the \hyperref[ch06.140]{analysis cells} inside the
buffer, using the workflow \texttt{egvtools::radius\_function()}. During the calculation of the landscape metric,
inverse distance weighted (power = 2) gap filling on the output is applied
to ensure no missing values at the edges. Then the layer is rewritten to set
its name. Finally, the layer is standardised by subtracting the arithmetic
mean and dividing by the root mean squared error.

\begin{Shaded}
\begin{Highlighting}[]
\CommentTok{\# libs {-}{-}{-}{-}}
\ControlFlowTok{if}\NormalTok{(}\SpecialCharTok{!}\FunctionTok{require}\NormalTok{(terra)) \{}\FunctionTok{install.packages}\NormalTok{(}\StringTok{"terra"}\NormalTok{); }\FunctionTok{require}\NormalTok{(terra)\}}
\ControlFlowTok{if}\NormalTok{(}\SpecialCharTok{!}\FunctionTok{require}\NormalTok{(egvtools)) \{remotes}\SpecialCharTok{::}\FunctionTok{install\_github}\NormalTok{(}\StringTok{"aavotins/egvtools"}\NormalTok{); }\FunctionTok{require}\NormalTok{(egvtools)\}}


\CommentTok{\# Templates {-}{-}{-}{-}{-}}
\NormalTok{template100}\OtherTok{=}\FunctionTok{rast}\NormalTok{(}\StringTok{"./Templates/TemplateRasters/LV100m\_10km.tif"}\NormalTok{)}

\CommentTok{\# radii {-}{-}{-}{-}}
\FunctionTok{radius\_function}\NormalTok{(}
 \AttributeTok{kvadrati\_path =} \StringTok{"./Templates/TemplateGrids/tiles/"}\NormalTok{,}
 \AttributeTok{radii\_path   =} \StringTok{"./Templates/TemplateGridPoints/tiles/"}\NormalTok{,}
 \AttributeTok{tikls100\_path =} \StringTok{"./Templates/TemplateGrids/tikls100\_sauzeme.parquet"}\NormalTok{,}
 \AttributeTok{template\_path =} \StringTok{"./Templates/TemplateRasters/LV100m\_10km.tif"}\NormalTok{,}
 \AttributeTok{input\_layers  =} \FunctionTok{c}\NormalTok{(}\StringTok{"./RasterGrids\_100m/2024/RAW/Edges\_Grasslands\_cell.tif"}\NormalTok{),}
 \AttributeTok{layer\_prefixes =} \FunctionTok{c}\NormalTok{(}\StringTok{"Edges\_Grasslands"}\NormalTok{),}
 \AttributeTok{output\_dir   =} \StringTok{"./RasterGrids\_100m/2024/RAW/"}\NormalTok{,}
 \AttributeTok{n\_workers   =} \DecValTok{12}\NormalTok{,}
 \AttributeTok{radii     =} \FunctionTok{c}\NormalTok{(}\StringTok{"r500"}\NormalTok{),}
 \AttributeTok{radius\_mode  =} \StringTok{"sparse"}\NormalTok{,}
 \AttributeTok{extract\_fun  =} \StringTok{"sum"}\NormalTok{,}
 \AttributeTok{fill\_missing  =} \ConstantTok{TRUE}\NormalTok{,}
 \AttributeTok{IDW\_weight   =} \DecValTok{2}\NormalTok{,}
 \AttributeTok{future\_max\_size =} \DecValTok{20} \SpecialCharTok{*} \DecValTok{1024}\SpecialCharTok{\^{}}\DecValTok{3}\NormalTok{)}


\CommentTok{\# Edges\_Grasslands\_r500.tif egv\_141 {-}{-}{-}{-}}
\NormalTok{slanis}\OtherTok{=}\FunctionTok{rast}\NormalTok{(}\StringTok{"./RasterGrids\_100m/2024/RAW/Edges\_Grasslands\_r500.tif"}\NormalTok{)}
\FunctionTok{names}\NormalTok{(slanis)}\OtherTok{=}\StringTok{"egv\_141"}
\NormalTok{slanis2}\OtherTok{=}\FunctionTok{project}\NormalTok{(slanis,template100)}
\FunctionTok{writeRaster}\NormalTok{(slanis2,}
      \StringTok{"./RasterGrids\_100m/2024/RAW/Edges\_Grasslands\_r500.tif"}\NormalTok{,}
      \AttributeTok{overwrite=}\ConstantTok{TRUE}\NormalTok{)}

\CommentTok{\# standardisation {-}{-}{-}{-}}
\ControlFlowTok{if}\NormalTok{(}\SpecialCharTok{!}\FunctionTok{require}\NormalTok{(terra)) \{}\FunctionTok{install.packages}\NormalTok{(}\StringTok{"terra"}\NormalTok{); }\FunctionTok{require}\NormalTok{(terra)\}}
\ControlFlowTok{if}\NormalTok{(}\SpecialCharTok{!}\FunctionTok{require}\NormalTok{(tidyverse)) \{}\FunctionTok{install.packages}\NormalTok{(}\StringTok{"tidyverse"}\NormalTok{); }\FunctionTok{require}\NormalTok{(tidyverse)\}}

\NormalTok{nosaukums}\OtherTok{=}\StringTok{"Edges\_Grasslands\_r500.tif"}
\NormalTok{ielasisanas\_cels}\OtherTok{=}\FunctionTok{paste0}\NormalTok{(}\StringTok{"./RasterGrids\_100m/2024/RAW/"}\NormalTok{,nosaukums)}
\NormalTok{saglabasanas\_cels}\OtherTok{=}\FunctionTok{paste0}\NormalTok{(}\StringTok{"./RasterGrids\_100m/2024/Scaled/"}\NormalTok{,nosaukums)}
\NormalTok{slanis}\OtherTok{=}\FunctionTok{rast}\NormalTok{(ielasisanas\_cels)}
\NormalTok{videjais}\OtherTok{=}\FunctionTok{global}\NormalTok{(slanis,}\AttributeTok{fun=}\StringTok{"mean"}\NormalTok{,}\AttributeTok{na.rm=}\ConstantTok{TRUE}\NormalTok{)}
\NormalTok{centrets}\OtherTok{=}\NormalTok{slanis}\SpecialCharTok{{-}}\NormalTok{videjais[,}\DecValTok{1}\NormalTok{]}
\NormalTok{standartnovirze}\OtherTok{=}\NormalTok{terra}\SpecialCharTok{::}\FunctionTok{global}\NormalTok{(centrets,}\AttributeTok{fun=}\StringTok{"rms"}\NormalTok{,}\AttributeTok{na.rm=}\ConstantTok{TRUE}\NormalTok{)}
\NormalTok{merogots}\OtherTok{=}\NormalTok{centrets}\SpecialCharTok{/}\NormalTok{standartnovirze[,}\DecValTok{1}\NormalTok{]}
\FunctionTok{writeRaster}\NormalTok{(merogots,}
      \AttributeTok{filename=}\NormalTok{saglabasanas\_cels,}
      \AttributeTok{overwrite=}\ConstantTok{TRUE}\NormalTok{)}
\end{Highlighting}
\end{Shaded}

\section{Edges\_Grasslands\_r1250}\label{ch06.142}

\textbf{filename:} \texttt{Edges\_Grasslands\_r1250.tif}

\textbf{layername:} \texttt{egv\_142}

\textbf{English name:} Edge pixels of Grassland within the 1.25 km landscape

\textbf{Latvian name:} Zālāju malu pikseļu skaits 1,25 km ainavā

\textbf{Procedure:} The total edge within a 1250 m radius around the analysis grid cell is
calculated as the area-weighted sum of the \hyperref[ch06.140]{analysis cells} inside the
buffer, using the workflow \texttt{egvtools::radius\_function()}. During the calculation of the landscape metric,
inverse distance weighted (power = 2) gap filling on the output is applied
to ensure no missing values at the edges. Then the layer is rewritten to set
its name. Finally, the layer is standardised by subtracting the arithmetic
mean and dividing by the root mean squared error.

\begin{Shaded}
\begin{Highlighting}[]
\CommentTok{\# libs {-}{-}{-}{-}}
\ControlFlowTok{if}\NormalTok{(}\SpecialCharTok{!}\FunctionTok{require}\NormalTok{(terra)) \{}\FunctionTok{install.packages}\NormalTok{(}\StringTok{"terra"}\NormalTok{); }\FunctionTok{require}\NormalTok{(terra)\}}
\ControlFlowTok{if}\NormalTok{(}\SpecialCharTok{!}\FunctionTok{require}\NormalTok{(egvtools)) \{remotes}\SpecialCharTok{::}\FunctionTok{install\_github}\NormalTok{(}\StringTok{"aavotins/egvtools"}\NormalTok{); }\FunctionTok{require}\NormalTok{(egvtools)\}}


\CommentTok{\# Templates {-}{-}{-}{-}{-}}
\NormalTok{template100}\OtherTok{=}\FunctionTok{rast}\NormalTok{(}\StringTok{"./Templates/TemplateRasters/LV100m\_10km.tif"}\NormalTok{)}

\CommentTok{\# radii {-}{-}{-}{-}}
\FunctionTok{radius\_function}\NormalTok{(}
 \AttributeTok{kvadrati\_path =} \StringTok{"./Templates/TemplateGrids/tiles/"}\NormalTok{,}
 \AttributeTok{radii\_path   =} \StringTok{"./Templates/TemplateGridPoints/tiles/"}\NormalTok{,}
 \AttributeTok{tikls100\_path =} \StringTok{"./Templates/TemplateGrids/tikls100\_sauzeme.parquet"}\NormalTok{,}
 \AttributeTok{template\_path =} \StringTok{"./Templates/TemplateRasters/LV100m\_10km.tif"}\NormalTok{,}
 \AttributeTok{input\_layers  =} \FunctionTok{c}\NormalTok{(}\StringTok{"./RasterGrids\_100m/2024/RAW/Edges\_Grasslands\_cell.tif"}\NormalTok{),}
 \AttributeTok{layer\_prefixes =} \FunctionTok{c}\NormalTok{(}\StringTok{"Edges\_Grasslands"}\NormalTok{),}
 \AttributeTok{output\_dir   =} \StringTok{"./RasterGrids\_100m/2024/RAW/"}\NormalTok{,}
 \AttributeTok{n\_workers   =} \DecValTok{12}\NormalTok{,}
 \AttributeTok{radii     =} \FunctionTok{c}\NormalTok{(}\StringTok{"r1250"}\NormalTok{),}
 \AttributeTok{radius\_mode  =} \StringTok{"sparse"}\NormalTok{,}
 \AttributeTok{extract\_fun  =} \StringTok{"sum"}\NormalTok{,}
 \AttributeTok{fill\_missing  =} \ConstantTok{TRUE}\NormalTok{,}
 \AttributeTok{IDW\_weight   =} \DecValTok{2}\NormalTok{,}
 \AttributeTok{future\_max\_size =} \DecValTok{20} \SpecialCharTok{*} \DecValTok{1024}\SpecialCharTok{\^{}}\DecValTok{3}\NormalTok{)}


\CommentTok{\# Edges\_Grasslands\_r1250.tif    egv\_142 {-}{-}{-}{-}}
\NormalTok{slanis}\OtherTok{=}\FunctionTok{rast}\NormalTok{(}\StringTok{"./RasterGrids\_100m/2024/RAW/Edges\_Grasslands\_r1250.tif"}\NormalTok{)}
\FunctionTok{names}\NormalTok{(slanis)}\OtherTok{=}\StringTok{"egv\_142"}
\NormalTok{slanis2}\OtherTok{=}\FunctionTok{project}\NormalTok{(slanis,template100)}
\FunctionTok{writeRaster}\NormalTok{(slanis2,}
      \StringTok{"./RasterGrids\_100m/2024/RAW/Edges\_Grasslands\_r1250.tif"}\NormalTok{,}
      \AttributeTok{overwrite=}\ConstantTok{TRUE}\NormalTok{)}

\CommentTok{\# standardisation {-}{-}{-}{-}}
\ControlFlowTok{if}\NormalTok{(}\SpecialCharTok{!}\FunctionTok{require}\NormalTok{(terra)) \{}\FunctionTok{install.packages}\NormalTok{(}\StringTok{"terra"}\NormalTok{); }\FunctionTok{require}\NormalTok{(terra)\}}
\ControlFlowTok{if}\NormalTok{(}\SpecialCharTok{!}\FunctionTok{require}\NormalTok{(tidyverse)) \{}\FunctionTok{install.packages}\NormalTok{(}\StringTok{"tidyverse"}\NormalTok{); }\FunctionTok{require}\NormalTok{(tidyverse)\}}

\NormalTok{nosaukums}\OtherTok{=}\StringTok{"Edges\_Grasslands\_r1250.tif"}
\NormalTok{ielasisanas\_cels}\OtherTok{=}\FunctionTok{paste0}\NormalTok{(}\StringTok{"./RasterGrids\_100m/2024/RAW/"}\NormalTok{,nosaukums)}
\NormalTok{saglabasanas\_cels}\OtherTok{=}\FunctionTok{paste0}\NormalTok{(}\StringTok{"./RasterGrids\_100m/2024/Scaled/"}\NormalTok{,nosaukums)}
\NormalTok{slanis}\OtherTok{=}\FunctionTok{rast}\NormalTok{(ielasisanas\_cels)}
\NormalTok{videjais}\OtherTok{=}\FunctionTok{global}\NormalTok{(slanis,}\AttributeTok{fun=}\StringTok{"mean"}\NormalTok{,}\AttributeTok{na.rm=}\ConstantTok{TRUE}\NormalTok{)}
\NormalTok{centrets}\OtherTok{=}\NormalTok{slanis}\SpecialCharTok{{-}}\NormalTok{videjais[,}\DecValTok{1}\NormalTok{]}
\NormalTok{standartnovirze}\OtherTok{=}\NormalTok{terra}\SpecialCharTok{::}\FunctionTok{global}\NormalTok{(centrets,}\AttributeTok{fun=}\StringTok{"rms"}\NormalTok{,}\AttributeTok{na.rm=}\ConstantTok{TRUE}\NormalTok{)}
\NormalTok{merogots}\OtherTok{=}\NormalTok{centrets}\SpecialCharTok{/}\NormalTok{standartnovirze[,}\DecValTok{1}\NormalTok{]}
\FunctionTok{writeRaster}\NormalTok{(merogots,}
      \AttributeTok{filename=}\NormalTok{saglabasanas\_cels,}
      \AttributeTok{overwrite=}\ConstantTok{TRUE}\NormalTok{)}
\end{Highlighting}
\end{Shaded}

\section{Edges\_Grasslands\_r3000}\label{ch06.143}

\textbf{filename:} \texttt{Edges\_Grasslands\_r3000.tif}

\textbf{layername:} \texttt{egv\_143}

\textbf{English name:} Edge pixels of Grassland within the 3 km landscape

\textbf{Latvian name:} Zālāju malu pikseļu skaits 3 km ainavā

\textbf{Procedure:} The total edge within a 3000 m radius around the analysis grid cell is
calculated as the area-weighted sum of the \hyperref[ch06.140]{analysis cells} inside the
buffer, using the workflow \texttt{egvtools::radius\_function()}. During the calculation of the landscape metric,
inverse distance weighted (power = 2) gap filling on the output is applied
to ensure no missing values at the edges. Then the layer is rewritten to set
its name. Finally, the layer is standardised by subtracting the arithmetic
mean and dividing by the root mean squared error.

\begin{Shaded}
\begin{Highlighting}[]
\CommentTok{\# libs {-}{-}{-}{-}}
\ControlFlowTok{if}\NormalTok{(}\SpecialCharTok{!}\FunctionTok{require}\NormalTok{(terra)) \{}\FunctionTok{install.packages}\NormalTok{(}\StringTok{"terra"}\NormalTok{); }\FunctionTok{require}\NormalTok{(terra)\}}
\ControlFlowTok{if}\NormalTok{(}\SpecialCharTok{!}\FunctionTok{require}\NormalTok{(egvtools)) \{remotes}\SpecialCharTok{::}\FunctionTok{install\_github}\NormalTok{(}\StringTok{"aavotins/egvtools"}\NormalTok{); }\FunctionTok{require}\NormalTok{(egvtools)\}}


\CommentTok{\# Templates {-}{-}{-}{-}{-}}
\NormalTok{template100}\OtherTok{=}\FunctionTok{rast}\NormalTok{(}\StringTok{"./Templates/TemplateRasters/LV100m\_10km.tif"}\NormalTok{)}

\CommentTok{\# radii {-}{-}{-}{-}}
\FunctionTok{radius\_function}\NormalTok{(}
 \AttributeTok{kvadrati\_path =} \StringTok{"./Templates/TemplateGrids/tiles/"}\NormalTok{,}
 \AttributeTok{radii\_path   =} \StringTok{"./Templates/TemplateGridPoints/tiles/"}\NormalTok{,}
 \AttributeTok{tikls100\_path =} \StringTok{"./Templates/TemplateGrids/tikls100\_sauzeme.parquet"}\NormalTok{,}
 \AttributeTok{template\_path =} \StringTok{"./Templates/TemplateRasters/LV100m\_10km.tif"}\NormalTok{,}
 \AttributeTok{input\_layers  =} \FunctionTok{c}\NormalTok{(}\StringTok{"./RasterGrids\_100m/2024/RAW/Edges\_Grasslands\_cell.tif"}\NormalTok{),}
 \AttributeTok{layer\_prefixes =} \FunctionTok{c}\NormalTok{(}\StringTok{"Edges\_Grasslands"}\NormalTok{),}
 \AttributeTok{output\_dir   =} \StringTok{"./RasterGrids\_100m/2024/RAW/"}\NormalTok{,}
 \AttributeTok{n\_workers   =} \DecValTok{12}\NormalTok{,}
 \AttributeTok{radii     =} \FunctionTok{c}\NormalTok{(}\StringTok{"r3000"}\NormalTok{),}
 \AttributeTok{radius\_mode  =} \StringTok{"sparse"}\NormalTok{,}
 \AttributeTok{extract\_fun  =} \StringTok{"sum"}\NormalTok{,}
 \AttributeTok{fill\_missing  =} \ConstantTok{TRUE}\NormalTok{,}
 \AttributeTok{IDW\_weight   =} \DecValTok{2}\NormalTok{,}
 \AttributeTok{future\_max\_size =} \DecValTok{20} \SpecialCharTok{*} \DecValTok{1024}\SpecialCharTok{\^{}}\DecValTok{3}\NormalTok{)}


\CommentTok{\# Edges\_Grasslands\_r3000.tif    egv\_143 {-}{-}{-}{-}}
\NormalTok{slanis}\OtherTok{=}\FunctionTok{rast}\NormalTok{(}\StringTok{"./RasterGrids\_100m/2024/RAW/Edges\_Grasslands\_r3000.tif"}\NormalTok{)}
\FunctionTok{names}\NormalTok{(slanis)}\OtherTok{=}\StringTok{"egv\_143"}
\NormalTok{slanis2}\OtherTok{=}\FunctionTok{project}\NormalTok{(slanis,template100)}
\FunctionTok{writeRaster}\NormalTok{(slanis2,}
      \StringTok{"./RasterGrids\_100m/2024/RAW/Edges\_Grasslands\_r3000.tif"}\NormalTok{,}
      \AttributeTok{overwrite=}\ConstantTok{TRUE}\NormalTok{)}

\CommentTok{\# standardisation {-}{-}{-}{-}}
\ControlFlowTok{if}\NormalTok{(}\SpecialCharTok{!}\FunctionTok{require}\NormalTok{(terra)) \{}\FunctionTok{install.packages}\NormalTok{(}\StringTok{"terra"}\NormalTok{); }\FunctionTok{require}\NormalTok{(terra)\}}
\ControlFlowTok{if}\NormalTok{(}\SpecialCharTok{!}\FunctionTok{require}\NormalTok{(tidyverse)) \{}\FunctionTok{install.packages}\NormalTok{(}\StringTok{"tidyverse"}\NormalTok{); }\FunctionTok{require}\NormalTok{(tidyverse)\}}

\NormalTok{nosaukums}\OtherTok{=}\StringTok{"Edges\_Grasslands\_r3000.tif"}
\NormalTok{ielasisanas\_cels}\OtherTok{=}\FunctionTok{paste0}\NormalTok{(}\StringTok{"./RasterGrids\_100m/2024/RAW/"}\NormalTok{,nosaukums)}
\NormalTok{saglabasanas\_cels}\OtherTok{=}\FunctionTok{paste0}\NormalTok{(}\StringTok{"./RasterGrids\_100m/2024/Scaled/"}\NormalTok{,nosaukums)}
\NormalTok{slanis}\OtherTok{=}\FunctionTok{rast}\NormalTok{(ielasisanas\_cels)}
\NormalTok{videjais}\OtherTok{=}\FunctionTok{global}\NormalTok{(slanis,}\AttributeTok{fun=}\StringTok{"mean"}\NormalTok{,}\AttributeTok{na.rm=}\ConstantTok{TRUE}\NormalTok{)}
\NormalTok{centrets}\OtherTok{=}\NormalTok{slanis}\SpecialCharTok{{-}}\NormalTok{videjais[,}\DecValTok{1}\NormalTok{]}
\NormalTok{standartnovirze}\OtherTok{=}\NormalTok{terra}\SpecialCharTok{::}\FunctionTok{global}\NormalTok{(centrets,}\AttributeTok{fun=}\StringTok{"rms"}\NormalTok{,}\AttributeTok{na.rm=}\ConstantTok{TRUE}\NormalTok{)}
\NormalTok{merogots}\OtherTok{=}\NormalTok{centrets}\SpecialCharTok{/}\NormalTok{standartnovirze[,}\DecValTok{1}\NormalTok{]}
\FunctionTok{writeRaster}\NormalTok{(merogots,}
      \AttributeTok{filename=}\NormalTok{saglabasanas\_cels,}
      \AttributeTok{overwrite=}\ConstantTok{TRUE}\NormalTok{)}
\end{Highlighting}
\end{Shaded}

\section{Edges\_Grasslands\_r10000}\label{ch06.144}

\textbf{filename:} \texttt{Edges\_Grasslands\_r10000.tif}

\textbf{layername:} \texttt{egv\_144}

\textbf{English name:} Edge pixels of Grassland within the 10 km landscape

\textbf{Latvian name:} Zālāju malu pikseļu skaits 10 km ainavā

\textbf{Procedure:} The total edge within a 10000 m radius around the analysis grid cell is
calculated as the area-weighted sum of the \hyperref[ch06.140]{analysis cells} inside the
buffer, using the workflow \texttt{egvtools::radius\_function()}. During the calculation of the landscape metric,
inverse distance weighted (power = 2) gap filling on the output is applied
to ensure no missing values at the edges. Then the layer is rewritten to set
its name. Finally, the layer is standardised by subtracting the arithmetic
mean and dividing by the root mean squared error.

\begin{Shaded}
\begin{Highlighting}[]
\CommentTok{\# libs {-}{-}{-}{-}}
\ControlFlowTok{if}\NormalTok{(}\SpecialCharTok{!}\FunctionTok{require}\NormalTok{(terra)) \{}\FunctionTok{install.packages}\NormalTok{(}\StringTok{"terra"}\NormalTok{); }\FunctionTok{require}\NormalTok{(terra)\}}
\ControlFlowTok{if}\NormalTok{(}\SpecialCharTok{!}\FunctionTok{require}\NormalTok{(egvtools)) \{remotes}\SpecialCharTok{::}\FunctionTok{install\_github}\NormalTok{(}\StringTok{"aavotins/egvtools"}\NormalTok{); }\FunctionTok{require}\NormalTok{(egvtools)\}}


\CommentTok{\# Templates {-}{-}{-}{-}{-}}
\NormalTok{template100}\OtherTok{=}\FunctionTok{rast}\NormalTok{(}\StringTok{"./Templates/TemplateRasters/LV100m\_10km.tif"}\NormalTok{)}

\CommentTok{\# radii {-}{-}{-}{-}}
\FunctionTok{radius\_function}\NormalTok{(}
 \AttributeTok{kvadrati\_path =} \StringTok{"./Templates/TemplateGrids/tiles/"}\NormalTok{,}
 \AttributeTok{radii\_path   =} \StringTok{"./Templates/TemplateGridPoints/tiles/"}\NormalTok{,}
 \AttributeTok{tikls100\_path =} \StringTok{"./Templates/TemplateGrids/tikls100\_sauzeme.parquet"}\NormalTok{,}
 \AttributeTok{template\_path =} \StringTok{"./Templates/TemplateRasters/LV100m\_10km.tif"}\NormalTok{,}
 \AttributeTok{input\_layers  =} \FunctionTok{c}\NormalTok{(}\StringTok{"./RasterGrids\_100m/2024/RAW/Edges\_Grasslands\_cell.tif"}\NormalTok{),}
 \AttributeTok{layer\_prefixes =} \FunctionTok{c}\NormalTok{(}\StringTok{"Edges\_Grasslands"}\NormalTok{),}
 \AttributeTok{output\_dir   =} \StringTok{"./RasterGrids\_100m/2024/RAW/"}\NormalTok{,}
 \AttributeTok{n\_workers   =} \DecValTok{12}\NormalTok{,}
 \AttributeTok{radii     =} \FunctionTok{c}\NormalTok{(}\StringTok{"r10000"}\NormalTok{),}
 \AttributeTok{radius\_mode  =} \StringTok{"sparse"}\NormalTok{,}
 \AttributeTok{extract\_fun  =} \StringTok{"sum"}\NormalTok{,}
 \AttributeTok{fill\_missing  =} \ConstantTok{TRUE}\NormalTok{,}
 \AttributeTok{IDW\_weight   =} \DecValTok{2}\NormalTok{,}
 \AttributeTok{future\_max\_size =} \DecValTok{20} \SpecialCharTok{*} \DecValTok{1024}\SpecialCharTok{\^{}}\DecValTok{3}\NormalTok{)}


\CommentTok{\# Edges\_Grasslands\_r10000.tif   egv\_144 {-}{-}{-}{-}}
\NormalTok{slanis}\OtherTok{=}\FunctionTok{rast}\NormalTok{(}\StringTok{"./RasterGrids\_100m/2024/RAW/Edges\_Grasslands\_r10000.tif"}\NormalTok{)}
\FunctionTok{names}\NormalTok{(slanis)}\OtherTok{=}\StringTok{"egv\_144"}
\NormalTok{slanis2}\OtherTok{=}\FunctionTok{project}\NormalTok{(slanis,template100)}
\FunctionTok{writeRaster}\NormalTok{(slanis2,}
      \StringTok{"./RasterGrids\_100m/2024/RAW/Edges\_Grasslands\_r10000.tif"}\NormalTok{,}
      \AttributeTok{overwrite=}\ConstantTok{TRUE}\NormalTok{)}

\CommentTok{\# standardisation {-}{-}{-}{-}}
\ControlFlowTok{if}\NormalTok{(}\SpecialCharTok{!}\FunctionTok{require}\NormalTok{(terra)) \{}\FunctionTok{install.packages}\NormalTok{(}\StringTok{"terra"}\NormalTok{); }\FunctionTok{require}\NormalTok{(terra)\}}
\ControlFlowTok{if}\NormalTok{(}\SpecialCharTok{!}\FunctionTok{require}\NormalTok{(tidyverse)) \{}\FunctionTok{install.packages}\NormalTok{(}\StringTok{"tidyverse"}\NormalTok{); }\FunctionTok{require}\NormalTok{(tidyverse)\}}

\NormalTok{nosaukums}\OtherTok{=}\StringTok{"Edges\_Grasslands\_r10000.tif"}
\NormalTok{ielasisanas\_cels}\OtherTok{=}\FunctionTok{paste0}\NormalTok{(}\StringTok{"./RasterGrids\_100m/2024/RAW/"}\NormalTok{,nosaukums)}
\NormalTok{saglabasanas\_cels}\OtherTok{=}\FunctionTok{paste0}\NormalTok{(}\StringTok{"./RasterGrids\_100m/2024/Scaled/"}\NormalTok{,nosaukums)}
\NormalTok{slanis}\OtherTok{=}\FunctionTok{rast}\NormalTok{(ielasisanas\_cels)}
\NormalTok{videjais}\OtherTok{=}\FunctionTok{global}\NormalTok{(slanis,}\AttributeTok{fun=}\StringTok{"mean"}\NormalTok{,}\AttributeTok{na.rm=}\ConstantTok{TRUE}\NormalTok{)}
\NormalTok{centrets}\OtherTok{=}\NormalTok{slanis}\SpecialCharTok{{-}}\NormalTok{videjais[,}\DecValTok{1}\NormalTok{]}
\NormalTok{standartnovirze}\OtherTok{=}\NormalTok{terra}\SpecialCharTok{::}\FunctionTok{global}\NormalTok{(centrets,}\AttributeTok{fun=}\StringTok{"rms"}\NormalTok{,}\AttributeTok{na.rm=}\ConstantTok{TRUE}\NormalTok{)}
\NormalTok{merogots}\OtherTok{=}\NormalTok{centrets}\SpecialCharTok{/}\NormalTok{standartnovirze[,}\DecValTok{1}\NormalTok{]}
\FunctionTok{writeRaster}\NormalTok{(merogots,}
      \AttributeTok{filename=}\NormalTok{saglabasanas\_cels,}
      \AttributeTok{overwrite=}\ConstantTok{TRUE}\NormalTok{)}
\end{Highlighting}
\end{Shaded}

\section{Edges\_OldForests\_cell}\label{ch06.145}

\textbf{filename:} \texttt{Edges\_OldForests\_cell.tif}

\textbf{layername:} \texttt{egv\_145}

\textbf{English name:} Edge pixels of Forests Over Rotation Age within the analysis
cell (1 ha)

\textbf{Latvian name:} Pieaugušo un pāraugušo mežaudžu malu pikseļu skaits analīzes šūnā (1
ha)

\textbf{Procedure:} First, the raster layer with forest stands from the \hyperref[Ch04.01]{MVR} at
age groups 4 and 5 is prepared (presence = 1, everything else = NA). Then, the
layer (1 = presence) is covered over the nulls layer (presence = 0) and written to
file (matching the input). Then, using the workflow
\texttt{egvtools::landscape\_function()} total edge between the two classes is
calculated. During the calculation of the landscape metric, inverse distance weighted
(power = 2) gap filling on the output is applied to ensure no missing values
at the edges. Finally, the layer is standardised by subtracting the arithmetic
mean and dividing by the root mean squared error.

\begin{Shaded}
\begin{Highlighting}[]
\CommentTok{\# libs {-}{-}{-}{-}}
\ControlFlowTok{if}\NormalTok{(}\SpecialCharTok{!}\FunctionTok{require}\NormalTok{(terra)) \{}\FunctionTok{install.packages}\NormalTok{(}\StringTok{"terra"}\NormalTok{); }\FunctionTok{require}\NormalTok{(terra)\}}
\ControlFlowTok{if}\NormalTok{(}\SpecialCharTok{!}\FunctionTok{require}\NormalTok{(egvtools)) \{remotes}\SpecialCharTok{::}\FunctionTok{install\_github}\NormalTok{(}\StringTok{"aavotins/egvtools"}\NormalTok{); }\FunctionTok{require}\NormalTok{(egvtools)\}}

\ControlFlowTok{if}\NormalTok{(}\SpecialCharTok{!}\FunctionTok{require}\NormalTok{(sf)) \{}\FunctionTok{install.packages}\NormalTok{(}\StringTok{"sf"}\NormalTok{); }\FunctionTok{require}\NormalTok{(sf)\}}
\ControlFlowTok{if}\NormalTok{(}\SpecialCharTok{!}\FunctionTok{require}\NormalTok{(sfarrow)) \{}\FunctionTok{install.packages}\NormalTok{(}\StringTok{"sfarrow"}\NormalTok{); }\FunctionTok{require}\NormalTok{(sfarrow)\}}
\ControlFlowTok{if}\NormalTok{(}\SpecialCharTok{!}\FunctionTok{require}\NormalTok{(raster)) \{}\FunctionTok{install.packages}\NormalTok{(}\StringTok{"raster"}\NormalTok{); }\FunctionTok{require}\NormalTok{(raster)\}}
\ControlFlowTok{if}\NormalTok{(}\SpecialCharTok{!}\FunctionTok{require}\NormalTok{(fasterize)) \{}\FunctionTok{install.packages}\NormalTok{(}\StringTok{"fasterize"}\NormalTok{); }\FunctionTok{require}\NormalTok{(fasterize)\}}
\ControlFlowTok{if}\NormalTok{(}\SpecialCharTok{!}\FunctionTok{require}\NormalTok{(tidyverse)) \{}\FunctionTok{install.packages}\NormalTok{(}\StringTok{"tidyverse"}\NormalTok{); }\FunctionTok{require}\NormalTok{(tidyverse)\}}


\CommentTok{\# Templates {-}{-}{-}{-}{-}}
\NormalTok{template10}\OtherTok{=}\FunctionTok{rast}\NormalTok{(}\StringTok{"./Templates/TemplateRasters/LV10m\_10km.tif"}\NormalTok{)}
\NormalTok{nulls10}\OtherTok{=}\FunctionTok{rast}\NormalTok{(}\StringTok{"./Templates/TemplateRasters/nulls\_LV10m\_10km.tif"}\NormalTok{)}

\CommentTok{\# simple landscape {-}{-}{-}{-}}
\NormalTok{simple\_landscape}\OtherTok{=}\FunctionTok{rast}\NormalTok{(}\StringTok{"./RasterGrids\_10m/2024/Ainava\_vienk\_mask.tif"}\NormalTok{)}

\CommentTok{\# Edges\_OldForests\_input.tif {-}{-}{-}{-}}
\NormalTok{mvr}\OtherTok{=}\NormalTok{sfarrow}\SpecialCharTok{::}\FunctionTok{st\_read\_parquet}\NormalTok{(}\StringTok{"./Geodata/2024/MVR/nogabali\_2024janv.parquet"}\NormalTok{)}
\NormalTok{mvr2}\OtherTok{=}\NormalTok{mvr }\SpecialCharTok{\%\textgreater{}\%} 
 \FunctionTok{mutate}\NormalTok{(}\AttributeTok{forest\_age=}\FunctionTok{ifelse}\NormalTok{(vgr}\SpecialCharTok{==}\StringTok{"4"}\SpecialCharTok{|}\NormalTok{vgr}\SpecialCharTok{==}\StringTok{"5"}\NormalTok{,}\DecValTok{1}\NormalTok{,}\ConstantTok{NA}\NormalTok{)) }\SpecialCharTok{\%\textgreater{}\%} 
 \FunctionTok{filter}\NormalTok{(}\SpecialCharTok{!}\FunctionTok{is.na}\NormalTok{(forest\_age))}

\NormalTok{rast\_old}\OtherTok{=}\FunctionTok{fasterize}\NormalTok{(mvr2,}\FunctionTok{raster}\NormalTok{(template10),}\AttributeTok{field=}\StringTok{"forest\_age"}\NormalTok{)}
\NormalTok{terra\_old}\OtherTok{=}\FunctionTok{rast}\NormalTok{(rast\_old)}
\FunctionTok{plot}\NormalTok{(terra\_old)}
\NormalTok{terra\_old}\OtherTok{=}\FunctionTok{cover}\NormalTok{(terra\_old,nulls10)}
\FunctionTok{plot}\NormalTok{(terra\_old)}

\NormalTok{edge\_old}\OtherTok{=}\FunctionTok{project}\NormalTok{(terra\_old,template10,}
            \AttributeTok{filename=}\StringTok{"./RasterGrids\_10m/2024/Edges\_OldForests\_input.tif"}\NormalTok{,}
            \AttributeTok{overwrite=}\ConstantTok{TRUE}\NormalTok{)}
\FunctionTok{rm}\NormalTok{(mvr)}
\FunctionTok{rm}\NormalTok{(mvr2)}
\FunctionTok{rm}\NormalTok{(rast\_old)}
\FunctionTok{rm}\NormalTok{(terra\_old)}
\FunctionTok{rm}\NormalTok{(edge\_old)}


\CommentTok{\# Edges\_OldForests\_cell.tif egv\_145 {-}{-}{-}{-}}
\FunctionTok{landscape\_function}\NormalTok{(}
 \AttributeTok{landscape   =} \StringTok{"./RasterGrids\_10m/2024/Edges\_OldForests\_input.tif"}\NormalTok{,}
 \AttributeTok{zones     =} \StringTok{"./Templates/TemplateGrids/tikls100\_sauzeme.parquet"}\NormalTok{,}
 \AttributeTok{id\_field    =} \StringTok{"id"}\NormalTok{,}
 \AttributeTok{tile\_field   =} \StringTok{"tks50km"}\NormalTok{,}
 \AttributeTok{template    =} \StringTok{"./Templates/TemplateRasters/LV100m\_10km.tif"}\NormalTok{,}
 \AttributeTok{out\_dir    =} \StringTok{"./RasterGrids\_100m/2024/RAW"}\NormalTok{,}
 \AttributeTok{out\_filename  =} \StringTok{"Edges\_OldForests\_cell.tif"}\NormalTok{,}
 \AttributeTok{out\_layername =} \StringTok{"egv\_145"}\NormalTok{,}
 \AttributeTok{what       =} \StringTok{"lsm\_l\_te"}\NormalTok{,}
 \AttributeTok{lm\_args     =} \FunctionTok{list}\NormalTok{(}\AttributeTok{count\_boundary =} \ConstantTok{FALSE}\NormalTok{),}
 \AttributeTok{rasterize\_engine =} \StringTok{"fasterize"}\NormalTok{,}
 \AttributeTok{n\_workers   =} \DecValTok{12}\NormalTok{,}
 \AttributeTok{future\_max\_size =} \DecValTok{20} \SpecialCharTok{*} \DecValTok{1024}\SpecialCharTok{\^{}}\DecValTok{3}\NormalTok{,}
 \AttributeTok{fill\_gaps   =} \ConstantTok{TRUE}\NormalTok{,}
 \AttributeTok{plot\_gaps   =} \ConstantTok{FALSE}\NormalTok{,}
 \AttributeTok{plot\_result  =} \ConstantTok{FALSE}
\NormalTok{)}

\CommentTok{\# standardisation {-}{-}{-}{-}}
\ControlFlowTok{if}\NormalTok{(}\SpecialCharTok{!}\FunctionTok{require}\NormalTok{(terra)) \{}\FunctionTok{install.packages}\NormalTok{(}\StringTok{"terra"}\NormalTok{); }\FunctionTok{require}\NormalTok{(terra)\}}
\ControlFlowTok{if}\NormalTok{(}\SpecialCharTok{!}\FunctionTok{require}\NormalTok{(tidyverse)) \{}\FunctionTok{install.packages}\NormalTok{(}\StringTok{"tidyverse"}\NormalTok{); }\FunctionTok{require}\NormalTok{(tidyverse)\}}

\NormalTok{nosaukums}\OtherTok{=}\StringTok{"Edges\_OldForests\_cell.tif"}
\NormalTok{ielasisanas\_cels}\OtherTok{=}\FunctionTok{paste0}\NormalTok{(}\StringTok{"./RasterGrids\_100m/2024/RAW/"}\NormalTok{,nosaukums)}
\NormalTok{saglabasanas\_cels}\OtherTok{=}\FunctionTok{paste0}\NormalTok{(}\StringTok{"./RasterGrids\_100m/2024/Scaled/"}\NormalTok{,nosaukums)}
\NormalTok{slanis}\OtherTok{=}\FunctionTok{rast}\NormalTok{(ielasisanas\_cels)}
\NormalTok{videjais}\OtherTok{=}\FunctionTok{global}\NormalTok{(slanis,}\AttributeTok{fun=}\StringTok{"mean"}\NormalTok{,}\AttributeTok{na.rm=}\ConstantTok{TRUE}\NormalTok{)}
\NormalTok{centrets}\OtherTok{=}\NormalTok{slanis}\SpecialCharTok{{-}}\NormalTok{videjais[,}\DecValTok{1}\NormalTok{]}
\NormalTok{standartnovirze}\OtherTok{=}\NormalTok{terra}\SpecialCharTok{::}\FunctionTok{global}\NormalTok{(centrets,}\AttributeTok{fun=}\StringTok{"rms"}\NormalTok{,}\AttributeTok{na.rm=}\ConstantTok{TRUE}\NormalTok{)}
\NormalTok{merogots}\OtherTok{=}\NormalTok{centrets}\SpecialCharTok{/}\NormalTok{standartnovirze[,}\DecValTok{1}\NormalTok{]}
\FunctionTok{writeRaster}\NormalTok{(merogots,}
      \AttributeTok{filename=}\NormalTok{saglabasanas\_cels,}
      \AttributeTok{overwrite=}\ConstantTok{TRUE}\NormalTok{)}
\end{Highlighting}
\end{Shaded}

\section{Edges\_OldForests\_r500}\label{ch06.146}

\textbf{filename:} \texttt{Edges\_OldForests\_r500.tif}

\textbf{layername:} \texttt{egv\_146}

\textbf{English name:} Edge pixels of Forests Over Rotation Age within the 0.5 km
landscape

\textbf{Latvian name:} Pieaugušo un pāraugušo mežaudžu malu pikseļu skaits 0,5 km ainavā

\textbf{Procedure:} The total edge within a 500 m radius around the analysis grid cell is
calculated as the area-weighted sum of the \hyperref[ch06.145]{analysis cells} inside the
buffer, using the workflow \texttt{egvtools::radius\_function()}. During the calculation of the landscape metric,
inverse distance weighted (power = 2) gap filling on the output is applied
to ensure no missing values at the edges. Then the layer is rewritten to set
its name. Finally, the layer is standardised by subtracting the arithmetic
mean and dividing by the root mean squared error.

\begin{Shaded}
\begin{Highlighting}[]
\CommentTok{\# libs {-}{-}{-}{-}}
\ControlFlowTok{if}\NormalTok{(}\SpecialCharTok{!}\FunctionTok{require}\NormalTok{(terra)) \{}\FunctionTok{install.packages}\NormalTok{(}\StringTok{"terra"}\NormalTok{); }\FunctionTok{require}\NormalTok{(terra)\}}
\ControlFlowTok{if}\NormalTok{(}\SpecialCharTok{!}\FunctionTok{require}\NormalTok{(egvtools)) \{remotes}\SpecialCharTok{::}\FunctionTok{install\_github}\NormalTok{(}\StringTok{"aavotins/egvtools"}\NormalTok{); }\FunctionTok{require}\NormalTok{(egvtools)\}}


\CommentTok{\# Templates {-}{-}{-}{-}{-}}
\NormalTok{template100}\OtherTok{=}\FunctionTok{rast}\NormalTok{(}\StringTok{"./Templates/TemplateRasters/LV100m\_10km.tif"}\NormalTok{)}

\CommentTok{\# radii {-}{-}{-}{-}}
\FunctionTok{radius\_function}\NormalTok{(}
 \AttributeTok{kvadrati\_path =} \StringTok{"./Templates/TemplateGrids/tiles/"}\NormalTok{,}
 \AttributeTok{radii\_path   =} \StringTok{"./Templates/TemplateGridPoints/tiles/"}\NormalTok{,}
 \AttributeTok{tikls100\_path =} \StringTok{"./Templates/TemplateGrids/tikls100\_sauzeme.parquet"}\NormalTok{,}
 \AttributeTok{template\_path =} \StringTok{"./Templates/TemplateRasters/LV100m\_10km.tif"}\NormalTok{,}
 \AttributeTok{input\_layers  =} \FunctionTok{c}\NormalTok{(}\StringTok{"./RasterGrids\_100m/2024/RAW/Edges\_OldForests\_cell.tif"}\NormalTok{),}
 \AttributeTok{layer\_prefixes =} \FunctionTok{c}\NormalTok{(}\StringTok{"Edges\_OldForests"}\NormalTok{),}
 \AttributeTok{output\_dir   =} \StringTok{"./RasterGrids\_100m/2024/RAW/"}\NormalTok{,}
 \AttributeTok{n\_workers   =} \DecValTok{12}\NormalTok{,}
 \AttributeTok{radii     =} \FunctionTok{c}\NormalTok{(}\StringTok{"r500"}\NormalTok{),}
 \AttributeTok{radius\_mode  =} \StringTok{"sparse"}\NormalTok{,}
 \AttributeTok{extract\_fun  =} \StringTok{"sum"}\NormalTok{,}
 \AttributeTok{fill\_missing  =} \ConstantTok{TRUE}\NormalTok{,}
 \AttributeTok{IDW\_weight   =} \DecValTok{2}\NormalTok{,}
 \AttributeTok{future\_max\_size =} \DecValTok{20} \SpecialCharTok{*} \DecValTok{1024}\SpecialCharTok{\^{}}\DecValTok{3}\NormalTok{)}


\CommentTok{\# Edges\_OldForests\_r500.tif egv\_146 {-}{-}{-}{-}}
\NormalTok{slanis}\OtherTok{=}\FunctionTok{rast}\NormalTok{(}\StringTok{"./RasterGrids\_100m/2024/RAW/Edges\_OldForests\_r500.tif"}\NormalTok{)}
\FunctionTok{names}\NormalTok{(slanis)}\OtherTok{=}\StringTok{"egv\_146"}
\NormalTok{slanis2}\OtherTok{=}\FunctionTok{project}\NormalTok{(slanis,template100)}
\FunctionTok{writeRaster}\NormalTok{(slanis2,}
      \StringTok{"./RasterGrids\_100m/2024/RAW/Edges\_OldForests\_r500.tif"}\NormalTok{,}
      \AttributeTok{overwrite=}\ConstantTok{TRUE}\NormalTok{)}

\CommentTok{\# standardisation {-}{-}{-}{-}}
\ControlFlowTok{if}\NormalTok{(}\SpecialCharTok{!}\FunctionTok{require}\NormalTok{(terra)) \{}\FunctionTok{install.packages}\NormalTok{(}\StringTok{"terra"}\NormalTok{); }\FunctionTok{require}\NormalTok{(terra)\}}
\ControlFlowTok{if}\NormalTok{(}\SpecialCharTok{!}\FunctionTok{require}\NormalTok{(tidyverse)) \{}\FunctionTok{install.packages}\NormalTok{(}\StringTok{"tidyverse"}\NormalTok{); }\FunctionTok{require}\NormalTok{(tidyverse)\}}

\NormalTok{nosaukums}\OtherTok{=}\StringTok{"Edges\_OldForests\_r500.tif"}
\NormalTok{ielasisanas\_cels}\OtherTok{=}\FunctionTok{paste0}\NormalTok{(}\StringTok{"./RasterGrids\_100m/2024/RAW/"}\NormalTok{,nosaukums)}
\NormalTok{saglabasanas\_cels}\OtherTok{=}\FunctionTok{paste0}\NormalTok{(}\StringTok{"./RasterGrids\_100m/2024/Scaled/"}\NormalTok{,nosaukums)}
\NormalTok{slanis}\OtherTok{=}\FunctionTok{rast}\NormalTok{(ielasisanas\_cels)}
\NormalTok{videjais}\OtherTok{=}\FunctionTok{global}\NormalTok{(slanis,}\AttributeTok{fun=}\StringTok{"mean"}\NormalTok{,}\AttributeTok{na.rm=}\ConstantTok{TRUE}\NormalTok{)}
\NormalTok{centrets}\OtherTok{=}\NormalTok{slanis}\SpecialCharTok{{-}}\NormalTok{videjais[,}\DecValTok{1}\NormalTok{]}
\NormalTok{standartnovirze}\OtherTok{=}\NormalTok{terra}\SpecialCharTok{::}\FunctionTok{global}\NormalTok{(centrets,}\AttributeTok{fun=}\StringTok{"rms"}\NormalTok{,}\AttributeTok{na.rm=}\ConstantTok{TRUE}\NormalTok{)}
\NormalTok{merogots}\OtherTok{=}\NormalTok{centrets}\SpecialCharTok{/}\NormalTok{standartnovirze[,}\DecValTok{1}\NormalTok{]}
\FunctionTok{writeRaster}\NormalTok{(merogots,}
      \AttributeTok{filename=}\NormalTok{saglabasanas\_cels,}
      \AttributeTok{overwrite=}\ConstantTok{TRUE}\NormalTok{)}
\end{Highlighting}
\end{Shaded}

\section{Edges\_OldForests\_r1250}\label{ch06.147}

\textbf{filename:} \texttt{Edges\_OldForests\_r1250.tif}

\textbf{layername:} \texttt{egv\_147}

\textbf{English name:} Edge pixels of Forests Over Rotation Age within the 1.25 km
landscape

\textbf{Latvian name:} Pieaugušo un pāraugušo mežaudžu malu pikseļu skaits 1,25 km ainavā

\textbf{Procedure:} The total edge within a 1250 m radius around the analysis grid cell is
calculated as the area-weighted sum of the \hyperref[ch06.145]{analysis cells} inside the
buffer, using the workflow \texttt{egvtools::radius\_function()}. During the calculation of the landscape metric,
inverse distance weighted (power = 2) gap filling on the output is applied
to ensure no missing values at the edges. Then the layer is rewritten to set
its name. Finally, the layer is standardised by subtracting the arithmetic
mean and dividing by the root mean squared error.

\begin{Shaded}
\begin{Highlighting}[]
\CommentTok{\# libs {-}{-}{-}{-}}
\ControlFlowTok{if}\NormalTok{(}\SpecialCharTok{!}\FunctionTok{require}\NormalTok{(terra)) \{}\FunctionTok{install.packages}\NormalTok{(}\StringTok{"terra"}\NormalTok{); }\FunctionTok{require}\NormalTok{(terra)\}}
\ControlFlowTok{if}\NormalTok{(}\SpecialCharTok{!}\FunctionTok{require}\NormalTok{(egvtools)) \{remotes}\SpecialCharTok{::}\FunctionTok{install\_github}\NormalTok{(}\StringTok{"aavotins/egvtools"}\NormalTok{); }\FunctionTok{require}\NormalTok{(egvtools)\}}


\CommentTok{\# Templates {-}{-}{-}{-}{-}}
\NormalTok{template100}\OtherTok{=}\FunctionTok{rast}\NormalTok{(}\StringTok{"./Templates/TemplateRasters/LV100m\_10km.tif"}\NormalTok{)}

\CommentTok{\# radii {-}{-}{-}{-}}
\FunctionTok{radius\_function}\NormalTok{(}
 \AttributeTok{kvadrati\_path =} \StringTok{"./Templates/TemplateGrids/tiles/"}\NormalTok{,}
 \AttributeTok{radii\_path   =} \StringTok{"./Templates/TemplateGridPoints/tiles/"}\NormalTok{,}
 \AttributeTok{tikls100\_path =} \StringTok{"./Templates/TemplateGrids/tikls100\_sauzeme.parquet"}\NormalTok{,}
 \AttributeTok{template\_path =} \StringTok{"./Templates/TemplateRasters/LV100m\_10km.tif"}\NormalTok{,}
 \AttributeTok{input\_layers  =} \FunctionTok{c}\NormalTok{(}\StringTok{"./RasterGrids\_100m/2024/RAW/Edges\_OldForests\_cell.tif"}\NormalTok{),}
 \AttributeTok{layer\_prefixes =} \FunctionTok{c}\NormalTok{(}\StringTok{"Edges\_OldForests"}\NormalTok{),}
 \AttributeTok{output\_dir   =} \StringTok{"./RasterGrids\_100m/2024/RAW/"}\NormalTok{,}
 \AttributeTok{n\_workers   =} \DecValTok{12}\NormalTok{,}
 \AttributeTok{radii     =} \FunctionTok{c}\NormalTok{(}\StringTok{"r1250"}\NormalTok{),}
 \AttributeTok{radius\_mode  =} \StringTok{"sparse"}\NormalTok{,}
 \AttributeTok{extract\_fun  =} \StringTok{"sum"}\NormalTok{,}
 \AttributeTok{fill\_missing  =} \ConstantTok{TRUE}\NormalTok{,}
 \AttributeTok{IDW\_weight   =} \DecValTok{2}\NormalTok{,}
 \AttributeTok{future\_max\_size =} \DecValTok{20} \SpecialCharTok{*} \DecValTok{1024}\SpecialCharTok{\^{}}\DecValTok{3}\NormalTok{)}


\CommentTok{\# Edges\_OldForests\_r1250.tif    egv\_147 {-}{-}{-}{-}}
\NormalTok{slanis}\OtherTok{=}\FunctionTok{rast}\NormalTok{(}\StringTok{"./RasterGrids\_100m/2024/RAW/Edges\_OldForests\_r1250.tif"}\NormalTok{)}
\FunctionTok{names}\NormalTok{(slanis)}\OtherTok{=}\StringTok{"egv\_147"}
\NormalTok{slanis2}\OtherTok{=}\FunctionTok{project}\NormalTok{(slanis,template100)}
\FunctionTok{writeRaster}\NormalTok{(slanis2,}
      \StringTok{"./RasterGrids\_100m/2024/RAW/Edges\_OldForests\_r1250.tif"}\NormalTok{,}
      \AttributeTok{overwrite=}\ConstantTok{TRUE}\NormalTok{)}

\CommentTok{\# standardisation {-}{-}{-}{-}}
\ControlFlowTok{if}\NormalTok{(}\SpecialCharTok{!}\FunctionTok{require}\NormalTok{(terra)) \{}\FunctionTok{install.packages}\NormalTok{(}\StringTok{"terra"}\NormalTok{); }\FunctionTok{require}\NormalTok{(terra)\}}
\ControlFlowTok{if}\NormalTok{(}\SpecialCharTok{!}\FunctionTok{require}\NormalTok{(tidyverse)) \{}\FunctionTok{install.packages}\NormalTok{(}\StringTok{"tidyverse"}\NormalTok{); }\FunctionTok{require}\NormalTok{(tidyverse)\}}

\NormalTok{nosaukums}\OtherTok{=}\StringTok{"Edges\_OldForests\_r1250.tif"}
\NormalTok{ielasisanas\_cels}\OtherTok{=}\FunctionTok{paste0}\NormalTok{(}\StringTok{"./RasterGrids\_100m/2024/RAW/"}\NormalTok{,nosaukums)}
\NormalTok{saglabasanas\_cels}\OtherTok{=}\FunctionTok{paste0}\NormalTok{(}\StringTok{"./RasterGrids\_100m/2024/Scaled/"}\NormalTok{,nosaukums)}
\NormalTok{slanis}\OtherTok{=}\FunctionTok{rast}\NormalTok{(ielasisanas\_cels)}
\NormalTok{videjais}\OtherTok{=}\FunctionTok{global}\NormalTok{(slanis,}\AttributeTok{fun=}\StringTok{"mean"}\NormalTok{,}\AttributeTok{na.rm=}\ConstantTok{TRUE}\NormalTok{)}
\NormalTok{centrets}\OtherTok{=}\NormalTok{slanis}\SpecialCharTok{{-}}\NormalTok{videjais[,}\DecValTok{1}\NormalTok{]}
\NormalTok{standartnovirze}\OtherTok{=}\NormalTok{terra}\SpecialCharTok{::}\FunctionTok{global}\NormalTok{(centrets,}\AttributeTok{fun=}\StringTok{"rms"}\NormalTok{,}\AttributeTok{na.rm=}\ConstantTok{TRUE}\NormalTok{)}
\NormalTok{merogots}\OtherTok{=}\NormalTok{centrets}\SpecialCharTok{/}\NormalTok{standartnovirze[,}\DecValTok{1}\NormalTok{]}
\FunctionTok{writeRaster}\NormalTok{(merogots,}
      \AttributeTok{filename=}\NormalTok{saglabasanas\_cels,}
      \AttributeTok{overwrite=}\ConstantTok{TRUE}\NormalTok{)}
\end{Highlighting}
\end{Shaded}

\section{Edges\_OldForests\_r3000}\label{ch06.148}

\textbf{filename:} \texttt{Edges\_OldForests\_r3000.tif}

\textbf{layername:} \texttt{egv\_148}

\textbf{English name:} Edge pixels of Forests Over Rotation Age within the 3 km
landscape

\textbf{Latvian name:} Pieaugušo un pāraugušo mežaudžu malu pikseļu skaits 3 km ainavā

\textbf{Procedure:} The total edge within a 3000 m radius around the analysis grid cell is
calculated as the area-weighted sum of the \hyperref[ch06.145]{analysis cells} inside the
buffer, using the workflow \texttt{egvtools::radius\_function()}. During the calculation of the landscape metric,
inverse distance weighted (power = 2) gap filling on the output is applied
to ensure no missing values at the edges. Then the layer is rewritten to set
its name. Finally, the layer is standardised by subtracting the arithmetic
mean and dividing by the root mean squared error.

\begin{Shaded}
\begin{Highlighting}[]
\CommentTok{\# libs {-}{-}{-}{-}}
\ControlFlowTok{if}\NormalTok{(}\SpecialCharTok{!}\FunctionTok{require}\NormalTok{(terra)) \{}\FunctionTok{install.packages}\NormalTok{(}\StringTok{"terra"}\NormalTok{); }\FunctionTok{require}\NormalTok{(terra)\}}
\ControlFlowTok{if}\NormalTok{(}\SpecialCharTok{!}\FunctionTok{require}\NormalTok{(egvtools)) \{remotes}\SpecialCharTok{::}\FunctionTok{install\_github}\NormalTok{(}\StringTok{"aavotins/egvtools"}\NormalTok{); }\FunctionTok{require}\NormalTok{(egvtools)\}}


\CommentTok{\# Templates {-}{-}{-}{-}{-}}
\NormalTok{template100}\OtherTok{=}\FunctionTok{rast}\NormalTok{(}\StringTok{"./Templates/TemplateRasters/LV100m\_10km.tif"}\NormalTok{)}

\CommentTok{\# radii {-}{-}{-}{-}}
\FunctionTok{radius\_function}\NormalTok{(}
 \AttributeTok{kvadrati\_path =} \StringTok{"./Templates/TemplateGrids/tiles/"}\NormalTok{,}
 \AttributeTok{radii\_path   =} \StringTok{"./Templates/TemplateGridPoints/tiles/"}\NormalTok{,}
 \AttributeTok{tikls100\_path =} \StringTok{"./Templates/TemplateGrids/tikls100\_sauzeme.parquet"}\NormalTok{,}
 \AttributeTok{template\_path =} \StringTok{"./Templates/TemplateRasters/LV100m\_10km.tif"}\NormalTok{,}
 \AttributeTok{input\_layers  =} \FunctionTok{c}\NormalTok{(}\StringTok{"./RasterGrids\_100m/2024/RAW/Edges\_OldForests\_cell.tif"}\NormalTok{),}
 \AttributeTok{layer\_prefixes =} \FunctionTok{c}\NormalTok{(}\StringTok{"Edges\_OldForests"}\NormalTok{),}
 \AttributeTok{output\_dir   =} \StringTok{"./RasterGrids\_100m/2024/RAW/"}\NormalTok{,}
 \AttributeTok{n\_workers   =} \DecValTok{12}\NormalTok{,}
 \AttributeTok{radii     =} \FunctionTok{c}\NormalTok{(}\StringTok{"r3000"}\NormalTok{),}
 \AttributeTok{radius\_mode  =} \StringTok{"sparse"}\NormalTok{,}
 \AttributeTok{extract\_fun  =} \StringTok{"sum"}\NormalTok{,}
 \AttributeTok{fill\_missing  =} \ConstantTok{TRUE}\NormalTok{,}
 \AttributeTok{IDW\_weight   =} \DecValTok{2}\NormalTok{,}
 \AttributeTok{future\_max\_size =} \DecValTok{20} \SpecialCharTok{*} \DecValTok{1024}\SpecialCharTok{\^{}}\DecValTok{3}\NormalTok{)}


\CommentTok{\# Edges\_OldForests\_r3000.tif    egv\_148 {-}{-}{-}{-}}
\NormalTok{slanis}\OtherTok{=}\FunctionTok{rast}\NormalTok{(}\StringTok{"./RasterGrids\_100m/2024/RAW/Edges\_OldForests\_r3000.tif"}\NormalTok{)}
\FunctionTok{names}\NormalTok{(slanis)}\OtherTok{=}\StringTok{"egv\_148"}
\NormalTok{slanis2}\OtherTok{=}\FunctionTok{project}\NormalTok{(slanis,template100)}
\FunctionTok{writeRaster}\NormalTok{(slanis2,}
      \StringTok{"./RasterGrids\_100m/2024/RAW/Edges\_OldForests\_r3000.tif"}\NormalTok{,}
      \AttributeTok{overwrite=}\ConstantTok{TRUE}\NormalTok{)}

\CommentTok{\# standardisation {-}{-}{-}{-}}
\ControlFlowTok{if}\NormalTok{(}\SpecialCharTok{!}\FunctionTok{require}\NormalTok{(terra)) \{}\FunctionTok{install.packages}\NormalTok{(}\StringTok{"terra"}\NormalTok{); }\FunctionTok{require}\NormalTok{(terra)\}}
\ControlFlowTok{if}\NormalTok{(}\SpecialCharTok{!}\FunctionTok{require}\NormalTok{(tidyverse)) \{}\FunctionTok{install.packages}\NormalTok{(}\StringTok{"tidyverse"}\NormalTok{); }\FunctionTok{require}\NormalTok{(tidyverse)\}}

\NormalTok{nosaukums}\OtherTok{=}\StringTok{"Edges\_OldForests\_r3000.tif"}
\NormalTok{ielasisanas\_cels}\OtherTok{=}\FunctionTok{paste0}\NormalTok{(}\StringTok{"./RasterGrids\_100m/2024/RAW/"}\NormalTok{,nosaukums)}
\NormalTok{saglabasanas\_cels}\OtherTok{=}\FunctionTok{paste0}\NormalTok{(}\StringTok{"./RasterGrids\_100m/2024/Scaled/"}\NormalTok{,nosaukums)}
\NormalTok{slanis}\OtherTok{=}\FunctionTok{rast}\NormalTok{(ielasisanas\_cels)}
\NormalTok{videjais}\OtherTok{=}\FunctionTok{global}\NormalTok{(slanis,}\AttributeTok{fun=}\StringTok{"mean"}\NormalTok{,}\AttributeTok{na.rm=}\ConstantTok{TRUE}\NormalTok{)}
\NormalTok{centrets}\OtherTok{=}\NormalTok{slanis}\SpecialCharTok{{-}}\NormalTok{videjais[,}\DecValTok{1}\NormalTok{]}
\NormalTok{standartnovirze}\OtherTok{=}\NormalTok{terra}\SpecialCharTok{::}\FunctionTok{global}\NormalTok{(centrets,}\AttributeTok{fun=}\StringTok{"rms"}\NormalTok{,}\AttributeTok{na.rm=}\ConstantTok{TRUE}\NormalTok{)}
\NormalTok{merogots}\OtherTok{=}\NormalTok{centrets}\SpecialCharTok{/}\NormalTok{standartnovirze[,}\DecValTok{1}\NormalTok{]}
\FunctionTok{writeRaster}\NormalTok{(merogots,}
      \AttributeTok{filename=}\NormalTok{saglabasanas\_cels,}
      \AttributeTok{overwrite=}\ConstantTok{TRUE}\NormalTok{)}
\end{Highlighting}
\end{Shaded}

\section{Edges\_OldForests\_r10000}\label{ch06.149}

\textbf{filename:} \texttt{Edges\_OldForests\_r10000.tif}

\textbf{layername:} \texttt{egv\_149}

\textbf{English name:} Edge pixels of Forests Over Rotation Age within the 10 km
landscape

\textbf{Latvian name:} Pieaugušo un pāraugušo mežaudžu malu pikseļu skaits 10 km ainavā

\textbf{Procedure:} The total edge within a 10000 m radius around the analysis grid cell is
calculated as the area-weighted sum of the \hyperref[ch06.145]{analysis cells} inside the
buffer, using the workflow \texttt{egvtools::radius\_function()}. During the calculation of the landscape metric,
inverse distance weighted (power = 2) gap filling on the output is applied
to ensure no missing values at the edges. Then the layer is rewritten to set
its name. Finally, the layer is standardised by subtracting the arithmetic
mean and dividing by the root mean squared error.

\begin{Shaded}
\begin{Highlighting}[]
\CommentTok{\# libs {-}{-}{-}{-}}
\ControlFlowTok{if}\NormalTok{(}\SpecialCharTok{!}\FunctionTok{require}\NormalTok{(terra)) \{}\FunctionTok{install.packages}\NormalTok{(}\StringTok{"terra"}\NormalTok{); }\FunctionTok{require}\NormalTok{(terra)\}}
\ControlFlowTok{if}\NormalTok{(}\SpecialCharTok{!}\FunctionTok{require}\NormalTok{(egvtools)) \{remotes}\SpecialCharTok{::}\FunctionTok{install\_github}\NormalTok{(}\StringTok{"aavotins/egvtools"}\NormalTok{); }\FunctionTok{require}\NormalTok{(egvtools)\}}


\CommentTok{\# Templates {-}{-}{-}{-}{-}}
\NormalTok{template100}\OtherTok{=}\FunctionTok{rast}\NormalTok{(}\StringTok{"./Templates/TemplateRasters/LV100m\_10km.tif"}\NormalTok{)}

\CommentTok{\# radii {-}{-}{-}{-}}
\FunctionTok{radius\_function}\NormalTok{(}
 \AttributeTok{kvadrati\_path =} \StringTok{"./Templates/TemplateGrids/tiles/"}\NormalTok{,}
 \AttributeTok{radii\_path   =} \StringTok{"./Templates/TemplateGridPoints/tiles/"}\NormalTok{,}
 \AttributeTok{tikls100\_path =} \StringTok{"./Templates/TemplateGrids/tikls100\_sauzeme.parquet"}\NormalTok{,}
 \AttributeTok{template\_path =} \StringTok{"./Templates/TemplateRasters/LV100m\_10km.tif"}\NormalTok{,}
 \AttributeTok{input\_layers  =} \FunctionTok{c}\NormalTok{(}\StringTok{"./RasterGrids\_100m/2024/RAW/Edges\_OldForests\_cell.tif"}\NormalTok{),}
 \AttributeTok{layer\_prefixes =} \FunctionTok{c}\NormalTok{(}\StringTok{"Edges\_OldForests"}\NormalTok{),}
 \AttributeTok{output\_dir   =} \StringTok{"./RasterGrids\_100m/2024/RAW/"}\NormalTok{,}
 \AttributeTok{n\_workers   =} \DecValTok{12}\NormalTok{,}
 \AttributeTok{radii     =} \FunctionTok{c}\NormalTok{(}\StringTok{"r10000"}\NormalTok{),}
 \AttributeTok{radius\_mode  =} \StringTok{"sparse"}\NormalTok{,}
 \AttributeTok{extract\_fun  =} \StringTok{"sum"}\NormalTok{,}
 \AttributeTok{fill\_missing  =} \ConstantTok{TRUE}\NormalTok{,}
 \AttributeTok{IDW\_weight   =} \DecValTok{2}\NormalTok{,}
 \AttributeTok{future\_max\_size =} \DecValTok{20} \SpecialCharTok{*} \DecValTok{1024}\SpecialCharTok{\^{}}\DecValTok{3}\NormalTok{)}


\CommentTok{\# Edges\_OldForests\_r10000.tif   egv\_149 {-}{-}{-}{-}}
\NormalTok{slanis}\OtherTok{=}\FunctionTok{rast}\NormalTok{(}\StringTok{"./RasterGrids\_100m/2024/RAW/Edges\_OldForests\_r10000.tif"}\NormalTok{)}
\FunctionTok{names}\NormalTok{(slanis)}\OtherTok{=}\StringTok{"egv\_149"}
\NormalTok{slanis2}\OtherTok{=}\FunctionTok{project}\NormalTok{(slanis,template100)}
\FunctionTok{writeRaster}\NormalTok{(slanis2,}
      \StringTok{"./RasterGrids\_100m/2024/RAW/Edges\_OldForests\_r10000.tif"}\NormalTok{,}
      \AttributeTok{overwrite=}\ConstantTok{TRUE}\NormalTok{)}

\CommentTok{\# standardisation {-}{-}{-}{-}}
\ControlFlowTok{if}\NormalTok{(}\SpecialCharTok{!}\FunctionTok{require}\NormalTok{(terra)) \{}\FunctionTok{install.packages}\NormalTok{(}\StringTok{"terra"}\NormalTok{); }\FunctionTok{require}\NormalTok{(terra)\}}
\ControlFlowTok{if}\NormalTok{(}\SpecialCharTok{!}\FunctionTok{require}\NormalTok{(tidyverse)) \{}\FunctionTok{install.packages}\NormalTok{(}\StringTok{"tidyverse"}\NormalTok{); }\FunctionTok{require}\NormalTok{(tidyverse)\}}

\NormalTok{nosaukums}\OtherTok{=}\StringTok{"Edges\_OldForests\_r10000.tif"}
\NormalTok{ielasisanas\_cels}\OtherTok{=}\FunctionTok{paste0}\NormalTok{(}\StringTok{"./RasterGrids\_100m/2024/RAW/"}\NormalTok{,nosaukums)}
\NormalTok{saglabasanas\_cels}\OtherTok{=}\FunctionTok{paste0}\NormalTok{(}\StringTok{"./RasterGrids\_100m/2024/Scaled/"}\NormalTok{,nosaukums)}
\NormalTok{slanis}\OtherTok{=}\FunctionTok{rast}\NormalTok{(ielasisanas\_cels)}
\NormalTok{videjais}\OtherTok{=}\FunctionTok{global}\NormalTok{(slanis,}\AttributeTok{fun=}\StringTok{"mean"}\NormalTok{,}\AttributeTok{na.rm=}\ConstantTok{TRUE}\NormalTok{)}
\NormalTok{centrets}\OtherTok{=}\NormalTok{slanis}\SpecialCharTok{{-}}\NormalTok{videjais[,}\DecValTok{1}\NormalTok{]}
\NormalTok{standartnovirze}\OtherTok{=}\NormalTok{terra}\SpecialCharTok{::}\FunctionTok{global}\NormalTok{(centrets,}\AttributeTok{fun=}\StringTok{"rms"}\NormalTok{,}\AttributeTok{na.rm=}\ConstantTok{TRUE}\NormalTok{)}
\NormalTok{merogots}\OtherTok{=}\NormalTok{centrets}\SpecialCharTok{/}\NormalTok{standartnovirze[,}\DecValTok{1}\NormalTok{]}
\FunctionTok{writeRaster}\NormalTok{(merogots,}
      \AttributeTok{filename=}\NormalTok{saglabasanas\_cels,}
      \AttributeTok{overwrite=}\ConstantTok{TRUE}\NormalTok{)}
\end{Highlighting}
\end{Shaded}

\section{Edges\_Roads\_cell}\label{ch06.150}

\textbf{filename:} \texttt{Edges\_Roads\_cell.tif}

\textbf{layername:} \texttt{egv\_150}

\textbf{English name:} Edge pixels of Roads within the analysis cell (1 ha)

\textbf{Latvian name:} Ceļu malu pikseļu skaits analīzes šūnā (1 ha)

\textbf{Procedure:} First, values equal to 100 from the \hyperref[Ch05.03]{Landscape
classification} are coded as 1, and other values as NA. Then, the
layer (1 = presence) is covered over the nulls layer (presence = 0) and written to
file (matching the input). Next, with the workflow
\texttt{egvtools::landscape\_function()} total edge between the two classes is
calculated. During the calculation of the landscape metric, inverse distance weighted
(power = 2) gap filling on the output is applied to ensure no missing values
at the edges. Finally, the layer is standardised by subtracting the arithmetic
mean and dividing by the root mean squared error.

\begin{Shaded}
\begin{Highlighting}[]
\CommentTok{\# libs {-}{-}{-}{-}}
\ControlFlowTok{if}\NormalTok{(}\SpecialCharTok{!}\FunctionTok{require}\NormalTok{(terra)) \{}\FunctionTok{install.packages}\NormalTok{(}\StringTok{"terra"}\NormalTok{); }\FunctionTok{require}\NormalTok{(terra)\}}
\ControlFlowTok{if}\NormalTok{(}\SpecialCharTok{!}\FunctionTok{require}\NormalTok{(egvtools)) \{remotes}\SpecialCharTok{::}\FunctionTok{install\_github}\NormalTok{(}\StringTok{"aavotins/egvtools"}\NormalTok{); }\FunctionTok{require}\NormalTok{(egvtools)\}}

\ControlFlowTok{if}\NormalTok{(}\SpecialCharTok{!}\FunctionTok{require}\NormalTok{(sf)) \{}\FunctionTok{install.packages}\NormalTok{(}\StringTok{"sf"}\NormalTok{); }\FunctionTok{require}\NormalTok{(sf)\}}
\ControlFlowTok{if}\NormalTok{(}\SpecialCharTok{!}\FunctionTok{require}\NormalTok{(sfarrow)) \{}\FunctionTok{install.packages}\NormalTok{(}\StringTok{"sfarrow"}\NormalTok{); }\FunctionTok{require}\NormalTok{(sfarrow)\}}
\ControlFlowTok{if}\NormalTok{(}\SpecialCharTok{!}\FunctionTok{require}\NormalTok{(raster)) \{}\FunctionTok{install.packages}\NormalTok{(}\StringTok{"raster"}\NormalTok{); }\FunctionTok{require}\NormalTok{(raster)\}}
\ControlFlowTok{if}\NormalTok{(}\SpecialCharTok{!}\FunctionTok{require}\NormalTok{(fasterize)) \{}\FunctionTok{install.packages}\NormalTok{(}\StringTok{"fasterize"}\NormalTok{); }\FunctionTok{require}\NormalTok{(fasterize)\}}
\ControlFlowTok{if}\NormalTok{(}\SpecialCharTok{!}\FunctionTok{require}\NormalTok{(tidyverse)) \{}\FunctionTok{install.packages}\NormalTok{(}\StringTok{"tidyverse"}\NormalTok{); }\FunctionTok{require}\NormalTok{(tidyverse)\}}


\CommentTok{\# Templates {-}{-}{-}{-}{-}}
\NormalTok{template10}\OtherTok{=}\FunctionTok{rast}\NormalTok{(}\StringTok{"./Templates/TemplateRasters/LV10m\_10km.tif"}\NormalTok{)}
\NormalTok{nulls10}\OtherTok{=}\FunctionTok{rast}\NormalTok{(}\StringTok{"./Templates/TemplateRasters/nulls\_LV10m\_10km.tif"}\NormalTok{)}

\CommentTok{\# simple landscape {-}{-}{-}{-}}
\NormalTok{simple\_landscape}\OtherTok{=}\FunctionTok{rast}\NormalTok{(}\StringTok{"./RasterGrids\_10m/2024/Ainava\_vienk\_mask.tif"}\NormalTok{)}

\CommentTok{\# Edges\_Roads\_input.tif {-}{-}{-}{-}}
\NormalTok{roads}\OtherTok{=}\FunctionTok{ifel}\NormalTok{(simple\_landscape}\SpecialCharTok{==}\DecValTok{100}\NormalTok{,}\DecValTok{1}\NormalTok{,}\ConstantTok{NA}\NormalTok{)}
\FunctionTok{plot}\NormalTok{(roads)}
\NormalTok{roads}\OtherTok{=}\FunctionTok{cover}\NormalTok{(roads,nulls10)}
\FunctionTok{plot}\NormalTok{(roads)}

\NormalTok{edge\_roads}\OtherTok{=}\FunctionTok{project}\NormalTok{(roads,template10,}
          \AttributeTok{filename=}\StringTok{"./RasterGrids\_10m/2024/Edges\_Roads\_input.tif"}\NormalTok{,}
          \AttributeTok{overwrite=}\ConstantTok{TRUE}\NormalTok{)}
\FunctionTok{rm}\NormalTok{(edge\_roads)}


\CommentTok{\# Edges\_Roads\_cell.tif  egv\_150 {-}{-}{-}{-}}
\FunctionTok{landscape\_function}\NormalTok{(}
 \AttributeTok{landscape   =} \StringTok{"./RasterGrids\_10m/2024/Edges\_Roads\_input.tif"}\NormalTok{,}
 \AttributeTok{zones     =} \StringTok{"./Templates/TemplateGrids/tikls100\_sauzeme.parquet"}\NormalTok{,}
 \AttributeTok{id\_field    =} \StringTok{"id"}\NormalTok{,}
 \AttributeTok{tile\_field   =} \StringTok{"tks50km"}\NormalTok{,}
 \AttributeTok{template    =} \StringTok{"./Templates/TemplateRasters/LV100m\_10km.tif"}\NormalTok{,}
 \AttributeTok{out\_dir    =} \StringTok{"./RasterGrids\_100m/2024/RAW"}\NormalTok{,}
 \AttributeTok{out\_filename  =} \StringTok{"Edges\_Roads\_cell.tif"}\NormalTok{,}
 \AttributeTok{out\_layername =} \StringTok{"egv\_150"}\NormalTok{,}
 \AttributeTok{what       =} \StringTok{"lsm\_l\_te"}\NormalTok{,}
 \AttributeTok{lm\_args     =} \FunctionTok{list}\NormalTok{(}\AttributeTok{count\_boundary =} \ConstantTok{FALSE}\NormalTok{),}
 \AttributeTok{rasterize\_engine =} \StringTok{"fasterize"}\NormalTok{,}
 \AttributeTok{n\_workers   =} \DecValTok{12}\NormalTok{,}
 \AttributeTok{future\_max\_size =} \DecValTok{20} \SpecialCharTok{*} \DecValTok{1024}\SpecialCharTok{\^{}}\DecValTok{3}\NormalTok{,}
 \AttributeTok{fill\_gaps   =} \ConstantTok{TRUE}\NormalTok{,}
 \AttributeTok{plot\_gaps   =} \ConstantTok{FALSE}\NormalTok{,}
 \AttributeTok{plot\_result  =} \ConstantTok{FALSE}
\NormalTok{)}

\CommentTok{\# standardisation {-}{-}{-}{-}}
\ControlFlowTok{if}\NormalTok{(}\SpecialCharTok{!}\FunctionTok{require}\NormalTok{(terra)) \{}\FunctionTok{install.packages}\NormalTok{(}\StringTok{"terra"}\NormalTok{); }\FunctionTok{require}\NormalTok{(terra)\}}
\ControlFlowTok{if}\NormalTok{(}\SpecialCharTok{!}\FunctionTok{require}\NormalTok{(tidyverse)) \{}\FunctionTok{install.packages}\NormalTok{(}\StringTok{"tidyverse"}\NormalTok{); }\FunctionTok{require}\NormalTok{(tidyverse)\}}

\NormalTok{nosaukums}\OtherTok{=}\StringTok{"Edges\_Roads\_cell.tif"}
\NormalTok{ielasisanas\_cels}\OtherTok{=}\FunctionTok{paste0}\NormalTok{(}\StringTok{"./RasterGrids\_100m/2024/RAW/"}\NormalTok{,nosaukums)}
\NormalTok{saglabasanas\_cels}\OtherTok{=}\FunctionTok{paste0}\NormalTok{(}\StringTok{"./RasterGrids\_100m/2024/Scaled/"}\NormalTok{,nosaukums)}
\NormalTok{slanis}\OtherTok{=}\FunctionTok{rast}\NormalTok{(ielasisanas\_cels)}
\NormalTok{videjais}\OtherTok{=}\FunctionTok{global}\NormalTok{(slanis,}\AttributeTok{fun=}\StringTok{"mean"}\NormalTok{,}\AttributeTok{na.rm=}\ConstantTok{TRUE}\NormalTok{)}
\NormalTok{centrets}\OtherTok{=}\NormalTok{slanis}\SpecialCharTok{{-}}\NormalTok{videjais[,}\DecValTok{1}\NormalTok{]}
\NormalTok{standartnovirze}\OtherTok{=}\NormalTok{terra}\SpecialCharTok{::}\FunctionTok{global}\NormalTok{(centrets,}\AttributeTok{fun=}\StringTok{"rms"}\NormalTok{,}\AttributeTok{na.rm=}\ConstantTok{TRUE}\NormalTok{)}
\NormalTok{merogots}\OtherTok{=}\NormalTok{centrets}\SpecialCharTok{/}\NormalTok{standartnovirze[,}\DecValTok{1}\NormalTok{]}
\FunctionTok{writeRaster}\NormalTok{(merogots,}
      \AttributeTok{filename=}\NormalTok{saglabasanas\_cels,}
      \AttributeTok{overwrite=}\ConstantTok{TRUE}\NormalTok{)}
\end{Highlighting}
\end{Shaded}

\section{Edges\_Roads\_r500}\label{ch06.151}

\textbf{filename:} \texttt{Edges\_Roads\_r500.tif}

\textbf{layername:} \texttt{egv\_151}

\textbf{English name:} Edge pixels of Roads within the 0.5 km landscape

\textbf{Latvian name:} Ceļu malu pikseļu skaits 0,5 km ainavā

\textbf{Procedure:} The total edge within a 500 m radius around the analysis grid cell is
calculated as the area-weighted sum of the \hyperref[ch06.145]{analysis cells} inside the
buffer, using the workflow \texttt{egvtools::radius\_function()}. During the calculation of the landscape metric,
inverse distance weighted (power = 2) gap filling on the output is applied
to ensure no missing values at the edges. Then the layer is rewritten to set
its name. Finally, the layer is standardised by subtracting the arithmetic
mean and dividing by the root mean squared error.

\begin{Shaded}
\begin{Highlighting}[]
\CommentTok{\# libs {-}{-}{-}{-}}
\ControlFlowTok{if}\NormalTok{(}\SpecialCharTok{!}\FunctionTok{require}\NormalTok{(terra)) \{}\FunctionTok{install.packages}\NormalTok{(}\StringTok{"terra"}\NormalTok{); }\FunctionTok{require}\NormalTok{(terra)\}}
\ControlFlowTok{if}\NormalTok{(}\SpecialCharTok{!}\FunctionTok{require}\NormalTok{(egvtools)) \{remotes}\SpecialCharTok{::}\FunctionTok{install\_github}\NormalTok{(}\StringTok{"aavotins/egvtools"}\NormalTok{); }\FunctionTok{require}\NormalTok{(egvtools)\}}


\CommentTok{\# Templates {-}{-}{-}{-}{-}}
\NormalTok{template100}\OtherTok{=}\FunctionTok{rast}\NormalTok{(}\StringTok{"./Templates/TemplateRasters/LV100m\_10km.tif"}\NormalTok{)}

\CommentTok{\# radii {-}{-}{-}{-}}
\FunctionTok{radius\_function}\NormalTok{(}
 \AttributeTok{kvadrati\_path =} \StringTok{"./Templates/TemplateGrids/tiles/"}\NormalTok{,}
 \AttributeTok{radii\_path   =} \StringTok{"./Templates/TemplateGridPoints/tiles/"}\NormalTok{,}
 \AttributeTok{tikls100\_path =} \StringTok{"./Templates/TemplateGrids/tikls100\_sauzeme.parquet"}\NormalTok{,}
 \AttributeTok{template\_path =} \StringTok{"./Templates/TemplateRasters/LV100m\_10km.tif"}\NormalTok{,}
 \AttributeTok{input\_layers  =} \FunctionTok{c}\NormalTok{(}\StringTok{"./RasterGrids\_100m/2024/RAW/Edges\_Roads\_cell.tif"}\NormalTok{),}
 \AttributeTok{layer\_prefixes =} \FunctionTok{c}\NormalTok{(}\StringTok{"Edges\_Roads"}\NormalTok{),}
 \AttributeTok{output\_dir   =} \StringTok{"./RasterGrids\_100m/2024/RAW/"}\NormalTok{,}
 \AttributeTok{n\_workers   =} \DecValTok{12}\NormalTok{,}
 \AttributeTok{radii     =} \FunctionTok{c}\NormalTok{(}\StringTok{"r500"}\NormalTok{),}
 \AttributeTok{radius\_mode  =} \StringTok{"sparse"}\NormalTok{,}
 \AttributeTok{extract\_fun  =} \StringTok{"sum"}\NormalTok{,}
 \AttributeTok{fill\_missing  =} \ConstantTok{TRUE}\NormalTok{,}
 \AttributeTok{IDW\_weight   =} \DecValTok{2}\NormalTok{,}
 \AttributeTok{future\_max\_size =} \DecValTok{20} \SpecialCharTok{*} \DecValTok{1024}\SpecialCharTok{\^{}}\DecValTok{3}\NormalTok{)}


\CommentTok{\# Edges\_Roads\_r500.tif  egv\_151 {-}{-}{-}{-}}
\NormalTok{slanis}\OtherTok{=}\FunctionTok{rast}\NormalTok{(}\StringTok{"./RasterGrids\_100m/2024/RAW/Edges\_Roads\_r500.tif"}\NormalTok{)}
\FunctionTok{names}\NormalTok{(slanis)}\OtherTok{=}\StringTok{"egv\_151"}
\NormalTok{slanis2}\OtherTok{=}\FunctionTok{project}\NormalTok{(slanis,template100)}
\FunctionTok{writeRaster}\NormalTok{(slanis2,}
      \StringTok{"./RasterGrids\_100m/2024/RAW/Edges\_Roads\_r500.tif"}\NormalTok{,}
      \AttributeTok{overwrite=}\ConstantTok{TRUE}\NormalTok{)}

\CommentTok{\# standardisation {-}{-}{-}{-}}
\ControlFlowTok{if}\NormalTok{(}\SpecialCharTok{!}\FunctionTok{require}\NormalTok{(terra)) \{}\FunctionTok{install.packages}\NormalTok{(}\StringTok{"terra"}\NormalTok{); }\FunctionTok{require}\NormalTok{(terra)\}}
\ControlFlowTok{if}\NormalTok{(}\SpecialCharTok{!}\FunctionTok{require}\NormalTok{(tidyverse)) \{}\FunctionTok{install.packages}\NormalTok{(}\StringTok{"tidyverse"}\NormalTok{); }\FunctionTok{require}\NormalTok{(tidyverse)\}}

\NormalTok{nosaukums}\OtherTok{=}\StringTok{"Edges\_Roads\_r500.tif"}
\NormalTok{ielasisanas\_cels}\OtherTok{=}\FunctionTok{paste0}\NormalTok{(}\StringTok{"./RasterGrids\_100m/2024/RAW/"}\NormalTok{,nosaukums)}
\NormalTok{saglabasanas\_cels}\OtherTok{=}\FunctionTok{paste0}\NormalTok{(}\StringTok{"./RasterGrids\_100m/2024/Scaled/"}\NormalTok{,nosaukums)}
\NormalTok{slanis}\OtherTok{=}\FunctionTok{rast}\NormalTok{(ielasisanas\_cels)}
\NormalTok{videjais}\OtherTok{=}\FunctionTok{global}\NormalTok{(slanis,}\AttributeTok{fun=}\StringTok{"mean"}\NormalTok{,}\AttributeTok{na.rm=}\ConstantTok{TRUE}\NormalTok{)}
\NormalTok{centrets}\OtherTok{=}\NormalTok{slanis}\SpecialCharTok{{-}}\NormalTok{videjais[,}\DecValTok{1}\NormalTok{]}
\NormalTok{standartnovirze}\OtherTok{=}\NormalTok{terra}\SpecialCharTok{::}\FunctionTok{global}\NormalTok{(centrets,}\AttributeTok{fun=}\StringTok{"rms"}\NormalTok{,}\AttributeTok{na.rm=}\ConstantTok{TRUE}\NormalTok{)}
\NormalTok{merogots}\OtherTok{=}\NormalTok{centrets}\SpecialCharTok{/}\NormalTok{standartnovirze[,}\DecValTok{1}\NormalTok{]}
\FunctionTok{writeRaster}\NormalTok{(merogots,}
      \AttributeTok{filename=}\NormalTok{saglabasanas\_cels,}
      \AttributeTok{overwrite=}\ConstantTok{TRUE}\NormalTok{)}
\end{Highlighting}
\end{Shaded}

\section{Edges\_Roads\_r1250}\label{ch06.152}

\textbf{filename:} \texttt{Edges\_Roads\_r1250.tif}

\textbf{layername:} \texttt{egv\_152}

\textbf{English name:} Edge pixels of Roads within the 1.25 km landscape

\textbf{Latvian name:} Ceļu malu pikseļu skaits 1,25 km ainavā

\textbf{Procedure:} The total edge within a 1250 m radius around the analysis grid cell is
calculated as the area-weighted sum of the \hyperref[ch06.145]{analysis cells} inside the
buffer, using the workflow \texttt{egvtools::radius\_function()}. During the calculation of the landscape metric,
inverse distance weighted (power = 2) gap filling on the output is applied
to ensure no missing values at the edges. Then the layer is rewritten to set
its name. Finally, the layer is standardised by subtracting the arithmetic
mean and dividing by the root mean squared error.

\begin{Shaded}
\begin{Highlighting}[]
\CommentTok{\# libs {-}{-}{-}{-}}
\ControlFlowTok{if}\NormalTok{(}\SpecialCharTok{!}\FunctionTok{require}\NormalTok{(terra)) \{}\FunctionTok{install.packages}\NormalTok{(}\StringTok{"terra"}\NormalTok{); }\FunctionTok{require}\NormalTok{(terra)\}}
\ControlFlowTok{if}\NormalTok{(}\SpecialCharTok{!}\FunctionTok{require}\NormalTok{(egvtools)) \{remotes}\SpecialCharTok{::}\FunctionTok{install\_github}\NormalTok{(}\StringTok{"aavotins/egvtools"}\NormalTok{); }\FunctionTok{require}\NormalTok{(egvtools)\}}


\CommentTok{\# Templates {-}{-}{-}{-}{-}}
\NormalTok{template100}\OtherTok{=}\FunctionTok{rast}\NormalTok{(}\StringTok{"./Templates/TemplateRasters/LV100m\_10km.tif"}\NormalTok{)}

\CommentTok{\# radii {-}{-}{-}{-}}
\FunctionTok{radius\_function}\NormalTok{(}
 \AttributeTok{kvadrati\_path =} \StringTok{"./Templates/TemplateGrids/tiles/"}\NormalTok{,}
 \AttributeTok{radii\_path   =} \StringTok{"./Templates/TemplateGridPoints/tiles/"}\NormalTok{,}
 \AttributeTok{tikls100\_path =} \StringTok{"./Templates/TemplateGrids/tikls100\_sauzeme.parquet"}\NormalTok{,}
 \AttributeTok{template\_path =} \StringTok{"./Templates/TemplateRasters/LV100m\_10km.tif"}\NormalTok{,}
 \AttributeTok{input\_layers  =} \FunctionTok{c}\NormalTok{(}\StringTok{"./RasterGrids\_100m/2024/RAW/Edges\_Roads\_cell.tif"}\NormalTok{),}
 \AttributeTok{layer\_prefixes =} \FunctionTok{c}\NormalTok{(}\StringTok{"Edges\_Roads"}\NormalTok{),}
 \AttributeTok{output\_dir   =} \StringTok{"./RasterGrids\_100m/2024/RAW/"}\NormalTok{,}
 \AttributeTok{n\_workers   =} \DecValTok{12}\NormalTok{,}
 \AttributeTok{radii     =} \FunctionTok{c}\NormalTok{(}\StringTok{"r1250"}\NormalTok{),}
 \AttributeTok{radius\_mode  =} \StringTok{"sparse"}\NormalTok{,}
 \AttributeTok{extract\_fun  =} \StringTok{"sum"}\NormalTok{,}
 \AttributeTok{fill\_missing  =} \ConstantTok{TRUE}\NormalTok{,}
 \AttributeTok{IDW\_weight   =} \DecValTok{2}\NormalTok{,}
 \AttributeTok{future\_max\_size =} \DecValTok{20} \SpecialCharTok{*} \DecValTok{1024}\SpecialCharTok{\^{}}\DecValTok{3}\NormalTok{)}


\CommentTok{\# Edges\_Roads\_r1250.tif egv\_152 {-}{-}{-}{-}}
\NormalTok{slanis}\OtherTok{=}\FunctionTok{rast}\NormalTok{(}\StringTok{"./RasterGrids\_100m/2024/RAW/Edges\_Roads\_r1250.tif"}\NormalTok{)}
\FunctionTok{names}\NormalTok{(slanis)}\OtherTok{=}\StringTok{"egv\_152"}
\NormalTok{slanis2}\OtherTok{=}\FunctionTok{project}\NormalTok{(slanis,template100)}
\FunctionTok{writeRaster}\NormalTok{(slanis2,}
      \StringTok{"./RasterGrids\_100m/2024/RAW/Edges\_Roads\_r1250.tif"}\NormalTok{,}
      \AttributeTok{overwrite=}\ConstantTok{TRUE}\NormalTok{)}

\CommentTok{\# standardisation {-}{-}{-}{-}}
\ControlFlowTok{if}\NormalTok{(}\SpecialCharTok{!}\FunctionTok{require}\NormalTok{(terra)) \{}\FunctionTok{install.packages}\NormalTok{(}\StringTok{"terra"}\NormalTok{); }\FunctionTok{require}\NormalTok{(terra)\}}
\ControlFlowTok{if}\NormalTok{(}\SpecialCharTok{!}\FunctionTok{require}\NormalTok{(tidyverse)) \{}\FunctionTok{install.packages}\NormalTok{(}\StringTok{"tidyverse"}\NormalTok{); }\FunctionTok{require}\NormalTok{(tidyverse)\}}

\NormalTok{nosaukums}\OtherTok{=}\StringTok{"Edges\_Roads\_r1250.tif"}
\NormalTok{ielasisanas\_cels}\OtherTok{=}\FunctionTok{paste0}\NormalTok{(}\StringTok{"./RasterGrids\_100m/2024/RAW/"}\NormalTok{,nosaukums)}
\NormalTok{saglabasanas\_cels}\OtherTok{=}\FunctionTok{paste0}\NormalTok{(}\StringTok{"./RasterGrids\_100m/2024/Scaled/"}\NormalTok{,nosaukums)}
\NormalTok{slanis}\OtherTok{=}\FunctionTok{rast}\NormalTok{(ielasisanas\_cels)}
\NormalTok{videjais}\OtherTok{=}\FunctionTok{global}\NormalTok{(slanis,}\AttributeTok{fun=}\StringTok{"mean"}\NormalTok{,}\AttributeTok{na.rm=}\ConstantTok{TRUE}\NormalTok{)}
\NormalTok{centrets}\OtherTok{=}\NormalTok{slanis}\SpecialCharTok{{-}}\NormalTok{videjais[,}\DecValTok{1}\NormalTok{]}
\NormalTok{standartnovirze}\OtherTok{=}\NormalTok{terra}\SpecialCharTok{::}\FunctionTok{global}\NormalTok{(centrets,}\AttributeTok{fun=}\StringTok{"rms"}\NormalTok{,}\AttributeTok{na.rm=}\ConstantTok{TRUE}\NormalTok{)}
\NormalTok{merogots}\OtherTok{=}\NormalTok{centrets}\SpecialCharTok{/}\NormalTok{standartnovirze[,}\DecValTok{1}\NormalTok{]}
\FunctionTok{writeRaster}\NormalTok{(merogots,}
      \AttributeTok{filename=}\NormalTok{saglabasanas\_cels,}
      \AttributeTok{overwrite=}\ConstantTok{TRUE}\NormalTok{)}
\end{Highlighting}
\end{Shaded}

\section{Edges\_Roads\_r3000}\label{ch06.153}

\textbf{filename:} \texttt{Edges\_Roads\_r3000.tif}

\textbf{layername:} \texttt{egv\_153}

\textbf{English name:} Edge pixels of Roads within the 3 km landscape

\textbf{Latvian name:} Ceļu malu pikseļu skaits 3 km ainavā

\textbf{Procedure:} The total edge within a 3000 m radius around the analysis grid cell is
calculated as the area-weighted sum of the \hyperref[ch06.145]{analysis cells} inside the
buffer, using the workflow \texttt{egvtools::radius\_function()}. During the calculation of the landscape metric,
inverse distance weighted (power = 2) gap filling on the output is applied
to ensure no missing values at the edges. Then the layer is rewritten to set
its name. Finally, the layer is standardised by subtracting the arithmetic
mean and dividing by the root mean squared error.

\begin{Shaded}
\begin{Highlighting}[]
\CommentTok{\# libs {-}{-}{-}{-}}
\ControlFlowTok{if}\NormalTok{(}\SpecialCharTok{!}\FunctionTok{require}\NormalTok{(terra)) \{}\FunctionTok{install.packages}\NormalTok{(}\StringTok{"terra"}\NormalTok{); }\FunctionTok{require}\NormalTok{(terra)\}}
\ControlFlowTok{if}\NormalTok{(}\SpecialCharTok{!}\FunctionTok{require}\NormalTok{(egvtools)) \{remotes}\SpecialCharTok{::}\FunctionTok{install\_github}\NormalTok{(}\StringTok{"aavotins/egvtools"}\NormalTok{); }\FunctionTok{require}\NormalTok{(egvtools)\}}


\CommentTok{\# Templates {-}{-}{-}{-}{-}}
\NormalTok{template100}\OtherTok{=}\FunctionTok{rast}\NormalTok{(}\StringTok{"./Templates/TemplateRasters/LV100m\_10km.tif"}\NormalTok{)}

\CommentTok{\# radii {-}{-}{-}{-}}
\FunctionTok{radius\_function}\NormalTok{(}
 \AttributeTok{kvadrati\_path =} \StringTok{"./Templates/TemplateGrids/tiles/"}\NormalTok{,}
 \AttributeTok{radii\_path   =} \StringTok{"./Templates/TemplateGridPoints/tiles/"}\NormalTok{,}
 \AttributeTok{tikls100\_path =} \StringTok{"./Templates/TemplateGrids/tikls100\_sauzeme.parquet"}\NormalTok{,}
 \AttributeTok{template\_path =} \StringTok{"./Templates/TemplateRasters/LV100m\_10km.tif"}\NormalTok{,}
 \AttributeTok{input\_layers  =} \FunctionTok{c}\NormalTok{(}\StringTok{"./RasterGrids\_100m/2024/RAW/Edges\_Roads\_cell.tif"}\NormalTok{),}
 \AttributeTok{layer\_prefixes =} \FunctionTok{c}\NormalTok{(}\StringTok{"Edges\_Roads"}\NormalTok{),}
 \AttributeTok{output\_dir   =} \StringTok{"./RasterGrids\_100m/2024/RAW/"}\NormalTok{,}
 \AttributeTok{n\_workers   =} \DecValTok{12}\NormalTok{,}
 \AttributeTok{radii     =} \FunctionTok{c}\NormalTok{(}\StringTok{"r3000"}\NormalTok{),}
 \AttributeTok{radius\_mode  =} \StringTok{"sparse"}\NormalTok{,}
 \AttributeTok{extract\_fun  =} \StringTok{"sum"}\NormalTok{,}
 \AttributeTok{fill\_missing  =} \ConstantTok{TRUE}\NormalTok{,}
 \AttributeTok{IDW\_weight   =} \DecValTok{2}\NormalTok{,}
 \AttributeTok{future\_max\_size =} \DecValTok{20} \SpecialCharTok{*} \DecValTok{1024}\SpecialCharTok{\^{}}\DecValTok{3}\NormalTok{)}


\CommentTok{\# Edges\_Roads\_r3000.tif egv\_153 {-}{-}{-}{-}}
\NormalTok{slanis}\OtherTok{=}\FunctionTok{rast}\NormalTok{(}\StringTok{"./RasterGrids\_100m/2024/RAW/Edges\_Roads\_r3000.tif"}\NormalTok{)}
\FunctionTok{names}\NormalTok{(slanis)}\OtherTok{=}\StringTok{"egv\_153"}
\NormalTok{slanis2}\OtherTok{=}\FunctionTok{project}\NormalTok{(slanis,template100)}
\FunctionTok{writeRaster}\NormalTok{(slanis2,}
      \StringTok{"./RasterGrids\_100m/2024/RAW/Edges\_Roads\_r3000.tif"}\NormalTok{,}
      \AttributeTok{overwrite=}\ConstantTok{TRUE}\NormalTok{)}

\CommentTok{\# standardisation {-}{-}{-}{-}}
\ControlFlowTok{if}\NormalTok{(}\SpecialCharTok{!}\FunctionTok{require}\NormalTok{(terra)) \{}\FunctionTok{install.packages}\NormalTok{(}\StringTok{"terra"}\NormalTok{); }\FunctionTok{require}\NormalTok{(terra)\}}
\ControlFlowTok{if}\NormalTok{(}\SpecialCharTok{!}\FunctionTok{require}\NormalTok{(tidyverse)) \{}\FunctionTok{install.packages}\NormalTok{(}\StringTok{"tidyverse"}\NormalTok{); }\FunctionTok{require}\NormalTok{(tidyverse)\}}

\NormalTok{nosaukums}\OtherTok{=}\StringTok{"Edges\_Roads\_r3000.tif"}
\NormalTok{ielasisanas\_cels}\OtherTok{=}\FunctionTok{paste0}\NormalTok{(}\StringTok{"./RasterGrids\_100m/2024/RAW/"}\NormalTok{,nosaukums)}
\NormalTok{saglabasanas\_cels}\OtherTok{=}\FunctionTok{paste0}\NormalTok{(}\StringTok{"./RasterGrids\_100m/2024/Scaled/"}\NormalTok{,nosaukums)}
\NormalTok{slanis}\OtherTok{=}\FunctionTok{rast}\NormalTok{(ielasisanas\_cels)}
\NormalTok{videjais}\OtherTok{=}\FunctionTok{global}\NormalTok{(slanis,}\AttributeTok{fun=}\StringTok{"mean"}\NormalTok{,}\AttributeTok{na.rm=}\ConstantTok{TRUE}\NormalTok{)}
\NormalTok{centrets}\OtherTok{=}\NormalTok{slanis}\SpecialCharTok{{-}}\NormalTok{videjais[,}\DecValTok{1}\NormalTok{]}
\NormalTok{standartnovirze}\OtherTok{=}\NormalTok{terra}\SpecialCharTok{::}\FunctionTok{global}\NormalTok{(centrets,}\AttributeTok{fun=}\StringTok{"rms"}\NormalTok{,}\AttributeTok{na.rm=}\ConstantTok{TRUE}\NormalTok{)}
\NormalTok{merogots}\OtherTok{=}\NormalTok{centrets}\SpecialCharTok{/}\NormalTok{standartnovirze[,}\DecValTok{1}\NormalTok{]}
\FunctionTok{writeRaster}\NormalTok{(merogots,}
      \AttributeTok{filename=}\NormalTok{saglabasanas\_cels,}
      \AttributeTok{overwrite=}\ConstantTok{TRUE}\NormalTok{)}
\end{Highlighting}
\end{Shaded}

\section{Edges\_Roads\_r10000}\label{ch06.154}

\textbf{filename:} \texttt{Edges\_Roads\_r10000.tif}

\textbf{layername:} \texttt{egv\_154}

\textbf{English name:} Edge pixels of Roads within the 10 km landscape

\textbf{Latvian name:} Ceļu malu pikseļu skaits 10 km ainavā

\textbf{Procedure:} The total edge within a 10000 m radius around the analysis grid cell is
calculated as the area-weighted sum of the \hyperref[ch06.145]{analysis cells} inside the
buffer, using the workflow \texttt{egvtools::radius\_function()}. During the calculation of the landscape metric,
inverse distance weighted (power = 2) gap filling on the output is applied
to ensure no missing values at the edges. Then the layer is rewritten to set
its name. Finally, the layer is standardised by subtracting the arithmetic
mean and dividing by the root mean squared error.

\begin{Shaded}
\begin{Highlighting}[]
\CommentTok{\# libs {-}{-}{-}{-}}
\ControlFlowTok{if}\NormalTok{(}\SpecialCharTok{!}\FunctionTok{require}\NormalTok{(terra)) \{}\FunctionTok{install.packages}\NormalTok{(}\StringTok{"terra"}\NormalTok{); }\FunctionTok{require}\NormalTok{(terra)\}}
\ControlFlowTok{if}\NormalTok{(}\SpecialCharTok{!}\FunctionTok{require}\NormalTok{(egvtools)) \{remotes}\SpecialCharTok{::}\FunctionTok{install\_github}\NormalTok{(}\StringTok{"aavotins/egvtools"}\NormalTok{); }\FunctionTok{require}\NormalTok{(egvtools)\}}


\CommentTok{\# Templates {-}{-}{-}{-}{-}}
\NormalTok{template100}\OtherTok{=}\FunctionTok{rast}\NormalTok{(}\StringTok{"./Templates/TemplateRasters/LV100m\_10km.tif"}\NormalTok{)}

\CommentTok{\# radii {-}{-}{-}{-}}
\FunctionTok{radius\_function}\NormalTok{(}
 \AttributeTok{kvadrati\_path =} \StringTok{"./Templates/TemplateGrids/tiles/"}\NormalTok{,}
 \AttributeTok{radii\_path   =} \StringTok{"./Templates/TemplateGridPoints/tiles/"}\NormalTok{,}
 \AttributeTok{tikls100\_path =} \StringTok{"./Templates/TemplateGrids/tikls100\_sauzeme.parquet"}\NormalTok{,}
 \AttributeTok{template\_path =} \StringTok{"./Templates/TemplateRasters/LV100m\_10km.tif"}\NormalTok{,}
 \AttributeTok{input\_layers  =} \FunctionTok{c}\NormalTok{(}\StringTok{"./RasterGrids\_100m/2024/RAW/Edges\_Roads\_cell.tif"}\NormalTok{),}
 \AttributeTok{layer\_prefixes =} \FunctionTok{c}\NormalTok{(}\StringTok{"Edges\_Roads"}\NormalTok{),}
 \AttributeTok{output\_dir   =} \StringTok{"./RasterGrids\_100m/2024/RAW/"}\NormalTok{,}
 \AttributeTok{n\_workers   =} \DecValTok{12}\NormalTok{,}
 \AttributeTok{radii     =} \FunctionTok{c}\NormalTok{(}\StringTok{"r10000"}\NormalTok{),}
 \AttributeTok{radius\_mode  =} \StringTok{"sparse"}\NormalTok{,}
 \AttributeTok{extract\_fun  =} \StringTok{"sum"}\NormalTok{,}
 \AttributeTok{fill\_missing  =} \ConstantTok{TRUE}\NormalTok{,}
 \AttributeTok{IDW\_weight   =} \DecValTok{2}\NormalTok{,}
 \AttributeTok{future\_max\_size =} \DecValTok{20} \SpecialCharTok{*} \DecValTok{1024}\SpecialCharTok{\^{}}\DecValTok{3}\NormalTok{)}


\CommentTok{\# Edges\_Roads\_r10000.tif    egv\_154 {-}{-}{-}{-}}
\NormalTok{slanis}\OtherTok{=}\FunctionTok{rast}\NormalTok{(}\StringTok{"./RasterGrids\_100m/2024/RAW/Edges\_Roads\_r10000.tif"}\NormalTok{)}
\FunctionTok{names}\NormalTok{(slanis)}\OtherTok{=}\StringTok{"egv\_154"}
\NormalTok{slanis2}\OtherTok{=}\FunctionTok{project}\NormalTok{(slanis,template100)}
\FunctionTok{writeRaster}\NormalTok{(slanis2,}
      \StringTok{"./RasterGrids\_100m/2024/RAW/Edges\_Roads\_r10000.tif"}\NormalTok{,}
      \AttributeTok{overwrite=}\ConstantTok{TRUE}\NormalTok{)}

\CommentTok{\# standardisation {-}{-}{-}{-}}
\ControlFlowTok{if}\NormalTok{(}\SpecialCharTok{!}\FunctionTok{require}\NormalTok{(terra)) \{}\FunctionTok{install.packages}\NormalTok{(}\StringTok{"terra"}\NormalTok{); }\FunctionTok{require}\NormalTok{(terra)\}}
\ControlFlowTok{if}\NormalTok{(}\SpecialCharTok{!}\FunctionTok{require}\NormalTok{(tidyverse)) \{}\FunctionTok{install.packages}\NormalTok{(}\StringTok{"tidyverse"}\NormalTok{); }\FunctionTok{require}\NormalTok{(tidyverse)\}}

\NormalTok{nosaukums}\OtherTok{=}\StringTok{"Edges\_Roads\_r10000.tif"}
\NormalTok{ielasisanas\_cels}\OtherTok{=}\FunctionTok{paste0}\NormalTok{(}\StringTok{"./RasterGrids\_100m/2024/RAW/"}\NormalTok{,nosaukums)}
\NormalTok{saglabasanas\_cels}\OtherTok{=}\FunctionTok{paste0}\NormalTok{(}\StringTok{"./RasterGrids\_100m/2024/Scaled/"}\NormalTok{,nosaukums)}
\NormalTok{slanis}\OtherTok{=}\FunctionTok{rast}\NormalTok{(ielasisanas\_cels)}
\NormalTok{videjais}\OtherTok{=}\FunctionTok{global}\NormalTok{(slanis,}\AttributeTok{fun=}\StringTok{"mean"}\NormalTok{,}\AttributeTok{na.rm=}\ConstantTok{TRUE}\NormalTok{)}
\NormalTok{centrets}\OtherTok{=}\NormalTok{slanis}\SpecialCharTok{{-}}\NormalTok{videjais[,}\DecValTok{1}\NormalTok{]}
\NormalTok{standartnovirze}\OtherTok{=}\NormalTok{terra}\SpecialCharTok{::}\FunctionTok{global}\NormalTok{(centrets,}\AttributeTok{fun=}\StringTok{"rms"}\NormalTok{,}\AttributeTok{na.rm=}\ConstantTok{TRUE}\NormalTok{)}
\NormalTok{merogots}\OtherTok{=}\NormalTok{centrets}\SpecialCharTok{/}\NormalTok{standartnovirze[,}\DecValTok{1}\NormalTok{]}
\FunctionTok{writeRaster}\NormalTok{(merogots,}
      \AttributeTok{filename=}\NormalTok{saglabasanas\_cels,}
      \AttributeTok{overwrite=}\ConstantTok{TRUE}\NormalTok{)}
\end{Highlighting}
\end{Shaded}

\section{Edges\_Trees\_cell}\label{ch06.155}

\textbf{filename:} \texttt{Edges\_Trees\_cell.tif}

\textbf{layername:} \texttt{egv\_155}

\textbf{English name:} Edge pixels of Trees within the analysis cell (1 ha)

\textbf{Latvian name:} Koku malu pikseļu skaits analīzes šūnā (1 ha)

\textbf{Procedure:} First, values larger or equal to 630 and smaller than 700 from
\hyperref[Ch05.03]{Landscape classification} are coded as 1, and all other values as NA.
Then, the layer (1 = presence) is covered over the nulls layer (presence = 0) and
written to file (matching the input). Next, with the workflow
\texttt{egvtools::landscape\_function()} total edge between the two classes is
calculated. During the calculation of the landscape metric, inverse distance weighted
(power = 2) gap filling on the output is applied to ensure no missing values
at the edges. Finally, the layer is standardised by subtracting the arithmetic
mean and dividing by the root mean squared error.

\begin{Shaded}
\begin{Highlighting}[]
\CommentTok{\# libs {-}{-}{-}{-}}
\ControlFlowTok{if}\NormalTok{(}\SpecialCharTok{!}\FunctionTok{require}\NormalTok{(terra)) \{}\FunctionTok{install.packages}\NormalTok{(}\StringTok{"terra"}\NormalTok{); }\FunctionTok{require}\NormalTok{(terra)\}}
\ControlFlowTok{if}\NormalTok{(}\SpecialCharTok{!}\FunctionTok{require}\NormalTok{(egvtools)) \{remotes}\SpecialCharTok{::}\FunctionTok{install\_github}\NormalTok{(}\StringTok{"aavotins/egvtools"}\NormalTok{); }\FunctionTok{require}\NormalTok{(egvtools)\}}

\ControlFlowTok{if}\NormalTok{(}\SpecialCharTok{!}\FunctionTok{require}\NormalTok{(sf)) \{}\FunctionTok{install.packages}\NormalTok{(}\StringTok{"sf"}\NormalTok{); }\FunctionTok{require}\NormalTok{(sf)\}}
\ControlFlowTok{if}\NormalTok{(}\SpecialCharTok{!}\FunctionTok{require}\NormalTok{(sfarrow)) \{}\FunctionTok{install.packages}\NormalTok{(}\StringTok{"sfarrow"}\NormalTok{); }\FunctionTok{require}\NormalTok{(sfarrow)\}}
\ControlFlowTok{if}\NormalTok{(}\SpecialCharTok{!}\FunctionTok{require}\NormalTok{(raster)) \{}\FunctionTok{install.packages}\NormalTok{(}\StringTok{"raster"}\NormalTok{); }\FunctionTok{require}\NormalTok{(raster)\}}
\ControlFlowTok{if}\NormalTok{(}\SpecialCharTok{!}\FunctionTok{require}\NormalTok{(fasterize)) \{}\FunctionTok{install.packages}\NormalTok{(}\StringTok{"fasterize"}\NormalTok{); }\FunctionTok{require}\NormalTok{(fasterize)\}}
\ControlFlowTok{if}\NormalTok{(}\SpecialCharTok{!}\FunctionTok{require}\NormalTok{(tidyverse)) \{}\FunctionTok{install.packages}\NormalTok{(}\StringTok{"tidyverse"}\NormalTok{); }\FunctionTok{require}\NormalTok{(tidyverse)\}}


\CommentTok{\# Templates {-}{-}{-}{-}{-}}
\NormalTok{template10}\OtherTok{=}\FunctionTok{rast}\NormalTok{(}\StringTok{"./Templates/TemplateRasters/LV10m\_10km.tif"}\NormalTok{)}
\NormalTok{nulls10}\OtherTok{=}\FunctionTok{rast}\NormalTok{(}\StringTok{"./Templates/TemplateRasters/nulls\_LV10m\_10km.tif"}\NormalTok{)}

\CommentTok{\# simple landscape {-}{-}{-}{-}}
\NormalTok{simple\_landscape}\OtherTok{=}\FunctionTok{rast}\NormalTok{(}\StringTok{"./RasterGrids\_10m/2024/Ainava\_vienk\_mask.tif"}\NormalTok{)}

\CommentTok{\# Edges\_Trees\_input.tif {-}{-}{-}{-}}
\NormalTok{trees\_from630}\OtherTok{=}\FunctionTok{ifel}\NormalTok{(simple\_landscape}\SpecialCharTok{\textgreater{}=}\DecValTok{630} \SpecialCharTok{\&}\NormalTok{ simple\_landscape}\SpecialCharTok{\textless{}}\DecValTok{700}\NormalTok{,}\DecValTok{1}\NormalTok{,}\ConstantTok{NA}\NormalTok{)}
\FunctionTok{plot}\NormalTok{(trees\_from630)}
\NormalTok{trees\_from630}\OtherTok{=}\FunctionTok{cover}\NormalTok{(trees\_from630,nulls10)}
\FunctionTok{plot}\NormalTok{(trees\_from630)}

\NormalTok{edge\_trees\_from630}\OtherTok{=}\FunctionTok{project}\NormalTok{(trees\_from630,template10,}
          \AttributeTok{filename=}\StringTok{"./RasterGrids\_10m/2024/Edges\_Trees\_input.tif"}\NormalTok{,}
          \AttributeTok{overwrite=}\ConstantTok{TRUE}\NormalTok{)}
\FunctionTok{rm}\NormalTok{(edge\_trees\_from630)}


\CommentTok{\# Edges\_Trees\_cell.tif  egv\_155}
\FunctionTok{landscape\_function}\NormalTok{(}
 \AttributeTok{landscape   =} \StringTok{"./RasterGrids\_10m/2024/Edges\_Trees\_input.tif"}\NormalTok{,}
 \AttributeTok{zones     =} \StringTok{"./Templates/TemplateGrids/tikls100\_sauzeme.parquet"}\NormalTok{,}
 \AttributeTok{id\_field    =} \StringTok{"id"}\NormalTok{,}
 \AttributeTok{tile\_field   =} \StringTok{"tks50km"}\NormalTok{,}
 \AttributeTok{template    =} \StringTok{"./Templates/TemplateRasters/LV100m\_10km.tif"}\NormalTok{,}
 \AttributeTok{out\_dir    =} \StringTok{"./RasterGrids\_100m/2024/RAW"}\NormalTok{,}
 \AttributeTok{out\_filename  =} \StringTok{"Edges\_Trees\_cell.tif"}\NormalTok{,}
 \AttributeTok{out\_layername =} \StringTok{"egv\_155"}\NormalTok{,}
 \AttributeTok{what       =} \StringTok{"lsm\_l\_te"}\NormalTok{,}
 \AttributeTok{lm\_args     =} \FunctionTok{list}\NormalTok{(}\AttributeTok{count\_boundary =} \ConstantTok{FALSE}\NormalTok{),}
 \AttributeTok{rasterize\_engine =} \StringTok{"fasterize"}\NormalTok{,}
 \AttributeTok{n\_workers   =} \DecValTok{12}\NormalTok{,}
 \AttributeTok{future\_max\_size =} \DecValTok{20} \SpecialCharTok{*} \DecValTok{1024}\SpecialCharTok{\^{}}\DecValTok{3}\NormalTok{,}
 \AttributeTok{fill\_gaps   =} \ConstantTok{TRUE}\NormalTok{,}
 \AttributeTok{plot\_gaps   =} \ConstantTok{FALSE}\NormalTok{,}
 \AttributeTok{plot\_result  =} \ConstantTok{FALSE}
\NormalTok{)}

\CommentTok{\# standardisation {-}{-}{-}{-}}
\ControlFlowTok{if}\NormalTok{(}\SpecialCharTok{!}\FunctionTok{require}\NormalTok{(terra)) \{}\FunctionTok{install.packages}\NormalTok{(}\StringTok{"terra"}\NormalTok{); }\FunctionTok{require}\NormalTok{(terra)\}}
\ControlFlowTok{if}\NormalTok{(}\SpecialCharTok{!}\FunctionTok{require}\NormalTok{(tidyverse)) \{}\FunctionTok{install.packages}\NormalTok{(}\StringTok{"tidyverse"}\NormalTok{); }\FunctionTok{require}\NormalTok{(tidyverse)\}}

\NormalTok{nosaukums}\OtherTok{=}\StringTok{"Edges\_Trees\_cell.tif"}
\NormalTok{ielasisanas\_cels}\OtherTok{=}\FunctionTok{paste0}\NormalTok{(}\StringTok{"./RasterGrids\_100m/2024/RAW/"}\NormalTok{,nosaukums)}
\NormalTok{saglabasanas\_cels}\OtherTok{=}\FunctionTok{paste0}\NormalTok{(}\StringTok{"./RasterGrids\_100m/2024/Scaled/"}\NormalTok{,nosaukums)}
\NormalTok{slanis}\OtherTok{=}\FunctionTok{rast}\NormalTok{(ielasisanas\_cels)}
\NormalTok{videjais}\OtherTok{=}\FunctionTok{global}\NormalTok{(slanis,}\AttributeTok{fun=}\StringTok{"mean"}\NormalTok{,}\AttributeTok{na.rm=}\ConstantTok{TRUE}\NormalTok{)}
\NormalTok{centrets}\OtherTok{=}\NormalTok{slanis}\SpecialCharTok{{-}}\NormalTok{videjais[,}\DecValTok{1}\NormalTok{]}
\NormalTok{standartnovirze}\OtherTok{=}\NormalTok{terra}\SpecialCharTok{::}\FunctionTok{global}\NormalTok{(centrets,}\AttributeTok{fun=}\StringTok{"rms"}\NormalTok{,}\AttributeTok{na.rm=}\ConstantTok{TRUE}\NormalTok{)}
\NormalTok{merogots}\OtherTok{=}\NormalTok{centrets}\SpecialCharTok{/}\NormalTok{standartnovirze[,}\DecValTok{1}\NormalTok{]}
\FunctionTok{writeRaster}\NormalTok{(merogots,}
      \AttributeTok{filename=}\NormalTok{saglabasanas\_cels,}
      \AttributeTok{overwrite=}\ConstantTok{TRUE}\NormalTok{)}
\end{Highlighting}
\end{Shaded}

\section{Edges\_Trees\_r500}\label{ch06.156}

\textbf{filename:} \texttt{Edges\_Trees\_r500.tif}

\textbf{layername:} \texttt{egv\_156}

\textbf{English name:} Edge pixels of Trees within the 0.5 km landscape

\textbf{Latvian name:} Koku malu pikseļu skaits 0,5 km ainavā

\textbf{Procedure:} The total edge within a 500 m radius around the analysis grid cell is
calculated as the area-weighted sum of the \hyperref[ch06.155]{analysis cells} inside the
buffer, using the workflow \texttt{egvtools::radius\_function()}. During the calculation of the landscape metric,
inverse distance weighted (power = 2) gap filling on the output is applied
to ensure no missing values at the edges. Then the layer is rewritten to set
its name. Finally, the layer is standardised by subtracting the arithmetic
mean and dividing by the root mean squared error.

\begin{Shaded}
\begin{Highlighting}[]
\CommentTok{\# libs {-}{-}{-}{-}}
\ControlFlowTok{if}\NormalTok{(}\SpecialCharTok{!}\FunctionTok{require}\NormalTok{(terra)) \{}\FunctionTok{install.packages}\NormalTok{(}\StringTok{"terra"}\NormalTok{); }\FunctionTok{require}\NormalTok{(terra)\}}
\ControlFlowTok{if}\NormalTok{(}\SpecialCharTok{!}\FunctionTok{require}\NormalTok{(egvtools)) \{remotes}\SpecialCharTok{::}\FunctionTok{install\_github}\NormalTok{(}\StringTok{"aavotins/egvtools"}\NormalTok{); }\FunctionTok{require}\NormalTok{(egvtools)\}}


\CommentTok{\# Templates {-}{-}{-}{-}{-}}
\NormalTok{template100}\OtherTok{=}\FunctionTok{rast}\NormalTok{(}\StringTok{"./Templates/TemplateRasters/LV100m\_10km.tif"}\NormalTok{)}

\CommentTok{\# radii {-}{-}{-}{-}}
\FunctionTok{radius\_function}\NormalTok{(}
 \AttributeTok{kvadrati\_path =} \StringTok{"./Templates/TemplateGrids/tiles/"}\NormalTok{,}
 \AttributeTok{radii\_path   =} \StringTok{"./Templates/TemplateGridPoints/tiles/"}\NormalTok{,}
 \AttributeTok{tikls100\_path =} \StringTok{"./Templates/TemplateGrids/tikls100\_sauzeme.parquet"}\NormalTok{,}
 \AttributeTok{template\_path =} \StringTok{"./Templates/TemplateRasters/LV100m\_10km.tif"}\NormalTok{,}
 \AttributeTok{input\_layers  =} \FunctionTok{c}\NormalTok{(}\StringTok{"./RasterGrids\_100m/2024/RAW/Edges\_Trees\_cell.tif"}\NormalTok{),}
 \AttributeTok{layer\_prefixes =} \FunctionTok{c}\NormalTok{(}\StringTok{"Edges\_Trees"}\NormalTok{),}
 \AttributeTok{output\_dir   =} \StringTok{"./RasterGrids\_100m/2024/RAW/"}\NormalTok{,}
 \AttributeTok{n\_workers   =} \DecValTok{12}\NormalTok{,}
 \AttributeTok{radii     =} \FunctionTok{c}\NormalTok{(}\StringTok{"r500"}\NormalTok{),}
 \AttributeTok{radius\_mode  =} \StringTok{"sparse"}\NormalTok{,}
 \AttributeTok{extract\_fun  =} \StringTok{"sum"}\NormalTok{,}
 \AttributeTok{fill\_missing  =} \ConstantTok{TRUE}\NormalTok{,}
 \AttributeTok{IDW\_weight   =} \DecValTok{2}\NormalTok{,}
 \AttributeTok{future\_max\_size =} \DecValTok{20} \SpecialCharTok{*} \DecValTok{1024}\SpecialCharTok{\^{}}\DecValTok{3}\NormalTok{)}


\CommentTok{\# Edges\_Trees\_r500.tif  egv\_156 {-}{-}{-}{-}}
\NormalTok{slanis}\OtherTok{=}\FunctionTok{rast}\NormalTok{(}\StringTok{"./RasterGrids\_100m/2024/RAW/Edges\_Trees\_r500.tif"}\NormalTok{)}
\FunctionTok{names}\NormalTok{(slanis)}\OtherTok{=}\StringTok{"egv\_156"}
\NormalTok{slanis2}\OtherTok{=}\FunctionTok{project}\NormalTok{(slanis,template100)}
\FunctionTok{writeRaster}\NormalTok{(slanis2,}
      \StringTok{"./RasterGrids\_100m/2024/RAW/Edges\_Trees\_r500.tif"}\NormalTok{,}
      \AttributeTok{overwrite=}\ConstantTok{TRUE}\NormalTok{)}

\CommentTok{\# standardisation {-}{-}{-}{-}}
\ControlFlowTok{if}\NormalTok{(}\SpecialCharTok{!}\FunctionTok{require}\NormalTok{(terra)) \{}\FunctionTok{install.packages}\NormalTok{(}\StringTok{"terra"}\NormalTok{); }\FunctionTok{require}\NormalTok{(terra)\}}
\ControlFlowTok{if}\NormalTok{(}\SpecialCharTok{!}\FunctionTok{require}\NormalTok{(tidyverse)) \{}\FunctionTok{install.packages}\NormalTok{(}\StringTok{"tidyverse"}\NormalTok{); }\FunctionTok{require}\NormalTok{(tidyverse)\}}

\NormalTok{nosaukums}\OtherTok{=}\StringTok{"Edges\_Trees\_r500.tif"}
\NormalTok{ielasisanas\_cels}\OtherTok{=}\FunctionTok{paste0}\NormalTok{(}\StringTok{"./RasterGrids\_100m/2024/RAW/"}\NormalTok{,nosaukums)}
\NormalTok{saglabasanas\_cels}\OtherTok{=}\FunctionTok{paste0}\NormalTok{(}\StringTok{"./RasterGrids\_100m/2024/Scaled/"}\NormalTok{,nosaukums)}
\NormalTok{slanis}\OtherTok{=}\FunctionTok{rast}\NormalTok{(ielasisanas\_cels)}
\NormalTok{videjais}\OtherTok{=}\FunctionTok{global}\NormalTok{(slanis,}\AttributeTok{fun=}\StringTok{"mean"}\NormalTok{,}\AttributeTok{na.rm=}\ConstantTok{TRUE}\NormalTok{)}
\NormalTok{centrets}\OtherTok{=}\NormalTok{slanis}\SpecialCharTok{{-}}\NormalTok{videjais[,}\DecValTok{1}\NormalTok{]}
\NormalTok{standartnovirze}\OtherTok{=}\NormalTok{terra}\SpecialCharTok{::}\FunctionTok{global}\NormalTok{(centrets,}\AttributeTok{fun=}\StringTok{"rms"}\NormalTok{,}\AttributeTok{na.rm=}\ConstantTok{TRUE}\NormalTok{)}
\NormalTok{merogots}\OtherTok{=}\NormalTok{centrets}\SpecialCharTok{/}\NormalTok{standartnovirze[,}\DecValTok{1}\NormalTok{]}
\FunctionTok{writeRaster}\NormalTok{(merogots,}
      \AttributeTok{filename=}\NormalTok{saglabasanas\_cels,}
      \AttributeTok{overwrite=}\ConstantTok{TRUE}\NormalTok{)}
\end{Highlighting}
\end{Shaded}

\section{Edges\_Trees\_r1250}\label{ch06.157}

\textbf{filename:} \texttt{Edges\_Trees\_r1250.tif}

\textbf{layername:} \texttt{egv\_157}

\textbf{English name:} Edge pixels of Trees within the 1.25 km landscape

\textbf{Latvian name:} Koku malu pikseļu skaits 1,25 km ainavā

\textbf{Procedure:} The total edge within a 1250 m radius around the analysis grid cell is
calculated as the area-weighted sum of the \hyperref[ch06.155]{analysis cells} inside the
buffer, using the workflow \texttt{egvtools::radius\_function()}. During the calculation of the landscape metric,
inverse distance weighted (power = 2) gap filling on the output is applied
to ensure no missing values at the edges. Then the layer is rewritten to set
its name. Finally, the layer is standardised by subtracting the arithmetic
mean and dividing by the root mean squared error.

\begin{Shaded}
\begin{Highlighting}[]
\CommentTok{\# libs {-}{-}{-}{-}}
\ControlFlowTok{if}\NormalTok{(}\SpecialCharTok{!}\FunctionTok{require}\NormalTok{(terra)) \{}\FunctionTok{install.packages}\NormalTok{(}\StringTok{"terra"}\NormalTok{); }\FunctionTok{require}\NormalTok{(terra)\}}
\ControlFlowTok{if}\NormalTok{(}\SpecialCharTok{!}\FunctionTok{require}\NormalTok{(egvtools)) \{remotes}\SpecialCharTok{::}\FunctionTok{install\_github}\NormalTok{(}\StringTok{"aavotins/egvtools"}\NormalTok{); }\FunctionTok{require}\NormalTok{(egvtools)\}}


\CommentTok{\# Templates {-}{-}{-}{-}{-}}
\NormalTok{template100}\OtherTok{=}\FunctionTok{rast}\NormalTok{(}\StringTok{"./Templates/TemplateRasters/LV100m\_10km.tif"}\NormalTok{)}

\CommentTok{\# radii {-}{-}{-}{-}}
\FunctionTok{radius\_function}\NormalTok{(}
 \AttributeTok{kvadrati\_path =} \StringTok{"./Templates/TemplateGrids/tiles/"}\NormalTok{,}
 \AttributeTok{radii\_path   =} \StringTok{"./Templates/TemplateGridPoints/tiles/"}\NormalTok{,}
 \AttributeTok{tikls100\_path =} \StringTok{"./Templates/TemplateGrids/tikls100\_sauzeme.parquet"}\NormalTok{,}
 \AttributeTok{template\_path =} \StringTok{"./Templates/TemplateRasters/LV100m\_10km.tif"}\NormalTok{,}
 \AttributeTok{input\_layers  =} \FunctionTok{c}\NormalTok{(}\StringTok{"./RasterGrids\_100m/2024/RAW/Edges\_Trees\_cell.tif"}\NormalTok{),}
 \AttributeTok{layer\_prefixes =} \FunctionTok{c}\NormalTok{(}\StringTok{"Edges\_Trees"}\NormalTok{),}
 \AttributeTok{output\_dir   =} \StringTok{"./RasterGrids\_100m/2024/RAW/"}\NormalTok{,}
 \AttributeTok{n\_workers   =} \DecValTok{12}\NormalTok{,}
 \AttributeTok{radii     =} \FunctionTok{c}\NormalTok{(}\StringTok{"r1250"}\NormalTok{),}
 \AttributeTok{radius\_mode  =} \StringTok{"sparse"}\NormalTok{,}
 \AttributeTok{extract\_fun  =} \StringTok{"sum"}\NormalTok{,}
 \AttributeTok{fill\_missing  =} \ConstantTok{TRUE}\NormalTok{,}
 \AttributeTok{IDW\_weight   =} \DecValTok{2}\NormalTok{,}
 \AttributeTok{future\_max\_size =} \DecValTok{20} \SpecialCharTok{*} \DecValTok{1024}\SpecialCharTok{\^{}}\DecValTok{3}\NormalTok{)}


\CommentTok{\# Edges\_Trees\_r1250.tif egv\_157 {-}{-}{-}{-}}
\NormalTok{slanis}\OtherTok{=}\FunctionTok{rast}\NormalTok{(}\StringTok{"./RasterGrids\_100m/2024/RAW/Edges\_Trees\_r1250.tif"}\NormalTok{)}
\FunctionTok{names}\NormalTok{(slanis)}\OtherTok{=}\StringTok{"egv\_157"}
\NormalTok{slanis2}\OtherTok{=}\FunctionTok{project}\NormalTok{(slanis,template100)}
\FunctionTok{writeRaster}\NormalTok{(slanis2,}
      \StringTok{"./RasterGrids\_100m/2024/RAW/Edges\_Trees\_r1250.tif"}\NormalTok{,}
      \AttributeTok{overwrite=}\ConstantTok{TRUE}\NormalTok{)}

\CommentTok{\# standardisation {-}{-}{-}{-}}
\ControlFlowTok{if}\NormalTok{(}\SpecialCharTok{!}\FunctionTok{require}\NormalTok{(terra)) \{}\FunctionTok{install.packages}\NormalTok{(}\StringTok{"terra"}\NormalTok{); }\FunctionTok{require}\NormalTok{(terra)\}}
\ControlFlowTok{if}\NormalTok{(}\SpecialCharTok{!}\FunctionTok{require}\NormalTok{(tidyverse)) \{}\FunctionTok{install.packages}\NormalTok{(}\StringTok{"tidyverse"}\NormalTok{); }\FunctionTok{require}\NormalTok{(tidyverse)\}}

\NormalTok{nosaukums}\OtherTok{=}\StringTok{"Edges\_Trees\_r1250.tif"}
\NormalTok{ielasisanas\_cels}\OtherTok{=}\FunctionTok{paste0}\NormalTok{(}\StringTok{"./RasterGrids\_100m/2024/RAW/"}\NormalTok{,nosaukums)}
\NormalTok{saglabasanas\_cels}\OtherTok{=}\FunctionTok{paste0}\NormalTok{(}\StringTok{"./RasterGrids\_100m/2024/Scaled/"}\NormalTok{,nosaukums)}
\NormalTok{slanis}\OtherTok{=}\FunctionTok{rast}\NormalTok{(ielasisanas\_cels)}
\NormalTok{videjais}\OtherTok{=}\FunctionTok{global}\NormalTok{(slanis,}\AttributeTok{fun=}\StringTok{"mean"}\NormalTok{,}\AttributeTok{na.rm=}\ConstantTok{TRUE}\NormalTok{)}
\NormalTok{centrets}\OtherTok{=}\NormalTok{slanis}\SpecialCharTok{{-}}\NormalTok{videjais[,}\DecValTok{1}\NormalTok{]}
\NormalTok{standartnovirze}\OtherTok{=}\NormalTok{terra}\SpecialCharTok{::}\FunctionTok{global}\NormalTok{(centrets,}\AttributeTok{fun=}\StringTok{"rms"}\NormalTok{,}\AttributeTok{na.rm=}\ConstantTok{TRUE}\NormalTok{)}
\NormalTok{merogots}\OtherTok{=}\NormalTok{centrets}\SpecialCharTok{/}\NormalTok{standartnovirze[,}\DecValTok{1}\NormalTok{]}
\FunctionTok{writeRaster}\NormalTok{(merogots,}
      \AttributeTok{filename=}\NormalTok{saglabasanas\_cels,}
      \AttributeTok{overwrite=}\ConstantTok{TRUE}\NormalTok{)}
\end{Highlighting}
\end{Shaded}

\section{Edges\_Trees\_r3000}\label{ch06.158}

\textbf{filename:} \texttt{Edges\_Trees\_r3000.tif}

\textbf{layername:} \texttt{egv\_158}

\textbf{English name:} Edge pixels of Trees within the 3 km landscape

\textbf{Latvian name:} Koku malu pikseļu skaits 3 km ainavā

\textbf{Procedure:} The total edge within a 3000 m radius around the analysis grid cell is
calculated as the area-weighted sum of the \hyperref[ch06.155]{analysis cells} inside the
buffer, using the workflow \texttt{egvtools::radius\_function()}. During the calculation of the landscape metric,
inverse distance weighted (power = 2) gap filling on the output is applied
to ensure no missing values at the edges. Then the layer is rewritten to set
its name. Finally, the layer is standardised by subtracting the arithmetic
mean and dividing by the root mean squared error.

\begin{Shaded}
\begin{Highlighting}[]
\CommentTok{\# libs {-}{-}{-}{-}}
\ControlFlowTok{if}\NormalTok{(}\SpecialCharTok{!}\FunctionTok{require}\NormalTok{(terra)) \{}\FunctionTok{install.packages}\NormalTok{(}\StringTok{"terra"}\NormalTok{); }\FunctionTok{require}\NormalTok{(terra)\}}
\ControlFlowTok{if}\NormalTok{(}\SpecialCharTok{!}\FunctionTok{require}\NormalTok{(egvtools)) \{remotes}\SpecialCharTok{::}\FunctionTok{install\_github}\NormalTok{(}\StringTok{"aavotins/egvtools"}\NormalTok{); }\FunctionTok{require}\NormalTok{(egvtools)\}}


\CommentTok{\# Templates {-}{-}{-}{-}{-}}
\NormalTok{template100}\OtherTok{=}\FunctionTok{rast}\NormalTok{(}\StringTok{"./Templates/TemplateRasters/LV100m\_10km.tif"}\NormalTok{)}

\CommentTok{\# radii {-}{-}{-}{-}}
\FunctionTok{radius\_function}\NormalTok{(}
 \AttributeTok{kvadrati\_path =} \StringTok{"./Templates/TemplateGrids/tiles/"}\NormalTok{,}
 \AttributeTok{radii\_path   =} \StringTok{"./Templates/TemplateGridPoints/tiles/"}\NormalTok{,}
 \AttributeTok{tikls100\_path =} \StringTok{"./Templates/TemplateGrids/tikls100\_sauzeme.parquet"}\NormalTok{,}
 \AttributeTok{template\_path =} \StringTok{"./Templates/TemplateRasters/LV100m\_10km.tif"}\NormalTok{,}
 \AttributeTok{input\_layers  =} \FunctionTok{c}\NormalTok{(}\StringTok{"./RasterGrids\_100m/2024/RAW/Edges\_Trees\_cell.tif"}\NormalTok{),}
 \AttributeTok{layer\_prefixes =} \FunctionTok{c}\NormalTok{(}\StringTok{"Edges\_Trees"}\NormalTok{),}
 \AttributeTok{output\_dir   =} \StringTok{"./RasterGrids\_100m/2024/RAW/"}\NormalTok{,}
 \AttributeTok{n\_workers   =} \DecValTok{12}\NormalTok{,}
 \AttributeTok{radii     =} \FunctionTok{c}\NormalTok{(}\StringTok{"r3000"}\NormalTok{),}
 \AttributeTok{radius\_mode  =} \StringTok{"sparse"}\NormalTok{,}
 \AttributeTok{extract\_fun  =} \StringTok{"sum"}\NormalTok{,}
 \AttributeTok{fill\_missing  =} \ConstantTok{TRUE}\NormalTok{,}
 \AttributeTok{IDW\_weight   =} \DecValTok{2}\NormalTok{,}
 \AttributeTok{future\_max\_size =} \DecValTok{20} \SpecialCharTok{*} \DecValTok{1024}\SpecialCharTok{\^{}}\DecValTok{3}\NormalTok{)}


\CommentTok{\# Edges\_Trees\_r3000.tif egv\_158 {-}{-}{-}{-}}
\NormalTok{slanis}\OtherTok{=}\FunctionTok{rast}\NormalTok{(}\StringTok{"./RasterGrids\_100m/2024/RAW/Edges\_Trees\_r3000.tif"}\NormalTok{)}
\FunctionTok{names}\NormalTok{(slanis)}\OtherTok{=}\StringTok{"egv\_158"}
\NormalTok{slanis2}\OtherTok{=}\FunctionTok{project}\NormalTok{(slanis,template100)}
\FunctionTok{writeRaster}\NormalTok{(slanis2,}
      \StringTok{"./RasterGrids\_100m/2024/RAW/Edges\_Trees\_r3000.tif"}\NormalTok{,}
      \AttributeTok{overwrite=}\ConstantTok{TRUE}\NormalTok{)}

\CommentTok{\# standardisation {-}{-}{-}{-}}
\ControlFlowTok{if}\NormalTok{(}\SpecialCharTok{!}\FunctionTok{require}\NormalTok{(terra)) \{}\FunctionTok{install.packages}\NormalTok{(}\StringTok{"terra"}\NormalTok{); }\FunctionTok{require}\NormalTok{(terra)\}}
\ControlFlowTok{if}\NormalTok{(}\SpecialCharTok{!}\FunctionTok{require}\NormalTok{(tidyverse)) \{}\FunctionTok{install.packages}\NormalTok{(}\StringTok{"tidyverse"}\NormalTok{); }\FunctionTok{require}\NormalTok{(tidyverse)\}}

\NormalTok{nosaukums}\OtherTok{=}\StringTok{"Edges\_Trees\_r3000.tif"}
\NormalTok{ielasisanas\_cels}\OtherTok{=}\FunctionTok{paste0}\NormalTok{(}\StringTok{"./RasterGrids\_100m/2024/RAW/"}\NormalTok{,nosaukums)}
\NormalTok{saglabasanas\_cels}\OtherTok{=}\FunctionTok{paste0}\NormalTok{(}\StringTok{"./RasterGrids\_100m/2024/Scaled/"}\NormalTok{,nosaukums)}
\NormalTok{slanis}\OtherTok{=}\FunctionTok{rast}\NormalTok{(ielasisanas\_cels)}
\NormalTok{videjais}\OtherTok{=}\FunctionTok{global}\NormalTok{(slanis,}\AttributeTok{fun=}\StringTok{"mean"}\NormalTok{,}\AttributeTok{na.rm=}\ConstantTok{TRUE}\NormalTok{)}
\NormalTok{centrets}\OtherTok{=}\NormalTok{slanis}\SpecialCharTok{{-}}\NormalTok{videjais[,}\DecValTok{1}\NormalTok{]}
\NormalTok{standartnovirze}\OtherTok{=}\NormalTok{terra}\SpecialCharTok{::}\FunctionTok{global}\NormalTok{(centrets,}\AttributeTok{fun=}\StringTok{"rms"}\NormalTok{,}\AttributeTok{na.rm=}\ConstantTok{TRUE}\NormalTok{)}
\NormalTok{merogots}\OtherTok{=}\NormalTok{centrets}\SpecialCharTok{/}\NormalTok{standartnovirze[,}\DecValTok{1}\NormalTok{]}
\FunctionTok{writeRaster}\NormalTok{(merogots,}
      \AttributeTok{filename=}\NormalTok{saglabasanas\_cels,}
      \AttributeTok{overwrite=}\ConstantTok{TRUE}\NormalTok{)}
\end{Highlighting}
\end{Shaded}

\section{Edges\_Trees\_r10000}\label{ch06.159}

\textbf{filename:} \texttt{Edges\_Trees\_r10000.tif}

\textbf{layername:} \texttt{egv\_159}

\textbf{English name:} Edge pixels of Trees within the 10 km landscape

\textbf{Latvian name:} Koku malu pikseļu skaits 10 km ainavā

\textbf{Procedure:} The total edge within a 10000 m radius around the analysis grid cell is
calculated as the area-weighted sum of the \hyperref[ch06.155]{analysis cells} inside the
buffer, using the workflow \texttt{egvtools::radius\_function()}. During the calculation of the landscape metric,
inverse distance weighted (power = 2) gap filling on the output is applied
to ensure no missing values at the edges. Then the layer is rewritten to set
its name. Finally, the layer is standardised by subtracting the arithmetic
mean and dividing by the root mean squared error.

\begin{Shaded}
\begin{Highlighting}[]
\CommentTok{\# libs {-}{-}{-}{-}}
\ControlFlowTok{if}\NormalTok{(}\SpecialCharTok{!}\FunctionTok{require}\NormalTok{(terra)) \{}\FunctionTok{install.packages}\NormalTok{(}\StringTok{"terra"}\NormalTok{); }\FunctionTok{require}\NormalTok{(terra)\}}
\ControlFlowTok{if}\NormalTok{(}\SpecialCharTok{!}\FunctionTok{require}\NormalTok{(egvtools)) \{remotes}\SpecialCharTok{::}\FunctionTok{install\_github}\NormalTok{(}\StringTok{"aavotins/egvtools"}\NormalTok{); }\FunctionTok{require}\NormalTok{(egvtools)\}}


\CommentTok{\# Templates {-}{-}{-}{-}{-}}
\NormalTok{template100}\OtherTok{=}\FunctionTok{rast}\NormalTok{(}\StringTok{"./Templates/TemplateRasters/LV100m\_10km.tif"}\NormalTok{)}

\CommentTok{\# radii {-}{-}{-}{-}}
\FunctionTok{radius\_function}\NormalTok{(}
 \AttributeTok{kvadrati\_path =} \StringTok{"./Templates/TemplateGrids/tiles/"}\NormalTok{,}
 \AttributeTok{radii\_path   =} \StringTok{"./Templates/TemplateGridPoints/tiles/"}\NormalTok{,}
 \AttributeTok{tikls100\_path =} \StringTok{"./Templates/TemplateGrids/tikls100\_sauzeme.parquet"}\NormalTok{,}
 \AttributeTok{template\_path =} \StringTok{"./Templates/TemplateRasters/LV100m\_10km.tif"}\NormalTok{,}
 \AttributeTok{input\_layers  =} \FunctionTok{c}\NormalTok{(}\StringTok{"./RasterGrids\_100m/2024/RAW/Edges\_Trees\_cell.tif"}\NormalTok{),}
 \AttributeTok{layer\_prefixes =} \FunctionTok{c}\NormalTok{(}\StringTok{"Edges\_Trees"}\NormalTok{),}
 \AttributeTok{output\_dir   =} \StringTok{"./RasterGrids\_100m/2024/RAW/"}\NormalTok{,}
 \AttributeTok{n\_workers   =} \DecValTok{12}\NormalTok{,}
 \AttributeTok{radii     =} \FunctionTok{c}\NormalTok{(}\StringTok{"r3000"}\NormalTok{),}
 \AttributeTok{radius\_mode  =} \StringTok{"sparse"}\NormalTok{,}
 \AttributeTok{extract\_fun  =} \StringTok{"sum"}\NormalTok{,}
 \AttributeTok{fill\_missing  =} \ConstantTok{TRUE}\NormalTok{,}
 \AttributeTok{IDW\_weight   =} \DecValTok{2}\NormalTok{,}
 \AttributeTok{future\_max\_size =} \DecValTok{20} \SpecialCharTok{*} \DecValTok{1024}\SpecialCharTok{\^{}}\DecValTok{3}\NormalTok{)}


\CommentTok{\# Edges\_Trees\_r10000.tif    egv\_159 {-}{-}{-}{-}}
\NormalTok{slanis}\OtherTok{=}\FunctionTok{rast}\NormalTok{(}\StringTok{"./RasterGrids\_100m/2024/RAW/Edges\_Trees\_r10000.tif"}\NormalTok{)}
\FunctionTok{names}\NormalTok{(slanis)}\OtherTok{=}\StringTok{"egv\_159"}
\NormalTok{slanis2}\OtherTok{=}\FunctionTok{project}\NormalTok{(slanis,template100)}
\FunctionTok{writeRaster}\NormalTok{(slanis2,}
      \StringTok{"./RasterGrids\_100m/2024/RAW/Edges\_Trees\_r10000.tif"}\NormalTok{,}
      \AttributeTok{overwrite=}\ConstantTok{TRUE}\NormalTok{)}

\CommentTok{\# standardisation {-}{-}{-}{-}}
\ControlFlowTok{if}\NormalTok{(}\SpecialCharTok{!}\FunctionTok{require}\NormalTok{(terra)) \{}\FunctionTok{install.packages}\NormalTok{(}\StringTok{"terra"}\NormalTok{); }\FunctionTok{require}\NormalTok{(terra)\}}
\ControlFlowTok{if}\NormalTok{(}\SpecialCharTok{!}\FunctionTok{require}\NormalTok{(tidyverse)) \{}\FunctionTok{install.packages}\NormalTok{(}\StringTok{"tidyverse"}\NormalTok{); }\FunctionTok{require}\NormalTok{(tidyverse)\}}

\NormalTok{nosaukums}\OtherTok{=}\StringTok{"Edges\_Trees\_r10000.tif"}
\NormalTok{ielasisanas\_cels}\OtherTok{=}\FunctionTok{paste0}\NormalTok{(}\StringTok{"./RasterGrids\_100m/2024/RAW/"}\NormalTok{,nosaukums)}
\NormalTok{saglabasanas\_cels}\OtherTok{=}\FunctionTok{paste0}\NormalTok{(}\StringTok{"./RasterGrids\_100m/2024/Scaled/"}\NormalTok{,nosaukums)}
\NormalTok{slanis}\OtherTok{=}\FunctionTok{rast}\NormalTok{(ielasisanas\_cels)}
\NormalTok{videjais}\OtherTok{=}\FunctionTok{global}\NormalTok{(slanis,}\AttributeTok{fun=}\StringTok{"mean"}\NormalTok{,}\AttributeTok{na.rm=}\ConstantTok{TRUE}\NormalTok{)}
\NormalTok{centrets}\OtherTok{=}\NormalTok{slanis}\SpecialCharTok{{-}}\NormalTok{videjais[,}\DecValTok{1}\NormalTok{]}
\NormalTok{standartnovirze}\OtherTok{=}\NormalTok{terra}\SpecialCharTok{::}\FunctionTok{global}\NormalTok{(centrets,}\AttributeTok{fun=}\StringTok{"rms"}\NormalTok{,}\AttributeTok{na.rm=}\ConstantTok{TRUE}\NormalTok{)}
\NormalTok{merogots}\OtherTok{=}\NormalTok{centrets}\SpecialCharTok{/}\NormalTok{standartnovirze[,}\DecValTok{1}\NormalTok{]}
\FunctionTok{writeRaster}\NormalTok{(merogots,}
      \AttributeTok{filename=}\NormalTok{saglabasanas\_cels,}
      \AttributeTok{overwrite=}\ConstantTok{TRUE}\NormalTok{)}
\end{Highlighting}
\end{Shaded}

\section{Edges\_Water\_cell}\label{ch06.160}

\textbf{filename:} \texttt{Edges\_Water\_cell.tif}

\textbf{layername:} \texttt{egv\_160}

\textbf{English name:} Edge pixels of Water within the analysis cell (1 ha)

\textbf{Latvian name:} Ūdenstilpju malu pikseļu skaits analīzes šūnā (1 ha)

\textbf{Procedure:} First, values equal to 200 from the \hyperref[Ch05.03]{Landscape
classification} are coded as 1 and everything else as NA. Then, the
layer (1 = presence) is covered over the nulls layer (presence = 0) and written to
file (matching the input). Next, with the workflow
\texttt{egvtools::landscape\_function()} total edge between the two classes is
calculated. During the calculation of the landscape metric, inverse distance weighted
(power = 2) gap filling on the output is applied to ensure no missing values
at the edges. Finally, the layer is standardised by subtracting the arithmetic
mean and dividing by the root mean squared error.

\begin{Shaded}
\begin{Highlighting}[]
\CommentTok{\# libs {-}{-}{-}{-}}
\ControlFlowTok{if}\NormalTok{(}\SpecialCharTok{!}\FunctionTok{require}\NormalTok{(terra)) \{}\FunctionTok{install.packages}\NormalTok{(}\StringTok{"terra"}\NormalTok{); }\FunctionTok{require}\NormalTok{(terra)\}}
\ControlFlowTok{if}\NormalTok{(}\SpecialCharTok{!}\FunctionTok{require}\NormalTok{(egvtools)) \{remotes}\SpecialCharTok{::}\FunctionTok{install\_github}\NormalTok{(}\StringTok{"aavotins/egvtools"}\NormalTok{); }\FunctionTok{require}\NormalTok{(egvtools)\}}

\ControlFlowTok{if}\NormalTok{(}\SpecialCharTok{!}\FunctionTok{require}\NormalTok{(sf)) \{}\FunctionTok{install.packages}\NormalTok{(}\StringTok{"sf"}\NormalTok{); }\FunctionTok{require}\NormalTok{(sf)\}}
\ControlFlowTok{if}\NormalTok{(}\SpecialCharTok{!}\FunctionTok{require}\NormalTok{(sfarrow)) \{}\FunctionTok{install.packages}\NormalTok{(}\StringTok{"sfarrow"}\NormalTok{); }\FunctionTok{require}\NormalTok{(sfarrow)\}}
\ControlFlowTok{if}\NormalTok{(}\SpecialCharTok{!}\FunctionTok{require}\NormalTok{(raster)) \{}\FunctionTok{install.packages}\NormalTok{(}\StringTok{"raster"}\NormalTok{); }\FunctionTok{require}\NormalTok{(raster)\}}
\ControlFlowTok{if}\NormalTok{(}\SpecialCharTok{!}\FunctionTok{require}\NormalTok{(fasterize)) \{}\FunctionTok{install.packages}\NormalTok{(}\StringTok{"fasterize"}\NormalTok{); }\FunctionTok{require}\NormalTok{(fasterize)\}}
\ControlFlowTok{if}\NormalTok{(}\SpecialCharTok{!}\FunctionTok{require}\NormalTok{(tidyverse)) \{}\FunctionTok{install.packages}\NormalTok{(}\StringTok{"tidyverse"}\NormalTok{); }\FunctionTok{require}\NormalTok{(tidyverse)\}}


\CommentTok{\# Templates {-}{-}{-}{-}{-}}
\NormalTok{template10}\OtherTok{=}\FunctionTok{rast}\NormalTok{(}\StringTok{"./Templates/TemplateRasters/LV10m\_10km.tif"}\NormalTok{)}
\NormalTok{nulls10}\OtherTok{=}\FunctionTok{rast}\NormalTok{(}\StringTok{"./Templates/TemplateRasters/nulls\_LV10m\_10km.tif"}\NormalTok{)}

\CommentTok{\# simple landscape {-}{-}{-}{-}}
\NormalTok{simple\_landscape}\OtherTok{=}\FunctionTok{rast}\NormalTok{(}\StringTok{"./RasterGrids\_10m/2024/Ainava\_vienk\_mask.tif"}\NormalTok{)}

\CommentTok{\# Edges\_Water\_input.tif {-}{-}{-}{-}}
\NormalTok{water}\OtherTok{=}\FunctionTok{ifel}\NormalTok{(simple\_landscape}\SpecialCharTok{==}\DecValTok{200}\NormalTok{,}\DecValTok{1}\NormalTok{,}\DecValTok{0}\NormalTok{)}
\FunctionTok{plot}\NormalTok{(water)}
\NormalTok{water}\OtherTok{=}\FunctionTok{cover}\NormalTok{(water,nulls10)}
\FunctionTok{plot}\NormalTok{(water)}

\NormalTok{edge\_water}\OtherTok{=}\FunctionTok{project}\NormalTok{(water,template10,}
              \AttributeTok{filename=}\StringTok{"./RasterGrids\_10m/2024/Edges\_Water\_input.tif"}\NormalTok{,}
              \AttributeTok{overwrite=}\ConstantTok{TRUE}\NormalTok{)}


\CommentTok{\# Edges\_Water\_cell.tif  egv\_160 {-}{-}{-}{-}}
\FunctionTok{landscape\_function}\NormalTok{(}
 \AttributeTok{landscape   =} \StringTok{"./RasterGrids\_10m/2024/Edges\_Water\_input.tif"}\NormalTok{,}
 \AttributeTok{zones     =} \StringTok{"./Templates/TemplateGrids/tikls100\_sauzeme.parquet"}\NormalTok{,}
 \AttributeTok{id\_field    =} \StringTok{"id"}\NormalTok{,}
 \AttributeTok{tile\_field   =} \StringTok{"tks50km"}\NormalTok{,}
 \AttributeTok{template    =} \StringTok{"./Templates/TemplateRasters/LV100m\_10km.tif"}\NormalTok{,}
 \AttributeTok{out\_dir    =} \StringTok{"./RasterGrids\_100m/2024/RAW"}\NormalTok{,}
 \AttributeTok{out\_filename  =} \StringTok{"Edges\_Water\_cell.tif"}\NormalTok{,}
 \AttributeTok{out\_layername =} \StringTok{"egv\_160"}\NormalTok{,}
 \AttributeTok{what       =} \StringTok{"lsm\_l\_te"}\NormalTok{,}
 \AttributeTok{lm\_args     =} \FunctionTok{list}\NormalTok{(}\AttributeTok{count\_boundary =} \ConstantTok{FALSE}\NormalTok{),}
 \AttributeTok{rasterize\_engine =} \StringTok{"fasterize"}\NormalTok{,}
 \AttributeTok{n\_workers   =} \DecValTok{12}\NormalTok{,}
 \AttributeTok{future\_max\_size =} \DecValTok{20} \SpecialCharTok{*} \DecValTok{1024}\SpecialCharTok{\^{}}\DecValTok{3}\NormalTok{,}
 \AttributeTok{fill\_gaps   =} \ConstantTok{TRUE}\NormalTok{,}
 \AttributeTok{plot\_gaps   =} \ConstantTok{FALSE}\NormalTok{,}
 \AttributeTok{plot\_result  =} \ConstantTok{FALSE}
\NormalTok{)}

\CommentTok{\# standardisation {-}{-}{-}{-}}
\ControlFlowTok{if}\NormalTok{(}\SpecialCharTok{!}\FunctionTok{require}\NormalTok{(terra)) \{}\FunctionTok{install.packages}\NormalTok{(}\StringTok{"terra"}\NormalTok{); }\FunctionTok{require}\NormalTok{(terra)\}}
\ControlFlowTok{if}\NormalTok{(}\SpecialCharTok{!}\FunctionTok{require}\NormalTok{(tidyverse)) \{}\FunctionTok{install.packages}\NormalTok{(}\StringTok{"tidyverse"}\NormalTok{); }\FunctionTok{require}\NormalTok{(tidyverse)\}}

\NormalTok{nosaukums}\OtherTok{=}\StringTok{"Edges\_Water\_cell.tif"}
\NormalTok{ielasisanas\_cels}\OtherTok{=}\FunctionTok{paste0}\NormalTok{(}\StringTok{"./RasterGrids\_100m/2024/RAW/"}\NormalTok{,nosaukums)}
\NormalTok{saglabasanas\_cels}\OtherTok{=}\FunctionTok{paste0}\NormalTok{(}\StringTok{"./RasterGrids\_100m/2024/Scaled/"}\NormalTok{,nosaukums)}
\NormalTok{slanis}\OtherTok{=}\FunctionTok{rast}\NormalTok{(ielasisanas\_cels)}
\NormalTok{videjais}\OtherTok{=}\FunctionTok{global}\NormalTok{(slanis,}\AttributeTok{fun=}\StringTok{"mean"}\NormalTok{,}\AttributeTok{na.rm=}\ConstantTok{TRUE}\NormalTok{)}
\NormalTok{centrets}\OtherTok{=}\NormalTok{slanis}\SpecialCharTok{{-}}\NormalTok{videjais[,}\DecValTok{1}\NormalTok{]}
\NormalTok{standartnovirze}\OtherTok{=}\NormalTok{terra}\SpecialCharTok{::}\FunctionTok{global}\NormalTok{(centrets,}\AttributeTok{fun=}\StringTok{"rms"}\NormalTok{,}\AttributeTok{na.rm=}\ConstantTok{TRUE}\NormalTok{)}
\NormalTok{merogots}\OtherTok{=}\NormalTok{centrets}\SpecialCharTok{/}\NormalTok{standartnovirze[,}\DecValTok{1}\NormalTok{]}
\FunctionTok{writeRaster}\NormalTok{(merogots,}
      \AttributeTok{filename=}\NormalTok{saglabasanas\_cels,}
      \AttributeTok{overwrite=}\ConstantTok{TRUE}\NormalTok{)}
\end{Highlighting}
\end{Shaded}

\section{Edges\_Water\_r500}\label{ch06.161}

\textbf{filename:} \texttt{Edges\_Water\_r500.tif}

\textbf{layername:} \texttt{egv\_161}

\textbf{English name:} Edge pixels of Water within the 0.5 km landscape

\textbf{Latvian name:} Ūdenstilpju malu pikseļu skaits 0,5 km ainavā

\textbf{Procedure:} The total edge within a 500 m radius around the analysis grid cell is
calculated as the area-weighted sum of the \hyperref[ch06.160]{analysis cells} inside the
buffer, using the workflow \texttt{egvtools::radius\_function()}. During the calculation of the landscape metric,
inverse distance weighted (power = 2) gap filling on the output is applied
to ensure no missing values at the edges. Then the layer is rewritten to set
its name. Finally, the layer is standardised by subtracting the arithmetic
mean and dividing by the root mean squared error.

\begin{Shaded}
\begin{Highlighting}[]
\CommentTok{\# libs {-}{-}{-}{-}}
\ControlFlowTok{if}\NormalTok{(}\SpecialCharTok{!}\FunctionTok{require}\NormalTok{(terra)) \{}\FunctionTok{install.packages}\NormalTok{(}\StringTok{"terra"}\NormalTok{); }\FunctionTok{require}\NormalTok{(terra)\}}
\ControlFlowTok{if}\NormalTok{(}\SpecialCharTok{!}\FunctionTok{require}\NormalTok{(egvtools)) \{remotes}\SpecialCharTok{::}\FunctionTok{install\_github}\NormalTok{(}\StringTok{"aavotins/egvtools"}\NormalTok{); }\FunctionTok{require}\NormalTok{(egvtools)\}}


\CommentTok{\# Templates {-}{-}{-}{-}{-}}
\NormalTok{template100}\OtherTok{=}\FunctionTok{rast}\NormalTok{(}\StringTok{"./Templates/TemplateRasters/LV100m\_10km.tif"}\NormalTok{)}

\CommentTok{\# radii {-}{-}{-}{-}}
\FunctionTok{radius\_function}\NormalTok{(}
 \AttributeTok{kvadrati\_path =} \StringTok{"./Templates/TemplateGrids/tiles/"}\NormalTok{,}
 \AttributeTok{radii\_path   =} \StringTok{"./Templates/TemplateGridPoints/tiles/"}\NormalTok{,}
 \AttributeTok{tikls100\_path =} \StringTok{"./Templates/TemplateGrids/tikls100\_sauzeme.parquet"}\NormalTok{,}
 \AttributeTok{template\_path =} \StringTok{"./Templates/TemplateRasters/LV100m\_10km.tif"}\NormalTok{,}
 \AttributeTok{input\_layers  =} \FunctionTok{c}\NormalTok{(}\StringTok{"./RasterGrids\_100m/2024/RAW/Edges\_Water\_cell.tif"}\NormalTok{),}
 \AttributeTok{layer\_prefixes =} \FunctionTok{c}\NormalTok{(}\StringTok{"Edges\_Water"}\NormalTok{),}
 \AttributeTok{output\_dir   =} \StringTok{"./RasterGrids\_100m/2024/RAW/"}\NormalTok{,}
 \AttributeTok{n\_workers   =} \DecValTok{12}\NormalTok{,}
 \AttributeTok{radii     =} \FunctionTok{c}\NormalTok{(}\StringTok{"r500"}\NormalTok{),}
 \AttributeTok{radius\_mode  =} \StringTok{"sparse"}\NormalTok{,}
 \AttributeTok{extract\_fun  =} \StringTok{"sum"}\NormalTok{,}
 \AttributeTok{fill\_missing  =} \ConstantTok{TRUE}\NormalTok{,}
 \AttributeTok{IDW\_weight   =} \DecValTok{2}\NormalTok{,}
 \AttributeTok{future\_max\_size =} \DecValTok{20} \SpecialCharTok{*} \DecValTok{1024}\SpecialCharTok{\^{}}\DecValTok{3}\NormalTok{)}


\CommentTok{\# Edges\_Water\_r500.tif  egv\_161 {-}{-}{-}{-}}
\NormalTok{slanis}\OtherTok{=}\FunctionTok{rast}\NormalTok{(}\StringTok{"./RasterGrids\_100m/2024/RAW/Edges\_Water\_r500.tif"}\NormalTok{)}
\FunctionTok{names}\NormalTok{(slanis)}\OtherTok{=}\StringTok{"egv\_161"}
\NormalTok{slanis2}\OtherTok{=}\FunctionTok{project}\NormalTok{(slanis,template100)}
\FunctionTok{writeRaster}\NormalTok{(slanis2,}
      \StringTok{"./RasterGrids\_100m/2024/RAW/Edges\_Water\_r500.tif"}\NormalTok{,}
      \AttributeTok{overwrite=}\ConstantTok{TRUE}\NormalTok{)}

\CommentTok{\# standardisation {-}{-}{-}{-}}
\ControlFlowTok{if}\NormalTok{(}\SpecialCharTok{!}\FunctionTok{require}\NormalTok{(terra)) \{}\FunctionTok{install.packages}\NormalTok{(}\StringTok{"terra"}\NormalTok{); }\FunctionTok{require}\NormalTok{(terra)\}}
\ControlFlowTok{if}\NormalTok{(}\SpecialCharTok{!}\FunctionTok{require}\NormalTok{(tidyverse)) \{}\FunctionTok{install.packages}\NormalTok{(}\StringTok{"tidyverse"}\NormalTok{); }\FunctionTok{require}\NormalTok{(tidyverse)\}}

\NormalTok{nosaukums}\OtherTok{=}\StringTok{"Edges\_Water\_r500.tif"}
\NormalTok{ielasisanas\_cels}\OtherTok{=}\FunctionTok{paste0}\NormalTok{(}\StringTok{"./RasterGrids\_100m/2024/RAW/"}\NormalTok{,nosaukums)}
\NormalTok{saglabasanas\_cels}\OtherTok{=}\FunctionTok{paste0}\NormalTok{(}\StringTok{"./RasterGrids\_100m/2024/Scaled/"}\NormalTok{,nosaukums)}
\NormalTok{slanis}\OtherTok{=}\FunctionTok{rast}\NormalTok{(ielasisanas\_cels)}
\NormalTok{videjais}\OtherTok{=}\FunctionTok{global}\NormalTok{(slanis,}\AttributeTok{fun=}\StringTok{"mean"}\NormalTok{,}\AttributeTok{na.rm=}\ConstantTok{TRUE}\NormalTok{)}
\NormalTok{centrets}\OtherTok{=}\NormalTok{slanis}\SpecialCharTok{{-}}\NormalTok{videjais[,}\DecValTok{1}\NormalTok{]}
\NormalTok{standartnovirze}\OtherTok{=}\NormalTok{terra}\SpecialCharTok{::}\FunctionTok{global}\NormalTok{(centrets,}\AttributeTok{fun=}\StringTok{"rms"}\NormalTok{,}\AttributeTok{na.rm=}\ConstantTok{TRUE}\NormalTok{)}
\NormalTok{merogots}\OtherTok{=}\NormalTok{centrets}\SpecialCharTok{/}\NormalTok{standartnovirze[,}\DecValTok{1}\NormalTok{]}
\FunctionTok{writeRaster}\NormalTok{(merogots,}
      \AttributeTok{filename=}\NormalTok{saglabasanas\_cels,}
      \AttributeTok{overwrite=}\ConstantTok{TRUE}\NormalTok{)}
\end{Highlighting}
\end{Shaded}

\section{Edges\_Water\_r1250}\label{ch06.162}

\textbf{filename:} \texttt{Edges\_Water\_r1250.tif}

\textbf{layername:} \texttt{egv\_162}

\textbf{English name:} Edge pixels of Water within the 1.25 km landscape

\textbf{Latvian name:} Ūdenstilpju malu pikseļu skaits 1,25 km ainavā

\textbf{Procedure:} The total edge within a 1250 m radius around the analysis grid cell is
calculated as the area-weighted sum of the \hyperref[ch06.160]{analysis cells} inside the
buffer, using the workflow \texttt{egvtools::radius\_function()}. During the calculation of the landscape metric,
inverse distance weighted (power = 2) gap filling on the output is applied
to ensure no missing values at the edges. Then the layer is rewritten to set
its name. Finally, the layer is standardised by subtracting the arithmetic
mean and dividing by the root mean squared error.

\begin{Shaded}
\begin{Highlighting}[]
\CommentTok{\# libs {-}{-}{-}{-}}
\ControlFlowTok{if}\NormalTok{(}\SpecialCharTok{!}\FunctionTok{require}\NormalTok{(terra)) \{}\FunctionTok{install.packages}\NormalTok{(}\StringTok{"terra"}\NormalTok{); }\FunctionTok{require}\NormalTok{(terra)\}}
\ControlFlowTok{if}\NormalTok{(}\SpecialCharTok{!}\FunctionTok{require}\NormalTok{(egvtools)) \{remotes}\SpecialCharTok{::}\FunctionTok{install\_github}\NormalTok{(}\StringTok{"aavotins/egvtools"}\NormalTok{); }\FunctionTok{require}\NormalTok{(egvtools)\}}


\CommentTok{\# Templates {-}{-}{-}{-}{-}}
\NormalTok{template100}\OtherTok{=}\FunctionTok{rast}\NormalTok{(}\StringTok{"./Templates/TemplateRasters/LV100m\_10km.tif"}\NormalTok{)}

\CommentTok{\# radii {-}{-}{-}{-}}
\FunctionTok{radius\_function}\NormalTok{(}
 \AttributeTok{kvadrati\_path =} \StringTok{"./Templates/TemplateGrids/tiles/"}\NormalTok{,}
 \AttributeTok{radii\_path   =} \StringTok{"./Templates/TemplateGridPoints/tiles/"}\NormalTok{,}
 \AttributeTok{tikls100\_path =} \StringTok{"./Templates/TemplateGrids/tikls100\_sauzeme.parquet"}\NormalTok{,}
 \AttributeTok{template\_path =} \StringTok{"./Templates/TemplateRasters/LV100m\_10km.tif"}\NormalTok{,}
 \AttributeTok{input\_layers  =} \FunctionTok{c}\NormalTok{(}\StringTok{"./RasterGrids\_100m/2024/RAW/Edges\_Water\_cell.tif"}\NormalTok{),}
 \AttributeTok{layer\_prefixes =} \FunctionTok{c}\NormalTok{(}\StringTok{"Edges\_Water"}\NormalTok{),}
 \AttributeTok{output\_dir   =} \StringTok{"./RasterGrids\_100m/2024/RAW/"}\NormalTok{,}
 \AttributeTok{n\_workers   =} \DecValTok{12}\NormalTok{,}
 \AttributeTok{radii     =} \FunctionTok{c}\NormalTok{(}\StringTok{"r1250"}\NormalTok{),}
 \AttributeTok{radius\_mode  =} \StringTok{"sparse"}\NormalTok{,}
 \AttributeTok{extract\_fun  =} \StringTok{"sum"}\NormalTok{,}
 \AttributeTok{fill\_missing  =} \ConstantTok{TRUE}\NormalTok{,}
 \AttributeTok{IDW\_weight   =} \DecValTok{2}\NormalTok{,}
 \AttributeTok{future\_max\_size =} \DecValTok{20} \SpecialCharTok{*} \DecValTok{1024}\SpecialCharTok{\^{}}\DecValTok{3}\NormalTok{)}


\CommentTok{\# Edges\_Water\_r1250.tif egv\_162 {-}{-}{-}{-}}
\NormalTok{slanis}\OtherTok{=}\FunctionTok{rast}\NormalTok{(}\StringTok{"./RasterGrids\_100m/2024/RAW/Edges\_Water\_r1250.tif"}\NormalTok{)}
\FunctionTok{names}\NormalTok{(slanis)}\OtherTok{=}\StringTok{"egv\_162"}
\NormalTok{slanis2}\OtherTok{=}\FunctionTok{project}\NormalTok{(slanis,template100)}
\FunctionTok{writeRaster}\NormalTok{(slanis2,}
      \StringTok{"./RasterGrids\_100m/2024/RAW/Edges\_Water\_r1250.tif"}\NormalTok{,}
      \AttributeTok{overwrite=}\ConstantTok{TRUE}\NormalTok{)}

\CommentTok{\# standardisation {-}{-}{-}{-}}
\ControlFlowTok{if}\NormalTok{(}\SpecialCharTok{!}\FunctionTok{require}\NormalTok{(terra)) \{}\FunctionTok{install.packages}\NormalTok{(}\StringTok{"terra"}\NormalTok{); }\FunctionTok{require}\NormalTok{(terra)\}}
\ControlFlowTok{if}\NormalTok{(}\SpecialCharTok{!}\FunctionTok{require}\NormalTok{(tidyverse)) \{}\FunctionTok{install.packages}\NormalTok{(}\StringTok{"tidyverse"}\NormalTok{); }\FunctionTok{require}\NormalTok{(tidyverse)\}}

\NormalTok{nosaukums}\OtherTok{=}\StringTok{"Edges\_Water\_r1250.tif"}
\NormalTok{ielasisanas\_cels}\OtherTok{=}\FunctionTok{paste0}\NormalTok{(}\StringTok{"./RasterGrids\_100m/2024/RAW/"}\NormalTok{,nosaukums)}
\NormalTok{saglabasanas\_cels}\OtherTok{=}\FunctionTok{paste0}\NormalTok{(}\StringTok{"./RasterGrids\_100m/2024/Scaled/"}\NormalTok{,nosaukums)}
\NormalTok{slanis}\OtherTok{=}\FunctionTok{rast}\NormalTok{(ielasisanas\_cels)}
\NormalTok{videjais}\OtherTok{=}\FunctionTok{global}\NormalTok{(slanis,}\AttributeTok{fun=}\StringTok{"mean"}\NormalTok{,}\AttributeTok{na.rm=}\ConstantTok{TRUE}\NormalTok{)}
\NormalTok{centrets}\OtherTok{=}\NormalTok{slanis}\SpecialCharTok{{-}}\NormalTok{videjais[,}\DecValTok{1}\NormalTok{]}
\NormalTok{standartnovirze}\OtherTok{=}\NormalTok{terra}\SpecialCharTok{::}\FunctionTok{global}\NormalTok{(centrets,}\AttributeTok{fun=}\StringTok{"rms"}\NormalTok{,}\AttributeTok{na.rm=}\ConstantTok{TRUE}\NormalTok{)}
\NormalTok{merogots}\OtherTok{=}\NormalTok{centrets}\SpecialCharTok{/}\NormalTok{standartnovirze[,}\DecValTok{1}\NormalTok{]}
\FunctionTok{writeRaster}\NormalTok{(merogots,}
      \AttributeTok{filename=}\NormalTok{saglabasanas\_cels,}
      \AttributeTok{overwrite=}\ConstantTok{TRUE}\NormalTok{)}
\end{Highlighting}
\end{Shaded}

\section{Edges\_Water\_r3000}\label{ch06.163}

\textbf{filename:} \texttt{Edges\_Water\_r3000.tif}

\textbf{layername:} \texttt{egv\_163}

\textbf{English name:} Edge pixels of Water within the 3 km landscape

\textbf{Latvian name:} Ūdenstilpju malu pikseļu skaits 3 km ainavā

\textbf{Procedure:} The total edge within a 3000 m radius around the analysis grid cell is
calculated as the area-weighted sum of the \hyperref[ch06.160]{analysis cells} inside the
buffer, using the workflow \texttt{egvtools::radius\_function()}. During the calculation of the landscape metric,
inverse distance weighted (power = 2) gap filling on the output is applied
to ensure no missing values at the edges. Then the layer is rewritten to set
its name. Finally, the layer is standardised by subtracting the arithmetic
mean and dividing by the root mean squared error.

\begin{Shaded}
\begin{Highlighting}[]
\CommentTok{\# libs {-}{-}{-}{-}}
\ControlFlowTok{if}\NormalTok{(}\SpecialCharTok{!}\FunctionTok{require}\NormalTok{(terra)) \{}\FunctionTok{install.packages}\NormalTok{(}\StringTok{"terra"}\NormalTok{); }\FunctionTok{require}\NormalTok{(terra)\}}
\ControlFlowTok{if}\NormalTok{(}\SpecialCharTok{!}\FunctionTok{require}\NormalTok{(egvtools)) \{remotes}\SpecialCharTok{::}\FunctionTok{install\_github}\NormalTok{(}\StringTok{"aavotins/egvtools"}\NormalTok{); }\FunctionTok{require}\NormalTok{(egvtools)\}}


\CommentTok{\# Templates {-}{-}{-}{-}{-}}
\NormalTok{template100}\OtherTok{=}\FunctionTok{rast}\NormalTok{(}\StringTok{"./Templates/TemplateRasters/LV100m\_10km.tif"}\NormalTok{)}

\CommentTok{\# radii {-}{-}{-}{-}}
\FunctionTok{radius\_function}\NormalTok{(}
 \AttributeTok{kvadrati\_path =} \StringTok{"./Templates/TemplateGrids/tiles/"}\NormalTok{,}
 \AttributeTok{radii\_path   =} \StringTok{"./Templates/TemplateGridPoints/tiles/"}\NormalTok{,}
 \AttributeTok{tikls100\_path =} \StringTok{"./Templates/TemplateGrids/tikls100\_sauzeme.parquet"}\NormalTok{,}
 \AttributeTok{template\_path =} \StringTok{"./Templates/TemplateRasters/LV100m\_10km.tif"}\NormalTok{,}
 \AttributeTok{input\_layers  =} \FunctionTok{c}\NormalTok{(}\StringTok{"./RasterGrids\_100m/2024/RAW/Edges\_Water\_cell.tif"}\NormalTok{),}
 \AttributeTok{layer\_prefixes =} \FunctionTok{c}\NormalTok{(}\StringTok{"Edges\_Water"}\NormalTok{),}
 \AttributeTok{output\_dir   =} \StringTok{"./RasterGrids\_100m/2024/RAW/"}\NormalTok{,}
 \AttributeTok{n\_workers   =} \DecValTok{12}\NormalTok{,}
 \AttributeTok{radii     =} \FunctionTok{c}\NormalTok{(}\StringTok{"r3000"}\NormalTok{),}
 \AttributeTok{radius\_mode  =} \StringTok{"sparse"}\NormalTok{,}
 \AttributeTok{extract\_fun  =} \StringTok{"sum"}\NormalTok{,}
 \AttributeTok{fill\_missing  =} \ConstantTok{TRUE}\NormalTok{,}
 \AttributeTok{IDW\_weight   =} \DecValTok{2}\NormalTok{,}
 \AttributeTok{future\_max\_size =} \DecValTok{20} \SpecialCharTok{*} \DecValTok{1024}\SpecialCharTok{\^{}}\DecValTok{3}\NormalTok{)}


\CommentTok{\# Edges\_Water\_r3000.tif egv\_163 {-}{-}{-}{-}}
\NormalTok{slanis}\OtherTok{=}\FunctionTok{rast}\NormalTok{(}\StringTok{"./RasterGrids\_100m/2024/RAW/Edges\_Water\_r3000.tif"}\NormalTok{)}
\FunctionTok{names}\NormalTok{(slanis)}\OtherTok{=}\StringTok{"egv\_163"}
\NormalTok{slanis2}\OtherTok{=}\FunctionTok{project}\NormalTok{(slanis,template100)}
\FunctionTok{writeRaster}\NormalTok{(slanis2,}
      \StringTok{"./RasterGrids\_100m/2024/RAW/Edges\_Water\_r3000.tif"}\NormalTok{,}
      \AttributeTok{overwrite=}\ConstantTok{TRUE}\NormalTok{)}

\CommentTok{\# standardisation {-}{-}{-}{-}}
\ControlFlowTok{if}\NormalTok{(}\SpecialCharTok{!}\FunctionTok{require}\NormalTok{(terra)) \{}\FunctionTok{install.packages}\NormalTok{(}\StringTok{"terra"}\NormalTok{); }\FunctionTok{require}\NormalTok{(terra)\}}
\ControlFlowTok{if}\NormalTok{(}\SpecialCharTok{!}\FunctionTok{require}\NormalTok{(tidyverse)) \{}\FunctionTok{install.packages}\NormalTok{(}\StringTok{"tidyverse"}\NormalTok{); }\FunctionTok{require}\NormalTok{(tidyverse)\}}

\NormalTok{nosaukums}\OtherTok{=}\StringTok{"Edges\_Water\_r3000.tif"}
\NormalTok{ielasisanas\_cels}\OtherTok{=}\FunctionTok{paste0}\NormalTok{(}\StringTok{"./RasterGrids\_100m/2024/RAW/"}\NormalTok{,nosaukums)}
\NormalTok{saglabasanas\_cels}\OtherTok{=}\FunctionTok{paste0}\NormalTok{(}\StringTok{"./RasterGrids\_100m/2024/Scaled/"}\NormalTok{,nosaukums)}
\NormalTok{slanis}\OtherTok{=}\FunctionTok{rast}\NormalTok{(ielasisanas\_cels)}
\NormalTok{videjais}\OtherTok{=}\FunctionTok{global}\NormalTok{(slanis,}\AttributeTok{fun=}\StringTok{"mean"}\NormalTok{,}\AttributeTok{na.rm=}\ConstantTok{TRUE}\NormalTok{)}
\NormalTok{centrets}\OtherTok{=}\NormalTok{slanis}\SpecialCharTok{{-}}\NormalTok{videjais[,}\DecValTok{1}\NormalTok{]}
\NormalTok{standartnovirze}\OtherTok{=}\NormalTok{terra}\SpecialCharTok{::}\FunctionTok{global}\NormalTok{(centrets,}\AttributeTok{fun=}\StringTok{"rms"}\NormalTok{,}\AttributeTok{na.rm=}\ConstantTok{TRUE}\NormalTok{)}
\NormalTok{merogots}\OtherTok{=}\NormalTok{centrets}\SpecialCharTok{/}\NormalTok{standartnovirze[,}\DecValTok{1}\NormalTok{]}
\FunctionTok{writeRaster}\NormalTok{(merogots,}
      \AttributeTok{filename=}\NormalTok{saglabasanas\_cels,}
      \AttributeTok{overwrite=}\ConstantTok{TRUE}\NormalTok{)}
\end{Highlighting}
\end{Shaded}

\section{Edges\_Water\_r10000}\label{ch06.164}

\textbf{filename:} \texttt{Edges\_Water\_r10000.tif}

\textbf{layername:} \texttt{egv\_164}

\textbf{English name:} Edge pixels of Water within the 10 km landscape

\textbf{Latvian name:} Ūdenstilpju malu pikseļu skaits 10 km ainavā

\textbf{Procedure:} The total edge within a 10000 m radius around the analysis grid cell is
calculated as the area-weighted sum of the \hyperref[ch06.160]{analysis cells} inside the
buffer, using the workflow \texttt{egvtools::radius\_function()}. During the calculation of the landscape metric,
inverse distance weighted (power = 2) gap filling on the output is applied
to ensure no missing values at the edges. Then the layer is rewritten to set
its name. Finally, the layer is standardised by subtracting the arithmetic
mean and dividing by the root mean squared error.

\begin{Shaded}
\begin{Highlighting}[]
\CommentTok{\# libs {-}{-}{-}{-}}
\ControlFlowTok{if}\NormalTok{(}\SpecialCharTok{!}\FunctionTok{require}\NormalTok{(terra)) \{}\FunctionTok{install.packages}\NormalTok{(}\StringTok{"terra"}\NormalTok{); }\FunctionTok{require}\NormalTok{(terra)\}}
\ControlFlowTok{if}\NormalTok{(}\SpecialCharTok{!}\FunctionTok{require}\NormalTok{(egvtools)) \{remotes}\SpecialCharTok{::}\FunctionTok{install\_github}\NormalTok{(}\StringTok{"aavotins/egvtools"}\NormalTok{); }\FunctionTok{require}\NormalTok{(egvtools)\}}


\CommentTok{\# Templates {-}{-}{-}{-}{-}}
\NormalTok{template100}\OtherTok{=}\FunctionTok{rast}\NormalTok{(}\StringTok{"./Templates/TemplateRasters/LV100m\_10km.tif"}\NormalTok{)}

\CommentTok{\# radii {-}{-}{-}{-}}
\FunctionTok{radius\_function}\NormalTok{(}
 \AttributeTok{kvadrati\_path =} \StringTok{"./Templates/TemplateGrids/tiles/"}\NormalTok{,}
 \AttributeTok{radii\_path   =} \StringTok{"./Templates/TemplateGridPoints/tiles/"}\NormalTok{,}
 \AttributeTok{tikls100\_path =} \StringTok{"./Templates/TemplateGrids/tikls100\_sauzeme.parquet"}\NormalTok{,}
 \AttributeTok{template\_path =} \StringTok{"./Templates/TemplateRasters/LV100m\_10km.tif"}\NormalTok{,}
 \AttributeTok{input\_layers  =} \FunctionTok{c}\NormalTok{(}\StringTok{"./RasterGrids\_100m/2024/RAW/Edges\_Water\_cell.tif"}\NormalTok{),}
 \AttributeTok{layer\_prefixes =} \FunctionTok{c}\NormalTok{(}\StringTok{"Edges\_Water"}\NormalTok{),}
 \AttributeTok{output\_dir   =} \StringTok{"./RasterGrids\_100m/2024/RAW/"}\NormalTok{,}
 \AttributeTok{n\_workers   =} \DecValTok{12}\NormalTok{,}
 \AttributeTok{radii     =} \FunctionTok{c}\NormalTok{(}\StringTok{"r10000"}\NormalTok{),}
 \AttributeTok{radius\_mode  =} \StringTok{"sparse"}\NormalTok{,}
 \AttributeTok{extract\_fun  =} \StringTok{"sum"}\NormalTok{,}
 \AttributeTok{fill\_missing  =} \ConstantTok{TRUE}\NormalTok{,}
 \AttributeTok{IDW\_weight   =} \DecValTok{2}\NormalTok{,}
 \AttributeTok{future\_max\_size =} \DecValTok{20} \SpecialCharTok{*} \DecValTok{1024}\SpecialCharTok{\^{}}\DecValTok{3}\NormalTok{)}


\CommentTok{\# Edges\_Water\_r10000.tif    egv\_164 {-}{-}{-}{-}}
\NormalTok{slanis}\OtherTok{=}\FunctionTok{rast}\NormalTok{(}\StringTok{"./RasterGrids\_100m/2024/RAW/Edges\_Water\_r10000.tif"}\NormalTok{)}
\FunctionTok{names}\NormalTok{(slanis)}\OtherTok{=}\StringTok{"egv\_164"}
\NormalTok{slanis2}\OtherTok{=}\FunctionTok{project}\NormalTok{(slanis,template100)}
\FunctionTok{writeRaster}\NormalTok{(slanis2,}
      \StringTok{"./RasterGrids\_100m/2024/RAW/Edges\_Water\_r10000.tif"}\NormalTok{,}
      \AttributeTok{overwrite=}\ConstantTok{TRUE}\NormalTok{)}

\CommentTok{\# standardisation {-}{-}{-}{-}}
\ControlFlowTok{if}\NormalTok{(}\SpecialCharTok{!}\FunctionTok{require}\NormalTok{(terra)) \{}\FunctionTok{install.packages}\NormalTok{(}\StringTok{"terra"}\NormalTok{); }\FunctionTok{require}\NormalTok{(terra)\}}
\ControlFlowTok{if}\NormalTok{(}\SpecialCharTok{!}\FunctionTok{require}\NormalTok{(tidyverse)) \{}\FunctionTok{install.packages}\NormalTok{(}\StringTok{"tidyverse"}\NormalTok{); }\FunctionTok{require}\NormalTok{(tidyverse)\}}

\NormalTok{nosaukums}\OtherTok{=}\StringTok{"Edges\_Water\_r10000.tif"}
\NormalTok{ielasisanas\_cels}\OtherTok{=}\FunctionTok{paste0}\NormalTok{(}\StringTok{"./RasterGrids\_100m/2024/RAW/"}\NormalTok{,nosaukums)}
\NormalTok{saglabasanas\_cels}\OtherTok{=}\FunctionTok{paste0}\NormalTok{(}\StringTok{"./RasterGrids\_100m/2024/Scaled/"}\NormalTok{,nosaukums)}
\NormalTok{slanis}\OtherTok{=}\FunctionTok{rast}\NormalTok{(ielasisanas\_cels)}
\NormalTok{videjais}\OtherTok{=}\FunctionTok{global}\NormalTok{(slanis,}\AttributeTok{fun=}\StringTok{"mean"}\NormalTok{,}\AttributeTok{na.rm=}\ConstantTok{TRUE}\NormalTok{)}
\NormalTok{centrets}\OtherTok{=}\NormalTok{slanis}\SpecialCharTok{{-}}\NormalTok{videjais[,}\DecValTok{1}\NormalTok{]}
\NormalTok{standartnovirze}\OtherTok{=}\NormalTok{terra}\SpecialCharTok{::}\FunctionTok{global}\NormalTok{(centrets,}\AttributeTok{fun=}\StringTok{"rms"}\NormalTok{,}\AttributeTok{na.rm=}\ConstantTok{TRUE}\NormalTok{)}
\NormalTok{merogots}\OtherTok{=}\NormalTok{centrets}\SpecialCharTok{/}\NormalTok{standartnovirze[,}\DecValTok{1}\NormalTok{]}
\FunctionTok{writeRaster}\NormalTok{(merogots,}
      \AttributeTok{filename=}\NormalTok{saglabasanas\_cels,}
      \AttributeTok{overwrite=}\ConstantTok{TRUE}\NormalTok{)}
\end{Highlighting}
\end{Shaded}

\section{Edges\_Water-Farmland\_cell}\label{ch06.165}

\textbf{filename:} \texttt{Edges\_Water-Farmland\_cell.tif}

\textbf{layername:} \texttt{egv\_165}

\textbf{English name:} Edge pixels of Water bordering with Farmland within the
analysis cell (1 ha)

\textbf{Latvian name:} Ūdenstilpju malu ar lauksaimniecības zemēm pikseļu skaits analīzes
šūnā (1 ha)

\textbf{Procedure:} First, values larger than 300 and smaller than 400 from
\hyperref[Ch05.03]{Landscape classification} are coded as 1, and all other values as NA.
Then values equal to 200 from the \hyperref[Ch05.03]{Landscape classification} are coded as
0, and all other values as NA. Then, the first layer (1 = presence) is covered over
the second layer (presence = 0) and written to file (matching the input). Next,
using the workflow \texttt{egvtools::landscape\_function()} total edge between the two
classes is calculated. During the calculation of the landscape metric, inverse distance
weighted (power = 2) gap filling on the output is applied to ensure no
missing values at the edges. Finally, the layer is standardised by
subtracting the arithmetic mean and dividing by the root mean squared error.

\begin{Shaded}
\begin{Highlighting}[]
\CommentTok{\# libs {-}{-}{-}{-}}
\ControlFlowTok{if}\NormalTok{(}\SpecialCharTok{!}\FunctionTok{require}\NormalTok{(terra)) \{}\FunctionTok{install.packages}\NormalTok{(}\StringTok{"terra"}\NormalTok{); }\FunctionTok{require}\NormalTok{(terra)\}}
\ControlFlowTok{if}\NormalTok{(}\SpecialCharTok{!}\FunctionTok{require}\NormalTok{(egvtools)) \{remotes}\SpecialCharTok{::}\FunctionTok{install\_github}\NormalTok{(}\StringTok{"aavotins/egvtools"}\NormalTok{); }\FunctionTok{require}\NormalTok{(egvtools)\}}

\ControlFlowTok{if}\NormalTok{(}\SpecialCharTok{!}\FunctionTok{require}\NormalTok{(sf)) \{}\FunctionTok{install.packages}\NormalTok{(}\StringTok{"sf"}\NormalTok{); }\FunctionTok{require}\NormalTok{(sf)\}}
\ControlFlowTok{if}\NormalTok{(}\SpecialCharTok{!}\FunctionTok{require}\NormalTok{(sfarrow)) \{}\FunctionTok{install.packages}\NormalTok{(}\StringTok{"sfarrow"}\NormalTok{); }\FunctionTok{require}\NormalTok{(sfarrow)\}}
\ControlFlowTok{if}\NormalTok{(}\SpecialCharTok{!}\FunctionTok{require}\NormalTok{(raster)) \{}\FunctionTok{install.packages}\NormalTok{(}\StringTok{"raster"}\NormalTok{); }\FunctionTok{require}\NormalTok{(raster)\}}
\ControlFlowTok{if}\NormalTok{(}\SpecialCharTok{!}\FunctionTok{require}\NormalTok{(fasterize)) \{}\FunctionTok{install.packages}\NormalTok{(}\StringTok{"fasterize"}\NormalTok{); }\FunctionTok{require}\NormalTok{(fasterize)\}}
\ControlFlowTok{if}\NormalTok{(}\SpecialCharTok{!}\FunctionTok{require}\NormalTok{(tidyverse)) \{}\FunctionTok{install.packages}\NormalTok{(}\StringTok{"tidyverse"}\NormalTok{); }\FunctionTok{require}\NormalTok{(tidyverse)\}}


\CommentTok{\# Templates {-}{-}{-}{-}{-}}
\NormalTok{template10}\OtherTok{=}\FunctionTok{rast}\NormalTok{(}\StringTok{"./Templates/TemplateRasters/LV10m\_10km.tif"}\NormalTok{)}
\NormalTok{nulls10}\OtherTok{=}\FunctionTok{rast}\NormalTok{(}\StringTok{"./Templates/TemplateRasters/nulls\_LV10m\_10km.tif"}\NormalTok{)}

\CommentTok{\# simple landscape {-}{-}{-}{-}}
\NormalTok{simple\_landscape}\OtherTok{=}\FunctionTok{rast}\NormalTok{(}\StringTok{"./RasterGrids\_10m/2024/Ainava\_vienk\_mask.tif"}\NormalTok{)}

\CommentTok{\# Edges\_Water{-}Farmland\_input.tif {-}{-}{-}{-}}
\NormalTok{water}\OtherTok{=}\FunctionTok{ifel}\NormalTok{(simple\_landscape}\SpecialCharTok{==}\DecValTok{200}\NormalTok{,}\DecValTok{0}\NormalTok{,}\ConstantTok{NA}\NormalTok{)}
\FunctionTok{plot}\NormalTok{(water)}

\NormalTok{farmland}\OtherTok{=}\FunctionTok{ifel}\NormalTok{(simple\_landscape}\SpecialCharTok{\textgreater{}}\DecValTok{300} \SpecialCharTok{\&}\NormalTok{ simple\_landscape}\SpecialCharTok{\textless{}}\DecValTok{400}\NormalTok{,}\DecValTok{1}\NormalTok{,}\ConstantTok{NA}\NormalTok{)}
\FunctionTok{plot}\NormalTok{(farmland)}

\NormalTok{water\_farmland}\OtherTok{=}\FunctionTok{cover}\NormalTok{(water,farmland)}
\FunctionTok{plot}\NormalTok{(water\_farmland)}

\NormalTok{edge\_water\_farmland}\OtherTok{=}\FunctionTok{project}\NormalTok{(water\_farmland,template10,}
               \AttributeTok{filename=}\StringTok{"./RasterGrids\_10m/2024/Edges\_Water{-}Farmland\_input.tif"}\NormalTok{,}
               \AttributeTok{overwrite=}\ConstantTok{TRUE}\NormalTok{)}
\FunctionTok{rm}\NormalTok{(edge\_water\_farmland)}
\FunctionTok{rm}\NormalTok{(water\_farmland)}


\CommentTok{\# Edges\_Water{-}Farmland\_cell.tif egv\_165 {-}{-}{-}{-}}
\FunctionTok{landscape\_function}\NormalTok{(}
 \AttributeTok{landscape   =} \StringTok{"./RasterGrids\_10m/2024/Edges\_Water{-}Farmland\_input.tif"}\NormalTok{,}
 \AttributeTok{zones     =} \StringTok{"./Templates/TemplateGrids/tikls100\_sauzeme.parquet"}\NormalTok{,}
 \AttributeTok{id\_field    =} \StringTok{"id"}\NormalTok{,}
 \AttributeTok{tile\_field   =} \StringTok{"tks50km"}\NormalTok{,}
 \AttributeTok{template    =} \StringTok{"./Templates/TemplateRasters/LV100m\_10km.tif"}\NormalTok{,}
 \AttributeTok{out\_dir    =} \StringTok{"./RasterGrids\_100m/2024/RAW"}\NormalTok{,}
 \AttributeTok{out\_filename  =} \StringTok{"Edges\_Water{-}Farmland\_cell.tif"}\NormalTok{,}
 \AttributeTok{out\_layername =} \StringTok{"egv\_165"}\NormalTok{,}
 \AttributeTok{what       =} \StringTok{"lsm\_l\_te"}\NormalTok{,}
 \AttributeTok{lm\_args     =} \FunctionTok{list}\NormalTok{(}\AttributeTok{count\_boundary =} \ConstantTok{FALSE}\NormalTok{),}
 \AttributeTok{rasterize\_engine =} \StringTok{"fasterize"}\NormalTok{,}
 \AttributeTok{n\_workers   =} \DecValTok{12}\NormalTok{,}
 \AttributeTok{future\_max\_size =} \DecValTok{20} \SpecialCharTok{*} \DecValTok{1024}\SpecialCharTok{\^{}}\DecValTok{3}\NormalTok{,}
 \AttributeTok{fill\_gaps   =} \ConstantTok{TRUE}\NormalTok{,}
 \AttributeTok{plot\_gaps   =} \ConstantTok{FALSE}\NormalTok{,}
 \AttributeTok{plot\_result  =} \ConstantTok{FALSE}
\NormalTok{)}

\CommentTok{\# standardisation {-}{-}{-}{-}}
\ControlFlowTok{if}\NormalTok{(}\SpecialCharTok{!}\FunctionTok{require}\NormalTok{(terra)) \{}\FunctionTok{install.packages}\NormalTok{(}\StringTok{"terra"}\NormalTok{); }\FunctionTok{require}\NormalTok{(terra)\}}
\ControlFlowTok{if}\NormalTok{(}\SpecialCharTok{!}\FunctionTok{require}\NormalTok{(tidyverse)) \{}\FunctionTok{install.packages}\NormalTok{(}\StringTok{"tidyverse"}\NormalTok{); }\FunctionTok{require}\NormalTok{(tidyverse)\}}

\NormalTok{nosaukums}\OtherTok{=}\StringTok{"Edges\_Water{-}Farmland\_cell.tif"}
\NormalTok{ielasisanas\_cels}\OtherTok{=}\FunctionTok{paste0}\NormalTok{(}\StringTok{"./RasterGrids\_100m/2024/RAW/"}\NormalTok{,nosaukums)}
\NormalTok{saglabasanas\_cels}\OtherTok{=}\FunctionTok{paste0}\NormalTok{(}\StringTok{"./RasterGrids\_100m/2024/Scaled/"}\NormalTok{,nosaukums)}
\NormalTok{slanis}\OtherTok{=}\FunctionTok{rast}\NormalTok{(ielasisanas\_cels)}
\NormalTok{videjais}\OtherTok{=}\FunctionTok{global}\NormalTok{(slanis,}\AttributeTok{fun=}\StringTok{"mean"}\NormalTok{,}\AttributeTok{na.rm=}\ConstantTok{TRUE}\NormalTok{)}
\NormalTok{centrets}\OtherTok{=}\NormalTok{slanis}\SpecialCharTok{{-}}\NormalTok{videjais[,}\DecValTok{1}\NormalTok{]}
\NormalTok{standartnovirze}\OtherTok{=}\NormalTok{terra}\SpecialCharTok{::}\FunctionTok{global}\NormalTok{(centrets,}\AttributeTok{fun=}\StringTok{"rms"}\NormalTok{,}\AttributeTok{na.rm=}\ConstantTok{TRUE}\NormalTok{)}
\NormalTok{merogots}\OtherTok{=}\NormalTok{centrets}\SpecialCharTok{/}\NormalTok{standartnovirze[,}\DecValTok{1}\NormalTok{]}
\FunctionTok{writeRaster}\NormalTok{(merogots,}
      \AttributeTok{filename=}\NormalTok{saglabasanas\_cels,}
      \AttributeTok{overwrite=}\ConstantTok{TRUE}\NormalTok{)}
\end{Highlighting}
\end{Shaded}

\section{Edges\_Water-Farmland\_r500}\label{ch06.166}

\textbf{filename:} \texttt{Edges\_Water-Farmland\_r500.tif}

\textbf{layername:} \texttt{egv\_166}

\textbf{English name:} Edge pixels of Water bordering with Farmland within the 0.5 km
landscape

\textbf{Latvian name:} Ūdenstilpju malu ar lauksaimniecības zemēm pikseļu skaits 0,5 km
ainavā

\textbf{Procedure:} The total edge within a 500 m radius around the analysis grid cell is
calculated as the area-weighted sum of the \hyperref[ch06.165]{analysis cells} inside the
buffer, using the workflow \texttt{egvtools::radius\_function()}. During the calculation of the landscape metric,
inverse distance weighted (power = 2) gap filling on the output is applied
to ensure no missing values at the edges. Then the layer is rewritten to set
its name. Finally, the layer is standardised by subtracting the arithmetic
mean and dividing by the root mean squared error.

\begin{Shaded}
\begin{Highlighting}[]
\CommentTok{\# libs {-}{-}{-}{-}}
\ControlFlowTok{if}\NormalTok{(}\SpecialCharTok{!}\FunctionTok{require}\NormalTok{(terra)) \{}\FunctionTok{install.packages}\NormalTok{(}\StringTok{"terra"}\NormalTok{); }\FunctionTok{require}\NormalTok{(terra)\}}
\ControlFlowTok{if}\NormalTok{(}\SpecialCharTok{!}\FunctionTok{require}\NormalTok{(egvtools)) \{remotes}\SpecialCharTok{::}\FunctionTok{install\_github}\NormalTok{(}\StringTok{"aavotins/egvtools"}\NormalTok{); }\FunctionTok{require}\NormalTok{(egvtools)\}}


\CommentTok{\# Templates {-}{-}{-}{-}{-}}
\NormalTok{template100}\OtherTok{=}\FunctionTok{rast}\NormalTok{(}\StringTok{"./Templates/TemplateRasters/LV100m\_10km.tif"}\NormalTok{)}

\CommentTok{\# radii {-}{-}{-}{-}}
\FunctionTok{radius\_function}\NormalTok{(}
 \AttributeTok{kvadrati\_path =} \StringTok{"./Templates/TemplateGrids/tiles/"}\NormalTok{,}
 \AttributeTok{radii\_path   =} \StringTok{"./Templates/TemplateGridPoints/tiles/"}\NormalTok{,}
 \AttributeTok{tikls100\_path =} \StringTok{"./Templates/TemplateGrids/tikls100\_sauzeme.parquet"}\NormalTok{,}
 \AttributeTok{template\_path =} \StringTok{"./Templates/TemplateRasters/LV100m\_10km.tif"}\NormalTok{,}
 \AttributeTok{input\_layers  =} \FunctionTok{c}\NormalTok{(}\StringTok{"./RasterGrids\_100m/2024/RAW/Edges\_Water{-}Farmland\_cell.tif"}\NormalTok{),}
 \AttributeTok{layer\_prefixes =} \FunctionTok{c}\NormalTok{(}\StringTok{"Edges\_Water{-}Farmland"}\NormalTok{),}
 \AttributeTok{output\_dir   =} \StringTok{"./RasterGrids\_100m/2024/RAW/"}\NormalTok{,}
 \AttributeTok{n\_workers   =} \DecValTok{12}\NormalTok{,}
 \AttributeTok{radii     =} \FunctionTok{c}\NormalTok{(}\StringTok{"r500"}\NormalTok{),}
 \AttributeTok{radius\_mode  =} \StringTok{"sparse"}\NormalTok{,}
 \AttributeTok{extract\_fun  =} \StringTok{"sum"}\NormalTok{,}
 \AttributeTok{fill\_missing  =} \ConstantTok{TRUE}\NormalTok{,}
 \AttributeTok{IDW\_weight   =} \DecValTok{2}\NormalTok{,}
 \AttributeTok{future\_max\_size =} \DecValTok{20} \SpecialCharTok{*} \DecValTok{1024}\SpecialCharTok{\^{}}\DecValTok{3}\NormalTok{)}


\CommentTok{\# Edges\_Water{-}Farmland\_r500.tif egv\_166 {-}{-}{-}{-}{-}}
\NormalTok{slanis}\OtherTok{=}\FunctionTok{rast}\NormalTok{(}\StringTok{"./RasterGrids\_100m/2024/RAW/Edges\_Water{-}Farmland\_r500.tif"}\NormalTok{)}
\FunctionTok{names}\NormalTok{(slanis)}\OtherTok{=}\StringTok{"egv\_166"}
\NormalTok{slanis2}\OtherTok{=}\FunctionTok{project}\NormalTok{(slanis,template100)}
\FunctionTok{writeRaster}\NormalTok{(slanis2,}
      \StringTok{"./RasterGrids\_100m/2024/RAW/Edges\_Water{-}Farmland\_r500.tif"}\NormalTok{,}
      \AttributeTok{overwrite=}\ConstantTok{TRUE}\NormalTok{)}

\CommentTok{\# standardisation {-}{-}{-}{-}}
\ControlFlowTok{if}\NormalTok{(}\SpecialCharTok{!}\FunctionTok{require}\NormalTok{(terra)) \{}\FunctionTok{install.packages}\NormalTok{(}\StringTok{"terra"}\NormalTok{); }\FunctionTok{require}\NormalTok{(terra)\}}
\ControlFlowTok{if}\NormalTok{(}\SpecialCharTok{!}\FunctionTok{require}\NormalTok{(tidyverse)) \{}\FunctionTok{install.packages}\NormalTok{(}\StringTok{"tidyverse"}\NormalTok{); }\FunctionTok{require}\NormalTok{(tidyverse)\}}

\NormalTok{nosaukums}\OtherTok{=}\StringTok{"Edges\_Water{-}Farmland\_r500.tif"}
\NormalTok{ielasisanas\_cels}\OtherTok{=}\FunctionTok{paste0}\NormalTok{(}\StringTok{"./RasterGrids\_100m/2024/RAW/"}\NormalTok{,nosaukums)}
\NormalTok{saglabasanas\_cels}\OtherTok{=}\FunctionTok{paste0}\NormalTok{(}\StringTok{"./RasterGrids\_100m/2024/Scaled/"}\NormalTok{,nosaukums)}
\NormalTok{slanis}\OtherTok{=}\FunctionTok{rast}\NormalTok{(ielasisanas\_cels)}
\NormalTok{videjais}\OtherTok{=}\FunctionTok{global}\NormalTok{(slanis,}\AttributeTok{fun=}\StringTok{"mean"}\NormalTok{,}\AttributeTok{na.rm=}\ConstantTok{TRUE}\NormalTok{)}
\NormalTok{centrets}\OtherTok{=}\NormalTok{slanis}\SpecialCharTok{{-}}\NormalTok{videjais[,}\DecValTok{1}\NormalTok{]}
\NormalTok{standartnovirze}\OtherTok{=}\NormalTok{terra}\SpecialCharTok{::}\FunctionTok{global}\NormalTok{(centrets,}\AttributeTok{fun=}\StringTok{"rms"}\NormalTok{,}\AttributeTok{na.rm=}\ConstantTok{TRUE}\NormalTok{)}
\NormalTok{merogots}\OtherTok{=}\NormalTok{centrets}\SpecialCharTok{/}\NormalTok{standartnovirze[,}\DecValTok{1}\NormalTok{]}
\FunctionTok{writeRaster}\NormalTok{(merogots,}
      \AttributeTok{filename=}\NormalTok{saglabasanas\_cels,}
      \AttributeTok{overwrite=}\ConstantTok{TRUE}\NormalTok{)}
\end{Highlighting}
\end{Shaded}

\section{Edges\_Water-Farmland\_r1250}\label{ch06.167}

\textbf{filename:} \texttt{Edges\_Water-Farmland\_r1250.tif}

\textbf{layername:} \texttt{egv\_167}

\textbf{English name:} Edge pixels of Water bordering with Farmland within the 1.25
km landscape

\textbf{Latvian name:} Ūdenstilpju malu ar lauksaimniecības zemēm pikseļu skaits 1,25 km
ainavā

\textbf{Procedure:} The total edge within a 1250 m radius around the analysis grid cell is
calculated as the area-weighted sum of the \hyperref[ch06.165]{analysis cells} inside the
buffer, using the workflow \texttt{egvtools::radius\_function()}. During the calculation of the landscape metric,
inverse distance weighted (power = 2) gap filling on the output is applied
to ensure no missing values at the edges. Then the layer is rewritten to set
its name. Finally, the layer is standardised by subtracting the arithmetic
mean and dividing by the root mean squared error.

\begin{Shaded}
\begin{Highlighting}[]
\CommentTok{\# libs {-}{-}{-}{-}}
\ControlFlowTok{if}\NormalTok{(}\SpecialCharTok{!}\FunctionTok{require}\NormalTok{(terra)) \{}\FunctionTok{install.packages}\NormalTok{(}\StringTok{"terra"}\NormalTok{); }\FunctionTok{require}\NormalTok{(terra)\}}
\ControlFlowTok{if}\NormalTok{(}\SpecialCharTok{!}\FunctionTok{require}\NormalTok{(egvtools)) \{remotes}\SpecialCharTok{::}\FunctionTok{install\_github}\NormalTok{(}\StringTok{"aavotins/egvtools"}\NormalTok{); }\FunctionTok{require}\NormalTok{(egvtools)\}}


\CommentTok{\# Templates {-}{-}{-}{-}{-}}
\NormalTok{template100}\OtherTok{=}\FunctionTok{rast}\NormalTok{(}\StringTok{"./Templates/TemplateRasters/LV100m\_10km.tif"}\NormalTok{)}

\CommentTok{\# radii {-}{-}{-}{-}}
\FunctionTok{radius\_function}\NormalTok{(}
 \AttributeTok{kvadrati\_path =} \StringTok{"./Templates/TemplateGrids/tiles/"}\NormalTok{,}
 \AttributeTok{radii\_path   =} \StringTok{"./Templates/TemplateGridPoints/tiles/"}\NormalTok{,}
 \AttributeTok{tikls100\_path =} \StringTok{"./Templates/TemplateGrids/tikls100\_sauzeme.parquet"}\NormalTok{,}
 \AttributeTok{template\_path =} \StringTok{"./Templates/TemplateRasters/LV100m\_10km.tif"}\NormalTok{,}
 \AttributeTok{input\_layers  =} \FunctionTok{c}\NormalTok{(}\StringTok{"./RasterGrids\_100m/2024/RAW/Edges\_Water{-}Farmland\_cell.tif"}\NormalTok{),}
 \AttributeTok{layer\_prefixes =} \FunctionTok{c}\NormalTok{(}\StringTok{"Edges\_Water{-}Farmland"}\NormalTok{),}
 \AttributeTok{output\_dir   =} \StringTok{"./RasterGrids\_100m/2024/RAW/"}\NormalTok{,}
 \AttributeTok{n\_workers   =} \DecValTok{12}\NormalTok{,}
 \AttributeTok{radii     =} \FunctionTok{c}\NormalTok{(}\StringTok{"r1250"}\NormalTok{),}
 \AttributeTok{radius\_mode  =} \StringTok{"sparse"}\NormalTok{,}
 \AttributeTok{extract\_fun  =} \StringTok{"sum"}\NormalTok{,}
 \AttributeTok{fill\_missing  =} \ConstantTok{TRUE}\NormalTok{,}
 \AttributeTok{IDW\_weight   =} \DecValTok{2}\NormalTok{,}
 \AttributeTok{future\_max\_size =} \DecValTok{20} \SpecialCharTok{*} \DecValTok{1024}\SpecialCharTok{\^{}}\DecValTok{3}\NormalTok{)}


\CommentTok{\# Edges\_Water{-}Farmland\_r1250.tif    egv\_167 {-}{-}{-}{-}}
\NormalTok{slanis}\OtherTok{=}\FunctionTok{rast}\NormalTok{(}\StringTok{"./RasterGrids\_100m/2024/RAW/Edges\_Water{-}Farmland\_r1250.tif"}\NormalTok{)}
\FunctionTok{names}\NormalTok{(slanis)}\OtherTok{=}\StringTok{"egv\_167"}
\NormalTok{slanis2}\OtherTok{=}\FunctionTok{project}\NormalTok{(slanis,template100)}
\FunctionTok{writeRaster}\NormalTok{(slanis2,}
      \StringTok{"./RasterGrids\_100m/2024/RAW/Edges\_Water{-}Farmland\_r1250.tif"}\NormalTok{,}
      \AttributeTok{overwrite=}\ConstantTok{TRUE}\NormalTok{)}

\CommentTok{\# standardisation {-}{-}{-}{-}}
\ControlFlowTok{if}\NormalTok{(}\SpecialCharTok{!}\FunctionTok{require}\NormalTok{(terra)) \{}\FunctionTok{install.packages}\NormalTok{(}\StringTok{"terra"}\NormalTok{); }\FunctionTok{require}\NormalTok{(terra)\}}
\ControlFlowTok{if}\NormalTok{(}\SpecialCharTok{!}\FunctionTok{require}\NormalTok{(tidyverse)) \{}\FunctionTok{install.packages}\NormalTok{(}\StringTok{"tidyverse"}\NormalTok{); }\FunctionTok{require}\NormalTok{(tidyverse)\}}

\NormalTok{nosaukums}\OtherTok{=}\StringTok{"Edges\_Water{-}Farmland\_r1250.tif"}
\NormalTok{ielasisanas\_cels}\OtherTok{=}\FunctionTok{paste0}\NormalTok{(}\StringTok{"./RasterGrids\_100m/2024/RAW/"}\NormalTok{,nosaukums)}
\NormalTok{saglabasanas\_cels}\OtherTok{=}\FunctionTok{paste0}\NormalTok{(}\StringTok{"./RasterGrids\_100m/2024/Scaled/"}\NormalTok{,nosaukums)}
\NormalTok{slanis}\OtherTok{=}\FunctionTok{rast}\NormalTok{(ielasisanas\_cels)}
\NormalTok{videjais}\OtherTok{=}\FunctionTok{global}\NormalTok{(slanis,}\AttributeTok{fun=}\StringTok{"mean"}\NormalTok{,}\AttributeTok{na.rm=}\ConstantTok{TRUE}\NormalTok{)}
\NormalTok{centrets}\OtherTok{=}\NormalTok{slanis}\SpecialCharTok{{-}}\NormalTok{videjais[,}\DecValTok{1}\NormalTok{]}
\NormalTok{standartnovirze}\OtherTok{=}\NormalTok{terra}\SpecialCharTok{::}\FunctionTok{global}\NormalTok{(centrets,}\AttributeTok{fun=}\StringTok{"rms"}\NormalTok{,}\AttributeTok{na.rm=}\ConstantTok{TRUE}\NormalTok{)}
\NormalTok{merogots}\OtherTok{=}\NormalTok{centrets}\SpecialCharTok{/}\NormalTok{standartnovirze[,}\DecValTok{1}\NormalTok{]}
\FunctionTok{writeRaster}\NormalTok{(merogots,}
      \AttributeTok{filename=}\NormalTok{saglabasanas\_cels,}
      \AttributeTok{overwrite=}\ConstantTok{TRUE}\NormalTok{)}
\end{Highlighting}
\end{Shaded}

\section{Edges\_Water-Farmland\_r3000}\label{ch06.168}

\textbf{filename:} \texttt{Edges\_Water-Farmland\_r3000.tif}

\textbf{layername:} \texttt{egv\_168}

\textbf{English name:} Edge pixels of Water bordering with Farmland within the 3 km
landscape

\textbf{Latvian name:} Ūdenstilpju malu ar lauksaimniecības zemēm pikseļu skaits 3 km ainavā

\textbf{Procedure:} The total edge within a 3000 m radius around the analysis grid cell is
calculated as the area-weighted sum of the \hyperref[ch06.165]{analysis cells} inside the
buffer, using the workflow \texttt{egvtools::radius\_function()}. During the calculation of the landscape metric,
inverse distance weighted (power = 2) gap filling on the output is applied
to ensure no missing values at the edges. Then the layer is rewritten to set
its name. Finally, the layer is standardised by subtracting the arithmetic
mean and dividing by the root mean squared error.

\begin{Shaded}
\begin{Highlighting}[]
\CommentTok{\# libs {-}{-}{-}{-}}
\ControlFlowTok{if}\NormalTok{(}\SpecialCharTok{!}\FunctionTok{require}\NormalTok{(terra)) \{}\FunctionTok{install.packages}\NormalTok{(}\StringTok{"terra"}\NormalTok{); }\FunctionTok{require}\NormalTok{(terra)\}}
\ControlFlowTok{if}\NormalTok{(}\SpecialCharTok{!}\FunctionTok{require}\NormalTok{(egvtools)) \{remotes}\SpecialCharTok{::}\FunctionTok{install\_github}\NormalTok{(}\StringTok{"aavotins/egvtools"}\NormalTok{); }\FunctionTok{require}\NormalTok{(egvtools)\}}


\CommentTok{\# Templates {-}{-}{-}{-}{-}}
\NormalTok{template100}\OtherTok{=}\FunctionTok{rast}\NormalTok{(}\StringTok{"./Templates/TemplateRasters/LV100m\_10km.tif"}\NormalTok{)}

\CommentTok{\# radii {-}{-}{-}{-}}
\FunctionTok{radius\_function}\NormalTok{(}
 \AttributeTok{kvadrati\_path =} \StringTok{"./Templates/TemplateGrids/tiles/"}\NormalTok{,}
 \AttributeTok{radii\_path   =} \StringTok{"./Templates/TemplateGridPoints/tiles/"}\NormalTok{,}
 \AttributeTok{tikls100\_path =} \StringTok{"./Templates/TemplateGrids/tikls100\_sauzeme.parquet"}\NormalTok{,}
 \AttributeTok{template\_path =} \StringTok{"./Templates/TemplateRasters/LV100m\_10km.tif"}\NormalTok{,}
 \AttributeTok{input\_layers  =} \FunctionTok{c}\NormalTok{(}\StringTok{"./RasterGrids\_100m/2024/RAW/Edges\_Water{-}Farmland\_cell.tif"}\NormalTok{),}
 \AttributeTok{layer\_prefixes =} \FunctionTok{c}\NormalTok{(}\StringTok{"Edges\_Water{-}Farmland"}\NormalTok{),}
 \AttributeTok{output\_dir   =} \StringTok{"./RasterGrids\_100m/2024/RAW/"}\NormalTok{,}
 \AttributeTok{n\_workers   =} \DecValTok{12}\NormalTok{,}
 \AttributeTok{radii     =} \FunctionTok{c}\NormalTok{(}\StringTok{"r3000"}\NormalTok{),}
 \AttributeTok{radius\_mode  =} \StringTok{"sparse"}\NormalTok{,}
 \AttributeTok{extract\_fun  =} \StringTok{"sum"}\NormalTok{,}
 \AttributeTok{fill\_missing  =} \ConstantTok{TRUE}\NormalTok{,}
 \AttributeTok{IDW\_weight   =} \DecValTok{2}\NormalTok{,}
 \AttributeTok{future\_max\_size =} \DecValTok{20} \SpecialCharTok{*} \DecValTok{1024}\SpecialCharTok{\^{}}\DecValTok{3}\NormalTok{)}


\CommentTok{\# Edges\_Water{-}Farmland\_r3000.tif    egv\_168 {-}{-}{-}{-}}
\NormalTok{slanis}\OtherTok{=}\FunctionTok{rast}\NormalTok{(}\StringTok{"./RasterGrids\_100m/2024/RAW/Edges\_Water{-}Farmland\_r3000.tif"}\NormalTok{)}
\FunctionTok{names}\NormalTok{(slanis)}\OtherTok{=}\StringTok{"egv\_168"}
\NormalTok{slanis2}\OtherTok{=}\FunctionTok{project}\NormalTok{(slanis,template100)}
\FunctionTok{writeRaster}\NormalTok{(slanis2,}
      \StringTok{"./RasterGrids\_100m/2024/RAW/Edges\_Water{-}Farmland\_r3000.tif"}\NormalTok{,}
      \AttributeTok{overwrite=}\ConstantTok{TRUE}\NormalTok{)}

\CommentTok{\# standardisation {-}{-}{-}{-}}
\ControlFlowTok{if}\NormalTok{(}\SpecialCharTok{!}\FunctionTok{require}\NormalTok{(terra)) \{}\FunctionTok{install.packages}\NormalTok{(}\StringTok{"terra"}\NormalTok{); }\FunctionTok{require}\NormalTok{(terra)\}}
\ControlFlowTok{if}\NormalTok{(}\SpecialCharTok{!}\FunctionTok{require}\NormalTok{(tidyverse)) \{}\FunctionTok{install.packages}\NormalTok{(}\StringTok{"tidyverse"}\NormalTok{); }\FunctionTok{require}\NormalTok{(tidyverse)\}}

\NormalTok{nosaukums}\OtherTok{=}\StringTok{"Edges\_Water{-}Farmland\_r3000.tif"}
\NormalTok{ielasisanas\_cels}\OtherTok{=}\FunctionTok{paste0}\NormalTok{(}\StringTok{"./RasterGrids\_100m/2024/RAW/"}\NormalTok{,nosaukums)}
\NormalTok{saglabasanas\_cels}\OtherTok{=}\FunctionTok{paste0}\NormalTok{(}\StringTok{"./RasterGrids\_100m/2024/Scaled/"}\NormalTok{,nosaukums)}
\NormalTok{slanis}\OtherTok{=}\FunctionTok{rast}\NormalTok{(ielasisanas\_cels)}
\NormalTok{videjais}\OtherTok{=}\FunctionTok{global}\NormalTok{(slanis,}\AttributeTok{fun=}\StringTok{"mean"}\NormalTok{,}\AttributeTok{na.rm=}\ConstantTok{TRUE}\NormalTok{)}
\NormalTok{centrets}\OtherTok{=}\NormalTok{slanis}\SpecialCharTok{{-}}\NormalTok{videjais[,}\DecValTok{1}\NormalTok{]}
\NormalTok{standartnovirze}\OtherTok{=}\NormalTok{terra}\SpecialCharTok{::}\FunctionTok{global}\NormalTok{(centrets,}\AttributeTok{fun=}\StringTok{"rms"}\NormalTok{,}\AttributeTok{na.rm=}\ConstantTok{TRUE}\NormalTok{)}
\NormalTok{merogots}\OtherTok{=}\NormalTok{centrets}\SpecialCharTok{/}\NormalTok{standartnovirze[,}\DecValTok{1}\NormalTok{]}
\FunctionTok{writeRaster}\NormalTok{(merogots,}
      \AttributeTok{filename=}\NormalTok{saglabasanas\_cels,}
      \AttributeTok{overwrite=}\ConstantTok{TRUE}\NormalTok{)}
\end{Highlighting}
\end{Shaded}

\section{Edges\_Water-Farmland\_r10000}\label{ch06.169}

\textbf{filename:} \texttt{Edges\_Water-Farmland\_r10000.tif}

\textbf{layername:} \texttt{egv\_169}

\textbf{English name:} Edge pixels of Water bordering with Farmland within the 10 km
landscape

\textbf{Latvian name:} Ūdenstilpju malu ar lauksaimniecības zemēm pikseļu skaits 10 km ainavā

\textbf{Procedure:} The total edge within a 10000 m radius around the analysis grid cell is
calculated as the area-weighted sum of the \hyperref[ch06.165]{analysis cells} inside the
buffer, using the workflow \texttt{egvtools::radius\_function()}. During the calculation of the landscape metric,
inverse distance weighted (power = 2) gap filling on the output is applied
to ensure no missing values at the edges. Then the layer is rewritten to set
its name. Finally, the layer is standardised by subtracting the arithmetic
mean and dividing by the root mean squared error.

\begin{Shaded}
\begin{Highlighting}[]
\CommentTok{\# libs {-}{-}{-}{-}}
\ControlFlowTok{if}\NormalTok{(}\SpecialCharTok{!}\FunctionTok{require}\NormalTok{(terra)) \{}\FunctionTok{install.packages}\NormalTok{(}\StringTok{"terra"}\NormalTok{); }\FunctionTok{require}\NormalTok{(terra)\}}
\ControlFlowTok{if}\NormalTok{(}\SpecialCharTok{!}\FunctionTok{require}\NormalTok{(egvtools)) \{remotes}\SpecialCharTok{::}\FunctionTok{install\_github}\NormalTok{(}\StringTok{"aavotins/egvtools"}\NormalTok{); }\FunctionTok{require}\NormalTok{(egvtools)\}}


\CommentTok{\# Templates {-}{-}{-}{-}{-}}
\NormalTok{template100}\OtherTok{=}\FunctionTok{rast}\NormalTok{(}\StringTok{"./Templates/TemplateRasters/LV100m\_10km.tif"}\NormalTok{)}

\CommentTok{\# radii {-}{-}{-}{-}}
\FunctionTok{radius\_function}\NormalTok{(}
 \AttributeTok{kvadrati\_path =} \StringTok{"./Templates/TemplateGrids/tiles/"}\NormalTok{,}
 \AttributeTok{radii\_path   =} \StringTok{"./Templates/TemplateGridPoints/tiles/"}\NormalTok{,}
 \AttributeTok{tikls100\_path =} \StringTok{"./Templates/TemplateGrids/tikls100\_sauzeme.parquet"}\NormalTok{,}
 \AttributeTok{template\_path =} \StringTok{"./Templates/TemplateRasters/LV100m\_10km.tif"}\NormalTok{,}
 \AttributeTok{input\_layers  =} \FunctionTok{c}\NormalTok{(}\StringTok{"./RasterGrids\_100m/2024/RAW/Edges\_Water{-}Farmland\_cell.tif"}\NormalTok{),}
 \AttributeTok{layer\_prefixes =} \FunctionTok{c}\NormalTok{(}\StringTok{"Edges\_Water{-}Farmland"}\NormalTok{),}
 \AttributeTok{output\_dir   =} \StringTok{"./RasterGrids\_100m/2024/RAW/"}\NormalTok{,}
 \AttributeTok{n\_workers   =} \DecValTok{12}\NormalTok{,}
 \AttributeTok{radii     =} \FunctionTok{c}\NormalTok{(}\StringTok{"r3000"}\NormalTok{),}
 \AttributeTok{radius\_mode  =} \StringTok{"sparse"}\NormalTok{,}
 \AttributeTok{extract\_fun  =} \StringTok{"sum"}\NormalTok{,}
 \AttributeTok{fill\_missing  =} \ConstantTok{TRUE}\NormalTok{,}
 \AttributeTok{IDW\_weight   =} \DecValTok{2}\NormalTok{,}
 \AttributeTok{future\_max\_size =} \DecValTok{20} \SpecialCharTok{*} \DecValTok{1024}\SpecialCharTok{\^{}}\DecValTok{3}\NormalTok{)}


\CommentTok{\# Edges\_Water{-}Farmland\_r10000.tif   egv\_169 {-}{-}{-}{-}}
\NormalTok{slanis}\OtherTok{=}\FunctionTok{rast}\NormalTok{(}\StringTok{"./RasterGrids\_100m/2024/RAW/Edges\_Water{-}Farmland\_r10000.tif"}\NormalTok{)}
\FunctionTok{names}\NormalTok{(slanis)}\OtherTok{=}\StringTok{"egv\_169"}
\NormalTok{slanis2}\OtherTok{=}\FunctionTok{project}\NormalTok{(slanis,template100)}
\FunctionTok{writeRaster}\NormalTok{(slanis2,}
      \StringTok{"./RasterGrids\_100m/2024/RAW/Edges\_Water{-}Farmland\_r10000.tif"}\NormalTok{,}
      \AttributeTok{overwrite=}\ConstantTok{TRUE}\NormalTok{)}

\CommentTok{\# standardisation {-}{-}{-}{-}}
\ControlFlowTok{if}\NormalTok{(}\SpecialCharTok{!}\FunctionTok{require}\NormalTok{(terra)) \{}\FunctionTok{install.packages}\NormalTok{(}\StringTok{"terra"}\NormalTok{); }\FunctionTok{require}\NormalTok{(terra)\}}
\ControlFlowTok{if}\NormalTok{(}\SpecialCharTok{!}\FunctionTok{require}\NormalTok{(tidyverse)) \{}\FunctionTok{install.packages}\NormalTok{(}\StringTok{"tidyverse"}\NormalTok{); }\FunctionTok{require}\NormalTok{(tidyverse)\}}

\NormalTok{nosaukums}\OtherTok{=}\StringTok{"Edges\_Water{-}Farmland\_r10000.tif"}
\NormalTok{ielasisanas\_cels}\OtherTok{=}\FunctionTok{paste0}\NormalTok{(}\StringTok{"./RasterGrids\_100m/2024/RAW/"}\NormalTok{,nosaukums)}
\NormalTok{saglabasanas\_cels}\OtherTok{=}\FunctionTok{paste0}\NormalTok{(}\StringTok{"./RasterGrids\_100m/2024/Scaled/"}\NormalTok{,nosaukums)}
\NormalTok{slanis}\OtherTok{=}\FunctionTok{rast}\NormalTok{(ielasisanas\_cels)}
\NormalTok{videjais}\OtherTok{=}\FunctionTok{global}\NormalTok{(slanis,}\AttributeTok{fun=}\StringTok{"mean"}\NormalTok{,}\AttributeTok{na.rm=}\ConstantTok{TRUE}\NormalTok{)}
\NormalTok{centrets}\OtherTok{=}\NormalTok{slanis}\SpecialCharTok{{-}}\NormalTok{videjais[,}\DecValTok{1}\NormalTok{]}
\NormalTok{standartnovirze}\OtherTok{=}\NormalTok{terra}\SpecialCharTok{::}\FunctionTok{global}\NormalTok{(centrets,}\AttributeTok{fun=}\StringTok{"rms"}\NormalTok{,}\AttributeTok{na.rm=}\ConstantTok{TRUE}\NormalTok{)}
\NormalTok{merogots}\OtherTok{=}\NormalTok{centrets}\SpecialCharTok{/}\NormalTok{standartnovirze[,}\DecValTok{1}\NormalTok{]}
\FunctionTok{writeRaster}\NormalTok{(merogots,}
      \AttributeTok{filename=}\NormalTok{saglabasanas\_cels,}
      \AttributeTok{overwrite=}\ConstantTok{TRUE}\NormalTok{)}
\end{Highlighting}
\end{Shaded}

\section{Edges\_Water-Grassland\_cell}\label{ch06.170}

\textbf{filename:} \texttt{Edges\_Water-Grassland\_cell.tif}

\textbf{layername:} \texttt{egv\_170}

\textbf{English name:} Edge pixels of Water bordering with Grassland within the
analysis cell (1 ha)

\textbf{Latvian name:} Ūdenstilpju malu ar zālājiem pikseļu skaits analīzes šūnā (1 ha)

\textbf{Procedure:} First, values lequal to 330 from the \hyperref[Ch05.03]{Landscape
classification} are coded as 1, and all other values as NA. Then values
equal to 200 from the \hyperref[Ch05.03]{Landscape classification} are coded as 0, and
all other values as NA. Then, the first layer (1 = presence) is covered over the
second layer (presence = 0) and written to file (matching the input). Next,
with the workflow \texttt{egvtools::landscape\_function()} total edge between the two
classes is calculated. During the calculation of the landscape metric, inverse distance
weighted (power = 2) gap filling on the output is applied to ensure no
missing values at the edges. Finally, the layer is standardised by
subtracting the arithmetic mean and dividing by the root mean squared error.

\begin{Shaded}
\begin{Highlighting}[]
\CommentTok{\# libs {-}{-}{-}{-}}
\ControlFlowTok{if}\NormalTok{(}\SpecialCharTok{!}\FunctionTok{require}\NormalTok{(terra)) \{}\FunctionTok{install.packages}\NormalTok{(}\StringTok{"terra"}\NormalTok{); }\FunctionTok{require}\NormalTok{(terra)\}}
\ControlFlowTok{if}\NormalTok{(}\SpecialCharTok{!}\FunctionTok{require}\NormalTok{(egvtools)) \{remotes}\SpecialCharTok{::}\FunctionTok{install\_github}\NormalTok{(}\StringTok{"aavotins/egvtools"}\NormalTok{); }\FunctionTok{require}\NormalTok{(egvtools)\}}

\ControlFlowTok{if}\NormalTok{(}\SpecialCharTok{!}\FunctionTok{require}\NormalTok{(sf)) \{}\FunctionTok{install.packages}\NormalTok{(}\StringTok{"sf"}\NormalTok{); }\FunctionTok{require}\NormalTok{(sf)\}}
\ControlFlowTok{if}\NormalTok{(}\SpecialCharTok{!}\FunctionTok{require}\NormalTok{(sfarrow)) \{}\FunctionTok{install.packages}\NormalTok{(}\StringTok{"sfarrow"}\NormalTok{); }\FunctionTok{require}\NormalTok{(sfarrow)\}}
\ControlFlowTok{if}\NormalTok{(}\SpecialCharTok{!}\FunctionTok{require}\NormalTok{(raster)) \{}\FunctionTok{install.packages}\NormalTok{(}\StringTok{"raster"}\NormalTok{); }\FunctionTok{require}\NormalTok{(raster)\}}
\ControlFlowTok{if}\NormalTok{(}\SpecialCharTok{!}\FunctionTok{require}\NormalTok{(fasterize)) \{}\FunctionTok{install.packages}\NormalTok{(}\StringTok{"fasterize"}\NormalTok{); }\FunctionTok{require}\NormalTok{(fasterize)\}}
\ControlFlowTok{if}\NormalTok{(}\SpecialCharTok{!}\FunctionTok{require}\NormalTok{(tidyverse)) \{}\FunctionTok{install.packages}\NormalTok{(}\StringTok{"tidyverse"}\NormalTok{); }\FunctionTok{require}\NormalTok{(tidyverse)\}}


\CommentTok{\# Templates {-}{-}{-}{-}{-}}
\NormalTok{template10}\OtherTok{=}\FunctionTok{rast}\NormalTok{(}\StringTok{"./Templates/TemplateRasters/LV10m\_10km.tif"}\NormalTok{)}
\NormalTok{nulls10}\OtherTok{=}\FunctionTok{rast}\NormalTok{(}\StringTok{"./Templates/TemplateRasters/nulls\_LV10m\_10km.tif"}\NormalTok{)}

\CommentTok{\# simple landscape {-}{-}{-}{-}}
\NormalTok{simple\_landscape}\OtherTok{=}\FunctionTok{rast}\NormalTok{(}\StringTok{"./RasterGrids\_10m/2024/Ainava\_vienk\_mask.tif"}\NormalTok{)}

\CommentTok{\# Edges\_Water{-}Grassland\_input.tif {-}{-}{-}{-}}
\NormalTok{water}\OtherTok{=}\FunctionTok{ifel}\NormalTok{(simple\_landscape}\SpecialCharTok{==}\DecValTok{200}\NormalTok{,}\DecValTok{0}\NormalTok{,}\ConstantTok{NA}\NormalTok{)}
\FunctionTok{plot}\NormalTok{(water)}

\NormalTok{grassland}\OtherTok{=}\FunctionTok{ifel}\NormalTok{(simple\_landscape}\SpecialCharTok{==}\DecValTok{330}\NormalTok{,}\DecValTok{1}\NormalTok{,}\ConstantTok{NA}\NormalTok{)}
\FunctionTok{plot}\NormalTok{(grassland)}

\NormalTok{water\_grassland}\OtherTok{=}\FunctionTok{cover}\NormalTok{(water,grassland)}
\FunctionTok{plot}\NormalTok{(water\_grassland)}

\NormalTok{edge\_water\_grassland}\OtherTok{=}\FunctionTok{project}\NormalTok{(water\_grassland,template10,}
              \AttributeTok{filename=}\StringTok{"./RasterGrids\_10m/2024/Edges\_Water{-}Grassland\_input.tif"}\NormalTok{,}
              \AttributeTok{overwrite=}\ConstantTok{TRUE}\NormalTok{)}
\FunctionTok{rm}\NormalTok{(edge\_water\_grassland)}
\FunctionTok{rm}\NormalTok{(water\_grassland)}


\CommentTok{\# Edges\_Water{-}Grassland\_cell.tif    egv\_170 {-}{-}{-}{-}}
\FunctionTok{landscape\_function}\NormalTok{(}
 \AttributeTok{landscape   =} \StringTok{"./RasterGrids\_10m/2024/Edges\_Water{-}Grassland\_input.tif"}\NormalTok{,}
 \AttributeTok{zones     =} \StringTok{"./Templates/TemplateGrids/tikls100\_sauzeme.parquet"}\NormalTok{,}
 \AttributeTok{id\_field    =} \StringTok{"id"}\NormalTok{,}
 \AttributeTok{tile\_field   =} \StringTok{"tks50km"}\NormalTok{,}
 \AttributeTok{template    =} \StringTok{"./Templates/TemplateRasters/LV100m\_10km.tif"}\NormalTok{,}
 \AttributeTok{out\_dir    =} \StringTok{"./RasterGrids\_100m/2024/RAW"}\NormalTok{,}
 \AttributeTok{out\_filename  =} \StringTok{"Edges\_Water{-}Grassland\_cell.tif"}\NormalTok{,}
 \AttributeTok{out\_layername =} \StringTok{"egv\_170"}\NormalTok{,}
 \AttributeTok{what       =} \StringTok{"lsm\_l\_te"}\NormalTok{,}
 \AttributeTok{lm\_args     =} \FunctionTok{list}\NormalTok{(}\AttributeTok{count\_boundary =} \ConstantTok{FALSE}\NormalTok{),}
 \AttributeTok{rasterize\_engine =} \StringTok{"fasterize"}\NormalTok{,}
 \AttributeTok{n\_workers   =} \DecValTok{12}\NormalTok{,}
 \AttributeTok{future\_max\_size =} \DecValTok{20} \SpecialCharTok{*} \DecValTok{1024}\SpecialCharTok{\^{}}\DecValTok{3}\NormalTok{,}
 \AttributeTok{fill\_gaps   =} \ConstantTok{TRUE}\NormalTok{,}
 \AttributeTok{plot\_gaps   =} \ConstantTok{FALSE}\NormalTok{,}
 \AttributeTok{plot\_result  =} \ConstantTok{FALSE}
\NormalTok{)}

\CommentTok{\# standardisation {-}{-}{-}{-}}
\ControlFlowTok{if}\NormalTok{(}\SpecialCharTok{!}\FunctionTok{require}\NormalTok{(terra)) \{}\FunctionTok{install.packages}\NormalTok{(}\StringTok{"terra"}\NormalTok{); }\FunctionTok{require}\NormalTok{(terra)\}}
\ControlFlowTok{if}\NormalTok{(}\SpecialCharTok{!}\FunctionTok{require}\NormalTok{(tidyverse)) \{}\FunctionTok{install.packages}\NormalTok{(}\StringTok{"tidyverse"}\NormalTok{); }\FunctionTok{require}\NormalTok{(tidyverse)\}}

\NormalTok{nosaukums}\OtherTok{=}\StringTok{"Edges\_Water{-}Grassland\_cell.tif"}
\NormalTok{ielasisanas\_cels}\OtherTok{=}\FunctionTok{paste0}\NormalTok{(}\StringTok{"./RasterGrids\_100m/2024/RAW/"}\NormalTok{,nosaukums)}
\NormalTok{saglabasanas\_cels}\OtherTok{=}\FunctionTok{paste0}\NormalTok{(}\StringTok{"./RasterGrids\_100m/2024/Scaled/"}\NormalTok{,nosaukums)}
\NormalTok{slanis}\OtherTok{=}\FunctionTok{rast}\NormalTok{(ielasisanas\_cels)}
\NormalTok{videjais}\OtherTok{=}\FunctionTok{global}\NormalTok{(slanis,}\AttributeTok{fun=}\StringTok{"mean"}\NormalTok{,}\AttributeTok{na.rm=}\ConstantTok{TRUE}\NormalTok{)}
\NormalTok{centrets}\OtherTok{=}\NormalTok{slanis}\SpecialCharTok{{-}}\NormalTok{videjais[,}\DecValTok{1}\NormalTok{]}
\NormalTok{standartnovirze}\OtherTok{=}\NormalTok{terra}\SpecialCharTok{::}\FunctionTok{global}\NormalTok{(centrets,}\AttributeTok{fun=}\StringTok{"rms"}\NormalTok{,}\AttributeTok{na.rm=}\ConstantTok{TRUE}\NormalTok{)}
\NormalTok{merogots}\OtherTok{=}\NormalTok{centrets}\SpecialCharTok{/}\NormalTok{standartnovirze[,}\DecValTok{1}\NormalTok{]}
\FunctionTok{writeRaster}\NormalTok{(merogots,}
      \AttributeTok{filename=}\NormalTok{saglabasanas\_cels,}
      \AttributeTok{overwrite=}\ConstantTok{TRUE}\NormalTok{)}
\end{Highlighting}
\end{Shaded}

\section{Edges\_Water-Grassland\_r500}\label{ch06.171}

\textbf{filename:} \texttt{Edges\_Water-Grassland\_r500.tif}

\textbf{layername:} \texttt{egv\_171}

\textbf{English name:} Edge pixels of Water bordering with Grassland within the 0.5
km landscape

\textbf{Latvian name:} Ūdenstilpju malu ar zālājiem pikseļu skaits 0,5 km ainavā

\textbf{Procedure:} The total edge within a 500 m radius around the analysis grid cell is
calculated as the area-weighted sum of the \hyperref[ch06.170]{analysis cells} inside the
buffer, using the workflow \texttt{egvtools::radius\_function()}. During the calculation of the landscape metric,
inverse distance weighted (power = 2) gap filling on the output is applied
to ensure no missing values at the edges. Then the layer is rewritten to set
its name. Finally, the layer is standardised by subtracting the arithmetic
mean and dividing by the root mean squared error.

\begin{Shaded}
\begin{Highlighting}[]
\CommentTok{\# libs {-}{-}{-}{-}}
\ControlFlowTok{if}\NormalTok{(}\SpecialCharTok{!}\FunctionTok{require}\NormalTok{(terra)) \{}\FunctionTok{install.packages}\NormalTok{(}\StringTok{"terra"}\NormalTok{); }\FunctionTok{require}\NormalTok{(terra)\}}
\ControlFlowTok{if}\NormalTok{(}\SpecialCharTok{!}\FunctionTok{require}\NormalTok{(egvtools)) \{remotes}\SpecialCharTok{::}\FunctionTok{install\_github}\NormalTok{(}\StringTok{"aavotins/egvtools"}\NormalTok{); }\FunctionTok{require}\NormalTok{(egvtools)\}}


\CommentTok{\# Templates {-}{-}{-}{-}{-}}
\NormalTok{template100}\OtherTok{=}\FunctionTok{rast}\NormalTok{(}\StringTok{"./Templates/TemplateRasters/LV100m\_10km.tif"}\NormalTok{)}

\CommentTok{\# radii {-}{-}{-}{-}}
\FunctionTok{radius\_function}\NormalTok{(}
 \AttributeTok{kvadrati\_path =} \StringTok{"./Templates/TemplateGrids/tiles/"}\NormalTok{,}
 \AttributeTok{radii\_path   =} \StringTok{"./Templates/TemplateGridPoints/tiles/"}\NormalTok{,}
 \AttributeTok{tikls100\_path =} \StringTok{"./Templates/TemplateGrids/tikls100\_sauzeme.parquet"}\NormalTok{,}
 \AttributeTok{template\_path =} \StringTok{"./Templates/TemplateRasters/LV100m\_10km.tif"}\NormalTok{,}
 \AttributeTok{input\_layers  =} \FunctionTok{c}\NormalTok{(}\StringTok{"./RasterGrids\_100m/2024/RAW/Edges\_Water{-}Grassland\_cell.tif"}\NormalTok{),}
 \AttributeTok{layer\_prefixes =} \FunctionTok{c}\NormalTok{(}\StringTok{"Edges\_Water{-}Grassland"}\NormalTok{),}
 \AttributeTok{output\_dir   =} \StringTok{"./RasterGrids\_100m/2024/RAW/"}\NormalTok{,}
 \AttributeTok{n\_workers   =} \DecValTok{12}\NormalTok{,}
 \AttributeTok{radii     =} \FunctionTok{c}\NormalTok{(}\StringTok{"r500"}\NormalTok{),}
 \AttributeTok{radius\_mode  =} \StringTok{"sparse"}\NormalTok{,}
 \AttributeTok{extract\_fun  =} \StringTok{"sum"}\NormalTok{,}
 \AttributeTok{fill\_missing  =} \ConstantTok{TRUE}\NormalTok{,}
 \AttributeTok{IDW\_weight   =} \DecValTok{2}\NormalTok{,}
 \AttributeTok{future\_max\_size =} \DecValTok{20} \SpecialCharTok{*} \DecValTok{1024}\SpecialCharTok{\^{}}\DecValTok{3}\NormalTok{)}


\CommentTok{\# Edges\_Water{-}Grassland\_r500.tif    egv\_171 {-}{-}{-}{-}}
\NormalTok{slanis}\OtherTok{=}\FunctionTok{rast}\NormalTok{(}\StringTok{"./RasterGrids\_100m/2024/RAW/Edges\_Water{-}Grassland\_r500.tif"}\NormalTok{)}
\FunctionTok{names}\NormalTok{(slanis)}\OtherTok{=}\StringTok{"egv\_171"}
\NormalTok{slanis2}\OtherTok{=}\FunctionTok{project}\NormalTok{(slanis,template100)}
\FunctionTok{writeRaster}\NormalTok{(slanis2,}
      \StringTok{"./RasterGrids\_100m/2024/RAW/Edges\_Water{-}Grassland\_r500.tif"}\NormalTok{,}
      \AttributeTok{overwrite=}\ConstantTok{TRUE}\NormalTok{)}

\CommentTok{\# standardisation {-}{-}{-}{-}}
\ControlFlowTok{if}\NormalTok{(}\SpecialCharTok{!}\FunctionTok{require}\NormalTok{(terra)) \{}\FunctionTok{install.packages}\NormalTok{(}\StringTok{"terra"}\NormalTok{); }\FunctionTok{require}\NormalTok{(terra)\}}
\ControlFlowTok{if}\NormalTok{(}\SpecialCharTok{!}\FunctionTok{require}\NormalTok{(tidyverse)) \{}\FunctionTok{install.packages}\NormalTok{(}\StringTok{"tidyverse"}\NormalTok{); }\FunctionTok{require}\NormalTok{(tidyverse)\}}

\NormalTok{nosaukums}\OtherTok{=}\StringTok{"Edges\_Water{-}Grassland\_r500.tif"}
\NormalTok{ielasisanas\_cels}\OtherTok{=}\FunctionTok{paste0}\NormalTok{(}\StringTok{"./RasterGrids\_100m/2024/RAW/"}\NormalTok{,nosaukums)}
\NormalTok{saglabasanas\_cels}\OtherTok{=}\FunctionTok{paste0}\NormalTok{(}\StringTok{"./RasterGrids\_100m/2024/Scaled/"}\NormalTok{,nosaukums)}
\NormalTok{slanis}\OtherTok{=}\FunctionTok{rast}\NormalTok{(ielasisanas\_cels)}
\NormalTok{videjais}\OtherTok{=}\FunctionTok{global}\NormalTok{(slanis,}\AttributeTok{fun=}\StringTok{"mean"}\NormalTok{,}\AttributeTok{na.rm=}\ConstantTok{TRUE}\NormalTok{)}
\NormalTok{centrets}\OtherTok{=}\NormalTok{slanis}\SpecialCharTok{{-}}\NormalTok{videjais[,}\DecValTok{1}\NormalTok{]}
\NormalTok{standartnovirze}\OtherTok{=}\NormalTok{terra}\SpecialCharTok{::}\FunctionTok{global}\NormalTok{(centrets,}\AttributeTok{fun=}\StringTok{"rms"}\NormalTok{,}\AttributeTok{na.rm=}\ConstantTok{TRUE}\NormalTok{)}
\NormalTok{merogots}\OtherTok{=}\NormalTok{centrets}\SpecialCharTok{/}\NormalTok{standartnovirze[,}\DecValTok{1}\NormalTok{]}
\FunctionTok{writeRaster}\NormalTok{(merogots,}
      \AttributeTok{filename=}\NormalTok{saglabasanas\_cels,}
      \AttributeTok{overwrite=}\ConstantTok{TRUE}\NormalTok{)}
\end{Highlighting}
\end{Shaded}

\section{Edges\_Water-Grassland\_r1250}\label{ch06.172}

\textbf{filename:} \texttt{Edges\_Water-Grassland\_r1250.tif}

\textbf{layername:} \texttt{egv\_172}

\textbf{English name:} Edge pixels of Water bordering with Grassland within the 1.25
km landscape

\textbf{Latvian name:} Ūdenstilpju malu ar zālājiem pikseļu skaits 1,25 km ainavā

\textbf{Procedure:} The total edge within a 1250 m radius around the analysis grid cell is
calculated as the area-weighted sum of the \hyperref[ch06.170]{analysis cells} inside the
buffer, using the workflow \texttt{egvtools::radius\_function()}. During the calculation of the landscape metric,
inverse distance weighted (power = 2) gap filling on the output is applied
to ensure no missing values at the edges. Then the layer is rewritten to set
its name. Finally, the layer is standardised by subtracting the arithmetic
mean and dividing by the root mean squared error.

\begin{Shaded}
\begin{Highlighting}[]
\CommentTok{\# libs {-}{-}{-}{-}}
\ControlFlowTok{if}\NormalTok{(}\SpecialCharTok{!}\FunctionTok{require}\NormalTok{(terra)) \{}\FunctionTok{install.packages}\NormalTok{(}\StringTok{"terra"}\NormalTok{); }\FunctionTok{require}\NormalTok{(terra)\}}
\ControlFlowTok{if}\NormalTok{(}\SpecialCharTok{!}\FunctionTok{require}\NormalTok{(egvtools)) \{remotes}\SpecialCharTok{::}\FunctionTok{install\_github}\NormalTok{(}\StringTok{"aavotins/egvtools"}\NormalTok{); }\FunctionTok{require}\NormalTok{(egvtools)\}}


\CommentTok{\# Templates {-}{-}{-}{-}{-}}
\NormalTok{template100}\OtherTok{=}\FunctionTok{rast}\NormalTok{(}\StringTok{"./Templates/TemplateRasters/LV100m\_10km.tif"}\NormalTok{)}

\CommentTok{\# radii {-}{-}{-}{-}}
\FunctionTok{radius\_function}\NormalTok{(}
 \AttributeTok{kvadrati\_path =} \StringTok{"./Templates/TemplateGrids/tiles/"}\NormalTok{,}
 \AttributeTok{radii\_path   =} \StringTok{"./Templates/TemplateGridPoints/tiles/"}\NormalTok{,}
 \AttributeTok{tikls100\_path =} \StringTok{"./Templates/TemplateGrids/tikls100\_sauzeme.parquet"}\NormalTok{,}
 \AttributeTok{template\_path =} \StringTok{"./Templates/TemplateRasters/LV100m\_10km.tif"}\NormalTok{,}
 \AttributeTok{input\_layers  =} \FunctionTok{c}\NormalTok{(}\StringTok{"./RasterGrids\_100m/2024/RAW/Edges\_Water{-}Grassland\_cell.tif"}\NormalTok{),}
 \AttributeTok{layer\_prefixes =} \FunctionTok{c}\NormalTok{(}\StringTok{"Edges\_Water{-}Grassland"}\NormalTok{),}
 \AttributeTok{output\_dir   =} \StringTok{"./RasterGrids\_100m/2024/RAW/"}\NormalTok{,}
 \AttributeTok{n\_workers   =} \DecValTok{12}\NormalTok{,}
 \AttributeTok{radii     =} \FunctionTok{c}\NormalTok{(}\StringTok{"r1250"}\NormalTok{),}
 \AttributeTok{radius\_mode  =} \StringTok{"sparse"}\NormalTok{,}
 \AttributeTok{extract\_fun  =} \StringTok{"sum"}\NormalTok{,}
 \AttributeTok{fill\_missing  =} \ConstantTok{TRUE}\NormalTok{,}
 \AttributeTok{IDW\_weight   =} \DecValTok{2}\NormalTok{,}
 \AttributeTok{future\_max\_size =} \DecValTok{20} \SpecialCharTok{*} \DecValTok{1024}\SpecialCharTok{\^{}}\DecValTok{3}\NormalTok{)}


\CommentTok{\# Edges\_Water{-}Grassland\_r1250.tif   egv\_172 {-}{-}{-}{-}}
\NormalTok{slanis}\OtherTok{=}\FunctionTok{rast}\NormalTok{(}\StringTok{"./RasterGrids\_100m/2024/RAW/Edges\_Water{-}Grassland\_r1250.tif"}\NormalTok{)}
\FunctionTok{names}\NormalTok{(slanis)}\OtherTok{=}\StringTok{"egv\_172"}
\NormalTok{slanis2}\OtherTok{=}\FunctionTok{project}\NormalTok{(slanis,template100)}
\FunctionTok{writeRaster}\NormalTok{(slanis2,}
      \StringTok{"./RasterGrids\_100m/2024/RAW/Edges\_Water{-}Grassland\_r1250.tif"}\NormalTok{,}
      \AttributeTok{overwrite=}\ConstantTok{TRUE}\NormalTok{)}

\CommentTok{\# standardisation {-}{-}{-}{-}}
\ControlFlowTok{if}\NormalTok{(}\SpecialCharTok{!}\FunctionTok{require}\NormalTok{(terra)) \{}\FunctionTok{install.packages}\NormalTok{(}\StringTok{"terra"}\NormalTok{); }\FunctionTok{require}\NormalTok{(terra)\}}
\ControlFlowTok{if}\NormalTok{(}\SpecialCharTok{!}\FunctionTok{require}\NormalTok{(tidyverse)) \{}\FunctionTok{install.packages}\NormalTok{(}\StringTok{"tidyverse"}\NormalTok{); }\FunctionTok{require}\NormalTok{(tidyverse)\}}

\NormalTok{nosaukums}\OtherTok{=}\StringTok{"Edges\_Water{-}Grassland\_r1250.tif"}
\NormalTok{ielasisanas\_cels}\OtherTok{=}\FunctionTok{paste0}\NormalTok{(}\StringTok{"./RasterGrids\_100m/2024/RAW/"}\NormalTok{,nosaukums)}
\NormalTok{saglabasanas\_cels}\OtherTok{=}\FunctionTok{paste0}\NormalTok{(}\StringTok{"./RasterGrids\_100m/2024/Scaled/"}\NormalTok{,nosaukums)}
\NormalTok{slanis}\OtherTok{=}\FunctionTok{rast}\NormalTok{(ielasisanas\_cels)}
\NormalTok{videjais}\OtherTok{=}\FunctionTok{global}\NormalTok{(slanis,}\AttributeTok{fun=}\StringTok{"mean"}\NormalTok{,}\AttributeTok{na.rm=}\ConstantTok{TRUE}\NormalTok{)}
\NormalTok{centrets}\OtherTok{=}\NormalTok{slanis}\SpecialCharTok{{-}}\NormalTok{videjais[,}\DecValTok{1}\NormalTok{]}
\NormalTok{standartnovirze}\OtherTok{=}\NormalTok{terra}\SpecialCharTok{::}\FunctionTok{global}\NormalTok{(centrets,}\AttributeTok{fun=}\StringTok{"rms"}\NormalTok{,}\AttributeTok{na.rm=}\ConstantTok{TRUE}\NormalTok{)}
\NormalTok{merogots}\OtherTok{=}\NormalTok{centrets}\SpecialCharTok{/}\NormalTok{standartnovirze[,}\DecValTok{1}\NormalTok{]}
\FunctionTok{writeRaster}\NormalTok{(merogots,}
      \AttributeTok{filename=}\NormalTok{saglabasanas\_cels,}
      \AttributeTok{overwrite=}\ConstantTok{TRUE}\NormalTok{)}
\end{Highlighting}
\end{Shaded}

\section{Edges\_Water-Grassland\_r3000}\label{ch06.173}

\textbf{filename:} \texttt{Edges\_Water-Grassland\_r3000.tif}

\textbf{layername:} \texttt{egv\_173}

\textbf{English name:} Edge pixels of Water bordering with Grassland within the 3 km
landscape

\textbf{Latvian name:} Ūdenstilpju malu ar zālājiem pikseļu skaits 3 km ainavā

\textbf{Procedure:} The total edge within a 3000 m radius around the analysis grid cell is
calculated as the area-weighted sum of the \hyperref[ch06.170]{analysis cells} inside the
buffer, using the workflow \texttt{egvtools::radius\_function()}. During the calculation of the landscape metric,
inverse distance weighted (power = 2) gap filling on the output is applied
to ensure no missing values at the edges. Then the layer is rewritten to set
its name. Finally, the layer is standardised by subtracting the arithmetic
mean and dividing by the root mean squared error.

\begin{Shaded}
\begin{Highlighting}[]
\CommentTok{\# libs {-}{-}{-}{-}}
\ControlFlowTok{if}\NormalTok{(}\SpecialCharTok{!}\FunctionTok{require}\NormalTok{(terra)) \{}\FunctionTok{install.packages}\NormalTok{(}\StringTok{"terra"}\NormalTok{); }\FunctionTok{require}\NormalTok{(terra)\}}
\ControlFlowTok{if}\NormalTok{(}\SpecialCharTok{!}\FunctionTok{require}\NormalTok{(egvtools)) \{remotes}\SpecialCharTok{::}\FunctionTok{install\_github}\NormalTok{(}\StringTok{"aavotins/egvtools"}\NormalTok{); }\FunctionTok{require}\NormalTok{(egvtools)\}}


\CommentTok{\# Templates {-}{-}{-}{-}{-}}
\NormalTok{template100}\OtherTok{=}\FunctionTok{rast}\NormalTok{(}\StringTok{"./Templates/TemplateRasters/LV100m\_10km.tif"}\NormalTok{)}

\CommentTok{\# radii {-}{-}{-}{-}}
\FunctionTok{radius\_function}\NormalTok{(}
 \AttributeTok{kvadrati\_path =} \StringTok{"./Templates/TemplateGrids/tiles/"}\NormalTok{,}
 \AttributeTok{radii\_path   =} \StringTok{"./Templates/TemplateGridPoints/tiles/"}\NormalTok{,}
 \AttributeTok{tikls100\_path =} \StringTok{"./Templates/TemplateGrids/tikls100\_sauzeme.parquet"}\NormalTok{,}
 \AttributeTok{template\_path =} \StringTok{"./Templates/TemplateRasters/LV100m\_10km.tif"}\NormalTok{,}
 \AttributeTok{input\_layers  =} \FunctionTok{c}\NormalTok{(}\StringTok{"./RasterGrids\_100m/2024/RAW/Edges\_Water{-}Grassland\_cell.tif"}\NormalTok{),}
 \AttributeTok{layer\_prefixes =} \FunctionTok{c}\NormalTok{(}\StringTok{"Edges\_Water{-}Grassland"}\NormalTok{),}
 \AttributeTok{output\_dir   =} \StringTok{"./RasterGrids\_100m/2024/RAW/"}\NormalTok{,}
 \AttributeTok{n\_workers   =} \DecValTok{12}\NormalTok{,}
 \AttributeTok{radii     =} \FunctionTok{c}\NormalTok{(}\StringTok{"r3000"}\NormalTok{),}
 \AttributeTok{radius\_mode  =} \StringTok{"sparse"}\NormalTok{,}
 \AttributeTok{extract\_fun  =} \StringTok{"sum"}\NormalTok{,}
 \AttributeTok{fill\_missing  =} \ConstantTok{TRUE}\NormalTok{,}
 \AttributeTok{IDW\_weight   =} \DecValTok{2}\NormalTok{,}
 \AttributeTok{future\_max\_size =} \DecValTok{20} \SpecialCharTok{*} \DecValTok{1024}\SpecialCharTok{\^{}}\DecValTok{3}\NormalTok{)}


\CommentTok{\# Edges\_Water{-}Grassland\_r3000.tif   egv\_173 {-}{-}{-}{-}}
\NormalTok{slanis}\OtherTok{=}\FunctionTok{rast}\NormalTok{(}\StringTok{"./RasterGrids\_100m/2024/RAW/Edges\_Water{-}Grassland\_r3000.tif"}\NormalTok{)}
\FunctionTok{names}\NormalTok{(slanis)}\OtherTok{=}\StringTok{"egv\_173"}
\NormalTok{slanis2}\OtherTok{=}\FunctionTok{project}\NormalTok{(slanis,template100)}
\FunctionTok{writeRaster}\NormalTok{(slanis2,}
      \StringTok{"./RasterGrids\_100m/2024/RAW/Edges\_Water{-}Grassland\_r3000.tif"}\NormalTok{,}
      \AttributeTok{overwrite=}\ConstantTok{TRUE}\NormalTok{)}

\CommentTok{\# standardisation {-}{-}{-}{-}}
\ControlFlowTok{if}\NormalTok{(}\SpecialCharTok{!}\FunctionTok{require}\NormalTok{(terra)) \{}\FunctionTok{install.packages}\NormalTok{(}\StringTok{"terra"}\NormalTok{); }\FunctionTok{require}\NormalTok{(terra)\}}
\ControlFlowTok{if}\NormalTok{(}\SpecialCharTok{!}\FunctionTok{require}\NormalTok{(tidyverse)) \{}\FunctionTok{install.packages}\NormalTok{(}\StringTok{"tidyverse"}\NormalTok{); }\FunctionTok{require}\NormalTok{(tidyverse)\}}

\NormalTok{nosaukums}\OtherTok{=}\StringTok{"Edges\_Water{-}Grassland\_r3000.tif"}
\NormalTok{ielasisanas\_cels}\OtherTok{=}\FunctionTok{paste0}\NormalTok{(}\StringTok{"./RasterGrids\_100m/2024/RAW/"}\NormalTok{,nosaukums)}
\NormalTok{saglabasanas\_cels}\OtherTok{=}\FunctionTok{paste0}\NormalTok{(}\StringTok{"./RasterGrids\_100m/2024/Scaled/"}\NormalTok{,nosaukums)}
\NormalTok{slanis}\OtherTok{=}\FunctionTok{rast}\NormalTok{(ielasisanas\_cels)}
\NormalTok{videjais}\OtherTok{=}\FunctionTok{global}\NormalTok{(slanis,}\AttributeTok{fun=}\StringTok{"mean"}\NormalTok{,}\AttributeTok{na.rm=}\ConstantTok{TRUE}\NormalTok{)}
\NormalTok{centrets}\OtherTok{=}\NormalTok{slanis}\SpecialCharTok{{-}}\NormalTok{videjais[,}\DecValTok{1}\NormalTok{]}
\NormalTok{standartnovirze}\OtherTok{=}\NormalTok{terra}\SpecialCharTok{::}\FunctionTok{global}\NormalTok{(centrets,}\AttributeTok{fun=}\StringTok{"rms"}\NormalTok{,}\AttributeTok{na.rm=}\ConstantTok{TRUE}\NormalTok{)}
\NormalTok{merogots}\OtherTok{=}\NormalTok{centrets}\SpecialCharTok{/}\NormalTok{standartnovirze[,}\DecValTok{1}\NormalTok{]}
\FunctionTok{writeRaster}\NormalTok{(merogots,}
      \AttributeTok{filename=}\NormalTok{saglabasanas\_cels,}
      \AttributeTok{overwrite=}\ConstantTok{TRUE}\NormalTok{)}
\end{Highlighting}
\end{Shaded}

\section{Edges\_Water-Grassland\_r10000}\label{ch06.174}

\textbf{filename:} \texttt{Edges\_Water-Grassland\_r10000.tif}

\textbf{layername:} \texttt{egv\_174}

\textbf{English name:} Edge pixels of Water bordering with Grassland within the 10 km
landscape

\textbf{Latvian name:} Ūdenstilpju malu ar zālājiem pikseļu skaits 10 km ainavā

\textbf{Procedure:} The total edge within a 10000 m radius around the analysis grid cell is
calculated as the area-weighted sum of the \hyperref[ch06.170]{analysis cells} inside the
buffer, using the workflow \texttt{egvtools::radius\_function()}. During the calculation of the landscape metric,
inverse distance weighted (power = 2) gap filling on the output is applied
to ensure no missing values at the edges. Then the layer is rewritten to set
its name. Finally, the layer is standardised by subtracting the arithmetic
mean and dividing by the root mean squared error.

\begin{Shaded}
\begin{Highlighting}[]
\CommentTok{\# libs {-}{-}{-}{-}}
\ControlFlowTok{if}\NormalTok{(}\SpecialCharTok{!}\FunctionTok{require}\NormalTok{(terra)) \{}\FunctionTok{install.packages}\NormalTok{(}\StringTok{"terra"}\NormalTok{); }\FunctionTok{require}\NormalTok{(terra)\}}
\ControlFlowTok{if}\NormalTok{(}\SpecialCharTok{!}\FunctionTok{require}\NormalTok{(egvtools)) \{remotes}\SpecialCharTok{::}\FunctionTok{install\_github}\NormalTok{(}\StringTok{"aavotins/egvtools"}\NormalTok{); }\FunctionTok{require}\NormalTok{(egvtools)\}}


\CommentTok{\# Templates {-}{-}{-}{-}{-}}
\NormalTok{template100}\OtherTok{=}\FunctionTok{rast}\NormalTok{(}\StringTok{"./Templates/TemplateRasters/LV100m\_10km.tif"}\NormalTok{)}

\CommentTok{\# radii {-}{-}{-}{-}}
\FunctionTok{radius\_function}\NormalTok{(}
 \AttributeTok{kvadrati\_path =} \StringTok{"./Templates/TemplateGrids/tiles/"}\NormalTok{,}
 \AttributeTok{radii\_path   =} \StringTok{"./Templates/TemplateGridPoints/tiles/"}\NormalTok{,}
 \AttributeTok{tikls100\_path =} \StringTok{"./Templates/TemplateGrids/tikls100\_sauzeme.parquet"}\NormalTok{,}
 \AttributeTok{template\_path =} \StringTok{"./Templates/TemplateRasters/LV100m\_10km.tif"}\NormalTok{,}
 \AttributeTok{input\_layers  =} \FunctionTok{c}\NormalTok{(}\StringTok{"./RasterGrids\_100m/2024/RAW/Edges\_Water{-}Grassland\_cell.tif"}\NormalTok{),}
 \AttributeTok{layer\_prefixes =} \FunctionTok{c}\NormalTok{(}\StringTok{"Edges\_Water{-}Grassland"}\NormalTok{),}
 \AttributeTok{output\_dir   =} \StringTok{"./RasterGrids\_100m/2024/RAW/"}\NormalTok{,}
 \AttributeTok{n\_workers   =} \DecValTok{12}\NormalTok{,}
 \AttributeTok{radii     =} \FunctionTok{c}\NormalTok{(}\StringTok{"r3000"}\NormalTok{),}
 \AttributeTok{radius\_mode  =} \StringTok{"sparse"}\NormalTok{,}
 \AttributeTok{extract\_fun  =} \StringTok{"sum"}\NormalTok{,}
 \AttributeTok{fill\_missing  =} \ConstantTok{TRUE}\NormalTok{,}
 \AttributeTok{IDW\_weight   =} \DecValTok{2}\NormalTok{,}
 \AttributeTok{future\_max\_size =} \DecValTok{20} \SpecialCharTok{*} \DecValTok{1024}\SpecialCharTok{\^{}}\DecValTok{3}\NormalTok{)}


\CommentTok{\# Edges\_Water{-}Grassland\_r10000.tif  egv\_174 {-}{-}{-}{-}}
\NormalTok{slanis}\OtherTok{=}\FunctionTok{rast}\NormalTok{(}\StringTok{"./RasterGrids\_100m/2024/RAW/Edges\_Water{-}Grassland\_r10000.tif"}\NormalTok{)}
\FunctionTok{names}\NormalTok{(slanis)}\OtherTok{=}\StringTok{"egv\_174"}
\NormalTok{slanis2}\OtherTok{=}\FunctionTok{project}\NormalTok{(slanis,template100)}
\FunctionTok{writeRaster}\NormalTok{(slanis2,}
      \StringTok{"./RasterGrids\_100m/2024/RAW/Edges\_Water{-}Grassland\_r10000.tif"}\NormalTok{,}
      \AttributeTok{overwrite=}\ConstantTok{TRUE}\NormalTok{)}

\CommentTok{\# standardisation {-}{-}{-}{-}}
\ControlFlowTok{if}\NormalTok{(}\SpecialCharTok{!}\FunctionTok{require}\NormalTok{(terra)) \{}\FunctionTok{install.packages}\NormalTok{(}\StringTok{"terra"}\NormalTok{); }\FunctionTok{require}\NormalTok{(terra)\}}
\ControlFlowTok{if}\NormalTok{(}\SpecialCharTok{!}\FunctionTok{require}\NormalTok{(tidyverse)) \{}\FunctionTok{install.packages}\NormalTok{(}\StringTok{"tidyverse"}\NormalTok{); }\FunctionTok{require}\NormalTok{(tidyverse)\}}

\NormalTok{nosaukums}\OtherTok{=}\StringTok{"Edges\_Water{-}Grassland\_r10000.tif"}
\NormalTok{ielasisanas\_cels}\OtherTok{=}\FunctionTok{paste0}\NormalTok{(}\StringTok{"./RasterGrids\_100m/2024/RAW/"}\NormalTok{,nosaukums)}
\NormalTok{saglabasanas\_cels}\OtherTok{=}\FunctionTok{paste0}\NormalTok{(}\StringTok{"./RasterGrids\_100m/2024/Scaled/"}\NormalTok{,nosaukums)}
\NormalTok{slanis}\OtherTok{=}\FunctionTok{rast}\NormalTok{(ielasisanas\_cels)}
\NormalTok{videjais}\OtherTok{=}\FunctionTok{global}\NormalTok{(slanis,}\AttributeTok{fun=}\StringTok{"mean"}\NormalTok{,}\AttributeTok{na.rm=}\ConstantTok{TRUE}\NormalTok{)}
\NormalTok{centrets}\OtherTok{=}\NormalTok{slanis}\SpecialCharTok{{-}}\NormalTok{videjais[,}\DecValTok{1}\NormalTok{]}
\NormalTok{standartnovirze}\OtherTok{=}\NormalTok{terra}\SpecialCharTok{::}\FunctionTok{global}\NormalTok{(centrets,}\AttributeTok{fun=}\StringTok{"rms"}\NormalTok{,}\AttributeTok{na.rm=}\ConstantTok{TRUE}\NormalTok{)}
\NormalTok{merogots}\OtherTok{=}\NormalTok{centrets}\SpecialCharTok{/}\NormalTok{standartnovirze[,}\DecValTok{1}\NormalTok{]}
\FunctionTok{writeRaster}\NormalTok{(merogots,}
      \AttributeTok{filename=}\NormalTok{saglabasanas\_cels,}
      \AttributeTok{overwrite=}\ConstantTok{TRUE}\NormalTok{)}
\end{Highlighting}
\end{Shaded}

\section{Edges\_ReedSedgeRushBeds-Water\_cell}\label{ch06.175}

\textbf{filename:} \texttt{Edges\_ReedSedgeRushBeds-Water\_cell.tif}

\textbf{layername:} \texttt{egv\_175}

\textbf{English name:} Edge pixels of Reed-, Sedge-, Rush- Beds bordering with Water
within the analysis cell (1 ha)

\textbf{Latvian name:} Niedrāju, grīslāju, meldrāju malu ar ūdeni pikseļu skaits analīzes
šūnā (1 ha)

\textbf{Procedure:} First, values equal to 720 from the \hyperref[Ch05.03]{Landscape
classification} are coded as 1, and all other values as NA. Then values
equal to 200 from the \hyperref[Ch05.03]{Landscape classification} are coded as 0, and
all other values as NA. Then, the first layer (1 = presence) is covered over the
second layer (presence = 0) and written to file (matching the input). Next,
with the workflow \texttt{egvtools::landscape\_function()} total edge between the two
classes is calculated. During the calculation of the landscape metric, inverse distance
weighted (power = 2) gap filling on the output is applied to ensure no
missing values at the edges. Finally, the layer is standardised by
subtracting the arithmetic mean and dividing by the root mean squared error.

\begin{Shaded}
\begin{Highlighting}[]
\CommentTok{\# libs {-}{-}{-}{-}}
\ControlFlowTok{if}\NormalTok{(}\SpecialCharTok{!}\FunctionTok{require}\NormalTok{(terra)) \{}\FunctionTok{install.packages}\NormalTok{(}\StringTok{"terra"}\NormalTok{); }\FunctionTok{require}\NormalTok{(terra)\}}
\ControlFlowTok{if}\NormalTok{(}\SpecialCharTok{!}\FunctionTok{require}\NormalTok{(egvtools)) \{remotes}\SpecialCharTok{::}\FunctionTok{install\_github}\NormalTok{(}\StringTok{"aavotins/egvtools"}\NormalTok{); }\FunctionTok{require}\NormalTok{(egvtools)\}}

\ControlFlowTok{if}\NormalTok{(}\SpecialCharTok{!}\FunctionTok{require}\NormalTok{(sf)) \{}\FunctionTok{install.packages}\NormalTok{(}\StringTok{"sf"}\NormalTok{); }\FunctionTok{require}\NormalTok{(sf)\}}
\ControlFlowTok{if}\NormalTok{(}\SpecialCharTok{!}\FunctionTok{require}\NormalTok{(sfarrow)) \{}\FunctionTok{install.packages}\NormalTok{(}\StringTok{"sfarrow"}\NormalTok{); }\FunctionTok{require}\NormalTok{(sfarrow)\}}
\ControlFlowTok{if}\NormalTok{(}\SpecialCharTok{!}\FunctionTok{require}\NormalTok{(raster)) \{}\FunctionTok{install.packages}\NormalTok{(}\StringTok{"raster"}\NormalTok{); }\FunctionTok{require}\NormalTok{(raster)\}}
\ControlFlowTok{if}\NormalTok{(}\SpecialCharTok{!}\FunctionTok{require}\NormalTok{(fasterize)) \{}\FunctionTok{install.packages}\NormalTok{(}\StringTok{"fasterize"}\NormalTok{); }\FunctionTok{require}\NormalTok{(fasterize)\}}
\ControlFlowTok{if}\NormalTok{(}\SpecialCharTok{!}\FunctionTok{require}\NormalTok{(tidyverse)) \{}\FunctionTok{install.packages}\NormalTok{(}\StringTok{"tidyverse"}\NormalTok{); }\FunctionTok{require}\NormalTok{(tidyverse)\}}


\CommentTok{\# Templates {-}{-}{-}{-}{-}}
\NormalTok{template10}\OtherTok{=}\FunctionTok{rast}\NormalTok{(}\StringTok{"./Templates/TemplateRasters/LV10m\_10km.tif"}\NormalTok{)}
\NormalTok{nulls10}\OtherTok{=}\FunctionTok{rast}\NormalTok{(}\StringTok{"./Templates/TemplateRasters/nulls\_LV10m\_10km.tif"}\NormalTok{)}

\CommentTok{\# simple landscape {-}{-}{-}{-}}
\NormalTok{simple\_landscape}\OtherTok{=}\FunctionTok{rast}\NormalTok{(}\StringTok{"./RasterGrids\_10m/2024/Ainava\_vienk\_mask.tif"}\NormalTok{)}

\CommentTok{\# Edges\_ReedSedgeRushBeds{-}Water\_input.tif {-}{-}{-}{-}}
\NormalTok{water}\OtherTok{=}\FunctionTok{ifel}\NormalTok{(simple\_landscape}\SpecialCharTok{==}\DecValTok{200}\NormalTok{,}\DecValTok{0}\NormalTok{,}\ConstantTok{NA}\NormalTok{)}
\FunctionTok{plot}\NormalTok{(water)}

\NormalTok{reedsedgerush}\OtherTok{=}\FunctionTok{ifel}\NormalTok{(simple\_landscape}\SpecialCharTok{==}\DecValTok{720}\NormalTok{,}\DecValTok{1}\NormalTok{,}\ConstantTok{NA}\NormalTok{)}
\FunctionTok{plot}\NormalTok{(reedsedgerush)}


\NormalTok{reedsedgerush\_water}\OtherTok{=}\FunctionTok{cover}\NormalTok{(reedsedgerush,water)}
\FunctionTok{plot}\NormalTok{(reedsedgerush\_water)}

\NormalTok{edge\_reedsedgerush\_water}\OtherTok{=}\FunctionTok{project}\NormalTok{(reedsedgerush\_water,template10,}
               \AttributeTok{filename=}\StringTok{"./RasterGrids\_10m/2024/Edges\_ReedSedgeRushBeds{-}Water\_input.tif"}\NormalTok{,}
               \AttributeTok{overwrite=}\ConstantTok{TRUE}\NormalTok{)}
\FunctionTok{rm}\NormalTok{(edge\_reedsedgerush\_water)}
\FunctionTok{rm}\NormalTok{(reedsedgerush\_water)}


\CommentTok{\# Edges\_ReedSedgeRushBeds{-}Water\_cell.tif    egv\_175 {-}{-}{-}{-}}
\FunctionTok{landscape\_function}\NormalTok{(}
 \AttributeTok{landscape   =} \StringTok{"./RasterGrids\_10m/2024/Edges\_ReedSedgeRushBeds{-}Water\_input.tif"}\NormalTok{,}
 \AttributeTok{zones     =} \StringTok{"./Templates/TemplateGrids/tikls100\_sauzeme.parquet"}\NormalTok{,}
 \AttributeTok{id\_field    =} \StringTok{"id"}\NormalTok{,}
 \AttributeTok{tile\_field   =} \StringTok{"tks50km"}\NormalTok{,}
 \AttributeTok{template    =} \StringTok{"./Templates/TemplateRasters/LV100m\_10km.tif"}\NormalTok{,}
 \AttributeTok{out\_dir    =} \StringTok{"./RasterGrids\_100m/2024/RAW"}\NormalTok{,}
 \AttributeTok{out\_filename  =} \StringTok{"Edges\_ReedSedgeRushBeds{-}Water\_cell.tif"}\NormalTok{,}
 \AttributeTok{out\_layername =} \StringTok{"egv\_175"}\NormalTok{,}
 \AttributeTok{what       =} \StringTok{"lsm\_l\_te"}\NormalTok{,}
 \AttributeTok{lm\_args     =} \FunctionTok{list}\NormalTok{(}\AttributeTok{count\_boundary =} \ConstantTok{FALSE}\NormalTok{),}
 \AttributeTok{rasterize\_engine =} \StringTok{"fasterize"}\NormalTok{,}
 \AttributeTok{n\_workers   =} \DecValTok{12}\NormalTok{,}
 \AttributeTok{future\_max\_size =} \DecValTok{20} \SpecialCharTok{*} \DecValTok{1024}\SpecialCharTok{\^{}}\DecValTok{3}\NormalTok{,}
 \AttributeTok{fill\_gaps   =} \ConstantTok{TRUE}\NormalTok{,}
 \AttributeTok{plot\_gaps   =} \ConstantTok{FALSE}\NormalTok{,}
 \AttributeTok{plot\_result  =} \ConstantTok{FALSE}
\NormalTok{)}

\CommentTok{\# standardisation {-}{-}{-}{-}}
\ControlFlowTok{if}\NormalTok{(}\SpecialCharTok{!}\FunctionTok{require}\NormalTok{(terra)) \{}\FunctionTok{install.packages}\NormalTok{(}\StringTok{"terra"}\NormalTok{); }\FunctionTok{require}\NormalTok{(terra)\}}
\ControlFlowTok{if}\NormalTok{(}\SpecialCharTok{!}\FunctionTok{require}\NormalTok{(tidyverse)) \{}\FunctionTok{install.packages}\NormalTok{(}\StringTok{"tidyverse"}\NormalTok{); }\FunctionTok{require}\NormalTok{(tidyverse)\}}

\NormalTok{nosaukums}\OtherTok{=}\StringTok{"Edges\_ReedSedgeRushBeds{-}Water\_cell.tif"}
\NormalTok{ielasisanas\_cels}\OtherTok{=}\FunctionTok{paste0}\NormalTok{(}\StringTok{"./RasterGrids\_100m/2024/RAW/"}\NormalTok{,nosaukums)}
\NormalTok{saglabasanas\_cels}\OtherTok{=}\FunctionTok{paste0}\NormalTok{(}\StringTok{"./RasterGrids\_100m/2024/Scaled/"}\NormalTok{,nosaukums)}
\NormalTok{slanis}\OtherTok{=}\FunctionTok{rast}\NormalTok{(ielasisanas\_cels)}
\NormalTok{videjais}\OtherTok{=}\FunctionTok{global}\NormalTok{(slanis,}\AttributeTok{fun=}\StringTok{"mean"}\NormalTok{,}\AttributeTok{na.rm=}\ConstantTok{TRUE}\NormalTok{)}
\NormalTok{centrets}\OtherTok{=}\NormalTok{slanis}\SpecialCharTok{{-}}\NormalTok{videjais[,}\DecValTok{1}\NormalTok{]}
\NormalTok{standartnovirze}\OtherTok{=}\NormalTok{terra}\SpecialCharTok{::}\FunctionTok{global}\NormalTok{(centrets,}\AttributeTok{fun=}\StringTok{"rms"}\NormalTok{,}\AttributeTok{na.rm=}\ConstantTok{TRUE}\NormalTok{)}
\NormalTok{merogots}\OtherTok{=}\NormalTok{centrets}\SpecialCharTok{/}\NormalTok{standartnovirze[,}\DecValTok{1}\NormalTok{]}
\FunctionTok{writeRaster}\NormalTok{(merogots,}
      \AttributeTok{filename=}\NormalTok{saglabasanas\_cels,}
      \AttributeTok{overwrite=}\ConstantTok{TRUE}\NormalTok{)}
\end{Highlighting}
\end{Shaded}

\section{Edges\_ReedSedgeRushBeds-Water\_r500}\label{ch06.176}

\textbf{filename:} \texttt{Edges\_ReedSedgeRushBeds-Water\_r500.tif}

\textbf{layername:} \texttt{egv\_176}

\textbf{English name:} Edge pixels of Reed-, Sedge-, Rush- Beds bordering with Water
within the 0.5 km landscape

\textbf{Latvian name:} Niedrāju, grīslāju, meldrāju malu ar ūdeni pikseļu skaits 0,5 km
ainavā

\textbf{Procedure:} The total edge within a 500 m radius around the analysis grid cell is
calculated as the area-weighted sum of the \hyperref[ch06.175]{analysis cells} inside the
buffer, using the workflow \texttt{egvtools::radius\_function()}. During the calculation of the landscape metric,
inverse distance weighted (power = 2) gap filling on the output is applied
to ensure no missing values at the edges. Then the layer is rewritten to set
its name. Finally, the layer is standardised by subtracting the arithmetic
mean and dividing by the root mean squared error.

\begin{Shaded}
\begin{Highlighting}[]
\CommentTok{\# libs {-}{-}{-}{-}}
\ControlFlowTok{if}\NormalTok{(}\SpecialCharTok{!}\FunctionTok{require}\NormalTok{(terra)) \{}\FunctionTok{install.packages}\NormalTok{(}\StringTok{"terra"}\NormalTok{); }\FunctionTok{require}\NormalTok{(terra)\}}
\ControlFlowTok{if}\NormalTok{(}\SpecialCharTok{!}\FunctionTok{require}\NormalTok{(egvtools)) \{remotes}\SpecialCharTok{::}\FunctionTok{install\_github}\NormalTok{(}\StringTok{"aavotins/egvtools"}\NormalTok{); }\FunctionTok{require}\NormalTok{(egvtools)\}}


\CommentTok{\# Templates {-}{-}{-}{-}{-}}
\NormalTok{template100}\OtherTok{=}\FunctionTok{rast}\NormalTok{(}\StringTok{"./Templates/TemplateRasters/LV100m\_10km.tif"}\NormalTok{)}

\CommentTok{\# radii {-}{-}{-}{-}}
\FunctionTok{radius\_function}\NormalTok{(}
 \AttributeTok{kvadrati\_path =} \StringTok{"./Templates/TemplateGrids/tiles/"}\NormalTok{,}
 \AttributeTok{radii\_path   =} \StringTok{"./Templates/TemplateGridPoints/tiles/"}\NormalTok{,}
 \AttributeTok{tikls100\_path =} \StringTok{"./Templates/TemplateGrids/tikls100\_sauzeme.parquet"}\NormalTok{,}
 \AttributeTok{template\_path =} \StringTok{"./Templates/TemplateRasters/LV100m\_10km.tif"}\NormalTok{,}
 \AttributeTok{input\_layers  =} \FunctionTok{c}\NormalTok{(}\StringTok{"./RasterGrids\_100m/2024/RAW/Edges\_ReedSedgeRushBeds{-}Water\_cell.tif"}\NormalTok{),}
 \AttributeTok{layer\_prefixes =} \FunctionTok{c}\NormalTok{(}\StringTok{"Edges\_ReedSedgeRushBeds{-}Water"}\NormalTok{),}
 \AttributeTok{output\_dir   =} \StringTok{"./RasterGrids\_100m/2024/RAW/"}\NormalTok{,}
 \AttributeTok{n\_workers   =} \DecValTok{12}\NormalTok{,}
 \AttributeTok{radii     =} \FunctionTok{c}\NormalTok{(}\StringTok{"r500"}\NormalTok{),}
 \AttributeTok{radius\_mode  =} \StringTok{"sparse"}\NormalTok{,}
 \AttributeTok{extract\_fun  =} \StringTok{"sum"}\NormalTok{,}
 \AttributeTok{fill\_missing  =} \ConstantTok{TRUE}\NormalTok{,}
 \AttributeTok{IDW\_weight   =} \DecValTok{2}\NormalTok{,}
 \AttributeTok{future\_max\_size =} \DecValTok{20} \SpecialCharTok{*} \DecValTok{1024}\SpecialCharTok{\^{}}\DecValTok{3}\NormalTok{)}


\CommentTok{\# Edges\_ReedSedgeRushBeds{-}Water\_r500.tif    egv\_176 {-}{-}{-}{-}}
\NormalTok{slanis}\OtherTok{=}\FunctionTok{rast}\NormalTok{(}\StringTok{"./RasterGrids\_100m/2024/RAW/Edges\_ReedSedgeRushBeds{-}Water\_r500.tif"}\NormalTok{)}
\FunctionTok{names}\NormalTok{(slanis)}\OtherTok{=}\StringTok{"egv\_176"}
\NormalTok{slanis2}\OtherTok{=}\FunctionTok{project}\NormalTok{(slanis,template100)}
\FunctionTok{writeRaster}\NormalTok{(slanis2,}
      \StringTok{"./RasterGrids\_100m/2024/RAW/Edges\_ReedSedgeRushBeds{-}Water\_r500.tif"}\NormalTok{,}
      \AttributeTok{overwrite=}\ConstantTok{TRUE}\NormalTok{)}

\CommentTok{\# standardisation {-}{-}{-}{-}}
\ControlFlowTok{if}\NormalTok{(}\SpecialCharTok{!}\FunctionTok{require}\NormalTok{(terra)) \{}\FunctionTok{install.packages}\NormalTok{(}\StringTok{"terra"}\NormalTok{); }\FunctionTok{require}\NormalTok{(terra)\}}
\ControlFlowTok{if}\NormalTok{(}\SpecialCharTok{!}\FunctionTok{require}\NormalTok{(tidyverse)) \{}\FunctionTok{install.packages}\NormalTok{(}\StringTok{"tidyverse"}\NormalTok{); }\FunctionTok{require}\NormalTok{(tidyverse)\}}

\NormalTok{nosaukums}\OtherTok{=}\StringTok{"Edges\_ReedSedgeRushBeds{-}Water\_r500.tif"}
\NormalTok{ielasisanas\_cels}\OtherTok{=}\FunctionTok{paste0}\NormalTok{(}\StringTok{"./RasterGrids\_100m/2024/RAW/"}\NormalTok{,nosaukums)}
\NormalTok{saglabasanas\_cels}\OtherTok{=}\FunctionTok{paste0}\NormalTok{(}\StringTok{"./RasterGrids\_100m/2024/Scaled/"}\NormalTok{,nosaukums)}
\NormalTok{slanis}\OtherTok{=}\FunctionTok{rast}\NormalTok{(ielasisanas\_cels)}
\NormalTok{videjais}\OtherTok{=}\FunctionTok{global}\NormalTok{(slanis,}\AttributeTok{fun=}\StringTok{"mean"}\NormalTok{,}\AttributeTok{na.rm=}\ConstantTok{TRUE}\NormalTok{)}
\NormalTok{centrets}\OtherTok{=}\NormalTok{slanis}\SpecialCharTok{{-}}\NormalTok{videjais[,}\DecValTok{1}\NormalTok{]}
\NormalTok{standartnovirze}\OtherTok{=}\NormalTok{terra}\SpecialCharTok{::}\FunctionTok{global}\NormalTok{(centrets,}\AttributeTok{fun=}\StringTok{"rms"}\NormalTok{,}\AttributeTok{na.rm=}\ConstantTok{TRUE}\NormalTok{)}
\NormalTok{merogots}\OtherTok{=}\NormalTok{centrets}\SpecialCharTok{/}\NormalTok{standartnovirze[,}\DecValTok{1}\NormalTok{]}
\FunctionTok{writeRaster}\NormalTok{(merogots,}
      \AttributeTok{filename=}\NormalTok{saglabasanas\_cels,}
      \AttributeTok{overwrite=}\ConstantTok{TRUE}\NormalTok{)}
\end{Highlighting}
\end{Shaded}

\section{Edges\_ReedSedgeRushBeds-Water\_r1250}\label{ch06.177}

\textbf{filename:} \texttt{Edges\_ReedSedgeRushBeds-Water\_r1250.tif}

\textbf{layername:} \texttt{egv\_177}

\textbf{English name:} Edge pixels of Reed-, Sedge-, Rush- Beds bordering with Water
within the 1.25 km landscape

\textbf{Latvian name:} Niedrāju, grīslāju, meldrāju malu ar ūdeni pikseļu skaits 1,25 km
ainavā

\textbf{Procedure:} The total edge within a 1250 m radius around the analysis grid cell is
calculated as the area-weighted sum of the \hyperref[ch06.175]{analysis cells} inside the
buffer, using the workflow \texttt{egvtools::radius\_function()}. During the calculation of the landscape metric,
inverse distance weighted (power = 2) gap filling on the output is applied
to ensure no missing values at the edges. Then the layer is rewritten to set
its name. Finally, the layer is standardised by subtracting the arithmetic
mean and dividing by the root mean squared error.

\begin{Shaded}
\begin{Highlighting}[]
\CommentTok{\# libs {-}{-}{-}{-}}
\ControlFlowTok{if}\NormalTok{(}\SpecialCharTok{!}\FunctionTok{require}\NormalTok{(terra)) \{}\FunctionTok{install.packages}\NormalTok{(}\StringTok{"terra"}\NormalTok{); }\FunctionTok{require}\NormalTok{(terra)\}}
\ControlFlowTok{if}\NormalTok{(}\SpecialCharTok{!}\FunctionTok{require}\NormalTok{(egvtools)) \{remotes}\SpecialCharTok{::}\FunctionTok{install\_github}\NormalTok{(}\StringTok{"aavotins/egvtools"}\NormalTok{); }\FunctionTok{require}\NormalTok{(egvtools)\}}


\CommentTok{\# Templates {-}{-}{-}{-}{-}}
\NormalTok{template100}\OtherTok{=}\FunctionTok{rast}\NormalTok{(}\StringTok{"./Templates/TemplateRasters/LV100m\_10km.tif"}\NormalTok{)}

\CommentTok{\# radii {-}{-}{-}{-}}
\FunctionTok{radius\_function}\NormalTok{(}
 \AttributeTok{kvadrati\_path =} \StringTok{"./Templates/TemplateGrids/tiles/"}\NormalTok{,}
 \AttributeTok{radii\_path   =} \StringTok{"./Templates/TemplateGridPoints/tiles/"}\NormalTok{,}
 \AttributeTok{tikls100\_path =} \StringTok{"./Templates/TemplateGrids/tikls100\_sauzeme.parquet"}\NormalTok{,}
 \AttributeTok{template\_path =} \StringTok{"./Templates/TemplateRasters/LV100m\_10km.tif"}\NormalTok{,}
 \AttributeTok{input\_layers  =} \FunctionTok{c}\NormalTok{(}\StringTok{"./RasterGrids\_100m/2024/RAW/Edges\_ReedSedgeRushBeds{-}Water\_cell.tif"}\NormalTok{),}
 \AttributeTok{layer\_prefixes =} \FunctionTok{c}\NormalTok{(}\StringTok{"Edges\_ReedSedgeRushBeds{-}Water"}\NormalTok{),}
 \AttributeTok{output\_dir   =} \StringTok{"./RasterGrids\_100m/2024/RAW/"}\NormalTok{,}
 \AttributeTok{n\_workers   =} \DecValTok{12}\NormalTok{,}
 \AttributeTok{radii     =} \FunctionTok{c}\NormalTok{(}\StringTok{"r1250"}\NormalTok{),}
 \AttributeTok{radius\_mode  =} \StringTok{"sparse"}\NormalTok{,}
 \AttributeTok{extract\_fun  =} \StringTok{"sum"}\NormalTok{,}
 \AttributeTok{fill\_missing  =} \ConstantTok{TRUE}\NormalTok{,}
 \AttributeTok{IDW\_weight   =} \DecValTok{2}\NormalTok{,}
 \AttributeTok{future\_max\_size =} \DecValTok{20} \SpecialCharTok{*} \DecValTok{1024}\SpecialCharTok{\^{}}\DecValTok{3}\NormalTok{)}


\CommentTok{\# Edges\_ReedSedgeRushBeds{-}Water\_r1250.tif   egv\_177 {-}{-}{-}{-}}
\NormalTok{slanis}\OtherTok{=}\FunctionTok{rast}\NormalTok{(}\StringTok{"./RasterGrids\_100m/2024/RAW/Edges\_ReedSedgeRushBeds{-}Water\_r1250.tif"}\NormalTok{)}
\FunctionTok{names}\NormalTok{(slanis)}\OtherTok{=}\StringTok{"egv\_177"}
\NormalTok{slanis2}\OtherTok{=}\FunctionTok{project}\NormalTok{(slanis,template100)}
\FunctionTok{writeRaster}\NormalTok{(slanis2,}
      \StringTok{"./RasterGrids\_100m/2024/RAW/Edges\_ReedSedgeRushBeds{-}Water\_r1250.tif"}\NormalTok{,}
      \AttributeTok{overwrite=}\ConstantTok{TRUE}\NormalTok{)}

\CommentTok{\# standardisation {-}{-}{-}{-}}
\ControlFlowTok{if}\NormalTok{(}\SpecialCharTok{!}\FunctionTok{require}\NormalTok{(terra)) \{}\FunctionTok{install.packages}\NormalTok{(}\StringTok{"terra"}\NormalTok{); }\FunctionTok{require}\NormalTok{(terra)\}}
\ControlFlowTok{if}\NormalTok{(}\SpecialCharTok{!}\FunctionTok{require}\NormalTok{(tidyverse)) \{}\FunctionTok{install.packages}\NormalTok{(}\StringTok{"tidyverse"}\NormalTok{); }\FunctionTok{require}\NormalTok{(tidyverse)\}}

\NormalTok{nosaukums}\OtherTok{=}\StringTok{"Edges\_ReedSedgeRushBeds{-}Water\_r1250.tif"}
\NormalTok{ielasisanas\_cels}\OtherTok{=}\FunctionTok{paste0}\NormalTok{(}\StringTok{"./RasterGrids\_100m/2024/RAW/"}\NormalTok{,nosaukums)}
\NormalTok{saglabasanas\_cels}\OtherTok{=}\FunctionTok{paste0}\NormalTok{(}\StringTok{"./RasterGrids\_100m/2024/Scaled/"}\NormalTok{,nosaukums)}
\NormalTok{slanis}\OtherTok{=}\FunctionTok{rast}\NormalTok{(ielasisanas\_cels)}
\NormalTok{videjais}\OtherTok{=}\FunctionTok{global}\NormalTok{(slanis,}\AttributeTok{fun=}\StringTok{"mean"}\NormalTok{,}\AttributeTok{na.rm=}\ConstantTok{TRUE}\NormalTok{)}
\NormalTok{centrets}\OtherTok{=}\NormalTok{slanis}\SpecialCharTok{{-}}\NormalTok{videjais[,}\DecValTok{1}\NormalTok{]}
\NormalTok{standartnovirze}\OtherTok{=}\NormalTok{terra}\SpecialCharTok{::}\FunctionTok{global}\NormalTok{(centrets,}\AttributeTok{fun=}\StringTok{"rms"}\NormalTok{,}\AttributeTok{na.rm=}\ConstantTok{TRUE}\NormalTok{)}
\NormalTok{merogots}\OtherTok{=}\NormalTok{centrets}\SpecialCharTok{/}\NormalTok{standartnovirze[,}\DecValTok{1}\NormalTok{]}
\FunctionTok{writeRaster}\NormalTok{(merogots,}
      \AttributeTok{filename=}\NormalTok{saglabasanas\_cels,}
      \AttributeTok{overwrite=}\ConstantTok{TRUE}\NormalTok{)}
\end{Highlighting}
\end{Shaded}

\section{Edges\_ReedSedgeRushBeds-Water\_r3000}\label{ch06.178}

\textbf{filename:} \texttt{Edges\_ReedSedgeRushBeds-Water\_r3000.tif}

\textbf{layername:} \texttt{egv\_178}

\textbf{English name:} Edge pixels of Reed-, Sedge-, Rush- Beds bordering with Water
within the 3 km landscape

\textbf{Latvian name:} Niedrāju, grīslāju, meldrāju malu ar ūdeni pikseļu skaits 3 km ainavā

\textbf{Procedure:} The total edge within a 3000 m radius around the analysis grid cell is
calculated as the area-weighted sum of the \hyperref[ch06.175]{analysis cells} inside the
buffer, using the workflow \texttt{egvtools::radius\_function()}. During the calculation of the landscape metric,
inverse distance weighted (power = 2) gap filling on the output is applied
to ensure no missing values at the edges. Then the layer is rewritten to set
its name. Finally, the layer is standardised by subtracting the arithmetic
mean and dividing by the root mean squared error.

\begin{Shaded}
\begin{Highlighting}[]
\CommentTok{\# libs {-}{-}{-}{-}}
\ControlFlowTok{if}\NormalTok{(}\SpecialCharTok{!}\FunctionTok{require}\NormalTok{(terra)) \{}\FunctionTok{install.packages}\NormalTok{(}\StringTok{"terra"}\NormalTok{); }\FunctionTok{require}\NormalTok{(terra)\}}
\ControlFlowTok{if}\NormalTok{(}\SpecialCharTok{!}\FunctionTok{require}\NormalTok{(egvtools)) \{remotes}\SpecialCharTok{::}\FunctionTok{install\_github}\NormalTok{(}\StringTok{"aavotins/egvtools"}\NormalTok{); }\FunctionTok{require}\NormalTok{(egvtools)\}}


\CommentTok{\# Templates {-}{-}{-}{-}{-}}
\NormalTok{template100}\OtherTok{=}\FunctionTok{rast}\NormalTok{(}\StringTok{"./Templates/TemplateRasters/LV100m\_10km.tif"}\NormalTok{)}

\CommentTok{\# radii {-}{-}{-}{-}}
\FunctionTok{radius\_function}\NormalTok{(}
 \AttributeTok{kvadrati\_path =} \StringTok{"./Templates/TemplateGrids/tiles/"}\NormalTok{,}
 \AttributeTok{radii\_path   =} \StringTok{"./Templates/TemplateGridPoints/tiles/"}\NormalTok{,}
 \AttributeTok{tikls100\_path =} \StringTok{"./Templates/TemplateGrids/tikls100\_sauzeme.parquet"}\NormalTok{,}
 \AttributeTok{template\_path =} \StringTok{"./Templates/TemplateRasters/LV100m\_10km.tif"}\NormalTok{,}
 \AttributeTok{input\_layers  =} \FunctionTok{c}\NormalTok{(}\StringTok{"./RasterGrids\_100m/2024/RAW/Edges\_ReedSedgeRushBeds{-}Water\_cell.tif"}\NormalTok{),}
 \AttributeTok{layer\_prefixes =} \FunctionTok{c}\NormalTok{(}\StringTok{"Edges\_ReedSedgeRushBeds{-}Water"}\NormalTok{),}
 \AttributeTok{output\_dir   =} \StringTok{"./RasterGrids\_100m/2024/RAW/"}\NormalTok{,}
 \AttributeTok{n\_workers   =} \DecValTok{12}\NormalTok{,}
 \AttributeTok{radii     =} \FunctionTok{c}\NormalTok{(}\StringTok{"r3000"}\NormalTok{),}
 \AttributeTok{radius\_mode  =} \StringTok{"sparse"}\NormalTok{,}
 \AttributeTok{extract\_fun  =} \StringTok{"sum"}\NormalTok{,}
 \AttributeTok{fill\_missing  =} \ConstantTok{TRUE}\NormalTok{,}
 \AttributeTok{IDW\_weight   =} \DecValTok{2}\NormalTok{,}
 \AttributeTok{future\_max\_size =} \DecValTok{20} \SpecialCharTok{*} \DecValTok{1024}\SpecialCharTok{\^{}}\DecValTok{3}\NormalTok{)}


\CommentTok{\# Edges\_ReedSedgeRushBeds{-}Water\_r3000.tif   egv\_178 {-}{-}{-}{-}}
\NormalTok{slanis}\OtherTok{=}\FunctionTok{rast}\NormalTok{(}\StringTok{"./RasterGrids\_100m/2024/RAW/Edges\_ReedSedgeRushBeds{-}Water\_r3000.tif"}\NormalTok{)}
\FunctionTok{names}\NormalTok{(slanis)}\OtherTok{=}\StringTok{"egv\_178"}
\NormalTok{slanis2}\OtherTok{=}\FunctionTok{project}\NormalTok{(slanis,template100)}
\FunctionTok{writeRaster}\NormalTok{(slanis2,}
      \StringTok{"./RasterGrids\_100m/2024/RAW/Edges\_ReedSedgeRushBeds{-}Water\_r3000.tif"}\NormalTok{,}
      \AttributeTok{overwrite=}\ConstantTok{TRUE}\NormalTok{)}

\CommentTok{\# standardisation {-}{-}{-}{-}}
\ControlFlowTok{if}\NormalTok{(}\SpecialCharTok{!}\FunctionTok{require}\NormalTok{(terra)) \{}\FunctionTok{install.packages}\NormalTok{(}\StringTok{"terra"}\NormalTok{); }\FunctionTok{require}\NormalTok{(terra)\}}
\ControlFlowTok{if}\NormalTok{(}\SpecialCharTok{!}\FunctionTok{require}\NormalTok{(tidyverse)) \{}\FunctionTok{install.packages}\NormalTok{(}\StringTok{"tidyverse"}\NormalTok{); }\FunctionTok{require}\NormalTok{(tidyverse)\}}

\NormalTok{nosaukums}\OtherTok{=}\StringTok{"Edges\_ReedSedgeRushBeds{-}Water\_r3000.tif"}
\NormalTok{ielasisanas\_cels}\OtherTok{=}\FunctionTok{paste0}\NormalTok{(}\StringTok{"./RasterGrids\_100m/2024/RAW/"}\NormalTok{,nosaukums)}
\NormalTok{saglabasanas\_cels}\OtherTok{=}\FunctionTok{paste0}\NormalTok{(}\StringTok{"./RasterGrids\_100m/2024/Scaled/"}\NormalTok{,nosaukums)}
\NormalTok{slanis}\OtherTok{=}\FunctionTok{rast}\NormalTok{(ielasisanas\_cels)}
\NormalTok{videjais}\OtherTok{=}\FunctionTok{global}\NormalTok{(slanis,}\AttributeTok{fun=}\StringTok{"mean"}\NormalTok{,}\AttributeTok{na.rm=}\ConstantTok{TRUE}\NormalTok{)}
\NormalTok{centrets}\OtherTok{=}\NormalTok{slanis}\SpecialCharTok{{-}}\NormalTok{videjais[,}\DecValTok{1}\NormalTok{]}
\NormalTok{standartnovirze}\OtherTok{=}\NormalTok{terra}\SpecialCharTok{::}\FunctionTok{global}\NormalTok{(centrets,}\AttributeTok{fun=}\StringTok{"rms"}\NormalTok{,}\AttributeTok{na.rm=}\ConstantTok{TRUE}\NormalTok{)}
\NormalTok{merogots}\OtherTok{=}\NormalTok{centrets}\SpecialCharTok{/}\NormalTok{standartnovirze[,}\DecValTok{1}\NormalTok{]}
\FunctionTok{writeRaster}\NormalTok{(merogots,}
      \AttributeTok{filename=}\NormalTok{saglabasanas\_cels,}
      \AttributeTok{overwrite=}\ConstantTok{TRUE}\NormalTok{)}
\end{Highlighting}
\end{Shaded}

\section{Edges\_ReedSedgeRushBeds-Water\_r10000}\label{ch06.179}

\textbf{filename:} \texttt{Edges\_ReedSedgeRushBeds-Water\_r10000.tif}

\textbf{layername:} \texttt{egv\_179}

\textbf{English name:} Edge pixels of Reed-, Sedge-, Rush- Beds bordering with Water
within the 10 km landscape

\textbf{Latvian name:} Niedrāju, grīslāju, meldrāju malu ar ūdeni pikseļu skaits 10 km ainavā

\textbf{Procedure:} The total edge within a 10000 m radius around the analysis grid cell is
calculated as the area-weighted sum of the \hyperref[ch06.175]{analysis cells} inside the
buffer, using the workflow \texttt{egvtools::radius\_function()}. During the calculation of the landscape metric,
inverse distance weighted (power = 2) gap filling on the output is applied
to ensure no missing values at the edges. Then the layer is rewritten to set
its name. Finally, the layer is standardised by subtracting the arithmetic
mean and dividing by the root mean squared error.

\begin{Shaded}
\begin{Highlighting}[]
\CommentTok{\# libs {-}{-}{-}{-}}
\ControlFlowTok{if}\NormalTok{(}\SpecialCharTok{!}\FunctionTok{require}\NormalTok{(terra)) \{}\FunctionTok{install.packages}\NormalTok{(}\StringTok{"terra"}\NormalTok{); }\FunctionTok{require}\NormalTok{(terra)\}}
\ControlFlowTok{if}\NormalTok{(}\SpecialCharTok{!}\FunctionTok{require}\NormalTok{(egvtools)) \{remotes}\SpecialCharTok{::}\FunctionTok{install\_github}\NormalTok{(}\StringTok{"aavotins/egvtools"}\NormalTok{); }\FunctionTok{require}\NormalTok{(egvtools)\}}


\CommentTok{\# Templates {-}{-}{-}{-}{-}}
\NormalTok{template100}\OtherTok{=}\FunctionTok{rast}\NormalTok{(}\StringTok{"./Templates/TemplateRasters/LV100m\_10km.tif"}\NormalTok{)}

\CommentTok{\# radii {-}{-}{-}{-}}
\FunctionTok{radius\_function}\NormalTok{(}
 \AttributeTok{kvadrati\_path =} \StringTok{"./Templates/TemplateGrids/tiles/"}\NormalTok{,}
 \AttributeTok{radii\_path   =} \StringTok{"./Templates/TemplateGridPoints/tiles/"}\NormalTok{,}
 \AttributeTok{tikls100\_path =} \StringTok{"./Templates/TemplateGrids/tikls100\_sauzeme.parquet"}\NormalTok{,}
 \AttributeTok{template\_path =} \StringTok{"./Templates/TemplateRasters/LV100m\_10km.tif"}\NormalTok{,}
 \AttributeTok{input\_layers  =} \FunctionTok{c}\NormalTok{(}\StringTok{"./RasterGrids\_100m/2024/RAW/Edges\_ReedSedgeRushBeds{-}Water\_cell.tif"}\NormalTok{),}
 \AttributeTok{layer\_prefixes =} \FunctionTok{c}\NormalTok{(}\StringTok{"Edges\_ReedSedgeRushBeds{-}Water"}\NormalTok{),}
 \AttributeTok{output\_dir   =} \StringTok{"./RasterGrids\_100m/2024/RAW/"}\NormalTok{,}
 \AttributeTok{n\_workers   =} \DecValTok{12}\NormalTok{,}
 \AttributeTok{radii     =} \FunctionTok{c}\NormalTok{(}\StringTok{"r10000"}\NormalTok{),}
 \AttributeTok{radius\_mode  =} \StringTok{"sparse"}\NormalTok{,}
 \AttributeTok{extract\_fun  =} \StringTok{"sum"}\NormalTok{,}
 \AttributeTok{fill\_missing  =} \ConstantTok{TRUE}\NormalTok{,}
 \AttributeTok{IDW\_weight   =} \DecValTok{2}\NormalTok{,}
 \AttributeTok{future\_max\_size =} \DecValTok{20} \SpecialCharTok{*} \DecValTok{1024}\SpecialCharTok{\^{}}\DecValTok{3}\NormalTok{)}


\CommentTok{\# Edges\_ReedSedgeRushBeds{-}Water\_r10000.tif  egv\_179 {-}{-}{-}{-}}
\NormalTok{slanis}\OtherTok{=}\FunctionTok{rast}\NormalTok{(}\StringTok{"./RasterGrids\_100m/2024/RAW/Edges\_ReedSedgeRushBeds{-}Water\_r10000.tif"}\NormalTok{)}
\FunctionTok{names}\NormalTok{(slanis)}\OtherTok{=}\StringTok{"egv\_179"}
\NormalTok{slanis2}\OtherTok{=}\FunctionTok{project}\NormalTok{(slanis,template100)}
\FunctionTok{writeRaster}\NormalTok{(slanis2,}
      \StringTok{"./RasterGrids\_100m/2024/RAW/Edges\_ReedSedgeRushBeds{-}Water\_r10000.tif"}\NormalTok{,}
      \AttributeTok{overwrite=}\ConstantTok{TRUE}\NormalTok{)}

\CommentTok{\# standardisation {-}{-}{-}{-}}
\ControlFlowTok{if}\NormalTok{(}\SpecialCharTok{!}\FunctionTok{require}\NormalTok{(terra)) \{}\FunctionTok{install.packages}\NormalTok{(}\StringTok{"terra"}\NormalTok{); }\FunctionTok{require}\NormalTok{(terra)\}}
\ControlFlowTok{if}\NormalTok{(}\SpecialCharTok{!}\FunctionTok{require}\NormalTok{(tidyverse)) \{}\FunctionTok{install.packages}\NormalTok{(}\StringTok{"tidyverse"}\NormalTok{); }\FunctionTok{require}\NormalTok{(tidyverse)\}}

\NormalTok{nosaukums}\OtherTok{=}\StringTok{"Edges\_ReedSedgeRushBeds{-}Water\_r10000.tif"}
\NormalTok{ielasisanas\_cels}\OtherTok{=}\FunctionTok{paste0}\NormalTok{(}\StringTok{"./RasterGrids\_100m/2024/RAW/"}\NormalTok{,nosaukums)}
\NormalTok{saglabasanas\_cels}\OtherTok{=}\FunctionTok{paste0}\NormalTok{(}\StringTok{"./RasterGrids\_100m/2024/Scaled/"}\NormalTok{,nosaukums)}
\NormalTok{slanis}\OtherTok{=}\FunctionTok{rast}\NormalTok{(ielasisanas\_cels)}
\NormalTok{videjais}\OtherTok{=}\FunctionTok{global}\NormalTok{(slanis,}\AttributeTok{fun=}\StringTok{"mean"}\NormalTok{,}\AttributeTok{na.rm=}\ConstantTok{TRUE}\NormalTok{)}
\NormalTok{centrets}\OtherTok{=}\NormalTok{slanis}\SpecialCharTok{{-}}\NormalTok{videjais[,}\DecValTok{1}\NormalTok{]}
\NormalTok{standartnovirze}\OtherTok{=}\NormalTok{terra}\SpecialCharTok{::}\FunctionTok{global}\NormalTok{(centrets,}\AttributeTok{fun=}\StringTok{"rms"}\NormalTok{,}\AttributeTok{na.rm=}\ConstantTok{TRUE}\NormalTok{)}
\NormalTok{merogots}\OtherTok{=}\NormalTok{centrets}\SpecialCharTok{/}\NormalTok{standartnovirze[,}\DecValTok{1}\NormalTok{]}
\FunctionTok{writeRaster}\NormalTok{(merogots,}
      \AttributeTok{filename=}\NormalTok{saglabasanas\_cels,}
      \AttributeTok{overwrite=}\ConstantTok{TRUE}\NormalTok{)}
\end{Highlighting}
\end{Shaded}

\section{FarmlandCrops\_CropsAll\_cell}\label{ch06.180}

\textbf{filename:} \texttt{FarmlandCrops\_CropsAll\_cell.tif}

\textbf{layername:} \texttt{egv\_180}

\textbf{English name:} Fractional cover of Crops (all types) within the analysis cell
(1 ha)

\textbf{Latvian name:} Aramzemju (dažādu lauksaimniecības kultūraugu) platības
īpatsvars analīzes šūnā (1 ha)

\textbf{Procedure:} First, agricultural parcels with any type of crops are selected
from the \hyperref[Ch04.02]{Rural Support Service's information on declared fields}. These
geometries are then rasterised to input resolution, ensuring value 1 at the
polygon locations and value 0 elsewhere. Rasterisation is performed with the workflow\\
\texttt{egvtools::polygon2input()}. Once rasterised, the layer is aggregated to EGV
resolution using the workflow \texttt{egvtools::input2egv()}, which calculates the arithmetic mean and thus
results in a cover fraction. During aggregation, inverse distance weighted
(power = 2) gap filling on the output is applied to ensure no missing
values at the edges. Finally, the layer is standardised by subtracting
the arithmetic mean and dividing by the root mean squared error.

\begin{Shaded}
\begin{Highlighting}[]
\CommentTok{\# libs {-}{-}{-}{-}}
\ControlFlowTok{if}\NormalTok{(}\SpecialCharTok{!}\FunctionTok{require}\NormalTok{(egvtools)) \{remotes}\SpecialCharTok{::}\FunctionTok{install\_github}\NormalTok{(}\StringTok{"aavotins/egvtools"}\NormalTok{); }\FunctionTok{require}\NormalTok{(egvtools)\}}
\ControlFlowTok{if}\NormalTok{(}\SpecialCharTok{!}\FunctionTok{require}\NormalTok{(terra)) \{}\FunctionTok{install.packages}\NormalTok{(}\StringTok{"terra"}\NormalTok{); }\FunctionTok{require}\NormalTok{(terra)\}}
\ControlFlowTok{if}\NormalTok{(}\SpecialCharTok{!}\FunctionTok{require}\NormalTok{(sf)) \{}\FunctionTok{install.packages}\NormalTok{(}\StringTok{"sf"}\NormalTok{); }\FunctionTok{require}\NormalTok{(sf)\}}
\ControlFlowTok{if}\NormalTok{(}\SpecialCharTok{!}\FunctionTok{require}\NormalTok{(tidyverse)) \{}\FunctionTok{install.packages}\NormalTok{(}\StringTok{"tidyverse"}\NormalTok{); }\FunctionTok{require}\NormalTok{(tidyverse)\}}
\ControlFlowTok{if}\NormalTok{(}\SpecialCharTok{!}\FunctionTok{require}\NormalTok{(sfarrow)) \{}\FunctionTok{install.packages}\NormalTok{(}\StringTok{"sfarrow"}\NormalTok{); }\FunctionTok{require}\NormalTok{(sfarrow)\}}
\ControlFlowTok{if}\NormalTok{(}\SpecialCharTok{!}\FunctionTok{require}\NormalTok{(readxl)) \{}\FunctionTok{install.packages}\NormalTok{(}\StringTok{"readxl"}\NormalTok{); }\FunctionTok{require}\NormalTok{(readxl)\}}
\ControlFlowTok{if}\NormalTok{(}\SpecialCharTok{!}\FunctionTok{require}\NormalTok{(raster)) \{}\FunctionTok{install.packages}\NormalTok{(}\StringTok{"raster"}\NormalTok{); }\FunctionTok{require}\NormalTok{(raster)\}}
\ControlFlowTok{if}\NormalTok{(}\SpecialCharTok{!}\FunctionTok{require}\NormalTok{(fasterize)) \{}\FunctionTok{install.packages}\NormalTok{(}\StringTok{"fasterize"}\NormalTok{); }\FunctionTok{require}\NormalTok{(fasterize)\}}

\CommentTok{\# templates {-}{-}{-}{-}}
\NormalTok{template100}\OtherTok{=}\FunctionTok{rast}\NormalTok{(}\StringTok{"./Templates/TemplateRasters/LV100m\_10km.tif"}\NormalTok{)}
\NormalTok{template10}\OtherTok{=}\FunctionTok{rast}\NormalTok{(}\StringTok{"./Templates/TemplateRasters/LV10m\_10km.tif"}\NormalTok{)}
\NormalTok{rastrs10}\OtherTok{=}\FunctionTok{raster}\NormalTok{(template10)}

\NormalTok{nulls10}\OtherTok{=}\FunctionTok{rast}\NormalTok{(}\StringTok{"./Templates/TemplateRasters/nulls\_LV10m\_10km.tif"}\NormalTok{)}
\NormalTok{nulls100}\OtherTok{=}\FunctionTok{rast}\NormalTok{(}\StringTok{"./Templates/TemplateRasters/nulls\_LV100m\_10km.tif"}\NormalTok{)}

\CommentTok{\# codes {-}{-}{-}{-}}
\NormalTok{kodi}\OtherTok{=}\FunctionTok{read\_excel}\NormalTok{(}\StringTok{"./Geodata/2024/LAD/KulturuKodi\_2024.xlsx"}\NormalTok{)}
\NormalTok{kodi}\SpecialCharTok{$}\NormalTok{kods}\OtherTok{=}\FunctionTok{as.character}\NormalTok{(kodi}\SpecialCharTok{$}\NormalTok{kods)}
\CommentTok{\# LAD {-}{-}{-}{-}}
\NormalTok{lad}\OtherTok{=}\NormalTok{sfarrow}\SpecialCharTok{::}\FunctionTok{st\_read\_parquet}\NormalTok{(}\StringTok{"./Geodata/2024/LAD/Lauki\_2024.parquet"}\NormalTok{)}
\NormalTok{lad}\SpecialCharTok{$}\NormalTok{yes}\OtherTok{=}\DecValTok{1}
\NormalTok{lad}\OtherTok{=}\NormalTok{lad }\SpecialCharTok{\%\textgreater{}\%} 
 \FunctionTok{left\_join}\NormalTok{(kodi,}\AttributeTok{by=}\FunctionTok{c}\NormalTok{(}\StringTok{"PRODUCT\_CODE"}\OtherTok{=}\StringTok{"kods"}\NormalTok{))}

\CommentTok{\# simple landscape {-}{-}{-}{-}}
\NormalTok{simple\_landscape}\OtherTok{=}\FunctionTok{rast}\NormalTok{(}\StringTok{"RasterGrids\_10m/2024/Ainava\_vienk\_mask.tif"}\NormalTok{)}


\CommentTok{\# FarmlandCrops\_CropsAll\_cell.tif   egv\_180 {-}{-}{-}{-}}
\NormalTok{aramzemes}\OtherTok{=}\NormalTok{lad }\SpecialCharTok{\%\textgreater{}\%} 
 \FunctionTok{filter}\NormalTok{(}\FunctionTok{str\_detect}\NormalTok{(SDM\_grupa\_sakums,}\StringTok{"aramz"}\NormalTok{))}

\NormalTok{p2i\_rez}\OtherTok{=}\NormalTok{egvtools}\SpecialCharTok{::}\FunctionTok{polygon2input}\NormalTok{(}\AttributeTok{vector\_data =}\NormalTok{ aramzemes,}
            \AttributeTok{template\_path =} \StringTok{"./Templates/TemplateRasters/LV10m\_10km.tif"}\NormalTok{,}
            \AttributeTok{out\_path =} \StringTok{"./RasterGrids\_10m/2024/"}\NormalTok{,}
            \AttributeTok{file\_name =} \StringTok{"FarmlandCrops\_CropsAll\_input.tif"}\NormalTok{,}
            \AttributeTok{value\_field =} \StringTok{"yes"}\NormalTok{,}
            \AttributeTok{prepare=}\ConstantTok{FALSE}\NormalTok{,}
            \AttributeTok{background\_raster =} \StringTok{"./Templates/TemplateRasters/nulls\_LV10m\_10km.tif"}\NormalTok{,}
            \AttributeTok{plot\_result =} \ConstantTok{TRUE}\NormalTok{)}
\NormalTok{p2i\_rez}
\NormalTok{i2e\_rez}\OtherTok{=}\NormalTok{egvtools}\SpecialCharTok{::}\FunctionTok{input2egv}\NormalTok{(}\AttributeTok{input=}\FunctionTok{paste0}\NormalTok{(}\StringTok{"./RasterGrids\_10m/2024/"}\NormalTok{,}
                     \StringTok{"FarmlandCrops\_CropsAll\_input.tif"}\NormalTok{),}
          \AttributeTok{egv\_template=} \StringTok{"./Templates/TemplateRasters/LV100m\_10km.tif"}\NormalTok{,}
          \AttributeTok{summary\_function =} \StringTok{"average"}\NormalTok{,}
          \AttributeTok{missing\_job =} \StringTok{"FillOutput"}\NormalTok{,}
          \AttributeTok{outlocation =} \StringTok{"./RasterGrids\_100m/2024/RAW/"}\NormalTok{,}
          \AttributeTok{outfilename =} \StringTok{"FarmlandCrops\_CropsAll\_cell.tif"}\NormalTok{,}
          \AttributeTok{layername =} \StringTok{"egv\_180"}\NormalTok{,}
          \AttributeTok{idw\_weight =} \DecValTok{2}\NormalTok{,}
          \AttributeTok{plot\_gaps =} \ConstantTok{FALSE}\NormalTok{,}\AttributeTok{plot\_final =} \ConstantTok{TRUE}\NormalTok{)}
\NormalTok{i2e\_rez}
\FunctionTok{rm}\NormalTok{(p2i\_rez)}
\FunctionTok{rm}\NormalTok{(i2e\_rez)}
\FunctionTok{rm}\NormalTok{(aramzemes)}
\FunctionTok{unlink}\NormalTok{(}\StringTok{"./RasterGrids\_10m/2024/FarmlandCrops\_CropsAll\_input.tif"}\NormalTok{)}

\CommentTok{\# standardisation {-}{-}{-}{-}}
\ControlFlowTok{if}\NormalTok{(}\SpecialCharTok{!}\FunctionTok{require}\NormalTok{(terra)) \{}\FunctionTok{install.packages}\NormalTok{(}\StringTok{"terra"}\NormalTok{); }\FunctionTok{require}\NormalTok{(terra)\}}
\ControlFlowTok{if}\NormalTok{(}\SpecialCharTok{!}\FunctionTok{require}\NormalTok{(tidyverse)) \{}\FunctionTok{install.packages}\NormalTok{(}\StringTok{"tidyverse"}\NormalTok{); }\FunctionTok{require}\NormalTok{(tidyverse)\}}

\NormalTok{nosaukums}\OtherTok{=}\StringTok{"FarmlandCrops\_CropsAll\_cell.tif"}
\NormalTok{ielasisanas\_cels}\OtherTok{=}\FunctionTok{paste0}\NormalTok{(}\StringTok{"./RasterGrids\_100m/2024/RAW/"}\NormalTok{,nosaukums)}
\NormalTok{saglabasanas\_cels}\OtherTok{=}\FunctionTok{paste0}\NormalTok{(}\StringTok{"./RasterGrids\_100m/2024/Scaled/"}\NormalTok{,nosaukums)}
\NormalTok{slanis}\OtherTok{=}\FunctionTok{rast}\NormalTok{(ielasisanas\_cels)}
\NormalTok{videjais}\OtherTok{=}\FunctionTok{global}\NormalTok{(slanis,}\AttributeTok{fun=}\StringTok{"mean"}\NormalTok{,}\AttributeTok{na.rm=}\ConstantTok{TRUE}\NormalTok{)}
\NormalTok{centrets}\OtherTok{=}\NormalTok{slanis}\SpecialCharTok{{-}}\NormalTok{videjais[,}\DecValTok{1}\NormalTok{]}
\NormalTok{standartnovirze}\OtherTok{=}\NormalTok{terra}\SpecialCharTok{::}\FunctionTok{global}\NormalTok{(centrets,}\AttributeTok{fun=}\StringTok{"rms"}\NormalTok{,}\AttributeTok{na.rm=}\ConstantTok{TRUE}\NormalTok{)}
\NormalTok{merogots}\OtherTok{=}\NormalTok{centrets}\SpecialCharTok{/}\NormalTok{standartnovirze[,}\DecValTok{1}\NormalTok{]}
\FunctionTok{writeRaster}\NormalTok{(merogots,}
      \AttributeTok{filename=}\NormalTok{saglabasanas\_cels,}
      \AttributeTok{overwrite=}\ConstantTok{TRUE}\NormalTok{)}
\end{Highlighting}
\end{Shaded}

\section{FarmlandCrops\_CropsAll\_r500}\label{ch06.181}

\textbf{filename:} \texttt{FarmlandCrops\_CropsAll\_r500.tif}

\textbf{layername:} \texttt{egv\_181}

\textbf{English name:} Fractional cover of Crops (all types) within the 0.5 km
landscape

\textbf{Latvian name:} Aramzemju (dažādu lauksaimniecības kultūraugu) platības
īpatsvars 0,5 km ainavā

\textbf{Procedure:} The cover fraction within a radius of 500 m around the analysis grid cell is
calculated as the area-weighted sum of the \hyperref[ch06.180]{analysis cells} inside the
buffer, using the workflow \texttt{egvtools::radius\_function()}. During the calculation of the landscape metric,
inverse distance weighted (power = 2) gap filling on the output is applied
to ensure no missing values at the edges. Then the layer is rewritten to set
its name. Finally, the layer is standardised by subtracting the arithmetic
mean and dividing by the root mean squared error.

\begin{Shaded}
\begin{Highlighting}[]
\CommentTok{\# libs {-}{-}{-}{-}}
\ControlFlowTok{if}\NormalTok{(}\SpecialCharTok{!}\FunctionTok{require}\NormalTok{(terra)) \{}\FunctionTok{install.packages}\NormalTok{(}\StringTok{"terra"}\NormalTok{); }\FunctionTok{require}\NormalTok{(terra)\}}
\ControlFlowTok{if}\NormalTok{(}\SpecialCharTok{!}\FunctionTok{require}\NormalTok{(egvtools)) \{remotes}\SpecialCharTok{::}\FunctionTok{install\_github}\NormalTok{(}\StringTok{"aavotins/egvtools"}\NormalTok{); }\FunctionTok{require}\NormalTok{(egvtools)\}}


\CommentTok{\# Templates {-}{-}{-}{-}{-}}
\NormalTok{template100}\OtherTok{=}\FunctionTok{rast}\NormalTok{(}\StringTok{"./Templates/TemplateRasters/LV100m\_10km.tif"}\NormalTok{)}

\CommentTok{\# radii {-}{-}{-}{-}}
\FunctionTok{radius\_function}\NormalTok{(}
 \AttributeTok{kvadrati\_path =} \StringTok{"./Templates/TemplateGrids/tiles/"}\NormalTok{,}
 \AttributeTok{radii\_path   =} \StringTok{"./Templates/TemplateGridPoints/tiles/"}\NormalTok{,}
 \AttributeTok{tikls100\_path =} \StringTok{"./Templates/TemplateGrids/tikls100\_sauzeme.parquet"}\NormalTok{,}
 \AttributeTok{template\_path =} \StringTok{"./Templates/TemplateRasters/LV100m\_10km.tif"}\NormalTok{,}
 \AttributeTok{input\_layers  =} \FunctionTok{c}\NormalTok{(}\StringTok{"./RasterGrids\_100m/2024/RAW/FarmlandCrops\_CropsAll\_cell.tif"}\NormalTok{),}
 \AttributeTok{layer\_prefixes =} \FunctionTok{c}\NormalTok{(}\StringTok{"FarmlandCrops\_CropsAll"}\NormalTok{),}
 \AttributeTok{output\_dir   =} \StringTok{"./RasterGrids\_100m/2024/RAW/"}\NormalTok{,}
 \AttributeTok{n\_workers   =} \DecValTok{6}\NormalTok{,}
 \AttributeTok{radii     =} \FunctionTok{c}\NormalTok{(}\StringTok{"r500"}\NormalTok{),}
 \AttributeTok{radius\_mode  =} \StringTok{"sparse"}\NormalTok{,}
 \AttributeTok{extract\_fun  =} \StringTok{"mean"}\NormalTok{,}
 \AttributeTok{fill\_missing  =} \ConstantTok{TRUE}\NormalTok{,}
 \AttributeTok{IDW\_weight   =} \DecValTok{2}\NormalTok{,}
 \AttributeTok{future\_max\_size =} \DecValTok{40} \SpecialCharTok{*} \DecValTok{1024}\SpecialCharTok{\^{}}\DecValTok{3}\NormalTok{)}



\CommentTok{\# FarmlandCrops\_CropsAll\_r500.tif   egv\_181 {-}{-}{-}{-}}
\NormalTok{slanis}\OtherTok{=}\FunctionTok{rast}\NormalTok{(}\StringTok{"./RasterGrids\_100m/2024/RAW/FarmlandCrops\_CropsAll\_r500.tif"}\NormalTok{)}
\FunctionTok{names}\NormalTok{(slanis)}\OtherTok{=}\StringTok{"egv\_181"}
\NormalTok{slanis2}\OtherTok{=}\FunctionTok{project}\NormalTok{(slanis,template100)}
\FunctionTok{writeRaster}\NormalTok{(slanis2,}
      \StringTok{"./RasterGrids\_100m/2024/RAW/FarmlandCrops\_CropsAll\_r500.tif"}\NormalTok{,}
      \AttributeTok{overwrite=}\ConstantTok{TRUE}\NormalTok{)}

\CommentTok{\# standardisation {-}{-}{-}{-}}
\ControlFlowTok{if}\NormalTok{(}\SpecialCharTok{!}\FunctionTok{require}\NormalTok{(terra)) \{}\FunctionTok{install.packages}\NormalTok{(}\StringTok{"terra"}\NormalTok{); }\FunctionTok{require}\NormalTok{(terra)\}}
\ControlFlowTok{if}\NormalTok{(}\SpecialCharTok{!}\FunctionTok{require}\NormalTok{(tidyverse)) \{}\FunctionTok{install.packages}\NormalTok{(}\StringTok{"tidyverse"}\NormalTok{); }\FunctionTok{require}\NormalTok{(tidyverse)\}}

\NormalTok{nosaukums}\OtherTok{=}\StringTok{"FarmlandCrops\_CropsAll\_r500.tif"}
\NormalTok{ielasisanas\_cels}\OtherTok{=}\FunctionTok{paste0}\NormalTok{(}\StringTok{"./RasterGrids\_100m/2024/RAW/"}\NormalTok{,nosaukums)}
\NormalTok{saglabasanas\_cels}\OtherTok{=}\FunctionTok{paste0}\NormalTok{(}\StringTok{"./RasterGrids\_100m/2024/Scaled/"}\NormalTok{,nosaukums)}
\NormalTok{slanis}\OtherTok{=}\FunctionTok{rast}\NormalTok{(ielasisanas\_cels)}
\NormalTok{videjais}\OtherTok{=}\FunctionTok{global}\NormalTok{(slanis,}\AttributeTok{fun=}\StringTok{"mean"}\NormalTok{,}\AttributeTok{na.rm=}\ConstantTok{TRUE}\NormalTok{)}
\NormalTok{centrets}\OtherTok{=}\NormalTok{slanis}\SpecialCharTok{{-}}\NormalTok{videjais[,}\DecValTok{1}\NormalTok{]}
\NormalTok{standartnovirze}\OtherTok{=}\NormalTok{terra}\SpecialCharTok{::}\FunctionTok{global}\NormalTok{(centrets,}\AttributeTok{fun=}\StringTok{"rms"}\NormalTok{,}\AttributeTok{na.rm=}\ConstantTok{TRUE}\NormalTok{)}
\NormalTok{merogots}\OtherTok{=}\NormalTok{centrets}\SpecialCharTok{/}\NormalTok{standartnovirze[,}\DecValTok{1}\NormalTok{]}
\FunctionTok{writeRaster}\NormalTok{(merogots,}
      \AttributeTok{filename=}\NormalTok{saglabasanas\_cels,}
      \AttributeTok{overwrite=}\ConstantTok{TRUE}\NormalTok{)}
\end{Highlighting}
\end{Shaded}

\section{FarmlandCrops\_CropsAll\_r1250}\label{ch06.182}

\textbf{filename:} \texttt{FarmlandCrops\_CropsAll\_r1250.tif}

\textbf{layername:} \texttt{egv\_182}

\textbf{English name:} Fractional cover of Crops (all types) within the 1.25 km
landscape

\textbf{Latvian name:} Aramzemju (dažādu lauksaimniecības kultūraugu) platības
īpatsvars 1,25 km ainavā

\textbf{Procedure:} The cover fraction within a radius of 1250 m around the analysis grid cell
is calculated as the area-weighted sum of the \hyperref[ch06.180]{analysis cells} inside
the buffer, using the workflow \texttt{egvtools::radius\_function()}. During the calculation of the landscape
metric, inverse distance weighted (power = 2) gap filling on the output is
applied to ensure no missing values at the edges. Then the layer is
rewritten to set its name. Finally, the layer is standardised by
subtracting the arithmetic mean and dividing by the root mean squared error.

\begin{Shaded}
\begin{Highlighting}[]
\CommentTok{\# libs {-}{-}{-}{-}}
\ControlFlowTok{if}\NormalTok{(}\SpecialCharTok{!}\FunctionTok{require}\NormalTok{(terra)) \{}\FunctionTok{install.packages}\NormalTok{(}\StringTok{"terra"}\NormalTok{); }\FunctionTok{require}\NormalTok{(terra)\}}
\ControlFlowTok{if}\NormalTok{(}\SpecialCharTok{!}\FunctionTok{require}\NormalTok{(egvtools)) \{remotes}\SpecialCharTok{::}\FunctionTok{install\_github}\NormalTok{(}\StringTok{"aavotins/egvtools"}\NormalTok{); }\FunctionTok{require}\NormalTok{(egvtools)\}}


\CommentTok{\# Templates {-}{-}{-}{-}{-}}
\NormalTok{template100}\OtherTok{=}\FunctionTok{rast}\NormalTok{(}\StringTok{"./Templates/TemplateRasters/LV100m\_10km.tif"}\NormalTok{)}

\CommentTok{\# radii {-}{-}{-}{-}}
\FunctionTok{radius\_function}\NormalTok{(}
 \AttributeTok{kvadrati\_path =} \StringTok{"./Templates/TemplateGrids/tiles/"}\NormalTok{,}
 \AttributeTok{radii\_path   =} \StringTok{"./Templates/TemplateGridPoints/tiles/"}\NormalTok{,}
 \AttributeTok{tikls100\_path =} \StringTok{"./Templates/TemplateGrids/tikls100\_sauzeme.parquet"}\NormalTok{,}
 \AttributeTok{template\_path =} \StringTok{"./Templates/TemplateRasters/LV100m\_10km.tif"}\NormalTok{,}
 \AttributeTok{input\_layers  =} \FunctionTok{c}\NormalTok{(}\StringTok{"./RasterGrids\_100m/2024/RAW/FarmlandCrops\_CropsAll\_cell.tif"}\NormalTok{),}
 \AttributeTok{layer\_prefixes =} \FunctionTok{c}\NormalTok{(}\StringTok{"FarmlandCrops\_CropsAll"}\NormalTok{),}
 \AttributeTok{output\_dir   =} \StringTok{"./RasterGrids\_100m/2024/RAW/"}\NormalTok{,}
 \AttributeTok{n\_workers   =} \DecValTok{6}\NormalTok{,}
 \AttributeTok{radii     =} \FunctionTok{c}\NormalTok{(}\StringTok{"r1250"}\NormalTok{),}
 \AttributeTok{radius\_mode  =} \StringTok{"sparse"}\NormalTok{,}
 \AttributeTok{extract\_fun  =} \StringTok{"mean"}\NormalTok{,}
 \AttributeTok{fill\_missing  =} \ConstantTok{TRUE}\NormalTok{,}
 \AttributeTok{IDW\_weight   =} \DecValTok{2}\NormalTok{,}
 \AttributeTok{future\_max\_size =} \DecValTok{40} \SpecialCharTok{*} \DecValTok{1024}\SpecialCharTok{\^{}}\DecValTok{3}\NormalTok{)}



\CommentTok{\# FarmlandCrops\_CropsAll\_r1250.tif  egv\_182 {-}{-}{-}{-}}
\NormalTok{slanis}\OtherTok{=}\FunctionTok{rast}\NormalTok{(}\StringTok{"./RasterGrids\_100m/2024/RAW/FarmlandCrops\_CropsAll\_r1250.tif"}\NormalTok{)}
\FunctionTok{names}\NormalTok{(slanis)}\OtherTok{=}\StringTok{"egv\_182"}
\NormalTok{slanis2}\OtherTok{=}\FunctionTok{project}\NormalTok{(slanis,template100)}
\FunctionTok{writeRaster}\NormalTok{(slanis2,}
      \StringTok{"./RasterGrids\_100m/2024/RAW/FarmlandCrops\_CropsAll\_r1250.tif"}\NormalTok{,}
      \AttributeTok{overwrite=}\ConstantTok{TRUE}\NormalTok{)}

\CommentTok{\# standardisation {-}{-}{-}{-}}
\ControlFlowTok{if}\NormalTok{(}\SpecialCharTok{!}\FunctionTok{require}\NormalTok{(terra)) \{}\FunctionTok{install.packages}\NormalTok{(}\StringTok{"terra"}\NormalTok{); }\FunctionTok{require}\NormalTok{(terra)\}}
\ControlFlowTok{if}\NormalTok{(}\SpecialCharTok{!}\FunctionTok{require}\NormalTok{(tidyverse)) \{}\FunctionTok{install.packages}\NormalTok{(}\StringTok{"tidyverse"}\NormalTok{); }\FunctionTok{require}\NormalTok{(tidyverse)\}}

\NormalTok{nosaukums}\OtherTok{=}\StringTok{"FarmlandCrops\_CropsAll\_r1250.tif"}
\NormalTok{ielasisanas\_cels}\OtherTok{=}\FunctionTok{paste0}\NormalTok{(}\StringTok{"./RasterGrids\_100m/2024/RAW/"}\NormalTok{,nosaukums)}
\NormalTok{saglabasanas\_cels}\OtherTok{=}\FunctionTok{paste0}\NormalTok{(}\StringTok{"./RasterGrids\_100m/2024/Scaled/"}\NormalTok{,nosaukums)}
\NormalTok{slanis}\OtherTok{=}\FunctionTok{rast}\NormalTok{(ielasisanas\_cels)}
\NormalTok{videjais}\OtherTok{=}\FunctionTok{global}\NormalTok{(slanis,}\AttributeTok{fun=}\StringTok{"mean"}\NormalTok{,}\AttributeTok{na.rm=}\ConstantTok{TRUE}\NormalTok{)}
\NormalTok{centrets}\OtherTok{=}\NormalTok{slanis}\SpecialCharTok{{-}}\NormalTok{videjais[,}\DecValTok{1}\NormalTok{]}
\NormalTok{standartnovirze}\OtherTok{=}\NormalTok{terra}\SpecialCharTok{::}\FunctionTok{global}\NormalTok{(centrets,}\AttributeTok{fun=}\StringTok{"rms"}\NormalTok{,}\AttributeTok{na.rm=}\ConstantTok{TRUE}\NormalTok{)}
\NormalTok{merogots}\OtherTok{=}\NormalTok{centrets}\SpecialCharTok{/}\NormalTok{standartnovirze[,}\DecValTok{1}\NormalTok{]}
\FunctionTok{writeRaster}\NormalTok{(merogots,}
      \AttributeTok{filename=}\NormalTok{saglabasanas\_cels,}
      \AttributeTok{overwrite=}\ConstantTok{TRUE}\NormalTok{)}
\end{Highlighting}
\end{Shaded}

\section{FarmlandCrops\_CropsAll\_r3000}\label{ch06.183}

\textbf{filename:} \texttt{FarmlandCrops\_CropsAll\_r3000.tif}

\textbf{layername:} \texttt{egv\_183}

\textbf{English name:} Fractional cover of Crops (all types) within the 3 km
landscape

\textbf{Latvian name:} Aramzemju (dažādu lauksaimniecības kultūraugu) platības
īpatsvars 3 km ainavā

\textbf{Procedure:} The cover fraction within a radius of 3000 m around the analysis grid cell
is calculated as the area-weighted sum of the \hyperref[ch06.180]{analysis cells} inside
the buffer, using the workflow \texttt{egvtools::radius\_function()}. During the calculation of the landscape
metric, inverse distance weighted (power = 2) gap filling on the output is
applied to ensure no missing values at the edges. Then the layer is
rewritten to set its name. Finally, the layer is standardised by
subtracting the arithmetic mean and dividing by the root mean squared error.

\begin{Shaded}
\begin{Highlighting}[]
\CommentTok{\# libs {-}{-}{-}{-}}
\ControlFlowTok{if}\NormalTok{(}\SpecialCharTok{!}\FunctionTok{require}\NormalTok{(terra)) \{}\FunctionTok{install.packages}\NormalTok{(}\StringTok{"terra"}\NormalTok{); }\FunctionTok{require}\NormalTok{(terra)\}}
\ControlFlowTok{if}\NormalTok{(}\SpecialCharTok{!}\FunctionTok{require}\NormalTok{(egvtools)) \{remotes}\SpecialCharTok{::}\FunctionTok{install\_github}\NormalTok{(}\StringTok{"aavotins/egvtools"}\NormalTok{); }\FunctionTok{require}\NormalTok{(egvtools)\}}


\CommentTok{\# Templates {-}{-}{-}{-}{-}}
\NormalTok{template100}\OtherTok{=}\FunctionTok{rast}\NormalTok{(}\StringTok{"./Templates/TemplateRasters/LV100m\_10km.tif"}\NormalTok{)}

\CommentTok{\# radii {-}{-}{-}{-}}
\FunctionTok{radius\_function}\NormalTok{(}
 \AttributeTok{kvadrati\_path =} \StringTok{"./Templates/TemplateGrids/tiles/"}\NormalTok{,}
 \AttributeTok{radii\_path   =} \StringTok{"./Templates/TemplateGridPoints/tiles/"}\NormalTok{,}
 \AttributeTok{tikls100\_path =} \StringTok{"./Templates/TemplateGrids/tikls100\_sauzeme.parquet"}\NormalTok{,}
 \AttributeTok{template\_path =} \StringTok{"./Templates/TemplateRasters/LV100m\_10km.tif"}\NormalTok{,}
 \AttributeTok{input\_layers  =} \FunctionTok{c}\NormalTok{(}\StringTok{"./RasterGrids\_100m/2024/RAW/FarmlandCrops\_CropsAll\_cell.tif"}\NormalTok{),}
 \AttributeTok{layer\_prefixes =} \FunctionTok{c}\NormalTok{(}\StringTok{"FarmlandCrops\_CropsAll"}\NormalTok{),}
 \AttributeTok{output\_dir   =} \StringTok{"./RasterGrids\_100m/2024/RAW/"}\NormalTok{,}
 \AttributeTok{n\_workers   =} \DecValTok{6}\NormalTok{,}
 \AttributeTok{radii     =} \FunctionTok{c}\NormalTok{(}\StringTok{"r3000"}\NormalTok{),}
 \AttributeTok{radius\_mode  =} \StringTok{"sparse"}\NormalTok{,}
 \AttributeTok{extract\_fun  =} \StringTok{"mean"}\NormalTok{,}
 \AttributeTok{fill\_missing  =} \ConstantTok{TRUE}\NormalTok{,}
 \AttributeTok{IDW\_weight   =} \DecValTok{2}\NormalTok{,}
 \AttributeTok{future\_max\_size =} \DecValTok{40} \SpecialCharTok{*} \DecValTok{1024}\SpecialCharTok{\^{}}\DecValTok{3}\NormalTok{)}



\CommentTok{\# FarmlandCrops\_CropsAll\_r3000.tif  egv\_183 {-}{-}{-}{-}}
\NormalTok{slanis}\OtherTok{=}\FunctionTok{rast}\NormalTok{(}\StringTok{"./RasterGrids\_100m/2024/RAW/FarmlandCrops\_CropsAll\_r3000.tif"}\NormalTok{)}
\FunctionTok{names}\NormalTok{(slanis)}\OtherTok{=}\StringTok{"egv\_183"}
\NormalTok{slanis2}\OtherTok{=}\FunctionTok{project}\NormalTok{(slanis,template100)}
\FunctionTok{writeRaster}\NormalTok{(slanis2,}
      \StringTok{"./RasterGrids\_100m/2024/RAW/FarmlandCrops\_CropsAll\_r3000.tif"}\NormalTok{,}
      \AttributeTok{overwrite=}\ConstantTok{TRUE}\NormalTok{)}

\CommentTok{\# standardisation {-}{-}{-}{-}}
\ControlFlowTok{if}\NormalTok{(}\SpecialCharTok{!}\FunctionTok{require}\NormalTok{(terra)) \{}\FunctionTok{install.packages}\NormalTok{(}\StringTok{"terra"}\NormalTok{); }\FunctionTok{require}\NormalTok{(terra)\}}
\ControlFlowTok{if}\NormalTok{(}\SpecialCharTok{!}\FunctionTok{require}\NormalTok{(tidyverse)) \{}\FunctionTok{install.packages}\NormalTok{(}\StringTok{"tidyverse"}\NormalTok{); }\FunctionTok{require}\NormalTok{(tidyverse)\}}

\NormalTok{nosaukums}\OtherTok{=}\StringTok{"FarmlandCrops\_CropsAll\_r3000.tif"}
\NormalTok{ielasisanas\_cels}\OtherTok{=}\FunctionTok{paste0}\NormalTok{(}\StringTok{"./RasterGrids\_100m/2024/RAW/"}\NormalTok{,nosaukums)}
\NormalTok{saglabasanas\_cels}\OtherTok{=}\FunctionTok{paste0}\NormalTok{(}\StringTok{"./RasterGrids\_100m/2024/Scaled/"}\NormalTok{,nosaukums)}
\NormalTok{slanis}\OtherTok{=}\FunctionTok{rast}\NormalTok{(ielasisanas\_cels)}
\NormalTok{videjais}\OtherTok{=}\FunctionTok{global}\NormalTok{(slanis,}\AttributeTok{fun=}\StringTok{"mean"}\NormalTok{,}\AttributeTok{na.rm=}\ConstantTok{TRUE}\NormalTok{)}
\NormalTok{centrets}\OtherTok{=}\NormalTok{slanis}\SpecialCharTok{{-}}\NormalTok{videjais[,}\DecValTok{1}\NormalTok{]}
\NormalTok{standartnovirze}\OtherTok{=}\NormalTok{terra}\SpecialCharTok{::}\FunctionTok{global}\NormalTok{(centrets,}\AttributeTok{fun=}\StringTok{"rms"}\NormalTok{,}\AttributeTok{na.rm=}\ConstantTok{TRUE}\NormalTok{)}
\NormalTok{merogots}\OtherTok{=}\NormalTok{centrets}\SpecialCharTok{/}\NormalTok{standartnovirze[,}\DecValTok{1}\NormalTok{]}
\FunctionTok{writeRaster}\NormalTok{(merogots,}
      \AttributeTok{filename=}\NormalTok{saglabasanas\_cels,}
      \AttributeTok{overwrite=}\ConstantTok{TRUE}\NormalTok{)}
\end{Highlighting}
\end{Shaded}

\section{FarmlandCrops\_CropsAll\_r10000}\label{ch06.184}

\textbf{filename:} \texttt{FarmlandCrops\_CropsAll\_r10000.tif}

\textbf{layername:} \texttt{egv\_184}

\textbf{English name:} Fractional cover of Crops (all types) within the 10 km
landscape

\textbf{Latvian name:} Aramzemju (dažādu lauksaimniecības kultūraugu) platības
īpatsvars 10 km ainavā

\textbf{Procedure:} The cover fraction within a radius of 10000 m around the analysis grid cell
is calculated as the area-weighted sum of the \hyperref[ch06.180]{analysis cells} inside
the buffer, using the workflow \texttt{egvtools::radius\_function()}. During the calculation of the landscape
metric, inverse distance weighted (power = 2) gap filling on the output is
applied to ensure no missing values at the edges. Then the layer is
rewritten to set its name. Finally, the layer is standardised by
subtracting the arithmetic mean and dividing by the root mean squared error.

\begin{Shaded}
\begin{Highlighting}[]
\CommentTok{\# libs {-}{-}{-}{-}}
\ControlFlowTok{if}\NormalTok{(}\SpecialCharTok{!}\FunctionTok{require}\NormalTok{(terra)) \{}\FunctionTok{install.packages}\NormalTok{(}\StringTok{"terra"}\NormalTok{); }\FunctionTok{require}\NormalTok{(terra)\}}
\ControlFlowTok{if}\NormalTok{(}\SpecialCharTok{!}\FunctionTok{require}\NormalTok{(egvtools)) \{remotes}\SpecialCharTok{::}\FunctionTok{install\_github}\NormalTok{(}\StringTok{"aavotins/egvtools"}\NormalTok{); }\FunctionTok{require}\NormalTok{(egvtools)\}}


\CommentTok{\# Templates {-}{-}{-}{-}{-}}
\NormalTok{template100}\OtherTok{=}\FunctionTok{rast}\NormalTok{(}\StringTok{"./Templates/TemplateRasters/LV100m\_10km.tif"}\NormalTok{)}

\CommentTok{\# radii {-}{-}{-}{-}}
\FunctionTok{radius\_function}\NormalTok{(}
 \AttributeTok{kvadrati\_path =} \StringTok{"./Templates/TemplateGrids/tiles/"}\NormalTok{,}
 \AttributeTok{radii\_path   =} \StringTok{"./Templates/TemplateGridPoints/tiles/"}\NormalTok{,}
 \AttributeTok{tikls100\_path =} \StringTok{"./Templates/TemplateGrids/tikls100\_sauzeme.parquet"}\NormalTok{,}
 \AttributeTok{template\_path =} \StringTok{"./Templates/TemplateRasters/LV100m\_10km.tif"}\NormalTok{,}
 \AttributeTok{input\_layers  =} \FunctionTok{c}\NormalTok{(}\StringTok{"./RasterGrids\_100m/2024/RAW/FarmlandCrops\_CropsAll\_cell.tif"}\NormalTok{),}
 \AttributeTok{layer\_prefixes =} \FunctionTok{c}\NormalTok{(}\StringTok{"FarmlandCrops\_CropsAll"}\NormalTok{),}
 \AttributeTok{output\_dir   =} \StringTok{"./RasterGrids\_100m/2024/RAW/"}\NormalTok{,}
 \AttributeTok{n\_workers   =} \DecValTok{6}\NormalTok{,}
 \AttributeTok{radii     =} \FunctionTok{c}\NormalTok{(}\StringTok{"r10000"}\NormalTok{),}
 \AttributeTok{radius\_mode  =} \StringTok{"sparse"}\NormalTok{,}
 \AttributeTok{extract\_fun  =} \StringTok{"mean"}\NormalTok{,}
 \AttributeTok{fill\_missing  =} \ConstantTok{TRUE}\NormalTok{,}
 \AttributeTok{IDW\_weight   =} \DecValTok{2}\NormalTok{,}
 \AttributeTok{future\_max\_size =} \DecValTok{40} \SpecialCharTok{*} \DecValTok{1024}\SpecialCharTok{\^{}}\DecValTok{3}\NormalTok{)}



\CommentTok{\# FarmlandCrops\_CropsAll\_r10000.tif egv\_184 {-}{-}{-}{-}}
\NormalTok{slanis}\OtherTok{=}\FunctionTok{rast}\NormalTok{(}\StringTok{"./RasterGrids\_100m/2024/RAW/FarmlandCrops\_CropsAll\_r10000.tif"}\NormalTok{)}
\FunctionTok{names}\NormalTok{(slanis)}\OtherTok{=}\StringTok{"egv\_184"}
\NormalTok{slanis2}\OtherTok{=}\FunctionTok{project}\NormalTok{(slanis,template100)}
\FunctionTok{writeRaster}\NormalTok{(slanis2,}
      \StringTok{"./RasterGrids\_100m/2024/RAW/FarmlandCrops\_CropsAll\_r10000.tif"}\NormalTok{,}
      \AttributeTok{overwrite=}\ConstantTok{TRUE}\NormalTok{)}

\CommentTok{\# standardisation {-}{-}{-}{-}}
\ControlFlowTok{if}\NormalTok{(}\SpecialCharTok{!}\FunctionTok{require}\NormalTok{(terra)) \{}\FunctionTok{install.packages}\NormalTok{(}\StringTok{"terra"}\NormalTok{); }\FunctionTok{require}\NormalTok{(terra)\}}
\ControlFlowTok{if}\NormalTok{(}\SpecialCharTok{!}\FunctionTok{require}\NormalTok{(tidyverse)) \{}\FunctionTok{install.packages}\NormalTok{(}\StringTok{"tidyverse"}\NormalTok{); }\FunctionTok{require}\NormalTok{(tidyverse)\}}

\NormalTok{nosaukums}\OtherTok{=}\StringTok{"FarmlandCrops\_CropsAll\_r10000.tif"}
\NormalTok{ielasisanas\_cels}\OtherTok{=}\FunctionTok{paste0}\NormalTok{(}\StringTok{"./RasterGrids\_100m/2024/RAW/"}\NormalTok{,nosaukums)}
\NormalTok{saglabasanas\_cels}\OtherTok{=}\FunctionTok{paste0}\NormalTok{(}\StringTok{"./RasterGrids\_100m/2024/Scaled/"}\NormalTok{,nosaukums)}
\NormalTok{slanis}\OtherTok{=}\FunctionTok{rast}\NormalTok{(ielasisanas\_cels)}
\NormalTok{videjais}\OtherTok{=}\FunctionTok{global}\NormalTok{(slanis,}\AttributeTok{fun=}\StringTok{"mean"}\NormalTok{,}\AttributeTok{na.rm=}\ConstantTok{TRUE}\NormalTok{)}
\NormalTok{centrets}\OtherTok{=}\NormalTok{slanis}\SpecialCharTok{{-}}\NormalTok{videjais[,}\DecValTok{1}\NormalTok{]}
\NormalTok{standartnovirze}\OtherTok{=}\NormalTok{terra}\SpecialCharTok{::}\FunctionTok{global}\NormalTok{(centrets,}\AttributeTok{fun=}\StringTok{"rms"}\NormalTok{,}\AttributeTok{na.rm=}\ConstantTok{TRUE}\NormalTok{)}
\NormalTok{merogots}\OtherTok{=}\NormalTok{centrets}\SpecialCharTok{/}\NormalTok{standartnovirze[,}\DecValTok{1}\NormalTok{]}
\FunctionTok{writeRaster}\NormalTok{(merogots,}
      \AttributeTok{filename=}\NormalTok{saglabasanas\_cels,}
      \AttributeTok{overwrite=}\ConstantTok{TRUE}\NormalTok{)}
\end{Highlighting}
\end{Shaded}

\section{FarmlandCrops\_CropsHoed\_cell}\label{ch06.185}

\textbf{filename:} \texttt{FarmlandCrops\_CropsHoed\_cell.tif}

\textbf{layername:} \texttt{egv\_185}

\textbf{English name:} Fractional cover of Hoed Crops within the analysis cell (1 ha)

\textbf{Latvian name:} Vagu un rušināmkultūru platības īpatsvars analīzes šūnā (1 ha)

\textbf{Procedure:} First, agricultural parcels declared as hoed crops are selected from the
\hyperref[Ch04.02]{Rural Support Service's information on declared fields}. These
geometries are then rasterised to input resolution, ensuring value 1 at the
polygon locations and value 0 elsewhere. Rasterisation is performed using the workflow
\texttt{egvtools::polygon2input()}. Once rasterised, the layer is aggregated to EGV
resolution using the workflow \texttt{egvtools::input2egv()}, which calculates the arithmetic mean and thus
results in a cover fraction. During aggregation, inverse distance weighted
(power = 2) gap filling on the output is applied to ensure no missing
values at the edges. Finally, the layer is standardised by subtracting
the arithmetic mean and dividing by the root mean squared error.

\begin{Shaded}
\begin{Highlighting}[]
\CommentTok{\# libs {-}{-}{-}{-}}
\ControlFlowTok{if}\NormalTok{(}\SpecialCharTok{!}\FunctionTok{require}\NormalTok{(egvtools)) \{remotes}\SpecialCharTok{::}\FunctionTok{install\_github}\NormalTok{(}\StringTok{"aavotins/egvtools"}\NormalTok{); }\FunctionTok{require}\NormalTok{(egvtools)\}}
\ControlFlowTok{if}\NormalTok{(}\SpecialCharTok{!}\FunctionTok{require}\NormalTok{(terra)) \{}\FunctionTok{install.packages}\NormalTok{(}\StringTok{"terra"}\NormalTok{); }\FunctionTok{require}\NormalTok{(terra)\}}
\ControlFlowTok{if}\NormalTok{(}\SpecialCharTok{!}\FunctionTok{require}\NormalTok{(sf)) \{}\FunctionTok{install.packages}\NormalTok{(}\StringTok{"sf"}\NormalTok{); }\FunctionTok{require}\NormalTok{(sf)\}}
\ControlFlowTok{if}\NormalTok{(}\SpecialCharTok{!}\FunctionTok{require}\NormalTok{(tidyverse)) \{}\FunctionTok{install.packages}\NormalTok{(}\StringTok{"tidyverse"}\NormalTok{); }\FunctionTok{require}\NormalTok{(tidyverse)\}}
\ControlFlowTok{if}\NormalTok{(}\SpecialCharTok{!}\FunctionTok{require}\NormalTok{(sfarrow)) \{}\FunctionTok{install.packages}\NormalTok{(}\StringTok{"sfarrow"}\NormalTok{); }\FunctionTok{require}\NormalTok{(sfarrow)\}}
\ControlFlowTok{if}\NormalTok{(}\SpecialCharTok{!}\FunctionTok{require}\NormalTok{(readxl)) \{}\FunctionTok{install.packages}\NormalTok{(}\StringTok{"readxl"}\NormalTok{); }\FunctionTok{require}\NormalTok{(readxl)\}}
\ControlFlowTok{if}\NormalTok{(}\SpecialCharTok{!}\FunctionTok{require}\NormalTok{(raster)) \{}\FunctionTok{install.packages}\NormalTok{(}\StringTok{"raster"}\NormalTok{); }\FunctionTok{require}\NormalTok{(raster)\}}
\ControlFlowTok{if}\NormalTok{(}\SpecialCharTok{!}\FunctionTok{require}\NormalTok{(fasterize)) \{}\FunctionTok{install.packages}\NormalTok{(}\StringTok{"fasterize"}\NormalTok{); }\FunctionTok{require}\NormalTok{(fasterize)\}}

\CommentTok{\# templates {-}{-}{-}{-}}
\NormalTok{template100}\OtherTok{=}\FunctionTok{rast}\NormalTok{(}\StringTok{"./Templates/TemplateRasters/LV100m\_10km.tif"}\NormalTok{)}
\NormalTok{template10}\OtherTok{=}\FunctionTok{rast}\NormalTok{(}\StringTok{"./Templates/TemplateRasters/LV10m\_10km.tif"}\NormalTok{)}
\NormalTok{rastrs10}\OtherTok{=}\FunctionTok{raster}\NormalTok{(template10)}

\NormalTok{nulls10}\OtherTok{=}\FunctionTok{rast}\NormalTok{(}\StringTok{"./Templates/TemplateRasters/nulls\_LV10m\_10km.tif"}\NormalTok{)}
\NormalTok{nulls100}\OtherTok{=}\FunctionTok{rast}\NormalTok{(}\StringTok{"./Templates/TemplateRasters/nulls\_LV100m\_10km.tif"}\NormalTok{)}

\CommentTok{\# codes {-}{-}{-}{-}}
\NormalTok{kodi}\OtherTok{=}\FunctionTok{read\_excel}\NormalTok{(}\StringTok{"./Geodata/2024/LAD/KulturuKodi\_2024.xlsx"}\NormalTok{)}
\NormalTok{kodi}\SpecialCharTok{$}\NormalTok{kods}\OtherTok{=}\FunctionTok{as.character}\NormalTok{(kodi}\SpecialCharTok{$}\NormalTok{kods)}
\CommentTok{\# LAD {-}{-}{-}{-}}
\NormalTok{lad}\OtherTok{=}\NormalTok{sfarrow}\SpecialCharTok{::}\FunctionTok{st\_read\_parquet}\NormalTok{(}\StringTok{"./Geodata/2024/LAD/Lauki\_2024.parquet"}\NormalTok{)}
\NormalTok{lad}\SpecialCharTok{$}\NormalTok{yes}\OtherTok{=}\DecValTok{1}
\NormalTok{lad}\OtherTok{=}\NormalTok{lad }\SpecialCharTok{\%\textgreater{}\%} 
 \FunctionTok{left\_join}\NormalTok{(kodi,}\AttributeTok{by=}\FunctionTok{c}\NormalTok{(}\StringTok{"PRODUCT\_CODE"}\OtherTok{=}\StringTok{"kods"}\NormalTok{))}

\CommentTok{\# simple landscape {-}{-}{-}{-}}
\NormalTok{simple\_landscape}\OtherTok{=}\FunctionTok{rast}\NormalTok{(}\StringTok{"RasterGrids\_10m/2024/Ainava\_vienk\_mask.tif"}\NormalTok{)}


\CommentTok{\# FarmlandCrops\_CropsHoed\_cell.tif  egv\_185 {-}{-}{-}{-}}
\NormalTok{dati}\OtherTok{=}\NormalTok{lad }\SpecialCharTok{\%\textgreater{}\%} 
 \FunctionTok{filter}\NormalTok{(}\FunctionTok{str\_detect}\NormalTok{(SDM\_grupa\_sakums,}\StringTok{"ruši"}\NormalTok{))}
\FunctionTok{table}\NormalTok{(dati}\SpecialCharTok{$}\NormalTok{SDM\_grupa\_sakums,}\AttributeTok{useNA=}\StringTok{"always"}\NormalTok{)}


\NormalTok{p2i\_rez}\OtherTok{=}\NormalTok{egvtools}\SpecialCharTok{::}\FunctionTok{polygon2input}\NormalTok{(}\AttributeTok{vector\_data =}\NormalTok{ dati,}
                \AttributeTok{template\_path =} \StringTok{"./Templates/TemplateRasters/LV10m\_10km.tif"}\NormalTok{,}
                \AttributeTok{out\_path =} \StringTok{"./RasterGrids\_10m/2024/"}\NormalTok{,}
                \AttributeTok{file\_name =} \StringTok{"FarmlandCrops\_CropsHoed\_input.tif"}\NormalTok{,}
                \AttributeTok{value\_field =} \StringTok{"yes"}\NormalTok{,}
                \AttributeTok{prepare=}\ConstantTok{FALSE}\NormalTok{,}
                \AttributeTok{background\_raster =} \StringTok{"./Templates/TemplateRasters/nulls\_LV10m\_10km.tif"}\NormalTok{,}
                \AttributeTok{plot\_result =} \ConstantTok{TRUE}\NormalTok{)}
\NormalTok{p2i\_rez}
\NormalTok{i2e\_rez}\OtherTok{=}\NormalTok{egvtools}\SpecialCharTok{::}\FunctionTok{input2egv}\NormalTok{(}\AttributeTok{input=}\FunctionTok{paste0}\NormalTok{(}\StringTok{"./RasterGrids\_10m/2024/"}\NormalTok{,}
                     \StringTok{"FarmlandCrops\_CropsHoed\_input.tif"}\NormalTok{),}
              \AttributeTok{egv\_template=} \StringTok{"./Templates/TemplateRasters/LV100m\_10km.tif"}\NormalTok{,}
              \AttributeTok{summary\_function =} \StringTok{"average"}\NormalTok{,}
              \AttributeTok{missing\_job =} \StringTok{"FillOutput"}\NormalTok{,}
              \AttributeTok{outlocation =} \StringTok{"./RasterGrids\_100m/2024/RAW/"}\NormalTok{,}
              \AttributeTok{outfilename =} \StringTok{"FarmlandCrops\_CropsHoed\_cell.tif"}\NormalTok{,}
              \AttributeTok{layername =} \StringTok{"egv\_185"}\NormalTok{,}
              \AttributeTok{idw\_weight =} \DecValTok{2}\NormalTok{,}
              \AttributeTok{plot\_gaps =} \ConstantTok{FALSE}\NormalTok{,}\AttributeTok{plot\_final =} \ConstantTok{TRUE}\NormalTok{)}
\NormalTok{i2e\_rez}
\FunctionTok{rm}\NormalTok{(p2i\_rez)}
\FunctionTok{rm}\NormalTok{(i2e\_rez)}
\FunctionTok{rm}\NormalTok{(dati)}
\FunctionTok{unlink}\NormalTok{(}\StringTok{"./RasterGrids\_10m/2024/FarmlandCrops\_CropsHoed\_input.tif"}\NormalTok{)}

\CommentTok{\# standardisation {-}{-}{-}{-}}
\ControlFlowTok{if}\NormalTok{(}\SpecialCharTok{!}\FunctionTok{require}\NormalTok{(terra)) \{}\FunctionTok{install.packages}\NormalTok{(}\StringTok{"terra"}\NormalTok{); }\FunctionTok{require}\NormalTok{(terra)\}}
\ControlFlowTok{if}\NormalTok{(}\SpecialCharTok{!}\FunctionTok{require}\NormalTok{(tidyverse)) \{}\FunctionTok{install.packages}\NormalTok{(}\StringTok{"tidyverse"}\NormalTok{); }\FunctionTok{require}\NormalTok{(tidyverse)\}}

\NormalTok{nosaukums}\OtherTok{=}\StringTok{"FarmlandCrops\_CropsHoed\_cell.tif"}
\NormalTok{ielasisanas\_cels}\OtherTok{=}\FunctionTok{paste0}\NormalTok{(}\StringTok{"./RasterGrids\_100m/2024/RAW/"}\NormalTok{,nosaukums)}
\NormalTok{saglabasanas\_cels}\OtherTok{=}\FunctionTok{paste0}\NormalTok{(}\StringTok{"./RasterGrids\_100m/2024/Scaled/"}\NormalTok{,nosaukums)}
\NormalTok{slanis}\OtherTok{=}\FunctionTok{rast}\NormalTok{(ielasisanas\_cels)}
\NormalTok{videjais}\OtherTok{=}\FunctionTok{global}\NormalTok{(slanis,}\AttributeTok{fun=}\StringTok{"mean"}\NormalTok{,}\AttributeTok{na.rm=}\ConstantTok{TRUE}\NormalTok{)}
\NormalTok{centrets}\OtherTok{=}\NormalTok{slanis}\SpecialCharTok{{-}}\NormalTok{videjais[,}\DecValTok{1}\NormalTok{]}
\NormalTok{standartnovirze}\OtherTok{=}\NormalTok{terra}\SpecialCharTok{::}\FunctionTok{global}\NormalTok{(centrets,}\AttributeTok{fun=}\StringTok{"rms"}\NormalTok{,}\AttributeTok{na.rm=}\ConstantTok{TRUE}\NormalTok{)}
\NormalTok{merogots}\OtherTok{=}\NormalTok{centrets}\SpecialCharTok{/}\NormalTok{standartnovirze[,}\DecValTok{1}\NormalTok{]}
\FunctionTok{writeRaster}\NormalTok{(merogots,}
      \AttributeTok{filename=}\NormalTok{saglabasanas\_cels,}
      \AttributeTok{overwrite=}\ConstantTok{TRUE}\NormalTok{)}
\end{Highlighting}
\end{Shaded}

\section{FarmlandCrops\_CropsHoed\_r500}\label{ch06.186}

\textbf{filename:} \texttt{FarmlandCrops\_CropsHoed\_r500.tif}

\textbf{layername:} \texttt{egv\_186}

\textbf{English name:} Fractional cover of Hoed Crops within the 0.5 km landscape

\textbf{Latvian name:} Vagu un rušināmkultūru platības īpatsvars 0,5 km ainavā

\textbf{Procedure:} The cover fraction within a radius of 500 m around the analysis grid cell is
calculated as the area-weighted sum of the \hyperref[ch06.185]{analysis cells} inside the
buffer, using the workflow \texttt{egvtools::radius\_function()}. During the calculation of the landscape metric,
inverse distance weighted (power = 2) gap filling on the output is applied
to ensure no missing values at the edges. Then the layer is rewritten to set
its name. Finally, the layer is standardised by subtracting the arithmetic
mean and dividing by the root mean squared error.

\begin{Shaded}
\begin{Highlighting}[]
\CommentTok{\# libs {-}{-}{-}{-}}
\ControlFlowTok{if}\NormalTok{(}\SpecialCharTok{!}\FunctionTok{require}\NormalTok{(terra)) \{}\FunctionTok{install.packages}\NormalTok{(}\StringTok{"terra"}\NormalTok{); }\FunctionTok{require}\NormalTok{(terra)\}}
\ControlFlowTok{if}\NormalTok{(}\SpecialCharTok{!}\FunctionTok{require}\NormalTok{(egvtools)) \{remotes}\SpecialCharTok{::}\FunctionTok{install\_github}\NormalTok{(}\StringTok{"aavotins/egvtools"}\NormalTok{); }\FunctionTok{require}\NormalTok{(egvtools)\}}


\CommentTok{\# Templates {-}{-}{-}{-}{-}}
\NormalTok{template100}\OtherTok{=}\FunctionTok{rast}\NormalTok{(}\StringTok{"./Templates/TemplateRasters/LV100m\_10km.tif"}\NormalTok{)}

\CommentTok{\# radii {-}{-}{-}{-}}
\FunctionTok{radius\_function}\NormalTok{(}
 \AttributeTok{kvadrati\_path =} \StringTok{"./Templates/TemplateGrids/tiles/"}\NormalTok{,}
 \AttributeTok{radii\_path   =} \StringTok{"./Templates/TemplateGridPoints/tiles/"}\NormalTok{,}
 \AttributeTok{tikls100\_path =} \StringTok{"./Templates/TemplateGrids/tikls100\_sauzeme.parquet"}\NormalTok{,}
 \AttributeTok{template\_path =} \StringTok{"./Templates/TemplateRasters/LV100m\_10km.tif"}\NormalTok{,}
 \AttributeTok{input\_layers  =} \FunctionTok{c}\NormalTok{(}\StringTok{"./RasterGrids\_100m/2024/RAW/FarmlandCrops\_CropsHoed\_cell.tif"}\NormalTok{),}
 \AttributeTok{layer\_prefixes =} \FunctionTok{c}\NormalTok{(}\StringTok{"FarmlandCrops\_CropsHoed"}\NormalTok{),}
 \AttributeTok{output\_dir   =} \StringTok{"./RasterGrids\_100m/2024/RAW/"}\NormalTok{,}
 \AttributeTok{n\_workers   =} \DecValTok{6}\NormalTok{,}
 \AttributeTok{radii     =} \FunctionTok{c}\NormalTok{(}\StringTok{"r500"}\NormalTok{),}
 \AttributeTok{radius\_mode  =} \StringTok{"sparse"}\NormalTok{,}
 \AttributeTok{extract\_fun  =} \StringTok{"mean"}\NormalTok{,}
 \AttributeTok{fill\_missing  =} \ConstantTok{TRUE}\NormalTok{,}
 \AttributeTok{IDW\_weight   =} \DecValTok{2}\NormalTok{,}
 \AttributeTok{future\_max\_size =} \DecValTok{40} \SpecialCharTok{*} \DecValTok{1024}\SpecialCharTok{\^{}}\DecValTok{3}\NormalTok{)}


\CommentTok{\# FarmlandCrops\_CropsHoed\_r500.tif  egv\_186 {-}{-}{-}{-}}
\NormalTok{slanis}\OtherTok{=}\FunctionTok{rast}\NormalTok{(}\StringTok{"./RasterGrids\_100m/2024/RAW/FarmlandCrops\_CropsHoed\_r500.tif"}\NormalTok{)}
\FunctionTok{names}\NormalTok{(slanis)}\OtherTok{=}\StringTok{"egv\_186"}
\NormalTok{slanis2}\OtherTok{=}\FunctionTok{project}\NormalTok{(slanis,template100)}
\FunctionTok{writeRaster}\NormalTok{(slanis2,}
      \StringTok{"./RasterGrids\_100m/2024/RAW/FarmlandCrops\_CropsHoed\_r500.tif"}\NormalTok{,}
      \AttributeTok{overwrite=}\ConstantTok{TRUE}\NormalTok{)}

\CommentTok{\# standardisation {-}{-}{-}{-}}
\ControlFlowTok{if}\NormalTok{(}\SpecialCharTok{!}\FunctionTok{require}\NormalTok{(terra)) \{}\FunctionTok{install.packages}\NormalTok{(}\StringTok{"terra"}\NormalTok{); }\FunctionTok{require}\NormalTok{(terra)\}}
\ControlFlowTok{if}\NormalTok{(}\SpecialCharTok{!}\FunctionTok{require}\NormalTok{(tidyverse)) \{}\FunctionTok{install.packages}\NormalTok{(}\StringTok{"tidyverse"}\NormalTok{); }\FunctionTok{require}\NormalTok{(tidyverse)\}}

\NormalTok{nosaukums}\OtherTok{=}\StringTok{"FarmlandCrops\_CropsHoed\_r500.tif"}
\NormalTok{ielasisanas\_cels}\OtherTok{=}\FunctionTok{paste0}\NormalTok{(}\StringTok{"./RasterGrids\_100m/2024/RAW/"}\NormalTok{,nosaukums)}
\NormalTok{saglabasanas\_cels}\OtherTok{=}\FunctionTok{paste0}\NormalTok{(}\StringTok{"./RasterGrids\_100m/2024/Scaled/"}\NormalTok{,nosaukums)}
\NormalTok{slanis}\OtherTok{=}\FunctionTok{rast}\NormalTok{(ielasisanas\_cels)}
\NormalTok{videjais}\OtherTok{=}\FunctionTok{global}\NormalTok{(slanis,}\AttributeTok{fun=}\StringTok{"mean"}\NormalTok{,}\AttributeTok{na.rm=}\ConstantTok{TRUE}\NormalTok{)}
\NormalTok{centrets}\OtherTok{=}\NormalTok{slanis}\SpecialCharTok{{-}}\NormalTok{videjais[,}\DecValTok{1}\NormalTok{]}
\NormalTok{standartnovirze}\OtherTok{=}\NormalTok{terra}\SpecialCharTok{::}\FunctionTok{global}\NormalTok{(centrets,}\AttributeTok{fun=}\StringTok{"rms"}\NormalTok{,}\AttributeTok{na.rm=}\ConstantTok{TRUE}\NormalTok{)}
\NormalTok{merogots}\OtherTok{=}\NormalTok{centrets}\SpecialCharTok{/}\NormalTok{standartnovirze[,}\DecValTok{1}\NormalTok{]}
\FunctionTok{writeRaster}\NormalTok{(merogots,}
      \AttributeTok{filename=}\NormalTok{saglabasanas\_cels,}
      \AttributeTok{overwrite=}\ConstantTok{TRUE}\NormalTok{)}
\end{Highlighting}
\end{Shaded}

\section{FarmlandCrops\_CropsHoed\_r1250}\label{ch06.187}

\textbf{filename:} \texttt{FarmlandCrops\_CropsHoed\_r1250.tif}

\textbf{layername:} \texttt{egv\_187}

\textbf{English name:} Fractional cover of Hoed Crops within the 1.25 km landscape

\textbf{Latvian name:} Vagu un rušināmkultūru platības īpatsvars 1,25 km ainavā

\textbf{Procedure:} The cover fraction within a radius of 1250 m around the analysis grid cell
is calculated as the area-weighted sum of the \hyperref[ch06.185]{analysis cells} inside
the buffer, using the workflow \texttt{egvtools::radius\_function()}. During the calculation of the landscape
metric, inverse distance weighted (power = 2) gap filling on the output is
applied to ensure no missing values at the edges. Then the layer is
rewritten to set its name. Finally, the layer is standardised by
subtracting the arithmetic mean and dividing by the root mean squared error.

\begin{Shaded}
\begin{Highlighting}[]
\CommentTok{\# libs {-}{-}{-}{-}}
\ControlFlowTok{if}\NormalTok{(}\SpecialCharTok{!}\FunctionTok{require}\NormalTok{(terra)) \{}\FunctionTok{install.packages}\NormalTok{(}\StringTok{"terra"}\NormalTok{); }\FunctionTok{require}\NormalTok{(terra)\}}
\ControlFlowTok{if}\NormalTok{(}\SpecialCharTok{!}\FunctionTok{require}\NormalTok{(egvtools)) \{remotes}\SpecialCharTok{::}\FunctionTok{install\_github}\NormalTok{(}\StringTok{"aavotins/egvtools"}\NormalTok{); }\FunctionTok{require}\NormalTok{(egvtools)\}}


\CommentTok{\# Templates {-}{-}{-}{-}{-}}
\NormalTok{template100}\OtherTok{=}\FunctionTok{rast}\NormalTok{(}\StringTok{"./Templates/TemplateRasters/LV100m\_10km.tif"}\NormalTok{)}

\CommentTok{\# radii {-}{-}{-}{-}}
\FunctionTok{radius\_function}\NormalTok{(}
 \AttributeTok{kvadrati\_path =} \StringTok{"./Templates/TemplateGrids/tiles/"}\NormalTok{,}
 \AttributeTok{radii\_path   =} \StringTok{"./Templates/TemplateGridPoints/tiles/"}\NormalTok{,}
 \AttributeTok{tikls100\_path =} \StringTok{"./Templates/TemplateGrids/tikls100\_sauzeme.parquet"}\NormalTok{,}
 \AttributeTok{template\_path =} \StringTok{"./Templates/TemplateRasters/LV100m\_10km.tif"}\NormalTok{,}
 \AttributeTok{input\_layers  =} \FunctionTok{c}\NormalTok{(}\StringTok{"./RasterGrids\_100m/2024/RAW/FarmlandCrops\_CropsHoed\_cell.tif"}\NormalTok{),}
 \AttributeTok{layer\_prefixes =} \FunctionTok{c}\NormalTok{(}\StringTok{"FarmlandCrops\_CropsHoed"}\NormalTok{),}
 \AttributeTok{output\_dir   =} \StringTok{"./RasterGrids\_100m/2024/RAW/"}\NormalTok{,}
 \AttributeTok{n\_workers   =} \DecValTok{6}\NormalTok{,}
 \AttributeTok{radii     =} \FunctionTok{c}\NormalTok{(}\StringTok{"r1250"}\NormalTok{),}
 \AttributeTok{radius\_mode  =} \StringTok{"sparse"}\NormalTok{,}
 \AttributeTok{extract\_fun  =} \StringTok{"mean"}\NormalTok{,}
 \AttributeTok{fill\_missing  =} \ConstantTok{TRUE}\NormalTok{,}
 \AttributeTok{IDW\_weight   =} \DecValTok{2}\NormalTok{,}
 \AttributeTok{future\_max\_size =} \DecValTok{40} \SpecialCharTok{*} \DecValTok{1024}\SpecialCharTok{\^{}}\DecValTok{3}\NormalTok{)}


\CommentTok{\# FarmlandCrops\_CropsHoed\_r1250.tif egv\_187 {-}{-}{-}{-}}
\NormalTok{slanis}\OtherTok{=}\FunctionTok{rast}\NormalTok{(}\StringTok{"./RasterGrids\_100m/2024/RAW/FarmlandCrops\_CropsHoed\_r1250.tif"}\NormalTok{)}
\FunctionTok{names}\NormalTok{(slanis)}\OtherTok{=}\StringTok{"egv\_187"}
\NormalTok{slanis2}\OtherTok{=}\FunctionTok{project}\NormalTok{(slanis,template100)}
\FunctionTok{writeRaster}\NormalTok{(slanis2,}
      \StringTok{"./RasterGrids\_100m/2024/RAW/FarmlandCrops\_CropsHoed\_r1250.tif"}\NormalTok{,}
      \AttributeTok{overwrite=}\ConstantTok{TRUE}\NormalTok{)}

\CommentTok{\# standardisation {-}{-}{-}{-}}
\ControlFlowTok{if}\NormalTok{(}\SpecialCharTok{!}\FunctionTok{require}\NormalTok{(terra)) \{}\FunctionTok{install.packages}\NormalTok{(}\StringTok{"terra"}\NormalTok{); }\FunctionTok{require}\NormalTok{(terra)\}}
\ControlFlowTok{if}\NormalTok{(}\SpecialCharTok{!}\FunctionTok{require}\NormalTok{(tidyverse)) \{}\FunctionTok{install.packages}\NormalTok{(}\StringTok{"tidyverse"}\NormalTok{); }\FunctionTok{require}\NormalTok{(tidyverse)\}}

\NormalTok{nosaukums}\OtherTok{=}\StringTok{"FarmlandCrops\_CropsHoed\_r1250.tif"}
\NormalTok{ielasisanas\_cels}\OtherTok{=}\FunctionTok{paste0}\NormalTok{(}\StringTok{"./RasterGrids\_100m/2024/RAW/"}\NormalTok{,nosaukums)}
\NormalTok{saglabasanas\_cels}\OtherTok{=}\FunctionTok{paste0}\NormalTok{(}\StringTok{"./RasterGrids\_100m/2024/Scaled/"}\NormalTok{,nosaukums)}
\NormalTok{slanis}\OtherTok{=}\FunctionTok{rast}\NormalTok{(ielasisanas\_cels)}
\NormalTok{videjais}\OtherTok{=}\FunctionTok{global}\NormalTok{(slanis,}\AttributeTok{fun=}\StringTok{"mean"}\NormalTok{,}\AttributeTok{na.rm=}\ConstantTok{TRUE}\NormalTok{)}
\NormalTok{centrets}\OtherTok{=}\NormalTok{slanis}\SpecialCharTok{{-}}\NormalTok{videjais[,}\DecValTok{1}\NormalTok{]}
\NormalTok{standartnovirze}\OtherTok{=}\NormalTok{terra}\SpecialCharTok{::}\FunctionTok{global}\NormalTok{(centrets,}\AttributeTok{fun=}\StringTok{"rms"}\NormalTok{,}\AttributeTok{na.rm=}\ConstantTok{TRUE}\NormalTok{)}
\NormalTok{merogots}\OtherTok{=}\NormalTok{centrets}\SpecialCharTok{/}\NormalTok{standartnovirze[,}\DecValTok{1}\NormalTok{]}
\FunctionTok{writeRaster}\NormalTok{(merogots,}
      \AttributeTok{filename=}\NormalTok{saglabasanas\_cels,}
      \AttributeTok{overwrite=}\ConstantTok{TRUE}\NormalTok{)}
\end{Highlighting}
\end{Shaded}

\section{FarmlandCrops\_CropsHoed\_r3000}\label{ch06.188}

\textbf{filename:} \texttt{FarmlandCrops\_CropsHoed\_r3000.tif}

\textbf{layername:} \texttt{egv\_188}

\textbf{English name:} Fractional cover of Hoed Crops within the 3 km landscape

\textbf{Latvian name:} Vagu un rušināmkultūru platības īpatsvars 3 km ainavā

\textbf{Procedure:} The cover fraction within a radius of 3000 m around the analysis grid cell
is calculated as the area-weighted sum of the \hyperref[ch06.185]{analysis cells} inside
the buffer, using the workflow \texttt{egvtools::radius\_function()}. During the calculation of the landscape
metric, inverse distance weighted (power = 2) gap filling on the output is
applied to ensure no missing values at the edges. Then the layer is
rewritten to set its name. Finally, the layer is standardised by
subtracting the arithmetic mean and dividing by the root mean squared error.

\begin{Shaded}
\begin{Highlighting}[]
\CommentTok{\# libs {-}{-}{-}{-}}
\ControlFlowTok{if}\NormalTok{(}\SpecialCharTok{!}\FunctionTok{require}\NormalTok{(terra)) \{}\FunctionTok{install.packages}\NormalTok{(}\StringTok{"terra"}\NormalTok{); }\FunctionTok{require}\NormalTok{(terra)\}}
\ControlFlowTok{if}\NormalTok{(}\SpecialCharTok{!}\FunctionTok{require}\NormalTok{(egvtools)) \{remotes}\SpecialCharTok{::}\FunctionTok{install\_github}\NormalTok{(}\StringTok{"aavotins/egvtools"}\NormalTok{); }\FunctionTok{require}\NormalTok{(egvtools)\}}


\CommentTok{\# Templates {-}{-}{-}{-}{-}}
\NormalTok{template100}\OtherTok{=}\FunctionTok{rast}\NormalTok{(}\StringTok{"./Templates/TemplateRasters/LV100m\_10km.tif"}\NormalTok{)}

\CommentTok{\# radii {-}{-}{-}{-}}
\FunctionTok{radius\_function}\NormalTok{(}
 \AttributeTok{kvadrati\_path =} \StringTok{"./Templates/TemplateGrids/tiles/"}\NormalTok{,}
 \AttributeTok{radii\_path   =} \StringTok{"./Templates/TemplateGridPoints/tiles/"}\NormalTok{,}
 \AttributeTok{tikls100\_path =} \StringTok{"./Templates/TemplateGrids/tikls100\_sauzeme.parquet"}\NormalTok{,}
 \AttributeTok{template\_path =} \StringTok{"./Templates/TemplateRasters/LV100m\_10km.tif"}\NormalTok{,}
 \AttributeTok{input\_layers  =} \FunctionTok{c}\NormalTok{(}\StringTok{"./RasterGrids\_100m/2024/RAW/FarmlandCrops\_CropsHoed\_cell.tif"}\NormalTok{),}
 \AttributeTok{layer\_prefixes =} \FunctionTok{c}\NormalTok{(}\StringTok{"FarmlandCrops\_CropsHoed"}\NormalTok{),}
 \AttributeTok{output\_dir   =} \StringTok{"./RasterGrids\_100m/2024/RAW/"}\NormalTok{,}
 \AttributeTok{n\_workers   =} \DecValTok{6}\NormalTok{,}
 \AttributeTok{radii     =} \FunctionTok{c}\NormalTok{(}\StringTok{"r3000"}\NormalTok{),}
 \AttributeTok{radius\_mode  =} \StringTok{"sparse"}\NormalTok{,}
 \AttributeTok{extract\_fun  =} \StringTok{"mean"}\NormalTok{,}
 \AttributeTok{fill\_missing  =} \ConstantTok{TRUE}\NormalTok{,}
 \AttributeTok{IDW\_weight   =} \DecValTok{2}\NormalTok{,}
 \AttributeTok{future\_max\_size =} \DecValTok{40} \SpecialCharTok{*} \DecValTok{1024}\SpecialCharTok{\^{}}\DecValTok{3}\NormalTok{)}


\CommentTok{\# FarmlandCrops\_CropsHoed\_r3000.tif egv\_188 {-}{-}{-}{-}}
\NormalTok{slanis}\OtherTok{=}\FunctionTok{rast}\NormalTok{(}\StringTok{"./RasterGrids\_100m/2024/RAW/FarmlandCrops\_CropsHoed\_r3000.tif"}\NormalTok{)}
\FunctionTok{names}\NormalTok{(slanis)}\OtherTok{=}\StringTok{"egv\_188"}
\NormalTok{slanis2}\OtherTok{=}\FunctionTok{project}\NormalTok{(slanis,template100)}
\FunctionTok{writeRaster}\NormalTok{(slanis2,}
      \StringTok{"./RasterGrids\_100m/2024/RAW/FarmlandCrops\_CropsHoed\_r3000.tif"}\NormalTok{,}
      \AttributeTok{overwrite=}\ConstantTok{TRUE}\NormalTok{)}

\CommentTok{\# standardisation {-}{-}{-}{-}}
\ControlFlowTok{if}\NormalTok{(}\SpecialCharTok{!}\FunctionTok{require}\NormalTok{(terra)) \{}\FunctionTok{install.packages}\NormalTok{(}\StringTok{"terra"}\NormalTok{); }\FunctionTok{require}\NormalTok{(terra)\}}
\ControlFlowTok{if}\NormalTok{(}\SpecialCharTok{!}\FunctionTok{require}\NormalTok{(tidyverse)) \{}\FunctionTok{install.packages}\NormalTok{(}\StringTok{"tidyverse"}\NormalTok{); }\FunctionTok{require}\NormalTok{(tidyverse)\}}

\NormalTok{nosaukums}\OtherTok{=}\StringTok{"FarmlandCrops\_CropsHoed\_r3000.tif"}
\NormalTok{ielasisanas\_cels}\OtherTok{=}\FunctionTok{paste0}\NormalTok{(}\StringTok{"./RasterGrids\_100m/2024/RAW/"}\NormalTok{,nosaukums)}
\NormalTok{saglabasanas\_cels}\OtherTok{=}\FunctionTok{paste0}\NormalTok{(}\StringTok{"./RasterGrids\_100m/2024/Scaled/"}\NormalTok{,nosaukums)}
\NormalTok{slanis}\OtherTok{=}\FunctionTok{rast}\NormalTok{(ielasisanas\_cels)}
\NormalTok{videjais}\OtherTok{=}\FunctionTok{global}\NormalTok{(slanis,}\AttributeTok{fun=}\StringTok{"mean"}\NormalTok{,}\AttributeTok{na.rm=}\ConstantTok{TRUE}\NormalTok{)}
\NormalTok{centrets}\OtherTok{=}\NormalTok{slanis}\SpecialCharTok{{-}}\NormalTok{videjais[,}\DecValTok{1}\NormalTok{]}
\NormalTok{standartnovirze}\OtherTok{=}\NormalTok{terra}\SpecialCharTok{::}\FunctionTok{global}\NormalTok{(centrets,}\AttributeTok{fun=}\StringTok{"rms"}\NormalTok{,}\AttributeTok{na.rm=}\ConstantTok{TRUE}\NormalTok{)}
\NormalTok{merogots}\OtherTok{=}\NormalTok{centrets}\SpecialCharTok{/}\NormalTok{standartnovirze[,}\DecValTok{1}\NormalTok{]}
\FunctionTok{writeRaster}\NormalTok{(merogots,}
      \AttributeTok{filename=}\NormalTok{saglabasanas\_cels,}
      \AttributeTok{overwrite=}\ConstantTok{TRUE}\NormalTok{)}
\end{Highlighting}
\end{Shaded}

\section{FarmlandCrops\_CropsHoed\_r10000}\label{ch06.189}

\textbf{filename:} \texttt{FarmlandCrops\_CropsHoed\_r10000.tif}

\textbf{layername:} \texttt{egv\_189}

\textbf{English name:} Fractional cover of Hoed Crops within the 10 km landscape

\textbf{Latvian name:} Vagu un rušināmkultūru platības īpatsvars 10 km ainavā

\textbf{Procedure:} The cover fraction within a radius of 10000 m around the analysis grid cell
is calculated as the area-weighted sum of the \hyperref[ch06.185]{analysis cells} inside
the buffer, using the workflow \texttt{egvtools::radius\_function()}. During the calculation of the landscape
metric, inverse distance weighted (power = 2) gap filling on the output is
applied to ensure no missing values at the edges. Then the layer is
rewritten to set its name. Finally, the layer is standardised by
subtracting the arithmetic mean and dividing by the root mean squared error.

\begin{Shaded}
\begin{Highlighting}[]
\CommentTok{\# libs {-}{-}{-}{-}}
\ControlFlowTok{if}\NormalTok{(}\SpecialCharTok{!}\FunctionTok{require}\NormalTok{(terra)) \{}\FunctionTok{install.packages}\NormalTok{(}\StringTok{"terra"}\NormalTok{); }\FunctionTok{require}\NormalTok{(terra)\}}
\ControlFlowTok{if}\NormalTok{(}\SpecialCharTok{!}\FunctionTok{require}\NormalTok{(egvtools)) \{remotes}\SpecialCharTok{::}\FunctionTok{install\_github}\NormalTok{(}\StringTok{"aavotins/egvtools"}\NormalTok{); }\FunctionTok{require}\NormalTok{(egvtools)\}}


\CommentTok{\# Templates {-}{-}{-}{-}{-}}
\NormalTok{template100}\OtherTok{=}\FunctionTok{rast}\NormalTok{(}\StringTok{"./Templates/TemplateRasters/LV100m\_10km.tif"}\NormalTok{)}

\CommentTok{\# radii {-}{-}{-}{-}}
\FunctionTok{radius\_function}\NormalTok{(}
 \AttributeTok{kvadrati\_path =} \StringTok{"./Templates/TemplateGrids/tiles/"}\NormalTok{,}
 \AttributeTok{radii\_path   =} \StringTok{"./Templates/TemplateGridPoints/tiles/"}\NormalTok{,}
 \AttributeTok{tikls100\_path =} \StringTok{"./Templates/TemplateGrids/tikls100\_sauzeme.parquet"}\NormalTok{,}
 \AttributeTok{template\_path =} \StringTok{"./Templates/TemplateRasters/LV100m\_10km.tif"}\NormalTok{,}
 \AttributeTok{input\_layers  =} \FunctionTok{c}\NormalTok{(}\StringTok{"./RasterGrids\_100m/2024/RAW/FarmlandCrops\_CropsHoed\_cell.tif"}\NormalTok{),}
 \AttributeTok{layer\_prefixes =} \FunctionTok{c}\NormalTok{(}\StringTok{"FarmlandCrops\_CropsHoed"}\NormalTok{),}
 \AttributeTok{output\_dir   =} \StringTok{"./RasterGrids\_100m/2024/RAW/"}\NormalTok{,}
 \AttributeTok{n\_workers   =} \DecValTok{6}\NormalTok{,}
 \AttributeTok{radii     =} \FunctionTok{c}\NormalTok{(}\StringTok{"r10000"}\NormalTok{),}
 \AttributeTok{radius\_mode  =} \StringTok{"sparse"}\NormalTok{,}
 \AttributeTok{extract\_fun  =} \StringTok{"mean"}\NormalTok{,}
 \AttributeTok{fill\_missing  =} \ConstantTok{TRUE}\NormalTok{,}
 \AttributeTok{IDW\_weight   =} \DecValTok{2}\NormalTok{,}
 \AttributeTok{future\_max\_size =} \DecValTok{40} \SpecialCharTok{*} \DecValTok{1024}\SpecialCharTok{\^{}}\DecValTok{3}\NormalTok{)}


\CommentTok{\# FarmlandCrops\_CropsHoed\_r10000.tif    egv\_189 {-}{-}{-}{-}}
\NormalTok{slanis}\OtherTok{=}\FunctionTok{rast}\NormalTok{(}\StringTok{"./RasterGrids\_100m/2024/RAW/FarmlandCrops\_CropsHoed\_r10000.tif"}\NormalTok{)}
\FunctionTok{names}\NormalTok{(slanis)}\OtherTok{=}\StringTok{"egv\_189"}
\NormalTok{slanis2}\OtherTok{=}\FunctionTok{project}\NormalTok{(slanis,template100)}
\FunctionTok{writeRaster}\NormalTok{(slanis2,}
      \StringTok{"./RasterGrids\_100m/2024/RAW/FarmlandCrops\_CropsHoed\_r10000.tif"}\NormalTok{,}
      \AttributeTok{overwrite=}\ConstantTok{TRUE}\NormalTok{)}

\CommentTok{\# standardisation {-}{-}{-}{-}}
\ControlFlowTok{if}\NormalTok{(}\SpecialCharTok{!}\FunctionTok{require}\NormalTok{(terra)) \{}\FunctionTok{install.packages}\NormalTok{(}\StringTok{"terra"}\NormalTok{); }\FunctionTok{require}\NormalTok{(terra)\}}
\ControlFlowTok{if}\NormalTok{(}\SpecialCharTok{!}\FunctionTok{require}\NormalTok{(tidyverse)) \{}\FunctionTok{install.packages}\NormalTok{(}\StringTok{"tidyverse"}\NormalTok{); }\FunctionTok{require}\NormalTok{(tidyverse)\}}

\NormalTok{nosaukums}\OtherTok{=}\StringTok{"FarmlandCrops\_CropsHoed\_r10000.tif"}
\NormalTok{ielasisanas\_cels}\OtherTok{=}\FunctionTok{paste0}\NormalTok{(}\StringTok{"./RasterGrids\_100m/2024/RAW/"}\NormalTok{,nosaukums)}
\NormalTok{saglabasanas\_cels}\OtherTok{=}\FunctionTok{paste0}\NormalTok{(}\StringTok{"./RasterGrids\_100m/2024/Scaled/"}\NormalTok{,nosaukums)}
\NormalTok{slanis}\OtherTok{=}\FunctionTok{rast}\NormalTok{(ielasisanas\_cels)}
\NormalTok{videjais}\OtherTok{=}\FunctionTok{global}\NormalTok{(slanis,}\AttributeTok{fun=}\StringTok{"mean"}\NormalTok{,}\AttributeTok{na.rm=}\ConstantTok{TRUE}\NormalTok{)}
\NormalTok{centrets}\OtherTok{=}\NormalTok{slanis}\SpecialCharTok{{-}}\NormalTok{videjais[,}\DecValTok{1}\NormalTok{]}
\NormalTok{standartnovirze}\OtherTok{=}\NormalTok{terra}\SpecialCharTok{::}\FunctionTok{global}\NormalTok{(centrets,}\AttributeTok{fun=}\StringTok{"rms"}\NormalTok{,}\AttributeTok{na.rm=}\ConstantTok{TRUE}\NormalTok{)}
\NormalTok{merogots}\OtherTok{=}\NormalTok{centrets}\SpecialCharTok{/}\NormalTok{standartnovirze[,}\DecValTok{1}\NormalTok{]}
\FunctionTok{writeRaster}\NormalTok{(merogots,}
      \AttributeTok{filename=}\NormalTok{saglabasanas\_cels,}
      \AttributeTok{overwrite=}\ConstantTok{TRUE}\NormalTok{)}
\end{Highlighting}
\end{Shaded}

\section{FarmlandCrops\_CropsOther\_cell}\label{ch06.190}

\textbf{filename:} \texttt{FarmlandCrops\_CropsOther\_cell.tif}

\textbf{layername:} \texttt{egv\_190}

\textbf{English name:} Fractional cover of Other Crops within the analysis cell (1
ha)

\textbf{Latvian name:} Citu lauksaimniecības kultūraugu aramzemēs platības īpatsvars
analīzes šūnā (1 ha)

\textbf{Procedure:} First, agricultural parcels with otherwise not differentiated
crops are selected from the \hyperref[Ch04.02]{Rural Support Service's information on declared
fields}. These geometries are then rasterised to input resolution,
ensuring value 1 at the polygon locations and value 0 elsewhere. Rasterisation is performed using the workflow \texttt{egvtools::polygon2input()}. Once rasterised, the layer is aggregated to EGV
resolution using the workflow \texttt{egvtools::input2egv()}, which calculates the arithmetic mean and thus
results in a cover fraction. During aggregation, inverse
distance weighted (power = 2) gap filling on the output is applied to
ensure no missing values at the edges. Finally, the layer is standardised
by subtracting the arithmetic mean and dividing by the root mean squared error.

\begin{Shaded}
\begin{Highlighting}[]
\CommentTok{\# libs {-}{-}{-}{-}}
\ControlFlowTok{if}\NormalTok{(}\SpecialCharTok{!}\FunctionTok{require}\NormalTok{(egvtools)) \{remotes}\SpecialCharTok{::}\FunctionTok{install\_github}\NormalTok{(}\StringTok{"aavotins/egvtools"}\NormalTok{); }\FunctionTok{require}\NormalTok{(egvtools)\}}
\ControlFlowTok{if}\NormalTok{(}\SpecialCharTok{!}\FunctionTok{require}\NormalTok{(terra)) \{}\FunctionTok{install.packages}\NormalTok{(}\StringTok{"terra"}\NormalTok{); }\FunctionTok{require}\NormalTok{(terra)\}}
\ControlFlowTok{if}\NormalTok{(}\SpecialCharTok{!}\FunctionTok{require}\NormalTok{(sf)) \{}\FunctionTok{install.packages}\NormalTok{(}\StringTok{"sf"}\NormalTok{); }\FunctionTok{require}\NormalTok{(sf)\}}
\ControlFlowTok{if}\NormalTok{(}\SpecialCharTok{!}\FunctionTok{require}\NormalTok{(tidyverse)) \{}\FunctionTok{install.packages}\NormalTok{(}\StringTok{"tidyverse"}\NormalTok{); }\FunctionTok{require}\NormalTok{(tidyverse)\}}
\ControlFlowTok{if}\NormalTok{(}\SpecialCharTok{!}\FunctionTok{require}\NormalTok{(sfarrow)) \{}\FunctionTok{install.packages}\NormalTok{(}\StringTok{"sfarrow"}\NormalTok{); }\FunctionTok{require}\NormalTok{(sfarrow)\}}
\ControlFlowTok{if}\NormalTok{(}\SpecialCharTok{!}\FunctionTok{require}\NormalTok{(readxl)) \{}\FunctionTok{install.packages}\NormalTok{(}\StringTok{"readxl"}\NormalTok{); }\FunctionTok{require}\NormalTok{(readxl)\}}
\ControlFlowTok{if}\NormalTok{(}\SpecialCharTok{!}\FunctionTok{require}\NormalTok{(raster)) \{}\FunctionTok{install.packages}\NormalTok{(}\StringTok{"raster"}\NormalTok{); }\FunctionTok{require}\NormalTok{(raster)\}}
\ControlFlowTok{if}\NormalTok{(}\SpecialCharTok{!}\FunctionTok{require}\NormalTok{(fasterize)) \{}\FunctionTok{install.packages}\NormalTok{(}\StringTok{"fasterize"}\NormalTok{); }\FunctionTok{require}\NormalTok{(fasterize)\}}

\CommentTok{\# templates {-}{-}{-}{-}}
\NormalTok{template100}\OtherTok{=}\FunctionTok{rast}\NormalTok{(}\StringTok{"./Templates/TemplateRasters/LV100m\_10km.tif"}\NormalTok{)}
\NormalTok{template10}\OtherTok{=}\FunctionTok{rast}\NormalTok{(}\StringTok{"./Templates/TemplateRasters/LV10m\_10km.tif"}\NormalTok{)}
\NormalTok{rastrs10}\OtherTok{=}\FunctionTok{raster}\NormalTok{(template10)}

\NormalTok{nulls10}\OtherTok{=}\FunctionTok{rast}\NormalTok{(}\StringTok{"./Templates/TemplateRasters/nulls\_LV10m\_10km.tif"}\NormalTok{)}
\NormalTok{nulls100}\OtherTok{=}\FunctionTok{rast}\NormalTok{(}\StringTok{"./Templates/TemplateRasters/nulls\_LV100m\_10km.tif"}\NormalTok{)}

\CommentTok{\# codes {-}{-}{-}{-}}
\NormalTok{kodi}\OtherTok{=}\FunctionTok{read\_excel}\NormalTok{(}\StringTok{"./Geodata/2024/LAD/KulturuKodi\_2024.xlsx"}\NormalTok{)}
\NormalTok{kodi}\SpecialCharTok{$}\NormalTok{kods}\OtherTok{=}\FunctionTok{as.character}\NormalTok{(kodi}\SpecialCharTok{$}\NormalTok{kods)}
\CommentTok{\# LAD {-}{-}{-}{-}}
\NormalTok{lad}\OtherTok{=}\NormalTok{sfarrow}\SpecialCharTok{::}\FunctionTok{st\_read\_parquet}\NormalTok{(}\StringTok{"./Geodata/2024/LAD/Lauki\_2024.parquet"}\NormalTok{)}
\NormalTok{lad}\SpecialCharTok{$}\NormalTok{yes}\OtherTok{=}\DecValTok{1}
\NormalTok{lad}\OtherTok{=}\NormalTok{lad }\SpecialCharTok{\%\textgreater{}\%} 
 \FunctionTok{left\_join}\NormalTok{(kodi,}\AttributeTok{by=}\FunctionTok{c}\NormalTok{(}\StringTok{"PRODUCT\_CODE"}\OtherTok{=}\StringTok{"kods"}\NormalTok{))}

\CommentTok{\# simple landscape {-}{-}{-}{-}}
\NormalTok{simple\_landscape}\OtherTok{=}\FunctionTok{rast}\NormalTok{(}\StringTok{"RasterGrids\_10m/2024/Ainava\_vienk\_mask.tif"}\NormalTok{)}


\CommentTok{\# FarmlandCrops\_CropsOther\_cell.tif egv\_190 {-}{-}{-}{-}}
\NormalTok{dati}\OtherTok{=}\NormalTok{lad }\SpecialCharTok{\%\textgreater{}\%} 
 \FunctionTok{filter}\NormalTok{(}\FunctionTok{str\_detect}\NormalTok{(SDM\_grupa\_sakums,}\StringTok{"citur neie"}\NormalTok{))}
\FunctionTok{table}\NormalTok{(dati}\SpecialCharTok{$}\NormalTok{SDM\_grupa\_sakums,}\AttributeTok{useNA=}\StringTok{"always"}\NormalTok{)}


\NormalTok{p2i\_rez}\OtherTok{=}\NormalTok{egvtools}\SpecialCharTok{::}\FunctionTok{polygon2input}\NormalTok{(}\AttributeTok{vector\_data =}\NormalTok{ dati,}
                \AttributeTok{template\_path =} \StringTok{"./Templates/TemplateRasters/LV10m\_10km.tif"}\NormalTok{,}
                \AttributeTok{out\_path =} \StringTok{"./RasterGrids\_10m/2024/"}\NormalTok{,}
                \AttributeTok{file\_name =} \StringTok{"FarmlandCrops\_CropsOther\_input.tif"}\NormalTok{,}
                \AttributeTok{value\_field =} \StringTok{"yes"}\NormalTok{,}
                \AttributeTok{prepare=}\ConstantTok{FALSE}\NormalTok{,}
                \AttributeTok{background\_raster =} \StringTok{"./Templates/TemplateRasters/nulls\_LV10m\_10km.tif"}\NormalTok{,}
                \AttributeTok{plot\_result =} \ConstantTok{TRUE}\NormalTok{)}
\NormalTok{p2i\_rez}
\NormalTok{i2e\_rez}\OtherTok{=}\NormalTok{egvtools}\SpecialCharTok{::}\FunctionTok{input2egv}\NormalTok{(}\AttributeTok{input=}\FunctionTok{paste0}\NormalTok{(}\StringTok{"./RasterGrids\_10m/2024/"}\NormalTok{,}
                     \StringTok{"FarmlandCrops\_CropsOther\_input.tif"}\NormalTok{),}
              \AttributeTok{egv\_template=} \StringTok{"./Templates/TemplateRasters/LV100m\_10km.tif"}\NormalTok{,}
              \AttributeTok{summary\_function =} \StringTok{"average"}\NormalTok{,}
              \AttributeTok{missing\_job =} \StringTok{"FillOutput"}\NormalTok{,}
              \AttributeTok{outlocation =} \StringTok{"./RasterGrids\_100m/2024/RAW/"}\NormalTok{,}
              \AttributeTok{outfilename =} \StringTok{"FarmlandCrops\_CropsOther\_cell.tif"}\NormalTok{,}
              \AttributeTok{layername =} \StringTok{"egv\_190"}\NormalTok{,}
              \AttributeTok{idw\_weight =} \DecValTok{2}\NormalTok{,}
              \AttributeTok{plot\_gaps =} \ConstantTok{FALSE}\NormalTok{,}\AttributeTok{plot\_final =} \ConstantTok{TRUE}\NormalTok{)}
\NormalTok{i2e\_rez}
\FunctionTok{rm}\NormalTok{(p2i\_rez)}
\FunctionTok{rm}\NormalTok{(i2e\_rez)}
\FunctionTok{rm}\NormalTok{(dati)}
\FunctionTok{unlink}\NormalTok{(}\StringTok{"./RasterGrids\_10m/2024/FarmlandCrops\_CropsOther\_input.tif"}\NormalTok{)}

\CommentTok{\# standardisation {-}{-}{-}{-}}
\ControlFlowTok{if}\NormalTok{(}\SpecialCharTok{!}\FunctionTok{require}\NormalTok{(terra)) \{}\FunctionTok{install.packages}\NormalTok{(}\StringTok{"terra"}\NormalTok{); }\FunctionTok{require}\NormalTok{(terra)\}}
\ControlFlowTok{if}\NormalTok{(}\SpecialCharTok{!}\FunctionTok{require}\NormalTok{(tidyverse)) \{}\FunctionTok{install.packages}\NormalTok{(}\StringTok{"tidyverse"}\NormalTok{); }\FunctionTok{require}\NormalTok{(tidyverse)\}}

\NormalTok{nosaukums}\OtherTok{=}\StringTok{"FarmlandCrops\_CropsOther\_cell.tif"}
\NormalTok{ielasisanas\_cels}\OtherTok{=}\FunctionTok{paste0}\NormalTok{(}\StringTok{"./RasterGrids\_100m/2024/RAW/"}\NormalTok{,nosaukums)}
\NormalTok{saglabasanas\_cels}\OtherTok{=}\FunctionTok{paste0}\NormalTok{(}\StringTok{"./RasterGrids\_100m/2024/Scaled/"}\NormalTok{,nosaukums)}
\NormalTok{slanis}\OtherTok{=}\FunctionTok{rast}\NormalTok{(ielasisanas\_cels)}
\NormalTok{videjais}\OtherTok{=}\FunctionTok{global}\NormalTok{(slanis,}\AttributeTok{fun=}\StringTok{"mean"}\NormalTok{,}\AttributeTok{na.rm=}\ConstantTok{TRUE}\NormalTok{)}
\NormalTok{centrets}\OtherTok{=}\NormalTok{slanis}\SpecialCharTok{{-}}\NormalTok{videjais[,}\DecValTok{1}\NormalTok{]}
\NormalTok{standartnovirze}\OtherTok{=}\NormalTok{terra}\SpecialCharTok{::}\FunctionTok{global}\NormalTok{(centrets,}\AttributeTok{fun=}\StringTok{"rms"}\NormalTok{,}\AttributeTok{na.rm=}\ConstantTok{TRUE}\NormalTok{)}
\NormalTok{merogots}\OtherTok{=}\NormalTok{centrets}\SpecialCharTok{/}\NormalTok{standartnovirze[,}\DecValTok{1}\NormalTok{]}
\FunctionTok{writeRaster}\NormalTok{(merogots,}
      \AttributeTok{filename=}\NormalTok{saglabasanas\_cels,}
      \AttributeTok{overwrite=}\ConstantTok{TRUE}\NormalTok{)}
\end{Highlighting}
\end{Shaded}

\section{FarmlandCrops\_CropsOther\_r500}\label{ch06.191}

\textbf{filename:} \texttt{FarmlandCrops\_CropsOther\_r500.tif}

\textbf{layername:} \texttt{egv\_191}

\textbf{English name:} Fractional cover of Other Crops within the 0.5 km landscape

\textbf{Latvian name:} Citu lauksaimniecības kultūraugu aramzemēs platības īpatsvars
0,5 km ainavā

\textbf{Procedure:} The cover fraction within a radius of 500 m around the analysis grid cell is
calculated as the area-weighted sum of the \hyperref[ch06.190]{analysis cells} inside the
buffer, using the workflow \texttt{egvtools::radius\_function()}. During the calculation of the landscape metric,
inverse distance weighted (power = 2) gap filling on the output is applied
to ensure no missing values at the edges. Then the layer is rewritten to set
its name. Finally, the layer is standardised by subtracting the arithmetic
mean and dividing by the root mean squared error.

\begin{Shaded}
\begin{Highlighting}[]
\CommentTok{\# libs {-}{-}{-}{-}}
\ControlFlowTok{if}\NormalTok{(}\SpecialCharTok{!}\FunctionTok{require}\NormalTok{(terra)) \{}\FunctionTok{install.packages}\NormalTok{(}\StringTok{"terra"}\NormalTok{); }\FunctionTok{require}\NormalTok{(terra)\}}
\ControlFlowTok{if}\NormalTok{(}\SpecialCharTok{!}\FunctionTok{require}\NormalTok{(egvtools)) \{remotes}\SpecialCharTok{::}\FunctionTok{install\_github}\NormalTok{(}\StringTok{"aavotins/egvtools"}\NormalTok{); }\FunctionTok{require}\NormalTok{(egvtools)\}}


\CommentTok{\# Templates {-}{-}{-}{-}{-}}
\NormalTok{template100}\OtherTok{=}\FunctionTok{rast}\NormalTok{(}\StringTok{"./Templates/TemplateRasters/LV100m\_10km.tif"}\NormalTok{)}

\CommentTok{\# radii {-}{-}{-}{-}}
\FunctionTok{radius\_function}\NormalTok{(}
 \AttributeTok{kvadrati\_path =} \StringTok{"./Templates/TemplateGrids/tiles/"}\NormalTok{,}
 \AttributeTok{radii\_path   =} \StringTok{"./Templates/TemplateGridPoints/tiles/"}\NormalTok{,}
 \AttributeTok{tikls100\_path =} \StringTok{"./Templates/TemplateGrids/tikls100\_sauzeme.parquet"}\NormalTok{,}
 \AttributeTok{template\_path =} \StringTok{"./Templates/TemplateRasters/LV100m\_10km.tif"}\NormalTok{,}
 \AttributeTok{input\_layers  =} \FunctionTok{c}\NormalTok{(}\StringTok{"./RasterGrids\_100m/2024/RAW/FarmlandCrops\_CropsOther\_cell.tif"}\NormalTok{),}
 \AttributeTok{layer\_prefixes =} \FunctionTok{c}\NormalTok{(}\StringTok{"FarmlandCrops\_CropsOther"}\NormalTok{),}
 \AttributeTok{output\_dir   =} \StringTok{"./RasterGrids\_100m/2024/RAW/"}\NormalTok{,}
 \AttributeTok{n\_workers   =} \DecValTok{6}\NormalTok{,}
 \AttributeTok{radii     =} \FunctionTok{c}\NormalTok{(}\StringTok{"r500"}\NormalTok{),}
 \AttributeTok{radius\_mode  =} \StringTok{"sparse"}\NormalTok{,}
 \AttributeTok{extract\_fun  =} \StringTok{"mean"}\NormalTok{,}
 \AttributeTok{fill\_missing  =} \ConstantTok{TRUE}\NormalTok{,}
 \AttributeTok{IDW\_weight   =} \DecValTok{2}\NormalTok{,}
 \AttributeTok{future\_max\_size =} \DecValTok{40} \SpecialCharTok{*} \DecValTok{1024}\SpecialCharTok{\^{}}\DecValTok{3}\NormalTok{)}


\CommentTok{\# FarmlandCrops\_CropsOther\_r500.tif egv\_191 {-}{-}{-}{-}}
\NormalTok{slanis}\OtherTok{=}\FunctionTok{rast}\NormalTok{(}\StringTok{"./RasterGrids\_100m/2024/RAW/FarmlandCrops\_CropsOther\_r500.tif"}\NormalTok{)}
\FunctionTok{names}\NormalTok{(slanis)}\OtherTok{=}\StringTok{"egv\_191"}
\NormalTok{slanis2}\OtherTok{=}\FunctionTok{project}\NormalTok{(slanis,template100)}
\FunctionTok{writeRaster}\NormalTok{(slanis2,}
      \StringTok{"./RasterGrids\_100m/2024/RAW/FarmlandCrops\_CropsOther\_r500.tif"}\NormalTok{,}
      \AttributeTok{overwrite=}\ConstantTok{TRUE}\NormalTok{)}

\CommentTok{\# standardisation {-}{-}{-}{-}}
\ControlFlowTok{if}\NormalTok{(}\SpecialCharTok{!}\FunctionTok{require}\NormalTok{(terra)) \{}\FunctionTok{install.packages}\NormalTok{(}\StringTok{"terra"}\NormalTok{); }\FunctionTok{require}\NormalTok{(terra)\}}
\ControlFlowTok{if}\NormalTok{(}\SpecialCharTok{!}\FunctionTok{require}\NormalTok{(tidyverse)) \{}\FunctionTok{install.packages}\NormalTok{(}\StringTok{"tidyverse"}\NormalTok{); }\FunctionTok{require}\NormalTok{(tidyverse)\}}

\NormalTok{nosaukums}\OtherTok{=}\StringTok{"FarmlandCrops\_CropsOther\_r500.tif"}
\NormalTok{ielasisanas\_cels}\OtherTok{=}\FunctionTok{paste0}\NormalTok{(}\StringTok{"./RasterGrids\_100m/2024/RAW/"}\NormalTok{,nosaukums)}
\NormalTok{saglabasanas\_cels}\OtherTok{=}\FunctionTok{paste0}\NormalTok{(}\StringTok{"./RasterGrids\_100m/2024/Scaled/"}\NormalTok{,nosaukums)}
\NormalTok{slanis}\OtherTok{=}\FunctionTok{rast}\NormalTok{(ielasisanas\_cels)}
\NormalTok{videjais}\OtherTok{=}\FunctionTok{global}\NormalTok{(slanis,}\AttributeTok{fun=}\StringTok{"mean"}\NormalTok{,}\AttributeTok{na.rm=}\ConstantTok{TRUE}\NormalTok{)}
\NormalTok{centrets}\OtherTok{=}\NormalTok{slanis}\SpecialCharTok{{-}}\NormalTok{videjais[,}\DecValTok{1}\NormalTok{]}
\NormalTok{standartnovirze}\OtherTok{=}\NormalTok{terra}\SpecialCharTok{::}\FunctionTok{global}\NormalTok{(centrets,}\AttributeTok{fun=}\StringTok{"rms"}\NormalTok{,}\AttributeTok{na.rm=}\ConstantTok{TRUE}\NormalTok{)}
\NormalTok{merogots}\OtherTok{=}\NormalTok{centrets}\SpecialCharTok{/}\NormalTok{standartnovirze[,}\DecValTok{1}\NormalTok{]}
\FunctionTok{writeRaster}\NormalTok{(merogots,}
      \AttributeTok{filename=}\NormalTok{saglabasanas\_cels,}
      \AttributeTok{overwrite=}\ConstantTok{TRUE}\NormalTok{)}
\end{Highlighting}
\end{Shaded}

\section{FarmlandCrops\_CropsOther\_r1250}\label{ch06.192}

\textbf{filename:} \texttt{FarmlandCrops\_CropsOther\_r1250.tif}

\textbf{layername:} \texttt{egv\_192}

\textbf{English name:} Fractional cover of Other Crops within the 1.25 km landscape

\textbf{Latvian name:} Citu lauksaimniecības kultūraugu aramzemēs platības īpatsvars
1,25 km ainavā

\textbf{Procedure:} The cover fraction within a radius of 1250 m around the analysis grid cell
is calculated as the area-weighted sum of the \hyperref[ch06.190]{analysis cells} inside
the buffer, using the workflow \texttt{egvtools::radius\_function()}. During the calculation of the landscape
metric, inverse distance weighted (power = 2) gap filling on the output is
applied to ensure no missing values at the edges. Then the layer is
rewritten to set its name. Finally, the layer is standardised by
subtracting the arithmetic mean and dividing by the root mean squared error.

\begin{Shaded}
\begin{Highlighting}[]
\CommentTok{\# libs {-}{-}{-}{-}}
\ControlFlowTok{if}\NormalTok{(}\SpecialCharTok{!}\FunctionTok{require}\NormalTok{(terra)) \{}\FunctionTok{install.packages}\NormalTok{(}\StringTok{"terra"}\NormalTok{); }\FunctionTok{require}\NormalTok{(terra)\}}
\ControlFlowTok{if}\NormalTok{(}\SpecialCharTok{!}\FunctionTok{require}\NormalTok{(egvtools)) \{remotes}\SpecialCharTok{::}\FunctionTok{install\_github}\NormalTok{(}\StringTok{"aavotins/egvtools"}\NormalTok{); }\FunctionTok{require}\NormalTok{(egvtools)\}}


\CommentTok{\# Templates {-}{-}{-}{-}{-}}
\NormalTok{template100}\OtherTok{=}\FunctionTok{rast}\NormalTok{(}\StringTok{"./Templates/TemplateRasters/LV100m\_10km.tif"}\NormalTok{)}

\CommentTok{\# radii {-}{-}{-}{-}}
\FunctionTok{radius\_function}\NormalTok{(}
 \AttributeTok{kvadrati\_path =} \StringTok{"./Templates/TemplateGrids/tiles/"}\NormalTok{,}
 \AttributeTok{radii\_path   =} \StringTok{"./Templates/TemplateGridPoints/tiles/"}\NormalTok{,}
 \AttributeTok{tikls100\_path =} \StringTok{"./Templates/TemplateGrids/tikls100\_sauzeme.parquet"}\NormalTok{,}
 \AttributeTok{template\_path =} \StringTok{"./Templates/TemplateRasters/LV100m\_10km.tif"}\NormalTok{,}
 \AttributeTok{input\_layers  =} \FunctionTok{c}\NormalTok{(}\StringTok{"./RasterGrids\_100m/2024/RAW/FarmlandCrops\_CropsOther\_cell.tif"}\NormalTok{),}
 \AttributeTok{layer\_prefixes =} \FunctionTok{c}\NormalTok{(}\StringTok{"FarmlandCrops\_CropsOther"}\NormalTok{),}
 \AttributeTok{output\_dir   =} \StringTok{"./RasterGrids\_100m/2024/RAW/"}\NormalTok{,}
 \AttributeTok{n\_workers   =} \DecValTok{6}\NormalTok{,}
 \AttributeTok{radii     =} \FunctionTok{c}\NormalTok{(}\StringTok{"r1250"}\NormalTok{),}
 \AttributeTok{radius\_mode  =} \StringTok{"sparse"}\NormalTok{,}
 \AttributeTok{extract\_fun  =} \StringTok{"mean"}\NormalTok{,}
 \AttributeTok{fill\_missing  =} \ConstantTok{TRUE}\NormalTok{,}
 \AttributeTok{IDW\_weight   =} \DecValTok{2}\NormalTok{,}
 \AttributeTok{future\_max\_size =} \DecValTok{40} \SpecialCharTok{*} \DecValTok{1024}\SpecialCharTok{\^{}}\DecValTok{3}\NormalTok{)}


\CommentTok{\# FarmlandCrops\_CropsOther\_r1250.tif    egv\_192 {-}{-}{-}{-}}
\NormalTok{slanis}\OtherTok{=}\FunctionTok{rast}\NormalTok{(}\StringTok{"./RasterGrids\_100m/2024/RAW/FarmlandCrops\_CropsOther\_r1250.tif"}\NormalTok{)}
\FunctionTok{names}\NormalTok{(slanis)}\OtherTok{=}\StringTok{"egv\_192"}
\NormalTok{slanis2}\OtherTok{=}\FunctionTok{project}\NormalTok{(slanis,template100)}
\FunctionTok{writeRaster}\NormalTok{(slanis2,}
      \StringTok{"./RasterGrids\_100m/2024/RAW/FarmlandCrops\_CropsOther\_r1250.tif"}\NormalTok{,}
      \AttributeTok{overwrite=}\ConstantTok{TRUE}\NormalTok{)}

\CommentTok{\# standardisation {-}{-}{-}{-}}
\ControlFlowTok{if}\NormalTok{(}\SpecialCharTok{!}\FunctionTok{require}\NormalTok{(terra)) \{}\FunctionTok{install.packages}\NormalTok{(}\StringTok{"terra"}\NormalTok{); }\FunctionTok{require}\NormalTok{(terra)\}}
\ControlFlowTok{if}\NormalTok{(}\SpecialCharTok{!}\FunctionTok{require}\NormalTok{(tidyverse)) \{}\FunctionTok{install.packages}\NormalTok{(}\StringTok{"tidyverse"}\NormalTok{); }\FunctionTok{require}\NormalTok{(tidyverse)\}}

\NormalTok{nosaukums}\OtherTok{=}\StringTok{"FarmlandCrops\_CropsOther\_r1250.tif"}
\NormalTok{ielasisanas\_cels}\OtherTok{=}\FunctionTok{paste0}\NormalTok{(}\StringTok{"./RasterGrids\_100m/2024/RAW/"}\NormalTok{,nosaukums)}
\NormalTok{saglabasanas\_cels}\OtherTok{=}\FunctionTok{paste0}\NormalTok{(}\StringTok{"./RasterGrids\_100m/2024/Scaled/"}\NormalTok{,nosaukums)}
\NormalTok{slanis}\OtherTok{=}\FunctionTok{rast}\NormalTok{(ielasisanas\_cels)}
\NormalTok{videjais}\OtherTok{=}\FunctionTok{global}\NormalTok{(slanis,}\AttributeTok{fun=}\StringTok{"mean"}\NormalTok{,}\AttributeTok{na.rm=}\ConstantTok{TRUE}\NormalTok{)}
\NormalTok{centrets}\OtherTok{=}\NormalTok{slanis}\SpecialCharTok{{-}}\NormalTok{videjais[,}\DecValTok{1}\NormalTok{]}
\NormalTok{standartnovirze}\OtherTok{=}\NormalTok{terra}\SpecialCharTok{::}\FunctionTok{global}\NormalTok{(centrets,}\AttributeTok{fun=}\StringTok{"rms"}\NormalTok{,}\AttributeTok{na.rm=}\ConstantTok{TRUE}\NormalTok{)}
\NormalTok{merogots}\OtherTok{=}\NormalTok{centrets}\SpecialCharTok{/}\NormalTok{standartnovirze[,}\DecValTok{1}\NormalTok{]}
\FunctionTok{writeRaster}\NormalTok{(merogots,}
      \AttributeTok{filename=}\NormalTok{saglabasanas\_cels,}
      \AttributeTok{overwrite=}\ConstantTok{TRUE}\NormalTok{)}
\end{Highlighting}
\end{Shaded}

\section{FarmlandCrops\_CropsOther\_r3000}\label{ch06.193}

\textbf{filename:} \texttt{FarmlandCrops\_CropsOther\_r3000.tif}

\textbf{layername:} \texttt{egv\_193}

\textbf{English name:} Fractional cover of Other Crops within the 3 km landscape

\textbf{Latvian name:} Citu lauksaimniecības kultūraugu aramzemēs platības īpatsvars
3 km ainavā

\textbf{Procedure:} The cover fraction within a radius of 3000 m around the analysis grid cell
is calculated as the area-weighted sum of the \hyperref[ch06.190]{analysis cells} inside
the buffer, using the workflow \texttt{egvtools::radius\_function()}. During the calculation of the landscape
metric, inverse distance weighted (power = 2) gap filling on the output is
applied to ensure no missing values at the edges. Then the layer is
rewritten to set its name. Finally, the layer is standardised by
subtracting the arithmetic mean and dividing by the root mean squared error.

\begin{Shaded}
\begin{Highlighting}[]
\CommentTok{\# libs {-}{-}{-}{-}}
\ControlFlowTok{if}\NormalTok{(}\SpecialCharTok{!}\FunctionTok{require}\NormalTok{(terra)) \{}\FunctionTok{install.packages}\NormalTok{(}\StringTok{"terra"}\NormalTok{); }\FunctionTok{require}\NormalTok{(terra)\}}
\ControlFlowTok{if}\NormalTok{(}\SpecialCharTok{!}\FunctionTok{require}\NormalTok{(egvtools)) \{remotes}\SpecialCharTok{::}\FunctionTok{install\_github}\NormalTok{(}\StringTok{"aavotins/egvtools"}\NormalTok{); }\FunctionTok{require}\NormalTok{(egvtools)\}}


\CommentTok{\# Templates {-}{-}{-}{-}{-}}
\NormalTok{template100}\OtherTok{=}\FunctionTok{rast}\NormalTok{(}\StringTok{"./Templates/TemplateRasters/LV100m\_10km.tif"}\NormalTok{)}

\CommentTok{\# radii {-}{-}{-}{-}}
\FunctionTok{radius\_function}\NormalTok{(}
 \AttributeTok{kvadrati\_path =} \StringTok{"./Templates/TemplateGrids/tiles/"}\NormalTok{,}
 \AttributeTok{radii\_path   =} \StringTok{"./Templates/TemplateGridPoints/tiles/"}\NormalTok{,}
 \AttributeTok{tikls100\_path =} \StringTok{"./Templates/TemplateGrids/tikls100\_sauzeme.parquet"}\NormalTok{,}
 \AttributeTok{template\_path =} \StringTok{"./Templates/TemplateRasters/LV100m\_10km.tif"}\NormalTok{,}
 \AttributeTok{input\_layers  =} \FunctionTok{c}\NormalTok{(}\StringTok{"./RasterGrids\_100m/2024/RAW/FarmlandCrops\_CropsOther\_cell.tif"}\NormalTok{),}
 \AttributeTok{layer\_prefixes =} \FunctionTok{c}\NormalTok{(}\StringTok{"FarmlandCrops\_CropsOther"}\NormalTok{),}
 \AttributeTok{output\_dir   =} \StringTok{"./RasterGrids\_100m/2024/RAW/"}\NormalTok{,}
 \AttributeTok{n\_workers   =} \DecValTok{6}\NormalTok{,}
 \AttributeTok{radii     =} \FunctionTok{c}\NormalTok{(}\StringTok{"r3000"}\NormalTok{),}
 \AttributeTok{radius\_mode  =} \StringTok{"sparse"}\NormalTok{,}
 \AttributeTok{extract\_fun  =} \StringTok{"mean"}\NormalTok{,}
 \AttributeTok{fill\_missing  =} \ConstantTok{TRUE}\NormalTok{,}
 \AttributeTok{IDW\_weight   =} \DecValTok{2}\NormalTok{,}
 \AttributeTok{future\_max\_size =} \DecValTok{40} \SpecialCharTok{*} \DecValTok{1024}\SpecialCharTok{\^{}}\DecValTok{3}\NormalTok{)}


\CommentTok{\# FarmlandCrops\_CropsOther\_r3000.tif    egv\_193 {-}{-}{-}{-}}
\NormalTok{slanis}\OtherTok{=}\FunctionTok{rast}\NormalTok{(}\StringTok{"./RasterGrids\_100m/2024/RAW/FarmlandCrops\_CropsOther\_r3000.tif"}\NormalTok{)}
\FunctionTok{names}\NormalTok{(slanis)}\OtherTok{=}\StringTok{"egv\_193"}
\NormalTok{slanis2}\OtherTok{=}\FunctionTok{project}\NormalTok{(slanis,template100)}
\FunctionTok{writeRaster}\NormalTok{(slanis2,}
      \StringTok{"./RasterGrids\_100m/2024/RAW/FarmlandCrops\_CropsOther\_r3000.tif"}\NormalTok{,}
      \AttributeTok{overwrite=}\ConstantTok{TRUE}\NormalTok{)}

\CommentTok{\# standardisation {-}{-}{-}{-}}
\ControlFlowTok{if}\NormalTok{(}\SpecialCharTok{!}\FunctionTok{require}\NormalTok{(terra)) \{}\FunctionTok{install.packages}\NormalTok{(}\StringTok{"terra"}\NormalTok{); }\FunctionTok{require}\NormalTok{(terra)\}}
\ControlFlowTok{if}\NormalTok{(}\SpecialCharTok{!}\FunctionTok{require}\NormalTok{(tidyverse)) \{}\FunctionTok{install.packages}\NormalTok{(}\StringTok{"tidyverse"}\NormalTok{); }\FunctionTok{require}\NormalTok{(tidyverse)\}}

\NormalTok{nosaukums}\OtherTok{=}\StringTok{"FarmlandCrops\_CropsOther\_r3000.tif"}
\NormalTok{ielasisanas\_cels}\OtherTok{=}\FunctionTok{paste0}\NormalTok{(}\StringTok{"./RasterGrids\_100m/2024/RAW/"}\NormalTok{,nosaukums)}
\NormalTok{saglabasanas\_cels}\OtherTok{=}\FunctionTok{paste0}\NormalTok{(}\StringTok{"./RasterGrids\_100m/2024/Scaled/"}\NormalTok{,nosaukums)}
\NormalTok{slanis}\OtherTok{=}\FunctionTok{rast}\NormalTok{(ielasisanas\_cels)}
\NormalTok{videjais}\OtherTok{=}\FunctionTok{global}\NormalTok{(slanis,}\AttributeTok{fun=}\StringTok{"mean"}\NormalTok{,}\AttributeTok{na.rm=}\ConstantTok{TRUE}\NormalTok{)}
\NormalTok{centrets}\OtherTok{=}\NormalTok{slanis}\SpecialCharTok{{-}}\NormalTok{videjais[,}\DecValTok{1}\NormalTok{]}
\NormalTok{standartnovirze}\OtherTok{=}\NormalTok{terra}\SpecialCharTok{::}\FunctionTok{global}\NormalTok{(centrets,}\AttributeTok{fun=}\StringTok{"rms"}\NormalTok{,}\AttributeTok{na.rm=}\ConstantTok{TRUE}\NormalTok{)}
\NormalTok{merogots}\OtherTok{=}\NormalTok{centrets}\SpecialCharTok{/}\NormalTok{standartnovirze[,}\DecValTok{1}\NormalTok{]}
\FunctionTok{writeRaster}\NormalTok{(merogots,}
      \AttributeTok{filename=}\NormalTok{saglabasanas\_cels,}
      \AttributeTok{overwrite=}\ConstantTok{TRUE}\NormalTok{)}
\end{Highlighting}
\end{Shaded}

\section{FarmlandCrops\_CropsOther\_r10000}\label{ch06.194}

\textbf{filename:} \texttt{FarmlandCrops\_CropsOther\_r10000.tif}

\textbf{layername:} \texttt{egv\_194}

\textbf{English name:} Fractional cover of Other Crops within the 10 km landscape

\textbf{Latvian name:} Citu lauksaimniecības kultūraugu aramzemēs platības īpatsvars
10 km ainavā

\textbf{Procedure:} The cover fraction within a radius of 10000 m around the analysis grid cell
is calculated as the area-weighted sum of the \hyperref[ch06.190]{analysis cells} inside
the buffer, using the workflow \texttt{egvtools::radius\_function()}. During the calculation of the landscape
metric, inverse distance weighted (power = 2) gap filling on the output is
applied to ensure no missing values at the edges. Then the layer is
rewritten to set its name. Finally, the layer is standardised by
subtracting the arithmetic mean and dividing by the root mean squared error.

\begin{Shaded}
\begin{Highlighting}[]
\CommentTok{\# libs {-}{-}{-}{-}}
\ControlFlowTok{if}\NormalTok{(}\SpecialCharTok{!}\FunctionTok{require}\NormalTok{(terra)) \{}\FunctionTok{install.packages}\NormalTok{(}\StringTok{"terra"}\NormalTok{); }\FunctionTok{require}\NormalTok{(terra)\}}
\ControlFlowTok{if}\NormalTok{(}\SpecialCharTok{!}\FunctionTok{require}\NormalTok{(egvtools)) \{remotes}\SpecialCharTok{::}\FunctionTok{install\_github}\NormalTok{(}\StringTok{"aavotins/egvtools"}\NormalTok{); }\FunctionTok{require}\NormalTok{(egvtools)\}}


\CommentTok{\# Templates {-}{-}{-}{-}{-}}
\NormalTok{template100}\OtherTok{=}\FunctionTok{rast}\NormalTok{(}\StringTok{"./Templates/TemplateRasters/LV100m\_10km.tif"}\NormalTok{)}

\CommentTok{\# radii {-}{-}{-}{-}}
\FunctionTok{radius\_function}\NormalTok{(}
 \AttributeTok{kvadrati\_path =} \StringTok{"./Templates/TemplateGrids/tiles/"}\NormalTok{,}
 \AttributeTok{radii\_path   =} \StringTok{"./Templates/TemplateGridPoints/tiles/"}\NormalTok{,}
 \AttributeTok{tikls100\_path =} \StringTok{"./Templates/TemplateGrids/tikls100\_sauzeme.parquet"}\NormalTok{,}
 \AttributeTok{template\_path =} \StringTok{"./Templates/TemplateRasters/LV100m\_10km.tif"}\NormalTok{,}
 \AttributeTok{input\_layers  =} \FunctionTok{c}\NormalTok{(}\StringTok{"./RasterGrids\_100m/2024/RAW/FarmlandCrops\_CropsOther\_cell.tif"}\NormalTok{),}
 \AttributeTok{layer\_prefixes =} \FunctionTok{c}\NormalTok{(}\StringTok{"FarmlandCrops\_CropsOther"}\NormalTok{),}
 \AttributeTok{output\_dir   =} \StringTok{"./RasterGrids\_100m/2024/RAW/"}\NormalTok{,}
 \AttributeTok{n\_workers   =} \DecValTok{6}\NormalTok{,}
 \AttributeTok{radii     =} \FunctionTok{c}\NormalTok{(}\StringTok{"r10000"}\NormalTok{),}
 \AttributeTok{radius\_mode  =} \StringTok{"sparse"}\NormalTok{,}
 \AttributeTok{extract\_fun  =} \StringTok{"mean"}\NormalTok{,}
 \AttributeTok{fill\_missing  =} \ConstantTok{TRUE}\NormalTok{,}
 \AttributeTok{IDW\_weight   =} \DecValTok{2}\NormalTok{,}
 \AttributeTok{future\_max\_size =} \DecValTok{40} \SpecialCharTok{*} \DecValTok{1024}\SpecialCharTok{\^{}}\DecValTok{3}\NormalTok{)}


\CommentTok{\# FarmlandCrops\_CropsOther\_r10000.tif   egv\_194 {-}{-}{-}{-}}
\NormalTok{slanis}\OtherTok{=}\FunctionTok{rast}\NormalTok{(}\StringTok{"./RasterGrids\_100m/2024/RAW/FarmlandCrops\_CropsOther\_r10000.tif"}\NormalTok{)}
\FunctionTok{names}\NormalTok{(slanis)}\OtherTok{=}\StringTok{"egv\_194"}
\NormalTok{slanis2}\OtherTok{=}\FunctionTok{project}\NormalTok{(slanis,template100)}
\FunctionTok{writeRaster}\NormalTok{(slanis2,}
      \StringTok{"./RasterGrids\_100m/2024/RAW/FarmlandCrops\_CropsOther\_r10000.tif"}\NormalTok{,}
      \AttributeTok{overwrite=}\ConstantTok{TRUE}\NormalTok{)}

\CommentTok{\# standardisation {-}{-}{-}{-}}
\ControlFlowTok{if}\NormalTok{(}\SpecialCharTok{!}\FunctionTok{require}\NormalTok{(terra)) \{}\FunctionTok{install.packages}\NormalTok{(}\StringTok{"terra"}\NormalTok{); }\FunctionTok{require}\NormalTok{(terra)\}}
\ControlFlowTok{if}\NormalTok{(}\SpecialCharTok{!}\FunctionTok{require}\NormalTok{(tidyverse)) \{}\FunctionTok{install.packages}\NormalTok{(}\StringTok{"tidyverse"}\NormalTok{); }\FunctionTok{require}\NormalTok{(tidyverse)\}}

\NormalTok{nosaukums}\OtherTok{=}\StringTok{"FarmlandCrops\_CropsOther\_r10000.tif"}
\NormalTok{ielasisanas\_cels}\OtherTok{=}\FunctionTok{paste0}\NormalTok{(}\StringTok{"./RasterGrids\_100m/2024/RAW/"}\NormalTok{,nosaukums)}
\NormalTok{saglabasanas\_cels}\OtherTok{=}\FunctionTok{paste0}\NormalTok{(}\StringTok{"./RasterGrids\_100m/2024/Scaled/"}\NormalTok{,nosaukums)}
\NormalTok{slanis}\OtherTok{=}\FunctionTok{rast}\NormalTok{(ielasisanas\_cels)}
\NormalTok{videjais}\OtherTok{=}\FunctionTok{global}\NormalTok{(slanis,}\AttributeTok{fun=}\StringTok{"mean"}\NormalTok{,}\AttributeTok{na.rm=}\ConstantTok{TRUE}\NormalTok{)}
\NormalTok{centrets}\OtherTok{=}\NormalTok{slanis}\SpecialCharTok{{-}}\NormalTok{videjais[,}\DecValTok{1}\NormalTok{]}
\NormalTok{standartnovirze}\OtherTok{=}\NormalTok{terra}\SpecialCharTok{::}\FunctionTok{global}\NormalTok{(centrets,}\AttributeTok{fun=}\StringTok{"rms"}\NormalTok{,}\AttributeTok{na.rm=}\ConstantTok{TRUE}\NormalTok{)}
\NormalTok{merogots}\OtherTok{=}\NormalTok{centrets}\SpecialCharTok{/}\NormalTok{standartnovirze[,}\DecValTok{1}\NormalTok{]}
\FunctionTok{writeRaster}\NormalTok{(merogots,}
      \AttributeTok{filename=}\NormalTok{saglabasanas\_cels,}
      \AttributeTok{overwrite=}\ConstantTok{TRUE}\NormalTok{)}
\end{Highlighting}
\end{Shaded}

\section{FarmlandCrops\_CropsSpring\_cell}\label{ch06.195}

\textbf{filename:} \texttt{FarmlandCrops\_CropsSpring\_cell.tif}

\textbf{layername:} \texttt{egv\_195}

\textbf{English name:} Fractional cover of Spring Sown Crops within the analysis cell
(1 ha)

\textbf{Latvian name:} Vasarāju aramzemēs platības īpatsvars analīzes šūnā (1 ha)

\textbf{Procedure:} First, agricultural parcels declared as spring sown crops
are selected from the \hyperref[Ch04.02]{Rural Support Service's information on declared
fields}. These geometries are then rasterised to input resolution,
ensuring value 1 at the polygon locations and value 0 elsewhere. Rasterisation is performed using the workflow \texttt{egvtools::polygon2input()}. Once rasterised, the layer is aggregated to EGV
resolution using the workflow \texttt{egvtools::input2egv()}, which calculates the arithmetic mean and thus
results in a cover fraction. During aggregation, inverse
distance weighted (power = 2) gap filling on the output is applied to
ensure no missing values at the edges. Finally, the layer is standardised
by subtracting the arithmetic mean and dividing by the root mean squared error.

\begin{Shaded}
\begin{Highlighting}[]
\CommentTok{\# libs {-}{-}{-}{-}}
\ControlFlowTok{if}\NormalTok{(}\SpecialCharTok{!}\FunctionTok{require}\NormalTok{(egvtools)) \{remotes}\SpecialCharTok{::}\FunctionTok{install\_github}\NormalTok{(}\StringTok{"aavotins/egvtools"}\NormalTok{); }\FunctionTok{require}\NormalTok{(egvtools)\}}
\ControlFlowTok{if}\NormalTok{(}\SpecialCharTok{!}\FunctionTok{require}\NormalTok{(terra)) \{}\FunctionTok{install.packages}\NormalTok{(}\StringTok{"terra"}\NormalTok{); }\FunctionTok{require}\NormalTok{(terra)\}}
\ControlFlowTok{if}\NormalTok{(}\SpecialCharTok{!}\FunctionTok{require}\NormalTok{(sf)) \{}\FunctionTok{install.packages}\NormalTok{(}\StringTok{"sf"}\NormalTok{); }\FunctionTok{require}\NormalTok{(sf)\}}
\ControlFlowTok{if}\NormalTok{(}\SpecialCharTok{!}\FunctionTok{require}\NormalTok{(tidyverse)) \{}\FunctionTok{install.packages}\NormalTok{(}\StringTok{"tidyverse"}\NormalTok{); }\FunctionTok{require}\NormalTok{(tidyverse)\}}
\ControlFlowTok{if}\NormalTok{(}\SpecialCharTok{!}\FunctionTok{require}\NormalTok{(sfarrow)) \{}\FunctionTok{install.packages}\NormalTok{(}\StringTok{"sfarrow"}\NormalTok{); }\FunctionTok{require}\NormalTok{(sfarrow)\}}
\ControlFlowTok{if}\NormalTok{(}\SpecialCharTok{!}\FunctionTok{require}\NormalTok{(readxl)) \{}\FunctionTok{install.packages}\NormalTok{(}\StringTok{"readxl"}\NormalTok{); }\FunctionTok{require}\NormalTok{(readxl)\}}
\ControlFlowTok{if}\NormalTok{(}\SpecialCharTok{!}\FunctionTok{require}\NormalTok{(raster)) \{}\FunctionTok{install.packages}\NormalTok{(}\StringTok{"raster"}\NormalTok{); }\FunctionTok{require}\NormalTok{(raster)\}}
\ControlFlowTok{if}\NormalTok{(}\SpecialCharTok{!}\FunctionTok{require}\NormalTok{(fasterize)) \{}\FunctionTok{install.packages}\NormalTok{(}\StringTok{"fasterize"}\NormalTok{); }\FunctionTok{require}\NormalTok{(fasterize)\}}

\CommentTok{\# templates {-}{-}{-}{-}}
\NormalTok{template100}\OtherTok{=}\FunctionTok{rast}\NormalTok{(}\StringTok{"./Templates/TemplateRasters/LV100m\_10km.tif"}\NormalTok{)}
\NormalTok{template10}\OtherTok{=}\FunctionTok{rast}\NormalTok{(}\StringTok{"./Templates/TemplateRasters/LV10m\_10km.tif"}\NormalTok{)}
\NormalTok{rastrs10}\OtherTok{=}\FunctionTok{raster}\NormalTok{(template10)}

\NormalTok{nulls10}\OtherTok{=}\FunctionTok{rast}\NormalTok{(}\StringTok{"./Templates/TemplateRasters/nulls\_LV10m\_10km.tif"}\NormalTok{)}
\NormalTok{nulls100}\OtherTok{=}\FunctionTok{rast}\NormalTok{(}\StringTok{"./Templates/TemplateRasters/nulls\_LV100m\_10km.tif"}\NormalTok{)}

\CommentTok{\# codes {-}{-}{-}{-}}
\NormalTok{kodi}\OtherTok{=}\FunctionTok{read\_excel}\NormalTok{(}\StringTok{"./Geodata/2024/LAD/KulturuKodi\_2024.xlsx"}\NormalTok{)}
\NormalTok{kodi}\SpecialCharTok{$}\NormalTok{kods}\OtherTok{=}\FunctionTok{as.character}\NormalTok{(kodi}\SpecialCharTok{$}\NormalTok{kods)}
\CommentTok{\# LAD {-}{-}{-}{-}}
\NormalTok{lad}\OtherTok{=}\NormalTok{sfarrow}\SpecialCharTok{::}\FunctionTok{st\_read\_parquet}\NormalTok{(}\StringTok{"./Geodata/2024/LAD/Lauki\_2024.parquet"}\NormalTok{)}
\NormalTok{lad}\SpecialCharTok{$}\NormalTok{yes}\OtherTok{=}\DecValTok{1}
\NormalTok{lad}\OtherTok{=}\NormalTok{lad }\SpecialCharTok{\%\textgreater{}\%} 
 \FunctionTok{left\_join}\NormalTok{(kodi,}\AttributeTok{by=}\FunctionTok{c}\NormalTok{(}\StringTok{"PRODUCT\_CODE"}\OtherTok{=}\StringTok{"kods"}\NormalTok{))}

\CommentTok{\# simple landscape {-}{-}{-}{-}}
\NormalTok{simple\_landscape}\OtherTok{=}\FunctionTok{rast}\NormalTok{(}\StringTok{"RasterGrids\_10m/2024/Ainava\_vienk\_mask.tif"}\NormalTok{)}


\CommentTok{\# FarmlandCrops\_CropsSpring\_cell.tif    egv\_195 {-}{-}{-}{-}}
\NormalTok{dati}\OtherTok{=}\NormalTok{lad }\SpecialCharTok{\%\textgreater{}\%} 
 \FunctionTok{filter}\NormalTok{(}\FunctionTok{str\_detect}\NormalTok{(SDM\_grupa\_sakums,}\StringTok{"labība{-}vasarāji"}\NormalTok{))}
\FunctionTok{table}\NormalTok{(dati}\SpecialCharTok{$}\NormalTok{SDM\_grupa\_sakums,}\AttributeTok{useNA=}\StringTok{"always"}\NormalTok{)}


\NormalTok{p2i\_rez}\OtherTok{=}\NormalTok{egvtools}\SpecialCharTok{::}\FunctionTok{polygon2input}\NormalTok{(}\AttributeTok{vector\_data =}\NormalTok{ dati,}
                \AttributeTok{template\_path =} \StringTok{"./Templates/TemplateRasters/LV10m\_10km.tif"}\NormalTok{,}
                \AttributeTok{out\_path =} \StringTok{"./RasterGrids\_10m/2024/"}\NormalTok{,}
                \AttributeTok{file\_name =} \StringTok{"FarmlandCrops\_CropsSpring\_input.tif"}\NormalTok{,}
                \AttributeTok{value\_field =} \StringTok{"yes"}\NormalTok{,}
                \AttributeTok{prepare=}\ConstantTok{FALSE}\NormalTok{,}
                \AttributeTok{background\_raster =} \StringTok{"./Templates/TemplateRasters/nulls\_LV10m\_10km.tif"}\NormalTok{,}
                \AttributeTok{plot\_result =} \ConstantTok{TRUE}\NormalTok{)}
\NormalTok{p2i\_rez}
\NormalTok{i2e\_rez}\OtherTok{=}\NormalTok{egvtools}\SpecialCharTok{::}\FunctionTok{input2egv}\NormalTok{(}\AttributeTok{input=}\FunctionTok{paste0}\NormalTok{(}\StringTok{"./RasterGrids\_10m/2024/"}\NormalTok{,}
                     \StringTok{"FarmlandCrops\_CropsSpring\_input.tif"}\NormalTok{),}
              \AttributeTok{egv\_template=} \StringTok{"./Templates/TemplateRasters/LV100m\_10km.tif"}\NormalTok{,}
              \AttributeTok{summary\_function =} \StringTok{"average"}\NormalTok{,}
              \AttributeTok{missing\_job =} \StringTok{"FillOutput"}\NormalTok{,}
              \AttributeTok{outlocation =} \StringTok{"./RasterGrids\_100m/2024/RAW/"}\NormalTok{,}
              \AttributeTok{outfilename =} \StringTok{"FarmlandCrops\_CropsSpring\_cell.tif"}\NormalTok{,}
              \AttributeTok{layername =} \StringTok{"egv\_195"}\NormalTok{,}
              \AttributeTok{idw\_weight =} \DecValTok{2}\NormalTok{,}
              \AttributeTok{plot\_gaps =} \ConstantTok{FALSE}\NormalTok{,}\AttributeTok{plot\_final =} \ConstantTok{TRUE}\NormalTok{)}
\NormalTok{i2e\_rez}
\FunctionTok{rm}\NormalTok{(p2i\_rez)}
\FunctionTok{rm}\NormalTok{(i2e\_rez)}
\FunctionTok{rm}\NormalTok{(dati)}
\FunctionTok{unlink}\NormalTok{(}\StringTok{"./RasterGrids\_10m/2024/FarmlandCrops\_CropsSpring\_input.tif"}\NormalTok{)}


\CommentTok{\# standardisation {-}{-}{-}{-}}
\ControlFlowTok{if}\NormalTok{(}\SpecialCharTok{!}\FunctionTok{require}\NormalTok{(terra)) \{}\FunctionTok{install.packages}\NormalTok{(}\StringTok{"terra"}\NormalTok{); }\FunctionTok{require}\NormalTok{(terra)\}}
\ControlFlowTok{if}\NormalTok{(}\SpecialCharTok{!}\FunctionTok{require}\NormalTok{(tidyverse)) \{}\FunctionTok{install.packages}\NormalTok{(}\StringTok{"tidyverse"}\NormalTok{); }\FunctionTok{require}\NormalTok{(tidyverse)\}}

\NormalTok{nosaukums}\OtherTok{=}\StringTok{"FarmlandCrops\_CropsSpring\_cell.tif"}
\NormalTok{ielasisanas\_cels}\OtherTok{=}\FunctionTok{paste0}\NormalTok{(}\StringTok{"./RasterGrids\_100m/2024/RAW/"}\NormalTok{,nosaukums)}
\NormalTok{saglabasanas\_cels}\OtherTok{=}\FunctionTok{paste0}\NormalTok{(}\StringTok{"./RasterGrids\_100m/2024/Scaled/"}\NormalTok{,nosaukums)}
\NormalTok{slanis}\OtherTok{=}\FunctionTok{rast}\NormalTok{(ielasisanas\_cels)}
\NormalTok{videjais}\OtherTok{=}\FunctionTok{global}\NormalTok{(slanis,}\AttributeTok{fun=}\StringTok{"mean"}\NormalTok{,}\AttributeTok{na.rm=}\ConstantTok{TRUE}\NormalTok{)}
\NormalTok{centrets}\OtherTok{=}\NormalTok{slanis}\SpecialCharTok{{-}}\NormalTok{videjais[,}\DecValTok{1}\NormalTok{]}
\NormalTok{standartnovirze}\OtherTok{=}\NormalTok{terra}\SpecialCharTok{::}\FunctionTok{global}\NormalTok{(centrets,}\AttributeTok{fun=}\StringTok{"rms"}\NormalTok{,}\AttributeTok{na.rm=}\ConstantTok{TRUE}\NormalTok{)}
\NormalTok{merogots}\OtherTok{=}\NormalTok{centrets}\SpecialCharTok{/}\NormalTok{standartnovirze[,}\DecValTok{1}\NormalTok{]}
\FunctionTok{writeRaster}\NormalTok{(merogots,}
      \AttributeTok{filename=}\NormalTok{saglabasanas\_cels,}
      \AttributeTok{overwrite=}\ConstantTok{TRUE}\NormalTok{)}
\end{Highlighting}
\end{Shaded}

\section{FarmlandCrops\_CropsSpring\_r500}\label{ch06.196}

\textbf{filename:} \texttt{FarmlandCrops\_CropsSpring\_r500.tif}

\textbf{layername:} \texttt{egv\_196}

\textbf{English name:} Fractional cover of Spring Sown Crops within the 0.5 km
landscape

\textbf{Latvian name:} Vasarāju aramzemēs platības īpatsvars 0,5 km ainavā

\textbf{Procedure:} The cover fraction within a radius of 500 m around the analysis grid cell is
calculated as the area-weighted sum of the \hyperref[ch06.195]{analysis cells} inside the
buffer, using the workflow \texttt{egvtools::radius\_function()}. During the calculation of the landscape metric,
inverse distance weighted (power = 2) gap filling on the output is applied
to ensure no missing values at the edges. Then the layer is rewritten to set
its name. Finally, the layer is standardised by subtracting the arithmetic
mean and dividing by the root mean squared error.

\begin{Shaded}
\begin{Highlighting}[]
\CommentTok{\# libs {-}{-}{-}{-}}
\ControlFlowTok{if}\NormalTok{(}\SpecialCharTok{!}\FunctionTok{require}\NormalTok{(terra)) \{}\FunctionTok{install.packages}\NormalTok{(}\StringTok{"terra"}\NormalTok{); }\FunctionTok{require}\NormalTok{(terra)\}}
\ControlFlowTok{if}\NormalTok{(}\SpecialCharTok{!}\FunctionTok{require}\NormalTok{(egvtools)) \{remotes}\SpecialCharTok{::}\FunctionTok{install\_github}\NormalTok{(}\StringTok{"aavotins/egvtools"}\NormalTok{); }\FunctionTok{require}\NormalTok{(egvtools)\}}


\CommentTok{\# Templates {-}{-}{-}{-}{-}}
\NormalTok{template100}\OtherTok{=}\FunctionTok{rast}\NormalTok{(}\StringTok{"./Templates/TemplateRasters/LV100m\_10km.tif"}\NormalTok{)}

\CommentTok{\# radii {-}{-}{-}{-}}
\FunctionTok{radius\_function}\NormalTok{(}
 \AttributeTok{kvadrati\_path =} \StringTok{"./Templates/TemplateGrids/tiles/"}\NormalTok{,}
 \AttributeTok{radii\_path   =} \StringTok{"./Templates/TemplateGridPoints/tiles/"}\NormalTok{,}
 \AttributeTok{tikls100\_path =} \StringTok{"./Templates/TemplateGrids/tikls100\_sauzeme.parquet"}\NormalTok{,}
 \AttributeTok{template\_path =} \StringTok{"./Templates/TemplateRasters/LV100m\_10km.tif"}\NormalTok{,}
 \AttributeTok{input\_layers  =} \FunctionTok{c}\NormalTok{(}\StringTok{"./RasterGrids\_100m/2024/RAW/FarmlandCrops\_CropsSpring\_cell.tif"}\NormalTok{),}
 \AttributeTok{layer\_prefixes =} \FunctionTok{c}\NormalTok{(}\StringTok{"FarmlandCrops\_CropsSpring"}\NormalTok{),}
 \AttributeTok{output\_dir   =} \StringTok{"./RasterGrids\_100m/2024/RAW/"}\NormalTok{,}
 \AttributeTok{n\_workers   =} \DecValTok{6}\NormalTok{,}
 \AttributeTok{radii     =} \FunctionTok{c}\NormalTok{(}\StringTok{"r500"}\NormalTok{),}
 \AttributeTok{radius\_mode  =} \StringTok{"sparse"}\NormalTok{,}
 \AttributeTok{extract\_fun  =} \StringTok{"mean"}\NormalTok{,}
 \AttributeTok{fill\_missing  =} \ConstantTok{TRUE}\NormalTok{,}
 \AttributeTok{IDW\_weight   =} \DecValTok{2}\NormalTok{,}
 \AttributeTok{future\_max\_size =} \DecValTok{40} \SpecialCharTok{*} \DecValTok{1024}\SpecialCharTok{\^{}}\DecValTok{3}\NormalTok{)}


\CommentTok{\# FarmlandCrops\_CropsSpring\_r500.tif    egv\_196 {-}{-}{-}{-}}
\NormalTok{slanis}\OtherTok{=}\FunctionTok{rast}\NormalTok{(}\StringTok{"./RasterGrids\_100m/2024/RAW/FarmlandCrops\_CropsSpring\_r500.tif"}\NormalTok{)}
\FunctionTok{names}\NormalTok{(slanis)}\OtherTok{=}\StringTok{"egv\_196"}
\NormalTok{slanis2}\OtherTok{=}\FunctionTok{project}\NormalTok{(slanis,template100)}
\FunctionTok{writeRaster}\NormalTok{(slanis2,}
      \StringTok{"./RasterGrids\_100m/2024/RAW/FarmlandCrops\_CropsSpring\_r500.tif"}\NormalTok{,}
      \AttributeTok{overwrite=}\ConstantTok{TRUE}\NormalTok{)}

\CommentTok{\# standardisation {-}{-}{-}{-}}
\ControlFlowTok{if}\NormalTok{(}\SpecialCharTok{!}\FunctionTok{require}\NormalTok{(terra)) \{}\FunctionTok{install.packages}\NormalTok{(}\StringTok{"terra"}\NormalTok{); }\FunctionTok{require}\NormalTok{(terra)\}}
\ControlFlowTok{if}\NormalTok{(}\SpecialCharTok{!}\FunctionTok{require}\NormalTok{(tidyverse)) \{}\FunctionTok{install.packages}\NormalTok{(}\StringTok{"tidyverse"}\NormalTok{); }\FunctionTok{require}\NormalTok{(tidyverse)\}}

\NormalTok{nosaukums}\OtherTok{=}\StringTok{"FarmlandCrops\_CropsSpring\_r500.tif"}
\NormalTok{ielasisanas\_cels}\OtherTok{=}\FunctionTok{paste0}\NormalTok{(}\StringTok{"./RasterGrids\_100m/2024/RAW/"}\NormalTok{,nosaukums)}
\NormalTok{saglabasanas\_cels}\OtherTok{=}\FunctionTok{paste0}\NormalTok{(}\StringTok{"./RasterGrids\_100m/2024/Scaled/"}\NormalTok{,nosaukums)}
\NormalTok{slanis}\OtherTok{=}\FunctionTok{rast}\NormalTok{(ielasisanas\_cels)}
\NormalTok{videjais}\OtherTok{=}\FunctionTok{global}\NormalTok{(slanis,}\AttributeTok{fun=}\StringTok{"mean"}\NormalTok{,}\AttributeTok{na.rm=}\ConstantTok{TRUE}\NormalTok{)}
\NormalTok{centrets}\OtherTok{=}\NormalTok{slanis}\SpecialCharTok{{-}}\NormalTok{videjais[,}\DecValTok{1}\NormalTok{]}
\NormalTok{standartnovirze}\OtherTok{=}\NormalTok{terra}\SpecialCharTok{::}\FunctionTok{global}\NormalTok{(centrets,}\AttributeTok{fun=}\StringTok{"rms"}\NormalTok{,}\AttributeTok{na.rm=}\ConstantTok{TRUE}\NormalTok{)}
\NormalTok{merogots}\OtherTok{=}\NormalTok{centrets}\SpecialCharTok{/}\NormalTok{standartnovirze[,}\DecValTok{1}\NormalTok{]}
\FunctionTok{writeRaster}\NormalTok{(merogots,}
      \AttributeTok{filename=}\NormalTok{saglabasanas\_cels,}
      \AttributeTok{overwrite=}\ConstantTok{TRUE}\NormalTok{)}
\end{Highlighting}
\end{Shaded}

\section{FarmlandCrops\_CropsSpring\_r1250}\label{ch06.197}

\textbf{filename:} \texttt{FarmlandCrops\_CropsSpring\_r1250.tif}

\textbf{layername:} \texttt{egv\_197}

\textbf{English name:} Fractional cover of Spring Sown Crops within the 1.25 km
landscape

\textbf{Latvian name:} Vasarāju aramzemēs platības īpatsvars 1,25 km ainavā

\textbf{Procedure:} The cover fraction within a radius of 1250 m around the analysis grid cell
is calculated as the area-weighted sum of the \hyperref[ch06.195]{analysis cells} inside
the buffer, using the workflow \texttt{egvtools::radius\_function()}. During the calculation of the landscape
metric, inverse distance weighted (power = 2) gap filling on the output is
applied to ensure no missing values at the edges. Then the layer is
rewritten to set its name. Finally, the layer is standardised by
subtracting the arithmetic mean and dividing by the root mean squared error.

\begin{Shaded}
\begin{Highlighting}[]
\CommentTok{\# libs {-}{-}{-}{-}}
\ControlFlowTok{if}\NormalTok{(}\SpecialCharTok{!}\FunctionTok{require}\NormalTok{(terra)) \{}\FunctionTok{install.packages}\NormalTok{(}\StringTok{"terra"}\NormalTok{); }\FunctionTok{require}\NormalTok{(terra)\}}
\ControlFlowTok{if}\NormalTok{(}\SpecialCharTok{!}\FunctionTok{require}\NormalTok{(egvtools)) \{remotes}\SpecialCharTok{::}\FunctionTok{install\_github}\NormalTok{(}\StringTok{"aavotins/egvtools"}\NormalTok{); }\FunctionTok{require}\NormalTok{(egvtools)\}}


\CommentTok{\# Templates {-}{-}{-}{-}{-}}
\NormalTok{template100}\OtherTok{=}\FunctionTok{rast}\NormalTok{(}\StringTok{"./Templates/TemplateRasters/LV100m\_10km.tif"}\NormalTok{)}

\CommentTok{\# radii {-}{-}{-}{-}}
\FunctionTok{radius\_function}\NormalTok{(}
 \AttributeTok{kvadrati\_path =} \StringTok{"./Templates/TemplateGrids/tiles/"}\NormalTok{,}
 \AttributeTok{radii\_path   =} \StringTok{"./Templates/TemplateGridPoints/tiles/"}\NormalTok{,}
 \AttributeTok{tikls100\_path =} \StringTok{"./Templates/TemplateGrids/tikls100\_sauzeme.parquet"}\NormalTok{,}
 \AttributeTok{template\_path =} \StringTok{"./Templates/TemplateRasters/LV100m\_10km.tif"}\NormalTok{,}
 \AttributeTok{input\_layers  =} \FunctionTok{c}\NormalTok{(}\StringTok{"./RasterGrids\_100m/2024/RAW/FarmlandCrops\_CropsSpring\_cell.tif"}\NormalTok{),}
 \AttributeTok{layer\_prefixes =} \FunctionTok{c}\NormalTok{(}\StringTok{"FarmlandCrops\_CropsSpring"}\NormalTok{),}
 \AttributeTok{output\_dir   =} \StringTok{"./RasterGrids\_100m/2024/RAW/"}\NormalTok{,}
 \AttributeTok{n\_workers   =} \DecValTok{6}\NormalTok{,}
 \AttributeTok{radii     =} \FunctionTok{c}\NormalTok{(}\StringTok{"r1250"}\NormalTok{),}
 \AttributeTok{radius\_mode  =} \StringTok{"sparse"}\NormalTok{,}
 \AttributeTok{extract\_fun  =} \StringTok{"mean"}\NormalTok{,}
 \AttributeTok{fill\_missing  =} \ConstantTok{TRUE}\NormalTok{,}
 \AttributeTok{IDW\_weight   =} \DecValTok{2}\NormalTok{,}
 \AttributeTok{future\_max\_size =} \DecValTok{40} \SpecialCharTok{*} \DecValTok{1024}\SpecialCharTok{\^{}}\DecValTok{3}\NormalTok{)}


\CommentTok{\# FarmlandCrops\_CropsSpring\_r1250.tif   egv\_197 {-}{-}{-}{-}}
\NormalTok{slanis}\OtherTok{=}\FunctionTok{rast}\NormalTok{(}\StringTok{"./RasterGrids\_100m/2024/RAW/FarmlandCrops\_CropsSpring\_r1250.tif"}\NormalTok{)}
\FunctionTok{names}\NormalTok{(slanis)}\OtherTok{=}\StringTok{"egv\_197"}
\NormalTok{slanis2}\OtherTok{=}\FunctionTok{project}\NormalTok{(slanis,template100)}
\FunctionTok{writeRaster}\NormalTok{(slanis2,}
      \StringTok{"./RasterGrids\_100m/2024/RAW/FarmlandCrops\_CropsSpring\_r1250.tif"}\NormalTok{,}
      \AttributeTok{overwrite=}\ConstantTok{TRUE}\NormalTok{)}

\CommentTok{\# standardisation {-}{-}{-}{-}}
\ControlFlowTok{if}\NormalTok{(}\SpecialCharTok{!}\FunctionTok{require}\NormalTok{(terra)) \{}\FunctionTok{install.packages}\NormalTok{(}\StringTok{"terra"}\NormalTok{); }\FunctionTok{require}\NormalTok{(terra)\}}
\ControlFlowTok{if}\NormalTok{(}\SpecialCharTok{!}\FunctionTok{require}\NormalTok{(tidyverse)) \{}\FunctionTok{install.packages}\NormalTok{(}\StringTok{"tidyverse"}\NormalTok{); }\FunctionTok{require}\NormalTok{(tidyverse)\}}

\NormalTok{nosaukums}\OtherTok{=}\StringTok{"FarmlandCrops\_CropsSpring\_r1250.tif"}
\NormalTok{ielasisanas\_cels}\OtherTok{=}\FunctionTok{paste0}\NormalTok{(}\StringTok{"./RasterGrids\_100m/2024/RAW/"}\NormalTok{,nosaukums)}
\NormalTok{saglabasanas\_cels}\OtherTok{=}\FunctionTok{paste0}\NormalTok{(}\StringTok{"./RasterGrids\_100m/2024/Scaled/"}\NormalTok{,nosaukums)}
\NormalTok{slanis}\OtherTok{=}\FunctionTok{rast}\NormalTok{(ielasisanas\_cels)}
\NormalTok{videjais}\OtherTok{=}\FunctionTok{global}\NormalTok{(slanis,}\AttributeTok{fun=}\StringTok{"mean"}\NormalTok{,}\AttributeTok{na.rm=}\ConstantTok{TRUE}\NormalTok{)}
\NormalTok{centrets}\OtherTok{=}\NormalTok{slanis}\SpecialCharTok{{-}}\NormalTok{videjais[,}\DecValTok{1}\NormalTok{]}
\NormalTok{standartnovirze}\OtherTok{=}\NormalTok{terra}\SpecialCharTok{::}\FunctionTok{global}\NormalTok{(centrets,}\AttributeTok{fun=}\StringTok{"rms"}\NormalTok{,}\AttributeTok{na.rm=}\ConstantTok{TRUE}\NormalTok{)}
\NormalTok{merogots}\OtherTok{=}\NormalTok{centrets}\SpecialCharTok{/}\NormalTok{standartnovirze[,}\DecValTok{1}\NormalTok{]}
\FunctionTok{writeRaster}\NormalTok{(merogots,}
      \AttributeTok{filename=}\NormalTok{saglabasanas\_cels,}
      \AttributeTok{overwrite=}\ConstantTok{TRUE}\NormalTok{)}
\end{Highlighting}
\end{Shaded}

\section{FarmlandCrops\_CropsSpring\_r3000}\label{ch06.198}

\textbf{filename:} \texttt{FarmlandCrops\_CropsSpring\_r3000.tif}

\textbf{layername:} \texttt{egv\_198}

\textbf{English name:} Fractional cover of Spring Sown Crops within the 3 km
landscape

\textbf{Latvian name:} Vasarāju aramzemēs platības īpatsvars 3 km ainavā

\textbf{Procedure:} The cover fraction within a radius of 3000 m around the analysis grid cell
is calculated as the area-weighted sum of the \hyperref[ch06.195]{analysis cells} inside
the buffer, using the workflow \texttt{egvtools::radius\_function()}. During the calculation of the landscape
metric, inverse distance weighted (power = 2) gap filling on the output is
applied to ensure no missing values at the edges. Then the layer is
rewritten to set its name. Finally, the layer is standardised by
subtracting the arithmetic mean and dividing by the root mean squared error.

\begin{Shaded}
\begin{Highlighting}[]
\CommentTok{\# libs {-}{-}{-}{-}}
\ControlFlowTok{if}\NormalTok{(}\SpecialCharTok{!}\FunctionTok{require}\NormalTok{(terra)) \{}\FunctionTok{install.packages}\NormalTok{(}\StringTok{"terra"}\NormalTok{); }\FunctionTok{require}\NormalTok{(terra)\}}
\ControlFlowTok{if}\NormalTok{(}\SpecialCharTok{!}\FunctionTok{require}\NormalTok{(egvtools)) \{remotes}\SpecialCharTok{::}\FunctionTok{install\_github}\NormalTok{(}\StringTok{"aavotins/egvtools"}\NormalTok{); }\FunctionTok{require}\NormalTok{(egvtools)\}}


\CommentTok{\# Templates {-}{-}{-}{-}{-}}
\NormalTok{template100}\OtherTok{=}\FunctionTok{rast}\NormalTok{(}\StringTok{"./Templates/TemplateRasters/LV100m\_10km.tif"}\NormalTok{)}

\CommentTok{\# radii {-}{-}{-}{-}}
\FunctionTok{radius\_function}\NormalTok{(}
 \AttributeTok{kvadrati\_path =} \StringTok{"./Templates/TemplateGrids/tiles/"}\NormalTok{,}
 \AttributeTok{radii\_path   =} \StringTok{"./Templates/TemplateGridPoints/tiles/"}\NormalTok{,}
 \AttributeTok{tikls100\_path =} \StringTok{"./Templates/TemplateGrids/tikls100\_sauzeme.parquet"}\NormalTok{,}
 \AttributeTok{template\_path =} \StringTok{"./Templates/TemplateRasters/LV100m\_10km.tif"}\NormalTok{,}
 \AttributeTok{input\_layers  =} \FunctionTok{c}\NormalTok{(}\StringTok{"./RasterGrids\_100m/2024/RAW/FarmlandCrops\_CropsSpring\_cell.tif"}\NormalTok{),}
 \AttributeTok{layer\_prefixes =} \FunctionTok{c}\NormalTok{(}\StringTok{"FarmlandCrops\_CropsSpring"}\NormalTok{),}
 \AttributeTok{output\_dir   =} \StringTok{"./RasterGrids\_100m/2024/RAW/"}\NormalTok{,}
 \AttributeTok{n\_workers   =} \DecValTok{6}\NormalTok{,}
 \AttributeTok{radii     =} \FunctionTok{c}\NormalTok{(}\StringTok{"r3000"}\NormalTok{),}
 \AttributeTok{radius\_mode  =} \StringTok{"sparse"}\NormalTok{,}
 \AttributeTok{extract\_fun  =} \StringTok{"mean"}\NormalTok{,}
 \AttributeTok{fill\_missing  =} \ConstantTok{TRUE}\NormalTok{,}
 \AttributeTok{IDW\_weight   =} \DecValTok{2}\NormalTok{,}
 \AttributeTok{future\_max\_size =} \DecValTok{40} \SpecialCharTok{*} \DecValTok{1024}\SpecialCharTok{\^{}}\DecValTok{3}\NormalTok{)}


\CommentTok{\# FarmlandCrops\_CropsSpring\_r3000.tif   egv\_198 {-}{-}{-}{-}}
\NormalTok{slanis}\OtherTok{=}\FunctionTok{rast}\NormalTok{(}\StringTok{"./RasterGrids\_100m/2024/RAW/FarmlandCrops\_CropsSpring\_r3000.tif"}\NormalTok{)}
\FunctionTok{names}\NormalTok{(slanis)}\OtherTok{=}\StringTok{"egv\_198"}
\NormalTok{slanis2}\OtherTok{=}\FunctionTok{project}\NormalTok{(slanis,template100)}
\FunctionTok{writeRaster}\NormalTok{(slanis2,}
      \StringTok{"./RasterGrids\_100m/2024/RAW/FarmlandCrops\_CropsSpring\_r3000.tif"}\NormalTok{,}
      \AttributeTok{overwrite=}\ConstantTok{TRUE}\NormalTok{)}

\CommentTok{\# standardisation {-}{-}{-}{-}}
\ControlFlowTok{if}\NormalTok{(}\SpecialCharTok{!}\FunctionTok{require}\NormalTok{(terra)) \{}\FunctionTok{install.packages}\NormalTok{(}\StringTok{"terra"}\NormalTok{); }\FunctionTok{require}\NormalTok{(terra)\}}
\ControlFlowTok{if}\NormalTok{(}\SpecialCharTok{!}\FunctionTok{require}\NormalTok{(tidyverse)) \{}\FunctionTok{install.packages}\NormalTok{(}\StringTok{"tidyverse"}\NormalTok{); }\FunctionTok{require}\NormalTok{(tidyverse)\}}

\NormalTok{nosaukums}\OtherTok{=}\StringTok{"FarmlandCrops\_CropsSpring\_r3000.tif"}
\NormalTok{ielasisanas\_cels}\OtherTok{=}\FunctionTok{paste0}\NormalTok{(}\StringTok{"./RasterGrids\_100m/2024/RAW/"}\NormalTok{,nosaukums)}
\NormalTok{saglabasanas\_cels}\OtherTok{=}\FunctionTok{paste0}\NormalTok{(}\StringTok{"./RasterGrids\_100m/2024/Scaled/"}\NormalTok{,nosaukums)}
\NormalTok{slanis}\OtherTok{=}\FunctionTok{rast}\NormalTok{(ielasisanas\_cels)}
\NormalTok{videjais}\OtherTok{=}\FunctionTok{global}\NormalTok{(slanis,}\AttributeTok{fun=}\StringTok{"mean"}\NormalTok{,}\AttributeTok{na.rm=}\ConstantTok{TRUE}\NormalTok{)}
\NormalTok{centrets}\OtherTok{=}\NormalTok{slanis}\SpecialCharTok{{-}}\NormalTok{videjais[,}\DecValTok{1}\NormalTok{]}
\NormalTok{standartnovirze}\OtherTok{=}\NormalTok{terra}\SpecialCharTok{::}\FunctionTok{global}\NormalTok{(centrets,}\AttributeTok{fun=}\StringTok{"rms"}\NormalTok{,}\AttributeTok{na.rm=}\ConstantTok{TRUE}\NormalTok{)}
\NormalTok{merogots}\OtherTok{=}\NormalTok{centrets}\SpecialCharTok{/}\NormalTok{standartnovirze[,}\DecValTok{1}\NormalTok{]}
\FunctionTok{writeRaster}\NormalTok{(merogots,}
      \AttributeTok{filename=}\NormalTok{saglabasanas\_cels,}
      \AttributeTok{overwrite=}\ConstantTok{TRUE}\NormalTok{)}
\end{Highlighting}
\end{Shaded}

\section{FarmlandCrops\_CropsSpring\_r10000}\label{ch06.199}

\textbf{filename:} \texttt{FarmlandCrops\_CropsSpring\_r10000.tif}

\textbf{layername:} \texttt{egv\_199}

\textbf{English name:} Fractional cover of Spring Sown Crops within the 10 km
landscape

\textbf{Latvian name:} Vasarāju aramzemēs platības īpatsvars 10 km ainavā

\textbf{Procedure:} The cover fraction within a radius of 10000 m around the analysis grid cell
is calculated as the area-weighted sum of the \hyperref[ch06.195]{analysis cells} inside
the buffer, using the workflow \texttt{egvtools::radius\_function()}. During the calculation of the landscape
metric, inverse distance weighted (power = 2) gap filling on the output is
applied to ensure no missing values at the edges. Then the layer is
rewritten to set its name. Finally, the layer is standardised by
subtracting the arithmetic mean and dividing by the root mean squared error.

\begin{Shaded}
\begin{Highlighting}[]
\CommentTok{\# libs {-}{-}{-}{-}}
\ControlFlowTok{if}\NormalTok{(}\SpecialCharTok{!}\FunctionTok{require}\NormalTok{(terra)) \{}\FunctionTok{install.packages}\NormalTok{(}\StringTok{"terra"}\NormalTok{); }\FunctionTok{require}\NormalTok{(terra)\}}
\ControlFlowTok{if}\NormalTok{(}\SpecialCharTok{!}\FunctionTok{require}\NormalTok{(egvtools)) \{remotes}\SpecialCharTok{::}\FunctionTok{install\_github}\NormalTok{(}\StringTok{"aavotins/egvtools"}\NormalTok{); }\FunctionTok{require}\NormalTok{(egvtools)\}}


\CommentTok{\# Templates {-}{-}{-}{-}{-}}
\NormalTok{template100}\OtherTok{=}\FunctionTok{rast}\NormalTok{(}\StringTok{"./Templates/TemplateRasters/LV100m\_10km.tif"}\NormalTok{)}

\CommentTok{\# radii {-}{-}{-}{-}}
\FunctionTok{radius\_function}\NormalTok{(}
 \AttributeTok{kvadrati\_path =} \StringTok{"./Templates/TemplateGrids/tiles/"}\NormalTok{,}
 \AttributeTok{radii\_path   =} \StringTok{"./Templates/TemplateGridPoints/tiles/"}\NormalTok{,}
 \AttributeTok{tikls100\_path =} \StringTok{"./Templates/TemplateGrids/tikls100\_sauzeme.parquet"}\NormalTok{,}
 \AttributeTok{template\_path =} \StringTok{"./Templates/TemplateRasters/LV100m\_10km.tif"}\NormalTok{,}
 \AttributeTok{input\_layers  =} \FunctionTok{c}\NormalTok{(}\StringTok{"./RasterGrids\_100m/2024/RAW/FarmlandCrops\_CropsSpring\_cell.tif"}\NormalTok{),}
 \AttributeTok{layer\_prefixes =} \FunctionTok{c}\NormalTok{(}\StringTok{"FarmlandCrops\_CropsSpring"}\NormalTok{),}
 \AttributeTok{output\_dir   =} \StringTok{"./RasterGrids\_100m/2024/RAW/"}\NormalTok{,}
 \AttributeTok{n\_workers   =} \DecValTok{6}\NormalTok{,}
 \AttributeTok{radii     =} \FunctionTok{c}\NormalTok{(}\StringTok{"r10000"}\NormalTok{),}
 \AttributeTok{radius\_mode  =} \StringTok{"sparse"}\NormalTok{,}
 \AttributeTok{extract\_fun  =} \StringTok{"mean"}\NormalTok{,}
 \AttributeTok{fill\_missing  =} \ConstantTok{TRUE}\NormalTok{,}
 \AttributeTok{IDW\_weight   =} \DecValTok{2}\NormalTok{,}
 \AttributeTok{future\_max\_size =} \DecValTok{40} \SpecialCharTok{*} \DecValTok{1024}\SpecialCharTok{\^{}}\DecValTok{3}\NormalTok{)}


\CommentTok{\# FarmlandCrops\_CropsSpring\_r10000.tif  egv\_199 {-}{-}{-}{-}}
\NormalTok{slanis}\OtherTok{=}\FunctionTok{rast}\NormalTok{(}\StringTok{"./RasterGrids\_100m/2024/RAW/FarmlandCrops\_CropsSpring\_r10000.tif"}\NormalTok{)}
\FunctionTok{names}\NormalTok{(slanis)}\OtherTok{=}\StringTok{"egv\_199"}
\NormalTok{slanis2}\OtherTok{=}\FunctionTok{project}\NormalTok{(slanis,template100)}
\FunctionTok{writeRaster}\NormalTok{(slanis2,}
      \StringTok{"./RasterGrids\_100m/2024/RAW/FarmlandCrops\_CropsSpring\_r10000.tif"}\NormalTok{,}
      \AttributeTok{overwrite=}\ConstantTok{TRUE}\NormalTok{)}

\CommentTok{\# standardisation {-}{-}{-}{-}}
\ControlFlowTok{if}\NormalTok{(}\SpecialCharTok{!}\FunctionTok{require}\NormalTok{(terra)) \{}\FunctionTok{install.packages}\NormalTok{(}\StringTok{"terra"}\NormalTok{); }\FunctionTok{require}\NormalTok{(terra)\}}
\ControlFlowTok{if}\NormalTok{(}\SpecialCharTok{!}\FunctionTok{require}\NormalTok{(tidyverse)) \{}\FunctionTok{install.packages}\NormalTok{(}\StringTok{"tidyverse"}\NormalTok{); }\FunctionTok{require}\NormalTok{(tidyverse)\}}

\NormalTok{nosaukums}\OtherTok{=}\StringTok{"FarmlandCrops\_CropsSpring\_r10000.tif"}
\NormalTok{ielasisanas\_cels}\OtherTok{=}\FunctionTok{paste0}\NormalTok{(}\StringTok{"./RasterGrids\_100m/2024/RAW/"}\NormalTok{,nosaukums)}
\NormalTok{saglabasanas\_cels}\OtherTok{=}\FunctionTok{paste0}\NormalTok{(}\StringTok{"./RasterGrids\_100m/2024/Scaled/"}\NormalTok{,nosaukums)}
\NormalTok{slanis}\OtherTok{=}\FunctionTok{rast}\NormalTok{(ielasisanas\_cels)}
\NormalTok{videjais}\OtherTok{=}\FunctionTok{global}\NormalTok{(slanis,}\AttributeTok{fun=}\StringTok{"mean"}\NormalTok{,}\AttributeTok{na.rm=}\ConstantTok{TRUE}\NormalTok{)}
\NormalTok{centrets}\OtherTok{=}\NormalTok{slanis}\SpecialCharTok{{-}}\NormalTok{videjais[,}\DecValTok{1}\NormalTok{]}
\NormalTok{standartnovirze}\OtherTok{=}\NormalTok{terra}\SpecialCharTok{::}\FunctionTok{global}\NormalTok{(centrets,}\AttributeTok{fun=}\StringTok{"rms"}\NormalTok{,}\AttributeTok{na.rm=}\ConstantTok{TRUE}\NormalTok{)}
\NormalTok{merogots}\OtherTok{=}\NormalTok{centrets}\SpecialCharTok{/}\NormalTok{standartnovirze[,}\DecValTok{1}\NormalTok{]}
\FunctionTok{writeRaster}\NormalTok{(merogots,}
      \AttributeTok{filename=}\NormalTok{saglabasanas\_cels,}
      \AttributeTok{overwrite=}\ConstantTok{TRUE}\NormalTok{)}
\end{Highlighting}
\end{Shaded}

\section{FarmlandCrops\_CropsWinter\_cell}\label{ch06.200}

\textbf{filename:} \texttt{FarmlandCrops\_CropsWinter\_cell.tif}

\textbf{layername:} \texttt{egv\_200}

\textbf{English name:} Fractional cover of Winter Crops within the analysis cell (1
ha)

\textbf{Latvian name:} Ziemāju aramzemēs platības īpatsvars analīzes šūnā (1 ha)

\textbf{Procedure:} First, agricultural parcels declared as winter crops are
selected from the \hyperref[Ch04.02]{Rural Support Service's information on declared
fields}. These geometries are then rasterised to input resolution,
ensuring value 1 at the polygon locations and value 0 elsewhere. Rasterisation is performed using the workflow \texttt{egvtools::polygon2input()}. Once rasterised, the layer is aggregated to EGV
resolution using the workflow \texttt{egvtools::input2egv()}, which calculates the arithmetic mean and thus
results in a cover fraction. During aggregation, inverse
distance weighted (power = 2) gap filling on the output is applied to
ensure no missing values at the edges. Finally, the layer is standardised
by subtracting the arithmetic mean and dividing by the root mean squared error.

\begin{Shaded}
\begin{Highlighting}[]
\CommentTok{\# libs {-}{-}{-}{-}}
\ControlFlowTok{if}\NormalTok{(}\SpecialCharTok{!}\FunctionTok{require}\NormalTok{(egvtools)) \{remotes}\SpecialCharTok{::}\FunctionTok{install\_github}\NormalTok{(}\StringTok{"aavotins/egvtools"}\NormalTok{); }\FunctionTok{require}\NormalTok{(egvtools)\}}
\ControlFlowTok{if}\NormalTok{(}\SpecialCharTok{!}\FunctionTok{require}\NormalTok{(terra)) \{}\FunctionTok{install.packages}\NormalTok{(}\StringTok{"terra"}\NormalTok{); }\FunctionTok{require}\NormalTok{(terra)\}}
\ControlFlowTok{if}\NormalTok{(}\SpecialCharTok{!}\FunctionTok{require}\NormalTok{(sf)) \{}\FunctionTok{install.packages}\NormalTok{(}\StringTok{"sf"}\NormalTok{); }\FunctionTok{require}\NormalTok{(sf)\}}
\ControlFlowTok{if}\NormalTok{(}\SpecialCharTok{!}\FunctionTok{require}\NormalTok{(tidyverse)) \{}\FunctionTok{install.packages}\NormalTok{(}\StringTok{"tidyverse"}\NormalTok{); }\FunctionTok{require}\NormalTok{(tidyverse)\}}
\ControlFlowTok{if}\NormalTok{(}\SpecialCharTok{!}\FunctionTok{require}\NormalTok{(sfarrow)) \{}\FunctionTok{install.packages}\NormalTok{(}\StringTok{"sfarrow"}\NormalTok{); }\FunctionTok{require}\NormalTok{(sfarrow)\}}
\ControlFlowTok{if}\NormalTok{(}\SpecialCharTok{!}\FunctionTok{require}\NormalTok{(readxl)) \{}\FunctionTok{install.packages}\NormalTok{(}\StringTok{"readxl"}\NormalTok{); }\FunctionTok{require}\NormalTok{(readxl)\}}
\ControlFlowTok{if}\NormalTok{(}\SpecialCharTok{!}\FunctionTok{require}\NormalTok{(raster)) \{}\FunctionTok{install.packages}\NormalTok{(}\StringTok{"raster"}\NormalTok{); }\FunctionTok{require}\NormalTok{(raster)\}}
\ControlFlowTok{if}\NormalTok{(}\SpecialCharTok{!}\FunctionTok{require}\NormalTok{(fasterize)) \{}\FunctionTok{install.packages}\NormalTok{(}\StringTok{"fasterize"}\NormalTok{); }\FunctionTok{require}\NormalTok{(fasterize)\}}

\CommentTok{\# templates {-}{-}{-}{-}}
\NormalTok{template100}\OtherTok{=}\FunctionTok{rast}\NormalTok{(}\StringTok{"./Templates/TemplateRasters/LV100m\_10km.tif"}\NormalTok{)}
\NormalTok{template10}\OtherTok{=}\FunctionTok{rast}\NormalTok{(}\StringTok{"./Templates/TemplateRasters/LV10m\_10km.tif"}\NormalTok{)}
\NormalTok{rastrs10}\OtherTok{=}\FunctionTok{raster}\NormalTok{(template10)}

\NormalTok{nulls10}\OtherTok{=}\FunctionTok{rast}\NormalTok{(}\StringTok{"./Templates/TemplateRasters/nulls\_LV10m\_10km.tif"}\NormalTok{)}
\NormalTok{nulls100}\OtherTok{=}\FunctionTok{rast}\NormalTok{(}\StringTok{"./Templates/TemplateRasters/nulls\_LV100m\_10km.tif"}\NormalTok{)}

\CommentTok{\# codes {-}{-}{-}{-}}
\NormalTok{kodi}\OtherTok{=}\FunctionTok{read\_excel}\NormalTok{(}\StringTok{"./Geodata/2024/LAD/KulturuKodi\_2024.xlsx"}\NormalTok{)}
\NormalTok{kodi}\SpecialCharTok{$}\NormalTok{kods}\OtherTok{=}\FunctionTok{as.character}\NormalTok{(kodi}\SpecialCharTok{$}\NormalTok{kods)}
\CommentTok{\# LAD {-}{-}{-}{-}}
\NormalTok{lad}\OtherTok{=}\NormalTok{sfarrow}\SpecialCharTok{::}\FunctionTok{st\_read\_parquet}\NormalTok{(}\StringTok{"./Geodata/2024/LAD/Lauki\_2024.parquet"}\NormalTok{)}
\NormalTok{lad}\SpecialCharTok{$}\NormalTok{yes}\OtherTok{=}\DecValTok{1}
\NormalTok{lad}\OtherTok{=}\NormalTok{lad }\SpecialCharTok{\%\textgreater{}\%} 
 \FunctionTok{left\_join}\NormalTok{(kodi,}\AttributeTok{by=}\FunctionTok{c}\NormalTok{(}\StringTok{"PRODUCT\_CODE"}\OtherTok{=}\StringTok{"kods"}\NormalTok{))}

\CommentTok{\# simple landscape {-}{-}{-}{-}}
\NormalTok{simple\_landscape}\OtherTok{=}\FunctionTok{rast}\NormalTok{(}\StringTok{"RasterGrids\_10m/2024/Ainava\_vienk\_mask.tif"}\NormalTok{)}


\CommentTok{\# FarmlandCrops\_CropsWinter\_cell.tif    egv\_200 {-}{-}{-}{-}}
\NormalTok{dati}\OtherTok{=}\NormalTok{lad }\SpecialCharTok{\%\textgreater{}\%} 
 \FunctionTok{filter}\NormalTok{(}\FunctionTok{str\_detect}\NormalTok{(SDM\_grupa\_sakums,}\StringTok{"labība{-}ziemāji"}\NormalTok{))}
\FunctionTok{table}\NormalTok{(dati}\SpecialCharTok{$}\NormalTok{SDM\_grupa\_sakums,}\AttributeTok{useNA=}\StringTok{"always"}\NormalTok{)}


\NormalTok{p2i\_rez}\OtherTok{=}\NormalTok{egvtools}\SpecialCharTok{::}\FunctionTok{polygon2input}\NormalTok{(}\AttributeTok{vector\_data =}\NormalTok{ dati,}
                \AttributeTok{template\_path =} \StringTok{"./Templates/TemplateRasters/LV10m\_10km.tif"}\NormalTok{,}
                \AttributeTok{out\_path =} \StringTok{"./RasterGrids\_10m/2024/"}\NormalTok{,}
                \AttributeTok{file\_name =} \StringTok{"FarmlandCrops\_CropsWinter\_input.tif"}\NormalTok{,}
                \AttributeTok{value\_field =} \StringTok{"yes"}\NormalTok{,}
                \AttributeTok{prepare=}\ConstantTok{FALSE}\NormalTok{,}
                \AttributeTok{background\_raster =} \StringTok{"./Templates/TemplateRasters/nulls\_LV10m\_10km.tif"}\NormalTok{,}
                \AttributeTok{plot\_result =} \ConstantTok{TRUE}\NormalTok{)}
\NormalTok{p2i\_rez}
\NormalTok{i2e\_rez}\OtherTok{=}\NormalTok{egvtools}\SpecialCharTok{::}\FunctionTok{input2egv}\NormalTok{(}\AttributeTok{input=}\FunctionTok{paste0}\NormalTok{(}\StringTok{"./RasterGrids\_10m/2024/"}\NormalTok{,}
                     \StringTok{"FarmlandCrops\_CropsWinter\_input.tif"}\NormalTok{),}
              \AttributeTok{egv\_template=} \StringTok{"./Templates/TemplateRasters/LV100m\_10km.tif"}\NormalTok{,}
              \AttributeTok{summary\_function =} \StringTok{"average"}\NormalTok{,}
              \AttributeTok{missing\_job =} \StringTok{"FillOutput"}\NormalTok{,}
              \AttributeTok{outlocation =} \StringTok{"./RasterGrids\_100m/2024/RAW/"}\NormalTok{,}
              \AttributeTok{outfilename =} \StringTok{"FarmlandCrops\_CropsWinter\_cell.tif"}\NormalTok{,}
              \AttributeTok{layername =} \StringTok{"egv\_200"}\NormalTok{,}
              \AttributeTok{idw\_weight =} \DecValTok{2}\NormalTok{,}
              \AttributeTok{plot\_gaps =} \ConstantTok{FALSE}\NormalTok{,}\AttributeTok{plot\_final =} \ConstantTok{TRUE}\NormalTok{)}
\NormalTok{i2e\_rez}
\FunctionTok{rm}\NormalTok{(p2i\_rez)}
\FunctionTok{rm}\NormalTok{(i2e\_rez)}
\FunctionTok{rm}\NormalTok{(dati)}
\FunctionTok{unlink}\NormalTok{(}\StringTok{"./RasterGrids\_10m/2024/FarmlandCrops\_CropsWinter\_input.tif"}\NormalTok{)}


\CommentTok{\# standardisation {-}{-}{-}{-}}
\ControlFlowTok{if}\NormalTok{(}\SpecialCharTok{!}\FunctionTok{require}\NormalTok{(terra)) \{}\FunctionTok{install.packages}\NormalTok{(}\StringTok{"terra"}\NormalTok{); }\FunctionTok{require}\NormalTok{(terra)\}}
\ControlFlowTok{if}\NormalTok{(}\SpecialCharTok{!}\FunctionTok{require}\NormalTok{(tidyverse)) \{}\FunctionTok{install.packages}\NormalTok{(}\StringTok{"tidyverse"}\NormalTok{); }\FunctionTok{require}\NormalTok{(tidyverse)\}}

\NormalTok{nosaukums}\OtherTok{=}\StringTok{"FarmlandCrops\_CropsWinter\_cell.tif"}
\NormalTok{ielasisanas\_cels}\OtherTok{=}\FunctionTok{paste0}\NormalTok{(}\StringTok{"./RasterGrids\_100m/2024/RAW/"}\NormalTok{,nosaukums)}
\NormalTok{saglabasanas\_cels}\OtherTok{=}\FunctionTok{paste0}\NormalTok{(}\StringTok{"./RasterGrids\_100m/2024/Scaled/"}\NormalTok{,nosaukums)}
\NormalTok{slanis}\OtherTok{=}\FunctionTok{rast}\NormalTok{(ielasisanas\_cels)}
\NormalTok{videjais}\OtherTok{=}\FunctionTok{global}\NormalTok{(slanis,}\AttributeTok{fun=}\StringTok{"mean"}\NormalTok{,}\AttributeTok{na.rm=}\ConstantTok{TRUE}\NormalTok{)}
\NormalTok{centrets}\OtherTok{=}\NormalTok{slanis}\SpecialCharTok{{-}}\NormalTok{videjais[,}\DecValTok{1}\NormalTok{]}
\NormalTok{standartnovirze}\OtherTok{=}\NormalTok{terra}\SpecialCharTok{::}\FunctionTok{global}\NormalTok{(centrets,}\AttributeTok{fun=}\StringTok{"rms"}\NormalTok{,}\AttributeTok{na.rm=}\ConstantTok{TRUE}\NormalTok{)}
\NormalTok{merogots}\OtherTok{=}\NormalTok{centrets}\SpecialCharTok{/}\NormalTok{standartnovirze[,}\DecValTok{1}\NormalTok{]}
\FunctionTok{writeRaster}\NormalTok{(merogots,}
      \AttributeTok{filename=}\NormalTok{saglabasanas\_cels,}
      \AttributeTok{overwrite=}\ConstantTok{TRUE}\NormalTok{)}
\end{Highlighting}
\end{Shaded}

\section{FarmlandCrops\_CropsWinter\_r500}\label{ch06.201}

\textbf{filename:} \texttt{FarmlandCrops\_CropsWinter\_r500.tif}

\textbf{layername:} \texttt{egv\_201}

\textbf{English name:} Fractional cover of Winter Crops within the 0.5 km landscape

\textbf{Latvian name:} Ziemāju aramzemēs platības īpatsvars 0,5 km ainavā

\textbf{Procedure:} The cover fraction within a radius of 500 m around the analysis grid cell is
calculated as the area-weighted sum of the \hyperref[ch06.200]{analysis cells} inside the
buffer, using the workflow \texttt{egvtools::radius\_function()}. During the calculation of the landscape metric,
inverse distance weighted (power = 2) gap filling on the output is applied
to ensure no missing values at the edges. Then the layer is rewritten to set
its name. Finally, the layer is standardised by subtracting the arithmetic
mean and dividing by the root mean squared error.

\begin{Shaded}
\begin{Highlighting}[]
\CommentTok{\# libs {-}{-}{-}{-}}
\ControlFlowTok{if}\NormalTok{(}\SpecialCharTok{!}\FunctionTok{require}\NormalTok{(terra)) \{}\FunctionTok{install.packages}\NormalTok{(}\StringTok{"terra"}\NormalTok{); }\FunctionTok{require}\NormalTok{(terra)\}}
\ControlFlowTok{if}\NormalTok{(}\SpecialCharTok{!}\FunctionTok{require}\NormalTok{(egvtools)) \{remotes}\SpecialCharTok{::}\FunctionTok{install\_github}\NormalTok{(}\StringTok{"aavotins/egvtools"}\NormalTok{); }\FunctionTok{require}\NormalTok{(egvtools)\}}


\CommentTok{\# Templates {-}{-}{-}{-}{-}}
\NormalTok{template100}\OtherTok{=}\FunctionTok{rast}\NormalTok{(}\StringTok{"./Templates/TemplateRasters/LV100m\_10km.tif"}\NormalTok{)}

\CommentTok{\# radii {-}{-}{-}{-}}
\FunctionTok{radius\_function}\NormalTok{(}
 \AttributeTok{kvadrati\_path =} \StringTok{"./Templates/TemplateGrids/tiles/"}\NormalTok{,}
 \AttributeTok{radii\_path   =} \StringTok{"./Templates/TemplateGridPoints/tiles/"}\NormalTok{,}
 \AttributeTok{tikls100\_path =} \StringTok{"./Templates/TemplateGrids/tikls100\_sauzeme.parquet"}\NormalTok{,}
 \AttributeTok{template\_path =} \StringTok{"./Templates/TemplateRasters/LV100m\_10km.tif"}\NormalTok{,}
 \AttributeTok{input\_layers  =} \FunctionTok{c}\NormalTok{(}\StringTok{"./RasterGrids\_100m/2024/RAW/FarmlandCrops\_CropsWinter\_cell.tif"}\NormalTok{),}
 \AttributeTok{layer\_prefixes =} \FunctionTok{c}\NormalTok{(}\StringTok{"FarmlandCrops\_CropsWinter"}\NormalTok{),}
 \AttributeTok{output\_dir   =} \StringTok{"./RasterGrids\_100m/2024/RAW/"}\NormalTok{,}
 \AttributeTok{n\_workers   =} \DecValTok{6}\NormalTok{,}
 \AttributeTok{radii     =} \FunctionTok{c}\NormalTok{(}\StringTok{"r500"}\NormalTok{),}
 \AttributeTok{radius\_mode  =} \StringTok{"sparse"}\NormalTok{,}
 \AttributeTok{extract\_fun  =} \StringTok{"mean"}\NormalTok{,}
 \AttributeTok{fill\_missing  =} \ConstantTok{TRUE}\NormalTok{,}
 \AttributeTok{IDW\_weight   =} \DecValTok{2}\NormalTok{,}
 \AttributeTok{future\_max\_size =} \DecValTok{40} \SpecialCharTok{*} \DecValTok{1024}\SpecialCharTok{\^{}}\DecValTok{3}\NormalTok{)}


\CommentTok{\# FarmlandCrops\_CropsWinter\_r500.tif    egv\_201 {-}{-}{-}{-}}
\NormalTok{slanis}\OtherTok{=}\FunctionTok{rast}\NormalTok{(}\StringTok{"./RasterGrids\_100m/2024/RAW/FarmlandCrops\_CropsWinter\_r500.tif"}\NormalTok{)}
\FunctionTok{names}\NormalTok{(slanis)}\OtherTok{=}\StringTok{"egv\_201"}
\NormalTok{slanis2}\OtherTok{=}\FunctionTok{project}\NormalTok{(slanis,template100)}
\FunctionTok{writeRaster}\NormalTok{(slanis2,}
      \StringTok{"./RasterGrids\_100m/2024/RAW/FarmlandCrops\_CropsWinter\_r500.tif"}\NormalTok{,}
      \AttributeTok{overwrite=}\ConstantTok{TRUE}\NormalTok{)}

\CommentTok{\# standardisation {-}{-}{-}{-}}
\ControlFlowTok{if}\NormalTok{(}\SpecialCharTok{!}\FunctionTok{require}\NormalTok{(terra)) \{}\FunctionTok{install.packages}\NormalTok{(}\StringTok{"terra"}\NormalTok{); }\FunctionTok{require}\NormalTok{(terra)\}}
\ControlFlowTok{if}\NormalTok{(}\SpecialCharTok{!}\FunctionTok{require}\NormalTok{(tidyverse)) \{}\FunctionTok{install.packages}\NormalTok{(}\StringTok{"tidyverse"}\NormalTok{); }\FunctionTok{require}\NormalTok{(tidyverse)\}}

\NormalTok{nosaukums}\OtherTok{=}\StringTok{"FarmlandCrops\_CropsWinter\_r500.tif"}
\NormalTok{ielasisanas\_cels}\OtherTok{=}\FunctionTok{paste0}\NormalTok{(}\StringTok{"./RasterGrids\_100m/2024/RAW/"}\NormalTok{,nosaukums)}
\NormalTok{saglabasanas\_cels}\OtherTok{=}\FunctionTok{paste0}\NormalTok{(}\StringTok{"./RasterGrids\_100m/2024/Scaled/"}\NormalTok{,nosaukums)}
\NormalTok{slanis}\OtherTok{=}\FunctionTok{rast}\NormalTok{(ielasisanas\_cels)}
\NormalTok{videjais}\OtherTok{=}\FunctionTok{global}\NormalTok{(slanis,}\AttributeTok{fun=}\StringTok{"mean"}\NormalTok{,}\AttributeTok{na.rm=}\ConstantTok{TRUE}\NormalTok{)}
\NormalTok{centrets}\OtherTok{=}\NormalTok{slanis}\SpecialCharTok{{-}}\NormalTok{videjais[,}\DecValTok{1}\NormalTok{]}
\NormalTok{standartnovirze}\OtherTok{=}\NormalTok{terra}\SpecialCharTok{::}\FunctionTok{global}\NormalTok{(centrets,}\AttributeTok{fun=}\StringTok{"rms"}\NormalTok{,}\AttributeTok{na.rm=}\ConstantTok{TRUE}\NormalTok{)}
\NormalTok{merogots}\OtherTok{=}\NormalTok{centrets}\SpecialCharTok{/}\NormalTok{standartnovirze[,}\DecValTok{1}\NormalTok{]}
\FunctionTok{writeRaster}\NormalTok{(merogots,}
      \AttributeTok{filename=}\NormalTok{saglabasanas\_cels,}
      \AttributeTok{overwrite=}\ConstantTok{TRUE}\NormalTok{)}
\end{Highlighting}
\end{Shaded}

\section{FarmlandCrops\_CropsWinter\_r1250}\label{ch06.202}

\textbf{filename:} \texttt{FarmlandCrops\_CropsWinter\_r1250.tif}

\textbf{layername:} \texttt{egv\_202}

\textbf{English name:} Fractional cover of Winter Crops within the 1.25 km landscape

\textbf{Latvian name:} Ziemāju aramzemēs platības īpatsvars 1,25 km ainavā

\textbf{Procedure:} The cover fraction within a radius of 1250 m around the analysis grid cell
is calculated as the area-weighted sum of the \hyperref[ch06.200]{analysis cells} inside
the buffer, using the workflow \texttt{egvtools::radius\_function()}. During the calculation of the landscape
metric, inverse distance weighted (power = 2) gap filling on the output is
applied to ensure no missing values at the edges. Then the layer is
rewritten to set its name. Finally, the layer is standardised by
subtracting the arithmetic mean and dividing by the root mean squared error.

\begin{Shaded}
\begin{Highlighting}[]
\CommentTok{\# libs {-}{-}{-}{-}}
\ControlFlowTok{if}\NormalTok{(}\SpecialCharTok{!}\FunctionTok{require}\NormalTok{(terra)) \{}\FunctionTok{install.packages}\NormalTok{(}\StringTok{"terra"}\NormalTok{); }\FunctionTok{require}\NormalTok{(terra)\}}
\ControlFlowTok{if}\NormalTok{(}\SpecialCharTok{!}\FunctionTok{require}\NormalTok{(egvtools)) \{remotes}\SpecialCharTok{::}\FunctionTok{install\_github}\NormalTok{(}\StringTok{"aavotins/egvtools"}\NormalTok{); }\FunctionTok{require}\NormalTok{(egvtools)\}}


\CommentTok{\# Templates {-}{-}{-}{-}{-}}
\NormalTok{template100}\OtherTok{=}\FunctionTok{rast}\NormalTok{(}\StringTok{"./Templates/TemplateRasters/LV100m\_10km.tif"}\NormalTok{)}

\CommentTok{\# radii {-}{-}{-}{-}}
\FunctionTok{radius\_function}\NormalTok{(}
 \AttributeTok{kvadrati\_path =} \StringTok{"./Templates/TemplateGrids/tiles/"}\NormalTok{,}
 \AttributeTok{radii\_path   =} \StringTok{"./Templates/TemplateGridPoints/tiles/"}\NormalTok{,}
 \AttributeTok{tikls100\_path =} \StringTok{"./Templates/TemplateGrids/tikls100\_sauzeme.parquet"}\NormalTok{,}
 \AttributeTok{template\_path =} \StringTok{"./Templates/TemplateRasters/LV100m\_10km.tif"}\NormalTok{,}
 \AttributeTok{input\_layers  =} \FunctionTok{c}\NormalTok{(}\StringTok{"./RasterGrids\_100m/2024/RAW/FarmlandCrops\_CropsWinter\_cell.tif"}\NormalTok{),}
 \AttributeTok{layer\_prefixes =} \FunctionTok{c}\NormalTok{(}\StringTok{"FarmlandCrops\_CropsWinter"}\NormalTok{),}
 \AttributeTok{output\_dir   =} \StringTok{"./RasterGrids\_100m/2024/RAW/"}\NormalTok{,}
 \AttributeTok{n\_workers   =} \DecValTok{6}\NormalTok{,}
 \AttributeTok{radii     =} \FunctionTok{c}\NormalTok{(}\StringTok{"r1250"}\NormalTok{),}
 \AttributeTok{radius\_mode  =} \StringTok{"sparse"}\NormalTok{,}
 \AttributeTok{extract\_fun  =} \StringTok{"mean"}\NormalTok{,}
 \AttributeTok{fill\_missing  =} \ConstantTok{TRUE}\NormalTok{,}
 \AttributeTok{IDW\_weight   =} \DecValTok{2}\NormalTok{,}
 \AttributeTok{future\_max\_size =} \DecValTok{40} \SpecialCharTok{*} \DecValTok{1024}\SpecialCharTok{\^{}}\DecValTok{3}\NormalTok{)}


\CommentTok{\# FarmlandCrops\_CropsWinter\_r1250.tif   egv\_202 {-}{-}{-}{-}}
\NormalTok{slanis}\OtherTok{=}\FunctionTok{rast}\NormalTok{(}\StringTok{"./RasterGrids\_100m/2024/RAW/FarmlandCrops\_CropsWinter\_r1250.tif"}\NormalTok{)}
\FunctionTok{names}\NormalTok{(slanis)}\OtherTok{=}\StringTok{"egv\_202"}
\NormalTok{slanis2}\OtherTok{=}\FunctionTok{project}\NormalTok{(slanis,template100)}
\FunctionTok{writeRaster}\NormalTok{(slanis2,}
      \StringTok{"./RasterGrids\_100m/2024/RAW/FarmlandCrops\_CropsWinter\_r1250.tif"}\NormalTok{,}
      \AttributeTok{overwrite=}\ConstantTok{TRUE}\NormalTok{)}

\CommentTok{\# standardisation {-}{-}{-}{-}}
\ControlFlowTok{if}\NormalTok{(}\SpecialCharTok{!}\FunctionTok{require}\NormalTok{(terra)) \{}\FunctionTok{install.packages}\NormalTok{(}\StringTok{"terra"}\NormalTok{); }\FunctionTok{require}\NormalTok{(terra)\}}
\ControlFlowTok{if}\NormalTok{(}\SpecialCharTok{!}\FunctionTok{require}\NormalTok{(tidyverse)) \{}\FunctionTok{install.packages}\NormalTok{(}\StringTok{"tidyverse"}\NormalTok{); }\FunctionTok{require}\NormalTok{(tidyverse)\}}

\NormalTok{nosaukums}\OtherTok{=}\StringTok{"FarmlandCrops\_CropsWinter\_r1250.tif"}
\NormalTok{ielasisanas\_cels}\OtherTok{=}\FunctionTok{paste0}\NormalTok{(}\StringTok{"./RasterGrids\_100m/2024/RAW/"}\NormalTok{,nosaukums)}
\NormalTok{saglabasanas\_cels}\OtherTok{=}\FunctionTok{paste0}\NormalTok{(}\StringTok{"./RasterGrids\_100m/2024/Scaled/"}\NormalTok{,nosaukums)}
\NormalTok{slanis}\OtherTok{=}\FunctionTok{rast}\NormalTok{(ielasisanas\_cels)}
\NormalTok{videjais}\OtherTok{=}\FunctionTok{global}\NormalTok{(slanis,}\AttributeTok{fun=}\StringTok{"mean"}\NormalTok{,}\AttributeTok{na.rm=}\ConstantTok{TRUE}\NormalTok{)}
\NormalTok{centrets}\OtherTok{=}\NormalTok{slanis}\SpecialCharTok{{-}}\NormalTok{videjais[,}\DecValTok{1}\NormalTok{]}
\NormalTok{standartnovirze}\OtherTok{=}\NormalTok{terra}\SpecialCharTok{::}\FunctionTok{global}\NormalTok{(centrets,}\AttributeTok{fun=}\StringTok{"rms"}\NormalTok{,}\AttributeTok{na.rm=}\ConstantTok{TRUE}\NormalTok{)}
\NormalTok{merogots}\OtherTok{=}\NormalTok{centrets}\SpecialCharTok{/}\NormalTok{standartnovirze[,}\DecValTok{1}\NormalTok{]}
\FunctionTok{writeRaster}\NormalTok{(merogots,}
      \AttributeTok{filename=}\NormalTok{saglabasanas\_cels,}
      \AttributeTok{overwrite=}\ConstantTok{TRUE}\NormalTok{)}
\end{Highlighting}
\end{Shaded}

\section{FarmlandCrops\_CropsWinter\_r3000}\label{ch06.203}

\textbf{filename:} \texttt{FarmlandCrops\_CropsWinter\_r3000.tif}

\textbf{layername:} \texttt{egv\_203}

\textbf{English name:} Fractional cover of Winter Crops within the 3 km landscape

\textbf{Latvian name:} Ziemāju aramzemēs platības īpatsvars 3 km ainavā

\textbf{Procedure:} The cover fraction within a radius of 3000 m around the analysis grid cell
is calculated as the area-weighted sum of the \hyperref[ch06.200]{analysis cells} inside
the buffer, using the workflow \texttt{egvtools::radius\_function()}. During the calculation of the landscape
metric, inverse distance weighted (power = 2) gap filling on the output is
applied to ensure no missing values at the edges. Then the layer is
rewritten to set its name. Finally, the layer is standardised by
subtracting the arithmetic mean and dividing by the root mean squared error.

\begin{Shaded}
\begin{Highlighting}[]
\CommentTok{\# libs {-}{-}{-}{-}}
\ControlFlowTok{if}\NormalTok{(}\SpecialCharTok{!}\FunctionTok{require}\NormalTok{(terra)) \{}\FunctionTok{install.packages}\NormalTok{(}\StringTok{"terra"}\NormalTok{); }\FunctionTok{require}\NormalTok{(terra)\}}
\ControlFlowTok{if}\NormalTok{(}\SpecialCharTok{!}\FunctionTok{require}\NormalTok{(egvtools)) \{remotes}\SpecialCharTok{::}\FunctionTok{install\_github}\NormalTok{(}\StringTok{"aavotins/egvtools"}\NormalTok{); }\FunctionTok{require}\NormalTok{(egvtools)\}}


\CommentTok{\# Templates {-}{-}{-}{-}{-}}
\NormalTok{template100}\OtherTok{=}\FunctionTok{rast}\NormalTok{(}\StringTok{"./Templates/TemplateRasters/LV100m\_10km.tif"}\NormalTok{)}

\CommentTok{\# radii {-}{-}{-}{-}}
\FunctionTok{radius\_function}\NormalTok{(}
 \AttributeTok{kvadrati\_path =} \StringTok{"./Templates/TemplateGrids/tiles/"}\NormalTok{,}
 \AttributeTok{radii\_path   =} \StringTok{"./Templates/TemplateGridPoints/tiles/"}\NormalTok{,}
 \AttributeTok{tikls100\_path =} \StringTok{"./Templates/TemplateGrids/tikls100\_sauzeme.parquet"}\NormalTok{,}
 \AttributeTok{template\_path =} \StringTok{"./Templates/TemplateRasters/LV100m\_10km.tif"}\NormalTok{,}
 \AttributeTok{input\_layers  =} \FunctionTok{c}\NormalTok{(}\StringTok{"./RasterGrids\_100m/2024/RAW/FarmlandCrops\_CropsWinter\_cell.tif"}\NormalTok{),}
 \AttributeTok{layer\_prefixes =} \FunctionTok{c}\NormalTok{(}\StringTok{"FarmlandCrops\_CropsWinter"}\NormalTok{),}
 \AttributeTok{output\_dir   =} \StringTok{"./RasterGrids\_100m/2024/RAW/"}\NormalTok{,}
 \AttributeTok{n\_workers   =} \DecValTok{6}\NormalTok{,}
 \AttributeTok{radii     =} \FunctionTok{c}\NormalTok{(}\StringTok{"r3000"}\NormalTok{),}
 \AttributeTok{radius\_mode  =} \StringTok{"sparse"}\NormalTok{,}
 \AttributeTok{extract\_fun  =} \StringTok{"mean"}\NormalTok{,}
 \AttributeTok{fill\_missing  =} \ConstantTok{TRUE}\NormalTok{,}
 \AttributeTok{IDW\_weight   =} \DecValTok{2}\NormalTok{,}
 \AttributeTok{future\_max\_size =} \DecValTok{40} \SpecialCharTok{*} \DecValTok{1024}\SpecialCharTok{\^{}}\DecValTok{3}\NormalTok{)}


\CommentTok{\# FarmlandCrops\_CropsWinter\_r3000.tif   egv\_203 {-}{-}{-}{-}}
\NormalTok{slanis}\OtherTok{=}\FunctionTok{rast}\NormalTok{(}\StringTok{"./RasterGrids\_100m/2024/RAW/FarmlandCrops\_CropsWinter\_r3000.tif"}\NormalTok{)}
\FunctionTok{names}\NormalTok{(slanis)}\OtherTok{=}\StringTok{"egv\_203"}
\NormalTok{slanis2}\OtherTok{=}\FunctionTok{project}\NormalTok{(slanis,template100)}
\FunctionTok{writeRaster}\NormalTok{(slanis2,}
      \StringTok{"./RasterGrids\_100m/2024/RAW/FarmlandCrops\_CropsWinter\_r3000.tif"}\NormalTok{,}
      \AttributeTok{overwrite=}\ConstantTok{TRUE}\NormalTok{)}

\CommentTok{\# standardisation {-}{-}{-}{-}}
\ControlFlowTok{if}\NormalTok{(}\SpecialCharTok{!}\FunctionTok{require}\NormalTok{(terra)) \{}\FunctionTok{install.packages}\NormalTok{(}\StringTok{"terra"}\NormalTok{); }\FunctionTok{require}\NormalTok{(terra)\}}
\ControlFlowTok{if}\NormalTok{(}\SpecialCharTok{!}\FunctionTok{require}\NormalTok{(tidyverse)) \{}\FunctionTok{install.packages}\NormalTok{(}\StringTok{"tidyverse"}\NormalTok{); }\FunctionTok{require}\NormalTok{(tidyverse)\}}

\NormalTok{nosaukums}\OtherTok{=}\StringTok{"FarmlandCrops\_CropsWinter\_r3000.tif"}
\NormalTok{ielasisanas\_cels}\OtherTok{=}\FunctionTok{paste0}\NormalTok{(}\StringTok{"./RasterGrids\_100m/2024/RAW/"}\NormalTok{,nosaukums)}
\NormalTok{saglabasanas\_cels}\OtherTok{=}\FunctionTok{paste0}\NormalTok{(}\StringTok{"./RasterGrids\_100m/2024/Scaled/"}\NormalTok{,nosaukums)}
\NormalTok{slanis}\OtherTok{=}\FunctionTok{rast}\NormalTok{(ielasisanas\_cels)}
\NormalTok{videjais}\OtherTok{=}\FunctionTok{global}\NormalTok{(slanis,}\AttributeTok{fun=}\StringTok{"mean"}\NormalTok{,}\AttributeTok{na.rm=}\ConstantTok{TRUE}\NormalTok{)}
\NormalTok{centrets}\OtherTok{=}\NormalTok{slanis}\SpecialCharTok{{-}}\NormalTok{videjais[,}\DecValTok{1}\NormalTok{]}
\NormalTok{standartnovirze}\OtherTok{=}\NormalTok{terra}\SpecialCharTok{::}\FunctionTok{global}\NormalTok{(centrets,}\AttributeTok{fun=}\StringTok{"rms"}\NormalTok{,}\AttributeTok{na.rm=}\ConstantTok{TRUE}\NormalTok{)}
\NormalTok{merogots}\OtherTok{=}\NormalTok{centrets}\SpecialCharTok{/}\NormalTok{standartnovirze[,}\DecValTok{1}\NormalTok{]}
\FunctionTok{writeRaster}\NormalTok{(merogots,}
      \AttributeTok{filename=}\NormalTok{saglabasanas\_cels,}
      \AttributeTok{overwrite=}\ConstantTok{TRUE}\NormalTok{)}
\end{Highlighting}
\end{Shaded}

\section{FarmlandCrops\_CropsWinter\_r10000}\label{ch06.204}

\textbf{filename:} \texttt{FarmlandCrops\_CropsWinter\_r10000.tif}

\textbf{layername:} \texttt{egv\_204}

\textbf{English name:} Fractional cover of Winter Crops within the 10 km landscape

\textbf{Latvian name:} Ziemāju aramzemēs platības īpatsvars 10 km ainavā

\textbf{Procedure:} The cover fraction within a radius of 10000 m around the analysis grid cell
is calculated as the area-weighted sum of the \hyperref[ch06.200]{analysis cells} inside
the buffer, using the workflow \texttt{egvtools::radius\_function()}. During the calculation of the landscape
metric, inverse distance weighted (power = 2) gap filling on the output is
applied to ensure no missing values at the edges. Then the layer is
rewritten to set its name. Finally, the layer is standardised by
subtracting the arithmetic mean and dividing by the root mean squared error.

\begin{Shaded}
\begin{Highlighting}[]
\CommentTok{\# libs {-}{-}{-}{-}}
\ControlFlowTok{if}\NormalTok{(}\SpecialCharTok{!}\FunctionTok{require}\NormalTok{(terra)) \{}\FunctionTok{install.packages}\NormalTok{(}\StringTok{"terra"}\NormalTok{); }\FunctionTok{require}\NormalTok{(terra)\}}
\ControlFlowTok{if}\NormalTok{(}\SpecialCharTok{!}\FunctionTok{require}\NormalTok{(egvtools)) \{remotes}\SpecialCharTok{::}\FunctionTok{install\_github}\NormalTok{(}\StringTok{"aavotins/egvtools"}\NormalTok{); }\FunctionTok{require}\NormalTok{(egvtools)\}}


\CommentTok{\# Templates {-}{-}{-}{-}{-}}
\NormalTok{template100}\OtherTok{=}\FunctionTok{rast}\NormalTok{(}\StringTok{"./Templates/TemplateRasters/LV100m\_10km.tif"}\NormalTok{)}

\CommentTok{\# radii {-}{-}{-}{-}}
\FunctionTok{radius\_function}\NormalTok{(}
 \AttributeTok{kvadrati\_path =} \StringTok{"./Templates/TemplateGrids/tiles/"}\NormalTok{,}
 \AttributeTok{radii\_path   =} \StringTok{"./Templates/TemplateGridPoints/tiles/"}\NormalTok{,}
 \AttributeTok{tikls100\_path =} \StringTok{"./Templates/TemplateGrids/tikls100\_sauzeme.parquet"}\NormalTok{,}
 \AttributeTok{template\_path =} \StringTok{"./Templates/TemplateRasters/LV100m\_10km.tif"}\NormalTok{,}
 \AttributeTok{input\_layers  =} \FunctionTok{c}\NormalTok{(}\StringTok{"./RasterGrids\_100m/2024/RAW/FarmlandCrops\_CropsWinter\_cell.tif"}\NormalTok{),}
 \AttributeTok{layer\_prefixes =} \FunctionTok{c}\NormalTok{(}\StringTok{"FarmlandCrops\_CropsWinter"}\NormalTok{),}
 \AttributeTok{output\_dir   =} \StringTok{"./RasterGrids\_100m/2024/RAW/"}\NormalTok{,}
 \AttributeTok{n\_workers   =} \DecValTok{6}\NormalTok{,}
 \AttributeTok{radii     =} \FunctionTok{c}\NormalTok{(}\StringTok{"r10000"}\NormalTok{),}
 \AttributeTok{radius\_mode  =} \StringTok{"sparse"}\NormalTok{,}
 \AttributeTok{extract\_fun  =} \StringTok{"mean"}\NormalTok{,}
 \AttributeTok{fill\_missing  =} \ConstantTok{TRUE}\NormalTok{,}
 \AttributeTok{IDW\_weight   =} \DecValTok{2}\NormalTok{,}
 \AttributeTok{future\_max\_size =} \DecValTok{40} \SpecialCharTok{*} \DecValTok{1024}\SpecialCharTok{\^{}}\DecValTok{3}\NormalTok{)}


\CommentTok{\# FarmlandCrops\_CropsWinter\_r10000.tif  egv\_204 {-}{-}{-}{-}}
\NormalTok{slanis}\OtherTok{=}\FunctionTok{rast}\NormalTok{(}\StringTok{"./RasterGrids\_100m/2024/RAW/FarmlandCrops\_CropsWinter\_r10000.tif"}\NormalTok{)}
\FunctionTok{names}\NormalTok{(slanis)}\OtherTok{=}\StringTok{"egv\_204"}
\NormalTok{slanis2}\OtherTok{=}\FunctionTok{project}\NormalTok{(slanis,template100)}
\FunctionTok{writeRaster}\NormalTok{(slanis2,}
      \StringTok{"./RasterGrids\_100m/2024/RAW/FarmlandCrops\_CropsWinter\_r10000.tif"}\NormalTok{,}
      \AttributeTok{overwrite=}\ConstantTok{TRUE}\NormalTok{)}

\CommentTok{\# standardisation {-}{-}{-}{-}}
\ControlFlowTok{if}\NormalTok{(}\SpecialCharTok{!}\FunctionTok{require}\NormalTok{(terra)) \{}\FunctionTok{install.packages}\NormalTok{(}\StringTok{"terra"}\NormalTok{); }\FunctionTok{require}\NormalTok{(terra)\}}
\ControlFlowTok{if}\NormalTok{(}\SpecialCharTok{!}\FunctionTok{require}\NormalTok{(tidyverse)) \{}\FunctionTok{install.packages}\NormalTok{(}\StringTok{"tidyverse"}\NormalTok{); }\FunctionTok{require}\NormalTok{(tidyverse)\}}

\NormalTok{nosaukums}\OtherTok{=}\StringTok{"FarmlandCrops\_CropsWinter\_r10000.tif"}
\NormalTok{ielasisanas\_cels}\OtherTok{=}\FunctionTok{paste0}\NormalTok{(}\StringTok{"./RasterGrids\_100m/2024/RAW/"}\NormalTok{,nosaukums)}
\NormalTok{saglabasanas\_cels}\OtherTok{=}\FunctionTok{paste0}\NormalTok{(}\StringTok{"./RasterGrids\_100m/2024/Scaled/"}\NormalTok{,nosaukums)}
\NormalTok{slanis}\OtherTok{=}\FunctionTok{rast}\NormalTok{(ielasisanas\_cels)}
\NormalTok{videjais}\OtherTok{=}\FunctionTok{global}\NormalTok{(slanis,}\AttributeTok{fun=}\StringTok{"mean"}\NormalTok{,}\AttributeTok{na.rm=}\ConstantTok{TRUE}\NormalTok{)}
\NormalTok{centrets}\OtherTok{=}\NormalTok{slanis}\SpecialCharTok{{-}}\NormalTok{videjais[,}\DecValTok{1}\NormalTok{]}
\NormalTok{standartnovirze}\OtherTok{=}\NormalTok{terra}\SpecialCharTok{::}\FunctionTok{global}\NormalTok{(centrets,}\AttributeTok{fun=}\StringTok{"rms"}\NormalTok{,}\AttributeTok{na.rm=}\ConstantTok{TRUE}\NormalTok{)}
\NormalTok{merogots}\OtherTok{=}\NormalTok{centrets}\SpecialCharTok{/}\NormalTok{standartnovirze[,}\DecValTok{1}\NormalTok{]}
\FunctionTok{writeRaster}\NormalTok{(merogots,}
      \AttributeTok{filename=}\NormalTok{saglabasanas\_cels,}
      \AttributeTok{overwrite=}\ConstantTok{TRUE}\NormalTok{)}
\end{Highlighting}
\end{Shaded}

\section{FarmlandCrops\_RapeseedsSpring\_cell}\label{ch06.205}

\textbf{filename:} \texttt{FarmlandCrops\_RapeseedsSpring\_cell.tif}

\textbf{layername:} \texttt{egv\_205}

\textbf{English name:} Fractional cover of Spring Sown Rapeseed, Turnip, Corn within
the analysis cell (1 ha)

\textbf{Latvian name:} Vasaras rapša, ripša, kukurūzas platība analīzes šūnā (1 ha)

\textbf{Procedure:} First, agricultural parcels declared as spring sown
rapeseed, turnip or corn are selected from the \hyperref[Ch04.02]{Rural Support Service's information
on declared fields}. These geometries are then rasterised to input
resolution, ensuring value 1 at the polygon locations and value 0 elsewhere.
Rasterisation is performed using the workflow \texttt{egvtools::polygon2input()}. Once
rasterised, the layer is aggregated to EGV
resolution using the workflow \texttt{egvtools::input2egv()}, which calculates the arithmetic mean and thus
results in a cover fraction. During aggregation, inverse
distance weighted (power = 2) gap filling on the output is applied to
ensure no missing values at the edges. Finally, the layer is standardised
by subtracting the arithmetic mean and dividing by the root mean squared error.

\begin{Shaded}
\begin{Highlighting}[]
\CommentTok{\# libs {-}{-}{-}{-}}
\ControlFlowTok{if}\NormalTok{(}\SpecialCharTok{!}\FunctionTok{require}\NormalTok{(egvtools)) \{remotes}\SpecialCharTok{::}\FunctionTok{install\_github}\NormalTok{(}\StringTok{"aavotins/egvtools"}\NormalTok{); }\FunctionTok{require}\NormalTok{(egvtools)\}}
\ControlFlowTok{if}\NormalTok{(}\SpecialCharTok{!}\FunctionTok{require}\NormalTok{(terra)) \{}\FunctionTok{install.packages}\NormalTok{(}\StringTok{"terra"}\NormalTok{); }\FunctionTok{require}\NormalTok{(terra)\}}
\ControlFlowTok{if}\NormalTok{(}\SpecialCharTok{!}\FunctionTok{require}\NormalTok{(sf)) \{}\FunctionTok{install.packages}\NormalTok{(}\StringTok{"sf"}\NormalTok{); }\FunctionTok{require}\NormalTok{(sf)\}}
\ControlFlowTok{if}\NormalTok{(}\SpecialCharTok{!}\FunctionTok{require}\NormalTok{(tidyverse)) \{}\FunctionTok{install.packages}\NormalTok{(}\StringTok{"tidyverse"}\NormalTok{); }\FunctionTok{require}\NormalTok{(tidyverse)\}}
\ControlFlowTok{if}\NormalTok{(}\SpecialCharTok{!}\FunctionTok{require}\NormalTok{(sfarrow)) \{}\FunctionTok{install.packages}\NormalTok{(}\StringTok{"sfarrow"}\NormalTok{); }\FunctionTok{require}\NormalTok{(sfarrow)\}}
\ControlFlowTok{if}\NormalTok{(}\SpecialCharTok{!}\FunctionTok{require}\NormalTok{(readxl)) \{}\FunctionTok{install.packages}\NormalTok{(}\StringTok{"readxl"}\NormalTok{); }\FunctionTok{require}\NormalTok{(readxl)\}}
\ControlFlowTok{if}\NormalTok{(}\SpecialCharTok{!}\FunctionTok{require}\NormalTok{(raster)) \{}\FunctionTok{install.packages}\NormalTok{(}\StringTok{"raster"}\NormalTok{); }\FunctionTok{require}\NormalTok{(raster)\}}
\ControlFlowTok{if}\NormalTok{(}\SpecialCharTok{!}\FunctionTok{require}\NormalTok{(fasterize)) \{}\FunctionTok{install.packages}\NormalTok{(}\StringTok{"fasterize"}\NormalTok{); }\FunctionTok{require}\NormalTok{(fasterize)\}}

\CommentTok{\# templates {-}{-}{-}{-}}
\NormalTok{template100}\OtherTok{=}\FunctionTok{rast}\NormalTok{(}\StringTok{"./Templates/TemplateRasters/LV100m\_10km.tif"}\NormalTok{)}
\NormalTok{template10}\OtherTok{=}\FunctionTok{rast}\NormalTok{(}\StringTok{"./Templates/TemplateRasters/LV10m\_10km.tif"}\NormalTok{)}
\NormalTok{rastrs10}\OtherTok{=}\FunctionTok{raster}\NormalTok{(template10)}

\NormalTok{nulls10}\OtherTok{=}\FunctionTok{rast}\NormalTok{(}\StringTok{"./Templates/TemplateRasters/nulls\_LV10m\_10km.tif"}\NormalTok{)}
\NormalTok{nulls100}\OtherTok{=}\FunctionTok{rast}\NormalTok{(}\StringTok{"./Templates/TemplateRasters/nulls\_LV100m\_10km.tif"}\NormalTok{)}

\CommentTok{\# codes {-}{-}{-}{-}}
\NormalTok{kodi}\OtherTok{=}\FunctionTok{read\_excel}\NormalTok{(}\StringTok{"./Geodata/2024/LAD/KulturuKodi\_2024.xlsx"}\NormalTok{)}
\NormalTok{kodi}\SpecialCharTok{$}\NormalTok{kods}\OtherTok{=}\FunctionTok{as.character}\NormalTok{(kodi}\SpecialCharTok{$}\NormalTok{kods)}
\CommentTok{\# LAD {-}{-}{-}{-}}
\NormalTok{lad}\OtherTok{=}\NormalTok{sfarrow}\SpecialCharTok{::}\FunctionTok{st\_read\_parquet}\NormalTok{(}\StringTok{"./Geodata/2024/LAD/Lauki\_2024.parquet"}\NormalTok{)}
\NormalTok{lad}\SpecialCharTok{$}\NormalTok{yes}\OtherTok{=}\DecValTok{1}
\NormalTok{lad}\OtherTok{=}\NormalTok{lad }\SpecialCharTok{\%\textgreater{}\%} 
 \FunctionTok{left\_join}\NormalTok{(kodi,}\AttributeTok{by=}\FunctionTok{c}\NormalTok{(}\StringTok{"PRODUCT\_CODE"}\OtherTok{=}\StringTok{"kods"}\NormalTok{))}

\CommentTok{\# simple landscape {-}{-}{-}{-}}
\NormalTok{simple\_landscape}\OtherTok{=}\FunctionTok{rast}\NormalTok{(}\StringTok{"RasterGrids\_10m/2024/Ainava\_vienk\_mask.tif"}\NormalTok{)}


\CommentTok{\# FarmlandCrops\_RapeseedsSpring\_cell.tif    egv\_205 {-}{-}{-}{-}}
\NormalTok{dati}\OtherTok{=}\NormalTok{lad }\SpecialCharTok{\%\textgreater{}\%} 
 \FunctionTok{filter}\NormalTok{(}\FunctionTok{str\_detect}\NormalTok{(SDM\_grupa\_sakums,}\StringTok{"vasaras rapsis"}\NormalTok{))}
\FunctionTok{table}\NormalTok{(dati}\SpecialCharTok{$}\NormalTok{SDM\_grupa\_sakums,}\AttributeTok{useNA=}\StringTok{"always"}\NormalTok{)}


\NormalTok{p2i\_rez}\OtherTok{=}\NormalTok{egvtools}\SpecialCharTok{::}\FunctionTok{polygon2input}\NormalTok{(}\AttributeTok{vector\_data =}\NormalTok{ dati,}
                \AttributeTok{template\_path =} \StringTok{"./Templates/TemplateRasters/LV10m\_10km.tif"}\NormalTok{,}
                \AttributeTok{out\_path =} \StringTok{"./RasterGrids\_10m/2024/"}\NormalTok{,}
                \AttributeTok{file\_name =} \StringTok{"FarmlandCrops\_RapeseedsSpring\_input.tif"}\NormalTok{,}
                \AttributeTok{value\_field =} \StringTok{"yes"}\NormalTok{,}
                \AttributeTok{prepare=}\ConstantTok{FALSE}\NormalTok{,}
                \AttributeTok{background\_raster =} \StringTok{"./Templates/TemplateRasters/nulls\_LV10m\_10km.tif"}\NormalTok{,}
                \AttributeTok{plot\_result =} \ConstantTok{TRUE}\NormalTok{)}
\NormalTok{p2i\_rez}
\NormalTok{i2e\_rez}\OtherTok{=}\NormalTok{egvtools}\SpecialCharTok{::}\FunctionTok{input2egv}\NormalTok{(}\AttributeTok{input=}\FunctionTok{paste0}\NormalTok{(}\StringTok{"./RasterGrids\_10m/2024/"}\NormalTok{,}
                     \StringTok{"FarmlandCrops\_RapeseedsSpring\_input.tif"}\NormalTok{),}
              \AttributeTok{egv\_template=} \StringTok{"./Templates/TemplateRasters/LV100m\_10km.tif"}\NormalTok{,}
              \AttributeTok{summary\_function =} \StringTok{"average"}\NormalTok{,}
              \AttributeTok{missing\_job =} \StringTok{"FillOutput"}\NormalTok{,}
              \AttributeTok{outlocation =} \StringTok{"./RasterGrids\_100m/2024/RAW/"}\NormalTok{,}
              \AttributeTok{outfilename =} \StringTok{"FarmlandCrops\_RapeseedsSpring\_cell.tif"}\NormalTok{,}
              \AttributeTok{layername =} \StringTok{"egv\_205"}\NormalTok{,}
              \AttributeTok{idw\_weight =} \DecValTok{2}\NormalTok{,}
              \AttributeTok{plot\_gaps =} \ConstantTok{FALSE}\NormalTok{,}\AttributeTok{plot\_final =} \ConstantTok{TRUE}\NormalTok{)}
\NormalTok{i2e\_rez}
\FunctionTok{rm}\NormalTok{(p2i\_rez)}
\FunctionTok{rm}\NormalTok{(i2e\_rez)}
\FunctionTok{rm}\NormalTok{(dati)}
\FunctionTok{unlink}\NormalTok{(}\StringTok{"./RasterGrids\_10m/2024/FarmlandCrops\_RapeseedsSpring\_input.tif"}\NormalTok{)}



\CommentTok{\# standardisation {-}{-}{-}{-}}
\ControlFlowTok{if}\NormalTok{(}\SpecialCharTok{!}\FunctionTok{require}\NormalTok{(terra)) \{}\FunctionTok{install.packages}\NormalTok{(}\StringTok{"terra"}\NormalTok{); }\FunctionTok{require}\NormalTok{(terra)\}}
\ControlFlowTok{if}\NormalTok{(}\SpecialCharTok{!}\FunctionTok{require}\NormalTok{(tidyverse)) \{}\FunctionTok{install.packages}\NormalTok{(}\StringTok{"tidyverse"}\NormalTok{); }\FunctionTok{require}\NormalTok{(tidyverse)\}}

\NormalTok{nosaukums}\OtherTok{=}\StringTok{"FarmlandCrops\_RapeseedsSpring\_cell.tif"}
\NormalTok{ielasisanas\_cels}\OtherTok{=}\FunctionTok{paste0}\NormalTok{(}\StringTok{"./RasterGrids\_100m/2024/RAW/"}\NormalTok{,nosaukums)}
\NormalTok{saglabasanas\_cels}\OtherTok{=}\FunctionTok{paste0}\NormalTok{(}\StringTok{"./RasterGrids\_100m/2024/Scaled/"}\NormalTok{,nosaukums)}
\NormalTok{slanis}\OtherTok{=}\FunctionTok{rast}\NormalTok{(ielasisanas\_cels)}
\NormalTok{videjais}\OtherTok{=}\FunctionTok{global}\NormalTok{(slanis,}\AttributeTok{fun=}\StringTok{"mean"}\NormalTok{,}\AttributeTok{na.rm=}\ConstantTok{TRUE}\NormalTok{)}
\NormalTok{centrets}\OtherTok{=}\NormalTok{slanis}\SpecialCharTok{{-}}\NormalTok{videjais[,}\DecValTok{1}\NormalTok{]}
\NormalTok{standartnovirze}\OtherTok{=}\NormalTok{terra}\SpecialCharTok{::}\FunctionTok{global}\NormalTok{(centrets,}\AttributeTok{fun=}\StringTok{"rms"}\NormalTok{,}\AttributeTok{na.rm=}\ConstantTok{TRUE}\NormalTok{)}
\NormalTok{merogots}\OtherTok{=}\NormalTok{centrets}\SpecialCharTok{/}\NormalTok{standartnovirze[,}\DecValTok{1}\NormalTok{]}
\FunctionTok{writeRaster}\NormalTok{(merogots,}
      \AttributeTok{filename=}\NormalTok{saglabasanas\_cels,}
      \AttributeTok{overwrite=}\ConstantTok{TRUE}\NormalTok{)}
\end{Highlighting}
\end{Shaded}

\section{FarmlandCrops\_RapeseedsSpring\_r500}\label{ch06.206}

\textbf{filename:} \texttt{FarmlandCrops\_RapeseedsSpring\_r500.tif}

\textbf{layername:} \texttt{egv\_206}

\textbf{English name:} Fractional cover of Spring Sown Rapeseed, Turnip, Corn within
the 0.5 km landscape

\textbf{Latvian name:} Vasaras rapša, ripša, kukurūzas platība 0,5 km ainavā

\textbf{Procedure:} The cover fraction within a radius of 500 m around the analysis grid cell is
calculated as the area-weighted sum of the \hyperref[ch06.205]{analysis cells} inside the
buffer, using the workflow \texttt{egvtools::radius\_function()}. During the calculation of the landscape metric,
inverse distance weighted (power = 2) gap filling on the output is applied
to ensure no missing values at the edges. Then the layer is rewritten to set
its name. Finally, the layer is standardised by subtracting the arithmetic
mean and dividing by the root mean squared error.

\begin{Shaded}
\begin{Highlighting}[]
\CommentTok{\# libs {-}{-}{-}{-}}
\ControlFlowTok{if}\NormalTok{(}\SpecialCharTok{!}\FunctionTok{require}\NormalTok{(terra)) \{}\FunctionTok{install.packages}\NormalTok{(}\StringTok{"terra"}\NormalTok{); }\FunctionTok{require}\NormalTok{(terra)\}}
\ControlFlowTok{if}\NormalTok{(}\SpecialCharTok{!}\FunctionTok{require}\NormalTok{(egvtools)) \{remotes}\SpecialCharTok{::}\FunctionTok{install\_github}\NormalTok{(}\StringTok{"aavotins/egvtools"}\NormalTok{); }\FunctionTok{require}\NormalTok{(egvtools)\}}


\CommentTok{\# Templates {-}{-}{-}{-}{-}}
\NormalTok{template100}\OtherTok{=}\FunctionTok{rast}\NormalTok{(}\StringTok{"./Templates/TemplateRasters/LV100m\_10km.tif"}\NormalTok{)}

\CommentTok{\# radii {-}{-}{-}{-}}
\FunctionTok{radius\_function}\NormalTok{(}
 \AttributeTok{kvadrati\_path =} \StringTok{"./Templates/TemplateGrids/tiles/"}\NormalTok{,}
 \AttributeTok{radii\_path   =} \StringTok{"./Templates/TemplateGridPoints/tiles/"}\NormalTok{,}
 \AttributeTok{tikls100\_path =} \StringTok{"./Templates/TemplateGrids/tikls100\_sauzeme.parquet"}\NormalTok{,}
 \AttributeTok{template\_path =} \StringTok{"./Templates/TemplateRasters/LV100m\_10km.tif"}\NormalTok{,}
 \AttributeTok{input\_layers  =} \FunctionTok{c}\NormalTok{(}\StringTok{"./RasterGrids\_100m/2024/RAW/FarmlandCrops\_RapeseedsSpring\_cell.tif"}\NormalTok{),}
 \AttributeTok{layer\_prefixes =} \FunctionTok{c}\NormalTok{(}\StringTok{"FarmlandCrops\_RapeseedsSpring"}\NormalTok{),}
 \AttributeTok{output\_dir   =} \StringTok{"./RasterGrids\_100m/2024/RAW/"}\NormalTok{,}
 \AttributeTok{n\_workers   =} \DecValTok{6}\NormalTok{,}
 \AttributeTok{radii     =} \FunctionTok{c}\NormalTok{(}\StringTok{"r500"}\NormalTok{),}
 \AttributeTok{radius\_mode  =} \StringTok{"sparse"}\NormalTok{,}
 \AttributeTok{extract\_fun  =} \StringTok{"mean"}\NormalTok{,}
 \AttributeTok{fill\_missing  =} \ConstantTok{TRUE}\NormalTok{,}
 \AttributeTok{IDW\_weight   =} \DecValTok{2}\NormalTok{,}
 \AttributeTok{future\_max\_size =} \DecValTok{40} \SpecialCharTok{*} \DecValTok{1024}\SpecialCharTok{\^{}}\DecValTok{3}\NormalTok{)}


\CommentTok{\# FarmlandCrops\_RapeseedsSpring\_r500.tif    egv\_206}
\NormalTok{slanis}\OtherTok{=}\FunctionTok{rast}\NormalTok{(}\StringTok{"./RasterGrids\_100m/2024/RAW/FarmlandCrops\_RapeseedsSpring\_r500.tif"}\NormalTok{)}
\FunctionTok{names}\NormalTok{(slanis)}\OtherTok{=}\StringTok{"egv\_206"}
\NormalTok{slanis2}\OtherTok{=}\FunctionTok{project}\NormalTok{(slanis,template100)}
\FunctionTok{writeRaster}\NormalTok{(slanis2,}
      \StringTok{"./RasterGrids\_100m/2024/RAW/FarmlandCrops\_RapeseedsSpring\_r500.tif"}\NormalTok{,}
      \AttributeTok{overwrite=}\ConstantTok{TRUE}\NormalTok{)}

\CommentTok{\# standardisation {-}{-}{-}{-}}
\ControlFlowTok{if}\NormalTok{(}\SpecialCharTok{!}\FunctionTok{require}\NormalTok{(terra)) \{}\FunctionTok{install.packages}\NormalTok{(}\StringTok{"terra"}\NormalTok{); }\FunctionTok{require}\NormalTok{(terra)\}}
\ControlFlowTok{if}\NormalTok{(}\SpecialCharTok{!}\FunctionTok{require}\NormalTok{(tidyverse)) \{}\FunctionTok{install.packages}\NormalTok{(}\StringTok{"tidyverse"}\NormalTok{); }\FunctionTok{require}\NormalTok{(tidyverse)\}}

\NormalTok{nosaukums}\OtherTok{=}\StringTok{"FarmlandCrops\_RapeseedsSpring\_r500.tif"}
\NormalTok{ielasisanas\_cels}\OtherTok{=}\FunctionTok{paste0}\NormalTok{(}\StringTok{"./RasterGrids\_100m/2024/RAW/"}\NormalTok{,nosaukums)}
\NormalTok{saglabasanas\_cels}\OtherTok{=}\FunctionTok{paste0}\NormalTok{(}\StringTok{"./RasterGrids\_100m/2024/Scaled/"}\NormalTok{,nosaukums)}
\NormalTok{slanis}\OtherTok{=}\FunctionTok{rast}\NormalTok{(ielasisanas\_cels)}
\NormalTok{videjais}\OtherTok{=}\FunctionTok{global}\NormalTok{(slanis,}\AttributeTok{fun=}\StringTok{"mean"}\NormalTok{,}\AttributeTok{na.rm=}\ConstantTok{TRUE}\NormalTok{)}
\NormalTok{centrets}\OtherTok{=}\NormalTok{slanis}\SpecialCharTok{{-}}\NormalTok{videjais[,}\DecValTok{1}\NormalTok{]}
\NormalTok{standartnovirze}\OtherTok{=}\NormalTok{terra}\SpecialCharTok{::}\FunctionTok{global}\NormalTok{(centrets,}\AttributeTok{fun=}\StringTok{"rms"}\NormalTok{,}\AttributeTok{na.rm=}\ConstantTok{TRUE}\NormalTok{)}
\NormalTok{merogots}\OtherTok{=}\NormalTok{centrets}\SpecialCharTok{/}\NormalTok{standartnovirze[,}\DecValTok{1}\NormalTok{]}
\FunctionTok{writeRaster}\NormalTok{(merogots,}
      \AttributeTok{filename=}\NormalTok{saglabasanas\_cels,}
      \AttributeTok{overwrite=}\ConstantTok{TRUE}\NormalTok{)}
\end{Highlighting}
\end{Shaded}

\section{FarmlandCrops\_RapeseedsSpring\_r1250}\label{ch06.207}

\textbf{filename:} \texttt{FarmlandCrops\_RapeseedsSpring\_r1250.tif}

\textbf{layername:} \texttt{egv\_207}

\textbf{English name:} Fractional cover of Spring Sown Rapeseed, Turnip, Corn within
the 1.25 km landscape

\textbf{Latvian name:} Vasaras rapša, ripša, kukurūzas platība 1,25 km ainavā

\textbf{Procedure:} The cover fraction within a radius of 1250 m around the analysis grid cell
is calculated as the area-weighted sum of the \hyperref[ch06.205]{analysis cells} inside
the buffer, using the workflow \texttt{egvtools::radius\_function()}. During the calculation of the landscape
metric, inverse distance weighted (power = 2) gap filling on the output is
applied to ensure no missing values at the edges. Then the layer is
rewritten to set its name. Finally, the layer is standardised by
subtracting the arithmetic mean and dividing by the root mean squared error.

\begin{Shaded}
\begin{Highlighting}[]
\CommentTok{\# libs {-}{-}{-}{-}}
\ControlFlowTok{if}\NormalTok{(}\SpecialCharTok{!}\FunctionTok{require}\NormalTok{(terra)) \{}\FunctionTok{install.packages}\NormalTok{(}\StringTok{"terra"}\NormalTok{); }\FunctionTok{require}\NormalTok{(terra)\}}
\ControlFlowTok{if}\NormalTok{(}\SpecialCharTok{!}\FunctionTok{require}\NormalTok{(egvtools)) \{remotes}\SpecialCharTok{::}\FunctionTok{install\_github}\NormalTok{(}\StringTok{"aavotins/egvtools"}\NormalTok{); }\FunctionTok{require}\NormalTok{(egvtools)\}}


\CommentTok{\# Templates {-}{-}{-}{-}{-}}
\NormalTok{template100}\OtherTok{=}\FunctionTok{rast}\NormalTok{(}\StringTok{"./Templates/TemplateRasters/LV100m\_10km.tif"}\NormalTok{)}

\CommentTok{\# radii {-}{-}{-}{-}}
\FunctionTok{radius\_function}\NormalTok{(}
 \AttributeTok{kvadrati\_path =} \StringTok{"./Templates/TemplateGrids/tiles/"}\NormalTok{,}
 \AttributeTok{radii\_path   =} \StringTok{"./Templates/TemplateGridPoints/tiles/"}\NormalTok{,}
 \AttributeTok{tikls100\_path =} \StringTok{"./Templates/TemplateGrids/tikls100\_sauzeme.parquet"}\NormalTok{,}
 \AttributeTok{template\_path =} \StringTok{"./Templates/TemplateRasters/LV100m\_10km.tif"}\NormalTok{,}
 \AttributeTok{input\_layers  =} \FunctionTok{c}\NormalTok{(}\StringTok{"./RasterGrids\_100m/2024/RAW/FarmlandCrops\_RapeseedsSpring\_cell.tif"}\NormalTok{),}
 \AttributeTok{layer\_prefixes =} \FunctionTok{c}\NormalTok{(}\StringTok{"FarmlandCrops\_RapeseedsSpring"}\NormalTok{),}
 \AttributeTok{output\_dir   =} \StringTok{"./RasterGrids\_100m/2024/RAW/"}\NormalTok{,}
 \AttributeTok{n\_workers   =} \DecValTok{6}\NormalTok{,}
 \AttributeTok{radii     =} \FunctionTok{c}\NormalTok{(}\StringTok{"r1250"}\NormalTok{),}
 \AttributeTok{radius\_mode  =} \StringTok{"sparse"}\NormalTok{,}
 \AttributeTok{extract\_fun  =} \StringTok{"mean"}\NormalTok{,}
 \AttributeTok{fill\_missing  =} \ConstantTok{TRUE}\NormalTok{,}
 \AttributeTok{IDW\_weight   =} \DecValTok{2}\NormalTok{,}
 \AttributeTok{future\_max\_size =} \DecValTok{40} \SpecialCharTok{*} \DecValTok{1024}\SpecialCharTok{\^{}}\DecValTok{3}\NormalTok{)}


\CommentTok{\# FarmlandCrops\_RapeseedsSpring\_r1250.tif   egv\_207}
\NormalTok{slanis}\OtherTok{=}\FunctionTok{rast}\NormalTok{(}\StringTok{"./RasterGrids\_100m/2024/RAW/FarmlandCrops\_RapeseedsSpring\_r1250.tif"}\NormalTok{)}
\FunctionTok{names}\NormalTok{(slanis)}\OtherTok{=}\StringTok{"egv\_207"}
\NormalTok{slanis2}\OtherTok{=}\FunctionTok{project}\NormalTok{(slanis,template100)}
\FunctionTok{writeRaster}\NormalTok{(slanis2,}
      \StringTok{"./RasterGrids\_100m/2024/RAW/FarmlandCrops\_RapeseedsSpring\_r1250.tif"}\NormalTok{,}
      \AttributeTok{overwrite=}\ConstantTok{TRUE}\NormalTok{)}

\CommentTok{\# standardisation {-}{-}{-}{-}}
\ControlFlowTok{if}\NormalTok{(}\SpecialCharTok{!}\FunctionTok{require}\NormalTok{(terra)) \{}\FunctionTok{install.packages}\NormalTok{(}\StringTok{"terra"}\NormalTok{); }\FunctionTok{require}\NormalTok{(terra)\}}
\ControlFlowTok{if}\NormalTok{(}\SpecialCharTok{!}\FunctionTok{require}\NormalTok{(tidyverse)) \{}\FunctionTok{install.packages}\NormalTok{(}\StringTok{"tidyverse"}\NormalTok{); }\FunctionTok{require}\NormalTok{(tidyverse)\}}

\NormalTok{nosaukums}\OtherTok{=}\StringTok{"FarmlandCrops\_RapeseedsSpring\_r1250.tif"}
\NormalTok{ielasisanas\_cels}\OtherTok{=}\FunctionTok{paste0}\NormalTok{(}\StringTok{"./RasterGrids\_100m/2024/RAW/"}\NormalTok{,nosaukums)}
\NormalTok{saglabasanas\_cels}\OtherTok{=}\FunctionTok{paste0}\NormalTok{(}\StringTok{"./RasterGrids\_100m/2024/Scaled/"}\NormalTok{,nosaukums)}
\NormalTok{slanis}\OtherTok{=}\FunctionTok{rast}\NormalTok{(ielasisanas\_cels)}
\NormalTok{videjais}\OtherTok{=}\FunctionTok{global}\NormalTok{(slanis,}\AttributeTok{fun=}\StringTok{"mean"}\NormalTok{,}\AttributeTok{na.rm=}\ConstantTok{TRUE}\NormalTok{)}
\NormalTok{centrets}\OtherTok{=}\NormalTok{slanis}\SpecialCharTok{{-}}\NormalTok{videjais[,}\DecValTok{1}\NormalTok{]}
\NormalTok{standartnovirze}\OtherTok{=}\NormalTok{terra}\SpecialCharTok{::}\FunctionTok{global}\NormalTok{(centrets,}\AttributeTok{fun=}\StringTok{"rms"}\NormalTok{,}\AttributeTok{na.rm=}\ConstantTok{TRUE}\NormalTok{)}
\NormalTok{merogots}\OtherTok{=}\NormalTok{centrets}\SpecialCharTok{/}\NormalTok{standartnovirze[,}\DecValTok{1}\NormalTok{]}
\FunctionTok{writeRaster}\NormalTok{(merogots,}
      \AttributeTok{filename=}\NormalTok{saglabasanas\_cels,}
      \AttributeTok{overwrite=}\ConstantTok{TRUE}\NormalTok{)}
\end{Highlighting}
\end{Shaded}

\section{FarmlandCrops\_RapeseedsSpring\_r3000}\label{ch06.208}

\textbf{filename:} \texttt{FarmlandCrops\_RapeseedsSpring\_r3000.tif}

\textbf{layername:} \texttt{egv\_208}

\textbf{English name:} Fractional cover of Spring Sown Rapeseed, Turnip, Corn within
the 3 km landscape

\textbf{Latvian name:} Vasaras rapša, ripša, kukurūzas platība 3 km ainavā

\textbf{Procedure:} The cover fraction within a radius of 3000 m around the analysis grid cell
is calculated as the area-weighted sum of the \hyperref[ch06.205]{analysis cells} inside
the buffer, using the workflow \texttt{egvtools::radius\_function()}. During the calculation of the landscape
metric, inverse distance weighted (power = 2) gap filling on the output is
applied to ensure no missing values at the edges. Then the layer is
rewritten to set its name. Finally, the layer is standardised by
subtracting the arithmetic mean and dividing by the root mean squared error.

\begin{Shaded}
\begin{Highlighting}[]
\CommentTok{\# libs {-}{-}{-}{-}}
\ControlFlowTok{if}\NormalTok{(}\SpecialCharTok{!}\FunctionTok{require}\NormalTok{(terra)) \{}\FunctionTok{install.packages}\NormalTok{(}\StringTok{"terra"}\NormalTok{); }\FunctionTok{require}\NormalTok{(terra)\}}
\ControlFlowTok{if}\NormalTok{(}\SpecialCharTok{!}\FunctionTok{require}\NormalTok{(egvtools)) \{remotes}\SpecialCharTok{::}\FunctionTok{install\_github}\NormalTok{(}\StringTok{"aavotins/egvtools"}\NormalTok{); }\FunctionTok{require}\NormalTok{(egvtools)\}}


\CommentTok{\# Templates {-}{-}{-}{-}{-}}
\NormalTok{template100}\OtherTok{=}\FunctionTok{rast}\NormalTok{(}\StringTok{"./Templates/TemplateRasters/LV100m\_10km.tif"}\NormalTok{)}

\CommentTok{\# radii {-}{-}{-}{-}}
\FunctionTok{radius\_function}\NormalTok{(}
 \AttributeTok{kvadrati\_path =} \StringTok{"./Templates/TemplateGrids/tiles/"}\NormalTok{,}
 \AttributeTok{radii\_path   =} \StringTok{"./Templates/TemplateGridPoints/tiles/"}\NormalTok{,}
 \AttributeTok{tikls100\_path =} \StringTok{"./Templates/TemplateGrids/tikls100\_sauzeme.parquet"}\NormalTok{,}
 \AttributeTok{template\_path =} \StringTok{"./Templates/TemplateRasters/LV100m\_10km.tif"}\NormalTok{,}
 \AttributeTok{input\_layers  =} \FunctionTok{c}\NormalTok{(}\StringTok{"./RasterGrids\_100m/2024/RAW/FarmlandCrops\_RapeseedsSpring\_cell.tif"}\NormalTok{),}
 \AttributeTok{layer\_prefixes =} \FunctionTok{c}\NormalTok{(}\StringTok{"FarmlandCrops\_RapeseedsSpring"}\NormalTok{),}
 \AttributeTok{output\_dir   =} \StringTok{"./RasterGrids\_100m/2024/RAW/"}\NormalTok{,}
 \AttributeTok{n\_workers   =} \DecValTok{6}\NormalTok{,}
 \AttributeTok{radii     =} \FunctionTok{c}\NormalTok{(}\StringTok{"r3000"}\NormalTok{),}
 \AttributeTok{radius\_mode  =} \StringTok{"sparse"}\NormalTok{,}
 \AttributeTok{extract\_fun  =} \StringTok{"mean"}\NormalTok{,}
 \AttributeTok{fill\_missing  =} \ConstantTok{TRUE}\NormalTok{,}
 \AttributeTok{IDW\_weight   =} \DecValTok{2}\NormalTok{,}
 \AttributeTok{future\_max\_size =} \DecValTok{40} \SpecialCharTok{*} \DecValTok{1024}\SpecialCharTok{\^{}}\DecValTok{3}\NormalTok{)}


\CommentTok{\# FarmlandCrops\_RapeseedsSpring\_r3000.tif   egv\_208}
\NormalTok{slanis}\OtherTok{=}\FunctionTok{rast}\NormalTok{(}\StringTok{"./RasterGrids\_100m/2024/RAW/FarmlandCrops\_RapeseedsSpring\_r3000.tif"}\NormalTok{)}
\FunctionTok{names}\NormalTok{(slanis)}\OtherTok{=}\StringTok{"egv\_208"}
\NormalTok{slanis2}\OtherTok{=}\FunctionTok{project}\NormalTok{(slanis,template100)}
\FunctionTok{writeRaster}\NormalTok{(slanis2,}
      \StringTok{"./RasterGrids\_100m/2024/RAW/FarmlandCrops\_RapeseedsSpring\_r3000.tif"}\NormalTok{,}
      \AttributeTok{overwrite=}\ConstantTok{TRUE}\NormalTok{)}

\CommentTok{\# standardisation {-}{-}{-}{-}}
\ControlFlowTok{if}\NormalTok{(}\SpecialCharTok{!}\FunctionTok{require}\NormalTok{(terra)) \{}\FunctionTok{install.packages}\NormalTok{(}\StringTok{"terra"}\NormalTok{); }\FunctionTok{require}\NormalTok{(terra)\}}
\ControlFlowTok{if}\NormalTok{(}\SpecialCharTok{!}\FunctionTok{require}\NormalTok{(tidyverse)) \{}\FunctionTok{install.packages}\NormalTok{(}\StringTok{"tidyverse"}\NormalTok{); }\FunctionTok{require}\NormalTok{(tidyverse)\}}

\NormalTok{nosaukums}\OtherTok{=}\StringTok{"FarmlandCrops\_RapeseedsSpring\_r3000.tif"}
\NormalTok{ielasisanas\_cels}\OtherTok{=}\FunctionTok{paste0}\NormalTok{(}\StringTok{"./RasterGrids\_100m/2024/RAW/"}\NormalTok{,nosaukums)}
\NormalTok{saglabasanas\_cels}\OtherTok{=}\FunctionTok{paste0}\NormalTok{(}\StringTok{"./RasterGrids\_100m/2024/Scaled/"}\NormalTok{,nosaukums)}
\NormalTok{slanis}\OtherTok{=}\FunctionTok{rast}\NormalTok{(ielasisanas\_cels)}
\NormalTok{videjais}\OtherTok{=}\FunctionTok{global}\NormalTok{(slanis,}\AttributeTok{fun=}\StringTok{"mean"}\NormalTok{,}\AttributeTok{na.rm=}\ConstantTok{TRUE}\NormalTok{)}
\NormalTok{centrets}\OtherTok{=}\NormalTok{slanis}\SpecialCharTok{{-}}\NormalTok{videjais[,}\DecValTok{1}\NormalTok{]}
\NormalTok{standartnovirze}\OtherTok{=}\NormalTok{terra}\SpecialCharTok{::}\FunctionTok{global}\NormalTok{(centrets,}\AttributeTok{fun=}\StringTok{"rms"}\NormalTok{,}\AttributeTok{na.rm=}\ConstantTok{TRUE}\NormalTok{)}
\NormalTok{merogots}\OtherTok{=}\NormalTok{centrets}\SpecialCharTok{/}\NormalTok{standartnovirze[,}\DecValTok{1}\NormalTok{]}
\FunctionTok{writeRaster}\NormalTok{(merogots,}
      \AttributeTok{filename=}\NormalTok{saglabasanas\_cels,}
      \AttributeTok{overwrite=}\ConstantTok{TRUE}\NormalTok{)}
\end{Highlighting}
\end{Shaded}

\section{FarmlandCrops\_RapeseedsSpring\_r10000}\label{ch06.209}

\textbf{filename:} \texttt{FarmlandCrops\_RapeseedsSpring\_r10000.tif}

\textbf{layername:} \texttt{egv\_209}

\textbf{English name:} Fractional cover of Spring Sown Rapeseed, Turnip, Corn within
the 10 km landscape

\textbf{Latvian name:} Vasaras rapša, ripša, kukurūzas platība 10 km ainavā

\textbf{Procedure:} The cover fraction within a radius of 10000 m around the analysis grid cell
is calculated as the area-weighted sum of the \hyperref[ch06.205]{analysis cells} inside
the buffer, using the workflow \texttt{egvtools::radius\_function()}. During the calculation of the landscape
metric, inverse distance weighted (power = 2) gap filling on the output is
applied to ensure no missing values at the edges. Then the layer is
rewritten to set its name. Finally, the layer is standardised by
subtracting the arithmetic mean and dividing by the root mean squared error.

\begin{Shaded}
\begin{Highlighting}[]
\CommentTok{\# libs {-}{-}{-}{-}}
\ControlFlowTok{if}\NormalTok{(}\SpecialCharTok{!}\FunctionTok{require}\NormalTok{(terra)) \{}\FunctionTok{install.packages}\NormalTok{(}\StringTok{"terra"}\NormalTok{); }\FunctionTok{require}\NormalTok{(terra)\}}
\ControlFlowTok{if}\NormalTok{(}\SpecialCharTok{!}\FunctionTok{require}\NormalTok{(egvtools)) \{remotes}\SpecialCharTok{::}\FunctionTok{install\_github}\NormalTok{(}\StringTok{"aavotins/egvtools"}\NormalTok{); }\FunctionTok{require}\NormalTok{(egvtools)\}}


\CommentTok{\# Templates {-}{-}{-}{-}{-}}
\NormalTok{template100}\OtherTok{=}\FunctionTok{rast}\NormalTok{(}\StringTok{"./Templates/TemplateRasters/LV100m\_10km.tif"}\NormalTok{)}

\CommentTok{\# radii {-}{-}{-}{-}}
\FunctionTok{radius\_function}\NormalTok{(}
 \AttributeTok{kvadrati\_path =} \StringTok{"./Templates/TemplateGrids/tiles/"}\NormalTok{,}
 \AttributeTok{radii\_path   =} \StringTok{"./Templates/TemplateGridPoints/tiles/"}\NormalTok{,}
 \AttributeTok{tikls100\_path =} \StringTok{"./Templates/TemplateGrids/tikls100\_sauzeme.parquet"}\NormalTok{,}
 \AttributeTok{template\_path =} \StringTok{"./Templates/TemplateRasters/LV100m\_10km.tif"}\NormalTok{,}
 \AttributeTok{input\_layers  =} \FunctionTok{c}\NormalTok{(}\StringTok{"./RasterGrids\_100m/2024/RAW/FarmlandCrops\_RapeseedsSpring\_cell.tif"}\NormalTok{),}
 \AttributeTok{layer\_prefixes =} \FunctionTok{c}\NormalTok{(}\StringTok{"FarmlandCrops\_RapeseedsSpring"}\NormalTok{),}
 \AttributeTok{output\_dir   =} \StringTok{"./RasterGrids\_100m/2024/RAW/"}\NormalTok{,}
 \AttributeTok{n\_workers   =} \DecValTok{6}\NormalTok{,}
 \AttributeTok{radii     =} \FunctionTok{c}\NormalTok{(}\StringTok{"r10000"}\NormalTok{),}
 \AttributeTok{radius\_mode  =} \StringTok{"sparse"}\NormalTok{,}
 \AttributeTok{extract\_fun  =} \StringTok{"mean"}\NormalTok{,}
 \AttributeTok{fill\_missing  =} \ConstantTok{TRUE}\NormalTok{,}
 \AttributeTok{IDW\_weight   =} \DecValTok{2}\NormalTok{,}
 \AttributeTok{future\_max\_size =} \DecValTok{40} \SpecialCharTok{*} \DecValTok{1024}\SpecialCharTok{\^{}}\DecValTok{3}\NormalTok{)}


\CommentTok{\# FarmlandCrops\_RapeseedsSpring\_r10000.tif  egv\_209}
\NormalTok{slanis}\OtherTok{=}\FunctionTok{rast}\NormalTok{(}\StringTok{"./RasterGrids\_100m/2024/RAW/FarmlandCrops\_RapeseedsSpring\_r10000.tif"}\NormalTok{)}
\FunctionTok{names}\NormalTok{(slanis)}\OtherTok{=}\StringTok{"egv\_209"}
\NormalTok{slanis2}\OtherTok{=}\FunctionTok{project}\NormalTok{(slanis,template100)}
\FunctionTok{writeRaster}\NormalTok{(slanis2,}
      \StringTok{"./RasterGrids\_100m/2024/RAW/FarmlandCrops\_RapeseedsSpring\_r10000.tif"}\NormalTok{,}
      \AttributeTok{overwrite=}\ConstantTok{TRUE}\NormalTok{)}

\CommentTok{\# standardisation {-}{-}{-}{-}}
\ControlFlowTok{if}\NormalTok{(}\SpecialCharTok{!}\FunctionTok{require}\NormalTok{(terra)) \{}\FunctionTok{install.packages}\NormalTok{(}\StringTok{"terra"}\NormalTok{); }\FunctionTok{require}\NormalTok{(terra)\}}
\ControlFlowTok{if}\NormalTok{(}\SpecialCharTok{!}\FunctionTok{require}\NormalTok{(tidyverse)) \{}\FunctionTok{install.packages}\NormalTok{(}\StringTok{"tidyverse"}\NormalTok{); }\FunctionTok{require}\NormalTok{(tidyverse)\}}

\NormalTok{nosaukums}\OtherTok{=}\StringTok{"FarmlandCrops\_RapeseedsSpring\_r10000.tif"}
\NormalTok{ielasisanas\_cels}\OtherTok{=}\FunctionTok{paste0}\NormalTok{(}\StringTok{"./RasterGrids\_100m/2024/RAW/"}\NormalTok{,nosaukums)}
\NormalTok{saglabasanas\_cels}\OtherTok{=}\FunctionTok{paste0}\NormalTok{(}\StringTok{"./RasterGrids\_100m/2024/Scaled/"}\NormalTok{,nosaukums)}
\NormalTok{slanis}\OtherTok{=}\FunctionTok{rast}\NormalTok{(ielasisanas\_cels)}
\NormalTok{videjais}\OtherTok{=}\FunctionTok{global}\NormalTok{(slanis,}\AttributeTok{fun=}\StringTok{"mean"}\NormalTok{,}\AttributeTok{na.rm=}\ConstantTok{TRUE}\NormalTok{)}
\NormalTok{centrets}\OtherTok{=}\NormalTok{slanis}\SpecialCharTok{{-}}\NormalTok{videjais[,}\DecValTok{1}\NormalTok{]}
\NormalTok{standartnovirze}\OtherTok{=}\NormalTok{terra}\SpecialCharTok{::}\FunctionTok{global}\NormalTok{(centrets,}\AttributeTok{fun=}\StringTok{"rms"}\NormalTok{,}\AttributeTok{na.rm=}\ConstantTok{TRUE}\NormalTok{)}
\NormalTok{merogots}\OtherTok{=}\NormalTok{centrets}\SpecialCharTok{/}\NormalTok{standartnovirze[,}\DecValTok{1}\NormalTok{]}
\FunctionTok{writeRaster}\NormalTok{(merogots,}
      \AttributeTok{filename=}\NormalTok{saglabasanas\_cels,}
      \AttributeTok{overwrite=}\ConstantTok{TRUE}\NormalTok{)}
\end{Highlighting}
\end{Shaded}

\section{FarmlandCrops\_RapeseedsWinter\_cell}\label{ch06.210}

\textbf{filename:} \texttt{FarmlandCrops\_RapeseedsWinter\_cell.tif}

\textbf{layername:} \texttt{egv\_210}

\textbf{English name:} Fractional cover of Winter Rapeseed, Turnip within the
analysis cell (1 ha)

\textbf{Latvian name:} Ziemas rapša, ripša platības īpatsvars analīzes šūnā (1 ha)

\textbf{Procedure:} First, agricultural parcels declared as winter rapeseed or
turnip are selected from the \hyperref[Ch04.02]{Rural Support Service's information on declared
fields}. These geometries are then rasterised to input resolution,
ensuring value 1 at the polygon locations and value 0 elsewhere. Rasterisation is performed using the workflow \texttt{egvtools::polygon2input()}. Once rasterised, the layer is aggregated to EGV
resolution using the workflow \texttt{egvtools::input2egv()}, which calculates the arithmetic mean and thus
results in a cover fraction. During aggregation, inverse
distance weighted (power = 2) gap filling on the output is applied to
ensure no missing values at the edges. Finally, the layer is standardised
by subtracting the arithmetic mean and dividing by the root mean squared error.

\begin{Shaded}
\begin{Highlighting}[]
\CommentTok{\# libs {-}{-}{-}{-}}
\ControlFlowTok{if}\NormalTok{(}\SpecialCharTok{!}\FunctionTok{require}\NormalTok{(egvtools)) \{remotes}\SpecialCharTok{::}\FunctionTok{install\_github}\NormalTok{(}\StringTok{"aavotins/egvtools"}\NormalTok{); }\FunctionTok{require}\NormalTok{(egvtools)\}}
\ControlFlowTok{if}\NormalTok{(}\SpecialCharTok{!}\FunctionTok{require}\NormalTok{(terra)) \{}\FunctionTok{install.packages}\NormalTok{(}\StringTok{"terra"}\NormalTok{); }\FunctionTok{require}\NormalTok{(terra)\}}
\ControlFlowTok{if}\NormalTok{(}\SpecialCharTok{!}\FunctionTok{require}\NormalTok{(sf)) \{}\FunctionTok{install.packages}\NormalTok{(}\StringTok{"sf"}\NormalTok{); }\FunctionTok{require}\NormalTok{(sf)\}}
\ControlFlowTok{if}\NormalTok{(}\SpecialCharTok{!}\FunctionTok{require}\NormalTok{(tidyverse)) \{}\FunctionTok{install.packages}\NormalTok{(}\StringTok{"tidyverse"}\NormalTok{); }\FunctionTok{require}\NormalTok{(tidyverse)\}}
\ControlFlowTok{if}\NormalTok{(}\SpecialCharTok{!}\FunctionTok{require}\NormalTok{(sfarrow)) \{}\FunctionTok{install.packages}\NormalTok{(}\StringTok{"sfarrow"}\NormalTok{); }\FunctionTok{require}\NormalTok{(sfarrow)\}}
\ControlFlowTok{if}\NormalTok{(}\SpecialCharTok{!}\FunctionTok{require}\NormalTok{(readxl)) \{}\FunctionTok{install.packages}\NormalTok{(}\StringTok{"readxl"}\NormalTok{); }\FunctionTok{require}\NormalTok{(readxl)\}}
\ControlFlowTok{if}\NormalTok{(}\SpecialCharTok{!}\FunctionTok{require}\NormalTok{(raster)) \{}\FunctionTok{install.packages}\NormalTok{(}\StringTok{"raster"}\NormalTok{); }\FunctionTok{require}\NormalTok{(raster)\}}
\ControlFlowTok{if}\NormalTok{(}\SpecialCharTok{!}\FunctionTok{require}\NormalTok{(fasterize)) \{}\FunctionTok{install.packages}\NormalTok{(}\StringTok{"fasterize"}\NormalTok{); }\FunctionTok{require}\NormalTok{(fasterize)\}}

\CommentTok{\# templates {-}{-}{-}{-}}
\NormalTok{template100}\OtherTok{=}\FunctionTok{rast}\NormalTok{(}\StringTok{"./Templates/TemplateRasters/LV100m\_10km.tif"}\NormalTok{)}
\NormalTok{template10}\OtherTok{=}\FunctionTok{rast}\NormalTok{(}\StringTok{"./Templates/TemplateRasters/LV10m\_10km.tif"}\NormalTok{)}
\NormalTok{rastrs10}\OtherTok{=}\FunctionTok{raster}\NormalTok{(template10)}

\NormalTok{nulls10}\OtherTok{=}\FunctionTok{rast}\NormalTok{(}\StringTok{"./Templates/TemplateRasters/nulls\_LV10m\_10km.tif"}\NormalTok{)}
\NormalTok{nulls100}\OtherTok{=}\FunctionTok{rast}\NormalTok{(}\StringTok{"./Templates/TemplateRasters/nulls\_LV100m\_10km.tif"}\NormalTok{)}

\CommentTok{\# codes {-}{-}{-}{-}}
\NormalTok{kodi}\OtherTok{=}\FunctionTok{read\_excel}\NormalTok{(}\StringTok{"./Geodata/2024/LAD/KulturuKodi\_2024.xlsx"}\NormalTok{)}
\NormalTok{kodi}\SpecialCharTok{$}\NormalTok{kods}\OtherTok{=}\FunctionTok{as.character}\NormalTok{(kodi}\SpecialCharTok{$}\NormalTok{kods)}
\CommentTok{\# LAD {-}{-}{-}{-}}
\NormalTok{lad}\OtherTok{=}\NormalTok{sfarrow}\SpecialCharTok{::}\FunctionTok{st\_read\_parquet}\NormalTok{(}\StringTok{"./Geodata/2024/LAD/Lauki\_2024.parquet"}\NormalTok{)}
\NormalTok{lad}\SpecialCharTok{$}\NormalTok{yes}\OtherTok{=}\DecValTok{1}
\NormalTok{lad}\OtherTok{=}\NormalTok{lad }\SpecialCharTok{\%\textgreater{}\%} 
 \FunctionTok{left\_join}\NormalTok{(kodi,}\AttributeTok{by=}\FunctionTok{c}\NormalTok{(}\StringTok{"PRODUCT\_CODE"}\OtherTok{=}\StringTok{"kods"}\NormalTok{))}

\CommentTok{\# simple landscape {-}{-}{-}{-}}
\NormalTok{simple\_landscape}\OtherTok{=}\FunctionTok{rast}\NormalTok{(}\StringTok{"RasterGrids\_10m/2024/Ainava\_vienk\_mask.tif"}\NormalTok{)}


\CommentTok{\# FarmlandCrops\_RapeseedsWinter\_cell.tif    egv\_210 {-}{-}{-}{-}}
\NormalTok{dati}\OtherTok{=}\NormalTok{lad }\SpecialCharTok{\%\textgreater{}\%} 
 \FunctionTok{filter}\NormalTok{(}\FunctionTok{str\_detect}\NormalTok{(SDM\_grupa\_sakums,}\StringTok{"ziemas rapsis"}\NormalTok{))}
\FunctionTok{table}\NormalTok{(dati}\SpecialCharTok{$}\NormalTok{SDM\_grupa\_sakums,}\AttributeTok{useNA=}\StringTok{"always"}\NormalTok{)}


\NormalTok{p2i\_rez}\OtherTok{=}\NormalTok{egvtools}\SpecialCharTok{::}\FunctionTok{polygon2input}\NormalTok{(}\AttributeTok{vector\_data =}\NormalTok{ dati,}
                \AttributeTok{template\_path =} \StringTok{"./Templates/TemplateRasters/LV10m\_10km.tif"}\NormalTok{,}
                \AttributeTok{out\_path =} \StringTok{"./RasterGrids\_10m/2024/"}\NormalTok{,}
                \AttributeTok{file\_name =} \StringTok{"FarmlandCrops\_RapeseedsWinter\_input.tif"}\NormalTok{,}
                \AttributeTok{value\_field =} \StringTok{"yes"}\NormalTok{,}
                \AttributeTok{prepare=}\ConstantTok{FALSE}\NormalTok{,}
                \AttributeTok{background\_raster =} \StringTok{"./Templates/TemplateRasters/nulls\_LV10m\_10km.tif"}\NormalTok{,}
                \AttributeTok{plot\_result =} \ConstantTok{TRUE}\NormalTok{)}
\NormalTok{p2i\_rez}
\NormalTok{i2e\_rez}\OtherTok{=}\NormalTok{egvtools}\SpecialCharTok{::}\FunctionTok{input2egv}\NormalTok{(}\AttributeTok{input=}\FunctionTok{paste0}\NormalTok{(}\StringTok{"./RasterGrids\_10m/2024/"}\NormalTok{,}
                     \StringTok{"FarmlandCrops\_RapeseedsWinter\_input.tif"}\NormalTok{),}
              \AttributeTok{egv\_template=} \StringTok{"./Templates/TemplateRasters/LV100m\_10km.tif"}\NormalTok{,}
              \AttributeTok{summary\_function =} \StringTok{"average"}\NormalTok{,}
              \AttributeTok{missing\_job =} \StringTok{"FillOutput"}\NormalTok{,}
              \AttributeTok{outlocation =} \StringTok{"./RasterGrids\_100m/2024/RAW/"}\NormalTok{,}
              \AttributeTok{outfilename =} \StringTok{"FarmlandCrops\_RapeseedsWinter\_cell.tif"}\NormalTok{,}
              \AttributeTok{layername =} \StringTok{"egv\_210"}\NormalTok{,}
              \AttributeTok{idw\_weight =} \DecValTok{2}\NormalTok{,}
              \AttributeTok{plot\_gaps =} \ConstantTok{FALSE}\NormalTok{,}\AttributeTok{plot\_final =} \ConstantTok{TRUE}\NormalTok{)}
\NormalTok{i2e\_rez}
\FunctionTok{rm}\NormalTok{(p2i\_rez)}
\FunctionTok{rm}\NormalTok{(i2e\_rez)}
\FunctionTok{rm}\NormalTok{(dati)}
\FunctionTok{unlink}\NormalTok{(}\StringTok{"./RasterGrids\_10m/2024/FarmlandCrops\_RapeseedsWinter\_input.tif"}\NormalTok{)}


\CommentTok{\# standardisation {-}{-}{-}{-}}
\ControlFlowTok{if}\NormalTok{(}\SpecialCharTok{!}\FunctionTok{require}\NormalTok{(terra)) \{}\FunctionTok{install.packages}\NormalTok{(}\StringTok{"terra"}\NormalTok{); }\FunctionTok{require}\NormalTok{(terra)\}}
\ControlFlowTok{if}\NormalTok{(}\SpecialCharTok{!}\FunctionTok{require}\NormalTok{(tidyverse)) \{}\FunctionTok{install.packages}\NormalTok{(}\StringTok{"tidyverse"}\NormalTok{); }\FunctionTok{require}\NormalTok{(tidyverse)\}}

\NormalTok{nosaukums}\OtherTok{=}\StringTok{"FarmlandCrops\_RapeseedsWinter\_cell.tif"}
\NormalTok{ielasisanas\_cels}\OtherTok{=}\FunctionTok{paste0}\NormalTok{(}\StringTok{"./RasterGrids\_100m/2024/RAW/"}\NormalTok{,nosaukums)}
\NormalTok{saglabasanas\_cels}\OtherTok{=}\FunctionTok{paste0}\NormalTok{(}\StringTok{"./RasterGrids\_100m/2024/Scaled/"}\NormalTok{,nosaukums)}
\NormalTok{slanis}\OtherTok{=}\FunctionTok{rast}\NormalTok{(ielasisanas\_cels)}
\NormalTok{videjais}\OtherTok{=}\FunctionTok{global}\NormalTok{(slanis,}\AttributeTok{fun=}\StringTok{"mean"}\NormalTok{,}\AttributeTok{na.rm=}\ConstantTok{TRUE}\NormalTok{)}
\NormalTok{centrets}\OtherTok{=}\NormalTok{slanis}\SpecialCharTok{{-}}\NormalTok{videjais[,}\DecValTok{1}\NormalTok{]}
\NormalTok{standartnovirze}\OtherTok{=}\NormalTok{terra}\SpecialCharTok{::}\FunctionTok{global}\NormalTok{(centrets,}\AttributeTok{fun=}\StringTok{"rms"}\NormalTok{,}\AttributeTok{na.rm=}\ConstantTok{TRUE}\NormalTok{)}
\NormalTok{merogots}\OtherTok{=}\NormalTok{centrets}\SpecialCharTok{/}\NormalTok{standartnovirze[,}\DecValTok{1}\NormalTok{]}
\FunctionTok{writeRaster}\NormalTok{(merogots,}
      \AttributeTok{filename=}\NormalTok{saglabasanas\_cels,}
      \AttributeTok{overwrite=}\ConstantTok{TRUE}\NormalTok{)}
\end{Highlighting}
\end{Shaded}

\section{FarmlandCrops\_RapeseedsWinter\_r500}\label{ch06.211}

\textbf{filename:} \texttt{FarmlandCrops\_RapeseedsWinter\_r500.tif}

\textbf{layername:} \texttt{egv\_211}

\textbf{English name:} Fractional cover of Winter Rapeseed, Turnip within the 0.5 km
landscape

\textbf{Latvian name:} Ziemas rapša, ripša platības īpatsvars 0,5 km ainavā

\textbf{Procedure:} The cover fraction within a radius of 500 m around the analysis grid cell is
calculated as the area-weighted sum of the \hyperref[ch06.210]{analysis cells} inside the
buffer, using the workflow \texttt{egvtools::radius\_function()}. During the calculation of the landscape metric,
inverse distance weighted (power = 2) gap filling on the output is applied
to ensure no missing values at the edges. Then the layer is rewritten to set
its name. Finally, the layer is standardised by subtracting the arithmetic
mean and dividing by the root mean squared error.

\begin{Shaded}
\begin{Highlighting}[]
\CommentTok{\# libs {-}{-}{-}{-}}
\ControlFlowTok{if}\NormalTok{(}\SpecialCharTok{!}\FunctionTok{require}\NormalTok{(terra)) \{}\FunctionTok{install.packages}\NormalTok{(}\StringTok{"terra"}\NormalTok{); }\FunctionTok{require}\NormalTok{(terra)\}}
\ControlFlowTok{if}\NormalTok{(}\SpecialCharTok{!}\FunctionTok{require}\NormalTok{(egvtools)) \{remotes}\SpecialCharTok{::}\FunctionTok{install\_github}\NormalTok{(}\StringTok{"aavotins/egvtools"}\NormalTok{); }\FunctionTok{require}\NormalTok{(egvtools)\}}


\CommentTok{\# Templates {-}{-}{-}{-}{-}}
\NormalTok{template100}\OtherTok{=}\FunctionTok{rast}\NormalTok{(}\StringTok{"./Templates/TemplateRasters/LV100m\_10km.tif"}\NormalTok{)}

\CommentTok{\# radii {-}{-}{-}{-}}
\FunctionTok{radius\_function}\NormalTok{(}
 \AttributeTok{kvadrati\_path =} \StringTok{"./Templates/TemplateGrids/tiles/"}\NormalTok{,}
 \AttributeTok{radii\_path   =} \StringTok{"./Templates/TemplateGridPoints/tiles/"}\NormalTok{,}
 \AttributeTok{tikls100\_path =} \StringTok{"./Templates/TemplateGrids/tikls100\_sauzeme.parquet"}\NormalTok{,}
 \AttributeTok{template\_path =} \StringTok{"./Templates/TemplateRasters/LV100m\_10km.tif"}\NormalTok{,}
 \AttributeTok{input\_layers  =} \FunctionTok{c}\NormalTok{(}\StringTok{"./RasterGrids\_100m/2024/RAW/FarmlandCrops\_RapeseedsWinter\_cell.tif"}\NormalTok{),}
 \AttributeTok{layer\_prefixes =} \FunctionTok{c}\NormalTok{(}\StringTok{"FarmlandCrops\_RapeseedsWinter"}\NormalTok{),}
 \AttributeTok{output\_dir   =} \StringTok{"./RasterGrids\_100m/2024/RAW/"}\NormalTok{,}
 \AttributeTok{n\_workers   =} \DecValTok{6}\NormalTok{,}
 \AttributeTok{radii     =} \FunctionTok{c}\NormalTok{(}\StringTok{"r500"}\NormalTok{),}
 \AttributeTok{radius\_mode  =} \StringTok{"sparse"}\NormalTok{,}
 \AttributeTok{extract\_fun  =} \StringTok{"mean"}\NormalTok{,}
 \AttributeTok{fill\_missing  =} \ConstantTok{TRUE}\NormalTok{,}
 \AttributeTok{IDW\_weight   =} \DecValTok{2}\NormalTok{,}
 \AttributeTok{future\_max\_size =} \DecValTok{40} \SpecialCharTok{*} \DecValTok{1024}\SpecialCharTok{\^{}}\DecValTok{3}\NormalTok{)}


\CommentTok{\# FarmlandCrops\_RapeseedsWinter\_r500.tif    egv\_211}
\NormalTok{slanis}\OtherTok{=}\FunctionTok{rast}\NormalTok{(}\StringTok{"./RasterGrids\_100m/2024/RAW/FarmlandCrops\_RapeseedsWinter\_r500.tif"}\NormalTok{)}
\FunctionTok{names}\NormalTok{(slanis)}\OtherTok{=}\StringTok{"egv\_211"}
\NormalTok{slanis2}\OtherTok{=}\FunctionTok{project}\NormalTok{(slanis,template100)}
\FunctionTok{writeRaster}\NormalTok{(slanis2,}
      \StringTok{"./RasterGrids\_100m/2024/RAW/FarmlandCrops\_RapeseedsWinter\_r500.tif"}\NormalTok{,}
      \AttributeTok{overwrite=}\ConstantTok{TRUE}\NormalTok{)}

\CommentTok{\# standardisation {-}{-}{-}{-}}
\ControlFlowTok{if}\NormalTok{(}\SpecialCharTok{!}\FunctionTok{require}\NormalTok{(terra)) \{}\FunctionTok{install.packages}\NormalTok{(}\StringTok{"terra"}\NormalTok{); }\FunctionTok{require}\NormalTok{(terra)\}}
\ControlFlowTok{if}\NormalTok{(}\SpecialCharTok{!}\FunctionTok{require}\NormalTok{(tidyverse)) \{}\FunctionTok{install.packages}\NormalTok{(}\StringTok{"tidyverse"}\NormalTok{); }\FunctionTok{require}\NormalTok{(tidyverse)\}}

\NormalTok{nosaukums}\OtherTok{=}\StringTok{"FarmlandCrops\_RapeseedsWinter\_r500.tif"}
\NormalTok{ielasisanas\_cels}\OtherTok{=}\FunctionTok{paste0}\NormalTok{(}\StringTok{"./RasterGrids\_100m/2024/RAW/"}\NormalTok{,nosaukums)}
\NormalTok{saglabasanas\_cels}\OtherTok{=}\FunctionTok{paste0}\NormalTok{(}\StringTok{"./RasterGrids\_100m/2024/Scaled/"}\NormalTok{,nosaukums)}
\NormalTok{slanis}\OtherTok{=}\FunctionTok{rast}\NormalTok{(ielasisanas\_cels)}
\NormalTok{videjais}\OtherTok{=}\FunctionTok{global}\NormalTok{(slanis,}\AttributeTok{fun=}\StringTok{"mean"}\NormalTok{,}\AttributeTok{na.rm=}\ConstantTok{TRUE}\NormalTok{)}
\NormalTok{centrets}\OtherTok{=}\NormalTok{slanis}\SpecialCharTok{{-}}\NormalTok{videjais[,}\DecValTok{1}\NormalTok{]}
\NormalTok{standartnovirze}\OtherTok{=}\NormalTok{terra}\SpecialCharTok{::}\FunctionTok{global}\NormalTok{(centrets,}\AttributeTok{fun=}\StringTok{"rms"}\NormalTok{,}\AttributeTok{na.rm=}\ConstantTok{TRUE}\NormalTok{)}
\NormalTok{merogots}\OtherTok{=}\NormalTok{centrets}\SpecialCharTok{/}\NormalTok{standartnovirze[,}\DecValTok{1}\NormalTok{]}
\FunctionTok{writeRaster}\NormalTok{(merogots,}
      \AttributeTok{filename=}\NormalTok{saglabasanas\_cels,}
      \AttributeTok{overwrite=}\ConstantTok{TRUE}\NormalTok{)}
\end{Highlighting}
\end{Shaded}

\section{FarmlandCrops\_RapeseedsWinter\_r1250}\label{ch06.212}

\textbf{filename:} \texttt{FarmlandCrops\_RapeseedsWinter\_r1250.tif}

\textbf{layername:} \texttt{egv\_212}

\textbf{English name:} Fractional cover of Winter Rapeseed, Turnip within the 1.25 km
landscape

\textbf{Latvian name:} Ziemas rapša, ripša platības īpatsvars 1,25 km ainavā

\textbf{Procedure:} The cover fraction within a radius of 1250 m around the analysis grid cell
is calculated as the area-weighted sum of the \hyperref[ch06.210]{analysis cells} inside
the buffer, using the workflow \texttt{egvtools::radius\_function()}. During the calculation of the landscape
metric, inverse distance weighted (power = 2) gap filling on the output is
applied to ensure no missing values at the edges. Then the layer is
rewritten to set its name. Finally, the layer is standardised by
subtracting the arithmetic mean and dividing by the root mean squared error.

\begin{Shaded}
\begin{Highlighting}[]
\CommentTok{\# libs {-}{-}{-}{-}}
\ControlFlowTok{if}\NormalTok{(}\SpecialCharTok{!}\FunctionTok{require}\NormalTok{(terra)) \{}\FunctionTok{install.packages}\NormalTok{(}\StringTok{"terra"}\NormalTok{); }\FunctionTok{require}\NormalTok{(terra)\}}
\ControlFlowTok{if}\NormalTok{(}\SpecialCharTok{!}\FunctionTok{require}\NormalTok{(egvtools)) \{remotes}\SpecialCharTok{::}\FunctionTok{install\_github}\NormalTok{(}\StringTok{"aavotins/egvtools"}\NormalTok{); }\FunctionTok{require}\NormalTok{(egvtools)\}}


\CommentTok{\# Templates {-}{-}{-}{-}{-}}
\NormalTok{template100}\OtherTok{=}\FunctionTok{rast}\NormalTok{(}\StringTok{"./Templates/TemplateRasters/LV100m\_10km.tif"}\NormalTok{)}

\CommentTok{\# radii {-}{-}{-}{-}}
\FunctionTok{radius\_function}\NormalTok{(}
 \AttributeTok{kvadrati\_path =} \StringTok{"./Templates/TemplateGrids/tiles/"}\NormalTok{,}
 \AttributeTok{radii\_path   =} \StringTok{"./Templates/TemplateGridPoints/tiles/"}\NormalTok{,}
 \AttributeTok{tikls100\_path =} \StringTok{"./Templates/TemplateGrids/tikls100\_sauzeme.parquet"}\NormalTok{,}
 \AttributeTok{template\_path =} \StringTok{"./Templates/TemplateRasters/LV100m\_10km.tif"}\NormalTok{,}
 \AttributeTok{input\_layers  =} \FunctionTok{c}\NormalTok{(}\StringTok{"./RasterGrids\_100m/2024/RAW/FarmlandCrops\_RapeseedsWinter\_cell.tif"}\NormalTok{),}
 \AttributeTok{layer\_prefixes =} \FunctionTok{c}\NormalTok{(}\StringTok{"FarmlandCrops\_RapeseedsWinter"}\NormalTok{),}
 \AttributeTok{output\_dir   =} \StringTok{"./RasterGrids\_100m/2024/RAW/"}\NormalTok{,}
 \AttributeTok{n\_workers   =} \DecValTok{6}\NormalTok{,}
 \AttributeTok{radii     =} \FunctionTok{c}\NormalTok{(}\StringTok{"r1250"}\NormalTok{),}
 \AttributeTok{radius\_mode  =} \StringTok{"sparse"}\NormalTok{,}
 \AttributeTok{extract\_fun  =} \StringTok{"mean"}\NormalTok{,}
 \AttributeTok{fill\_missing  =} \ConstantTok{TRUE}\NormalTok{,}
 \AttributeTok{IDW\_weight   =} \DecValTok{2}\NormalTok{,}
 \AttributeTok{future\_max\_size =} \DecValTok{40} \SpecialCharTok{*} \DecValTok{1024}\SpecialCharTok{\^{}}\DecValTok{3}\NormalTok{)}


\CommentTok{\# FarmlandCrops\_RapeseedsWinter\_r1250.tif   egv\_212}
\NormalTok{slanis}\OtherTok{=}\FunctionTok{rast}\NormalTok{(}\StringTok{"./RasterGrids\_100m/2024/RAW/FarmlandCrops\_RapeseedsWinter\_r1250.tif"}\NormalTok{)}
\FunctionTok{names}\NormalTok{(slanis)}\OtherTok{=}\StringTok{"egv\_212"}
\NormalTok{slanis2}\OtherTok{=}\FunctionTok{project}\NormalTok{(slanis,template100)}
\FunctionTok{writeRaster}\NormalTok{(slanis2,}
      \StringTok{"./RasterGrids\_100m/2024/RAW/FarmlandCrops\_RapeseedsWinter\_r1250.tif"}\NormalTok{,}
      \AttributeTok{overwrite=}\ConstantTok{TRUE}\NormalTok{)}

\CommentTok{\# standardisation {-}{-}{-}{-}}
\ControlFlowTok{if}\NormalTok{(}\SpecialCharTok{!}\FunctionTok{require}\NormalTok{(terra)) \{}\FunctionTok{install.packages}\NormalTok{(}\StringTok{"terra"}\NormalTok{); }\FunctionTok{require}\NormalTok{(terra)\}}
\ControlFlowTok{if}\NormalTok{(}\SpecialCharTok{!}\FunctionTok{require}\NormalTok{(tidyverse)) \{}\FunctionTok{install.packages}\NormalTok{(}\StringTok{"tidyverse"}\NormalTok{); }\FunctionTok{require}\NormalTok{(tidyverse)\}}

\NormalTok{nosaukums}\OtherTok{=}\StringTok{"FarmlandCrops\_RapeseedsWinter\_r1250.tif"}
\NormalTok{ielasisanas\_cels}\OtherTok{=}\FunctionTok{paste0}\NormalTok{(}\StringTok{"./RasterGrids\_100m/2024/RAW/"}\NormalTok{,nosaukums)}
\NormalTok{saglabasanas\_cels}\OtherTok{=}\FunctionTok{paste0}\NormalTok{(}\StringTok{"./RasterGrids\_100m/2024/Scaled/"}\NormalTok{,nosaukums)}
\NormalTok{slanis}\OtherTok{=}\FunctionTok{rast}\NormalTok{(ielasisanas\_cels)}
\NormalTok{videjais}\OtherTok{=}\FunctionTok{global}\NormalTok{(slanis,}\AttributeTok{fun=}\StringTok{"mean"}\NormalTok{,}\AttributeTok{na.rm=}\ConstantTok{TRUE}\NormalTok{)}
\NormalTok{centrets}\OtherTok{=}\NormalTok{slanis}\SpecialCharTok{{-}}\NormalTok{videjais[,}\DecValTok{1}\NormalTok{]}
\NormalTok{standartnovirze}\OtherTok{=}\NormalTok{terra}\SpecialCharTok{::}\FunctionTok{global}\NormalTok{(centrets,}\AttributeTok{fun=}\StringTok{"rms"}\NormalTok{,}\AttributeTok{na.rm=}\ConstantTok{TRUE}\NormalTok{)}
\NormalTok{merogots}\OtherTok{=}\NormalTok{centrets}\SpecialCharTok{/}\NormalTok{standartnovirze[,}\DecValTok{1}\NormalTok{]}
\FunctionTok{writeRaster}\NormalTok{(merogots,}
      \AttributeTok{filename=}\NormalTok{saglabasanas\_cels,}
      \AttributeTok{overwrite=}\ConstantTok{TRUE}\NormalTok{)}
\end{Highlighting}
\end{Shaded}

\section{FarmlandCrops\_RapeseedsWinter\_r3000}\label{ch06.213}

\textbf{filename:} \texttt{FarmlandCrops\_RapeseedsWinter\_r3000.tif}

\textbf{layername:} \texttt{egv\_213}

\textbf{English name:} Fractional cover of Winter Rapeseed, Turnip within the 3 km
landscape

\textbf{Latvian name:} Ziemas rapša, ripša platības īpatsvars 3 km ainavā

\textbf{Procedure:} The cover fraction within a radius of 3000 m around the analysis grid cell
is calculated as the area-weighted sum of the \hyperref[ch06.210]{analysis cells} inside
the buffer, using the workflow \texttt{egvtools::radius\_function()}. During the calculation of the landscape
metric, inverse distance weighted (power = 2) gap filling on the output is
applied to ensure no missing values at the edges. Then the layer is
rewritten to set its name. Finally, the layer is standardised by
subtracting the arithmetic mean and dividing by the root mean squared error.

\begin{Shaded}
\begin{Highlighting}[]
\CommentTok{\# libs {-}{-}{-}{-}}
\ControlFlowTok{if}\NormalTok{(}\SpecialCharTok{!}\FunctionTok{require}\NormalTok{(terra)) \{}\FunctionTok{install.packages}\NormalTok{(}\StringTok{"terra"}\NormalTok{); }\FunctionTok{require}\NormalTok{(terra)\}}
\ControlFlowTok{if}\NormalTok{(}\SpecialCharTok{!}\FunctionTok{require}\NormalTok{(egvtools)) \{remotes}\SpecialCharTok{::}\FunctionTok{install\_github}\NormalTok{(}\StringTok{"aavotins/egvtools"}\NormalTok{); }\FunctionTok{require}\NormalTok{(egvtools)\}}


\CommentTok{\# Templates {-}{-}{-}{-}{-}}
\NormalTok{template100}\OtherTok{=}\FunctionTok{rast}\NormalTok{(}\StringTok{"./Templates/TemplateRasters/LV100m\_10km.tif"}\NormalTok{)}

\CommentTok{\# radii {-}{-}{-}{-}}
\FunctionTok{radius\_function}\NormalTok{(}
 \AttributeTok{kvadrati\_path =} \StringTok{"./Templates/TemplateGrids/tiles/"}\NormalTok{,}
 \AttributeTok{radii\_path   =} \StringTok{"./Templates/TemplateGridPoints/tiles/"}\NormalTok{,}
 \AttributeTok{tikls100\_path =} \StringTok{"./Templates/TemplateGrids/tikls100\_sauzeme.parquet"}\NormalTok{,}
 \AttributeTok{template\_path =} \StringTok{"./Templates/TemplateRasters/LV100m\_10km.tif"}\NormalTok{,}
 \AttributeTok{input\_layers  =} \FunctionTok{c}\NormalTok{(}\StringTok{"./RasterGrids\_100m/2024/RAW/FarmlandCrops\_RapeseedsWinter\_cell.tif"}\NormalTok{),}
 \AttributeTok{layer\_prefixes =} \FunctionTok{c}\NormalTok{(}\StringTok{"FarmlandCrops\_RapeseedsWinter"}\NormalTok{),}
 \AttributeTok{output\_dir   =} \StringTok{"./RasterGrids\_100m/2024/RAW/"}\NormalTok{,}
 \AttributeTok{n\_workers   =} \DecValTok{6}\NormalTok{,}
 \AttributeTok{radii     =} \FunctionTok{c}\NormalTok{(}\StringTok{"r3000"}\NormalTok{),}
 \AttributeTok{radius\_mode  =} \StringTok{"sparse"}\NormalTok{,}
 \AttributeTok{extract\_fun  =} \StringTok{"mean"}\NormalTok{,}
 \AttributeTok{fill\_missing  =} \ConstantTok{TRUE}\NormalTok{,}
 \AttributeTok{IDW\_weight   =} \DecValTok{2}\NormalTok{,}
 \AttributeTok{future\_max\_size =} \DecValTok{40} \SpecialCharTok{*} \DecValTok{1024}\SpecialCharTok{\^{}}\DecValTok{3}\NormalTok{)}


\CommentTok{\# FarmlandCrops\_RapeseedsWinter\_r3000.tif   egv\_213}
\NormalTok{slanis}\OtherTok{=}\FunctionTok{rast}\NormalTok{(}\StringTok{"./RasterGrids\_100m/2024/RAW/FarmlandCrops\_RapeseedsWinter\_r3000.tif"}\NormalTok{)}
\FunctionTok{names}\NormalTok{(slanis)}\OtherTok{=}\StringTok{"egv\_213"}
\NormalTok{slanis2}\OtherTok{=}\FunctionTok{project}\NormalTok{(slanis,template100)}
\FunctionTok{writeRaster}\NormalTok{(slanis2,}
      \StringTok{"./RasterGrids\_100m/2024/RAW/FarmlandCrops\_RapeseedsWinter\_r3000.tif"}\NormalTok{,}
      \AttributeTok{overwrite=}\ConstantTok{TRUE}\NormalTok{)}

\CommentTok{\# standardisation {-}{-}{-}{-}}
\ControlFlowTok{if}\NormalTok{(}\SpecialCharTok{!}\FunctionTok{require}\NormalTok{(terra)) \{}\FunctionTok{install.packages}\NormalTok{(}\StringTok{"terra"}\NormalTok{); }\FunctionTok{require}\NormalTok{(terra)\}}
\ControlFlowTok{if}\NormalTok{(}\SpecialCharTok{!}\FunctionTok{require}\NormalTok{(tidyverse)) \{}\FunctionTok{install.packages}\NormalTok{(}\StringTok{"tidyverse"}\NormalTok{); }\FunctionTok{require}\NormalTok{(tidyverse)\}}

\NormalTok{nosaukums}\OtherTok{=}\StringTok{"FarmlandCrops\_RapeseedsWinter\_r3000.tif"}
\NormalTok{ielasisanas\_cels}\OtherTok{=}\FunctionTok{paste0}\NormalTok{(}\StringTok{"./RasterGrids\_100m/2024/RAW/"}\NormalTok{,nosaukums)}
\NormalTok{saglabasanas\_cels}\OtherTok{=}\FunctionTok{paste0}\NormalTok{(}\StringTok{"./RasterGrids\_100m/2024/Scaled/"}\NormalTok{,nosaukums)}
\NormalTok{slanis}\OtherTok{=}\FunctionTok{rast}\NormalTok{(ielasisanas\_cels)}
\NormalTok{videjais}\OtherTok{=}\FunctionTok{global}\NormalTok{(slanis,}\AttributeTok{fun=}\StringTok{"mean"}\NormalTok{,}\AttributeTok{na.rm=}\ConstantTok{TRUE}\NormalTok{)}
\NormalTok{centrets}\OtherTok{=}\NormalTok{slanis}\SpecialCharTok{{-}}\NormalTok{videjais[,}\DecValTok{1}\NormalTok{]}
\NormalTok{standartnovirze}\OtherTok{=}\NormalTok{terra}\SpecialCharTok{::}\FunctionTok{global}\NormalTok{(centrets,}\AttributeTok{fun=}\StringTok{"rms"}\NormalTok{,}\AttributeTok{na.rm=}\ConstantTok{TRUE}\NormalTok{)}
\NormalTok{merogots}\OtherTok{=}\NormalTok{centrets}\SpecialCharTok{/}\NormalTok{standartnovirze[,}\DecValTok{1}\NormalTok{]}
\FunctionTok{writeRaster}\NormalTok{(merogots,}
      \AttributeTok{filename=}\NormalTok{saglabasanas\_cels,}
      \AttributeTok{overwrite=}\ConstantTok{TRUE}\NormalTok{)}
\end{Highlighting}
\end{Shaded}

\section{FarmlandCrops\_RapeseedsWinter\_r10000}\label{ch06.214}

\textbf{filename:} \texttt{FarmlandCrops\_RapeseedsWinter\_r10000.tif}

\textbf{layername:} \texttt{egv\_214}

\textbf{English name:} Fractional cover of Winter Rapeseed, Turnip within the 10 km
landscape

\textbf{Latvian name:} Ziemas rapša, ripša platības īpatsvars 10 km ainavā

\textbf{Procedure:} The cover fraction within a radius of 10000 m around the analysis grid cell
is calculated as the area-weighted sum of the \hyperref[ch06.210]{analysis cells} inside
the buffer, using the workflow \texttt{egvtools::radius\_function()}. During the calculation of the landscape
metric, inverse distance weighted (power = 2) gap filling on the output is
applied to ensure no missing values at the edges. Then the layer is
rewritten to set its name. Finally, the layer is standardised by
subtracting the arithmetic mean and dividing by the root mean squared error.

\begin{Shaded}
\begin{Highlighting}[]
\CommentTok{\# libs {-}{-}{-}{-}}
\ControlFlowTok{if}\NormalTok{(}\SpecialCharTok{!}\FunctionTok{require}\NormalTok{(terra)) \{}\FunctionTok{install.packages}\NormalTok{(}\StringTok{"terra"}\NormalTok{); }\FunctionTok{require}\NormalTok{(terra)\}}
\ControlFlowTok{if}\NormalTok{(}\SpecialCharTok{!}\FunctionTok{require}\NormalTok{(egvtools)) \{remotes}\SpecialCharTok{::}\FunctionTok{install\_github}\NormalTok{(}\StringTok{"aavotins/egvtools"}\NormalTok{); }\FunctionTok{require}\NormalTok{(egvtools)\}}


\CommentTok{\# Templates {-}{-}{-}{-}{-}}
\NormalTok{template100}\OtherTok{=}\FunctionTok{rast}\NormalTok{(}\StringTok{"./Templates/TemplateRasters/LV100m\_10km.tif"}\NormalTok{)}

\CommentTok{\# radii {-}{-}{-}{-}}
\FunctionTok{radius\_function}\NormalTok{(}
 \AttributeTok{kvadrati\_path =} \StringTok{"./Templates/TemplateGrids/tiles/"}\NormalTok{,}
 \AttributeTok{radii\_path   =} \StringTok{"./Templates/TemplateGridPoints/tiles/"}\NormalTok{,}
 \AttributeTok{tikls100\_path =} \StringTok{"./Templates/TemplateGrids/tikls100\_sauzeme.parquet"}\NormalTok{,}
 \AttributeTok{template\_path =} \StringTok{"./Templates/TemplateRasters/LV100m\_10km.tif"}\NormalTok{,}
 \AttributeTok{input\_layers  =} \FunctionTok{c}\NormalTok{(}\StringTok{"./RasterGrids\_100m/2024/RAW/FarmlandCrops\_RapeseedsWinter\_cell.tif"}\NormalTok{),}
 \AttributeTok{layer\_prefixes =} \FunctionTok{c}\NormalTok{(}\StringTok{"FarmlandCrops\_RapeseedsWinter"}\NormalTok{),}
 \AttributeTok{output\_dir   =} \StringTok{"./RasterGrids\_100m/2024/RAW/"}\NormalTok{,}
 \AttributeTok{n\_workers   =} \DecValTok{6}\NormalTok{,}
 \AttributeTok{radii     =} \FunctionTok{c}\NormalTok{(}\StringTok{"r10000"}\NormalTok{),}
 \AttributeTok{radius\_mode  =} \StringTok{"sparse"}\NormalTok{,}
 \AttributeTok{extract\_fun  =} \StringTok{"mean"}\NormalTok{,}
 \AttributeTok{fill\_missing  =} \ConstantTok{TRUE}\NormalTok{,}
 \AttributeTok{IDW\_weight   =} \DecValTok{2}\NormalTok{,}
 \AttributeTok{future\_max\_size =} \DecValTok{40} \SpecialCharTok{*} \DecValTok{1024}\SpecialCharTok{\^{}}\DecValTok{3}\NormalTok{)}


\CommentTok{\# FarmlandCrops\_RapeseedsWinter\_r10000.tif  egv\_214}
\NormalTok{slanis}\OtherTok{=}\FunctionTok{rast}\NormalTok{(}\StringTok{"./RasterGrids\_100m/2024/RAW/FarmlandCrops\_RapeseedsWinter\_r10000.tif"}\NormalTok{)}
\FunctionTok{names}\NormalTok{(slanis)}\OtherTok{=}\StringTok{"egv\_214"}
\NormalTok{slanis2}\OtherTok{=}\FunctionTok{project}\NormalTok{(slanis,template100)}
\FunctionTok{writeRaster}\NormalTok{(slanis2,}
      \StringTok{"./RasterGrids\_100m/2024/RAW/FarmlandCrops\_RapeseedsWinter\_r10000.tif"}\NormalTok{,}
      \AttributeTok{overwrite=}\ConstantTok{TRUE}\NormalTok{)}

\CommentTok{\# standardisation {-}{-}{-}{-}}
\ControlFlowTok{if}\NormalTok{(}\SpecialCharTok{!}\FunctionTok{require}\NormalTok{(terra)) \{}\FunctionTok{install.packages}\NormalTok{(}\StringTok{"terra"}\NormalTok{); }\FunctionTok{require}\NormalTok{(terra)\}}
\ControlFlowTok{if}\NormalTok{(}\SpecialCharTok{!}\FunctionTok{require}\NormalTok{(tidyverse)) \{}\FunctionTok{install.packages}\NormalTok{(}\StringTok{"tidyverse"}\NormalTok{); }\FunctionTok{require}\NormalTok{(tidyverse)\}}

\NormalTok{nosaukums}\OtherTok{=}\StringTok{"FarmlandCrops\_RapeseedsWinter\_r10000.tif"}
\NormalTok{ielasisanas\_cels}\OtherTok{=}\FunctionTok{paste0}\NormalTok{(}\StringTok{"./RasterGrids\_100m/2024/RAW/"}\NormalTok{,nosaukums)}
\NormalTok{saglabasanas\_cels}\OtherTok{=}\FunctionTok{paste0}\NormalTok{(}\StringTok{"./RasterGrids\_100m/2024/Scaled/"}\NormalTok{,nosaukums)}
\NormalTok{slanis}\OtherTok{=}\FunctionTok{rast}\NormalTok{(ielasisanas\_cels)}
\NormalTok{videjais}\OtherTok{=}\FunctionTok{global}\NormalTok{(slanis,}\AttributeTok{fun=}\StringTok{"mean"}\NormalTok{,}\AttributeTok{na.rm=}\ConstantTok{TRUE}\NormalTok{)}
\NormalTok{centrets}\OtherTok{=}\NormalTok{slanis}\SpecialCharTok{{-}}\NormalTok{videjais[,}\DecValTok{1}\NormalTok{]}
\NormalTok{standartnovirze}\OtherTok{=}\NormalTok{terra}\SpecialCharTok{::}\FunctionTok{global}\NormalTok{(centrets,}\AttributeTok{fun=}\StringTok{"rms"}\NormalTok{,}\AttributeTok{na.rm=}\ConstantTok{TRUE}\NormalTok{)}
\NormalTok{merogots}\OtherTok{=}\NormalTok{centrets}\SpecialCharTok{/}\NormalTok{standartnovirze[,}\DecValTok{1}\NormalTok{]}
\FunctionTok{writeRaster}\NormalTok{(merogots,}
      \AttributeTok{filename=}\NormalTok{saglabasanas\_cels,}
      \AttributeTok{overwrite=}\ConstantTok{TRUE}\NormalTok{)}
\end{Highlighting}
\end{Shaded}

\section{FarmlandGrassland\_GrasslandsAbandoned\_cell}\label{ch06.215}

\textbf{filename:} \texttt{FarmlandGrassland\_GrasslandsAbandoned\_cell.tif}

\textbf{layername:} \texttt{egv\_215}

\textbf{English name:} Fractional cover of Abandoned Grassland within the analysis
cell (1 ha)

\textbf{Latvian name:} Neapsaimniekotu zālāju platības īpatsvars analīzes šūnā (1 ha)

\textbf{Procedure:} First, the grasslands from the \hyperref[Ch05.03]{Landscape classification} are
selected (value 330 reclassified as value 1, others as NA). Next, agricultural
parcels declared as grasslands are selected from the \hyperref[Ch04.02]{Rural Support Service's
information on declared fields}. Next, cells with grasslands in
\hyperref[Ch05.03]{Landscape classification} but not in the \hyperref[Ch04.02]{Rural Support Service's
information on declared fields} are selected and matched to input
layer. Once matched, the layer is aggregated to EGV
resolution using the workflow \texttt{egvtools::input2egv()}, which calculates the arithmetic mean and thus
results in a cover fraction. During aggregation, inverse distance weighted (power = 2) gap filling
on the output is applied to ensure no missing values at the edges. Finally, the layer is standardised
by subtracting the arithmetic mean and dividing by the root mean squared error.

\begin{Shaded}
\begin{Highlighting}[]
\CommentTok{\# libs {-}{-}{-}{-}}
\ControlFlowTok{if}\NormalTok{(}\SpecialCharTok{!}\FunctionTok{require}\NormalTok{(egvtools)) \{remotes}\SpecialCharTok{::}\FunctionTok{install\_github}\NormalTok{(}\StringTok{"aavotins/egvtools"}\NormalTok{); }\FunctionTok{require}\NormalTok{(egvtools)\}}
\ControlFlowTok{if}\NormalTok{(}\SpecialCharTok{!}\FunctionTok{require}\NormalTok{(terra)) \{}\FunctionTok{install.packages}\NormalTok{(}\StringTok{"terra"}\NormalTok{); }\FunctionTok{require}\NormalTok{(terra)\}}
\ControlFlowTok{if}\NormalTok{(}\SpecialCharTok{!}\FunctionTok{require}\NormalTok{(sf)) \{}\FunctionTok{install.packages}\NormalTok{(}\StringTok{"sf"}\NormalTok{); }\FunctionTok{require}\NormalTok{(sf)\}}
\ControlFlowTok{if}\NormalTok{(}\SpecialCharTok{!}\FunctionTok{require}\NormalTok{(tidyverse)) \{}\FunctionTok{install.packages}\NormalTok{(}\StringTok{"tidyverse"}\NormalTok{); }\FunctionTok{require}\NormalTok{(tidyverse)\}}
\ControlFlowTok{if}\NormalTok{(}\SpecialCharTok{!}\FunctionTok{require}\NormalTok{(sfarrow)) \{}\FunctionTok{install.packages}\NormalTok{(}\StringTok{"sfarrow"}\NormalTok{); }\FunctionTok{require}\NormalTok{(sfarrow)\}}
\ControlFlowTok{if}\NormalTok{(}\SpecialCharTok{!}\FunctionTok{require}\NormalTok{(readxl)) \{}\FunctionTok{install.packages}\NormalTok{(}\StringTok{"readxl"}\NormalTok{); }\FunctionTok{require}\NormalTok{(readxl)\}}
\ControlFlowTok{if}\NormalTok{(}\SpecialCharTok{!}\FunctionTok{require}\NormalTok{(raster)) \{}\FunctionTok{install.packages}\NormalTok{(}\StringTok{"raster"}\NormalTok{); }\FunctionTok{require}\NormalTok{(raster)\}}
\ControlFlowTok{if}\NormalTok{(}\SpecialCharTok{!}\FunctionTok{require}\NormalTok{(fasterize)) \{}\FunctionTok{install.packages}\NormalTok{(}\StringTok{"fasterize"}\NormalTok{); }\FunctionTok{require}\NormalTok{(fasterize)\}}

\CommentTok{\# templates {-}{-}{-}{-}}
\NormalTok{template100}\OtherTok{=}\FunctionTok{rast}\NormalTok{(}\StringTok{"./Templates/TemplateRasters/LV100m\_10km.tif"}\NormalTok{)}
\NormalTok{template10}\OtherTok{=}\FunctionTok{rast}\NormalTok{(}\StringTok{"./Templates/TemplateRasters/LV10m\_10km.tif"}\NormalTok{)}
\NormalTok{rastrs10}\OtherTok{=}\FunctionTok{raster}\NormalTok{(template10)}

\NormalTok{nulls10}\OtherTok{=}\FunctionTok{rast}\NormalTok{(}\StringTok{"./Templates/TemplateRasters/nulls\_LV10m\_10km.tif"}\NormalTok{)}
\NormalTok{nulls100}\OtherTok{=}\FunctionTok{rast}\NormalTok{(}\StringTok{"./Templates/TemplateRasters/nulls\_LV100m\_10km.tif"}\NormalTok{)}

\CommentTok{\# codes {-}{-}{-}{-}}
\NormalTok{kodi}\OtherTok{=}\FunctionTok{read\_excel}\NormalTok{(}\StringTok{"./Geodata/2024/LAD/KulturuKodi\_2024.xlsx"}\NormalTok{)}
\NormalTok{kodi}\SpecialCharTok{$}\NormalTok{kods}\OtherTok{=}\FunctionTok{as.character}\NormalTok{(kodi}\SpecialCharTok{$}\NormalTok{kods)}
\CommentTok{\# LAD {-}{-}{-}{-}}
\NormalTok{lad}\OtherTok{=}\NormalTok{sfarrow}\SpecialCharTok{::}\FunctionTok{st\_read\_parquet}\NormalTok{(}\StringTok{"./Geodata/2024/LAD/Lauki\_2024.parquet"}\NormalTok{)}
\NormalTok{lad}\SpecialCharTok{$}\NormalTok{yes}\OtherTok{=}\DecValTok{1}
\NormalTok{lad}\OtherTok{=}\NormalTok{lad }\SpecialCharTok{\%\textgreater{}\%} 
 \FunctionTok{left\_join}\NormalTok{(kodi,}\AttributeTok{by=}\FunctionTok{c}\NormalTok{(}\StringTok{"PRODUCT\_CODE"}\OtherTok{=}\StringTok{"kods"}\NormalTok{))}

\CommentTok{\# simple landscape {-}{-}{-}{-}}
\NormalTok{simple\_landscape}\OtherTok{=}\FunctionTok{rast}\NormalTok{(}\StringTok{"RasterGrids\_10m/2024/Ainava\_vienk\_mask.tif"}\NormalTok{)}


\CommentTok{\# FarmlandGrassland\_GrasslandsAbandoned\_cell.tif    egv\_215 {-}{-}{-}{-}}
\NormalTok{landscape\_grasslands}\OtherTok{=}\FunctionTok{ifel}\NormalTok{(simple\_landscape}\SpecialCharTok{==}\DecValTok{330}\NormalTok{,}\DecValTok{1}\NormalTok{,}\DecValTok{0}\NormalTok{)}

\NormalTok{dati}\OtherTok{=}\NormalTok{lad }\SpecialCharTok{\%\textgreater{}\%} 
 \FunctionTok{filter}\NormalTok{(}\FunctionTok{str\_detect}\NormalTok{(SDM\_grupa\_sakums,}\StringTok{"zālāji"}\NormalTok{))}
\FunctionTok{table}\NormalTok{(dati}\SpecialCharTok{$}\NormalTok{SDM\_grupa\_sakums,}\AttributeTok{useNA=}\StringTok{"always"}\NormalTok{)}

\NormalTok{lad\_zalajiem}\OtherTok{=}\FunctionTok{fasterize}\NormalTok{(dati,rastrs10,}\AttributeTok{field=}\StringTok{"yes"}\NormalTok{,}\AttributeTok{fun=}\StringTok{"first"}\NormalTok{)}
\NormalTok{lad\_zalaji}\OtherTok{=}\FunctionTok{rast}\NormalTok{(lad\_zalajiem)}

\NormalTok{abandoned}\OtherTok{=}\FunctionTok{ifel}\NormalTok{(landscape\_grasslands}\SpecialCharTok{==}\DecValTok{1}\SpecialCharTok{\&}\FunctionTok{is.na}\NormalTok{(lad\_zalaji),}\DecValTok{1}\NormalTok{,}\DecValTok{0}\NormalTok{)}
\FunctionTok{plot}\NormalTok{(abandoned)}

\NormalTok{i2e\_rez}\OtherTok{=}\NormalTok{egvtools}\SpecialCharTok{::}\FunctionTok{input2egv}\NormalTok{(}\AttributeTok{input=}\NormalTok{abandoned,}
              \AttributeTok{egv\_template=} \StringTok{"./Templates/TemplateRasters/LV100m\_10km.tif"}\NormalTok{,}
              \AttributeTok{summary\_function =} \StringTok{"average"}\NormalTok{,}
              \AttributeTok{missing\_job =} \StringTok{"FillOutput"}\NormalTok{,}
              \AttributeTok{outlocation =} \StringTok{"./RasterGrids\_100m/2024/RAW/"}\NormalTok{,}
              \AttributeTok{outfilename =} \StringTok{"FarmlandGrassland\_GrasslandsAbandoned\_cell.tif"}\NormalTok{,}
              \AttributeTok{layername =} \StringTok{"egv\_215"}\NormalTok{,}
              \AttributeTok{idw\_weight =} \DecValTok{2}\NormalTok{,}
              \AttributeTok{plot\_gaps =} \ConstantTok{FALSE}\NormalTok{,}\AttributeTok{plot\_final =} \ConstantTok{TRUE}\NormalTok{)}
\NormalTok{i2e\_rez}
\FunctionTok{rm}\NormalTok{(i2e\_rez)}
\FunctionTok{rm}\NormalTok{(dati)}
\FunctionTok{rm}\NormalTok{(lad\_zalajiem)}
\FunctionTok{rm}\NormalTok{(lad\_zalaji)}


\CommentTok{\# standardisation {-}{-}{-}{-}}
\ControlFlowTok{if}\NormalTok{(}\SpecialCharTok{!}\FunctionTok{require}\NormalTok{(terra)) \{}\FunctionTok{install.packages}\NormalTok{(}\StringTok{"terra"}\NormalTok{); }\FunctionTok{require}\NormalTok{(terra)\}}
\ControlFlowTok{if}\NormalTok{(}\SpecialCharTok{!}\FunctionTok{require}\NormalTok{(tidyverse)) \{}\FunctionTok{install.packages}\NormalTok{(}\StringTok{"tidyverse"}\NormalTok{); }\FunctionTok{require}\NormalTok{(tidyverse)\}}

\NormalTok{nosaukums}\OtherTok{=}\StringTok{"FarmlandGrassland\_GrasslandsAbandoned\_cell.tif"}
\NormalTok{ielasisanas\_cels}\OtherTok{=}\FunctionTok{paste0}\NormalTok{(}\StringTok{"./RasterGrids\_100m/2024/RAW/"}\NormalTok{,nosaukums)}
\NormalTok{saglabasanas\_cels}\OtherTok{=}\FunctionTok{paste0}\NormalTok{(}\StringTok{"./RasterGrids\_100m/2024/Scaled/"}\NormalTok{,nosaukums)}
\NormalTok{slanis}\OtherTok{=}\FunctionTok{rast}\NormalTok{(ielasisanas\_cels)}
\NormalTok{videjais}\OtherTok{=}\FunctionTok{global}\NormalTok{(slanis,}\AttributeTok{fun=}\StringTok{"mean"}\NormalTok{,}\AttributeTok{na.rm=}\ConstantTok{TRUE}\NormalTok{)}
\NormalTok{centrets}\OtherTok{=}\NormalTok{slanis}\SpecialCharTok{{-}}\NormalTok{videjais[,}\DecValTok{1}\NormalTok{]}
\NormalTok{standartnovirze}\OtherTok{=}\NormalTok{terra}\SpecialCharTok{::}\FunctionTok{global}\NormalTok{(centrets,}\AttributeTok{fun=}\StringTok{"rms"}\NormalTok{,}\AttributeTok{na.rm=}\ConstantTok{TRUE}\NormalTok{)}
\NormalTok{merogots}\OtherTok{=}\NormalTok{centrets}\SpecialCharTok{/}\NormalTok{standartnovirze[,}\DecValTok{1}\NormalTok{]}
\FunctionTok{writeRaster}\NormalTok{(merogots,}
      \AttributeTok{filename=}\NormalTok{saglabasanas\_cels,}
      \AttributeTok{overwrite=}\ConstantTok{TRUE}\NormalTok{)}
\end{Highlighting}
\end{Shaded}

\section{FarmlandGrassland\_GrasslandsAbandoned\_r500}\label{ch06.216}

\textbf{filename:} \texttt{FarmlandGrassland\_GrasslandsAbandoned\_r500.tif}

\textbf{layername:} \texttt{egv\_216}

\textbf{English name:} Fractional cover of Abandoned Grassland within the 0.5 km
landscape

\textbf{Latvian name:} Neapsaimniekotu zālāju platības īpatsvars 0,5 km ainavā

\textbf{Procedure:} The cover fraction within a radius of 500 m around the analysis grid cell is
calculated as the area-weighted sum of the \hyperref[ch06.215]{analysis cells} inside the
buffer, using the workflow \texttt{egvtools::radius\_function()}. During the calculation of the landscape metric,
inverse distance weighted (power = 2) gap filling on the output is applied
to ensure no missing values at the edges. Then the layer is rewritten to set
its name. Finally, the layer is standardised by subtracting the arithmetic
mean and dividing by the root mean squared error.

\begin{Shaded}
\begin{Highlighting}[]
\CommentTok{\# libs {-}{-}{-}{-}}
\ControlFlowTok{if}\NormalTok{(}\SpecialCharTok{!}\FunctionTok{require}\NormalTok{(terra)) \{}\FunctionTok{install.packages}\NormalTok{(}\StringTok{"terra"}\NormalTok{); }\FunctionTok{require}\NormalTok{(terra)\}}
\ControlFlowTok{if}\NormalTok{(}\SpecialCharTok{!}\FunctionTok{require}\NormalTok{(egvtools)) \{remotes}\SpecialCharTok{::}\FunctionTok{install\_github}\NormalTok{(}\StringTok{"aavotins/egvtools"}\NormalTok{); }\FunctionTok{require}\NormalTok{(egvtools)\}}


\CommentTok{\# Templates {-}{-}{-}{-}{-}}
\NormalTok{template100}\OtherTok{=}\FunctionTok{rast}\NormalTok{(}\StringTok{"./Templates/TemplateRasters/LV100m\_10km.tif"}\NormalTok{)}

\CommentTok{\# radii {-}{-}{-}{-}}
\FunctionTok{radius\_function}\NormalTok{(}
 \AttributeTok{kvadrati\_path =} \StringTok{"./Templates/TemplateGrids/tiles/"}\NormalTok{,}
 \AttributeTok{radii\_path   =} \StringTok{"./Templates/TemplateGridPoints/tiles/"}\NormalTok{,}
 \AttributeTok{tikls100\_path =} \StringTok{"./Templates/TemplateGrids/tikls100\_sauzeme.parquet"}\NormalTok{,}
 \AttributeTok{template\_path =} \StringTok{"./Templates/TemplateRasters/LV100m\_10km.tif"}\NormalTok{,}
 \AttributeTok{input\_layers  =} \FunctionTok{c}\NormalTok{(}\StringTok{"./RasterGrids\_100m/2024/RAW/FarmlandGrassland\_GrasslandsAbandoned\_cell.tif"}\NormalTok{),}
 \AttributeTok{layer\_prefixes =} \FunctionTok{c}\NormalTok{(}\StringTok{"FarmlandGrassland\_GrasslandsAbandoned"}\NormalTok{),}
 \AttributeTok{output\_dir   =} \StringTok{"./RasterGrids\_100m/2024/RAW/"}\NormalTok{,}
 \AttributeTok{n\_workers   =} \DecValTok{6}\NormalTok{,}
 \AttributeTok{radii     =} \FunctionTok{c}\NormalTok{(}\StringTok{"r500"}\NormalTok{),}
 \AttributeTok{radius\_mode  =} \StringTok{"sparse"}\NormalTok{,}
 \AttributeTok{extract\_fun  =} \StringTok{"mean"}\NormalTok{,}
 \AttributeTok{fill\_missing  =} \ConstantTok{TRUE}\NormalTok{,}
 \AttributeTok{IDW\_weight   =} \DecValTok{2}\NormalTok{,}
 \AttributeTok{future\_max\_size =} \DecValTok{40} \SpecialCharTok{*} \DecValTok{1024}\SpecialCharTok{\^{}}\DecValTok{3}\NormalTok{)}


\CommentTok{\# FarmlandGrassland\_GrasslandsAbandoned\_r500.tif    egv\_216}
\NormalTok{slanis}\OtherTok{=}\FunctionTok{rast}\NormalTok{(}\StringTok{"./RasterGrids\_100m/2024/RAW/FarmlandGrassland\_GrasslandsAbandoned\_r500.tif"}\NormalTok{)}
\FunctionTok{names}\NormalTok{(slanis)}\OtherTok{=}\StringTok{"egv\_216"}
\NormalTok{slanis2}\OtherTok{=}\FunctionTok{project}\NormalTok{(slanis,template100)}
\FunctionTok{writeRaster}\NormalTok{(slanis2,}
      \StringTok{"./RasterGrids\_100m/2024/RAW/FarmlandGrassland\_GrasslandsAbandoned\_r500.tif"}\NormalTok{,}
      \AttributeTok{overwrite=}\ConstantTok{TRUE}\NormalTok{)}

\CommentTok{\# standardisation {-}{-}{-}{-}}
\ControlFlowTok{if}\NormalTok{(}\SpecialCharTok{!}\FunctionTok{require}\NormalTok{(terra)) \{}\FunctionTok{install.packages}\NormalTok{(}\StringTok{"terra"}\NormalTok{); }\FunctionTok{require}\NormalTok{(terra)\}}
\ControlFlowTok{if}\NormalTok{(}\SpecialCharTok{!}\FunctionTok{require}\NormalTok{(tidyverse)) \{}\FunctionTok{install.packages}\NormalTok{(}\StringTok{"tidyverse"}\NormalTok{); }\FunctionTok{require}\NormalTok{(tidyverse)\}}

\NormalTok{nosaukums}\OtherTok{=}\StringTok{"FarmlandGrassland\_GrasslandsAbandoned\_r500.tif"}
\NormalTok{ielasisanas\_cels}\OtherTok{=}\FunctionTok{paste0}\NormalTok{(}\StringTok{"./RasterGrids\_100m/2024/RAW/"}\NormalTok{,nosaukums)}
\NormalTok{saglabasanas\_cels}\OtherTok{=}\FunctionTok{paste0}\NormalTok{(}\StringTok{"./RasterGrids\_100m/2024/Scaled/"}\NormalTok{,nosaukums)}
\NormalTok{slanis}\OtherTok{=}\FunctionTok{rast}\NormalTok{(ielasisanas\_cels)}
\NormalTok{videjais}\OtherTok{=}\FunctionTok{global}\NormalTok{(slanis,}\AttributeTok{fun=}\StringTok{"mean"}\NormalTok{,}\AttributeTok{na.rm=}\ConstantTok{TRUE}\NormalTok{)}
\NormalTok{centrets}\OtherTok{=}\NormalTok{slanis}\SpecialCharTok{{-}}\NormalTok{videjais[,}\DecValTok{1}\NormalTok{]}
\NormalTok{standartnovirze}\OtherTok{=}\NormalTok{terra}\SpecialCharTok{::}\FunctionTok{global}\NormalTok{(centrets,}\AttributeTok{fun=}\StringTok{"rms"}\NormalTok{,}\AttributeTok{na.rm=}\ConstantTok{TRUE}\NormalTok{)}
\NormalTok{merogots}\OtherTok{=}\NormalTok{centrets}\SpecialCharTok{/}\NormalTok{standartnovirze[,}\DecValTok{1}\NormalTok{]}
\FunctionTok{writeRaster}\NormalTok{(merogots,}
      \AttributeTok{filename=}\NormalTok{saglabasanas\_cels,}
      \AttributeTok{overwrite=}\ConstantTok{TRUE}\NormalTok{)}
\end{Highlighting}
\end{Shaded}

\section{FarmlandGrassland\_GrasslandsAbandoned\_r1250}\label{ch06.217}

\textbf{filename:} \texttt{FarmlandGrassland\_GrasslandsAbandoned\_r1250.tif}

\textbf{layername:} \texttt{egv\_217}

\textbf{English name:} Fractional cover of Abandoned Grassland within the 1.25 km
landscape

\textbf{Latvian name:} Neapsaimniekotu zālāju platības īpatsvars 1,25 km ainavā

\textbf{Procedure:} The cover fraction within a radius of 1250 m around the analysis grid cell
is calculated as the area-weighted sum of the \hyperref[ch06.215]{analysis cells} inside
the buffer, using the workflow \texttt{egvtools::radius\_function()}. During the calculation of the landscape
metric, inverse distance weighted (power = 2) gap filling on the output is
applied to ensure no missing values at the edges. Then the layer is
rewritten to set its name. Finally, the layer is standardised by
subtracting the arithmetic mean and dividing by the root mean squared error.

\begin{Shaded}
\begin{Highlighting}[]
\CommentTok{\# libs {-}{-}{-}{-}}
\ControlFlowTok{if}\NormalTok{(}\SpecialCharTok{!}\FunctionTok{require}\NormalTok{(terra)) \{}\FunctionTok{install.packages}\NormalTok{(}\StringTok{"terra"}\NormalTok{); }\FunctionTok{require}\NormalTok{(terra)\}}
\ControlFlowTok{if}\NormalTok{(}\SpecialCharTok{!}\FunctionTok{require}\NormalTok{(egvtools)) \{remotes}\SpecialCharTok{::}\FunctionTok{install\_github}\NormalTok{(}\StringTok{"aavotins/egvtools"}\NormalTok{); }\FunctionTok{require}\NormalTok{(egvtools)\}}


\CommentTok{\# Templates {-}{-}{-}{-}{-}}
\NormalTok{template100}\OtherTok{=}\FunctionTok{rast}\NormalTok{(}\StringTok{"./Templates/TemplateRasters/LV100m\_10km.tif"}\NormalTok{)}

\CommentTok{\# radii {-}{-}{-}{-}}
\FunctionTok{radius\_function}\NormalTok{(}
 \AttributeTok{kvadrati\_path =} \StringTok{"./Templates/TemplateGrids/tiles/"}\NormalTok{,}
 \AttributeTok{radii\_path   =} \StringTok{"./Templates/TemplateGridPoints/tiles/"}\NormalTok{,}
 \AttributeTok{tikls100\_path =} \StringTok{"./Templates/TemplateGrids/tikls100\_sauzeme.parquet"}\NormalTok{,}
 \AttributeTok{template\_path =} \StringTok{"./Templates/TemplateRasters/LV100m\_10km.tif"}\NormalTok{,}
 \AttributeTok{input\_layers  =} \FunctionTok{c}\NormalTok{(}\StringTok{"./RasterGrids\_100m/2024/RAW/FarmlandGrassland\_GrasslandsAbandoned\_cell.tif"}\NormalTok{),}
 \AttributeTok{layer\_prefixes =} \FunctionTok{c}\NormalTok{(}\StringTok{"FarmlandGrassland\_GrasslandsAbandoned"}\NormalTok{),}
 \AttributeTok{output\_dir   =} \StringTok{"./RasterGrids\_100m/2024/RAW/"}\NormalTok{,}
 \AttributeTok{n\_workers   =} \DecValTok{6}\NormalTok{,}
 \AttributeTok{radii     =} \FunctionTok{c}\NormalTok{(}\StringTok{"r1250"}\NormalTok{),}
 \AttributeTok{radius\_mode  =} \StringTok{"sparse"}\NormalTok{,}
 \AttributeTok{extract\_fun  =} \StringTok{"mean"}\NormalTok{,}
 \AttributeTok{fill\_missing  =} \ConstantTok{TRUE}\NormalTok{,}
 \AttributeTok{IDW\_weight   =} \DecValTok{2}\NormalTok{,}
 \AttributeTok{future\_max\_size =} \DecValTok{40} \SpecialCharTok{*} \DecValTok{1024}\SpecialCharTok{\^{}}\DecValTok{3}\NormalTok{)}


\CommentTok{\# FarmlandGrassland\_GrasslandsAbandoned\_r1250.tif   egv\_217}
\NormalTok{slanis}\OtherTok{=}\FunctionTok{rast}\NormalTok{(}\StringTok{"./RasterGrids\_100m/2024/RAW/FarmlandGrassland\_GrasslandsAbandoned\_r1250.tif"}\NormalTok{)}
\FunctionTok{names}\NormalTok{(slanis)}\OtherTok{=}\StringTok{"egv\_217"}
\NormalTok{slanis2}\OtherTok{=}\FunctionTok{project}\NormalTok{(slanis,template100)}
\FunctionTok{writeRaster}\NormalTok{(slanis2,}
      \StringTok{"./RasterGrids\_100m/2024/RAW/FarmlandGrassland\_GrasslandsAbandoned\_r1250.tif"}\NormalTok{,}
      \AttributeTok{overwrite=}\ConstantTok{TRUE}\NormalTok{)}

\CommentTok{\# standardisation {-}{-}{-}{-}}
\ControlFlowTok{if}\NormalTok{(}\SpecialCharTok{!}\FunctionTok{require}\NormalTok{(terra)) \{}\FunctionTok{install.packages}\NormalTok{(}\StringTok{"terra"}\NormalTok{); }\FunctionTok{require}\NormalTok{(terra)\}}
\ControlFlowTok{if}\NormalTok{(}\SpecialCharTok{!}\FunctionTok{require}\NormalTok{(tidyverse)) \{}\FunctionTok{install.packages}\NormalTok{(}\StringTok{"tidyverse"}\NormalTok{); }\FunctionTok{require}\NormalTok{(tidyverse)\}}

\NormalTok{nosaukums}\OtherTok{=}\StringTok{"FarmlandGrassland\_GrasslandsAbandoned\_r1250.tif"}
\NormalTok{ielasisanas\_cels}\OtherTok{=}\FunctionTok{paste0}\NormalTok{(}\StringTok{"./RasterGrids\_100m/2024/RAW/"}\NormalTok{,nosaukums)}
\NormalTok{saglabasanas\_cels}\OtherTok{=}\FunctionTok{paste0}\NormalTok{(}\StringTok{"./RasterGrids\_100m/2024/Scaled/"}\NormalTok{,nosaukums)}
\NormalTok{slanis}\OtherTok{=}\FunctionTok{rast}\NormalTok{(ielasisanas\_cels)}
\NormalTok{videjais}\OtherTok{=}\FunctionTok{global}\NormalTok{(slanis,}\AttributeTok{fun=}\StringTok{"mean"}\NormalTok{,}\AttributeTok{na.rm=}\ConstantTok{TRUE}\NormalTok{)}
\NormalTok{centrets}\OtherTok{=}\NormalTok{slanis}\SpecialCharTok{{-}}\NormalTok{videjais[,}\DecValTok{1}\NormalTok{]}
\NormalTok{standartnovirze}\OtherTok{=}\NormalTok{terra}\SpecialCharTok{::}\FunctionTok{global}\NormalTok{(centrets,}\AttributeTok{fun=}\StringTok{"rms"}\NormalTok{,}\AttributeTok{na.rm=}\ConstantTok{TRUE}\NormalTok{)}
\NormalTok{merogots}\OtherTok{=}\NormalTok{centrets}\SpecialCharTok{/}\NormalTok{standartnovirze[,}\DecValTok{1}\NormalTok{]}
\FunctionTok{writeRaster}\NormalTok{(merogots,}
      \AttributeTok{filename=}\NormalTok{saglabasanas\_cels,}
      \AttributeTok{overwrite=}\ConstantTok{TRUE}\NormalTok{)}
\end{Highlighting}
\end{Shaded}

\section{FarmlandGrassland\_GrasslandsAbandoned\_r3000}\label{ch06.218}

\textbf{filename:} \texttt{FarmlandGrassland\_GrasslandsAbandoned\_r3000.tif}

\textbf{layername:} \texttt{egv\_218}

\textbf{English name:} Fractional cover of Abandoned Grassland within the 3 km
landscape

\textbf{Latvian name:} Neapsaimniekotu zālāju platības īpatsvars 3 km ainavā

\textbf{Procedure:} The cover fraction within a radius of 3000 m around the analysis grid cell
is calculated as the area-weighted sum of the \hyperref[ch06.215]{analysis cells} inside
the buffer, using the workflow \texttt{egvtools::radius\_function()}. During the calculation of the landscape
metric, inverse distance weighted (power = 2) gap filling on the output is
applied to ensure no missing values at the edges. Then the layer is
rewritten to set its name. Finally, the layer is standardised by
subtracting the arithmetic mean and dividing by the root mean squared error.

\begin{Shaded}
\begin{Highlighting}[]
\CommentTok{\# libs {-}{-}{-}{-}}
\ControlFlowTok{if}\NormalTok{(}\SpecialCharTok{!}\FunctionTok{require}\NormalTok{(terra)) \{}\FunctionTok{install.packages}\NormalTok{(}\StringTok{"terra"}\NormalTok{); }\FunctionTok{require}\NormalTok{(terra)\}}
\ControlFlowTok{if}\NormalTok{(}\SpecialCharTok{!}\FunctionTok{require}\NormalTok{(egvtools)) \{remotes}\SpecialCharTok{::}\FunctionTok{install\_github}\NormalTok{(}\StringTok{"aavotins/egvtools"}\NormalTok{); }\FunctionTok{require}\NormalTok{(egvtools)\}}


\CommentTok{\# Templates {-}{-}{-}{-}{-}}
\NormalTok{template100}\OtherTok{=}\FunctionTok{rast}\NormalTok{(}\StringTok{"./Templates/TemplateRasters/LV100m\_10km.tif"}\NormalTok{)}

\CommentTok{\# radii {-}{-}{-}{-}}
\FunctionTok{radius\_function}\NormalTok{(}
 \AttributeTok{kvadrati\_path =} \StringTok{"./Templates/TemplateGrids/tiles/"}\NormalTok{,}
 \AttributeTok{radii\_path   =} \StringTok{"./Templates/TemplateGridPoints/tiles/"}\NormalTok{,}
 \AttributeTok{tikls100\_path =} \StringTok{"./Templates/TemplateGrids/tikls100\_sauzeme.parquet"}\NormalTok{,}
 \AttributeTok{template\_path =} \StringTok{"./Templates/TemplateRasters/LV100m\_10km.tif"}\NormalTok{,}
 \AttributeTok{input\_layers  =} \FunctionTok{c}\NormalTok{(}\StringTok{"./RasterGrids\_100m/2024/RAW/FarmlandGrassland\_GrasslandsAbandoned\_cell.tif"}\NormalTok{),}
 \AttributeTok{layer\_prefixes =} \FunctionTok{c}\NormalTok{(}\StringTok{"FarmlandGrassland\_GrasslandsAbandoned"}\NormalTok{),}
 \AttributeTok{output\_dir   =} \StringTok{"./RasterGrids\_100m/2024/RAW/"}\NormalTok{,}
 \AttributeTok{n\_workers   =} \DecValTok{6}\NormalTok{,}
 \AttributeTok{radii     =} \FunctionTok{c}\NormalTok{(}\StringTok{"r3000"}\NormalTok{),}
 \AttributeTok{radius\_mode  =} \StringTok{"sparse"}\NormalTok{,}
 \AttributeTok{extract\_fun  =} \StringTok{"mean"}\NormalTok{,}
 \AttributeTok{fill\_missing  =} \ConstantTok{TRUE}\NormalTok{,}
 \AttributeTok{IDW\_weight   =} \DecValTok{2}\NormalTok{,}
 \AttributeTok{future\_max\_size =} \DecValTok{40} \SpecialCharTok{*} \DecValTok{1024}\SpecialCharTok{\^{}}\DecValTok{3}\NormalTok{)}


\CommentTok{\# FarmlandGrassland\_GrasslandsAbandoned\_r3000.tif   egv\_218}
\NormalTok{slanis}\OtherTok{=}\FunctionTok{rast}\NormalTok{(}\StringTok{"./RasterGrids\_100m/2024/RAW/FarmlandGrassland\_GrasslandsAbandoned\_r3000.tif"}\NormalTok{)}
\FunctionTok{names}\NormalTok{(slanis)}\OtherTok{=}\StringTok{"egv\_218"}
\NormalTok{slanis2}\OtherTok{=}\FunctionTok{project}\NormalTok{(slanis,template100)}
\FunctionTok{writeRaster}\NormalTok{(slanis2,}
      \StringTok{"./RasterGrids\_100m/2024/RAW/FarmlandGrassland\_GrasslandsAbandoned\_r3000.tif"}\NormalTok{,}
      \AttributeTok{overwrite=}\ConstantTok{TRUE}\NormalTok{)}

\CommentTok{\# standardisation {-}{-}{-}{-}}
\ControlFlowTok{if}\NormalTok{(}\SpecialCharTok{!}\FunctionTok{require}\NormalTok{(terra)) \{}\FunctionTok{install.packages}\NormalTok{(}\StringTok{"terra"}\NormalTok{); }\FunctionTok{require}\NormalTok{(terra)\}}
\ControlFlowTok{if}\NormalTok{(}\SpecialCharTok{!}\FunctionTok{require}\NormalTok{(tidyverse)) \{}\FunctionTok{install.packages}\NormalTok{(}\StringTok{"tidyverse"}\NormalTok{); }\FunctionTok{require}\NormalTok{(tidyverse)\}}

\NormalTok{nosaukums}\OtherTok{=}\StringTok{"FarmlandGrassland\_GrasslandsAbandoned\_r3000.tif"}
\NormalTok{ielasisanas\_cels}\OtherTok{=}\FunctionTok{paste0}\NormalTok{(}\StringTok{"./RasterGrids\_100m/2024/RAW/"}\NormalTok{,nosaukums)}
\NormalTok{saglabasanas\_cels}\OtherTok{=}\FunctionTok{paste0}\NormalTok{(}\StringTok{"./RasterGrids\_100m/2024/Scaled/"}\NormalTok{,nosaukums)}
\NormalTok{slanis}\OtherTok{=}\FunctionTok{rast}\NormalTok{(ielasisanas\_cels)}
\NormalTok{videjais}\OtherTok{=}\FunctionTok{global}\NormalTok{(slanis,}\AttributeTok{fun=}\StringTok{"mean"}\NormalTok{,}\AttributeTok{na.rm=}\ConstantTok{TRUE}\NormalTok{)}
\NormalTok{centrets}\OtherTok{=}\NormalTok{slanis}\SpecialCharTok{{-}}\NormalTok{videjais[,}\DecValTok{1}\NormalTok{]}
\NormalTok{standartnovirze}\OtherTok{=}\NormalTok{terra}\SpecialCharTok{::}\FunctionTok{global}\NormalTok{(centrets,}\AttributeTok{fun=}\StringTok{"rms"}\NormalTok{,}\AttributeTok{na.rm=}\ConstantTok{TRUE}\NormalTok{)}
\NormalTok{merogots}\OtherTok{=}\NormalTok{centrets}\SpecialCharTok{/}\NormalTok{standartnovirze[,}\DecValTok{1}\NormalTok{]}
\FunctionTok{writeRaster}\NormalTok{(merogots,}
      \AttributeTok{filename=}\NormalTok{saglabasanas\_cels,}
      \AttributeTok{overwrite=}\ConstantTok{TRUE}\NormalTok{)}
\end{Highlighting}
\end{Shaded}

\section{FarmlandGrassland\_GrasslandsAbandoned\_r10000}\label{ch06.219}

\textbf{filename:} \texttt{FarmlandGrassland\_GrasslandsAbandoned\_r10000.tif}

\textbf{layername:} \texttt{egv\_219}

\textbf{English name:} Fractional cover of Abandoned Grassland within the 10 km
landscape

\textbf{Latvian name:} Neapsaimniekotu zālāju platības īpatsvars 10 km ainavā

\textbf{Procedure:} The cover fraction within a radius of 10000 m around the analysis grid cell
is calculated as the area-weighted sum of the \hyperref[ch06.215]{analysis cells} inside
the buffer, using the workflow \texttt{egvtools::radius\_function()}. During the calculation of the landscape
metric, inverse distance weighted (power = 2) gap filling on the output is
applied to ensure no missing values at the edges. Then the layer is
rewritten to set its name. Finally, the layer is standardised by
subtracting the arithmetic mean and dividing by the root mean squared error.

\begin{Shaded}
\begin{Highlighting}[]
\CommentTok{\# libs {-}{-}{-}{-}}
\ControlFlowTok{if}\NormalTok{(}\SpecialCharTok{!}\FunctionTok{require}\NormalTok{(terra)) \{}\FunctionTok{install.packages}\NormalTok{(}\StringTok{"terra"}\NormalTok{); }\FunctionTok{require}\NormalTok{(terra)\}}
\ControlFlowTok{if}\NormalTok{(}\SpecialCharTok{!}\FunctionTok{require}\NormalTok{(egvtools)) \{remotes}\SpecialCharTok{::}\FunctionTok{install\_github}\NormalTok{(}\StringTok{"aavotins/egvtools"}\NormalTok{); }\FunctionTok{require}\NormalTok{(egvtools)\}}


\CommentTok{\# Templates {-}{-}{-}{-}{-}}
\NormalTok{template100}\OtherTok{=}\FunctionTok{rast}\NormalTok{(}\StringTok{"./Templates/TemplateRasters/LV100m\_10km.tif"}\NormalTok{)}

\CommentTok{\# radii {-}{-}{-}{-}}
\FunctionTok{radius\_function}\NormalTok{(}
 \AttributeTok{kvadrati\_path =} \StringTok{"./Templates/TemplateGrids/tiles/"}\NormalTok{,}
 \AttributeTok{radii\_path   =} \StringTok{"./Templates/TemplateGridPoints/tiles/"}\NormalTok{,}
 \AttributeTok{tikls100\_path =} \StringTok{"./Templates/TemplateGrids/tikls100\_sauzeme.parquet"}\NormalTok{,}
 \AttributeTok{template\_path =} \StringTok{"./Templates/TemplateRasters/LV100m\_10km.tif"}\NormalTok{,}
 \AttributeTok{input\_layers  =} \FunctionTok{c}\NormalTok{(}\StringTok{"./RasterGrids\_100m/2024/RAW/FarmlandGrassland\_GrasslandsAbandoned\_cell.tif"}\NormalTok{),}
 \AttributeTok{layer\_prefixes =} \FunctionTok{c}\NormalTok{(}\StringTok{"FarmlandGrassland\_GrasslandsAbandoned"}\NormalTok{),}
 \AttributeTok{output\_dir   =} \StringTok{"./RasterGrids\_100m/2024/RAW/"}\NormalTok{,}
 \AttributeTok{n\_workers   =} \DecValTok{6}\NormalTok{,}
 \AttributeTok{radii     =} \FunctionTok{c}\NormalTok{(}\StringTok{"r10000"}\NormalTok{),}
 \AttributeTok{radius\_mode  =} \StringTok{"sparse"}\NormalTok{,}
 \AttributeTok{extract\_fun  =} \StringTok{"mean"}\NormalTok{,}
 \AttributeTok{fill\_missing  =} \ConstantTok{TRUE}\NormalTok{,}
 \AttributeTok{IDW\_weight   =} \DecValTok{2}\NormalTok{,}
 \AttributeTok{future\_max\_size =} \DecValTok{40} \SpecialCharTok{*} \DecValTok{1024}\SpecialCharTok{\^{}}\DecValTok{3}\NormalTok{)}


\CommentTok{\# FarmlandGrassland\_GrasslandsAbandoned\_r10000.tif  egv\_219}
\NormalTok{slanis}\OtherTok{=}\FunctionTok{rast}\NormalTok{(}\StringTok{"./RasterGrids\_100m/2024/RAW/FarmlandGrassland\_GrasslandsAbandoned\_r10000.tif"}\NormalTok{)}
\FunctionTok{names}\NormalTok{(slanis)}\OtherTok{=}\StringTok{"egv\_219"}
\NormalTok{slanis2}\OtherTok{=}\FunctionTok{project}\NormalTok{(slanis,template100)}
\FunctionTok{writeRaster}\NormalTok{(slanis2,}
      \StringTok{"./RasterGrids\_100m/2024/RAW/FarmlandGrassland\_GrasslandsAbandoned\_r10000.tif"}\NormalTok{,}
      \AttributeTok{overwrite=}\ConstantTok{TRUE}\NormalTok{)}

\CommentTok{\# standardisation {-}{-}{-}{-}}
\ControlFlowTok{if}\NormalTok{(}\SpecialCharTok{!}\FunctionTok{require}\NormalTok{(terra)) \{}\FunctionTok{install.packages}\NormalTok{(}\StringTok{"terra"}\NormalTok{); }\FunctionTok{require}\NormalTok{(terra)\}}
\ControlFlowTok{if}\NormalTok{(}\SpecialCharTok{!}\FunctionTok{require}\NormalTok{(tidyverse)) \{}\FunctionTok{install.packages}\NormalTok{(}\StringTok{"tidyverse"}\NormalTok{); }\FunctionTok{require}\NormalTok{(tidyverse)\}}

\NormalTok{nosaukums}\OtherTok{=}\StringTok{"FarmlandGrassland\_GrasslandsAbandoned\_r10000.tif"}
\NormalTok{ielasisanas\_cels}\OtherTok{=}\FunctionTok{paste0}\NormalTok{(}\StringTok{"./RasterGrids\_100m/2024/RAW/"}\NormalTok{,nosaukums)}
\NormalTok{saglabasanas\_cels}\OtherTok{=}\FunctionTok{paste0}\NormalTok{(}\StringTok{"./RasterGrids\_100m/2024/Scaled/"}\NormalTok{,nosaukums)}
\NormalTok{slanis}\OtherTok{=}\FunctionTok{rast}\NormalTok{(ielasisanas\_cels)}
\NormalTok{videjais}\OtherTok{=}\FunctionTok{global}\NormalTok{(slanis,}\AttributeTok{fun=}\StringTok{"mean"}\NormalTok{,}\AttributeTok{na.rm=}\ConstantTok{TRUE}\NormalTok{)}
\NormalTok{centrets}\OtherTok{=}\NormalTok{slanis}\SpecialCharTok{{-}}\NormalTok{videjais[,}\DecValTok{1}\NormalTok{]}
\NormalTok{standartnovirze}\OtherTok{=}\NormalTok{terra}\SpecialCharTok{::}\FunctionTok{global}\NormalTok{(centrets,}\AttributeTok{fun=}\StringTok{"rms"}\NormalTok{,}\AttributeTok{na.rm=}\ConstantTok{TRUE}\NormalTok{)}
\NormalTok{merogots}\OtherTok{=}\NormalTok{centrets}\SpecialCharTok{/}\NormalTok{standartnovirze[,}\DecValTok{1}\NormalTok{]}
\FunctionTok{writeRaster}\NormalTok{(merogots,}
      \AttributeTok{filename=}\NormalTok{saglabasanas\_cels,}
      \AttributeTok{overwrite=}\ConstantTok{TRUE}\NormalTok{)}
\end{Highlighting}
\end{Shaded}

\section{FarmlandGrassland\_GrasslandsAll\_cell}\label{ch06.220}

\textbf{filename:} \texttt{FarmlandGrassland\_GrasslandsAll\_cell.tif}

\textbf{layername:} \texttt{egv\_220}

\textbf{English name:} Fractional cover of any Grassland within the analysis cell (1
ha)

\textbf{Latvian name:} Zālāju (visu veidu) platības īpatsvars analīzes šūnā (1 ha)

\textbf{Procedure:} First, the grasslands from the \hyperref[Ch05.03]{Landscape classification} are
selected (value 330 reclassified to value 1, others as 0). Once selected, the layer
is aggregated to EGV resolution using the workflow \texttt{egvtools::input2egv()}, which
calculates the arithmetic mean and thus results in a cover fraction. During aggregation, inverse
distance weighted (power = 2) gap filling on the output is applied to
ensure no missing values at the edges. Finally, the layer is standardised
by subtracting the arithmetic mean and dividing by the root mean squared error.

\begin{Shaded}
\begin{Highlighting}[]
\CommentTok{\# libs {-}{-}{-}{-}}
\ControlFlowTok{if}\NormalTok{(}\SpecialCharTok{!}\FunctionTok{require}\NormalTok{(egvtools)) \{remotes}\SpecialCharTok{::}\FunctionTok{install\_github}\NormalTok{(}\StringTok{"aavotins/egvtools"}\NormalTok{); }\FunctionTok{require}\NormalTok{(egvtools)\}}
\ControlFlowTok{if}\NormalTok{(}\SpecialCharTok{!}\FunctionTok{require}\NormalTok{(terra)) \{}\FunctionTok{install.packages}\NormalTok{(}\StringTok{"terra"}\NormalTok{); }\FunctionTok{require}\NormalTok{(terra)\}}
\ControlFlowTok{if}\NormalTok{(}\SpecialCharTok{!}\FunctionTok{require}\NormalTok{(sf)) \{}\FunctionTok{install.packages}\NormalTok{(}\StringTok{"sf"}\NormalTok{); }\FunctionTok{require}\NormalTok{(sf)\}}
\ControlFlowTok{if}\NormalTok{(}\SpecialCharTok{!}\FunctionTok{require}\NormalTok{(tidyverse)) \{}\FunctionTok{install.packages}\NormalTok{(}\StringTok{"tidyverse"}\NormalTok{); }\FunctionTok{require}\NormalTok{(tidyverse)\}}
\ControlFlowTok{if}\NormalTok{(}\SpecialCharTok{!}\FunctionTok{require}\NormalTok{(sfarrow)) \{}\FunctionTok{install.packages}\NormalTok{(}\StringTok{"sfarrow"}\NormalTok{); }\FunctionTok{require}\NormalTok{(sfarrow)\}}
\ControlFlowTok{if}\NormalTok{(}\SpecialCharTok{!}\FunctionTok{require}\NormalTok{(readxl)) \{}\FunctionTok{install.packages}\NormalTok{(}\StringTok{"readxl"}\NormalTok{); }\FunctionTok{require}\NormalTok{(readxl)\}}
\ControlFlowTok{if}\NormalTok{(}\SpecialCharTok{!}\FunctionTok{require}\NormalTok{(raster)) \{}\FunctionTok{install.packages}\NormalTok{(}\StringTok{"raster"}\NormalTok{); }\FunctionTok{require}\NormalTok{(raster)\}}
\ControlFlowTok{if}\NormalTok{(}\SpecialCharTok{!}\FunctionTok{require}\NormalTok{(fasterize)) \{}\FunctionTok{install.packages}\NormalTok{(}\StringTok{"fasterize"}\NormalTok{); }\FunctionTok{require}\NormalTok{(fasterize)\}}

\CommentTok{\# templates {-}{-}{-}{-}}
\NormalTok{template100}\OtherTok{=}\FunctionTok{rast}\NormalTok{(}\StringTok{"./Templates/TemplateRasters/LV100m\_10km.tif"}\NormalTok{)}
\NormalTok{template10}\OtherTok{=}\FunctionTok{rast}\NormalTok{(}\StringTok{"./Templates/TemplateRasters/LV10m\_10km.tif"}\NormalTok{)}
\NormalTok{rastrs10}\OtherTok{=}\FunctionTok{raster}\NormalTok{(template10)}

\NormalTok{nulls10}\OtherTok{=}\FunctionTok{rast}\NormalTok{(}\StringTok{"./Templates/TemplateRasters/nulls\_LV10m\_10km.tif"}\NormalTok{)}
\NormalTok{nulls100}\OtherTok{=}\FunctionTok{rast}\NormalTok{(}\StringTok{"./Templates/TemplateRasters/nulls\_LV100m\_10km.tif"}\NormalTok{)}

\CommentTok{\# codes {-}{-}{-}{-}}
\NormalTok{kodi}\OtherTok{=}\FunctionTok{read\_excel}\NormalTok{(}\StringTok{"./Geodata/2024/LAD/KulturuKodi\_2024.xlsx"}\NormalTok{)}
\NormalTok{kodi}\SpecialCharTok{$}\NormalTok{kods}\OtherTok{=}\FunctionTok{as.character}\NormalTok{(kodi}\SpecialCharTok{$}\NormalTok{kods)}
\CommentTok{\# LAD {-}{-}{-}{-}}
\NormalTok{lad}\OtherTok{=}\NormalTok{sfarrow}\SpecialCharTok{::}\FunctionTok{st\_read\_parquet}\NormalTok{(}\StringTok{"./Geodata/2024/LAD/Lauki\_2024.parquet"}\NormalTok{)}
\NormalTok{lad}\SpecialCharTok{$}\NormalTok{yes}\OtherTok{=}\DecValTok{1}
\NormalTok{lad}\OtherTok{=}\NormalTok{lad }\SpecialCharTok{\%\textgreater{}\%} 
 \FunctionTok{left\_join}\NormalTok{(kodi,}\AttributeTok{by=}\FunctionTok{c}\NormalTok{(}\StringTok{"PRODUCT\_CODE"}\OtherTok{=}\StringTok{"kods"}\NormalTok{))}

\CommentTok{\# simple landscape {-}{-}{-}{-}}
\NormalTok{simple\_landscape}\OtherTok{=}\FunctionTok{rast}\NormalTok{(}\StringTok{"RasterGrids\_10m/2024/Ainava\_vienk\_mask.tif"}\NormalTok{)}


\CommentTok{\# FarmlandGrassland\_GrasslandsAll\_cell.tif  egv\_220 {-}{-}{-}{-}}
\NormalTok{landscape\_grasslands}\OtherTok{=}\FunctionTok{ifel}\NormalTok{(simple\_landscape}\SpecialCharTok{==}\DecValTok{330}\NormalTok{,}\DecValTok{1}\NormalTok{,}\DecValTok{0}\NormalTok{)}

\NormalTok{i2e\_rez}\OtherTok{=}\NormalTok{egvtools}\SpecialCharTok{::}\FunctionTok{input2egv}\NormalTok{(}\AttributeTok{input=}\NormalTok{landscape\_grasslands,}
              \AttributeTok{egv\_template=} \StringTok{"./Templates/TemplateRasters/LV100m\_10km.tif"}\NormalTok{,}
              \AttributeTok{summary\_function =} \StringTok{"average"}\NormalTok{,}
              \AttributeTok{missing\_job =} \StringTok{"FillOutput"}\NormalTok{,}
              \AttributeTok{outlocation =} \StringTok{"./RasterGrids\_100m/2024/RAW/"}\NormalTok{,}
              \AttributeTok{outfilename =} \StringTok{"FarmlandGrassland\_GrasslandsAll\_cell.tif"}\NormalTok{,}
              \AttributeTok{layername =} \StringTok{"egv\_220"}\NormalTok{,}
              \AttributeTok{idw\_weight =} \DecValTok{2}\NormalTok{,}
              \AttributeTok{plot\_gaps =} \ConstantTok{FALSE}\NormalTok{,}\AttributeTok{plot\_final =} \ConstantTok{TRUE}\NormalTok{)}
\NormalTok{i2e\_rez}
\FunctionTok{rm}\NormalTok{(i2e\_rez)}
\FunctionTok{rm}\NormalTok{(landscape\_grasslands)}

\CommentTok{\# standardisation {-}{-}{-}{-}}
\ControlFlowTok{if}\NormalTok{(}\SpecialCharTok{!}\FunctionTok{require}\NormalTok{(terra)) \{}\FunctionTok{install.packages}\NormalTok{(}\StringTok{"terra"}\NormalTok{); }\FunctionTok{require}\NormalTok{(terra)\}}
\ControlFlowTok{if}\NormalTok{(}\SpecialCharTok{!}\FunctionTok{require}\NormalTok{(tidyverse)) \{}\FunctionTok{install.packages}\NormalTok{(}\StringTok{"tidyverse"}\NormalTok{); }\FunctionTok{require}\NormalTok{(tidyverse)\}}

\NormalTok{nosaukums}\OtherTok{=}\StringTok{"FarmlandGrassland\_GrasslandsAll\_cell.tif"}
\NormalTok{ielasisanas\_cels}\OtherTok{=}\FunctionTok{paste0}\NormalTok{(}\StringTok{"./RasterGrids\_100m/2024/RAW/"}\NormalTok{,nosaukums)}
\NormalTok{saglabasanas\_cels}\OtherTok{=}\FunctionTok{paste0}\NormalTok{(}\StringTok{"./RasterGrids\_100m/2024/Scaled/"}\NormalTok{,nosaukums)}
\NormalTok{slanis}\OtherTok{=}\FunctionTok{rast}\NormalTok{(ielasisanas\_cels)}
\NormalTok{videjais}\OtherTok{=}\FunctionTok{global}\NormalTok{(slanis,}\AttributeTok{fun=}\StringTok{"mean"}\NormalTok{,}\AttributeTok{na.rm=}\ConstantTok{TRUE}\NormalTok{)}
\NormalTok{centrets}\OtherTok{=}\NormalTok{slanis}\SpecialCharTok{{-}}\NormalTok{videjais[,}\DecValTok{1}\NormalTok{]}
\NormalTok{standartnovirze}\OtherTok{=}\NormalTok{terra}\SpecialCharTok{::}\FunctionTok{global}\NormalTok{(centrets,}\AttributeTok{fun=}\StringTok{"rms"}\NormalTok{,}\AttributeTok{na.rm=}\ConstantTok{TRUE}\NormalTok{)}
\NormalTok{merogots}\OtherTok{=}\NormalTok{centrets}\SpecialCharTok{/}\NormalTok{standartnovirze[,}\DecValTok{1}\NormalTok{]}
\FunctionTok{writeRaster}\NormalTok{(merogots,}
      \AttributeTok{filename=}\NormalTok{saglabasanas\_cels,}
      \AttributeTok{overwrite=}\ConstantTok{TRUE}\NormalTok{)}
\end{Highlighting}
\end{Shaded}

\section{FarmlandGrassland\_GrasslandsAll\_r500}\label{ch06.221}

\textbf{filename:} \texttt{FarmlandGrassland\_GrasslandsAll\_r500.tif}

\textbf{layername:} \texttt{egv\_221}

\textbf{English name:} Fractional cover of any Grassland within the 0.5 km landscape

\textbf{Latvian name:} Zālāju (visu veidu) platības īpatsvars 0,5 km ainavā

\textbf{Procedure:} The cover fraction within a radius of 500 m around the analysis grid cell is
calculated as the area-weighted sum of the \hyperref[ch06.220]{analysis cells} inside the
buffer, using the workflow \texttt{egvtools::radius\_function()}. During the calculation of the landscape metric,
inverse distance weighted (power = 2) gap filling on the output is applied
to ensure no missing values at the edges. Then the layer is rewritten to set
its name. Finally, the layer is standardised by subtracting the arithmetic
mean and dividing by the root mean squared error.

\begin{Shaded}
\begin{Highlighting}[]
\CommentTok{\# libs {-}{-}{-}{-}}
\ControlFlowTok{if}\NormalTok{(}\SpecialCharTok{!}\FunctionTok{require}\NormalTok{(terra)) \{}\FunctionTok{install.packages}\NormalTok{(}\StringTok{"terra"}\NormalTok{); }\FunctionTok{require}\NormalTok{(terra)\}}
\ControlFlowTok{if}\NormalTok{(}\SpecialCharTok{!}\FunctionTok{require}\NormalTok{(egvtools)) \{remotes}\SpecialCharTok{::}\FunctionTok{install\_github}\NormalTok{(}\StringTok{"aavotins/egvtools"}\NormalTok{); }\FunctionTok{require}\NormalTok{(egvtools)\}}


\CommentTok{\# Templates {-}{-}{-}{-}{-}}
\NormalTok{template100}\OtherTok{=}\FunctionTok{rast}\NormalTok{(}\StringTok{"./Templates/TemplateRasters/LV100m\_10km.tif"}\NormalTok{)}

\CommentTok{\# radii {-}{-}{-}{-}}
\FunctionTok{radius\_function}\NormalTok{(}
 \AttributeTok{kvadrati\_path =} \StringTok{"./Templates/TemplateGrids/tiles/"}\NormalTok{,}
 \AttributeTok{radii\_path   =} \StringTok{"./Templates/TemplateGridPoints/tiles/"}\NormalTok{,}
 \AttributeTok{tikls100\_path =} \StringTok{"./Templates/TemplateGrids/tikls100\_sauzeme.parquet"}\NormalTok{,}
 \AttributeTok{template\_path =} \StringTok{"./Templates/TemplateRasters/LV100m\_10km.tif"}\NormalTok{,}
 \AttributeTok{input\_layers  =} \FunctionTok{c}\NormalTok{(}\StringTok{"./RasterGrids\_100m/2024/RAW/FarmlandGrassland\_GrasslandsAll\_cell.tif"}\NormalTok{),}
 \AttributeTok{layer\_prefixes =} \FunctionTok{c}\NormalTok{(}\StringTok{"FarmlandGrassland\_GrasslandsAll"}\NormalTok{),}
 \AttributeTok{output\_dir   =} \StringTok{"./RasterGrids\_100m/2024/RAW/"}\NormalTok{,}
 \AttributeTok{n\_workers   =} \DecValTok{6}\NormalTok{,}
 \AttributeTok{radii     =} \FunctionTok{c}\NormalTok{(}\StringTok{"r500"}\NormalTok{),}
 \AttributeTok{radius\_mode  =} \StringTok{"sparse"}\NormalTok{,}
 \AttributeTok{extract\_fun  =} \StringTok{"mean"}\NormalTok{,}
 \AttributeTok{fill\_missing  =} \ConstantTok{TRUE}\NormalTok{,}
 \AttributeTok{IDW\_weight   =} \DecValTok{2}\NormalTok{,}
 \AttributeTok{future\_max\_size =} \DecValTok{40} \SpecialCharTok{*} \DecValTok{1024}\SpecialCharTok{\^{}}\DecValTok{3}\NormalTok{)}


\CommentTok{\# FarmlandGrassland\_GrasslandsAll\_r500.tif  egv\_221}
\NormalTok{slanis}\OtherTok{=}\FunctionTok{rast}\NormalTok{(}\StringTok{"./RasterGrids\_100m/2024/RAW/FarmlandGrassland\_GrasslandsAll\_r500.tif"}\NormalTok{)}
\FunctionTok{names}\NormalTok{(slanis)}\OtherTok{=}\StringTok{"egv\_221"}
\NormalTok{slanis2}\OtherTok{=}\FunctionTok{project}\NormalTok{(slanis,template100)}
\FunctionTok{writeRaster}\NormalTok{(slanis2,}
      \StringTok{"./RasterGrids\_100m/2024/RAW/FarmlandGrassland\_GrasslandsAll\_r500.tif"}\NormalTok{,}
      \AttributeTok{overwrite=}\ConstantTok{TRUE}\NormalTok{)}

\CommentTok{\# standardisation {-}{-}{-}{-}}
\ControlFlowTok{if}\NormalTok{(}\SpecialCharTok{!}\FunctionTok{require}\NormalTok{(terra)) \{}\FunctionTok{install.packages}\NormalTok{(}\StringTok{"terra"}\NormalTok{); }\FunctionTok{require}\NormalTok{(terra)\}}
\ControlFlowTok{if}\NormalTok{(}\SpecialCharTok{!}\FunctionTok{require}\NormalTok{(tidyverse)) \{}\FunctionTok{install.packages}\NormalTok{(}\StringTok{"tidyverse"}\NormalTok{); }\FunctionTok{require}\NormalTok{(tidyverse)\}}

\NormalTok{nosaukums}\OtherTok{=}\StringTok{"FarmlandGrassland\_GrasslandsAll\_r500.tif"}
\NormalTok{ielasisanas\_cels}\OtherTok{=}\FunctionTok{paste0}\NormalTok{(}\StringTok{"./RasterGrids\_100m/2024/RAW/"}\NormalTok{,nosaukums)}
\NormalTok{saglabasanas\_cels}\OtherTok{=}\FunctionTok{paste0}\NormalTok{(}\StringTok{"./RasterGrids\_100m/2024/Scaled/"}\NormalTok{,nosaukums)}
\NormalTok{slanis}\OtherTok{=}\FunctionTok{rast}\NormalTok{(ielasisanas\_cels)}
\NormalTok{videjais}\OtherTok{=}\FunctionTok{global}\NormalTok{(slanis,}\AttributeTok{fun=}\StringTok{"mean"}\NormalTok{,}\AttributeTok{na.rm=}\ConstantTok{TRUE}\NormalTok{)}
\NormalTok{centrets}\OtherTok{=}\NormalTok{slanis}\SpecialCharTok{{-}}\NormalTok{videjais[,}\DecValTok{1}\NormalTok{]}
\NormalTok{standartnovirze}\OtherTok{=}\NormalTok{terra}\SpecialCharTok{::}\FunctionTok{global}\NormalTok{(centrets,}\AttributeTok{fun=}\StringTok{"rms"}\NormalTok{,}\AttributeTok{na.rm=}\ConstantTok{TRUE}\NormalTok{)}
\NormalTok{merogots}\OtherTok{=}\NormalTok{centrets}\SpecialCharTok{/}\NormalTok{standartnovirze[,}\DecValTok{1}\NormalTok{]}
\FunctionTok{writeRaster}\NormalTok{(merogots,}
      \AttributeTok{filename=}\NormalTok{saglabasanas\_cels,}
      \AttributeTok{overwrite=}\ConstantTok{TRUE}\NormalTok{)}
\end{Highlighting}
\end{Shaded}

\section{FarmlandGrassland\_GrasslandsAll\_r1250}\label{ch06.222}

\textbf{filename:} \texttt{FarmlandGrassland\_GrasslandsAll\_r1250.tif}

\textbf{layername:} \texttt{egv\_222}

\textbf{English name:} Fractional cover of any Grassland within the 1.25 km landscape

\textbf{Latvian name:} Zālāju (visu veidu) platības īpatsvars 1,25 km ainavā

\textbf{Procedure:} The cover fraction within a radius of 1250 m around the analysis grid cell
is calculated as the area-weighted sum of the \hyperref[ch06.220]{analysis cells} inside
the buffer, using the workflow \texttt{egvtools::radius\_function()}. During the calculation of the landscape
metric, inverse distance weighted (power = 2) gap filling on the output is
applied to ensure no missing values at the edges. Then the layer is
rewritten to set its name. Finally, the layer is standardised by
subtracting the arithmetic mean and dividing by the root mean squared error.

\begin{Shaded}
\begin{Highlighting}[]
\CommentTok{\# libs {-}{-}{-}{-}}
\ControlFlowTok{if}\NormalTok{(}\SpecialCharTok{!}\FunctionTok{require}\NormalTok{(terra)) \{}\FunctionTok{install.packages}\NormalTok{(}\StringTok{"terra"}\NormalTok{); }\FunctionTok{require}\NormalTok{(terra)\}}
\ControlFlowTok{if}\NormalTok{(}\SpecialCharTok{!}\FunctionTok{require}\NormalTok{(egvtools)) \{remotes}\SpecialCharTok{::}\FunctionTok{install\_github}\NormalTok{(}\StringTok{"aavotins/egvtools"}\NormalTok{); }\FunctionTok{require}\NormalTok{(egvtools)\}}


\CommentTok{\# Templates {-}{-}{-}{-}{-}}
\NormalTok{template100}\OtherTok{=}\FunctionTok{rast}\NormalTok{(}\StringTok{"./Templates/TemplateRasters/LV100m\_10km.tif"}\NormalTok{)}

\CommentTok{\# radii {-}{-}{-}{-}}
\FunctionTok{radius\_function}\NormalTok{(}
 \AttributeTok{kvadrati\_path =} \StringTok{"./Templates/TemplateGrids/tiles/"}\NormalTok{,}
 \AttributeTok{radii\_path   =} \StringTok{"./Templates/TemplateGridPoints/tiles/"}\NormalTok{,}
 \AttributeTok{tikls100\_path =} \StringTok{"./Templates/TemplateGrids/tikls100\_sauzeme.parquet"}\NormalTok{,}
 \AttributeTok{template\_path =} \StringTok{"./Templates/TemplateRasters/LV100m\_10km.tif"}\NormalTok{,}
 \AttributeTok{input\_layers  =} \FunctionTok{c}\NormalTok{(}\StringTok{"./RasterGrids\_100m/2024/RAW/FarmlandGrassland\_GrasslandsAll\_cell.tif"}\NormalTok{),}
 \AttributeTok{layer\_prefixes =} \FunctionTok{c}\NormalTok{(}\StringTok{"FarmlandGrassland\_GrasslandsAll"}\NormalTok{),}
 \AttributeTok{output\_dir   =} \StringTok{"./RasterGrids\_100m/2024/RAW/"}\NormalTok{,}
 \AttributeTok{n\_workers   =} \DecValTok{6}\NormalTok{,}
 \AttributeTok{radii     =} \FunctionTok{c}\NormalTok{(}\StringTok{"r1250"}\NormalTok{),}
 \AttributeTok{radius\_mode  =} \StringTok{"sparse"}\NormalTok{,}
 \AttributeTok{extract\_fun  =} \StringTok{"mean"}\NormalTok{,}
 \AttributeTok{fill\_missing  =} \ConstantTok{TRUE}\NormalTok{,}
 \AttributeTok{IDW\_weight   =} \DecValTok{2}\NormalTok{,}
 \AttributeTok{future\_max\_size =} \DecValTok{40} \SpecialCharTok{*} \DecValTok{1024}\SpecialCharTok{\^{}}\DecValTok{3}\NormalTok{)}


\CommentTok{\# FarmlandGrassland\_GrasslandsAll\_r1250.tif egv\_222}
\NormalTok{slanis}\OtherTok{=}\FunctionTok{rast}\NormalTok{(}\StringTok{"./RasterGrids\_100m/2024/RAW/FarmlandGrassland\_GrasslandsAll\_r1250.tif"}\NormalTok{)}
\FunctionTok{names}\NormalTok{(slanis)}\OtherTok{=}\StringTok{"egv\_222"}
\NormalTok{slanis2}\OtherTok{=}\FunctionTok{project}\NormalTok{(slanis,template100)}
\FunctionTok{writeRaster}\NormalTok{(slanis2,}
      \StringTok{"./RasterGrids\_100m/2024/RAW/FarmlandGrassland\_GrasslandsAll\_r1250.tif"}\NormalTok{,}
      \AttributeTok{overwrite=}\ConstantTok{TRUE}\NormalTok{)}

\CommentTok{\# standardisation {-}{-}{-}{-}}
\ControlFlowTok{if}\NormalTok{(}\SpecialCharTok{!}\FunctionTok{require}\NormalTok{(terra)) \{}\FunctionTok{install.packages}\NormalTok{(}\StringTok{"terra"}\NormalTok{); }\FunctionTok{require}\NormalTok{(terra)\}}
\ControlFlowTok{if}\NormalTok{(}\SpecialCharTok{!}\FunctionTok{require}\NormalTok{(tidyverse)) \{}\FunctionTok{install.packages}\NormalTok{(}\StringTok{"tidyverse"}\NormalTok{); }\FunctionTok{require}\NormalTok{(tidyverse)\}}

\NormalTok{nosaukums}\OtherTok{=}\StringTok{"FarmlandGrassland\_GrasslandsAll\_r1250.tif"}
\NormalTok{ielasisanas\_cels}\OtherTok{=}\FunctionTok{paste0}\NormalTok{(}\StringTok{"./RasterGrids\_100m/2024/RAW/"}\NormalTok{,nosaukums)}
\NormalTok{saglabasanas\_cels}\OtherTok{=}\FunctionTok{paste0}\NormalTok{(}\StringTok{"./RasterGrids\_100m/2024/Scaled/"}\NormalTok{,nosaukums)}
\NormalTok{slanis}\OtherTok{=}\FunctionTok{rast}\NormalTok{(ielasisanas\_cels)}
\NormalTok{videjais}\OtherTok{=}\FunctionTok{global}\NormalTok{(slanis,}\AttributeTok{fun=}\StringTok{"mean"}\NormalTok{,}\AttributeTok{na.rm=}\ConstantTok{TRUE}\NormalTok{)}
\NormalTok{centrets}\OtherTok{=}\NormalTok{slanis}\SpecialCharTok{{-}}\NormalTok{videjais[,}\DecValTok{1}\NormalTok{]}
\NormalTok{standartnovirze}\OtherTok{=}\NormalTok{terra}\SpecialCharTok{::}\FunctionTok{global}\NormalTok{(centrets,}\AttributeTok{fun=}\StringTok{"rms"}\NormalTok{,}\AttributeTok{na.rm=}\ConstantTok{TRUE}\NormalTok{)}
\NormalTok{merogots}\OtherTok{=}\NormalTok{centrets}\SpecialCharTok{/}\NormalTok{standartnovirze[,}\DecValTok{1}\NormalTok{]}
\FunctionTok{writeRaster}\NormalTok{(merogots,}
      \AttributeTok{filename=}\NormalTok{saglabasanas\_cels,}
      \AttributeTok{overwrite=}\ConstantTok{TRUE}\NormalTok{)}
\end{Highlighting}
\end{Shaded}

\section{FarmlandGrassland\_GrasslandsAll\_r3000}\label{ch06.223}

\textbf{filename:} \texttt{FarmlandGrassland\_GrasslandsAll\_r3000.tif}

\textbf{layername:} \texttt{egv\_223}

\textbf{English name:} Fractional cover of any Grassland within the 3 km landscape

\textbf{Latvian name:} Zālāju (visu veidu) platības īpatsvars 3 km ainavā

\textbf{Procedure:} The cover fraction within a radius of 3000 m around the analysis grid cell
is calculated as the area-weighted sum of the \hyperref[ch06.220]{analysis cells} inside
the buffer, using the workflow \texttt{egvtools::radius\_function()}. During the calculation of the landscape
metric, inverse distance weighted (power = 2) gap filling on the output is
applied to ensure no missing values at the edges. Then the layer is
rewritten to set its name. Finally, the layer is standardised by
subtracting the arithmetic mean and dividing by the root mean squared error.

\begin{Shaded}
\begin{Highlighting}[]
\CommentTok{\# libs {-}{-}{-}{-}}
\ControlFlowTok{if}\NormalTok{(}\SpecialCharTok{!}\FunctionTok{require}\NormalTok{(terra)) \{}\FunctionTok{install.packages}\NormalTok{(}\StringTok{"terra"}\NormalTok{); }\FunctionTok{require}\NormalTok{(terra)\}}
\ControlFlowTok{if}\NormalTok{(}\SpecialCharTok{!}\FunctionTok{require}\NormalTok{(egvtools)) \{remotes}\SpecialCharTok{::}\FunctionTok{install\_github}\NormalTok{(}\StringTok{"aavotins/egvtools"}\NormalTok{); }\FunctionTok{require}\NormalTok{(egvtools)\}}


\CommentTok{\# Templates {-}{-}{-}{-}{-}}
\NormalTok{template100}\OtherTok{=}\FunctionTok{rast}\NormalTok{(}\StringTok{"./Templates/TemplateRasters/LV100m\_10km.tif"}\NormalTok{)}

\CommentTok{\# radii {-}{-}{-}{-}}
\FunctionTok{radius\_function}\NormalTok{(}
 \AttributeTok{kvadrati\_path =} \StringTok{"./Templates/TemplateGrids/tiles/"}\NormalTok{,}
 \AttributeTok{radii\_path   =} \StringTok{"./Templates/TemplateGridPoints/tiles/"}\NormalTok{,}
 \AttributeTok{tikls100\_path =} \StringTok{"./Templates/TemplateGrids/tikls100\_sauzeme.parquet"}\NormalTok{,}
 \AttributeTok{template\_path =} \StringTok{"./Templates/TemplateRasters/LV100m\_10km.tif"}\NormalTok{,}
 \AttributeTok{input\_layers  =} \FunctionTok{c}\NormalTok{(}\StringTok{"./RasterGrids\_100m/2024/RAW/FarmlandGrassland\_GrasslandsAll\_cell.tif"}\NormalTok{),}
 \AttributeTok{layer\_prefixes =} \FunctionTok{c}\NormalTok{(}\StringTok{"FarmlandGrassland\_GrasslandsAll"}\NormalTok{),}
 \AttributeTok{output\_dir   =} \StringTok{"./RasterGrids\_100m/2024/RAW/"}\NormalTok{,}
 \AttributeTok{n\_workers   =} \DecValTok{6}\NormalTok{,}
 \AttributeTok{radii     =} \FunctionTok{c}\NormalTok{(}\StringTok{"r3000"}\NormalTok{),}
 \AttributeTok{radius\_mode  =} \StringTok{"sparse"}\NormalTok{,}
 \AttributeTok{extract\_fun  =} \StringTok{"mean"}\NormalTok{,}
 \AttributeTok{fill\_missing  =} \ConstantTok{TRUE}\NormalTok{,}
 \AttributeTok{IDW\_weight   =} \DecValTok{2}\NormalTok{,}
 \AttributeTok{future\_max\_size =} \DecValTok{40} \SpecialCharTok{*} \DecValTok{1024}\SpecialCharTok{\^{}}\DecValTok{3}\NormalTok{)}


\CommentTok{\# FarmlandGrassland\_GrasslandsAll\_r3000.tif egv\_223}
\NormalTok{slanis}\OtherTok{=}\FunctionTok{rast}\NormalTok{(}\StringTok{"./RasterGrids\_100m/2024/RAW/FarmlandGrassland\_GrasslandsAll\_r3000.tif"}\NormalTok{)}
\FunctionTok{names}\NormalTok{(slanis)}\OtherTok{=}\StringTok{"egv\_223"}
\NormalTok{slanis2}\OtherTok{=}\FunctionTok{project}\NormalTok{(slanis,template100)}
\FunctionTok{writeRaster}\NormalTok{(slanis2,}
      \StringTok{"./RasterGrids\_100m/2024/RAW/FarmlandGrassland\_GrasslandsAll\_r3000.tif"}\NormalTok{,}
      \AttributeTok{overwrite=}\ConstantTok{TRUE}\NormalTok{)}

\CommentTok{\# standardisation {-}{-}{-}{-}}
\ControlFlowTok{if}\NormalTok{(}\SpecialCharTok{!}\FunctionTok{require}\NormalTok{(terra)) \{}\FunctionTok{install.packages}\NormalTok{(}\StringTok{"terra"}\NormalTok{); }\FunctionTok{require}\NormalTok{(terra)\}}
\ControlFlowTok{if}\NormalTok{(}\SpecialCharTok{!}\FunctionTok{require}\NormalTok{(tidyverse)) \{}\FunctionTok{install.packages}\NormalTok{(}\StringTok{"tidyverse"}\NormalTok{); }\FunctionTok{require}\NormalTok{(tidyverse)\}}

\NormalTok{nosaukums}\OtherTok{=}\StringTok{"FarmlandGrassland\_GrasslandsAll\_r3000.tif"}
\NormalTok{ielasisanas\_cels}\OtherTok{=}\FunctionTok{paste0}\NormalTok{(}\StringTok{"./RasterGrids\_100m/2024/RAW/"}\NormalTok{,nosaukums)}
\NormalTok{saglabasanas\_cels}\OtherTok{=}\FunctionTok{paste0}\NormalTok{(}\StringTok{"./RasterGrids\_100m/2024/Scaled/"}\NormalTok{,nosaukums)}
\NormalTok{slanis}\OtherTok{=}\FunctionTok{rast}\NormalTok{(ielasisanas\_cels)}
\NormalTok{videjais}\OtherTok{=}\FunctionTok{global}\NormalTok{(slanis,}\AttributeTok{fun=}\StringTok{"mean"}\NormalTok{,}\AttributeTok{na.rm=}\ConstantTok{TRUE}\NormalTok{)}
\NormalTok{centrets}\OtherTok{=}\NormalTok{slanis}\SpecialCharTok{{-}}\NormalTok{videjais[,}\DecValTok{1}\NormalTok{]}
\NormalTok{standartnovirze}\OtherTok{=}\NormalTok{terra}\SpecialCharTok{::}\FunctionTok{global}\NormalTok{(centrets,}\AttributeTok{fun=}\StringTok{"rms"}\NormalTok{,}\AttributeTok{na.rm=}\ConstantTok{TRUE}\NormalTok{)}
\NormalTok{merogots}\OtherTok{=}\NormalTok{centrets}\SpecialCharTok{/}\NormalTok{standartnovirze[,}\DecValTok{1}\NormalTok{]}
\FunctionTok{writeRaster}\NormalTok{(merogots,}
      \AttributeTok{filename=}\NormalTok{saglabasanas\_cels,}
      \AttributeTok{overwrite=}\ConstantTok{TRUE}\NormalTok{)}
\end{Highlighting}
\end{Shaded}

\section{FarmlandGrassland\_GrasslandsAll\_r10000}\label{ch06.224}

\textbf{filename:} \texttt{FarmlandGrassland\_GrasslandsAll\_r10000.tif}

\textbf{layername:} \texttt{egv\_224}

\textbf{English name:} Fractional cover of any Grassland within the 10 km landscape

\textbf{Latvian name:} Zālāju (visu veidu) platības īpatsvars 10 km ainavā

\textbf{Procedure:} The cover fraction within a radius of 10000 m around the analysis grid cell
is calculated as the area-weighted sum of the \hyperref[ch06.220]{analysis cells} inside
the buffer, using the workflow \texttt{egvtools::radius\_function()}. During the calculation of the landscape
metric, inverse distance weighted (power = 2) gap filling on the output is
applied to ensure no missing values at the edges. Then the layer is
rewritten to set its name. Finally, the layer is standardised by
subtracting the arithmetic mean and dividing by the root mean squared error.

\begin{Shaded}
\begin{Highlighting}[]
\CommentTok{\# libs {-}{-}{-}{-}}
\ControlFlowTok{if}\NormalTok{(}\SpecialCharTok{!}\FunctionTok{require}\NormalTok{(terra)) \{}\FunctionTok{install.packages}\NormalTok{(}\StringTok{"terra"}\NormalTok{); }\FunctionTok{require}\NormalTok{(terra)\}}
\ControlFlowTok{if}\NormalTok{(}\SpecialCharTok{!}\FunctionTok{require}\NormalTok{(egvtools)) \{remotes}\SpecialCharTok{::}\FunctionTok{install\_github}\NormalTok{(}\StringTok{"aavotins/egvtools"}\NormalTok{); }\FunctionTok{require}\NormalTok{(egvtools)\}}


\CommentTok{\# Templates {-}{-}{-}{-}{-}}
\NormalTok{template100}\OtherTok{=}\FunctionTok{rast}\NormalTok{(}\StringTok{"./Templates/TemplateRasters/LV100m\_10km.tif"}\NormalTok{)}

\CommentTok{\# radii {-}{-}{-}{-}}
\FunctionTok{radius\_function}\NormalTok{(}
 \AttributeTok{kvadrati\_path =} \StringTok{"./Templates/TemplateGrids/tiles/"}\NormalTok{,}
 \AttributeTok{radii\_path   =} \StringTok{"./Templates/TemplateGridPoints/tiles/"}\NormalTok{,}
 \AttributeTok{tikls100\_path =} \StringTok{"./Templates/TemplateGrids/tikls100\_sauzeme.parquet"}\NormalTok{,}
 \AttributeTok{template\_path =} \StringTok{"./Templates/TemplateRasters/LV100m\_10km.tif"}\NormalTok{,}
 \AttributeTok{input\_layers  =} \FunctionTok{c}\NormalTok{(}\StringTok{"./RasterGrids\_100m/2024/RAW/FarmlandGrassland\_GrasslandsAll\_cell.tif"}\NormalTok{),}
 \AttributeTok{layer\_prefixes =} \FunctionTok{c}\NormalTok{(}\StringTok{"FarmlandGrassland\_GrasslandsAll"}\NormalTok{),}
 \AttributeTok{output\_dir   =} \StringTok{"./RasterGrids\_100m/2024/RAW/"}\NormalTok{,}
 \AttributeTok{n\_workers   =} \DecValTok{6}\NormalTok{,}
 \AttributeTok{radii     =} \FunctionTok{c}\NormalTok{(}\StringTok{"r10000"}\NormalTok{),}
 \AttributeTok{radius\_mode  =} \StringTok{"sparse"}\NormalTok{,}
 \AttributeTok{extract\_fun  =} \StringTok{"mean"}\NormalTok{,}
 \AttributeTok{fill\_missing  =} \ConstantTok{TRUE}\NormalTok{,}
 \AttributeTok{IDW\_weight   =} \DecValTok{2}\NormalTok{,}
 \AttributeTok{future\_max\_size =} \DecValTok{40} \SpecialCharTok{*} \DecValTok{1024}\SpecialCharTok{\^{}}\DecValTok{3}\NormalTok{)}


\CommentTok{\# FarmlandGrassland\_GrasslandsAll\_r10000.tif    egv\_224}
\NormalTok{slanis}\OtherTok{=}\FunctionTok{rast}\NormalTok{(}\StringTok{"./RasterGrids\_100m/2024/RAW/FarmlandGrassland\_GrasslandsAll\_r10000.tif"}\NormalTok{)}
\FunctionTok{names}\NormalTok{(slanis)}\OtherTok{=}\StringTok{"egv\_224"}
\NormalTok{slanis2}\OtherTok{=}\FunctionTok{project}\NormalTok{(slanis,template100)}
\FunctionTok{writeRaster}\NormalTok{(slanis2,}
      \StringTok{"./RasterGrids\_100m/2024/RAW/FarmlandGrassland\_GrasslandsAll\_r10000.tif"}\NormalTok{,}
      \AttributeTok{overwrite=}\ConstantTok{TRUE}\NormalTok{)}

\CommentTok{\# standardisation {-}{-}{-}{-}}
\ControlFlowTok{if}\NormalTok{(}\SpecialCharTok{!}\FunctionTok{require}\NormalTok{(terra)) \{}\FunctionTok{install.packages}\NormalTok{(}\StringTok{"terra"}\NormalTok{); }\FunctionTok{require}\NormalTok{(terra)\}}
\ControlFlowTok{if}\NormalTok{(}\SpecialCharTok{!}\FunctionTok{require}\NormalTok{(tidyverse)) \{}\FunctionTok{install.packages}\NormalTok{(}\StringTok{"tidyverse"}\NormalTok{); }\FunctionTok{require}\NormalTok{(tidyverse)\}}

\NormalTok{nosaukums}\OtherTok{=}\StringTok{"FarmlandGrassland\_GrasslandsAll\_r10000.tif"}
\NormalTok{ielasisanas\_cels}\OtherTok{=}\FunctionTok{paste0}\NormalTok{(}\StringTok{"./RasterGrids\_100m/2024/RAW/"}\NormalTok{,nosaukums)}
\NormalTok{saglabasanas\_cels}\OtherTok{=}\FunctionTok{paste0}\NormalTok{(}\StringTok{"./RasterGrids\_100m/2024/Scaled/"}\NormalTok{,nosaukums)}
\NormalTok{slanis}\OtherTok{=}\FunctionTok{rast}\NormalTok{(ielasisanas\_cels)}
\NormalTok{videjais}\OtherTok{=}\FunctionTok{global}\NormalTok{(slanis,}\AttributeTok{fun=}\StringTok{"mean"}\NormalTok{,}\AttributeTok{na.rm=}\ConstantTok{TRUE}\NormalTok{)}
\NormalTok{centrets}\OtherTok{=}\NormalTok{slanis}\SpecialCharTok{{-}}\NormalTok{videjais[,}\DecValTok{1}\NormalTok{]}
\NormalTok{standartnovirze}\OtherTok{=}\NormalTok{terra}\SpecialCharTok{::}\FunctionTok{global}\NormalTok{(centrets,}\AttributeTok{fun=}\StringTok{"rms"}\NormalTok{,}\AttributeTok{na.rm=}\ConstantTok{TRUE}\NormalTok{)}
\NormalTok{merogots}\OtherTok{=}\NormalTok{centrets}\SpecialCharTok{/}\NormalTok{standartnovirze[,}\DecValTok{1}\NormalTok{]}
\FunctionTok{writeRaster}\NormalTok{(merogots,}
      \AttributeTok{filename=}\NormalTok{saglabasanas\_cels,}
      \AttributeTok{overwrite=}\ConstantTok{TRUE}\NormalTok{)}
\end{Highlighting}
\end{Shaded}

\section{FarmlandGrassland\_GrasslandsPermanent\_cell}\label{ch06.225}

\textbf{filename:} \texttt{FarmlandGrassland\_GrasslandsPermanent\_cell.tif}

\textbf{layername:} \texttt{egv\_225}

\textbf{English name:} Fractional cover of Permanent Grassland within the analysis
cell (1 ha)

\textbf{Latvian name:} Ilggadīgu zālāju platības īpatsvars analīzes šūnā (1 ha)

\textbf{Procedure:} First, agricultural parcels declared as permanent grasslands are
selected from the \hyperref[Ch04.02]{Rural Support Service's information on declared
fields}. These geometries are then rasterised to input resolution,
ensuring value 1 at the polygon locations and value 0 elsewhere. Rasterisation
is performed with the workflow \texttt{egvtools::polygon2input()}. Once rasterised, the layer is
aggregated to EGV resolution using the workflow \texttt{egvtools::input2egv()}, which
calculates the arithmetic mean and thus results in a cover fraction. During aggregation, inverse
distance weighted (power = 2) gap filling on the output is applied to
ensure no missing values at the edges. Finally, the layer is standardised
by subtracting the arithmetic mean and dividing by the root mean squared error.

\begin{Shaded}
\begin{Highlighting}[]
\CommentTok{\# libs {-}{-}{-}{-}}
\ControlFlowTok{if}\NormalTok{(}\SpecialCharTok{!}\FunctionTok{require}\NormalTok{(egvtools)) \{remotes}\SpecialCharTok{::}\FunctionTok{install\_github}\NormalTok{(}\StringTok{"aavotins/egvtools"}\NormalTok{); }\FunctionTok{require}\NormalTok{(egvtools)\}}
\ControlFlowTok{if}\NormalTok{(}\SpecialCharTok{!}\FunctionTok{require}\NormalTok{(terra)) \{}\FunctionTok{install.packages}\NormalTok{(}\StringTok{"terra"}\NormalTok{); }\FunctionTok{require}\NormalTok{(terra)\}}
\ControlFlowTok{if}\NormalTok{(}\SpecialCharTok{!}\FunctionTok{require}\NormalTok{(sf)) \{}\FunctionTok{install.packages}\NormalTok{(}\StringTok{"sf"}\NormalTok{); }\FunctionTok{require}\NormalTok{(sf)\}}
\ControlFlowTok{if}\NormalTok{(}\SpecialCharTok{!}\FunctionTok{require}\NormalTok{(tidyverse)) \{}\FunctionTok{install.packages}\NormalTok{(}\StringTok{"tidyverse"}\NormalTok{); }\FunctionTok{require}\NormalTok{(tidyverse)\}}
\ControlFlowTok{if}\NormalTok{(}\SpecialCharTok{!}\FunctionTok{require}\NormalTok{(sfarrow)) \{}\FunctionTok{install.packages}\NormalTok{(}\StringTok{"sfarrow"}\NormalTok{); }\FunctionTok{require}\NormalTok{(sfarrow)\}}
\ControlFlowTok{if}\NormalTok{(}\SpecialCharTok{!}\FunctionTok{require}\NormalTok{(readxl)) \{}\FunctionTok{install.packages}\NormalTok{(}\StringTok{"readxl"}\NormalTok{); }\FunctionTok{require}\NormalTok{(readxl)\}}
\ControlFlowTok{if}\NormalTok{(}\SpecialCharTok{!}\FunctionTok{require}\NormalTok{(raster)) \{}\FunctionTok{install.packages}\NormalTok{(}\StringTok{"raster"}\NormalTok{); }\FunctionTok{require}\NormalTok{(raster)\}}
\ControlFlowTok{if}\NormalTok{(}\SpecialCharTok{!}\FunctionTok{require}\NormalTok{(fasterize)) \{}\FunctionTok{install.packages}\NormalTok{(}\StringTok{"fasterize"}\NormalTok{); }\FunctionTok{require}\NormalTok{(fasterize)\}}

\CommentTok{\# templates {-}{-}{-}{-}}
\NormalTok{template100}\OtherTok{=}\FunctionTok{rast}\NormalTok{(}\StringTok{"./Templates/TemplateRasters/LV100m\_10km.tif"}\NormalTok{)}
\NormalTok{template10}\OtherTok{=}\FunctionTok{rast}\NormalTok{(}\StringTok{"./Templates/TemplateRasters/LV10m\_10km.tif"}\NormalTok{)}
\NormalTok{rastrs10}\OtherTok{=}\FunctionTok{raster}\NormalTok{(template10)}

\NormalTok{nulls10}\OtherTok{=}\FunctionTok{rast}\NormalTok{(}\StringTok{"./Templates/TemplateRasters/nulls\_LV10m\_10km.tif"}\NormalTok{)}
\NormalTok{nulls100}\OtherTok{=}\FunctionTok{rast}\NormalTok{(}\StringTok{"./Templates/TemplateRasters/nulls\_LV100m\_10km.tif"}\NormalTok{)}

\CommentTok{\# codes {-}{-}{-}{-}}
\NormalTok{kodi}\OtherTok{=}\FunctionTok{read\_excel}\NormalTok{(}\StringTok{"./Geodata/2024/LAD/KulturuKodi\_2024.xlsx"}\NormalTok{)}
\NormalTok{kodi}\SpecialCharTok{$}\NormalTok{kods}\OtherTok{=}\FunctionTok{as.character}\NormalTok{(kodi}\SpecialCharTok{$}\NormalTok{kods)}
\CommentTok{\# LAD {-}{-}{-}{-}}
\NormalTok{lad}\OtherTok{=}\NormalTok{sfarrow}\SpecialCharTok{::}\FunctionTok{st\_read\_parquet}\NormalTok{(}\StringTok{"./Geodata/2024/LAD/Lauki\_2024.parquet"}\NormalTok{)}
\NormalTok{lad}\SpecialCharTok{$}\NormalTok{yes}\OtherTok{=}\DecValTok{1}
\NormalTok{lad}\OtherTok{=}\NormalTok{lad }\SpecialCharTok{\%\textgreater{}\%} 
 \FunctionTok{left\_join}\NormalTok{(kodi,}\AttributeTok{by=}\FunctionTok{c}\NormalTok{(}\StringTok{"PRODUCT\_CODE"}\OtherTok{=}\StringTok{"kods"}\NormalTok{))}

\CommentTok{\# simple landscape {-}{-}{-}{-}}
\NormalTok{simple\_landscape}\OtherTok{=}\FunctionTok{rast}\NormalTok{(}\StringTok{"RasterGrids\_10m/2024/Ainava\_vienk\_mask.tif"}\NormalTok{)}


\CommentTok{\# FarmlandGrassland\_GrasslandsPermanent\_cell.tif    egv\_225 {-}{-}{-}{-}}
\NormalTok{dati}\OtherTok{=}\NormalTok{lad }\SpecialCharTok{\%\textgreater{}\%} 
 \FunctionTok{filter}\NormalTok{(SDM\_grupa\_sakums}\SpecialCharTok{==}\StringTok{"zālāji (ilggadīgie)"}\NormalTok{)}
 
\FunctionTok{table}\NormalTok{(dati}\SpecialCharTok{$}\NormalTok{SDM\_grupa\_sakums,}\AttributeTok{useNA=}\StringTok{"always"}\NormalTok{)}

\NormalTok{p2i\_rez}\OtherTok{=}\NormalTok{egvtools}\SpecialCharTok{::}\FunctionTok{polygon2input}\NormalTok{(}\AttributeTok{vector\_data =}\NormalTok{ dati,}
                \AttributeTok{template\_path =} \StringTok{"./Templates/TemplateRasters/LV10m\_10km.tif"}\NormalTok{,}
                \AttributeTok{out\_path =} \StringTok{"./RasterGrids\_10m/2024/"}\NormalTok{,}
                \AttributeTok{file\_name =} \StringTok{"FarmlandGrassland\_GrasslandsPermanent\_input.tif"}\NormalTok{,}
                \AttributeTok{value\_field =} \StringTok{"yes"}\NormalTok{,}
                \AttributeTok{prepare=}\ConstantTok{FALSE}\NormalTok{,}
                \AttributeTok{background\_raster =} \StringTok{"./Templates/TemplateRasters/nulls\_LV10m\_10km.tif"}\NormalTok{,}
                \AttributeTok{plot\_result =} \ConstantTok{TRUE}\NormalTok{)}
\NormalTok{p2i\_rez}
\NormalTok{i2e\_rez}\OtherTok{=}\NormalTok{egvtools}\SpecialCharTok{::}\FunctionTok{input2egv}\NormalTok{(}\AttributeTok{input=}\FunctionTok{paste0}\NormalTok{(}\StringTok{"./RasterGrids\_10m/2024/"}\NormalTok{,}
                     \StringTok{"FarmlandGrassland\_GrasslandsPermanent\_input.tif"}\NormalTok{),}
              \AttributeTok{egv\_template=} \StringTok{"./Templates/TemplateRasters/LV100m\_10km.tif"}\NormalTok{,}
              \AttributeTok{summary\_function =} \StringTok{"average"}\NormalTok{,}
              \AttributeTok{missing\_job =} \StringTok{"FillOutput"}\NormalTok{,}
              \AttributeTok{outlocation =} \StringTok{"./RasterGrids\_100m/2024/RAW/"}\NormalTok{,}
              \AttributeTok{outfilename =} \StringTok{"FarmlandGrassland\_GrasslandsPermanent\_cell.tif"}\NormalTok{,}
              \AttributeTok{layername =} \StringTok{"egv\_225"}\NormalTok{,}
              \AttributeTok{idw\_weight =} \DecValTok{2}\NormalTok{,}
              \AttributeTok{plot\_gaps =} \ConstantTok{FALSE}\NormalTok{,}\AttributeTok{plot\_final =} \ConstantTok{TRUE}\NormalTok{)}
\NormalTok{i2e\_rez}
\FunctionTok{rm}\NormalTok{(p2i\_rez)}
\FunctionTok{rm}\NormalTok{(i2e\_rez)}
\FunctionTok{rm}\NormalTok{(dati)}
\FunctionTok{unlink}\NormalTok{(}\StringTok{"./RasterGrids\_10m/2024/FarmlandGrassland\_GrasslandsPermanent\_input.tif"}\NormalTok{)}


\CommentTok{\# standardisation {-}{-}{-}{-}}
\ControlFlowTok{if}\NormalTok{(}\SpecialCharTok{!}\FunctionTok{require}\NormalTok{(terra)) \{}\FunctionTok{install.packages}\NormalTok{(}\StringTok{"terra"}\NormalTok{); }\FunctionTok{require}\NormalTok{(terra)\}}
\ControlFlowTok{if}\NormalTok{(}\SpecialCharTok{!}\FunctionTok{require}\NormalTok{(tidyverse)) \{}\FunctionTok{install.packages}\NormalTok{(}\StringTok{"tidyverse"}\NormalTok{); }\FunctionTok{require}\NormalTok{(tidyverse)\}}

\NormalTok{nosaukums}\OtherTok{=}\StringTok{"FarmlandGrassland\_GrasslandsPermanent\_cell.tif"}
\NormalTok{ielasisanas\_cels}\OtherTok{=}\FunctionTok{paste0}\NormalTok{(}\StringTok{"./RasterGrids\_100m/2024/RAW/"}\NormalTok{,nosaukums)}
\NormalTok{saglabasanas\_cels}\OtherTok{=}\FunctionTok{paste0}\NormalTok{(}\StringTok{"./RasterGrids\_100m/2024/Scaled/"}\NormalTok{,nosaukums)}
\NormalTok{slanis}\OtherTok{=}\FunctionTok{rast}\NormalTok{(ielasisanas\_cels)}
\NormalTok{videjais}\OtherTok{=}\FunctionTok{global}\NormalTok{(slanis,}\AttributeTok{fun=}\StringTok{"mean"}\NormalTok{,}\AttributeTok{na.rm=}\ConstantTok{TRUE}\NormalTok{)}
\NormalTok{centrets}\OtherTok{=}\NormalTok{slanis}\SpecialCharTok{{-}}\NormalTok{videjais[,}\DecValTok{1}\NormalTok{]}
\NormalTok{standartnovirze}\OtherTok{=}\NormalTok{terra}\SpecialCharTok{::}\FunctionTok{global}\NormalTok{(centrets,}\AttributeTok{fun=}\StringTok{"rms"}\NormalTok{,}\AttributeTok{na.rm=}\ConstantTok{TRUE}\NormalTok{)}
\NormalTok{merogots}\OtherTok{=}\NormalTok{centrets}\SpecialCharTok{/}\NormalTok{standartnovirze[,}\DecValTok{1}\NormalTok{]}
\FunctionTok{writeRaster}\NormalTok{(merogots,}
      \AttributeTok{filename=}\NormalTok{saglabasanas\_cels,}
      \AttributeTok{overwrite=}\ConstantTok{TRUE}\NormalTok{)}
\end{Highlighting}
\end{Shaded}

\section{FarmlandGrassland\_GrasslandsPermanent\_r500}\label{ch06.226}

\textbf{filename:} \texttt{FarmlandGrassland\_GrasslandsPermanent\_r500.tif}

\textbf{layername:} \texttt{egv\_226}

\textbf{English name:} Fractional cover of Permanent Grassland within the 0.5 km
landscape

\textbf{Latvian name:} Ilggadīgu zālāju platības īpatsvars 0,5 km ainavā

\textbf{Procedure:} The cover fraction within a radius of 500 m around the analysis grid cell is
calculated as the area-weighted sum of the \hyperref[ch06.225]{analysis cells} inside the
buffer, using the workflow \texttt{egvtools::radius\_function()}. During the calculation of the landscape metric,
inverse distance weighted (power = 2) gap filling on the output is applied
to ensure no missing values at the edges. Then the layer is rewritten to set
its name. Finally, the layer is standardised by subtracting the arithmetic
mean and dividing by the root mean squared error.

\begin{Shaded}
\begin{Highlighting}[]
\CommentTok{\# libs {-}{-}{-}{-}}
\ControlFlowTok{if}\NormalTok{(}\SpecialCharTok{!}\FunctionTok{require}\NormalTok{(terra)) \{}\FunctionTok{install.packages}\NormalTok{(}\StringTok{"terra"}\NormalTok{); }\FunctionTok{require}\NormalTok{(terra)\}}
\ControlFlowTok{if}\NormalTok{(}\SpecialCharTok{!}\FunctionTok{require}\NormalTok{(egvtools)) \{remotes}\SpecialCharTok{::}\FunctionTok{install\_github}\NormalTok{(}\StringTok{"aavotins/egvtools"}\NormalTok{); }\FunctionTok{require}\NormalTok{(egvtools)\}}


\CommentTok{\# Templates {-}{-}{-}{-}{-}}
\NormalTok{template100}\OtherTok{=}\FunctionTok{rast}\NormalTok{(}\StringTok{"./Templates/TemplateRasters/LV100m\_10km.tif"}\NormalTok{)}

\CommentTok{\# radii {-}{-}{-}{-}}
\FunctionTok{radius\_function}\NormalTok{(}
 \AttributeTok{kvadrati\_path =} \StringTok{"./Templates/TemplateGrids/tiles/"}\NormalTok{,}
 \AttributeTok{radii\_path   =} \StringTok{"./Templates/TemplateGridPoints/tiles/"}\NormalTok{,}
 \AttributeTok{tikls100\_path =} \StringTok{"./Templates/TemplateGrids/tikls100\_sauzeme.parquet"}\NormalTok{,}
 \AttributeTok{template\_path =} \StringTok{"./Templates/TemplateRasters/LV100m\_10km.tif"}\NormalTok{,}
 \AttributeTok{input\_layers  =} \FunctionTok{c}\NormalTok{(}\StringTok{"./RasterGrids\_100m/2024/RAW/FarmlandGrassland\_GrasslandsPermanent\_cell.tif"}\NormalTok{),}
 \AttributeTok{layer\_prefixes =} \FunctionTok{c}\NormalTok{(}\StringTok{"FarmlandGrassland\_GrasslandsPermanent"}\NormalTok{),}
 \AttributeTok{output\_dir   =} \StringTok{"./RasterGrids\_100m/2024/RAW/"}\NormalTok{,}
 \AttributeTok{n\_workers   =} \DecValTok{6}\NormalTok{,}
 \AttributeTok{radii     =} \FunctionTok{c}\NormalTok{(}\StringTok{"r500"}\NormalTok{),}
 \AttributeTok{radius\_mode  =} \StringTok{"sparse"}\NormalTok{,}
 \AttributeTok{extract\_fun  =} \StringTok{"mean"}\NormalTok{,}
 \AttributeTok{fill\_missing  =} \ConstantTok{TRUE}\NormalTok{,}
 \AttributeTok{IDW\_weight   =} \DecValTok{2}\NormalTok{,}
 \AttributeTok{future\_max\_size =} \DecValTok{40} \SpecialCharTok{*} \DecValTok{1024}\SpecialCharTok{\^{}}\DecValTok{3}\NormalTok{)}


\CommentTok{\# FarmlandGrassland\_GrasslandsPermanent\_r500.tif    egv\_226}
\NormalTok{slanis}\OtherTok{=}\FunctionTok{rast}\NormalTok{(}\StringTok{"./RasterGrids\_100m/2024/RAW/FarmlandGrassland\_GrasslandsPermanent\_r500.tif"}\NormalTok{)}
\FunctionTok{names}\NormalTok{(slanis)}\OtherTok{=}\StringTok{"egv\_226"}
\NormalTok{slanis2}\OtherTok{=}\FunctionTok{project}\NormalTok{(slanis,template100)}
\FunctionTok{writeRaster}\NormalTok{(slanis2,}
      \StringTok{"./RasterGrids\_100m/2024/RAW/FarmlandGrassland\_GrasslandsPermanent\_r500.tif"}\NormalTok{,}
      \AttributeTok{overwrite=}\ConstantTok{TRUE}\NormalTok{)}

\CommentTok{\# standardisation {-}{-}{-}{-}}
\ControlFlowTok{if}\NormalTok{(}\SpecialCharTok{!}\FunctionTok{require}\NormalTok{(terra)) \{}\FunctionTok{install.packages}\NormalTok{(}\StringTok{"terra"}\NormalTok{); }\FunctionTok{require}\NormalTok{(terra)\}}
\ControlFlowTok{if}\NormalTok{(}\SpecialCharTok{!}\FunctionTok{require}\NormalTok{(tidyverse)) \{}\FunctionTok{install.packages}\NormalTok{(}\StringTok{"tidyverse"}\NormalTok{); }\FunctionTok{require}\NormalTok{(tidyverse)\}}

\NormalTok{nosaukums}\OtherTok{=}\StringTok{"FarmlandGrassland\_GrasslandsPermanent\_r500.tif"}
\NormalTok{ielasisanas\_cels}\OtherTok{=}\FunctionTok{paste0}\NormalTok{(}\StringTok{"./RasterGrids\_100m/2024/RAW/"}\NormalTok{,nosaukums)}
\NormalTok{saglabasanas\_cels}\OtherTok{=}\FunctionTok{paste0}\NormalTok{(}\StringTok{"./RasterGrids\_100m/2024/Scaled/"}\NormalTok{,nosaukums)}
\NormalTok{slanis}\OtherTok{=}\FunctionTok{rast}\NormalTok{(ielasisanas\_cels)}
\NormalTok{videjais}\OtherTok{=}\FunctionTok{global}\NormalTok{(slanis,}\AttributeTok{fun=}\StringTok{"mean"}\NormalTok{,}\AttributeTok{na.rm=}\ConstantTok{TRUE}\NormalTok{)}
\NormalTok{centrets}\OtherTok{=}\NormalTok{slanis}\SpecialCharTok{{-}}\NormalTok{videjais[,}\DecValTok{1}\NormalTok{]}
\NormalTok{standartnovirze}\OtherTok{=}\NormalTok{terra}\SpecialCharTok{::}\FunctionTok{global}\NormalTok{(centrets,}\AttributeTok{fun=}\StringTok{"rms"}\NormalTok{,}\AttributeTok{na.rm=}\ConstantTok{TRUE}\NormalTok{)}
\NormalTok{merogots}\OtherTok{=}\NormalTok{centrets}\SpecialCharTok{/}\NormalTok{standartnovirze[,}\DecValTok{1}\NormalTok{]}
\FunctionTok{writeRaster}\NormalTok{(merogots,}
      \AttributeTok{filename=}\NormalTok{saglabasanas\_cels,}
      \AttributeTok{overwrite=}\ConstantTok{TRUE}\NormalTok{)}
\end{Highlighting}
\end{Shaded}

\section{FarmlandGrassland\_GrasslandsPermanent\_r1250}\label{ch06.227}

\textbf{filename:} \texttt{FarmlandGrassland\_GrasslandsPermanent\_r1250.tif}

\textbf{layername:} \texttt{egv\_227}

\textbf{English name:} Fractional cover of Permanent Grassland within the 1.25 km
landscape

\textbf{Latvian name:} Ilggadīgu zālāju platības īpatsvars 1,25 km ainavā

\textbf{Procedure:} The cover fraction within a radius of 1250 m around the analysis grid cell
is calculated as the area-weighted sum of the \hyperref[ch06.225]{analysis cells} inside
the buffer, using the workflow \texttt{egvtools::radius\_function()}. During the calculation of the landscape
metric, inverse distance weighted (power = 2) gap filling on the output is
applied to ensure no missing values at the edges. Then the layer is
rewritten to set its name. Finally, the layer is standardised by
subtracting the arithmetic mean and dividing by the root mean squared error.

\begin{Shaded}
\begin{Highlighting}[]
\CommentTok{\# libs {-}{-}{-}{-}}
\ControlFlowTok{if}\NormalTok{(}\SpecialCharTok{!}\FunctionTok{require}\NormalTok{(terra)) \{}\FunctionTok{install.packages}\NormalTok{(}\StringTok{"terra"}\NormalTok{); }\FunctionTok{require}\NormalTok{(terra)\}}
\ControlFlowTok{if}\NormalTok{(}\SpecialCharTok{!}\FunctionTok{require}\NormalTok{(egvtools)) \{remotes}\SpecialCharTok{::}\FunctionTok{install\_github}\NormalTok{(}\StringTok{"aavotins/egvtools"}\NormalTok{); }\FunctionTok{require}\NormalTok{(egvtools)\}}


\CommentTok{\# Templates {-}{-}{-}{-}{-}}
\NormalTok{template100}\OtherTok{=}\FunctionTok{rast}\NormalTok{(}\StringTok{"./Templates/TemplateRasters/LV100m\_10km.tif"}\NormalTok{)}

\CommentTok{\# radii {-}{-}{-}{-}}
\FunctionTok{radius\_function}\NormalTok{(}
 \AttributeTok{kvadrati\_path =} \StringTok{"./Templates/TemplateGrids/tiles/"}\NormalTok{,}
 \AttributeTok{radii\_path   =} \StringTok{"./Templates/TemplateGridPoints/tiles/"}\NormalTok{,}
 \AttributeTok{tikls100\_path =} \StringTok{"./Templates/TemplateGrids/tikls100\_sauzeme.parquet"}\NormalTok{,}
 \AttributeTok{template\_path =} \StringTok{"./Templates/TemplateRasters/LV100m\_10km.tif"}\NormalTok{,}
 \AttributeTok{input\_layers  =} \FunctionTok{c}\NormalTok{(}\StringTok{"./RasterGrids\_100m/2024/RAW/FarmlandGrassland\_GrasslandsPermanent\_cell.tif"}\NormalTok{),}
 \AttributeTok{layer\_prefixes =} \FunctionTok{c}\NormalTok{(}\StringTok{"FarmlandGrassland\_GrasslandsPermanent"}\NormalTok{),}
 \AttributeTok{output\_dir   =} \StringTok{"./RasterGrids\_100m/2024/RAW/"}\NormalTok{,}
 \AttributeTok{n\_workers   =} \DecValTok{6}\NormalTok{,}
 \AttributeTok{radii     =} \FunctionTok{c}\NormalTok{(}\StringTok{"r1250"}\NormalTok{),}
 \AttributeTok{radius\_mode  =} \StringTok{"sparse"}\NormalTok{,}
 \AttributeTok{extract\_fun  =} \StringTok{"mean"}\NormalTok{,}
 \AttributeTok{fill\_missing  =} \ConstantTok{TRUE}\NormalTok{,}
 \AttributeTok{IDW\_weight   =} \DecValTok{2}\NormalTok{,}
 \AttributeTok{future\_max\_size =} \DecValTok{40} \SpecialCharTok{*} \DecValTok{1024}\SpecialCharTok{\^{}}\DecValTok{3}\NormalTok{)}


\CommentTok{\# FarmlandGrassland\_GrasslandsPermanent\_r1250.tif   egv\_227}
\NormalTok{slanis}\OtherTok{=}\FunctionTok{rast}\NormalTok{(}\StringTok{"./RasterGrids\_100m/2024/RAW/FarmlandGrassland\_GrasslandsPermanent\_r1250.tif"}\NormalTok{)}
\FunctionTok{names}\NormalTok{(slanis)}\OtherTok{=}\StringTok{"egv\_227"}
\NormalTok{slanis2}\OtherTok{=}\FunctionTok{project}\NormalTok{(slanis,template100)}
\FunctionTok{writeRaster}\NormalTok{(slanis2,}
      \StringTok{"./RasterGrids\_100m/2024/RAW/FarmlandGrassland\_GrasslandsPermanent\_r1250.tif"}\NormalTok{,}
      \AttributeTok{overwrite=}\ConstantTok{TRUE}\NormalTok{)}

\CommentTok{\# standardisation {-}{-}{-}{-}}
\ControlFlowTok{if}\NormalTok{(}\SpecialCharTok{!}\FunctionTok{require}\NormalTok{(terra)) \{}\FunctionTok{install.packages}\NormalTok{(}\StringTok{"terra"}\NormalTok{); }\FunctionTok{require}\NormalTok{(terra)\}}
\ControlFlowTok{if}\NormalTok{(}\SpecialCharTok{!}\FunctionTok{require}\NormalTok{(tidyverse)) \{}\FunctionTok{install.packages}\NormalTok{(}\StringTok{"tidyverse"}\NormalTok{); }\FunctionTok{require}\NormalTok{(tidyverse)\}}

\NormalTok{nosaukums}\OtherTok{=}\StringTok{"FarmlandGrassland\_GrasslandsPermanent\_r1250.tif"}
\NormalTok{ielasisanas\_cels}\OtherTok{=}\FunctionTok{paste0}\NormalTok{(}\StringTok{"./RasterGrids\_100m/2024/RAW/"}\NormalTok{,nosaukums)}
\NormalTok{saglabasanas\_cels}\OtherTok{=}\FunctionTok{paste0}\NormalTok{(}\StringTok{"./RasterGrids\_100m/2024/Scaled/"}\NormalTok{,nosaukums)}
\NormalTok{slanis}\OtherTok{=}\FunctionTok{rast}\NormalTok{(ielasisanas\_cels)}
\NormalTok{videjais}\OtherTok{=}\FunctionTok{global}\NormalTok{(slanis,}\AttributeTok{fun=}\StringTok{"mean"}\NormalTok{,}\AttributeTok{na.rm=}\ConstantTok{TRUE}\NormalTok{)}
\NormalTok{centrets}\OtherTok{=}\NormalTok{slanis}\SpecialCharTok{{-}}\NormalTok{videjais[,}\DecValTok{1}\NormalTok{]}
\NormalTok{standartnovirze}\OtherTok{=}\NormalTok{terra}\SpecialCharTok{::}\FunctionTok{global}\NormalTok{(centrets,}\AttributeTok{fun=}\StringTok{"rms"}\NormalTok{,}\AttributeTok{na.rm=}\ConstantTok{TRUE}\NormalTok{)}
\NormalTok{merogots}\OtherTok{=}\NormalTok{centrets}\SpecialCharTok{/}\NormalTok{standartnovirze[,}\DecValTok{1}\NormalTok{]}
\FunctionTok{writeRaster}\NormalTok{(merogots,}
      \AttributeTok{filename=}\NormalTok{saglabasanas\_cels,}
      \AttributeTok{overwrite=}\ConstantTok{TRUE}\NormalTok{)}
\end{Highlighting}
\end{Shaded}

\section{FarmlandGrassland\_GrasslandsPermanent\_r3000}\label{ch06.228}

\textbf{filename:} \texttt{FarmlandGrassland\_GrasslandsPermanent\_r3000.tif}

\textbf{layername:} \texttt{egv\_228}

\textbf{English name:} Fractional cover of Permanent Grassland within the 3 km
landscape

\textbf{Latvian name:} Ilggadīgu zālāju platības īpatsvars 3 km ainavā

\textbf{Procedure:} The cover fraction within a radius of 3000 m around the analysis grid cell
is calculated as the area-weighted sum of the \hyperref[ch06.225]{analysis cells} inside
the buffer, using the workflow \texttt{egvtools::radius\_function()}. During the calculation of the landscape
metric, inverse distance weighted (power = 2) gap filling on the output is
applied to ensure no missing values at the edges. Then the layer is
rewritten to set its name. Finally, the layer is standardised by
subtracting the arithmetic mean and dividing by the root mean squared error.

\begin{Shaded}
\begin{Highlighting}[]
\CommentTok{\# libs {-}{-}{-}{-}}
\ControlFlowTok{if}\NormalTok{(}\SpecialCharTok{!}\FunctionTok{require}\NormalTok{(terra)) \{}\FunctionTok{install.packages}\NormalTok{(}\StringTok{"terra"}\NormalTok{); }\FunctionTok{require}\NormalTok{(terra)\}}
\ControlFlowTok{if}\NormalTok{(}\SpecialCharTok{!}\FunctionTok{require}\NormalTok{(egvtools)) \{remotes}\SpecialCharTok{::}\FunctionTok{install\_github}\NormalTok{(}\StringTok{"aavotins/egvtools"}\NormalTok{); }\FunctionTok{require}\NormalTok{(egvtools)\}}


\CommentTok{\# Templates {-}{-}{-}{-}{-}}
\NormalTok{template100}\OtherTok{=}\FunctionTok{rast}\NormalTok{(}\StringTok{"./Templates/TemplateRasters/LV100m\_10km.tif"}\NormalTok{)}

\CommentTok{\# radii {-}{-}{-}{-}}
\FunctionTok{radius\_function}\NormalTok{(}
 \AttributeTok{kvadrati\_path =} \StringTok{"./Templates/TemplateGrids/tiles/"}\NormalTok{,}
 \AttributeTok{radii\_path   =} \StringTok{"./Templates/TemplateGridPoints/tiles/"}\NormalTok{,}
 \AttributeTok{tikls100\_path =} \StringTok{"./Templates/TemplateGrids/tikls100\_sauzeme.parquet"}\NormalTok{,}
 \AttributeTok{template\_path =} \StringTok{"./Templates/TemplateRasters/LV100m\_10km.tif"}\NormalTok{,}
 \AttributeTok{input\_layers  =} \FunctionTok{c}\NormalTok{(}\StringTok{"./RasterGrids\_100m/2024/RAW/FarmlandGrassland\_GrasslandsPermanent\_cell.tif"}\NormalTok{),}
 \AttributeTok{layer\_prefixes =} \FunctionTok{c}\NormalTok{(}\StringTok{"FarmlandGrassland\_GrasslandsPermanent"}\NormalTok{),}
 \AttributeTok{output\_dir   =} \StringTok{"./RasterGrids\_100m/2024/RAW/"}\NormalTok{,}
 \AttributeTok{n\_workers   =} \DecValTok{6}\NormalTok{,}
 \AttributeTok{radii     =} \FunctionTok{c}\NormalTok{(}\StringTok{"r3000"}\NormalTok{),}
 \AttributeTok{radius\_mode  =} \StringTok{"sparse"}\NormalTok{,}
 \AttributeTok{extract\_fun  =} \StringTok{"mean"}\NormalTok{,}
 \AttributeTok{fill\_missing  =} \ConstantTok{TRUE}\NormalTok{,}
 \AttributeTok{IDW\_weight   =} \DecValTok{2}\NormalTok{,}
 \AttributeTok{future\_max\_size =} \DecValTok{40} \SpecialCharTok{*} \DecValTok{1024}\SpecialCharTok{\^{}}\DecValTok{3}\NormalTok{)}


\CommentTok{\# FarmlandGrassland\_GrasslandsPermanent\_r3000.tif   egv\_228}
\NormalTok{slanis}\OtherTok{=}\FunctionTok{rast}\NormalTok{(}\StringTok{"./RasterGrids\_100m/2024/RAW/FarmlandGrassland\_GrasslandsPermanent\_r3000.tif"}\NormalTok{)}
\FunctionTok{names}\NormalTok{(slanis)}\OtherTok{=}\StringTok{"egv\_228"}
\NormalTok{slanis2}\OtherTok{=}\FunctionTok{project}\NormalTok{(slanis,template100)}
\FunctionTok{writeRaster}\NormalTok{(slanis2,}
      \StringTok{"./RasterGrids\_100m/2024/RAW/FarmlandGrassland\_GrasslandsPermanent\_r3000.tif"}\NormalTok{,}
      \AttributeTok{overwrite=}\ConstantTok{TRUE}\NormalTok{)}

\CommentTok{\# standardisation {-}{-}{-}{-}}
\ControlFlowTok{if}\NormalTok{(}\SpecialCharTok{!}\FunctionTok{require}\NormalTok{(terra)) \{}\FunctionTok{install.packages}\NormalTok{(}\StringTok{"terra"}\NormalTok{); }\FunctionTok{require}\NormalTok{(terra)\}}
\ControlFlowTok{if}\NormalTok{(}\SpecialCharTok{!}\FunctionTok{require}\NormalTok{(tidyverse)) \{}\FunctionTok{install.packages}\NormalTok{(}\StringTok{"tidyverse"}\NormalTok{); }\FunctionTok{require}\NormalTok{(tidyverse)\}}

\NormalTok{nosaukums}\OtherTok{=}\StringTok{"FarmlandGrassland\_GrasslandsPermanent\_r3000.tif"}
\NormalTok{ielasisanas\_cels}\OtherTok{=}\FunctionTok{paste0}\NormalTok{(}\StringTok{"./RasterGrids\_100m/2024/RAW/"}\NormalTok{,nosaukums)}
\NormalTok{saglabasanas\_cels}\OtherTok{=}\FunctionTok{paste0}\NormalTok{(}\StringTok{"./RasterGrids\_100m/2024/Scaled/"}\NormalTok{,nosaukums)}
\NormalTok{slanis}\OtherTok{=}\FunctionTok{rast}\NormalTok{(ielasisanas\_cels)}
\NormalTok{videjais}\OtherTok{=}\FunctionTok{global}\NormalTok{(slanis,}\AttributeTok{fun=}\StringTok{"mean"}\NormalTok{,}\AttributeTok{na.rm=}\ConstantTok{TRUE}\NormalTok{)}
\NormalTok{centrets}\OtherTok{=}\NormalTok{slanis}\SpecialCharTok{{-}}\NormalTok{videjais[,}\DecValTok{1}\NormalTok{]}
\NormalTok{standartnovirze}\OtherTok{=}\NormalTok{terra}\SpecialCharTok{::}\FunctionTok{global}\NormalTok{(centrets,}\AttributeTok{fun=}\StringTok{"rms"}\NormalTok{,}\AttributeTok{na.rm=}\ConstantTok{TRUE}\NormalTok{)}
\NormalTok{merogots}\OtherTok{=}\NormalTok{centrets}\SpecialCharTok{/}\NormalTok{standartnovirze[,}\DecValTok{1}\NormalTok{]}
\FunctionTok{writeRaster}\NormalTok{(merogots,}
      \AttributeTok{filename=}\NormalTok{saglabasanas\_cels,}
      \AttributeTok{overwrite=}\ConstantTok{TRUE}\NormalTok{)}
\end{Highlighting}
\end{Shaded}

\section{FarmlandGrassland\_GrasslandsPermanent\_r10000}\label{ch06.229}

\textbf{filename:} \texttt{FarmlandGrassland\_GrasslandsPermanent\_r10000.tif}

\textbf{layername:} \texttt{egv\_229}

\textbf{English name:} Fractional cover of Permanent Grassland within the 10 km
landscape

\textbf{Latvian name:} Ilggadīgu zālāju platības īpatsvars 10 km ainavā

\textbf{Procedure:} The cover fraction within a radius of 10000 m around the analysis grid cell
is calculated as the area-weighted sum of the \hyperref[ch06.225]{analysis cells} inside
the buffer, using the workflow \texttt{egvtools::radius\_function()}. During the calculation of the landscape
metric, inverse distance weighted (power = 2) gap filling on the output is
applied to ensure no missing values at the edges. Then the layer is
rewritten to set its name. Finally, the layer is standardised by
subtracting the arithmetic mean and dividing by the root mean squared error.

\begin{Shaded}
\begin{Highlighting}[]
\CommentTok{\# libs {-}{-}{-}{-}}
\ControlFlowTok{if}\NormalTok{(}\SpecialCharTok{!}\FunctionTok{require}\NormalTok{(terra)) \{}\FunctionTok{install.packages}\NormalTok{(}\StringTok{"terra"}\NormalTok{); }\FunctionTok{require}\NormalTok{(terra)\}}
\ControlFlowTok{if}\NormalTok{(}\SpecialCharTok{!}\FunctionTok{require}\NormalTok{(egvtools)) \{remotes}\SpecialCharTok{::}\FunctionTok{install\_github}\NormalTok{(}\StringTok{"aavotins/egvtools"}\NormalTok{); }\FunctionTok{require}\NormalTok{(egvtools)\}}


\CommentTok{\# Templates {-}{-}{-}{-}{-}}
\NormalTok{template100}\OtherTok{=}\FunctionTok{rast}\NormalTok{(}\StringTok{"./Templates/TemplateRasters/LV100m\_10km.tif"}\NormalTok{)}

\CommentTok{\# radii {-}{-}{-}{-}}
\FunctionTok{radius\_function}\NormalTok{(}
 \AttributeTok{kvadrati\_path =} \StringTok{"./Templates/TemplateGrids/tiles/"}\NormalTok{,}
 \AttributeTok{radii\_path   =} \StringTok{"./Templates/TemplateGridPoints/tiles/"}\NormalTok{,}
 \AttributeTok{tikls100\_path =} \StringTok{"./Templates/TemplateGrids/tikls100\_sauzeme.parquet"}\NormalTok{,}
 \AttributeTok{template\_path =} \StringTok{"./Templates/TemplateRasters/LV100m\_10km.tif"}\NormalTok{,}
 \AttributeTok{input\_layers  =} \FunctionTok{c}\NormalTok{(}\StringTok{"./RasterGrids\_100m/2024/RAW/FarmlandGrassland\_GrasslandsPermanent\_cell.tif"}\NormalTok{),}
 \AttributeTok{layer\_prefixes =} \FunctionTok{c}\NormalTok{(}\StringTok{"FarmlandGrassland\_GrasslandsPermanent"}\NormalTok{),}
 \AttributeTok{output\_dir   =} \StringTok{"./RasterGrids\_100m/2024/RAW/"}\NormalTok{,}
 \AttributeTok{n\_workers   =} \DecValTok{6}\NormalTok{,}
 \AttributeTok{radii     =} \FunctionTok{c}\NormalTok{(}\StringTok{"r10000"}\NormalTok{),}
 \AttributeTok{radius\_mode  =} \StringTok{"sparse"}\NormalTok{,}
 \AttributeTok{extract\_fun  =} \StringTok{"mean"}\NormalTok{,}
 \AttributeTok{fill\_missing  =} \ConstantTok{TRUE}\NormalTok{,}
 \AttributeTok{IDW\_weight   =} \DecValTok{2}\NormalTok{,}
 \AttributeTok{future\_max\_size =} \DecValTok{40} \SpecialCharTok{*} \DecValTok{1024}\SpecialCharTok{\^{}}\DecValTok{3}\NormalTok{)}


\CommentTok{\# FarmlandGrassland\_GrasslandsPermanent\_r10000.tif  egv\_229}
\NormalTok{slanis}\OtherTok{=}\FunctionTok{rast}\NormalTok{(}\StringTok{"./RasterGrids\_100m/2024/RAW/FarmlandGrassland\_GrasslandsPermanent\_r10000.tif"}\NormalTok{)}
\FunctionTok{names}\NormalTok{(slanis)}\OtherTok{=}\StringTok{"egv\_229"}
\NormalTok{slanis2}\OtherTok{=}\FunctionTok{project}\NormalTok{(slanis,template100)}
\FunctionTok{writeRaster}\NormalTok{(slanis2,}
      \StringTok{"./RasterGrids\_100m/2024/RAW/FarmlandGrassland\_GrasslandsPermanent\_r10000.tif"}\NormalTok{,}
      \AttributeTok{overwrite=}\ConstantTok{TRUE}\NormalTok{)}

\CommentTok{\# standardisation {-}{-}{-}{-}}
\ControlFlowTok{if}\NormalTok{(}\SpecialCharTok{!}\FunctionTok{require}\NormalTok{(terra)) \{}\FunctionTok{install.packages}\NormalTok{(}\StringTok{"terra"}\NormalTok{); }\FunctionTok{require}\NormalTok{(terra)\}}
\ControlFlowTok{if}\NormalTok{(}\SpecialCharTok{!}\FunctionTok{require}\NormalTok{(tidyverse)) \{}\FunctionTok{install.packages}\NormalTok{(}\StringTok{"tidyverse"}\NormalTok{); }\FunctionTok{require}\NormalTok{(tidyverse)\}}

\NormalTok{nosaukums}\OtherTok{=}\StringTok{"FarmlandGrassland\_GrasslandsPermanent\_r10000.tif"}
\NormalTok{ielasisanas\_cels}\OtherTok{=}\FunctionTok{paste0}\NormalTok{(}\StringTok{"./RasterGrids\_100m/2024/RAW/"}\NormalTok{,nosaukums)}
\NormalTok{saglabasanas\_cels}\OtherTok{=}\FunctionTok{paste0}\NormalTok{(}\StringTok{"./RasterGrids\_100m/2024/Scaled/"}\NormalTok{,nosaukums)}
\NormalTok{slanis}\OtherTok{=}\FunctionTok{rast}\NormalTok{(ielasisanas\_cels)}
\NormalTok{videjais}\OtherTok{=}\FunctionTok{global}\NormalTok{(slanis,}\AttributeTok{fun=}\StringTok{"mean"}\NormalTok{,}\AttributeTok{na.rm=}\ConstantTok{TRUE}\NormalTok{)}
\NormalTok{centrets}\OtherTok{=}\NormalTok{slanis}\SpecialCharTok{{-}}\NormalTok{videjais[,}\DecValTok{1}\NormalTok{]}
\NormalTok{standartnovirze}\OtherTok{=}\NormalTok{terra}\SpecialCharTok{::}\FunctionTok{global}\NormalTok{(centrets,}\AttributeTok{fun=}\StringTok{"rms"}\NormalTok{,}\AttributeTok{na.rm=}\ConstantTok{TRUE}\NormalTok{)}
\NormalTok{merogots}\OtherTok{=}\NormalTok{centrets}\SpecialCharTok{/}\NormalTok{standartnovirze[,}\DecValTok{1}\NormalTok{]}
\FunctionTok{writeRaster}\NormalTok{(merogots,}
      \AttributeTok{filename=}\NormalTok{saglabasanas\_cels,}
      \AttributeTok{overwrite=}\ConstantTok{TRUE}\NormalTok{)}
\end{Highlighting}
\end{Shaded}

\section{FarmlandGrassland\_GrasslandsTemporary\_cell}\label{ch06.230}

\textbf{filename:} \texttt{FarmlandGrassland\_GrasslandsTemporary\_cell.tif}

\textbf{layername:} \texttt{egv\_230}

\textbf{English name:} Fractional cover of Temporary Grassland within the analysis
cell (1 ha)

\textbf{Latvian name:} Zālāju-aramzemē platības īpatsvars analīzes šūnā (1 ha)

\textbf{Procedure:} First, agricultural parcels declared as grasslands in arable
lands are selected from the \hyperref[Ch04.02]{Rural Support Service's information on declared
fields}. These geometries are then rasterised to input resolution,
ensuring value 1 at the polygon locations and value 0 elsewhere. Rasterisation
is performed with the workflow \texttt{egvtools::polygon2input()}. Once rasterised, the layer is
aggregated to EGV resolution using the workflow \texttt{egvtools::input2egv()}, which
calculates the arithmetic mean and thus results in a cover fraction. During aggregation, inverse
distance weighted (power = 2) gap filling on the output is applied to
ensure no missing values at the edges. Finally, the layer is standardised
by subtracting the arithmetic mean and dividing by the root mean squared error.

\begin{Shaded}
\begin{Highlighting}[]
\CommentTok{\# libs {-}{-}{-}{-}}
\ControlFlowTok{if}\NormalTok{(}\SpecialCharTok{!}\FunctionTok{require}\NormalTok{(egvtools)) \{remotes}\SpecialCharTok{::}\FunctionTok{install\_github}\NormalTok{(}\StringTok{"aavotins/egvtools"}\NormalTok{); }\FunctionTok{require}\NormalTok{(egvtools)\}}
\ControlFlowTok{if}\NormalTok{(}\SpecialCharTok{!}\FunctionTok{require}\NormalTok{(terra)) \{}\FunctionTok{install.packages}\NormalTok{(}\StringTok{"terra"}\NormalTok{); }\FunctionTok{require}\NormalTok{(terra)\}}
\ControlFlowTok{if}\NormalTok{(}\SpecialCharTok{!}\FunctionTok{require}\NormalTok{(sf)) \{}\FunctionTok{install.packages}\NormalTok{(}\StringTok{"sf"}\NormalTok{); }\FunctionTok{require}\NormalTok{(sf)\}}
\ControlFlowTok{if}\NormalTok{(}\SpecialCharTok{!}\FunctionTok{require}\NormalTok{(tidyverse)) \{}\FunctionTok{install.packages}\NormalTok{(}\StringTok{"tidyverse"}\NormalTok{); }\FunctionTok{require}\NormalTok{(tidyverse)\}}
\ControlFlowTok{if}\NormalTok{(}\SpecialCharTok{!}\FunctionTok{require}\NormalTok{(sfarrow)) \{}\FunctionTok{install.packages}\NormalTok{(}\StringTok{"sfarrow"}\NormalTok{); }\FunctionTok{require}\NormalTok{(sfarrow)\}}
\ControlFlowTok{if}\NormalTok{(}\SpecialCharTok{!}\FunctionTok{require}\NormalTok{(readxl)) \{}\FunctionTok{install.packages}\NormalTok{(}\StringTok{"readxl"}\NormalTok{); }\FunctionTok{require}\NormalTok{(readxl)\}}
\ControlFlowTok{if}\NormalTok{(}\SpecialCharTok{!}\FunctionTok{require}\NormalTok{(raster)) \{}\FunctionTok{install.packages}\NormalTok{(}\StringTok{"raster"}\NormalTok{); }\FunctionTok{require}\NormalTok{(raster)\}}
\ControlFlowTok{if}\NormalTok{(}\SpecialCharTok{!}\FunctionTok{require}\NormalTok{(fasterize)) \{}\FunctionTok{install.packages}\NormalTok{(}\StringTok{"fasterize"}\NormalTok{); }\FunctionTok{require}\NormalTok{(fasterize)\}}

\CommentTok{\# templates {-}{-}{-}{-}}
\NormalTok{template100}\OtherTok{=}\FunctionTok{rast}\NormalTok{(}\StringTok{"./Templates/TemplateRasters/LV100m\_10km.tif"}\NormalTok{)}
\NormalTok{template10}\OtherTok{=}\FunctionTok{rast}\NormalTok{(}\StringTok{"./Templates/TemplateRasters/LV10m\_10km.tif"}\NormalTok{)}
\NormalTok{rastrs10}\OtherTok{=}\FunctionTok{raster}\NormalTok{(template10)}

\NormalTok{nulls10}\OtherTok{=}\FunctionTok{rast}\NormalTok{(}\StringTok{"./Templates/TemplateRasters/nulls\_LV10m\_10km.tif"}\NormalTok{)}
\NormalTok{nulls100}\OtherTok{=}\FunctionTok{rast}\NormalTok{(}\StringTok{"./Templates/TemplateRasters/nulls\_LV100m\_10km.tif"}\NormalTok{)}

\CommentTok{\# codes {-}{-}{-}{-}}
\NormalTok{kodi}\OtherTok{=}\FunctionTok{read\_excel}\NormalTok{(}\StringTok{"./Geodata/2024/LAD/KulturuKodi\_2024.xlsx"}\NormalTok{)}
\NormalTok{kodi}\SpecialCharTok{$}\NormalTok{kods}\OtherTok{=}\FunctionTok{as.character}\NormalTok{(kodi}\SpecialCharTok{$}\NormalTok{kods)}
\CommentTok{\# LAD {-}{-}{-}{-}}
\NormalTok{lad}\OtherTok{=}\NormalTok{sfarrow}\SpecialCharTok{::}\FunctionTok{st\_read\_parquet}\NormalTok{(}\StringTok{"./Geodata/2024/LAD/Lauki\_2024.parquet"}\NormalTok{)}
\NormalTok{lad}\SpecialCharTok{$}\NormalTok{yes}\OtherTok{=}\DecValTok{1}
\NormalTok{lad}\OtherTok{=}\NormalTok{lad }\SpecialCharTok{\%\textgreater{}\%} 
 \FunctionTok{left\_join}\NormalTok{(kodi,}\AttributeTok{by=}\FunctionTok{c}\NormalTok{(}\StringTok{"PRODUCT\_CODE"}\OtherTok{=}\StringTok{"kods"}\NormalTok{))}

\CommentTok{\# simple landscape {-}{-}{-}{-}}
\NormalTok{simple\_landscape}\OtherTok{=}\FunctionTok{rast}\NormalTok{(}\StringTok{"RasterGrids\_10m/2024/Ainava\_vienk\_mask.tif"}\NormalTok{)}


\CommentTok{\# FarmlandGrassland\_GrasslandsTemporary\_cell.tif    egv\_230 {-}{-}{-}{-}}
\NormalTok{dati}\OtherTok{=}\NormalTok{lad }\SpecialCharTok{\%\textgreater{}\%} 
 \FunctionTok{filter}\NormalTok{(}\FunctionTok{str\_detect}\NormalTok{(SDM\_grupa\_sakums,}\StringTok{"kultivēt"}\NormalTok{))}
\FunctionTok{table}\NormalTok{(dati}\SpecialCharTok{$}\NormalTok{SDM\_grupa\_sakums,}\AttributeTok{useNA=}\StringTok{"always"}\NormalTok{)}

\NormalTok{p2i\_rez}\OtherTok{=}\NormalTok{egvtools}\SpecialCharTok{::}\FunctionTok{polygon2input}\NormalTok{(}\AttributeTok{vector\_data =}\NormalTok{ dati,}
                \AttributeTok{template\_path =} \StringTok{"./Templates/TemplateRasters/LV10m\_10km.tif"}\NormalTok{,}
                \AttributeTok{out\_path =} \StringTok{"./RasterGrids\_10m/2024/"}\NormalTok{,}
                \AttributeTok{file\_name =} \StringTok{"FarmlandGrassland\_GrasslandsTemporary\_input.tif"}\NormalTok{,}
                \AttributeTok{value\_field =} \StringTok{"yes"}\NormalTok{,}
                \AttributeTok{prepare=}\ConstantTok{FALSE}\NormalTok{,}
                \AttributeTok{background\_raster =} \StringTok{"./Templates/TemplateRasters/nulls\_LV10m\_10km.tif"}\NormalTok{,}
                \AttributeTok{plot\_result =} \ConstantTok{TRUE}\NormalTok{)}
\NormalTok{p2i\_rez}
\NormalTok{i2e\_rez}\OtherTok{=}\NormalTok{egvtools}\SpecialCharTok{::}\FunctionTok{input2egv}\NormalTok{(}\AttributeTok{input=}\FunctionTok{paste0}\NormalTok{(}\StringTok{"./RasterGrids\_10m/2024/"}\NormalTok{,}
                     \StringTok{"FarmlandGrassland\_GrasslandsTemporary\_input.tif"}\NormalTok{),}
              \AttributeTok{egv\_template=} \StringTok{"./Templates/TemplateRasters/LV100m\_10km.tif"}\NormalTok{,}
              \AttributeTok{summary\_function =} \StringTok{"average"}\NormalTok{,}
              \AttributeTok{missing\_job =} \StringTok{"FillOutput"}\NormalTok{,}
              \AttributeTok{outlocation =} \StringTok{"./RasterGrids\_100m/2024/RAW/"}\NormalTok{,}
              \AttributeTok{outfilename =} \StringTok{"FarmlandGrassland\_GrasslandsTemporary\_cell.tif"}\NormalTok{,}
              \AttributeTok{layername =} \StringTok{"egv\_230"}\NormalTok{,}
              \AttributeTok{idw\_weight =} \DecValTok{2}\NormalTok{,}
              \AttributeTok{plot\_gaps =} \ConstantTok{FALSE}\NormalTok{,}\AttributeTok{plot\_final =} \ConstantTok{TRUE}\NormalTok{)}
\NormalTok{i2e\_rez}
\FunctionTok{rm}\NormalTok{(p2i\_rez)}
\FunctionTok{rm}\NormalTok{(i2e\_rez)}
\FunctionTok{rm}\NormalTok{(dati)}
\FunctionTok{unlink}\NormalTok{(}\StringTok{"./RasterGrids\_10m/2024/FarmlandGrassland\_GrasslandsTemporary\_input.tif"}\NormalTok{)}


\CommentTok{\# standardisation {-}{-}{-}{-}}
\ControlFlowTok{if}\NormalTok{(}\SpecialCharTok{!}\FunctionTok{require}\NormalTok{(terra)) \{}\FunctionTok{install.packages}\NormalTok{(}\StringTok{"terra"}\NormalTok{); }\FunctionTok{require}\NormalTok{(terra)\}}
\ControlFlowTok{if}\NormalTok{(}\SpecialCharTok{!}\FunctionTok{require}\NormalTok{(tidyverse)) \{}\FunctionTok{install.packages}\NormalTok{(}\StringTok{"tidyverse"}\NormalTok{); }\FunctionTok{require}\NormalTok{(tidyverse)\}}

\NormalTok{nosaukums}\OtherTok{=}\StringTok{"FarmlandGrassland\_GrasslandsTemporary\_cell.tif"}
\NormalTok{ielasisanas\_cels}\OtherTok{=}\FunctionTok{paste0}\NormalTok{(}\StringTok{"./RasterGrids\_100m/2024/RAW/"}\NormalTok{,nosaukums)}
\NormalTok{saglabasanas\_cels}\OtherTok{=}\FunctionTok{paste0}\NormalTok{(}\StringTok{"./RasterGrids\_100m/2024/Scaled/"}\NormalTok{,nosaukums)}
\NormalTok{slanis}\OtherTok{=}\FunctionTok{rast}\NormalTok{(ielasisanas\_cels)}
\NormalTok{videjais}\OtherTok{=}\FunctionTok{global}\NormalTok{(slanis,}\AttributeTok{fun=}\StringTok{"mean"}\NormalTok{,}\AttributeTok{na.rm=}\ConstantTok{TRUE}\NormalTok{)}
\NormalTok{centrets}\OtherTok{=}\NormalTok{slanis}\SpecialCharTok{{-}}\NormalTok{videjais[,}\DecValTok{1}\NormalTok{]}
\NormalTok{standartnovirze}\OtherTok{=}\NormalTok{terra}\SpecialCharTok{::}\FunctionTok{global}\NormalTok{(centrets,}\AttributeTok{fun=}\StringTok{"rms"}\NormalTok{,}\AttributeTok{na.rm=}\ConstantTok{TRUE}\NormalTok{)}
\NormalTok{merogots}\OtherTok{=}\NormalTok{centrets}\SpecialCharTok{/}\NormalTok{standartnovirze[,}\DecValTok{1}\NormalTok{]}
\FunctionTok{writeRaster}\NormalTok{(merogots,}
      \AttributeTok{filename=}\NormalTok{saglabasanas\_cels,}
      \AttributeTok{overwrite=}\ConstantTok{TRUE}\NormalTok{)}
\end{Highlighting}
\end{Shaded}

\section{FarmlandGrassland\_GrasslandsTemporary\_r500}\label{ch06.231}

\textbf{filename:} \texttt{FarmlandGrassland\_GrasslandsTemporary\_r500.tif}

\textbf{layername:} \texttt{egv\_231}

\textbf{English name:} Fractional cover of Temporary Grassland within the 0.5 km
landscape

\textbf{Latvian name:} Zālāju-aramzemē platības īpatsvars 0,5 km ainavā

\textbf{Procedure:} The cover fraction within a radius of 500 m around the analysis grid cell is
calculated as the area-weighted sum of the \hyperref[ch06.230]{analysis cells} inside the
buffer, using the workflow \texttt{egvtools::radius\_function()}. During the calculation of the landscape metric,
inverse distance weighted (power = 2) gap filling on the output is applied
to ensure no missing values at the edges. Then the layer is rewritten to set
its name. Finally, the layer is standardised by subtracting the arithmetic
mean and dividing by the root mean squared error.

\begin{Shaded}
\begin{Highlighting}[]
\CommentTok{\# libs {-}{-}{-}{-}}
\ControlFlowTok{if}\NormalTok{(}\SpecialCharTok{!}\FunctionTok{require}\NormalTok{(terra)) \{}\FunctionTok{install.packages}\NormalTok{(}\StringTok{"terra"}\NormalTok{); }\FunctionTok{require}\NormalTok{(terra)\}}
\ControlFlowTok{if}\NormalTok{(}\SpecialCharTok{!}\FunctionTok{require}\NormalTok{(egvtools)) \{remotes}\SpecialCharTok{::}\FunctionTok{install\_github}\NormalTok{(}\StringTok{"aavotins/egvtools"}\NormalTok{); }\FunctionTok{require}\NormalTok{(egvtools)\}}


\CommentTok{\# Templates {-}{-}{-}{-}{-}}
\NormalTok{template100}\OtherTok{=}\FunctionTok{rast}\NormalTok{(}\StringTok{"./Templates/TemplateRasters/LV100m\_10km.tif"}\NormalTok{)}

\CommentTok{\# radii {-}{-}{-}{-}}
\FunctionTok{radius\_function}\NormalTok{(}
 \AttributeTok{kvadrati\_path =} \StringTok{"./Templates/TemplateGrids/tiles/"}\NormalTok{,}
 \AttributeTok{radii\_path   =} \StringTok{"./Templates/TemplateGridPoints/tiles/"}\NormalTok{,}
 \AttributeTok{tikls100\_path =} \StringTok{"./Templates/TemplateGrids/tikls100\_sauzeme.parquet"}\NormalTok{,}
 \AttributeTok{template\_path =} \StringTok{"./Templates/TemplateRasters/LV100m\_10km.tif"}\NormalTok{,}
 \AttributeTok{input\_layers  =} \FunctionTok{c}\NormalTok{(}\StringTok{"./RasterGrids\_100m/2024/RAW/FarmlandGrassland\_GrasslandsTemporary\_cell.tif"}\NormalTok{),}
 \AttributeTok{layer\_prefixes =} \FunctionTok{c}\NormalTok{(}\StringTok{"FarmlandGrassland\_GrasslandsTemporary"}\NormalTok{),}
 \AttributeTok{output\_dir   =} \StringTok{"./RasterGrids\_100m/2024/RAW/"}\NormalTok{,}
 \AttributeTok{n\_workers   =} \DecValTok{6}\NormalTok{,}
 \AttributeTok{radii     =} \FunctionTok{c}\NormalTok{(}\StringTok{"r500"}\NormalTok{),}
 \AttributeTok{radius\_mode  =} \StringTok{"sparse"}\NormalTok{,}
 \AttributeTok{extract\_fun  =} \StringTok{"mean"}\NormalTok{,}
 \AttributeTok{fill\_missing  =} \ConstantTok{TRUE}\NormalTok{,}
 \AttributeTok{IDW\_weight   =} \DecValTok{2}\NormalTok{,}
 \AttributeTok{future\_max\_size =} \DecValTok{40} \SpecialCharTok{*} \DecValTok{1024}\SpecialCharTok{\^{}}\DecValTok{3}\NormalTok{)}


\CommentTok{\# FarmlandGrassland\_GrasslandsTemporary\_r500.tif    egv\_231}
\NormalTok{slanis}\OtherTok{=}\FunctionTok{rast}\NormalTok{(}\StringTok{"./RasterGrids\_100m/2024/RAW/FarmlandGrassland\_GrasslandsTemporary\_r500.tif"}\NormalTok{)}
\FunctionTok{names}\NormalTok{(slanis)}\OtherTok{=}\StringTok{"egv\_231"}
\NormalTok{slanis2}\OtherTok{=}\FunctionTok{project}\NormalTok{(slanis,template100)}
\FunctionTok{writeRaster}\NormalTok{(slanis2,}
      \StringTok{"./RasterGrids\_100m/2024/RAW/FarmlandGrassland\_GrasslandsTemporary\_r500.tif"}\NormalTok{,}
      \AttributeTok{overwrite=}\ConstantTok{TRUE}\NormalTok{)}

\CommentTok{\# standardisation {-}{-}{-}{-}}
\ControlFlowTok{if}\NormalTok{(}\SpecialCharTok{!}\FunctionTok{require}\NormalTok{(terra)) \{}\FunctionTok{install.packages}\NormalTok{(}\StringTok{"terra"}\NormalTok{); }\FunctionTok{require}\NormalTok{(terra)\}}
\ControlFlowTok{if}\NormalTok{(}\SpecialCharTok{!}\FunctionTok{require}\NormalTok{(tidyverse)) \{}\FunctionTok{install.packages}\NormalTok{(}\StringTok{"tidyverse"}\NormalTok{); }\FunctionTok{require}\NormalTok{(tidyverse)\}}

\NormalTok{nosaukums}\OtherTok{=}\StringTok{"FarmlandGrassland\_GrasslandsTemporary\_r500.tif"}
\NormalTok{ielasisanas\_cels}\OtherTok{=}\FunctionTok{paste0}\NormalTok{(}\StringTok{"./RasterGrids\_100m/2024/RAW/"}\NormalTok{,nosaukums)}
\NormalTok{saglabasanas\_cels}\OtherTok{=}\FunctionTok{paste0}\NormalTok{(}\StringTok{"./RasterGrids\_100m/2024/Scaled/"}\NormalTok{,nosaukums)}
\NormalTok{slanis}\OtherTok{=}\FunctionTok{rast}\NormalTok{(ielasisanas\_cels)}
\NormalTok{videjais}\OtherTok{=}\FunctionTok{global}\NormalTok{(slanis,}\AttributeTok{fun=}\StringTok{"mean"}\NormalTok{,}\AttributeTok{na.rm=}\ConstantTok{TRUE}\NormalTok{)}
\NormalTok{centrets}\OtherTok{=}\NormalTok{slanis}\SpecialCharTok{{-}}\NormalTok{videjais[,}\DecValTok{1}\NormalTok{]}
\NormalTok{standartnovirze}\OtherTok{=}\NormalTok{terra}\SpecialCharTok{::}\FunctionTok{global}\NormalTok{(centrets,}\AttributeTok{fun=}\StringTok{"rms"}\NormalTok{,}\AttributeTok{na.rm=}\ConstantTok{TRUE}\NormalTok{)}
\NormalTok{merogots}\OtherTok{=}\NormalTok{centrets}\SpecialCharTok{/}\NormalTok{standartnovirze[,}\DecValTok{1}\NormalTok{]}
\FunctionTok{writeRaster}\NormalTok{(merogots,}
      \AttributeTok{filename=}\NormalTok{saglabasanas\_cels,}
      \AttributeTok{overwrite=}\ConstantTok{TRUE}\NormalTok{)}
\end{Highlighting}
\end{Shaded}

\section{FarmlandGrassland\_GrasslandsTemporary\_r1250}\label{ch06.232}

\textbf{filename:} \texttt{FarmlandGrassland\_GrasslandsTemporary\_r1250.tif}

\textbf{layername:} \texttt{egv\_232}

\textbf{English name:} Fractional cover of Temporary Grassland within the 1.25 km
landscape

\textbf{Latvian name:} Zālāju-aramzemē platības īpatsvars 1,25 km ainavā

\textbf{Procedure:} The cover fraction within a radius of 1250 m around the analysis grid cell
is calculated as the area-weighted sum of the \hyperref[ch06.230]{analysis cells} inside
the buffer, using the workflow \texttt{egvtools::radius\_function()}. During the calculation of the landscape
metric, inverse distance weighted (power = 2) gap filling on the output is
applied to ensure no missing values at the edges. Then the layer is
rewritten to set its name. Finally, the layer is standardised by
subtracting the arithmetic mean and dividing by the root mean squared error.

\begin{Shaded}
\begin{Highlighting}[]
\CommentTok{\# libs {-}{-}{-}{-}}
\ControlFlowTok{if}\NormalTok{(}\SpecialCharTok{!}\FunctionTok{require}\NormalTok{(terra)) \{}\FunctionTok{install.packages}\NormalTok{(}\StringTok{"terra"}\NormalTok{); }\FunctionTok{require}\NormalTok{(terra)\}}
\ControlFlowTok{if}\NormalTok{(}\SpecialCharTok{!}\FunctionTok{require}\NormalTok{(egvtools)) \{remotes}\SpecialCharTok{::}\FunctionTok{install\_github}\NormalTok{(}\StringTok{"aavotins/egvtools"}\NormalTok{); }\FunctionTok{require}\NormalTok{(egvtools)\}}


\CommentTok{\# Templates {-}{-}{-}{-}{-}}
\NormalTok{template100}\OtherTok{=}\FunctionTok{rast}\NormalTok{(}\StringTok{"./Templates/TemplateRasters/LV100m\_10km.tif"}\NormalTok{)}

\CommentTok{\# radii {-}{-}{-}{-}}
\FunctionTok{radius\_function}\NormalTok{(}
 \AttributeTok{kvadrati\_path =} \StringTok{"./Templates/TemplateGrids/tiles/"}\NormalTok{,}
 \AttributeTok{radii\_path   =} \StringTok{"./Templates/TemplateGridPoints/tiles/"}\NormalTok{,}
 \AttributeTok{tikls100\_path =} \StringTok{"./Templates/TemplateGrids/tikls100\_sauzeme.parquet"}\NormalTok{,}
 \AttributeTok{template\_path =} \StringTok{"./Templates/TemplateRasters/LV100m\_10km.tif"}\NormalTok{,}
 \AttributeTok{input\_layers  =} \FunctionTok{c}\NormalTok{(}\StringTok{"./RasterGrids\_100m/2024/RAW/FarmlandGrassland\_GrasslandsTemporary\_cell.tif"}\NormalTok{),}
 \AttributeTok{layer\_prefixes =} \FunctionTok{c}\NormalTok{(}\StringTok{"FarmlandGrassland\_GrasslandsTemporary"}\NormalTok{),}
 \AttributeTok{output\_dir   =} \StringTok{"./RasterGrids\_100m/2024/RAW/"}\NormalTok{,}
 \AttributeTok{n\_workers   =} \DecValTok{6}\NormalTok{,}
 \AttributeTok{radii     =} \FunctionTok{c}\NormalTok{(}\StringTok{"r1250"}\NormalTok{),}
 \AttributeTok{radius\_mode  =} \StringTok{"sparse"}\NormalTok{,}
 \AttributeTok{extract\_fun  =} \StringTok{"mean"}\NormalTok{,}
 \AttributeTok{fill\_missing  =} \ConstantTok{TRUE}\NormalTok{,}
 \AttributeTok{IDW\_weight   =} \DecValTok{2}\NormalTok{,}
 \AttributeTok{future\_max\_size =} \DecValTok{40} \SpecialCharTok{*} \DecValTok{1024}\SpecialCharTok{\^{}}\DecValTok{3}\NormalTok{)}


\CommentTok{\# FarmlandGrassland\_GrasslandsTemporary\_r1250.tif   egv\_232}
\NormalTok{slanis}\OtherTok{=}\FunctionTok{rast}\NormalTok{(}\StringTok{"./RasterGrids\_100m/2024/RAW/FarmlandGrassland\_GrasslandsTemporary\_r1250.tif"}\NormalTok{)}
\FunctionTok{names}\NormalTok{(slanis)}\OtherTok{=}\StringTok{"egv\_232"}
\NormalTok{slanis2}\OtherTok{=}\FunctionTok{project}\NormalTok{(slanis,template100)}
\FunctionTok{writeRaster}\NormalTok{(slanis2,}
      \StringTok{"./RasterGrids\_100m/2024/RAW/FarmlandGrassland\_GrasslandsTemporary\_r1250.tif"}\NormalTok{,}
      \AttributeTok{overwrite=}\ConstantTok{TRUE}\NormalTok{)}

\CommentTok{\# standardisation {-}{-}{-}{-}}
\ControlFlowTok{if}\NormalTok{(}\SpecialCharTok{!}\FunctionTok{require}\NormalTok{(terra)) \{}\FunctionTok{install.packages}\NormalTok{(}\StringTok{"terra"}\NormalTok{); }\FunctionTok{require}\NormalTok{(terra)\}}
\ControlFlowTok{if}\NormalTok{(}\SpecialCharTok{!}\FunctionTok{require}\NormalTok{(tidyverse)) \{}\FunctionTok{install.packages}\NormalTok{(}\StringTok{"tidyverse"}\NormalTok{); }\FunctionTok{require}\NormalTok{(tidyverse)\}}

\NormalTok{nosaukums}\OtherTok{=}\StringTok{"FarmlandGrassland\_GrasslandsTemporary\_r1250.tif"}
\NormalTok{ielasisanas\_cels}\OtherTok{=}\FunctionTok{paste0}\NormalTok{(}\StringTok{"./RasterGrids\_100m/2024/RAW/"}\NormalTok{,nosaukums)}
\NormalTok{saglabasanas\_cels}\OtherTok{=}\FunctionTok{paste0}\NormalTok{(}\StringTok{"./RasterGrids\_100m/2024/Scaled/"}\NormalTok{,nosaukums)}
\NormalTok{slanis}\OtherTok{=}\FunctionTok{rast}\NormalTok{(ielasisanas\_cels)}
\NormalTok{videjais}\OtherTok{=}\FunctionTok{global}\NormalTok{(slanis,}\AttributeTok{fun=}\StringTok{"mean"}\NormalTok{,}\AttributeTok{na.rm=}\ConstantTok{TRUE}\NormalTok{)}
\NormalTok{centrets}\OtherTok{=}\NormalTok{slanis}\SpecialCharTok{{-}}\NormalTok{videjais[,}\DecValTok{1}\NormalTok{]}
\NormalTok{standartnovirze}\OtherTok{=}\NormalTok{terra}\SpecialCharTok{::}\FunctionTok{global}\NormalTok{(centrets,}\AttributeTok{fun=}\StringTok{"rms"}\NormalTok{,}\AttributeTok{na.rm=}\ConstantTok{TRUE}\NormalTok{)}
\NormalTok{merogots}\OtherTok{=}\NormalTok{centrets}\SpecialCharTok{/}\NormalTok{standartnovirze[,}\DecValTok{1}\NormalTok{]}
\FunctionTok{writeRaster}\NormalTok{(merogots,}
      \AttributeTok{filename=}\NormalTok{saglabasanas\_cels,}
      \AttributeTok{overwrite=}\ConstantTok{TRUE}\NormalTok{)}
\end{Highlighting}
\end{Shaded}

\section{FarmlandGrassland\_GrasslandsTemporary\_r3000}\label{ch06.233}

\textbf{filename:} \texttt{FarmlandGrassland\_GrasslandsTemporary\_r3000.tif}

\textbf{layername:} \texttt{egv\_233}

\textbf{English name:} Fractional cover of Temporary Grassland within the 3 km
landscape

\textbf{Latvian name:} Zālāju-aramzemē platības īpatsvars 3 km ainavā

\textbf{Procedure:} The cover fraction within a radius of 3000 m around the analysis grid cell
is calculated as the area-weighted sum of the \hyperref[ch06.230]{analysis cells} inside
the buffer, using the workflow \texttt{egvtools::radius\_function()}. During the calculation of the landscape
metric, inverse distance weighted (power = 2) gap filling on the output is
applied to ensure no missing values at the edges. Then the layer is
rewritten to set its name. Finally, the layer is standardised by
subtracting the arithmetic mean and dividing by the root mean squared error.

\begin{Shaded}
\begin{Highlighting}[]
\CommentTok{\# libs {-}{-}{-}{-}}
\ControlFlowTok{if}\NormalTok{(}\SpecialCharTok{!}\FunctionTok{require}\NormalTok{(terra)) \{}\FunctionTok{install.packages}\NormalTok{(}\StringTok{"terra"}\NormalTok{); }\FunctionTok{require}\NormalTok{(terra)\}}
\ControlFlowTok{if}\NormalTok{(}\SpecialCharTok{!}\FunctionTok{require}\NormalTok{(egvtools)) \{remotes}\SpecialCharTok{::}\FunctionTok{install\_github}\NormalTok{(}\StringTok{"aavotins/egvtools"}\NormalTok{); }\FunctionTok{require}\NormalTok{(egvtools)\}}


\CommentTok{\# Templates {-}{-}{-}{-}{-}}
\NormalTok{template100}\OtherTok{=}\FunctionTok{rast}\NormalTok{(}\StringTok{"./Templates/TemplateRasters/LV100m\_10km.tif"}\NormalTok{)}

\CommentTok{\# radii {-}{-}{-}{-}}
\FunctionTok{radius\_function}\NormalTok{(}
 \AttributeTok{kvadrati\_path =} \StringTok{"./Templates/TemplateGrids/tiles/"}\NormalTok{,}
 \AttributeTok{radii\_path   =} \StringTok{"./Templates/TemplateGridPoints/tiles/"}\NormalTok{,}
 \AttributeTok{tikls100\_path =} \StringTok{"./Templates/TemplateGrids/tikls100\_sauzeme.parquet"}\NormalTok{,}
 \AttributeTok{template\_path =} \StringTok{"./Templates/TemplateRasters/LV100m\_10km.tif"}\NormalTok{,}
 \AttributeTok{input\_layers  =} \FunctionTok{c}\NormalTok{(}\StringTok{"./RasterGrids\_100m/2024/RAW/FarmlandGrassland\_GrasslandsTemporary\_cell.tif"}\NormalTok{),}
 \AttributeTok{layer\_prefixes =} \FunctionTok{c}\NormalTok{(}\StringTok{"FarmlandGrassland\_GrasslandsTemporary"}\NormalTok{),}
 \AttributeTok{output\_dir   =} \StringTok{"./RasterGrids\_100m/2024/RAW/"}\NormalTok{,}
 \AttributeTok{n\_workers   =} \DecValTok{6}\NormalTok{,}
 \AttributeTok{radii     =} \FunctionTok{c}\NormalTok{(}\StringTok{"r3000"}\NormalTok{),}
 \AttributeTok{radius\_mode  =} \StringTok{"sparse"}\NormalTok{,}
 \AttributeTok{extract\_fun  =} \StringTok{"mean"}\NormalTok{,}
 \AttributeTok{fill\_missing  =} \ConstantTok{TRUE}\NormalTok{,}
 \AttributeTok{IDW\_weight   =} \DecValTok{2}\NormalTok{,}
 \AttributeTok{future\_max\_size =} \DecValTok{40} \SpecialCharTok{*} \DecValTok{1024}\SpecialCharTok{\^{}}\DecValTok{3}\NormalTok{)}


\CommentTok{\# FarmlandGrassland\_GrasslandsTemporary\_r3000.tif   egv\_233}
\NormalTok{slanis}\OtherTok{=}\FunctionTok{rast}\NormalTok{(}\StringTok{"./RasterGrids\_100m/2024/RAW/FarmlandGrassland\_GrasslandsTemporary\_r3000.tif"}\NormalTok{)}
\FunctionTok{names}\NormalTok{(slanis)}\OtherTok{=}\StringTok{"egv\_233"}
\NormalTok{slanis2}\OtherTok{=}\FunctionTok{project}\NormalTok{(slanis,template100)}
\FunctionTok{writeRaster}\NormalTok{(slanis2,}
      \StringTok{"./RasterGrids\_100m/2024/RAW/FarmlandGrassland\_GrasslandsTemporary\_r3000.tif"}\NormalTok{,}
      \AttributeTok{overwrite=}\ConstantTok{TRUE}\NormalTok{)}

\CommentTok{\# standardisation {-}{-}{-}{-}}
\ControlFlowTok{if}\NormalTok{(}\SpecialCharTok{!}\FunctionTok{require}\NormalTok{(terra)) \{}\FunctionTok{install.packages}\NormalTok{(}\StringTok{"terra"}\NormalTok{); }\FunctionTok{require}\NormalTok{(terra)\}}
\ControlFlowTok{if}\NormalTok{(}\SpecialCharTok{!}\FunctionTok{require}\NormalTok{(tidyverse)) \{}\FunctionTok{install.packages}\NormalTok{(}\StringTok{"tidyverse"}\NormalTok{); }\FunctionTok{require}\NormalTok{(tidyverse)\}}

\NormalTok{nosaukums}\OtherTok{=}\StringTok{"FarmlandGrassland\_GrasslandsTemporary\_r3000.tif"}
\NormalTok{ielasisanas\_cels}\OtherTok{=}\FunctionTok{paste0}\NormalTok{(}\StringTok{"./RasterGrids\_100m/2024/RAW/"}\NormalTok{,nosaukums)}
\NormalTok{saglabasanas\_cels}\OtherTok{=}\FunctionTok{paste0}\NormalTok{(}\StringTok{"./RasterGrids\_100m/2024/Scaled/"}\NormalTok{,nosaukums)}
\NormalTok{slanis}\OtherTok{=}\FunctionTok{rast}\NormalTok{(ielasisanas\_cels)}
\NormalTok{videjais}\OtherTok{=}\FunctionTok{global}\NormalTok{(slanis,}\AttributeTok{fun=}\StringTok{"mean"}\NormalTok{,}\AttributeTok{na.rm=}\ConstantTok{TRUE}\NormalTok{)}
\NormalTok{centrets}\OtherTok{=}\NormalTok{slanis}\SpecialCharTok{{-}}\NormalTok{videjais[,}\DecValTok{1}\NormalTok{]}
\NormalTok{standartnovirze}\OtherTok{=}\NormalTok{terra}\SpecialCharTok{::}\FunctionTok{global}\NormalTok{(centrets,}\AttributeTok{fun=}\StringTok{"rms"}\NormalTok{,}\AttributeTok{na.rm=}\ConstantTok{TRUE}\NormalTok{)}
\NormalTok{merogots}\OtherTok{=}\NormalTok{centrets}\SpecialCharTok{/}\NormalTok{standartnovirze[,}\DecValTok{1}\NormalTok{]}
\FunctionTok{writeRaster}\NormalTok{(merogots,}
      \AttributeTok{filename=}\NormalTok{saglabasanas\_cels,}
      \AttributeTok{overwrite=}\ConstantTok{TRUE}\NormalTok{)}
\end{Highlighting}
\end{Shaded}

\section{FarmlandGrassland\_GrasslandsTemporary\_r10000}\label{ch06.234}

\textbf{filename:} \texttt{FarmlandGrassland\_GrasslandsTemporary\_r10000.tif}

\textbf{layername:} \texttt{egv\_234}

\textbf{English name:} Fractional cover of Temporary Grassland within the 10 km
landscape

\textbf{Latvian name:} Zālāju-aramzemē platības īpatsvars 10 km ainavā

\textbf{Procedure:} The cover fraction within a radius of 10000 m around the analysis grid cell
is calculated as the area-weighted sum of the \hyperref[ch06.230]{analysis cells} inside
the buffer, using the workflow \texttt{egvtools::radius\_function()}. During the calculation of the landscape
metric, inverse distance weighted (power = 2) gap filling on the output is
applied to ensure no missing values at the edges. Then the layer is
rewritten to set its name. Finally, the layer is standardised by
subtracting the arithmetic mean and dividing by the root mean squared error.

\begin{Shaded}
\begin{Highlighting}[]
\CommentTok{\# libs {-}{-}{-}{-}}
\ControlFlowTok{if}\NormalTok{(}\SpecialCharTok{!}\FunctionTok{require}\NormalTok{(terra)) \{}\FunctionTok{install.packages}\NormalTok{(}\StringTok{"terra"}\NormalTok{); }\FunctionTok{require}\NormalTok{(terra)\}}
\ControlFlowTok{if}\NormalTok{(}\SpecialCharTok{!}\FunctionTok{require}\NormalTok{(egvtools)) \{remotes}\SpecialCharTok{::}\FunctionTok{install\_github}\NormalTok{(}\StringTok{"aavotins/egvtools"}\NormalTok{); }\FunctionTok{require}\NormalTok{(egvtools)\}}


\CommentTok{\# Templates {-}{-}{-}{-}{-}}
\NormalTok{template100}\OtherTok{=}\FunctionTok{rast}\NormalTok{(}\StringTok{"./Templates/TemplateRasters/LV100m\_10km.tif"}\NormalTok{)}

\CommentTok{\# radii {-}{-}{-}{-}}
\FunctionTok{radius\_function}\NormalTok{(}
 \AttributeTok{kvadrati\_path =} \StringTok{"./Templates/TemplateGrids/tiles/"}\NormalTok{,}
 \AttributeTok{radii\_path   =} \StringTok{"./Templates/TemplateGridPoints/tiles/"}\NormalTok{,}
 \AttributeTok{tikls100\_path =} \StringTok{"./Templates/TemplateGrids/tikls100\_sauzeme.parquet"}\NormalTok{,}
 \AttributeTok{template\_path =} \StringTok{"./Templates/TemplateRasters/LV100m\_10km.tif"}\NormalTok{,}
 \AttributeTok{input\_layers  =} \FunctionTok{c}\NormalTok{(}\StringTok{"./RasterGrids\_100m/2024/RAW/FarmlandGrassland\_GrasslandsTemporary\_cell.tif"}\NormalTok{),}
 \AttributeTok{layer\_prefixes =} \FunctionTok{c}\NormalTok{(}\StringTok{"FarmlandGrassland\_GrasslandsTemporary"}\NormalTok{),}
 \AttributeTok{output\_dir   =} \StringTok{"./RasterGrids\_100m/2024/RAW/"}\NormalTok{,}
 \AttributeTok{n\_workers   =} \DecValTok{6}\NormalTok{,}
 \AttributeTok{radii     =} \FunctionTok{c}\NormalTok{(}\StringTok{"r10000"}\NormalTok{),}
 \AttributeTok{radius\_mode  =} \StringTok{"sparse"}\NormalTok{,}
 \AttributeTok{extract\_fun  =} \StringTok{"mean"}\NormalTok{,}
 \AttributeTok{fill\_missing  =} \ConstantTok{TRUE}\NormalTok{,}
 \AttributeTok{IDW\_weight   =} \DecValTok{2}\NormalTok{,}
 \AttributeTok{future\_max\_size =} \DecValTok{40} \SpecialCharTok{*} \DecValTok{1024}\SpecialCharTok{\^{}}\DecValTok{3}\NormalTok{)}


\CommentTok{\# FarmlandGrassland\_GrasslandsTemporary\_r10000.tif  egv\_234}
\NormalTok{slanis}\OtherTok{=}\FunctionTok{rast}\NormalTok{(}\StringTok{"./RasterGrids\_100m/2024/RAW/FarmlandGrassland\_GrasslandsTemporary\_r10000.tif"}\NormalTok{)}
\FunctionTok{names}\NormalTok{(slanis)}\OtherTok{=}\StringTok{"egv\_234"}
\NormalTok{slanis2}\OtherTok{=}\FunctionTok{project}\NormalTok{(slanis,template100)}
\FunctionTok{writeRaster}\NormalTok{(slanis2,}
      \StringTok{"./RasterGrids\_100m/2024/RAW/FarmlandGrassland\_GrasslandsTemporary\_r10000.tif"}\NormalTok{,}
      \AttributeTok{overwrite=}\ConstantTok{TRUE}\NormalTok{)}

\CommentTok{\# standardisation {-}{-}{-}{-}}
\ControlFlowTok{if}\NormalTok{(}\SpecialCharTok{!}\FunctionTok{require}\NormalTok{(terra)) \{}\FunctionTok{install.packages}\NormalTok{(}\StringTok{"terra"}\NormalTok{); }\FunctionTok{require}\NormalTok{(terra)\}}
\ControlFlowTok{if}\NormalTok{(}\SpecialCharTok{!}\FunctionTok{require}\NormalTok{(tidyverse)) \{}\FunctionTok{install.packages}\NormalTok{(}\StringTok{"tidyverse"}\NormalTok{); }\FunctionTok{require}\NormalTok{(tidyverse)\}}

\NormalTok{nosaukums}\OtherTok{=}\StringTok{"FarmlandGrassland\_GrasslandsTemporary\_r10000.tif"}
\NormalTok{ielasisanas\_cels}\OtherTok{=}\FunctionTok{paste0}\NormalTok{(}\StringTok{"./RasterGrids\_100m/2024/RAW/"}\NormalTok{,nosaukums)}
\NormalTok{saglabasanas\_cels}\OtherTok{=}\FunctionTok{paste0}\NormalTok{(}\StringTok{"./RasterGrids\_100m/2024/Scaled/"}\NormalTok{,nosaukums)}
\NormalTok{slanis}\OtherTok{=}\FunctionTok{rast}\NormalTok{(ielasisanas\_cels)}
\NormalTok{videjais}\OtherTok{=}\FunctionTok{global}\NormalTok{(slanis,}\AttributeTok{fun=}\StringTok{"mean"}\NormalTok{,}\AttributeTok{na.rm=}\ConstantTok{TRUE}\NormalTok{)}
\NormalTok{centrets}\OtherTok{=}\NormalTok{slanis}\SpecialCharTok{{-}}\NormalTok{videjais[,}\DecValTok{1}\NormalTok{]}
\NormalTok{standartnovirze}\OtherTok{=}\NormalTok{terra}\SpecialCharTok{::}\FunctionTok{global}\NormalTok{(centrets,}\AttributeTok{fun=}\StringTok{"rms"}\NormalTok{,}\AttributeTok{na.rm=}\ConstantTok{TRUE}\NormalTok{)}
\NormalTok{merogots}\OtherTok{=}\NormalTok{centrets}\SpecialCharTok{/}\NormalTok{standartnovirze[,}\DecValTok{1}\NormalTok{]}
\FunctionTok{writeRaster}\NormalTok{(merogots,}
      \AttributeTok{filename=}\NormalTok{saglabasanas\_cels,}
      \AttributeTok{overwrite=}\ConstantTok{TRUE}\NormalTok{)}
\end{Highlighting}
\end{Shaded}

\section{FarmlandParcels\_FieldsActive\_cell}\label{ch06.235}

\textbf{filename:} \texttt{FarmlandParcels\_FieldsActive\_cell.tif}

\textbf{layername:} \texttt{egv\_235}

\textbf{English name:} Fractional cover of Agricultural Land Parcels within the
analysis cell (1 ha)

\textbf{Latvian name:} Lauku bloku platības īpatsvars analīzes šūnā (1 ha)

\textbf{Procedure:} First, agricultural parcels from the \hyperref[Ch04.02]{Rural Support Service's
information on declared fields} are rasterised to input resolution,
ensuring value 1 at the polygon locations and value 0 elsewhere. Rasterisation
is performed with the workflow \texttt{egvtools::polygon2input()}. Once rasterised, the layer is
aggregated to EGV resolution using the workflow \texttt{egvtools::input2egv()}, which
calculates the arithmetic mean and thus results in a cover fraction. During aggregation, inverse
distance weighted (power = 2) gap filling on the output is applied to
ensure no missing values at the edges. Finally, the layer is standardised
by subtracting the arithmetic mean and dividing by the root mean squared error.

\begin{Shaded}
\begin{Highlighting}[]
\CommentTok{\# libs {-}{-}{-}{-}}
\ControlFlowTok{if}\NormalTok{(}\SpecialCharTok{!}\FunctionTok{require}\NormalTok{(egvtools)) \{remotes}\SpecialCharTok{::}\FunctionTok{install\_github}\NormalTok{(}\StringTok{"aavotins/egvtools"}\NormalTok{); }\FunctionTok{require}\NormalTok{(egvtools)\}}
\ControlFlowTok{if}\NormalTok{(}\SpecialCharTok{!}\FunctionTok{require}\NormalTok{(terra)) \{}\FunctionTok{install.packages}\NormalTok{(}\StringTok{"terra"}\NormalTok{); }\FunctionTok{require}\NormalTok{(terra)\}}
\ControlFlowTok{if}\NormalTok{(}\SpecialCharTok{!}\FunctionTok{require}\NormalTok{(sf)) \{}\FunctionTok{install.packages}\NormalTok{(}\StringTok{"sf"}\NormalTok{); }\FunctionTok{require}\NormalTok{(sf)\}}
\ControlFlowTok{if}\NormalTok{(}\SpecialCharTok{!}\FunctionTok{require}\NormalTok{(tidyverse)) \{}\FunctionTok{install.packages}\NormalTok{(}\StringTok{"tidyverse"}\NormalTok{); }\FunctionTok{require}\NormalTok{(tidyverse)\}}
\ControlFlowTok{if}\NormalTok{(}\SpecialCharTok{!}\FunctionTok{require}\NormalTok{(sfarrow)) \{}\FunctionTok{install.packages}\NormalTok{(}\StringTok{"sfarrow"}\NormalTok{); }\FunctionTok{require}\NormalTok{(sfarrow)\}}
\ControlFlowTok{if}\NormalTok{(}\SpecialCharTok{!}\FunctionTok{require}\NormalTok{(readxl)) \{}\FunctionTok{install.packages}\NormalTok{(}\StringTok{"readxl"}\NormalTok{); }\FunctionTok{require}\NormalTok{(readxl)\}}
\ControlFlowTok{if}\NormalTok{(}\SpecialCharTok{!}\FunctionTok{require}\NormalTok{(raster)) \{}\FunctionTok{install.packages}\NormalTok{(}\StringTok{"raster"}\NormalTok{); }\FunctionTok{require}\NormalTok{(raster)\}}
\ControlFlowTok{if}\NormalTok{(}\SpecialCharTok{!}\FunctionTok{require}\NormalTok{(fasterize)) \{}\FunctionTok{install.packages}\NormalTok{(}\StringTok{"fasterize"}\NormalTok{); }\FunctionTok{require}\NormalTok{(fasterize)\}}

\CommentTok{\# templates {-}{-}{-}{-}}
\NormalTok{template100}\OtherTok{=}\FunctionTok{rast}\NormalTok{(}\StringTok{"./Templates/TemplateRasters/LV100m\_10km.tif"}\NormalTok{)}
\NormalTok{template10}\OtherTok{=}\FunctionTok{rast}\NormalTok{(}\StringTok{"./Templates/TemplateRasters/LV10m\_10km.tif"}\NormalTok{)}
\NormalTok{rastrs10}\OtherTok{=}\FunctionTok{raster}\NormalTok{(template10)}

\NormalTok{nulls10}\OtherTok{=}\FunctionTok{rast}\NormalTok{(}\StringTok{"./Templates/TemplateRasters/nulls\_LV10m\_10km.tif"}\NormalTok{)}
\NormalTok{nulls100}\OtherTok{=}\FunctionTok{rast}\NormalTok{(}\StringTok{"./Templates/TemplateRasters/nulls\_LV100m\_10km.tif"}\NormalTok{)}

\CommentTok{\# codes {-}{-}{-}{-}}
\NormalTok{kodi}\OtherTok{=}\FunctionTok{read\_excel}\NormalTok{(}\StringTok{"./Geodata/2024/LAD/KulturuKodi\_2024.xlsx"}\NormalTok{)}
\NormalTok{kodi}\SpecialCharTok{$}\NormalTok{kods}\OtherTok{=}\FunctionTok{as.character}\NormalTok{(kodi}\SpecialCharTok{$}\NormalTok{kods)}
\CommentTok{\# LAD {-}{-}{-}{-}}
\NormalTok{lad}\OtherTok{=}\NormalTok{sfarrow}\SpecialCharTok{::}\FunctionTok{st\_read\_parquet}\NormalTok{(}\StringTok{"./Geodata/2024/LAD/Lauki\_2024.parquet"}\NormalTok{)}
\NormalTok{lad}\SpecialCharTok{$}\NormalTok{yes}\OtherTok{=}\DecValTok{1}
\NormalTok{lad}\OtherTok{=}\NormalTok{lad }\SpecialCharTok{\%\textgreater{}\%} 
 \FunctionTok{left\_join}\NormalTok{(kodi,}\AttributeTok{by=}\FunctionTok{c}\NormalTok{(}\StringTok{"PRODUCT\_CODE"}\OtherTok{=}\StringTok{"kods"}\NormalTok{))}

\CommentTok{\# simple landscape {-}{-}{-}{-}}
\NormalTok{simple\_landscape}\OtherTok{=}\FunctionTok{rast}\NormalTok{(}\StringTok{"RasterGrids\_10m/2024/Ainava\_vienk\_mask.tif"}\NormalTok{)}


\CommentTok{\# FarmlandParcels\_FieldsActive\_cell.tif egv\_235 {-}{-}{-}{-}}
\NormalTok{p2i\_rez}\OtherTok{=}\NormalTok{egvtools}\SpecialCharTok{::}\FunctionTok{polygon2input}\NormalTok{(}\AttributeTok{vector\_data =}\NormalTok{ lad,}
                \AttributeTok{template\_path =} \StringTok{"./Templates/TemplateRasters/LV10m\_10km.tif"}\NormalTok{,}
                \AttributeTok{out\_path =} \StringTok{"./RasterGrids\_10m/2024/"}\NormalTok{,}
                \AttributeTok{file\_name =} \StringTok{"FarmlandParcels\_FieldsActive\_input.tif"}\NormalTok{,}
                \AttributeTok{value\_field =} \StringTok{"yes"}\NormalTok{,}
                \AttributeTok{prepare=}\ConstantTok{FALSE}\NormalTok{,}
                \AttributeTok{background\_raster =} \StringTok{"./Templates/TemplateRasters/nulls\_LV10m\_10km.tif"}\NormalTok{,}
                \AttributeTok{plot\_result =} \ConstantTok{TRUE}\NormalTok{)}
\NormalTok{p2i\_rez}
\NormalTok{i2e\_rez}\OtherTok{=}\NormalTok{egvtools}\SpecialCharTok{::}\FunctionTok{input2egv}\NormalTok{(}\AttributeTok{input=}\FunctionTok{paste0}\NormalTok{(}\StringTok{"./RasterGrids\_10m/2024/"}\NormalTok{,}
                     \StringTok{"FarmlandParcels\_FieldsActive\_input.tif"}\NormalTok{),}
              \AttributeTok{egv\_template=} \StringTok{"./Templates/TemplateRasters/LV100m\_10km.tif"}\NormalTok{,}
              \AttributeTok{summary\_function =} \StringTok{"average"}\NormalTok{,}
              \AttributeTok{missing\_job =} \StringTok{"FillOutput"}\NormalTok{,}
              \AttributeTok{outlocation =} \StringTok{"./RasterGrids\_100m/2024/RAW/"}\NormalTok{,}
              \AttributeTok{outfilename =} \StringTok{"FarmlandParcels\_FieldsActive\_cell.tif"}\NormalTok{,}
              \AttributeTok{layername =} \StringTok{"egv\_235"}\NormalTok{,}
              \AttributeTok{idw\_weight =} \DecValTok{2}\NormalTok{,}
              \AttributeTok{plot\_gaps =} \ConstantTok{FALSE}\NormalTok{,}\AttributeTok{plot\_final =} \ConstantTok{TRUE}\NormalTok{)}
\NormalTok{i2e\_rez}
\FunctionTok{rm}\NormalTok{(p2i\_rez)}
\FunctionTok{rm}\NormalTok{(i2e\_rez)}
\FunctionTok{rm}\NormalTok{(dati)}
\FunctionTok{unlink}\NormalTok{(}\StringTok{"./RasterGrids\_10m/2024/FarmlandParcels\_FieldsActive\_input.tif"}\NormalTok{)}


\CommentTok{\# standardisation {-}{-}{-}{-}}
\ControlFlowTok{if}\NormalTok{(}\SpecialCharTok{!}\FunctionTok{require}\NormalTok{(terra)) \{}\FunctionTok{install.packages}\NormalTok{(}\StringTok{"terra"}\NormalTok{); }\FunctionTok{require}\NormalTok{(terra)\}}
\ControlFlowTok{if}\NormalTok{(}\SpecialCharTok{!}\FunctionTok{require}\NormalTok{(tidyverse)) \{}\FunctionTok{install.packages}\NormalTok{(}\StringTok{"tidyverse"}\NormalTok{); }\FunctionTok{require}\NormalTok{(tidyverse)\}}

\NormalTok{nosaukums}\OtherTok{=}\StringTok{"FarmlandParcels\_FieldsActive\_cell.tif"}
\NormalTok{ielasisanas\_cels}\OtherTok{=}\FunctionTok{paste0}\NormalTok{(}\StringTok{"./RasterGrids\_100m/2024/RAW/"}\NormalTok{,nosaukums)}
\NormalTok{saglabasanas\_cels}\OtherTok{=}\FunctionTok{paste0}\NormalTok{(}\StringTok{"./RasterGrids\_100m/2024/Scaled/"}\NormalTok{,nosaukums)}
\NormalTok{slanis}\OtherTok{=}\FunctionTok{rast}\NormalTok{(ielasisanas\_cels)}
\NormalTok{videjais}\OtherTok{=}\FunctionTok{global}\NormalTok{(slanis,}\AttributeTok{fun=}\StringTok{"mean"}\NormalTok{,}\AttributeTok{na.rm=}\ConstantTok{TRUE}\NormalTok{)}
\NormalTok{centrets}\OtherTok{=}\NormalTok{slanis}\SpecialCharTok{{-}}\NormalTok{videjais[,}\DecValTok{1}\NormalTok{]}
\NormalTok{standartnovirze}\OtherTok{=}\NormalTok{terra}\SpecialCharTok{::}\FunctionTok{global}\NormalTok{(centrets,}\AttributeTok{fun=}\StringTok{"rms"}\NormalTok{,}\AttributeTok{na.rm=}\ConstantTok{TRUE}\NormalTok{)}
\NormalTok{merogots}\OtherTok{=}\NormalTok{centrets}\SpecialCharTok{/}\NormalTok{standartnovirze[,}\DecValTok{1}\NormalTok{]}
\FunctionTok{writeRaster}\NormalTok{(merogots,}
      \AttributeTok{filename=}\NormalTok{saglabasanas\_cels,}
      \AttributeTok{overwrite=}\ConstantTok{TRUE}\NormalTok{)}
\end{Highlighting}
\end{Shaded}

\section{FarmlandParcels\_FieldsActive\_r500}\label{ch06.236}

\textbf{filename:} \texttt{FarmlandParcels\_FieldsActive\_r500.tif}

\textbf{layername:} \texttt{egv\_236}

\textbf{English name:} Fractional cover of Agricultural Land Parcels within the 0.5
km landscape

\textbf{Latvian name:} Lauku bloku platības īpatsvars 0,5 km ainavā

\textbf{Procedure:} The cover fraction within a radius of 500 m around the analysis grid cell is
calculated as the area-weighted sum of the \hyperref[ch06.235]{analysis cells} inside the
buffer, using the workflow \texttt{egvtools::radius\_function()}. During the calculation of the landscape metric,
inverse distance weighted (power = 2) gap filling on the output is applied
to ensure no missing values at the edges. Then the layer is rewritten to set
its name. Finally, the layer is standardised by subtracting the arithmetic
mean and dividing by the root mean squared error.

\begin{Shaded}
\begin{Highlighting}[]
\CommentTok{\# libs {-}{-}{-}{-}}
\ControlFlowTok{if}\NormalTok{(}\SpecialCharTok{!}\FunctionTok{require}\NormalTok{(terra)) \{}\FunctionTok{install.packages}\NormalTok{(}\StringTok{"terra"}\NormalTok{); }\FunctionTok{require}\NormalTok{(terra)\}}
\ControlFlowTok{if}\NormalTok{(}\SpecialCharTok{!}\FunctionTok{require}\NormalTok{(egvtools)) \{remotes}\SpecialCharTok{::}\FunctionTok{install\_github}\NormalTok{(}\StringTok{"aavotins/egvtools"}\NormalTok{); }\FunctionTok{require}\NormalTok{(egvtools)\}}


\CommentTok{\# Templates {-}{-}{-}{-}{-}}
\NormalTok{template100}\OtherTok{=}\FunctionTok{rast}\NormalTok{(}\StringTok{"./Templates/TemplateRasters/LV100m\_10km.tif"}\NormalTok{)}

\CommentTok{\# radii {-}{-}{-}{-}}
\FunctionTok{radius\_function}\NormalTok{(}
 \AttributeTok{kvadrati\_path =} \StringTok{"./Templates/TemplateGrids/tiles/"}\NormalTok{,}
 \AttributeTok{radii\_path   =} \StringTok{"./Templates/TemplateGridPoints/tiles/"}\NormalTok{,}
 \AttributeTok{tikls100\_path =} \StringTok{"./Templates/TemplateGrids/tikls100\_sauzeme.parquet"}\NormalTok{,}
 \AttributeTok{template\_path =} \StringTok{"./Templates/TemplateRasters/LV100m\_10km.tif"}\NormalTok{,}
 \AttributeTok{input\_layers  =} \FunctionTok{c}\NormalTok{(}\StringTok{"./RasterGrids\_100m/2024/RAW/FarmlandParcels\_FieldsActive\_cell.tif"}\NormalTok{),}
 \AttributeTok{layer\_prefixes =} \FunctionTok{c}\NormalTok{(}\StringTok{"FarmlandParcels\_FieldsActive"}\NormalTok{),}
 \AttributeTok{output\_dir   =} \StringTok{"./RasterGrids\_100m/2024/RAW/"}\NormalTok{,}
 \AttributeTok{n\_workers   =} \DecValTok{6}\NormalTok{,}
 \AttributeTok{radii     =} \FunctionTok{c}\NormalTok{(}\StringTok{"r500"}\NormalTok{),}
 \AttributeTok{radius\_mode  =} \StringTok{"sparse"}\NormalTok{,}
 \AttributeTok{extract\_fun  =} \StringTok{"mean"}\NormalTok{,}
 \AttributeTok{fill\_missing  =} \ConstantTok{TRUE}\NormalTok{,}
 \AttributeTok{IDW\_weight   =} \DecValTok{2}\NormalTok{,}
 \AttributeTok{future\_max\_size =} \DecValTok{40} \SpecialCharTok{*} \DecValTok{1024}\SpecialCharTok{\^{}}\DecValTok{3}\NormalTok{)}


\CommentTok{\# FarmlandParcels\_FieldsActive\_r500.tif egv\_236}
\NormalTok{slanis}\OtherTok{=}\FunctionTok{rast}\NormalTok{(}\StringTok{"./RasterGrids\_100m/2024/RAW/FarmlandParcels\_FieldsActive\_r500.tif"}\NormalTok{)}
\FunctionTok{names}\NormalTok{(slanis)}\OtherTok{=}\StringTok{"egv\_236"}
\NormalTok{slanis2}\OtherTok{=}\FunctionTok{project}\NormalTok{(slanis,template100)}
\FunctionTok{writeRaster}\NormalTok{(slanis2,}
      \StringTok{"./RasterGrids\_100m/2024/RAW/FarmlandParcels\_FieldsActive\_r500.tif"}\NormalTok{,}
      \AttributeTok{overwrite=}\ConstantTok{TRUE}\NormalTok{)}

\CommentTok{\# standardisation {-}{-}{-}{-}}
\ControlFlowTok{if}\NormalTok{(}\SpecialCharTok{!}\FunctionTok{require}\NormalTok{(terra)) \{}\FunctionTok{install.packages}\NormalTok{(}\StringTok{"terra"}\NormalTok{); }\FunctionTok{require}\NormalTok{(terra)\}}
\ControlFlowTok{if}\NormalTok{(}\SpecialCharTok{!}\FunctionTok{require}\NormalTok{(tidyverse)) \{}\FunctionTok{install.packages}\NormalTok{(}\StringTok{"tidyverse"}\NormalTok{); }\FunctionTok{require}\NormalTok{(tidyverse)\}}

\NormalTok{nosaukums}\OtherTok{=}\StringTok{"FarmlandParcels\_FieldsActive\_r500.tif"}
\NormalTok{ielasisanas\_cels}\OtherTok{=}\FunctionTok{paste0}\NormalTok{(}\StringTok{"./RasterGrids\_100m/2024/RAW/"}\NormalTok{,nosaukums)}
\NormalTok{saglabasanas\_cels}\OtherTok{=}\FunctionTok{paste0}\NormalTok{(}\StringTok{"./RasterGrids\_100m/2024/Scaled/"}\NormalTok{,nosaukums)}
\NormalTok{slanis}\OtherTok{=}\FunctionTok{rast}\NormalTok{(ielasisanas\_cels)}
\NormalTok{videjais}\OtherTok{=}\FunctionTok{global}\NormalTok{(slanis,}\AttributeTok{fun=}\StringTok{"mean"}\NormalTok{,}\AttributeTok{na.rm=}\ConstantTok{TRUE}\NormalTok{)}
\NormalTok{centrets}\OtherTok{=}\NormalTok{slanis}\SpecialCharTok{{-}}\NormalTok{videjais[,}\DecValTok{1}\NormalTok{]}
\NormalTok{standartnovirze}\OtherTok{=}\NormalTok{terra}\SpecialCharTok{::}\FunctionTok{global}\NormalTok{(centrets,}\AttributeTok{fun=}\StringTok{"rms"}\NormalTok{,}\AttributeTok{na.rm=}\ConstantTok{TRUE}\NormalTok{)}
\NormalTok{merogots}\OtherTok{=}\NormalTok{centrets}\SpecialCharTok{/}\NormalTok{standartnovirze[,}\DecValTok{1}\NormalTok{]}
\FunctionTok{writeRaster}\NormalTok{(merogots,}
      \AttributeTok{filename=}\NormalTok{saglabasanas\_cels,}
      \AttributeTok{overwrite=}\ConstantTok{TRUE}\NormalTok{)}
\end{Highlighting}
\end{Shaded}

\section{FarmlandParcels\_FieldsActive\_r1250}\label{ch06.237}

\textbf{filename:} \texttt{FarmlandParcels\_FieldsActive\_r1250.tif}

\textbf{layername:} \texttt{egv\_237}

\textbf{English name:} Fractional cover of Agricultural Land Parcels within the 1.25
km landscape

\textbf{Latvian name:} Lauku bloku platības īpatsvars 1,25 km ainavā

\textbf{Procedure:} The cover fraction within a radius of 1250 m around the analysis grid cell
is calculated as the area-weighted sum of the \hyperref[ch06.235]{analysis cells} inside
the buffer, using the workflow \texttt{egvtools::radius\_function()}. During the calculation of the landscape
metric, inverse distance weighted (power = 2) gap filling on the output is
applied to ensure no missing values at the edges. Then the layer is
rewritten to set its name. Finally, the layer is standardised by
subtracting the arithmetic mean and dividing by the root mean squared error.

\begin{Shaded}
\begin{Highlighting}[]
\CommentTok{\# libs {-}{-}{-}{-}}
\ControlFlowTok{if}\NormalTok{(}\SpecialCharTok{!}\FunctionTok{require}\NormalTok{(terra)) \{}\FunctionTok{install.packages}\NormalTok{(}\StringTok{"terra"}\NormalTok{); }\FunctionTok{require}\NormalTok{(terra)\}}
\ControlFlowTok{if}\NormalTok{(}\SpecialCharTok{!}\FunctionTok{require}\NormalTok{(egvtools)) \{remotes}\SpecialCharTok{::}\FunctionTok{install\_github}\NormalTok{(}\StringTok{"aavotins/egvtools"}\NormalTok{); }\FunctionTok{require}\NormalTok{(egvtools)\}}


\CommentTok{\# Templates {-}{-}{-}{-}{-}}
\NormalTok{template100}\OtherTok{=}\FunctionTok{rast}\NormalTok{(}\StringTok{"./Templates/TemplateRasters/LV100m\_10km.tif"}\NormalTok{)}

\CommentTok{\# radii {-}{-}{-}{-}}
\FunctionTok{radius\_function}\NormalTok{(}
 \AttributeTok{kvadrati\_path =} \StringTok{"./Templates/TemplateGrids/tiles/"}\NormalTok{,}
 \AttributeTok{radii\_path   =} \StringTok{"./Templates/TemplateGridPoints/tiles/"}\NormalTok{,}
 \AttributeTok{tikls100\_path =} \StringTok{"./Templates/TemplateGrids/tikls100\_sauzeme.parquet"}\NormalTok{,}
 \AttributeTok{template\_path =} \StringTok{"./Templates/TemplateRasters/LV100m\_10km.tif"}\NormalTok{,}
 \AttributeTok{input\_layers  =} \FunctionTok{c}\NormalTok{(}\StringTok{"./RasterGrids\_100m/2024/RAW/FarmlandParcels\_FieldsActive\_cell.tif"}\NormalTok{),}
 \AttributeTok{layer\_prefixes =} \FunctionTok{c}\NormalTok{(}\StringTok{"FarmlandParcels\_FieldsActive"}\NormalTok{),}
 \AttributeTok{output\_dir   =} \StringTok{"./RasterGrids\_100m/2024/RAW/"}\NormalTok{,}
 \AttributeTok{n\_workers   =} \DecValTok{6}\NormalTok{,}
 \AttributeTok{radii     =} \FunctionTok{c}\NormalTok{(}\StringTok{"r1250"}\NormalTok{),}
 \AttributeTok{radius\_mode  =} \StringTok{"sparse"}\NormalTok{,}
 \AttributeTok{extract\_fun  =} \StringTok{"mean"}\NormalTok{,}
 \AttributeTok{fill\_missing  =} \ConstantTok{TRUE}\NormalTok{,}
 \AttributeTok{IDW\_weight   =} \DecValTok{2}\NormalTok{,}
 \AttributeTok{future\_max\_size =} \DecValTok{40} \SpecialCharTok{*} \DecValTok{1024}\SpecialCharTok{\^{}}\DecValTok{3}\NormalTok{)}


\CommentTok{\# FarmlandParcels\_FieldsActive\_r1250.tif    egv\_237}
\NormalTok{slanis}\OtherTok{=}\FunctionTok{rast}\NormalTok{(}\StringTok{"./RasterGrids\_100m/2024/RAW/FarmlandParcels\_FieldsActive\_r1250.tif"}\NormalTok{)}
\FunctionTok{names}\NormalTok{(slanis)}\OtherTok{=}\StringTok{"egv\_237"}
\NormalTok{slanis2}\OtherTok{=}\FunctionTok{project}\NormalTok{(slanis,template100)}
\FunctionTok{writeRaster}\NormalTok{(slanis2,}
      \StringTok{"./RasterGrids\_100m/2024/RAW/FarmlandParcels\_FieldsActive\_r1250.tif"}\NormalTok{,}
      \AttributeTok{overwrite=}\ConstantTok{TRUE}\NormalTok{)}

\CommentTok{\# standardisation {-}{-}{-}{-}}
\ControlFlowTok{if}\NormalTok{(}\SpecialCharTok{!}\FunctionTok{require}\NormalTok{(terra)) \{}\FunctionTok{install.packages}\NormalTok{(}\StringTok{"terra"}\NormalTok{); }\FunctionTok{require}\NormalTok{(terra)\}}
\ControlFlowTok{if}\NormalTok{(}\SpecialCharTok{!}\FunctionTok{require}\NormalTok{(tidyverse)) \{}\FunctionTok{install.packages}\NormalTok{(}\StringTok{"tidyverse"}\NormalTok{); }\FunctionTok{require}\NormalTok{(tidyverse)\}}

\NormalTok{nosaukums}\OtherTok{=}\StringTok{"FarmlandParcels\_FieldsActive\_r1250.tif"}
\NormalTok{ielasisanas\_cels}\OtherTok{=}\FunctionTok{paste0}\NormalTok{(}\StringTok{"./RasterGrids\_100m/2024/RAW/"}\NormalTok{,nosaukums)}
\NormalTok{saglabasanas\_cels}\OtherTok{=}\FunctionTok{paste0}\NormalTok{(}\StringTok{"./RasterGrids\_100m/2024/Scaled/"}\NormalTok{,nosaukums)}
\NormalTok{slanis}\OtherTok{=}\FunctionTok{rast}\NormalTok{(ielasisanas\_cels)}
\NormalTok{videjais}\OtherTok{=}\FunctionTok{global}\NormalTok{(slanis,}\AttributeTok{fun=}\StringTok{"mean"}\NormalTok{,}\AttributeTok{na.rm=}\ConstantTok{TRUE}\NormalTok{)}
\NormalTok{centrets}\OtherTok{=}\NormalTok{slanis}\SpecialCharTok{{-}}\NormalTok{videjais[,}\DecValTok{1}\NormalTok{]}
\NormalTok{standartnovirze}\OtherTok{=}\NormalTok{terra}\SpecialCharTok{::}\FunctionTok{global}\NormalTok{(centrets,}\AttributeTok{fun=}\StringTok{"rms"}\NormalTok{,}\AttributeTok{na.rm=}\ConstantTok{TRUE}\NormalTok{)}
\NormalTok{merogots}\OtherTok{=}\NormalTok{centrets}\SpecialCharTok{/}\NormalTok{standartnovirze[,}\DecValTok{1}\NormalTok{]}
\FunctionTok{writeRaster}\NormalTok{(merogots,}
      \AttributeTok{filename=}\NormalTok{saglabasanas\_cels,}
      \AttributeTok{overwrite=}\ConstantTok{TRUE}\NormalTok{)}
\end{Highlighting}
\end{Shaded}

\section{FarmlandParcels\_FieldsActive\_r3000}\label{ch06.238}

\textbf{filename:} \texttt{FarmlandParcels\_FieldsActive\_r3000.tif}

\textbf{layername:} \texttt{egv\_238}

\textbf{English name:} Fractional cover of Agricultural Land Parcels within the 3 km
landscape

\textbf{Latvian name:} Lauku bloku platības īpatsvars 3 km ainavā

\textbf{Procedure:} The cover fraction within a radius of 3000 m around the analysis grid cell
is calculated as the area-weighted sum of the \hyperref[ch06.235]{analysis cells} inside
the buffer, using the workflow \texttt{egvtools::radius\_function()}. During the calculation of the landscape
metric, inverse distance weighted (power = 2) gap filling on the output is
applied to ensure no missing values at the edges. Then the layer is
rewritten to set its name. Finally, the layer is standardised by
subtracting the arithmetic mean and dividing by the root mean squared error.

\begin{Shaded}
\begin{Highlighting}[]
\CommentTok{\# libs {-}{-}{-}{-}}
\ControlFlowTok{if}\NormalTok{(}\SpecialCharTok{!}\FunctionTok{require}\NormalTok{(terra)) \{}\FunctionTok{install.packages}\NormalTok{(}\StringTok{"terra"}\NormalTok{); }\FunctionTok{require}\NormalTok{(terra)\}}
\ControlFlowTok{if}\NormalTok{(}\SpecialCharTok{!}\FunctionTok{require}\NormalTok{(egvtools)) \{remotes}\SpecialCharTok{::}\FunctionTok{install\_github}\NormalTok{(}\StringTok{"aavotins/egvtools"}\NormalTok{); }\FunctionTok{require}\NormalTok{(egvtools)\}}


\CommentTok{\# Templates {-}{-}{-}{-}{-}}
\NormalTok{template100}\OtherTok{=}\FunctionTok{rast}\NormalTok{(}\StringTok{"./Templates/TemplateRasters/LV100m\_10km.tif"}\NormalTok{)}

\CommentTok{\# radii {-}{-}{-}{-}}
\FunctionTok{radius\_function}\NormalTok{(}
 \AttributeTok{kvadrati\_path =} \StringTok{"./Templates/TemplateGrids/tiles/"}\NormalTok{,}
 \AttributeTok{radii\_path   =} \StringTok{"./Templates/TemplateGridPoints/tiles/"}\NormalTok{,}
 \AttributeTok{tikls100\_path =} \StringTok{"./Templates/TemplateGrids/tikls100\_sauzeme.parquet"}\NormalTok{,}
 \AttributeTok{template\_path =} \StringTok{"./Templates/TemplateRasters/LV100m\_10km.tif"}\NormalTok{,}
 \AttributeTok{input\_layers  =} \FunctionTok{c}\NormalTok{(}\StringTok{"./RasterGrids\_100m/2024/RAW/FarmlandParcels\_FieldsActive\_cell.tif"}\NormalTok{),}
 \AttributeTok{layer\_prefixes =} \FunctionTok{c}\NormalTok{(}\StringTok{"FarmlandParcels\_FieldsActive"}\NormalTok{),}
 \AttributeTok{output\_dir   =} \StringTok{"./RasterGrids\_100m/2024/RAW/"}\NormalTok{,}
 \AttributeTok{n\_workers   =} \DecValTok{6}\NormalTok{,}
 \AttributeTok{radii     =} \FunctionTok{c}\NormalTok{(}\StringTok{"r3000"}\NormalTok{),}
 \AttributeTok{radius\_mode  =} \StringTok{"sparse"}\NormalTok{,}
 \AttributeTok{extract\_fun  =} \StringTok{"mean"}\NormalTok{,}
 \AttributeTok{fill\_missing  =} \ConstantTok{TRUE}\NormalTok{,}
 \AttributeTok{IDW\_weight   =} \DecValTok{2}\NormalTok{,}
 \AttributeTok{future\_max\_size =} \DecValTok{40} \SpecialCharTok{*} \DecValTok{1024}\SpecialCharTok{\^{}}\DecValTok{3}\NormalTok{)}


\CommentTok{\# FarmlandParcels\_FieldsActive\_r3000.tif    egv\_238}
\NormalTok{slanis}\OtherTok{=}\FunctionTok{rast}\NormalTok{(}\StringTok{"./RasterGrids\_100m/2024/RAW/FarmlandParcels\_FieldsActive\_r3000.tif"}\NormalTok{)}
\FunctionTok{names}\NormalTok{(slanis)}\OtherTok{=}\StringTok{"egv\_238"}
\NormalTok{slanis2}\OtherTok{=}\FunctionTok{project}\NormalTok{(slanis,template100)}
\FunctionTok{writeRaster}\NormalTok{(slanis2,}
      \StringTok{"./RasterGrids\_100m/2024/RAW/FarmlandParcels\_FieldsActive\_r3000.tif"}\NormalTok{,}
      \AttributeTok{overwrite=}\ConstantTok{TRUE}\NormalTok{)}

\CommentTok{\# standardisation {-}{-}{-}{-}}
\ControlFlowTok{if}\NormalTok{(}\SpecialCharTok{!}\FunctionTok{require}\NormalTok{(terra)) \{}\FunctionTok{install.packages}\NormalTok{(}\StringTok{"terra"}\NormalTok{); }\FunctionTok{require}\NormalTok{(terra)\}}
\ControlFlowTok{if}\NormalTok{(}\SpecialCharTok{!}\FunctionTok{require}\NormalTok{(tidyverse)) \{}\FunctionTok{install.packages}\NormalTok{(}\StringTok{"tidyverse"}\NormalTok{); }\FunctionTok{require}\NormalTok{(tidyverse)\}}

\NormalTok{nosaukums}\OtherTok{=}\StringTok{"FarmlandParcels\_FieldsActive\_r3000.tif"}
\NormalTok{ielasisanas\_cels}\OtherTok{=}\FunctionTok{paste0}\NormalTok{(}\StringTok{"./RasterGrids\_100m/2024/RAW/"}\NormalTok{,nosaukums)}
\NormalTok{saglabasanas\_cels}\OtherTok{=}\FunctionTok{paste0}\NormalTok{(}\StringTok{"./RasterGrids\_100m/2024/Scaled/"}\NormalTok{,nosaukums)}
\NormalTok{slanis}\OtherTok{=}\FunctionTok{rast}\NormalTok{(ielasisanas\_cels)}
\NormalTok{videjais}\OtherTok{=}\FunctionTok{global}\NormalTok{(slanis,}\AttributeTok{fun=}\StringTok{"mean"}\NormalTok{,}\AttributeTok{na.rm=}\ConstantTok{TRUE}\NormalTok{)}
\NormalTok{centrets}\OtherTok{=}\NormalTok{slanis}\SpecialCharTok{{-}}\NormalTok{videjais[,}\DecValTok{1}\NormalTok{]}
\NormalTok{standartnovirze}\OtherTok{=}\NormalTok{terra}\SpecialCharTok{::}\FunctionTok{global}\NormalTok{(centrets,}\AttributeTok{fun=}\StringTok{"rms"}\NormalTok{,}\AttributeTok{na.rm=}\ConstantTok{TRUE}\NormalTok{)}
\NormalTok{merogots}\OtherTok{=}\NormalTok{centrets}\SpecialCharTok{/}\NormalTok{standartnovirze[,}\DecValTok{1}\NormalTok{]}
\FunctionTok{writeRaster}\NormalTok{(merogots,}
      \AttributeTok{filename=}\NormalTok{saglabasanas\_cels,}
      \AttributeTok{overwrite=}\ConstantTok{TRUE}\NormalTok{)}
\end{Highlighting}
\end{Shaded}

\section{FarmlandParcels\_FieldsActive\_r10000}\label{ch06.239}

\textbf{filename:} \texttt{FarmlandParcels\_FieldsActive\_r10000.tif}

\textbf{layername:} \texttt{egv\_239}

\textbf{English name:} Fractional cover of Agricultural Land Parcels within the 10 km
landscape

\textbf{Latvian name:} Lauku bloku platības īpatsvars 10 km ainavā

\textbf{Procedure:} The cover fraction within a radius of 10000 m around the analysis grid cell
is calculated as the area-weighted sum of the \hyperref[ch06.235]{analysis cells} inside
the buffer, using the workflow \texttt{egvtools::radius\_function()}. During the calculation of the landscape
metric, inverse distance weighted (power = 2) gap filling on the output is
applied to ensure no missing values at the edges. Then the layer is
rewritten to set its name. Finally, the layer is standardised by
subtracting the arithmetic mean and dividing by the root mean squared error.

\begin{Shaded}
\begin{Highlighting}[]
\CommentTok{\# libs {-}{-}{-}{-}}
\ControlFlowTok{if}\NormalTok{(}\SpecialCharTok{!}\FunctionTok{require}\NormalTok{(terra)) \{}\FunctionTok{install.packages}\NormalTok{(}\StringTok{"terra"}\NormalTok{); }\FunctionTok{require}\NormalTok{(terra)\}}
\ControlFlowTok{if}\NormalTok{(}\SpecialCharTok{!}\FunctionTok{require}\NormalTok{(egvtools)) \{remotes}\SpecialCharTok{::}\FunctionTok{install\_github}\NormalTok{(}\StringTok{"aavotins/egvtools"}\NormalTok{); }\FunctionTok{require}\NormalTok{(egvtools)\}}


\CommentTok{\# Templates {-}{-}{-}{-}{-}}
\NormalTok{template100}\OtherTok{=}\FunctionTok{rast}\NormalTok{(}\StringTok{"./Templates/TemplateRasters/LV100m\_10km.tif"}\NormalTok{)}

\CommentTok{\# radii {-}{-}{-}{-}}
\FunctionTok{radius\_function}\NormalTok{(}
 \AttributeTok{kvadrati\_path =} \StringTok{"./Templates/TemplateGrids/tiles/"}\NormalTok{,}
 \AttributeTok{radii\_path   =} \StringTok{"./Templates/TemplateGridPoints/tiles/"}\NormalTok{,}
 \AttributeTok{tikls100\_path =} \StringTok{"./Templates/TemplateGrids/tikls100\_sauzeme.parquet"}\NormalTok{,}
 \AttributeTok{template\_path =} \StringTok{"./Templates/TemplateRasters/LV100m\_10km.tif"}\NormalTok{,}
 \AttributeTok{input\_layers  =} \FunctionTok{c}\NormalTok{(}\StringTok{"./RasterGrids\_100m/2024/RAW/FarmlandParcels\_FieldsActive\_cell.tif"}\NormalTok{),}
 \AttributeTok{layer\_prefixes =} \FunctionTok{c}\NormalTok{(}\StringTok{"FarmlandParcels\_FieldsActive"}\NormalTok{),}
 \AttributeTok{output\_dir   =} \StringTok{"./RasterGrids\_100m/2024/RAW/"}\NormalTok{,}
 \AttributeTok{n\_workers   =} \DecValTok{6}\NormalTok{,}
 \AttributeTok{radii     =} \FunctionTok{c}\NormalTok{(}\StringTok{"r10000"}\NormalTok{),}
 \AttributeTok{radius\_mode  =} \StringTok{"sparse"}\NormalTok{,}
 \AttributeTok{extract\_fun  =} \StringTok{"mean"}\NormalTok{,}
 \AttributeTok{fill\_missing  =} \ConstantTok{TRUE}\NormalTok{,}
 \AttributeTok{IDW\_weight   =} \DecValTok{2}\NormalTok{,}
 \AttributeTok{future\_max\_size =} \DecValTok{40} \SpecialCharTok{*} \DecValTok{1024}\SpecialCharTok{\^{}}\DecValTok{3}\NormalTok{)}


\CommentTok{\# FarmlandParcels\_FieldsActive\_r10000.tif   egv\_239}
\NormalTok{slanis}\OtherTok{=}\FunctionTok{rast}\NormalTok{(}\StringTok{"./RasterGrids\_100m/2024/RAW/FarmlandParcels\_FieldsActive\_r10000.tif"}\NormalTok{)}
\FunctionTok{names}\NormalTok{(slanis)}\OtherTok{=}\StringTok{"egv\_239"}
\NormalTok{slanis2}\OtherTok{=}\FunctionTok{project}\NormalTok{(slanis,template100)}
\FunctionTok{writeRaster}\NormalTok{(slanis2,}
      \StringTok{"./RasterGrids\_100m/2024/RAW/FarmlandParcels\_FieldsActive\_r10000.tif"}\NormalTok{,}
      \AttributeTok{overwrite=}\ConstantTok{TRUE}\NormalTok{)}

\CommentTok{\# standardisation {-}{-}{-}{-}}
\ControlFlowTok{if}\NormalTok{(}\SpecialCharTok{!}\FunctionTok{require}\NormalTok{(terra)) \{}\FunctionTok{install.packages}\NormalTok{(}\StringTok{"terra"}\NormalTok{); }\FunctionTok{require}\NormalTok{(terra)\}}
\ControlFlowTok{if}\NormalTok{(}\SpecialCharTok{!}\FunctionTok{require}\NormalTok{(tidyverse)) \{}\FunctionTok{install.packages}\NormalTok{(}\StringTok{"tidyverse"}\NormalTok{); }\FunctionTok{require}\NormalTok{(tidyverse)\}}

\NormalTok{nosaukums}\OtherTok{=}\StringTok{"FarmlandParcels\_FieldsActive\_r10000.tif"}
\NormalTok{ielasisanas\_cels}\OtherTok{=}\FunctionTok{paste0}\NormalTok{(}\StringTok{"./RasterGrids\_100m/2024/RAW/"}\NormalTok{,nosaukums)}
\NormalTok{saglabasanas\_cels}\OtherTok{=}\FunctionTok{paste0}\NormalTok{(}\StringTok{"./RasterGrids\_100m/2024/Scaled/"}\NormalTok{,nosaukums)}
\NormalTok{slanis}\OtherTok{=}\FunctionTok{rast}\NormalTok{(ielasisanas\_cels)}
\NormalTok{videjais}\OtherTok{=}\FunctionTok{global}\NormalTok{(slanis,}\AttributeTok{fun=}\StringTok{"mean"}\NormalTok{,}\AttributeTok{na.rm=}\ConstantTok{TRUE}\NormalTok{)}
\NormalTok{centrets}\OtherTok{=}\NormalTok{slanis}\SpecialCharTok{{-}}\NormalTok{videjais[,}\DecValTok{1}\NormalTok{]}
\NormalTok{standartnovirze}\OtherTok{=}\NormalTok{terra}\SpecialCharTok{::}\FunctionTok{global}\NormalTok{(centrets,}\AttributeTok{fun=}\StringTok{"rms"}\NormalTok{,}\AttributeTok{na.rm=}\ConstantTok{TRUE}\NormalTok{)}
\NormalTok{merogots}\OtherTok{=}\NormalTok{centrets}\SpecialCharTok{/}\NormalTok{standartnovirze[,}\DecValTok{1}\NormalTok{]}
\FunctionTok{writeRaster}\NormalTok{(merogots,}
      \AttributeTok{filename=}\NormalTok{saglabasanas\_cels,}
      \AttributeTok{overwrite=}\ConstantTok{TRUE}\NormalTok{)}
\end{Highlighting}
\end{Shaded}

\section{FarmlandPloughed\_CropsFallow\_cell}\label{ch06.240}

\textbf{filename:} \texttt{FarmlandPloughed\_CropsFallow\_cell.tif}

\textbf{layername:} \texttt{egv\_240}

\textbf{English name:} Fractional cover of Crop-, Fallow- Land within the analysis
cell (1 ha)

\textbf{Latvian name:} Aramzemju, papuvju platības īpatsvars analīzes šūnā (1 ha)

\textbf{Procedure:} First, agricultural parcels declared as crops or fallow land are
selected from the \hyperref[Ch04.02]{Rural Support Service's information on declared
fields}. Geometries are then rasterised to input resolution, ensuring
value 1 at the polygon locations and value 0 elsewhere. Rasterisation is
performed using the workflow \texttt{egvtools::polygon2input()}. Once rasterised, the
layer is aggregated to EGV resolution using the workflow \texttt{egvtools::input2egv()},
which calculates the arithmetic mean and thus
results in a cover fraction. During aggregation, inverse
distance weighted (power = 2) gap filling on the output is applied to
ensure no missing values at the edges. Finally, the layer is standardised
by subtracting the arithmetic mean and dividing by the root mean squared error.

\begin{Shaded}
\begin{Highlighting}[]
\CommentTok{\# libs {-}{-}{-}{-}}
\ControlFlowTok{if}\NormalTok{(}\SpecialCharTok{!}\FunctionTok{require}\NormalTok{(egvtools)) \{remotes}\SpecialCharTok{::}\FunctionTok{install\_github}\NormalTok{(}\StringTok{"aavotins/egvtools"}\NormalTok{); }\FunctionTok{require}\NormalTok{(egvtools)\}}
\ControlFlowTok{if}\NormalTok{(}\SpecialCharTok{!}\FunctionTok{require}\NormalTok{(terra)) \{}\FunctionTok{install.packages}\NormalTok{(}\StringTok{"terra"}\NormalTok{); }\FunctionTok{require}\NormalTok{(terra)\}}
\ControlFlowTok{if}\NormalTok{(}\SpecialCharTok{!}\FunctionTok{require}\NormalTok{(sf)) \{}\FunctionTok{install.packages}\NormalTok{(}\StringTok{"sf"}\NormalTok{); }\FunctionTok{require}\NormalTok{(sf)\}}
\ControlFlowTok{if}\NormalTok{(}\SpecialCharTok{!}\FunctionTok{require}\NormalTok{(tidyverse)) \{}\FunctionTok{install.packages}\NormalTok{(}\StringTok{"tidyverse"}\NormalTok{); }\FunctionTok{require}\NormalTok{(tidyverse)\}}
\ControlFlowTok{if}\NormalTok{(}\SpecialCharTok{!}\FunctionTok{require}\NormalTok{(sfarrow)) \{}\FunctionTok{install.packages}\NormalTok{(}\StringTok{"sfarrow"}\NormalTok{); }\FunctionTok{require}\NormalTok{(sfarrow)\}}
\ControlFlowTok{if}\NormalTok{(}\SpecialCharTok{!}\FunctionTok{require}\NormalTok{(readxl)) \{}\FunctionTok{install.packages}\NormalTok{(}\StringTok{"readxl"}\NormalTok{); }\FunctionTok{require}\NormalTok{(readxl)\}}
\ControlFlowTok{if}\NormalTok{(}\SpecialCharTok{!}\FunctionTok{require}\NormalTok{(raster)) \{}\FunctionTok{install.packages}\NormalTok{(}\StringTok{"raster"}\NormalTok{); }\FunctionTok{require}\NormalTok{(raster)\}}
\ControlFlowTok{if}\NormalTok{(}\SpecialCharTok{!}\FunctionTok{require}\NormalTok{(fasterize)) \{}\FunctionTok{install.packages}\NormalTok{(}\StringTok{"fasterize"}\NormalTok{); }\FunctionTok{require}\NormalTok{(fasterize)\}}

\CommentTok{\# templates {-}{-}{-}{-}}
\NormalTok{template100}\OtherTok{=}\FunctionTok{rast}\NormalTok{(}\StringTok{"./Templates/TemplateRasters/LV100m\_10km.tif"}\NormalTok{)}
\NormalTok{template10}\OtherTok{=}\FunctionTok{rast}\NormalTok{(}\StringTok{"./Templates/TemplateRasters/LV10m\_10km.tif"}\NormalTok{)}
\NormalTok{rastrs10}\OtherTok{=}\FunctionTok{raster}\NormalTok{(template10)}

\NormalTok{nulls10}\OtherTok{=}\FunctionTok{rast}\NormalTok{(}\StringTok{"./Templates/TemplateRasters/nulls\_LV10m\_10km.tif"}\NormalTok{)}
\NormalTok{nulls100}\OtherTok{=}\FunctionTok{rast}\NormalTok{(}\StringTok{"./Templates/TemplateRasters/nulls\_LV100m\_10km.tif"}\NormalTok{)}

\CommentTok{\# codes {-}{-}{-}{-}}
\NormalTok{kodi}\OtherTok{=}\FunctionTok{read\_excel}\NormalTok{(}\StringTok{"./Geodata/2024/LAD/KulturuKodi\_2024.xlsx"}\NormalTok{)}
\NormalTok{kodi}\SpecialCharTok{$}\NormalTok{kods}\OtherTok{=}\FunctionTok{as.character}\NormalTok{(kodi}\SpecialCharTok{$}\NormalTok{kods)}
\CommentTok{\# LAD {-}{-}{-}{-}}
\NormalTok{lad}\OtherTok{=}\NormalTok{sfarrow}\SpecialCharTok{::}\FunctionTok{st\_read\_parquet}\NormalTok{(}\StringTok{"./Geodata/2024/LAD/Lauki\_2024.parquet"}\NormalTok{)}
\NormalTok{lad}\SpecialCharTok{$}\NormalTok{yes}\OtherTok{=}\DecValTok{1}
\NormalTok{lad}\OtherTok{=}\NormalTok{lad }\SpecialCharTok{\%\textgreater{}\%} 
 \FunctionTok{left\_join}\NormalTok{(kodi,}\AttributeTok{by=}\FunctionTok{c}\NormalTok{(}\StringTok{"PRODUCT\_CODE"}\OtherTok{=}\StringTok{"kods"}\NormalTok{))}

\CommentTok{\# simple landscape {-}{-}{-}{-}}
\NormalTok{simple\_landscape}\OtherTok{=}\FunctionTok{rast}\NormalTok{(}\StringTok{"RasterGrids\_10m/2024/Ainava\_vienk\_mask.tif"}\NormalTok{)}


\CommentTok{\# FarmlandPloughed\_CropsFallow\_cell.tif egv\_240 {-}{-}{-}{-}}
\NormalTok{dati}\OtherTok{=}\NormalTok{lad }\SpecialCharTok{\%\textgreater{}\%} 
 \FunctionTok{filter}\NormalTok{(SDM\_grupa\_sakums }\SpecialCharTok{\%in\%} \FunctionTok{c}\NormalTok{(}\StringTok{"aramzemes (citur neiekļautās)"}\NormalTok{,}
                 \StringTok{"aramzemes (labība{-}vasarāji)"}\NormalTok{,}
                 \StringTok{"aramzemes (labība{-}ziemāji)"}\NormalTok{,}
                 \StringTok{"aramzemes (vagu un rušināmkultūru)"}\NormalTok{,}
                 \StringTok{"aramzemes (vasaras rapsis un rispsis, kukurūzas, zirņi un pupas, soja, kaņepes)"}\NormalTok{,}
                 \StringTok{"aramzemes (ziemas rapsis un ripsis)"}\NormalTok{,}
                 \StringTok{"papuves"}\NormalTok{))}
\FunctionTok{table}\NormalTok{(dati}\SpecialCharTok{$}\NormalTok{SDM\_grupa\_sakums,}\AttributeTok{useNA=}\StringTok{"always"}\NormalTok{)}

\NormalTok{p2i\_rez}\OtherTok{=}\NormalTok{egvtools}\SpecialCharTok{::}\FunctionTok{polygon2input}\NormalTok{(}\AttributeTok{vector\_data =}\NormalTok{ dati,}
                \AttributeTok{template\_path =} \StringTok{"./Templates/TemplateRasters/LV10m\_10km.tif"}\NormalTok{,}
                \AttributeTok{out\_path =} \StringTok{"./RasterGrids\_10m/2024/"}\NormalTok{,}
                \AttributeTok{file\_name =} \StringTok{"FarmlandPloughed\_CropsFallow\_input.tif"}\NormalTok{,}
                \AttributeTok{value\_field =} \StringTok{"yes"}\NormalTok{,}
                \AttributeTok{prepare=}\ConstantTok{FALSE}\NormalTok{,}
                \AttributeTok{background\_raster =} \StringTok{"./Templates/TemplateRasters/nulls\_LV10m\_10km.tif"}\NormalTok{,}
                \AttributeTok{plot\_result =} \ConstantTok{TRUE}\NormalTok{)}
\NormalTok{p2i\_rez}
\NormalTok{i2e\_rez}\OtherTok{=}\NormalTok{egvtools}\SpecialCharTok{::}\FunctionTok{input2egv}\NormalTok{(}\AttributeTok{input=}\FunctionTok{paste0}\NormalTok{(}\StringTok{"./RasterGrids\_10m/2024/"}\NormalTok{,}
                     \StringTok{"FarmlandPloughed\_CropsFallow\_input.tif"}\NormalTok{),}
              \AttributeTok{egv\_template=} \StringTok{"./Templates/TemplateRasters/LV100m\_10km.tif"}\NormalTok{,}
              \AttributeTok{summary\_function =} \StringTok{"average"}\NormalTok{,}
              \AttributeTok{missing\_job =} \StringTok{"FillOutput"}\NormalTok{,}
              \AttributeTok{outlocation =} \StringTok{"./RasterGrids\_100m/2024/RAW/"}\NormalTok{,}
              \AttributeTok{outfilename =} \StringTok{"FarmlandPloughed\_CropsFallow\_cell.tif"}\NormalTok{,}
              \AttributeTok{layername =} \StringTok{"egv\_240"}\NormalTok{,}
              \AttributeTok{idw\_weight =} \DecValTok{2}\NormalTok{,}
              \AttributeTok{plot\_gaps =} \ConstantTok{FALSE}\NormalTok{,}\AttributeTok{plot\_final =} \ConstantTok{TRUE}\NormalTok{)}
\NormalTok{i2e\_rez}
\FunctionTok{rm}\NormalTok{(p2i\_rez)}
\FunctionTok{rm}\NormalTok{(i2e\_rez)}
\FunctionTok{rm}\NormalTok{(dati)}
\FunctionTok{unlink}\NormalTok{(}\StringTok{"./RasterGrids\_10m/2024/FarmlandPloughed\_CropsFallow\_input.tif"}\NormalTok{)}


\CommentTok{\# standardisation {-}{-}{-}{-}}
\ControlFlowTok{if}\NormalTok{(}\SpecialCharTok{!}\FunctionTok{require}\NormalTok{(terra)) \{}\FunctionTok{install.packages}\NormalTok{(}\StringTok{"terra"}\NormalTok{); }\FunctionTok{require}\NormalTok{(terra)\}}
\ControlFlowTok{if}\NormalTok{(}\SpecialCharTok{!}\FunctionTok{require}\NormalTok{(tidyverse)) \{}\FunctionTok{install.packages}\NormalTok{(}\StringTok{"tidyverse"}\NormalTok{); }\FunctionTok{require}\NormalTok{(tidyverse)\}}

\NormalTok{nosaukums}\OtherTok{=}\StringTok{"FarmlandPloughed\_CropsFallow\_cell.tif"}
\NormalTok{ielasisanas\_cels}\OtherTok{=}\FunctionTok{paste0}\NormalTok{(}\StringTok{"./RasterGrids\_100m/2024/RAW/"}\NormalTok{,nosaukums)}
\NormalTok{saglabasanas\_cels}\OtherTok{=}\FunctionTok{paste0}\NormalTok{(}\StringTok{"./RasterGrids\_100m/2024/Scaled/"}\NormalTok{,nosaukums)}
\NormalTok{slanis}\OtherTok{=}\FunctionTok{rast}\NormalTok{(ielasisanas\_cels)}
\NormalTok{videjais}\OtherTok{=}\FunctionTok{global}\NormalTok{(slanis,}\AttributeTok{fun=}\StringTok{"mean"}\NormalTok{,}\AttributeTok{na.rm=}\ConstantTok{TRUE}\NormalTok{)}
\NormalTok{centrets}\OtherTok{=}\NormalTok{slanis}\SpecialCharTok{{-}}\NormalTok{videjais[,}\DecValTok{1}\NormalTok{]}
\NormalTok{standartnovirze}\OtherTok{=}\NormalTok{terra}\SpecialCharTok{::}\FunctionTok{global}\NormalTok{(centrets,}\AttributeTok{fun=}\StringTok{"rms"}\NormalTok{,}\AttributeTok{na.rm=}\ConstantTok{TRUE}\NormalTok{)}
\NormalTok{merogots}\OtherTok{=}\NormalTok{centrets}\SpecialCharTok{/}\NormalTok{standartnovirze[,}\DecValTok{1}\NormalTok{]}
\FunctionTok{writeRaster}\NormalTok{(merogots,}
      \AttributeTok{filename=}\NormalTok{saglabasanas\_cels,}
      \AttributeTok{overwrite=}\ConstantTok{TRUE}\NormalTok{)}
\end{Highlighting}
\end{Shaded}

\section{FarmlandPloughed\_CropsFallow\_r500}\label{ch06.241}

\textbf{filename:} \texttt{FarmlandPloughed\_CropsFallow\_r500.tif}

\textbf{layername:} \texttt{egv\_241}

\textbf{English name:} Fractional cover of Crop-, Fallow- Land within the 0.5 km
landscape

\textbf{Latvian name:} Aramzemju, papuvju platības īpatsvars 0,5 km ainavā

\textbf{Procedure:} The cover fraction within a radius of 500 m around the analysis grid cell is
calculated as the area-weighted sum of the \hyperref[ch06.240]{analysis cells} inside the
buffer, using the workflow \texttt{egvtools::radius\_function()}. During the calculation of the landscape metric,
inverse distance weighted (power = 2) gap filling on the output is applied
to ensure no missing values at the edges. Then the layer is rewritten to set
its name. Finally, the layer is standardised by subtracting the arithmetic
mean and dividing by the root mean squared error.

\begin{Shaded}
\begin{Highlighting}[]
\CommentTok{\# libs {-}{-}{-}{-}}
\ControlFlowTok{if}\NormalTok{(}\SpecialCharTok{!}\FunctionTok{require}\NormalTok{(terra)) \{}\FunctionTok{install.packages}\NormalTok{(}\StringTok{"terra"}\NormalTok{); }\FunctionTok{require}\NormalTok{(terra)\}}
\ControlFlowTok{if}\NormalTok{(}\SpecialCharTok{!}\FunctionTok{require}\NormalTok{(egvtools)) \{remotes}\SpecialCharTok{::}\FunctionTok{install\_github}\NormalTok{(}\StringTok{"aavotins/egvtools"}\NormalTok{); }\FunctionTok{require}\NormalTok{(egvtools)\}}


\CommentTok{\# Templates {-}{-}{-}{-}{-}}
\NormalTok{template100}\OtherTok{=}\FunctionTok{rast}\NormalTok{(}\StringTok{"./Templates/TemplateRasters/LV100m\_10km.tif"}\NormalTok{)}

\CommentTok{\# radii {-}{-}{-}{-}}
\FunctionTok{radius\_function}\NormalTok{(}
 \AttributeTok{kvadrati\_path =} \StringTok{"./Templates/TemplateGrids/tiles/"}\NormalTok{,}
 \AttributeTok{radii\_path   =} \StringTok{"./Templates/TemplateGridPoints/tiles/"}\NormalTok{,}
 \AttributeTok{tikls100\_path =} \StringTok{"./Templates/TemplateGrids/tikls100\_sauzeme.parquet"}\NormalTok{,}
 \AttributeTok{template\_path =} \StringTok{"./Templates/TemplateRasters/LV100m\_10km.tif"}\NormalTok{,}
 \AttributeTok{input\_layers  =} \FunctionTok{c}\NormalTok{(}\StringTok{"./RasterGrids\_100m/2024/RAW/FarmlandPloughed\_CropsFallow\_cell.tif"}\NormalTok{),}
 \AttributeTok{layer\_prefixes =} \FunctionTok{c}\NormalTok{(}\StringTok{"FarmlandPloughed\_CropsFallow"}\NormalTok{),}
 \AttributeTok{output\_dir   =} \StringTok{"./RasterGrids\_100m/2024/RAW/"}\NormalTok{,}
 \AttributeTok{n\_workers   =} \DecValTok{6}\NormalTok{,}
 \AttributeTok{radii     =} \FunctionTok{c}\NormalTok{(}\StringTok{"r500"}\NormalTok{),}
 \AttributeTok{radius\_mode  =} \StringTok{"sparse"}\NormalTok{,}
 \AttributeTok{extract\_fun  =} \StringTok{"mean"}\NormalTok{,}
 \AttributeTok{fill\_missing  =} \ConstantTok{TRUE}\NormalTok{,}
 \AttributeTok{IDW\_weight   =} \DecValTok{2}\NormalTok{,}
 \AttributeTok{future\_max\_size =} \DecValTok{40} \SpecialCharTok{*} \DecValTok{1024}\SpecialCharTok{\^{}}\DecValTok{3}\NormalTok{)}


\CommentTok{\# FarmlandPloughed\_CropsFallow\_r500.tif egv\_241}
\NormalTok{slanis}\OtherTok{=}\FunctionTok{rast}\NormalTok{(}\StringTok{"./RasterGrids\_100m/2024/RAW/FarmlandPloughed\_CropsFallow\_r500.tif"}\NormalTok{)}
\FunctionTok{names}\NormalTok{(slanis)}\OtherTok{=}\StringTok{"egv\_241"}
\NormalTok{slanis2}\OtherTok{=}\FunctionTok{project}\NormalTok{(slanis,template100)}
\FunctionTok{writeRaster}\NormalTok{(slanis2,}
      \StringTok{"./RasterGrids\_100m/2024/RAW/FarmlandPloughed\_CropsFallow\_r500.tif"}\NormalTok{,}
      \AttributeTok{overwrite=}\ConstantTok{TRUE}\NormalTok{)}

\CommentTok{\# standardisation {-}{-}{-}{-}}
\ControlFlowTok{if}\NormalTok{(}\SpecialCharTok{!}\FunctionTok{require}\NormalTok{(terra)) \{}\FunctionTok{install.packages}\NormalTok{(}\StringTok{"terra"}\NormalTok{); }\FunctionTok{require}\NormalTok{(terra)\}}
\ControlFlowTok{if}\NormalTok{(}\SpecialCharTok{!}\FunctionTok{require}\NormalTok{(tidyverse)) \{}\FunctionTok{install.packages}\NormalTok{(}\StringTok{"tidyverse"}\NormalTok{); }\FunctionTok{require}\NormalTok{(tidyverse)\}}

\NormalTok{nosaukums}\OtherTok{=}\StringTok{"FarmlandPloughed\_CropsFallow\_r500.tif"}
\NormalTok{ielasisanas\_cels}\OtherTok{=}\FunctionTok{paste0}\NormalTok{(}\StringTok{"./RasterGrids\_100m/2024/RAW/"}\NormalTok{,nosaukums)}
\NormalTok{saglabasanas\_cels}\OtherTok{=}\FunctionTok{paste0}\NormalTok{(}\StringTok{"./RasterGrids\_100m/2024/Scaled/"}\NormalTok{,nosaukums)}
\NormalTok{slanis}\OtherTok{=}\FunctionTok{rast}\NormalTok{(ielasisanas\_cels)}
\NormalTok{videjais}\OtherTok{=}\FunctionTok{global}\NormalTok{(slanis,}\AttributeTok{fun=}\StringTok{"mean"}\NormalTok{,}\AttributeTok{na.rm=}\ConstantTok{TRUE}\NormalTok{)}
\NormalTok{centrets}\OtherTok{=}\NormalTok{slanis}\SpecialCharTok{{-}}\NormalTok{videjais[,}\DecValTok{1}\NormalTok{]}
\NormalTok{standartnovirze}\OtherTok{=}\NormalTok{terra}\SpecialCharTok{::}\FunctionTok{global}\NormalTok{(centrets,}\AttributeTok{fun=}\StringTok{"rms"}\NormalTok{,}\AttributeTok{na.rm=}\ConstantTok{TRUE}\NormalTok{)}
\NormalTok{merogots}\OtherTok{=}\NormalTok{centrets}\SpecialCharTok{/}\NormalTok{standartnovirze[,}\DecValTok{1}\NormalTok{]}
\FunctionTok{writeRaster}\NormalTok{(merogots,}
      \AttributeTok{filename=}\NormalTok{saglabasanas\_cels,}
      \AttributeTok{overwrite=}\ConstantTok{TRUE}\NormalTok{)}
\end{Highlighting}
\end{Shaded}

\section{FarmlandPloughed\_CropsFallow\_r1250}\label{ch06.242}

\textbf{filename:} \texttt{FarmlandPloughed\_CropsFallow\_r1250.tif}

\textbf{layername:} \texttt{egv\_242}

\textbf{English name:} Fractional cover of Crop-, Fallow- Land within the 1.25 km
landscape

\textbf{Latvian name:} Aramzemju, papuvju platības īpatsvars 1,25 km ainavā

\textbf{Procedure:} The cover fraction within a radius of 1250 m around the analysis grid cell
is calculated as the area-weighted sum of the \hyperref[ch06.240]{analysis cells} inside
the buffer, using the workflow \texttt{egvtools::radius\_function()}. During the calculation of the landscape
metric, inverse distance weighted (power = 2) gap filling on the output is
applied to ensure no missing values at the edges. Then the layer is
rewritten to set its name. Finally, the layer is standardised by
subtracting the arithmetic mean and dividing by the root mean squared error.

\begin{Shaded}
\begin{Highlighting}[]
\CommentTok{\# libs {-}{-}{-}{-}}
\ControlFlowTok{if}\NormalTok{(}\SpecialCharTok{!}\FunctionTok{require}\NormalTok{(terra)) \{}\FunctionTok{install.packages}\NormalTok{(}\StringTok{"terra"}\NormalTok{); }\FunctionTok{require}\NormalTok{(terra)\}}
\ControlFlowTok{if}\NormalTok{(}\SpecialCharTok{!}\FunctionTok{require}\NormalTok{(egvtools)) \{remotes}\SpecialCharTok{::}\FunctionTok{install\_github}\NormalTok{(}\StringTok{"aavotins/egvtools"}\NormalTok{); }\FunctionTok{require}\NormalTok{(egvtools)\}}


\CommentTok{\# Templates {-}{-}{-}{-}{-}}
\NormalTok{template100}\OtherTok{=}\FunctionTok{rast}\NormalTok{(}\StringTok{"./Templates/TemplateRasters/LV100m\_10km.tif"}\NormalTok{)}

\CommentTok{\# radii {-}{-}{-}{-}}
\FunctionTok{radius\_function}\NormalTok{(}
 \AttributeTok{kvadrati\_path =} \StringTok{"./Templates/TemplateGrids/tiles/"}\NormalTok{,}
 \AttributeTok{radii\_path   =} \StringTok{"./Templates/TemplateGridPoints/tiles/"}\NormalTok{,}
 \AttributeTok{tikls100\_path =} \StringTok{"./Templates/TemplateGrids/tikls100\_sauzeme.parquet"}\NormalTok{,}
 \AttributeTok{template\_path =} \StringTok{"./Templates/TemplateRasters/LV100m\_10km.tif"}\NormalTok{,}
 \AttributeTok{input\_layers  =} \FunctionTok{c}\NormalTok{(}\StringTok{"./RasterGrids\_100m/2024/RAW/FarmlandPloughed\_CropsFallow\_cell.tif"}\NormalTok{),}
 \AttributeTok{layer\_prefixes =} \FunctionTok{c}\NormalTok{(}\StringTok{"FarmlandPloughed\_CropsFallow"}\NormalTok{),}
 \AttributeTok{output\_dir   =} \StringTok{"./RasterGrids\_100m/2024/RAW/"}\NormalTok{,}
 \AttributeTok{n\_workers   =} \DecValTok{6}\NormalTok{,}
 \AttributeTok{radii     =} \FunctionTok{c}\NormalTok{(}\StringTok{"r1250"}\NormalTok{),}
 \AttributeTok{radius\_mode  =} \StringTok{"sparse"}\NormalTok{,}
 \AttributeTok{extract\_fun  =} \StringTok{"mean"}\NormalTok{,}
 \AttributeTok{fill\_missing  =} \ConstantTok{TRUE}\NormalTok{,}
 \AttributeTok{IDW\_weight   =} \DecValTok{2}\NormalTok{,}
 \AttributeTok{future\_max\_size =} \DecValTok{40} \SpecialCharTok{*} \DecValTok{1024}\SpecialCharTok{\^{}}\DecValTok{3}\NormalTok{)}


\CommentTok{\# FarmlandPloughed\_CropsFallow\_r1250.tif    egv\_242}
\NormalTok{slanis}\OtherTok{=}\FunctionTok{rast}\NormalTok{(}\StringTok{"./RasterGrids\_100m/2024/RAW/FarmlandPloughed\_CropsFallow\_r1250.tif"}\NormalTok{)}
\FunctionTok{names}\NormalTok{(slanis)}\OtherTok{=}\StringTok{"egv\_242"}
\NormalTok{slanis2}\OtherTok{=}\FunctionTok{project}\NormalTok{(slanis,template100)}
\FunctionTok{writeRaster}\NormalTok{(slanis2,}
      \StringTok{"./RasterGrids\_100m/2024/RAW/FarmlandPloughed\_CropsFallow\_r1250.tif"}\NormalTok{,}
      \AttributeTok{overwrite=}\ConstantTok{TRUE}\NormalTok{)}

\CommentTok{\# standardisation {-}{-}{-}{-}}
\ControlFlowTok{if}\NormalTok{(}\SpecialCharTok{!}\FunctionTok{require}\NormalTok{(terra)) \{}\FunctionTok{install.packages}\NormalTok{(}\StringTok{"terra"}\NormalTok{); }\FunctionTok{require}\NormalTok{(terra)\}}
\ControlFlowTok{if}\NormalTok{(}\SpecialCharTok{!}\FunctionTok{require}\NormalTok{(tidyverse)) \{}\FunctionTok{install.packages}\NormalTok{(}\StringTok{"tidyverse"}\NormalTok{); }\FunctionTok{require}\NormalTok{(tidyverse)\}}

\NormalTok{nosaukums}\OtherTok{=}\StringTok{"FarmlandPloughed\_CropsFallow\_r1250.tif"}
\NormalTok{ielasisanas\_cels}\OtherTok{=}\FunctionTok{paste0}\NormalTok{(}\StringTok{"./RasterGrids\_100m/2024/RAW/"}\NormalTok{,nosaukums)}
\NormalTok{saglabasanas\_cels}\OtherTok{=}\FunctionTok{paste0}\NormalTok{(}\StringTok{"./RasterGrids\_100m/2024/Scaled/"}\NormalTok{,nosaukums)}
\NormalTok{slanis}\OtherTok{=}\FunctionTok{rast}\NormalTok{(ielasisanas\_cels)}
\NormalTok{videjais}\OtherTok{=}\FunctionTok{global}\NormalTok{(slanis,}\AttributeTok{fun=}\StringTok{"mean"}\NormalTok{,}\AttributeTok{na.rm=}\ConstantTok{TRUE}\NormalTok{)}
\NormalTok{centrets}\OtherTok{=}\NormalTok{slanis}\SpecialCharTok{{-}}\NormalTok{videjais[,}\DecValTok{1}\NormalTok{]}
\NormalTok{standartnovirze}\OtherTok{=}\NormalTok{terra}\SpecialCharTok{::}\FunctionTok{global}\NormalTok{(centrets,}\AttributeTok{fun=}\StringTok{"rms"}\NormalTok{,}\AttributeTok{na.rm=}\ConstantTok{TRUE}\NormalTok{)}
\NormalTok{merogots}\OtherTok{=}\NormalTok{centrets}\SpecialCharTok{/}\NormalTok{standartnovirze[,}\DecValTok{1}\NormalTok{]}
\FunctionTok{writeRaster}\NormalTok{(merogots,}
      \AttributeTok{filename=}\NormalTok{saglabasanas\_cels,}
      \AttributeTok{overwrite=}\ConstantTok{TRUE}\NormalTok{)}
\end{Highlighting}
\end{Shaded}

\section{FarmlandPloughed\_CropsFallow\_r3000}\label{ch06.243}

\textbf{filename:} \texttt{FarmlandPloughed\_CropsFallow\_r3000.tif}

\textbf{layername:} \texttt{egv\_243}

\textbf{English name:} Fractional cover of Crop-, Fallow- Land within the 3 km
landscape

\textbf{Latvian name:} Aramzemju, papuvju platības īpatsvars 3 km ainavā

\textbf{Procedure:} The cover fraction within a radius of 3000 m around the analysis grid cell
is calculated as the area-weighted sum of the \hyperref[ch06.240]{analysis cells} inside
the buffer, using the workflow \texttt{egvtools::radius\_function()}. During the calculation of the landscape
metric, inverse distance weighted (power = 2) gap filling on the output is
applied to ensure no missing values at the edges. Then the layer is
rewritten to set its name. Finally, the layer is standardised by
subtracting the arithmetic mean and dividing by the root mean squared error.

\begin{Shaded}
\begin{Highlighting}[]
\CommentTok{\# libs {-}{-}{-}{-}}
\ControlFlowTok{if}\NormalTok{(}\SpecialCharTok{!}\FunctionTok{require}\NormalTok{(terra)) \{}\FunctionTok{install.packages}\NormalTok{(}\StringTok{"terra"}\NormalTok{); }\FunctionTok{require}\NormalTok{(terra)\}}
\ControlFlowTok{if}\NormalTok{(}\SpecialCharTok{!}\FunctionTok{require}\NormalTok{(egvtools)) \{remotes}\SpecialCharTok{::}\FunctionTok{install\_github}\NormalTok{(}\StringTok{"aavotins/egvtools"}\NormalTok{); }\FunctionTok{require}\NormalTok{(egvtools)\}}


\CommentTok{\# Templates {-}{-}{-}{-}{-}}
\NormalTok{template100}\OtherTok{=}\FunctionTok{rast}\NormalTok{(}\StringTok{"./Templates/TemplateRasters/LV100m\_10km.tif"}\NormalTok{)}

\CommentTok{\# radii {-}{-}{-}{-}}
\FunctionTok{radius\_function}\NormalTok{(}
 \AttributeTok{kvadrati\_path =} \StringTok{"./Templates/TemplateGrids/tiles/"}\NormalTok{,}
 \AttributeTok{radii\_path   =} \StringTok{"./Templates/TemplateGridPoints/tiles/"}\NormalTok{,}
 \AttributeTok{tikls100\_path =} \StringTok{"./Templates/TemplateGrids/tikls100\_sauzeme.parquet"}\NormalTok{,}
 \AttributeTok{template\_path =} \StringTok{"./Templates/TemplateRasters/LV100m\_10km.tif"}\NormalTok{,}
 \AttributeTok{input\_layers  =} \FunctionTok{c}\NormalTok{(}\StringTok{"./RasterGrids\_100m/2024/RAW/FarmlandPloughed\_CropsFallow\_cell.tif"}\NormalTok{),}
 \AttributeTok{layer\_prefixes =} \FunctionTok{c}\NormalTok{(}\StringTok{"FarmlandPloughed\_CropsFallow"}\NormalTok{),}
 \AttributeTok{output\_dir   =} \StringTok{"./RasterGrids\_100m/2024/RAW/"}\NormalTok{,}
 \AttributeTok{n\_workers   =} \DecValTok{6}\NormalTok{,}
 \AttributeTok{radii     =} \FunctionTok{c}\NormalTok{(}\StringTok{"r3000"}\NormalTok{),}
 \AttributeTok{radius\_mode  =} \StringTok{"sparse"}\NormalTok{,}
 \AttributeTok{extract\_fun  =} \StringTok{"mean"}\NormalTok{,}
 \AttributeTok{fill\_missing  =} \ConstantTok{TRUE}\NormalTok{,}
 \AttributeTok{IDW\_weight   =} \DecValTok{2}\NormalTok{,}
 \AttributeTok{future\_max\_size =} \DecValTok{40} \SpecialCharTok{*} \DecValTok{1024}\SpecialCharTok{\^{}}\DecValTok{3}\NormalTok{)}


\CommentTok{\# FarmlandPloughed\_CropsFallow\_r3000.tif    egv\_243}
\NormalTok{slanis}\OtherTok{=}\FunctionTok{rast}\NormalTok{(}\StringTok{"./RasterGrids\_100m/2024/RAW/FarmlandPloughed\_CropsFallow\_r3000.tif"}\NormalTok{)}
\FunctionTok{names}\NormalTok{(slanis)}\OtherTok{=}\StringTok{"egv\_243"}
\NormalTok{slanis2}\OtherTok{=}\FunctionTok{project}\NormalTok{(slanis,template100)}
\FunctionTok{writeRaster}\NormalTok{(slanis2,}
      \StringTok{"./RasterGrids\_100m/2024/RAW/FarmlandPloughed\_CropsFallow\_r3000.tif"}\NormalTok{,}
      \AttributeTok{overwrite=}\ConstantTok{TRUE}\NormalTok{)}

\CommentTok{\# standardisation {-}{-}{-}{-}}
\ControlFlowTok{if}\NormalTok{(}\SpecialCharTok{!}\FunctionTok{require}\NormalTok{(terra)) \{}\FunctionTok{install.packages}\NormalTok{(}\StringTok{"terra"}\NormalTok{); }\FunctionTok{require}\NormalTok{(terra)\}}
\ControlFlowTok{if}\NormalTok{(}\SpecialCharTok{!}\FunctionTok{require}\NormalTok{(tidyverse)) \{}\FunctionTok{install.packages}\NormalTok{(}\StringTok{"tidyverse"}\NormalTok{); }\FunctionTok{require}\NormalTok{(tidyverse)\}}

\NormalTok{nosaukums}\OtherTok{=}\StringTok{"FarmlandPloughed\_CropsFallow\_r3000.tif"}
\NormalTok{ielasisanas\_cels}\OtherTok{=}\FunctionTok{paste0}\NormalTok{(}\StringTok{"./RasterGrids\_100m/2024/RAW/"}\NormalTok{,nosaukums)}
\NormalTok{saglabasanas\_cels}\OtherTok{=}\FunctionTok{paste0}\NormalTok{(}\StringTok{"./RasterGrids\_100m/2024/Scaled/"}\NormalTok{,nosaukums)}
\NormalTok{slanis}\OtherTok{=}\FunctionTok{rast}\NormalTok{(ielasisanas\_cels)}
\NormalTok{videjais}\OtherTok{=}\FunctionTok{global}\NormalTok{(slanis,}\AttributeTok{fun=}\StringTok{"mean"}\NormalTok{,}\AttributeTok{na.rm=}\ConstantTok{TRUE}\NormalTok{)}
\NormalTok{centrets}\OtherTok{=}\NormalTok{slanis}\SpecialCharTok{{-}}\NormalTok{videjais[,}\DecValTok{1}\NormalTok{]}
\NormalTok{standartnovirze}\OtherTok{=}\NormalTok{terra}\SpecialCharTok{::}\FunctionTok{global}\NormalTok{(centrets,}\AttributeTok{fun=}\StringTok{"rms"}\NormalTok{,}\AttributeTok{na.rm=}\ConstantTok{TRUE}\NormalTok{)}
\NormalTok{merogots}\OtherTok{=}\NormalTok{centrets}\SpecialCharTok{/}\NormalTok{standartnovirze[,}\DecValTok{1}\NormalTok{]}
\FunctionTok{writeRaster}\NormalTok{(merogots,}
      \AttributeTok{filename=}\NormalTok{saglabasanas\_cels,}
      \AttributeTok{overwrite=}\ConstantTok{TRUE}\NormalTok{)}
\end{Highlighting}
\end{Shaded}

\section{FarmlandPloughed\_CropsFallow\_r10000}\label{ch06.244}

\textbf{filename:} \texttt{FarmlandPloughed\_CropsFallow\_r10000.tif}

\textbf{layername:} \texttt{egv\_244}

\textbf{English name:} Fractional cover of Crop-, Fallow- Land within the 10 km
landscape

\textbf{Latvian name:} Aramzemju, papuvju platības īpatsvars 10 km ainavā

\textbf{Procedure:} The cover fraction within a radius of 10000 m around the analysis grid cell
is calculated as the area-weighted sum of the \hyperref[ch06.240]{analysis cells} inside
the buffer, using the workflow \texttt{egvtools::radius\_function()}. During the calculation of the landscape
metric, inverse distance weighted (power = 2) gap filling on the output is
applied to ensure no missing values at the edges. Then the layer is
rewritten to set its name. Finally, the layer is standardised by
subtracting the arithmetic mean and dividing by the root mean squared error.

\begin{Shaded}
\begin{Highlighting}[]
\CommentTok{\# libs {-}{-}{-}{-}}
\ControlFlowTok{if}\NormalTok{(}\SpecialCharTok{!}\FunctionTok{require}\NormalTok{(terra)) \{}\FunctionTok{install.packages}\NormalTok{(}\StringTok{"terra"}\NormalTok{); }\FunctionTok{require}\NormalTok{(terra)\}}
\ControlFlowTok{if}\NormalTok{(}\SpecialCharTok{!}\FunctionTok{require}\NormalTok{(egvtools)) \{remotes}\SpecialCharTok{::}\FunctionTok{install\_github}\NormalTok{(}\StringTok{"aavotins/egvtools"}\NormalTok{); }\FunctionTok{require}\NormalTok{(egvtools)\}}


\CommentTok{\# Templates {-}{-}{-}{-}{-}}
\NormalTok{template100}\OtherTok{=}\FunctionTok{rast}\NormalTok{(}\StringTok{"./Templates/TemplateRasters/LV100m\_10km.tif"}\NormalTok{)}

\CommentTok{\# radii {-}{-}{-}{-}}
\FunctionTok{radius\_function}\NormalTok{(}
 \AttributeTok{kvadrati\_path =} \StringTok{"./Templates/TemplateGrids/tiles/"}\NormalTok{,}
 \AttributeTok{radii\_path   =} \StringTok{"./Templates/TemplateGridPoints/tiles/"}\NormalTok{,}
 \AttributeTok{tikls100\_path =} \StringTok{"./Templates/TemplateGrids/tikls100\_sauzeme.parquet"}\NormalTok{,}
 \AttributeTok{template\_path =} \StringTok{"./Templates/TemplateRasters/LV100m\_10km.tif"}\NormalTok{,}
 \AttributeTok{input\_layers  =} \FunctionTok{c}\NormalTok{(}\StringTok{"./RasterGrids\_100m/2024/RAW/FarmlandPloughed\_CropsFallow\_cell.tif"}\NormalTok{),}
 \AttributeTok{layer\_prefixes =} \FunctionTok{c}\NormalTok{(}\StringTok{"FarmlandPloughed\_CropsFallow"}\NormalTok{),}
 \AttributeTok{output\_dir   =} \StringTok{"./RasterGrids\_100m/2024/RAW/"}\NormalTok{,}
 \AttributeTok{n\_workers   =} \DecValTok{6}\NormalTok{,}
 \AttributeTok{radii     =} \FunctionTok{c}\NormalTok{(}\StringTok{"r10000"}\NormalTok{),}
 \AttributeTok{radius\_mode  =} \StringTok{"sparse"}\NormalTok{,}
 \AttributeTok{extract\_fun  =} \StringTok{"mean"}\NormalTok{,}
 \AttributeTok{fill\_missing  =} \ConstantTok{TRUE}\NormalTok{,}
 \AttributeTok{IDW\_weight   =} \DecValTok{2}\NormalTok{,}
 \AttributeTok{future\_max\_size =} \DecValTok{40} \SpecialCharTok{*} \DecValTok{1024}\SpecialCharTok{\^{}}\DecValTok{3}\NormalTok{)}


\CommentTok{\# FarmlandPloughed\_CropsFallow\_r10000.tif   egv\_244}
\NormalTok{slanis}\OtherTok{=}\FunctionTok{rast}\NormalTok{(}\StringTok{"./RasterGrids\_100m/2024/RAW/FarmlandPloughed\_CropsFallow\_r10000.tif"}\NormalTok{)}
\FunctionTok{names}\NormalTok{(slanis)}\OtherTok{=}\StringTok{"egv\_244"}
\NormalTok{slanis2}\OtherTok{=}\FunctionTok{project}\NormalTok{(slanis,template100)}
\FunctionTok{writeRaster}\NormalTok{(slanis2,}
      \StringTok{"./RasterGrids\_100m/2024/RAW/FarmlandPloughed\_CropsFallow\_r10000.tif"}\NormalTok{,}
      \AttributeTok{overwrite=}\ConstantTok{TRUE}\NormalTok{)}

\CommentTok{\# standardisation {-}{-}{-}{-}}
\ControlFlowTok{if}\NormalTok{(}\SpecialCharTok{!}\FunctionTok{require}\NormalTok{(terra)) \{}\FunctionTok{install.packages}\NormalTok{(}\StringTok{"terra"}\NormalTok{); }\FunctionTok{require}\NormalTok{(terra)\}}
\ControlFlowTok{if}\NormalTok{(}\SpecialCharTok{!}\FunctionTok{require}\NormalTok{(tidyverse)) \{}\FunctionTok{install.packages}\NormalTok{(}\StringTok{"tidyverse"}\NormalTok{); }\FunctionTok{require}\NormalTok{(tidyverse)\}}

\NormalTok{nosaukums}\OtherTok{=}\StringTok{"FarmlandPloughed\_CropsFallow\_r10000.tif"}
\NormalTok{ielasisanas\_cels}\OtherTok{=}\FunctionTok{paste0}\NormalTok{(}\StringTok{"./RasterGrids\_100m/2024/RAW/"}\NormalTok{,nosaukums)}
\NormalTok{saglabasanas\_cels}\OtherTok{=}\FunctionTok{paste0}\NormalTok{(}\StringTok{"./RasterGrids\_100m/2024/Scaled/"}\NormalTok{,nosaukums)}
\NormalTok{slanis}\OtherTok{=}\FunctionTok{rast}\NormalTok{(ielasisanas\_cels)}
\NormalTok{videjais}\OtherTok{=}\FunctionTok{global}\NormalTok{(slanis,}\AttributeTok{fun=}\StringTok{"mean"}\NormalTok{,}\AttributeTok{na.rm=}\ConstantTok{TRUE}\NormalTok{)}
\NormalTok{centrets}\OtherTok{=}\NormalTok{slanis}\SpecialCharTok{{-}}\NormalTok{videjais[,}\DecValTok{1}\NormalTok{]}
\NormalTok{standartnovirze}\OtherTok{=}\NormalTok{terra}\SpecialCharTok{::}\FunctionTok{global}\NormalTok{(centrets,}\AttributeTok{fun=}\StringTok{"rms"}\NormalTok{,}\AttributeTok{na.rm=}\ConstantTok{TRUE}\NormalTok{)}
\NormalTok{merogots}\OtherTok{=}\NormalTok{centrets}\SpecialCharTok{/}\NormalTok{standartnovirze[,}\DecValTok{1}\NormalTok{]}
\FunctionTok{writeRaster}\NormalTok{(merogots,}
      \AttributeTok{filename=}\NormalTok{saglabasanas\_cels,}
      \AttributeTok{overwrite=}\ConstantTok{TRUE}\NormalTok{)}
\end{Highlighting}
\end{Shaded}

\section{FarmlandPloughed\_CropsFallowTempGrass\_cell}\label{ch06.245}

\textbf{filename:} \texttt{FarmlandPloughed\_CropsFallowTempGrass\_cell.tif}

\textbf{layername:} \texttt{egv\_245}

\textbf{English name:} Fractional cover of Crop-, Fallow-, Temporary Grass- Lands
within the analysis cell (1 ha)

\textbf{Latvian name:} Aramzemju, papuvju, zālāju-aramzemē platības īpatsvars
analīzes šūnā (1 ha)

\textbf{Procedure:} First, agricultural parcels declared as crops, fallow land or
grasslands in arable land are selected from the \hyperref[Ch04.02]{Rural Support Service's
information on declared fields}. Geometries are then rasterised to
input resolution, ensuring value 1 at the polygon locations and value 0
elsewhere. Rasterisation is
performed using the workflow \texttt{egvtools::polygon2input()}. Once rasterised, the
layer is aggregated to EGV resolution using the workflow \texttt{egvtools::input2egv()},
which calculates the arithmetic mean and thus
results in a cover fraction. During
aggregation, inverse distance weighted (power = 2) gap filling on the output is
applied to ensure no missing values at the edges. Finally, the layer is
standardised by subtracting the arithmetic mean and dividing by the root mean squared
error.

\begin{Shaded}
\begin{Highlighting}[]
\CommentTok{\# libs {-}{-}{-}{-}}
\ControlFlowTok{if}\NormalTok{(}\SpecialCharTok{!}\FunctionTok{require}\NormalTok{(egvtools)) \{remotes}\SpecialCharTok{::}\FunctionTok{install\_github}\NormalTok{(}\StringTok{"aavotins/egvtools"}\NormalTok{); }\FunctionTok{require}\NormalTok{(egvtools)\}}
\ControlFlowTok{if}\NormalTok{(}\SpecialCharTok{!}\FunctionTok{require}\NormalTok{(terra)) \{}\FunctionTok{install.packages}\NormalTok{(}\StringTok{"terra"}\NormalTok{); }\FunctionTok{require}\NormalTok{(terra)\}}
\ControlFlowTok{if}\NormalTok{(}\SpecialCharTok{!}\FunctionTok{require}\NormalTok{(sf)) \{}\FunctionTok{install.packages}\NormalTok{(}\StringTok{"sf"}\NormalTok{); }\FunctionTok{require}\NormalTok{(sf)\}}
\ControlFlowTok{if}\NormalTok{(}\SpecialCharTok{!}\FunctionTok{require}\NormalTok{(tidyverse)) \{}\FunctionTok{install.packages}\NormalTok{(}\StringTok{"tidyverse"}\NormalTok{); }\FunctionTok{require}\NormalTok{(tidyverse)\}}
\ControlFlowTok{if}\NormalTok{(}\SpecialCharTok{!}\FunctionTok{require}\NormalTok{(sfarrow)) \{}\FunctionTok{install.packages}\NormalTok{(}\StringTok{"sfarrow"}\NormalTok{); }\FunctionTok{require}\NormalTok{(sfarrow)\}}
\ControlFlowTok{if}\NormalTok{(}\SpecialCharTok{!}\FunctionTok{require}\NormalTok{(readxl)) \{}\FunctionTok{install.packages}\NormalTok{(}\StringTok{"readxl"}\NormalTok{); }\FunctionTok{require}\NormalTok{(readxl)\}}
\ControlFlowTok{if}\NormalTok{(}\SpecialCharTok{!}\FunctionTok{require}\NormalTok{(raster)) \{}\FunctionTok{install.packages}\NormalTok{(}\StringTok{"raster"}\NormalTok{); }\FunctionTok{require}\NormalTok{(raster)\}}
\ControlFlowTok{if}\NormalTok{(}\SpecialCharTok{!}\FunctionTok{require}\NormalTok{(fasterize)) \{}\FunctionTok{install.packages}\NormalTok{(}\StringTok{"fasterize"}\NormalTok{); }\FunctionTok{require}\NormalTok{(fasterize)\}}

\CommentTok{\# templates {-}{-}{-}{-}}
\NormalTok{template100}\OtherTok{=}\FunctionTok{rast}\NormalTok{(}\StringTok{"./Templates/TemplateRasters/LV100m\_10km.tif"}\NormalTok{)}
\NormalTok{template10}\OtherTok{=}\FunctionTok{rast}\NormalTok{(}\StringTok{"./Templates/TemplateRasters/LV10m\_10km.tif"}\NormalTok{)}
\NormalTok{rastrs10}\OtherTok{=}\FunctionTok{raster}\NormalTok{(template10)}

\NormalTok{nulls10}\OtherTok{=}\FunctionTok{rast}\NormalTok{(}\StringTok{"./Templates/TemplateRasters/nulls\_LV10m\_10km.tif"}\NormalTok{)}
\NormalTok{nulls100}\OtherTok{=}\FunctionTok{rast}\NormalTok{(}\StringTok{"./Templates/TemplateRasters/nulls\_LV100m\_10km.tif"}\NormalTok{)}

\CommentTok{\# codes {-}{-}{-}{-}}
\NormalTok{kodi}\OtherTok{=}\FunctionTok{read\_excel}\NormalTok{(}\StringTok{"./Geodata/2024/LAD/KulturuKodi\_2024.xlsx"}\NormalTok{)}
\NormalTok{kodi}\SpecialCharTok{$}\NormalTok{kods}\OtherTok{=}\FunctionTok{as.character}\NormalTok{(kodi}\SpecialCharTok{$}\NormalTok{kods)}
\CommentTok{\# LAD {-}{-}{-}{-}}
\NormalTok{lad}\OtherTok{=}\NormalTok{sfarrow}\SpecialCharTok{::}\FunctionTok{st\_read\_parquet}\NormalTok{(}\StringTok{"./Geodata/2024/LAD/Lauki\_2024.parquet"}\NormalTok{)}
\NormalTok{lad}\SpecialCharTok{$}\NormalTok{yes}\OtherTok{=}\DecValTok{1}
\NormalTok{lad}\OtherTok{=}\NormalTok{lad }\SpecialCharTok{\%\textgreater{}\%} 
 \FunctionTok{left\_join}\NormalTok{(kodi,}\AttributeTok{by=}\FunctionTok{c}\NormalTok{(}\StringTok{"PRODUCT\_CODE"}\OtherTok{=}\StringTok{"kods"}\NormalTok{))}

\CommentTok{\# simple landscape {-}{-}{-}{-}}
\NormalTok{simple\_landscape}\OtherTok{=}\FunctionTok{rast}\NormalTok{(}\StringTok{"RasterGrids\_10m/2024/Ainava\_vienk\_mask.tif"}\NormalTok{)}


\CommentTok{\# FarmlandPloughed\_CropsFallowTempGrass\_cell.tif    egv\_245 {-}{-}{-}{-}}
\NormalTok{dati}\OtherTok{=}\NormalTok{lad }\SpecialCharTok{\%\textgreater{}\%} 
 \FunctionTok{filter}\NormalTok{(SDM\_grupa\_sakums }\SpecialCharTok{\%in\%} \FunctionTok{c}\NormalTok{(}\StringTok{"aramzemes (citur neiekļautās)"}\NormalTok{,}
                 \StringTok{"aramzemes (labība{-}vasarāji)"}\NormalTok{,}
                 \StringTok{"aramzemes (labība{-}ziemāji)"}\NormalTok{,}
                 \StringTok{"aramzemes (vagu un rušināmkultūru)"}\NormalTok{,}
                 \StringTok{"aramzemes (vasaras rapsis un rispsis, kukurūzas, zirņi un pupas, soja, kaņepes)"}\NormalTok{,}
                 \StringTok{"aramzemes (ziemas rapsis un ripsis)"}\NormalTok{,}
                 \StringTok{"papuves"}\NormalTok{,}
                 \StringTok{"zālāji (kultivētie)"}\NormalTok{))}
\FunctionTok{table}\NormalTok{(dati}\SpecialCharTok{$}\NormalTok{SDM\_grupa\_sakums,}\AttributeTok{useNA=}\StringTok{"always"}\NormalTok{)}

\NormalTok{p2i\_rez}\OtherTok{=}\NormalTok{egvtools}\SpecialCharTok{::}\FunctionTok{polygon2input}\NormalTok{(}\AttributeTok{vector\_data =}\NormalTok{ dati,}
                \AttributeTok{template\_path =} \StringTok{"./Templates/TemplateRasters/LV10m\_10km.tif"}\NormalTok{,}
                \AttributeTok{out\_path =} \StringTok{"./RasterGrids\_10m/2024/"}\NormalTok{,}
                \AttributeTok{file\_name =} \StringTok{"FarmlandPloughed\_CropsFallowTempGrass\_input.tif"}\NormalTok{,}
                \AttributeTok{value\_field =} \StringTok{"yes"}\NormalTok{,}
                \AttributeTok{prepare=}\ConstantTok{FALSE}\NormalTok{,}
                \AttributeTok{background\_raster =} \StringTok{"./Templates/TemplateRasters/nulls\_LV10m\_10km.tif"}\NormalTok{,}
                \AttributeTok{plot\_result =} \ConstantTok{TRUE}\NormalTok{)}
\NormalTok{p2i\_rez}
\NormalTok{i2e\_rez}\OtherTok{=}\NormalTok{egvtools}\SpecialCharTok{::}\FunctionTok{input2egv}\NormalTok{(}\AttributeTok{input=}\FunctionTok{paste0}\NormalTok{(}\StringTok{"./RasterGrids\_10m/2024/"}\NormalTok{,}
                     \StringTok{"FarmlandPloughed\_CropsFallowTempGrass\_input.tif"}\NormalTok{),}
              \AttributeTok{egv\_template=} \StringTok{"./Templates/TemplateRasters/LV100m\_10km.tif"}\NormalTok{,}
              \AttributeTok{summary\_function =} \StringTok{"average"}\NormalTok{,}
              \AttributeTok{missing\_job =} \StringTok{"FillOutput"}\NormalTok{,}
              \AttributeTok{outlocation =} \StringTok{"./RasterGrids\_100m/2024/RAW/"}\NormalTok{,}
              \AttributeTok{outfilename =} \StringTok{"FarmlandPloughed\_CropsFallowTempGrass\_cell.tif"}\NormalTok{,}
              \AttributeTok{layername =} \StringTok{"egv\_245"}\NormalTok{,}
              \AttributeTok{idw\_weight =} \DecValTok{2}\NormalTok{,}
              \AttributeTok{plot\_gaps =} \ConstantTok{FALSE}\NormalTok{,}\AttributeTok{plot\_final =} \ConstantTok{TRUE}\NormalTok{)}
\NormalTok{i2e\_rez}
\FunctionTok{rm}\NormalTok{(p2i\_rez)}
\FunctionTok{rm}\NormalTok{(i2e\_rez)}
\FunctionTok{rm}\NormalTok{(dati)}
\FunctionTok{unlink}\NormalTok{(}\StringTok{"./RasterGrids\_10m/2024/FarmlandPloughed\_CropsFallowTempGrass\_input.tif"}\NormalTok{)}


\CommentTok{\# standardisation {-}{-}{-}{-}}
\ControlFlowTok{if}\NormalTok{(}\SpecialCharTok{!}\FunctionTok{require}\NormalTok{(terra)) \{}\FunctionTok{install.packages}\NormalTok{(}\StringTok{"terra"}\NormalTok{); }\FunctionTok{require}\NormalTok{(terra)\}}
\ControlFlowTok{if}\NormalTok{(}\SpecialCharTok{!}\FunctionTok{require}\NormalTok{(tidyverse)) \{}\FunctionTok{install.packages}\NormalTok{(}\StringTok{"tidyverse"}\NormalTok{); }\FunctionTok{require}\NormalTok{(tidyverse)\}}

\NormalTok{nosaukums}\OtherTok{=}\StringTok{"FarmlandPloughed\_CropsFallowTempGrass\_cell.tif"}
\NormalTok{ielasisanas\_cels}\OtherTok{=}\FunctionTok{paste0}\NormalTok{(}\StringTok{"./RasterGrids\_100m/2024/RAW/"}\NormalTok{,nosaukums)}
\NormalTok{saglabasanas\_cels}\OtherTok{=}\FunctionTok{paste0}\NormalTok{(}\StringTok{"./RasterGrids\_100m/2024/Scaled/"}\NormalTok{,nosaukums)}
\NormalTok{slanis}\OtherTok{=}\FunctionTok{rast}\NormalTok{(ielasisanas\_cels)}
\NormalTok{videjais}\OtherTok{=}\FunctionTok{global}\NormalTok{(slanis,}\AttributeTok{fun=}\StringTok{"mean"}\NormalTok{,}\AttributeTok{na.rm=}\ConstantTok{TRUE}\NormalTok{)}
\NormalTok{centrets}\OtherTok{=}\NormalTok{slanis}\SpecialCharTok{{-}}\NormalTok{videjais[,}\DecValTok{1}\NormalTok{]}
\NormalTok{standartnovirze}\OtherTok{=}\NormalTok{terra}\SpecialCharTok{::}\FunctionTok{global}\NormalTok{(centrets,}\AttributeTok{fun=}\StringTok{"rms"}\NormalTok{,}\AttributeTok{na.rm=}\ConstantTok{TRUE}\NormalTok{)}
\NormalTok{merogots}\OtherTok{=}\NormalTok{centrets}\SpecialCharTok{/}\NormalTok{standartnovirze[,}\DecValTok{1}\NormalTok{]}
\FunctionTok{writeRaster}\NormalTok{(merogots,}
      \AttributeTok{filename=}\NormalTok{saglabasanas\_cels,}
      \AttributeTok{overwrite=}\ConstantTok{TRUE}\NormalTok{)}
\end{Highlighting}
\end{Shaded}

\section{FarmlandPloughed\_CropsFallowTempGrass\_r500}\label{ch06.246}

\textbf{filename:} \texttt{FarmlandPloughed\_CropsFallowTempGrass\_r500.tif}

\textbf{layername:} \texttt{egv\_246}

\textbf{English name:} Fractional cover of Crop-, Fallow-, Temporary Grass- Lands
within the 0.5 km landscape

\textbf{Latvian name:} Aramzemju, papuvju, zālāju-aramzemē platības īpatsvars 0,5 km
ainavā

\textbf{Procedure:} The cover fraction within a radius of 500 m around the analysis grid cell is
calculated as the area-weighted sum of the \hyperref[ch06.245]{analysis cells} inside the
buffer, using the workflow \texttt{egvtools::radius\_function()}. During the calculation of the landscape metric,
inverse distance weighted (power = 2) gap filling on the output is applied
to ensure no missing values at the edges. Then the layer is rewritten to set
its name. Finally, the layer is standardised by subtracting the arithmetic
mean and dividing by the root mean squared error.

\begin{Shaded}
\begin{Highlighting}[]
\CommentTok{\# libs {-}{-}{-}{-}}
\ControlFlowTok{if}\NormalTok{(}\SpecialCharTok{!}\FunctionTok{require}\NormalTok{(terra)) \{}\FunctionTok{install.packages}\NormalTok{(}\StringTok{"terra"}\NormalTok{); }\FunctionTok{require}\NormalTok{(terra)\}}
\ControlFlowTok{if}\NormalTok{(}\SpecialCharTok{!}\FunctionTok{require}\NormalTok{(egvtools)) \{remotes}\SpecialCharTok{::}\FunctionTok{install\_github}\NormalTok{(}\StringTok{"aavotins/egvtools"}\NormalTok{); }\FunctionTok{require}\NormalTok{(egvtools)\}}


\CommentTok{\# Templates {-}{-}{-}{-}{-}}
\NormalTok{template100}\OtherTok{=}\FunctionTok{rast}\NormalTok{(}\StringTok{"./Templates/TemplateRasters/LV100m\_10km.tif"}\NormalTok{)}

\CommentTok{\# radii {-}{-}{-}{-}}
\FunctionTok{radius\_function}\NormalTok{(}
 \AttributeTok{kvadrati\_path =} \StringTok{"./Templates/TemplateGrids/tiles/"}\NormalTok{,}
 \AttributeTok{radii\_path   =} \StringTok{"./Templates/TemplateGridPoints/tiles/"}\NormalTok{,}
 \AttributeTok{tikls100\_path =} \StringTok{"./Templates/TemplateGrids/tikls100\_sauzeme.parquet"}\NormalTok{,}
 \AttributeTok{template\_path =} \StringTok{"./Templates/TemplateRasters/LV100m\_10km.tif"}\NormalTok{,}
 \AttributeTok{input\_layers  =} \FunctionTok{c}\NormalTok{(}\StringTok{"./RasterGrids\_100m/2024/RAW/FarmlandPloughed\_CropsFallowTempGrass\_cell.tif"}\NormalTok{),}
 \AttributeTok{layer\_prefixes =} \FunctionTok{c}\NormalTok{(}\StringTok{"FarmlandPloughed\_CropsFallowTempGrass"}\NormalTok{),}
 \AttributeTok{output\_dir   =} \StringTok{"./RasterGrids\_100m/2024/RAW/"}\NormalTok{,}
 \AttributeTok{n\_workers   =} \DecValTok{6}\NormalTok{,}
 \AttributeTok{radii     =} \FunctionTok{c}\NormalTok{(}\StringTok{"r500"}\NormalTok{),}
 \AttributeTok{radius\_mode  =} \StringTok{"sparse"}\NormalTok{,}
 \AttributeTok{extract\_fun  =} \StringTok{"mean"}\NormalTok{,}
 \AttributeTok{fill\_missing  =} \ConstantTok{TRUE}\NormalTok{,}
 \AttributeTok{IDW\_weight   =} \DecValTok{2}\NormalTok{,}
 \AttributeTok{future\_max\_size =} \DecValTok{40} \SpecialCharTok{*} \DecValTok{1024}\SpecialCharTok{\^{}}\DecValTok{3}\NormalTok{)}


\CommentTok{\# FarmlandPloughed\_CropsFallowTempGrass\_r500.tif    egv\_246}
\NormalTok{slanis}\OtherTok{=}\FunctionTok{rast}\NormalTok{(}\StringTok{"./RasterGrids\_100m/2024/RAW/FarmlandPloughed\_CropsFallowTempGrass\_r500.tif"}\NormalTok{)}
\FunctionTok{names}\NormalTok{(slanis)}\OtherTok{=}\StringTok{"egv\_246"}
\NormalTok{slanis2}\OtherTok{=}\FunctionTok{project}\NormalTok{(slanis,template100)}
\FunctionTok{writeRaster}\NormalTok{(slanis2,}
      \StringTok{"./RasterGrids\_100m/2024/RAW/FarmlandPloughed\_CropsFallowTempGrass\_r500.tif"}\NormalTok{,}
      \AttributeTok{overwrite=}\ConstantTok{TRUE}\NormalTok{)}

\CommentTok{\# standardisation {-}{-}{-}{-}}
\ControlFlowTok{if}\NormalTok{(}\SpecialCharTok{!}\FunctionTok{require}\NormalTok{(terra)) \{}\FunctionTok{install.packages}\NormalTok{(}\StringTok{"terra"}\NormalTok{); }\FunctionTok{require}\NormalTok{(terra)\}}
\ControlFlowTok{if}\NormalTok{(}\SpecialCharTok{!}\FunctionTok{require}\NormalTok{(tidyverse)) \{}\FunctionTok{install.packages}\NormalTok{(}\StringTok{"tidyverse"}\NormalTok{); }\FunctionTok{require}\NormalTok{(tidyverse)\}}

\NormalTok{nosaukums}\OtherTok{=}\StringTok{"FarmlandPloughed\_CropsFallowTempGrass\_r500.tif"}
\NormalTok{ielasisanas\_cels}\OtherTok{=}\FunctionTok{paste0}\NormalTok{(}\StringTok{"./RasterGrids\_100m/2024/RAW/"}\NormalTok{,nosaukums)}
\NormalTok{saglabasanas\_cels}\OtherTok{=}\FunctionTok{paste0}\NormalTok{(}\StringTok{"./RasterGrids\_100m/2024/Scaled/"}\NormalTok{,nosaukums)}
\NormalTok{slanis}\OtherTok{=}\FunctionTok{rast}\NormalTok{(ielasisanas\_cels)}
\NormalTok{videjais}\OtherTok{=}\FunctionTok{global}\NormalTok{(slanis,}\AttributeTok{fun=}\StringTok{"mean"}\NormalTok{,}\AttributeTok{na.rm=}\ConstantTok{TRUE}\NormalTok{)}
\NormalTok{centrets}\OtherTok{=}\NormalTok{slanis}\SpecialCharTok{{-}}\NormalTok{videjais[,}\DecValTok{1}\NormalTok{]}
\NormalTok{standartnovirze}\OtherTok{=}\NormalTok{terra}\SpecialCharTok{::}\FunctionTok{global}\NormalTok{(centrets,}\AttributeTok{fun=}\StringTok{"rms"}\NormalTok{,}\AttributeTok{na.rm=}\ConstantTok{TRUE}\NormalTok{)}
\NormalTok{merogots}\OtherTok{=}\NormalTok{centrets}\SpecialCharTok{/}\NormalTok{standartnovirze[,}\DecValTok{1}\NormalTok{]}
\FunctionTok{writeRaster}\NormalTok{(merogots,}
      \AttributeTok{filename=}\NormalTok{saglabasanas\_cels,}
      \AttributeTok{overwrite=}\ConstantTok{TRUE}\NormalTok{)}
\end{Highlighting}
\end{Shaded}

\section{FarmlandPloughed\_CropsFallowTempGrass\_r1250}\label{ch06.247}

\textbf{filename:} \texttt{FarmlandPloughed\_CropsFallowTempGrass\_r1250.tif}

\textbf{layername:} \texttt{egv\_247}

\textbf{English name:} Fractional cover of Crop-, Fallow-, Temporary Grass- Lands
within the 1.25 km landscape

\textbf{Latvian name:} Aramzemju, papuvju, zālāju-aramzemē platības īpatsvars 1,25 km
ainavā

\textbf{Procedure:} The cover fraction within a radius of 1250 m around the analysis grid cell
is calculated as the area-weighted sum of the \hyperref[ch06.245]{analysis cells} inside
the buffer, using the workflow \texttt{egvtools::radius\_function()}. During the calculation of the landscape
metric, inverse distance weighted (power = 2) gap filling on the output is
applied to ensure no missing values at the edges. Then the layer is
rewritten to set its name. Finally, the layer is standardised by
subtracting the arithmetic mean and dividing by the root mean squared error.

\begin{Shaded}
\begin{Highlighting}[]
\CommentTok{\# libs {-}{-}{-}{-}}
\ControlFlowTok{if}\NormalTok{(}\SpecialCharTok{!}\FunctionTok{require}\NormalTok{(terra)) \{}\FunctionTok{install.packages}\NormalTok{(}\StringTok{"terra"}\NormalTok{); }\FunctionTok{require}\NormalTok{(terra)\}}
\ControlFlowTok{if}\NormalTok{(}\SpecialCharTok{!}\FunctionTok{require}\NormalTok{(egvtools)) \{remotes}\SpecialCharTok{::}\FunctionTok{install\_github}\NormalTok{(}\StringTok{"aavotins/egvtools"}\NormalTok{); }\FunctionTok{require}\NormalTok{(egvtools)\}}


\CommentTok{\# Templates {-}{-}{-}{-}{-}}
\NormalTok{template100}\OtherTok{=}\FunctionTok{rast}\NormalTok{(}\StringTok{"./Templates/TemplateRasters/LV100m\_10km.tif"}\NormalTok{)}

\CommentTok{\# radii {-}{-}{-}{-}}
\FunctionTok{radius\_function}\NormalTok{(}
 \AttributeTok{kvadrati\_path =} \StringTok{"./Templates/TemplateGrids/tiles/"}\NormalTok{,}
 \AttributeTok{radii\_path   =} \StringTok{"./Templates/TemplateGridPoints/tiles/"}\NormalTok{,}
 \AttributeTok{tikls100\_path =} \StringTok{"./Templates/TemplateGrids/tikls100\_sauzeme.parquet"}\NormalTok{,}
 \AttributeTok{template\_path =} \StringTok{"./Templates/TemplateRasters/LV100m\_10km.tif"}\NormalTok{,}
 \AttributeTok{input\_layers  =} \FunctionTok{c}\NormalTok{(}\StringTok{"./RasterGrids\_100m/2024/RAW/FarmlandPloughed\_CropsFallowTempGrass\_cell.tif"}\NormalTok{),}
 \AttributeTok{layer\_prefixes =} \FunctionTok{c}\NormalTok{(}\StringTok{"FarmlandPloughed\_CropsFallowTempGrass"}\NormalTok{),}
 \AttributeTok{output\_dir   =} \StringTok{"./RasterGrids\_100m/2024/RAW/"}\NormalTok{,}
 \AttributeTok{n\_workers   =} \DecValTok{6}\NormalTok{,}
 \AttributeTok{radii     =} \FunctionTok{c}\NormalTok{(}\StringTok{"r1250"}\NormalTok{),}
 \AttributeTok{radius\_mode  =} \StringTok{"sparse"}\NormalTok{,}
 \AttributeTok{extract\_fun  =} \StringTok{"mean"}\NormalTok{,}
 \AttributeTok{fill\_missing  =} \ConstantTok{TRUE}\NormalTok{,}
 \AttributeTok{IDW\_weight   =} \DecValTok{2}\NormalTok{,}
 \AttributeTok{future\_max\_size =} \DecValTok{40} \SpecialCharTok{*} \DecValTok{1024}\SpecialCharTok{\^{}}\DecValTok{3}\NormalTok{)}


\CommentTok{\# FarmlandPloughed\_CropsFallowTempGrass\_r1250.tif   egv\_247}
\NormalTok{slanis}\OtherTok{=}\FunctionTok{rast}\NormalTok{(}\StringTok{"./RasterGrids\_100m/2024/RAW/FarmlandPloughed\_CropsFallowTempGrass\_r1250.tif"}\NormalTok{)}
\FunctionTok{names}\NormalTok{(slanis)}\OtherTok{=}\StringTok{"egv\_247"}
\NormalTok{slanis2}\OtherTok{=}\FunctionTok{project}\NormalTok{(slanis,template100)}
\FunctionTok{writeRaster}\NormalTok{(slanis2,}
      \StringTok{"./RasterGrids\_100m/2024/RAW/FarmlandPloughed\_CropsFallowTempGrass\_r1250.tif"}\NormalTok{,}
      \AttributeTok{overwrite=}\ConstantTok{TRUE}\NormalTok{)}

\CommentTok{\# standardisation {-}{-}{-}{-}}
\ControlFlowTok{if}\NormalTok{(}\SpecialCharTok{!}\FunctionTok{require}\NormalTok{(terra)) \{}\FunctionTok{install.packages}\NormalTok{(}\StringTok{"terra"}\NormalTok{); }\FunctionTok{require}\NormalTok{(terra)\}}
\ControlFlowTok{if}\NormalTok{(}\SpecialCharTok{!}\FunctionTok{require}\NormalTok{(tidyverse)) \{}\FunctionTok{install.packages}\NormalTok{(}\StringTok{"tidyverse"}\NormalTok{); }\FunctionTok{require}\NormalTok{(tidyverse)\}}

\NormalTok{nosaukums}\OtherTok{=}\StringTok{"FarmlandPloughed\_CropsFallowTempGrass\_r1250.tif"}
\NormalTok{ielasisanas\_cels}\OtherTok{=}\FunctionTok{paste0}\NormalTok{(}\StringTok{"./RasterGrids\_100m/2024/RAW/"}\NormalTok{,nosaukums)}
\NormalTok{saglabasanas\_cels}\OtherTok{=}\FunctionTok{paste0}\NormalTok{(}\StringTok{"./RasterGrids\_100m/2024/Scaled/"}\NormalTok{,nosaukums)}
\NormalTok{slanis}\OtherTok{=}\FunctionTok{rast}\NormalTok{(ielasisanas\_cels)}
\NormalTok{videjais}\OtherTok{=}\FunctionTok{global}\NormalTok{(slanis,}\AttributeTok{fun=}\StringTok{"mean"}\NormalTok{,}\AttributeTok{na.rm=}\ConstantTok{TRUE}\NormalTok{)}
\NormalTok{centrets}\OtherTok{=}\NormalTok{slanis}\SpecialCharTok{{-}}\NormalTok{videjais[,}\DecValTok{1}\NormalTok{]}
\NormalTok{standartnovirze}\OtherTok{=}\NormalTok{terra}\SpecialCharTok{::}\FunctionTok{global}\NormalTok{(centrets,}\AttributeTok{fun=}\StringTok{"rms"}\NormalTok{,}\AttributeTok{na.rm=}\ConstantTok{TRUE}\NormalTok{)}
\NormalTok{merogots}\OtherTok{=}\NormalTok{centrets}\SpecialCharTok{/}\NormalTok{standartnovirze[,}\DecValTok{1}\NormalTok{]}
\FunctionTok{writeRaster}\NormalTok{(merogots,}
      \AttributeTok{filename=}\NormalTok{saglabasanas\_cels,}
      \AttributeTok{overwrite=}\ConstantTok{TRUE}\NormalTok{)}
\end{Highlighting}
\end{Shaded}

\section{FarmlandPloughed\_CropsFallowTempGrass\_r3000}\label{ch06.248}

\textbf{filename:} \texttt{FarmlandPloughed\_CropsFallowTempGrass\_r3000.tif}

\textbf{layername:} \texttt{egv\_248}

\textbf{English name:} Fractional cover of Crop-, Fallow-, Temporary Grass- Lands
within the 3 km landscape

\textbf{Latvian name:} Aramzemju, papuvju, zālāju-aramzemē platības īpatsvars 3 km
ainavā

\textbf{Procedure:} The cover fraction within a radius of 3000 m around the analysis grid cell
is calculated as the area-weighted sum of the \hyperref[ch06.245]{analysis cells} inside
the buffer, using the workflow \texttt{egvtools::radius\_function()}. During the calculation of the landscape
metric, inverse distance weighted (power = 2) gap filling on the output is
applied to ensure no missing values at the edges. Then the layer is
rewritten to set its name. Finally, the layer is standardised by
subtracting the arithmetic mean and dividing by the root mean squared error.

\begin{Shaded}
\begin{Highlighting}[]
\CommentTok{\# libs {-}{-}{-}{-}}
\ControlFlowTok{if}\NormalTok{(}\SpecialCharTok{!}\FunctionTok{require}\NormalTok{(terra)) \{}\FunctionTok{install.packages}\NormalTok{(}\StringTok{"terra"}\NormalTok{); }\FunctionTok{require}\NormalTok{(terra)\}}
\ControlFlowTok{if}\NormalTok{(}\SpecialCharTok{!}\FunctionTok{require}\NormalTok{(egvtools)) \{remotes}\SpecialCharTok{::}\FunctionTok{install\_github}\NormalTok{(}\StringTok{"aavotins/egvtools"}\NormalTok{); }\FunctionTok{require}\NormalTok{(egvtools)\}}


\CommentTok{\# Templates {-}{-}{-}{-}{-}}
\NormalTok{template100}\OtherTok{=}\FunctionTok{rast}\NormalTok{(}\StringTok{"./Templates/TemplateRasters/LV100m\_10km.tif"}\NormalTok{)}

\CommentTok{\# radii {-}{-}{-}{-}}
\FunctionTok{radius\_function}\NormalTok{(}
 \AttributeTok{kvadrati\_path =} \StringTok{"./Templates/TemplateGrids/tiles/"}\NormalTok{,}
 \AttributeTok{radii\_path   =} \StringTok{"./Templates/TemplateGridPoints/tiles/"}\NormalTok{,}
 \AttributeTok{tikls100\_path =} \StringTok{"./Templates/TemplateGrids/tikls100\_sauzeme.parquet"}\NormalTok{,}
 \AttributeTok{template\_path =} \StringTok{"./Templates/TemplateRasters/LV100m\_10km.tif"}\NormalTok{,}
 \AttributeTok{input\_layers  =} \FunctionTok{c}\NormalTok{(}\StringTok{"./RasterGrids\_100m/2024/RAW/FarmlandPloughed\_CropsFallowTempGrass\_cell.tif"}\NormalTok{),}
 \AttributeTok{layer\_prefixes =} \FunctionTok{c}\NormalTok{(}\StringTok{"FarmlandPloughed\_CropsFallowTempGrass"}\NormalTok{),}
 \AttributeTok{output\_dir   =} \StringTok{"./RasterGrids\_100m/2024/RAW/"}\NormalTok{,}
 \AttributeTok{n\_workers   =} \DecValTok{6}\NormalTok{,}
 \AttributeTok{radii     =} \FunctionTok{c}\NormalTok{(}\StringTok{"r3000"}\NormalTok{),}
 \AttributeTok{radius\_mode  =} \StringTok{"sparse"}\NormalTok{,}
 \AttributeTok{extract\_fun  =} \StringTok{"mean"}\NormalTok{,}
 \AttributeTok{fill\_missing  =} \ConstantTok{TRUE}\NormalTok{,}
 \AttributeTok{IDW\_weight   =} \DecValTok{2}\NormalTok{,}
 \AttributeTok{future\_max\_size =} \DecValTok{40} \SpecialCharTok{*} \DecValTok{1024}\SpecialCharTok{\^{}}\DecValTok{3}\NormalTok{)}


\CommentTok{\# FarmlandPloughed\_CropsFallowTempGrass\_r3000.tif   egv\_248}
\NormalTok{slanis}\OtherTok{=}\FunctionTok{rast}\NormalTok{(}\StringTok{"./RasterGrids\_100m/2024/RAW/FarmlandPloughed\_CropsFallowTempGrass\_r3000.tif"}\NormalTok{)}
\FunctionTok{names}\NormalTok{(slanis)}\OtherTok{=}\StringTok{"egv\_248"}
\NormalTok{slanis2}\OtherTok{=}\FunctionTok{project}\NormalTok{(slanis,template100)}
\FunctionTok{writeRaster}\NormalTok{(slanis2,}
      \StringTok{"./RasterGrids\_100m/2024/RAW/FarmlandPloughed\_CropsFallowTempGrass\_r3000.tif"}\NormalTok{,}
      \AttributeTok{overwrite=}\ConstantTok{TRUE}\NormalTok{)}

\CommentTok{\# standardisation {-}{-}{-}{-}}
\ControlFlowTok{if}\NormalTok{(}\SpecialCharTok{!}\FunctionTok{require}\NormalTok{(terra)) \{}\FunctionTok{install.packages}\NormalTok{(}\StringTok{"terra"}\NormalTok{); }\FunctionTok{require}\NormalTok{(terra)\}}
\ControlFlowTok{if}\NormalTok{(}\SpecialCharTok{!}\FunctionTok{require}\NormalTok{(tidyverse)) \{}\FunctionTok{install.packages}\NormalTok{(}\StringTok{"tidyverse"}\NormalTok{); }\FunctionTok{require}\NormalTok{(tidyverse)\}}

\NormalTok{nosaukums}\OtherTok{=}\StringTok{"FarmlandPloughed\_CropsFallowTempGrass\_r3000.tif"}
\NormalTok{ielasisanas\_cels}\OtherTok{=}\FunctionTok{paste0}\NormalTok{(}\StringTok{"./RasterGrids\_100m/2024/RAW/"}\NormalTok{,nosaukums)}
\NormalTok{saglabasanas\_cels}\OtherTok{=}\FunctionTok{paste0}\NormalTok{(}\StringTok{"./RasterGrids\_100m/2024/Scaled/"}\NormalTok{,nosaukums)}
\NormalTok{slanis}\OtherTok{=}\FunctionTok{rast}\NormalTok{(ielasisanas\_cels)}
\NormalTok{videjais}\OtherTok{=}\FunctionTok{global}\NormalTok{(slanis,}\AttributeTok{fun=}\StringTok{"mean"}\NormalTok{,}\AttributeTok{na.rm=}\ConstantTok{TRUE}\NormalTok{)}
\NormalTok{centrets}\OtherTok{=}\NormalTok{slanis}\SpecialCharTok{{-}}\NormalTok{videjais[,}\DecValTok{1}\NormalTok{]}
\NormalTok{standartnovirze}\OtherTok{=}\NormalTok{terra}\SpecialCharTok{::}\FunctionTok{global}\NormalTok{(centrets,}\AttributeTok{fun=}\StringTok{"rms"}\NormalTok{,}\AttributeTok{na.rm=}\ConstantTok{TRUE}\NormalTok{)}
\NormalTok{merogots}\OtherTok{=}\NormalTok{centrets}\SpecialCharTok{/}\NormalTok{standartnovirze[,}\DecValTok{1}\NormalTok{]}
\FunctionTok{writeRaster}\NormalTok{(merogots,}
      \AttributeTok{filename=}\NormalTok{saglabasanas\_cels,}
      \AttributeTok{overwrite=}\ConstantTok{TRUE}\NormalTok{)}
\end{Highlighting}
\end{Shaded}

\section{FarmlandPloughed\_CropsFallowTempGrass\_r10000}\label{ch06.249}

\textbf{filename:} \texttt{FarmlandPloughed\_CropsFallowTempGrass\_r10000.tif}

\textbf{layername:} \texttt{egv\_249}

\textbf{English name:} Fractional cover of Crop-, Fallow-, Temporary Grass- Lands
within the 10 km landscape

\textbf{Latvian name:} Aramzemju, papuvju, zālāju-aramzemē platības īpatsvars 10 km
ainavā

\textbf{Procedure:} The cover fraction within a radius of 10000 m around the analysis grid cell
is calculated as the area-weighted sum of the \hyperref[ch06.245]{analysis cells} inside
the buffer, using the workflow \texttt{egvtools::radius\_function()}. During the calculation of the landscape
metric, inverse distance weighted (power = 2) gap filling on the output is
applied to ensure no missing values at the edges. Then the layer is
rewritten to set its name. Finally, the layer is standardised by
subtracting the arithmetic mean and dividing by the root mean squared error.

\begin{Shaded}
\begin{Highlighting}[]
\CommentTok{\# libs {-}{-}{-}{-}}
\ControlFlowTok{if}\NormalTok{(}\SpecialCharTok{!}\FunctionTok{require}\NormalTok{(terra)) \{}\FunctionTok{install.packages}\NormalTok{(}\StringTok{"terra"}\NormalTok{); }\FunctionTok{require}\NormalTok{(terra)\}}
\ControlFlowTok{if}\NormalTok{(}\SpecialCharTok{!}\FunctionTok{require}\NormalTok{(egvtools)) \{remotes}\SpecialCharTok{::}\FunctionTok{install\_github}\NormalTok{(}\StringTok{"aavotins/egvtools"}\NormalTok{); }\FunctionTok{require}\NormalTok{(egvtools)\}}


\CommentTok{\# Templates {-}{-}{-}{-}{-}}
\NormalTok{template100}\OtherTok{=}\FunctionTok{rast}\NormalTok{(}\StringTok{"./Templates/TemplateRasters/LV100m\_10km.tif"}\NormalTok{)}

\CommentTok{\# radii {-}{-}{-}{-}}
\FunctionTok{radius\_function}\NormalTok{(}
 \AttributeTok{kvadrati\_path =} \StringTok{"./Templates/TemplateGrids/tiles/"}\NormalTok{,}
 \AttributeTok{radii\_path   =} \StringTok{"./Templates/TemplateGridPoints/tiles/"}\NormalTok{,}
 \AttributeTok{tikls100\_path =} \StringTok{"./Templates/TemplateGrids/tikls100\_sauzeme.parquet"}\NormalTok{,}
 \AttributeTok{template\_path =} \StringTok{"./Templates/TemplateRasters/LV100m\_10km.tif"}\NormalTok{,}
 \AttributeTok{input\_layers  =} \FunctionTok{c}\NormalTok{(}\StringTok{"./RasterGrids\_100m/2024/RAW/FarmlandPloughed\_CropsFallowTempGrass\_cell.tif"}\NormalTok{),}
 \AttributeTok{layer\_prefixes =} \FunctionTok{c}\NormalTok{(}\StringTok{"FarmlandPloughed\_CropsFallowTempGrass"}\NormalTok{),}
 \AttributeTok{output\_dir   =} \StringTok{"./RasterGrids\_100m/2024/RAW/"}\NormalTok{,}
 \AttributeTok{n\_workers   =} \DecValTok{6}\NormalTok{,}
 \AttributeTok{radii     =} \FunctionTok{c}\NormalTok{(}\StringTok{"r10000"}\NormalTok{),}
 \AttributeTok{radius\_mode  =} \StringTok{"sparse"}\NormalTok{,}
 \AttributeTok{extract\_fun  =} \StringTok{"mean"}\NormalTok{,}
 \AttributeTok{fill\_missing  =} \ConstantTok{TRUE}\NormalTok{,}
 \AttributeTok{IDW\_weight   =} \DecValTok{2}\NormalTok{,}
 \AttributeTok{future\_max\_size =} \DecValTok{40} \SpecialCharTok{*} \DecValTok{1024}\SpecialCharTok{\^{}}\DecValTok{3}\NormalTok{)}


\CommentTok{\# FarmlandPloughed\_CropsFallowTempGrass\_r10000.tif  egv\_249}
\NormalTok{slanis}\OtherTok{=}\FunctionTok{rast}\NormalTok{(}\StringTok{"./RasterGrids\_100m/2024/RAW/FarmlandPloughed\_CropsFallowTempGrass\_r10000.tif"}\NormalTok{)}
\FunctionTok{names}\NormalTok{(slanis)}\OtherTok{=}\StringTok{"egv\_249"}
\NormalTok{slanis2}\OtherTok{=}\FunctionTok{project}\NormalTok{(slanis,template100)}
\FunctionTok{writeRaster}\NormalTok{(slanis2,}
      \StringTok{"./RasterGrids\_100m/2024/RAW/FarmlandPloughed\_CropsFallowTempGrass\_r10000.tif"}\NormalTok{,}
      \AttributeTok{overwrite=}\ConstantTok{TRUE}\NormalTok{)}

\CommentTok{\# standardisation {-}{-}{-}{-}}
\ControlFlowTok{if}\NormalTok{(}\SpecialCharTok{!}\FunctionTok{require}\NormalTok{(terra)) \{}\FunctionTok{install.packages}\NormalTok{(}\StringTok{"terra"}\NormalTok{); }\FunctionTok{require}\NormalTok{(terra)\}}
\ControlFlowTok{if}\NormalTok{(}\SpecialCharTok{!}\FunctionTok{require}\NormalTok{(tidyverse)) \{}\FunctionTok{install.packages}\NormalTok{(}\StringTok{"tidyverse"}\NormalTok{); }\FunctionTok{require}\NormalTok{(tidyverse)\}}

\NormalTok{nosaukums}\OtherTok{=}\StringTok{"FarmlandPloughed\_CropsFallowTempGrass\_r10000.tif"}
\NormalTok{ielasisanas\_cels}\OtherTok{=}\FunctionTok{paste0}\NormalTok{(}\StringTok{"./RasterGrids\_100m/2024/RAW/"}\NormalTok{,nosaukums)}
\NormalTok{saglabasanas\_cels}\OtherTok{=}\FunctionTok{paste0}\NormalTok{(}\StringTok{"./RasterGrids\_100m/2024/Scaled/"}\NormalTok{,nosaukums)}
\NormalTok{slanis}\OtherTok{=}\FunctionTok{rast}\NormalTok{(ielasisanas\_cels)}
\NormalTok{videjais}\OtherTok{=}\FunctionTok{global}\NormalTok{(slanis,}\AttributeTok{fun=}\StringTok{"mean"}\NormalTok{,}\AttributeTok{na.rm=}\ConstantTok{TRUE}\NormalTok{)}
\NormalTok{centrets}\OtherTok{=}\NormalTok{slanis}\SpecialCharTok{{-}}\NormalTok{videjais[,}\DecValTok{1}\NormalTok{]}
\NormalTok{standartnovirze}\OtherTok{=}\NormalTok{terra}\SpecialCharTok{::}\FunctionTok{global}\NormalTok{(centrets,}\AttributeTok{fun=}\StringTok{"rms"}\NormalTok{,}\AttributeTok{na.rm=}\ConstantTok{TRUE}\NormalTok{)}
\NormalTok{merogots}\OtherTok{=}\NormalTok{centrets}\SpecialCharTok{/}\NormalTok{standartnovirze[,}\DecValTok{1}\NormalTok{]}
\FunctionTok{writeRaster}\NormalTok{(merogots,}
      \AttributeTok{filename=}\NormalTok{saglabasanas\_cels,}
      \AttributeTok{overwrite=}\ConstantTok{TRUE}\NormalTok{)}
\end{Highlighting}
\end{Shaded}

\section{FarmlandPloughed\_Fallow\_cell}\label{ch06.250}

\textbf{filename:} \texttt{FarmlandPloughed\_Fallow\_cell.tif}

\textbf{layername:} \texttt{egv\_250}

\textbf{English name:} Fractional cover of Fallow Land within the analysis cell (1
ha)

\textbf{Latvian name:} Papuvju platības īpatsvars analīzes šūnā (1 ha)

\textbf{Procedure:} First, agricultural parcels declared as fallow land are selected
from the \hyperref[Ch04.02]{Rural Support Service's information on declared fields}.
Geometries are then rasterised to input resolution, ensuring value 1 at the
polygon locations and value 0 elsewhere. Rasterisation is
performed using the workflow \texttt{egvtools::polygon2input()}. Once rasterised, the
layer is aggregated to EGV resolution using the workflow \texttt{egvtools::input2egv()},
which calculates the arithmetic mean and thus
results in a cover fraction. During aggregation, inverse distance weighted
(power = 2) gap filling on the output is applied to ensure no missing
values at the edges. Finally, the layer is standardised by subtracting
the arithmetic mean and dividing by the root mean squared error.

\begin{Shaded}
\begin{Highlighting}[]
\CommentTok{\# libs {-}{-}{-}{-}}
\ControlFlowTok{if}\NormalTok{(}\SpecialCharTok{!}\FunctionTok{require}\NormalTok{(egvtools)) \{remotes}\SpecialCharTok{::}\FunctionTok{install\_github}\NormalTok{(}\StringTok{"aavotins/egvtools"}\NormalTok{); }\FunctionTok{require}\NormalTok{(egvtools)\}}
\ControlFlowTok{if}\NormalTok{(}\SpecialCharTok{!}\FunctionTok{require}\NormalTok{(terra)) \{}\FunctionTok{install.packages}\NormalTok{(}\StringTok{"terra"}\NormalTok{); }\FunctionTok{require}\NormalTok{(terra)\}}
\ControlFlowTok{if}\NormalTok{(}\SpecialCharTok{!}\FunctionTok{require}\NormalTok{(sf)) \{}\FunctionTok{install.packages}\NormalTok{(}\StringTok{"sf"}\NormalTok{); }\FunctionTok{require}\NormalTok{(sf)\}}
\ControlFlowTok{if}\NormalTok{(}\SpecialCharTok{!}\FunctionTok{require}\NormalTok{(tidyverse)) \{}\FunctionTok{install.packages}\NormalTok{(}\StringTok{"tidyverse"}\NormalTok{); }\FunctionTok{require}\NormalTok{(tidyverse)\}}
\ControlFlowTok{if}\NormalTok{(}\SpecialCharTok{!}\FunctionTok{require}\NormalTok{(sfarrow)) \{}\FunctionTok{install.packages}\NormalTok{(}\StringTok{"sfarrow"}\NormalTok{); }\FunctionTok{require}\NormalTok{(sfarrow)\}}
\ControlFlowTok{if}\NormalTok{(}\SpecialCharTok{!}\FunctionTok{require}\NormalTok{(readxl)) \{}\FunctionTok{install.packages}\NormalTok{(}\StringTok{"readxl"}\NormalTok{); }\FunctionTok{require}\NormalTok{(readxl)\}}
\ControlFlowTok{if}\NormalTok{(}\SpecialCharTok{!}\FunctionTok{require}\NormalTok{(raster)) \{}\FunctionTok{install.packages}\NormalTok{(}\StringTok{"raster"}\NormalTok{); }\FunctionTok{require}\NormalTok{(raster)\}}
\ControlFlowTok{if}\NormalTok{(}\SpecialCharTok{!}\FunctionTok{require}\NormalTok{(fasterize)) \{}\FunctionTok{install.packages}\NormalTok{(}\StringTok{"fasterize"}\NormalTok{); }\FunctionTok{require}\NormalTok{(fasterize)\}}

\CommentTok{\# templates {-}{-}{-}{-}}
\NormalTok{template100}\OtherTok{=}\FunctionTok{rast}\NormalTok{(}\StringTok{"./Templates/TemplateRasters/LV100m\_10km.tif"}\NormalTok{)}
\NormalTok{template10}\OtherTok{=}\FunctionTok{rast}\NormalTok{(}\StringTok{"./Templates/TemplateRasters/LV10m\_10km.tif"}\NormalTok{)}
\NormalTok{rastrs10}\OtherTok{=}\FunctionTok{raster}\NormalTok{(template10)}

\NormalTok{nulls10}\OtherTok{=}\FunctionTok{rast}\NormalTok{(}\StringTok{"./Templates/TemplateRasters/nulls\_LV10m\_10km.tif"}\NormalTok{)}
\NormalTok{nulls100}\OtherTok{=}\FunctionTok{rast}\NormalTok{(}\StringTok{"./Templates/TemplateRasters/nulls\_LV100m\_10km.tif"}\NormalTok{)}

\CommentTok{\# codes {-}{-}{-}{-}}
\NormalTok{kodi}\OtherTok{=}\FunctionTok{read\_excel}\NormalTok{(}\StringTok{"./Geodata/2024/LAD/KulturuKodi\_2024.xlsx"}\NormalTok{)}
\NormalTok{kodi}\SpecialCharTok{$}\NormalTok{kods}\OtherTok{=}\FunctionTok{as.character}\NormalTok{(kodi}\SpecialCharTok{$}\NormalTok{kods)}
\CommentTok{\# LAD {-}{-}{-}{-}}
\NormalTok{lad}\OtherTok{=}\NormalTok{sfarrow}\SpecialCharTok{::}\FunctionTok{st\_read\_parquet}\NormalTok{(}\StringTok{"./Geodata/2024/LAD/Lauki\_2024.parquet"}\NormalTok{)}
\NormalTok{lad}\SpecialCharTok{$}\NormalTok{yes}\OtherTok{=}\DecValTok{1}
\NormalTok{lad}\OtherTok{=}\NormalTok{lad }\SpecialCharTok{\%\textgreater{}\%} 
 \FunctionTok{left\_join}\NormalTok{(kodi,}\AttributeTok{by=}\FunctionTok{c}\NormalTok{(}\StringTok{"PRODUCT\_CODE"}\OtherTok{=}\StringTok{"kods"}\NormalTok{))}

\CommentTok{\# simple landscape {-}{-}{-}{-}}
\NormalTok{simple\_landscape}\OtherTok{=}\FunctionTok{rast}\NormalTok{(}\StringTok{"RasterGrids\_10m/2024/Ainava\_vienk\_mask.tif"}\NormalTok{)}


\CommentTok{\# FarmlandPloughed\_Fallow\_cell.tif  egv\_250 {-}{-}{-}{-}}
\NormalTok{dati}\OtherTok{=}\NormalTok{lad }\SpecialCharTok{\%\textgreater{}\%} 
 \FunctionTok{filter}\NormalTok{(SDM\_grupa\_sakums }\SpecialCharTok{==} \StringTok{"papuves"}\NormalTok{)}
\FunctionTok{table}\NormalTok{(dati}\SpecialCharTok{$}\NormalTok{SDM\_grupa\_sakums,}\AttributeTok{useNA=}\StringTok{"always"}\NormalTok{)}

\NormalTok{p2i\_rez}\OtherTok{=}\NormalTok{egvtools}\SpecialCharTok{::}\FunctionTok{polygon2input}\NormalTok{(}\AttributeTok{vector\_data =}\NormalTok{ dati,}
                \AttributeTok{template\_path =} \StringTok{"./Templates/TemplateRasters/LV10m\_10km.tif"}\NormalTok{,}
                \AttributeTok{out\_path =} \StringTok{"./RasterGrids\_10m/2024/"}\NormalTok{,}
                \AttributeTok{file\_name =} \StringTok{"FarmlandPloughed\_Fallow\_input.tif"}\NormalTok{,}
                \AttributeTok{value\_field =} \StringTok{"yes"}\NormalTok{,}
                \AttributeTok{prepare=}\ConstantTok{FALSE}\NormalTok{,}
                \AttributeTok{background\_raster =} \StringTok{"./Templates/TemplateRasters/nulls\_LV10m\_10km.tif"}\NormalTok{,}
                \AttributeTok{plot\_result =} \ConstantTok{TRUE}\NormalTok{)}
\NormalTok{p2i\_rez}
\NormalTok{i2e\_rez}\OtherTok{=}\NormalTok{egvtools}\SpecialCharTok{::}\FunctionTok{input2egv}\NormalTok{(}\AttributeTok{input=}\FunctionTok{paste0}\NormalTok{(}\StringTok{"./RasterGrids\_10m/2024/"}\NormalTok{,}
                     \StringTok{"FarmlandPloughed\_Fallow\_input.tif"}\NormalTok{),}
              \AttributeTok{egv\_template=} \StringTok{"./Templates/TemplateRasters/LV100m\_10km.tif"}\NormalTok{,}
              \AttributeTok{summary\_function =} \StringTok{"average"}\NormalTok{,}
              \AttributeTok{missing\_job =} \StringTok{"FillOutput"}\NormalTok{,}
              \AttributeTok{outlocation =} \StringTok{"./RasterGrids\_100m/2024/RAW/"}\NormalTok{,}
              \AttributeTok{outfilename =} \StringTok{"FarmlandPloughed\_Fallow\_cell.tif"}\NormalTok{,}
              \AttributeTok{layername =} \StringTok{"egv\_250"}\NormalTok{,}
              \AttributeTok{idw\_weight =} \DecValTok{2}\NormalTok{,}
              \AttributeTok{plot\_gaps =} \ConstantTok{FALSE}\NormalTok{,}\AttributeTok{plot\_final =} \ConstantTok{TRUE}\NormalTok{)}
\NormalTok{i2e\_rez}
\FunctionTok{rm}\NormalTok{(p2i\_rez)}
\FunctionTok{rm}\NormalTok{(i2e\_rez)}
\FunctionTok{rm}\NormalTok{(dati)}
\FunctionTok{unlink}\NormalTok{(}\StringTok{"./RasterGrids\_10m/2024/FarmlandPloughed\_Fallow\_input.tif"}\NormalTok{)}

\CommentTok{\# standardisation {-}{-}{-}{-}}
\ControlFlowTok{if}\NormalTok{(}\SpecialCharTok{!}\FunctionTok{require}\NormalTok{(terra)) \{}\FunctionTok{install.packages}\NormalTok{(}\StringTok{"terra"}\NormalTok{); }\FunctionTok{require}\NormalTok{(terra)\}}
\ControlFlowTok{if}\NormalTok{(}\SpecialCharTok{!}\FunctionTok{require}\NormalTok{(tidyverse)) \{}\FunctionTok{install.packages}\NormalTok{(}\StringTok{"tidyverse"}\NormalTok{); }\FunctionTok{require}\NormalTok{(tidyverse)\}}

\NormalTok{nosaukums}\OtherTok{=}\StringTok{"FarmlandPloughed\_Fallow\_cell.tif"}
\NormalTok{ielasisanas\_cels}\OtherTok{=}\FunctionTok{paste0}\NormalTok{(}\StringTok{"./RasterGrids\_100m/2024/RAW/"}\NormalTok{,nosaukums)}
\NormalTok{saglabasanas\_cels}\OtherTok{=}\FunctionTok{paste0}\NormalTok{(}\StringTok{"./RasterGrids\_100m/2024/Scaled/"}\NormalTok{,nosaukums)}
\NormalTok{slanis}\OtherTok{=}\FunctionTok{rast}\NormalTok{(ielasisanas\_cels)}
\NormalTok{videjais}\OtherTok{=}\FunctionTok{global}\NormalTok{(slanis,}\AttributeTok{fun=}\StringTok{"mean"}\NormalTok{,}\AttributeTok{na.rm=}\ConstantTok{TRUE}\NormalTok{)}
\NormalTok{centrets}\OtherTok{=}\NormalTok{slanis}\SpecialCharTok{{-}}\NormalTok{videjais[,}\DecValTok{1}\NormalTok{]}
\NormalTok{standartnovirze}\OtherTok{=}\NormalTok{terra}\SpecialCharTok{::}\FunctionTok{global}\NormalTok{(centrets,}\AttributeTok{fun=}\StringTok{"rms"}\NormalTok{,}\AttributeTok{na.rm=}\ConstantTok{TRUE}\NormalTok{)}
\NormalTok{merogots}\OtherTok{=}\NormalTok{centrets}\SpecialCharTok{/}\NormalTok{standartnovirze[,}\DecValTok{1}\NormalTok{]}
\FunctionTok{writeRaster}\NormalTok{(merogots,}
      \AttributeTok{filename=}\NormalTok{saglabasanas\_cels,}
      \AttributeTok{overwrite=}\ConstantTok{TRUE}\NormalTok{)}
\end{Highlighting}
\end{Shaded}

\section{FarmlandPloughed\_Fallow\_r500}\label{ch06.251}

\textbf{filename:} \texttt{FarmlandPloughed\_Fallow\_r500.tif}

\textbf{layername:} \texttt{egv\_251}

\textbf{English name:} Fractional cover of Fallow Land within the 0.5 km landscape

\textbf{Latvian name:} Papuvju platības īpatsvars 0,5 km ainavā

\textbf{Procedure:} The cover fraction within a radius of 500 m around the analysis grid cell is
calculated as the area-weighted sum of the \hyperref[ch06.250]{analysis cells} inside the
buffer, using the workflow \texttt{egvtools::radius\_function()}. During the calculation of the landscape metric,
inverse distance weighted (power = 2) gap filling on the output is applied
to ensure no missing values at the edges. Then the layer is rewritten to set
its name. Finally, the layer is standardised by subtracting the arithmetic
mean and dividing by the root mean squared error.

\begin{Shaded}
\begin{Highlighting}[]
\CommentTok{\# libs {-}{-}{-}{-}}
\ControlFlowTok{if}\NormalTok{(}\SpecialCharTok{!}\FunctionTok{require}\NormalTok{(terra)) \{}\FunctionTok{install.packages}\NormalTok{(}\StringTok{"terra"}\NormalTok{); }\FunctionTok{require}\NormalTok{(terra)\}}
\ControlFlowTok{if}\NormalTok{(}\SpecialCharTok{!}\FunctionTok{require}\NormalTok{(egvtools)) \{remotes}\SpecialCharTok{::}\FunctionTok{install\_github}\NormalTok{(}\StringTok{"aavotins/egvtools"}\NormalTok{); }\FunctionTok{require}\NormalTok{(egvtools)\}}


\CommentTok{\# Templates {-}{-}{-}{-}{-}}
\NormalTok{template100}\OtherTok{=}\FunctionTok{rast}\NormalTok{(}\StringTok{"./Templates/TemplateRasters/LV100m\_10km.tif"}\NormalTok{)}

\CommentTok{\# radii {-}{-}{-}{-}}
\FunctionTok{radius\_function}\NormalTok{(}
 \AttributeTok{kvadrati\_path =} \StringTok{"./Templates/TemplateGrids/tiles/"}\NormalTok{,}
 \AttributeTok{radii\_path   =} \StringTok{"./Templates/TemplateGridPoints/tiles/"}\NormalTok{,}
 \AttributeTok{tikls100\_path =} \StringTok{"./Templates/TemplateGrids/tikls100\_sauzeme.parquet"}\NormalTok{,}
 \AttributeTok{template\_path =} \StringTok{"./Templates/TemplateRasters/LV100m\_10km.tif"}\NormalTok{,}
 \AttributeTok{input\_layers  =} \FunctionTok{c}\NormalTok{(}\StringTok{"./RasterGrids\_100m/2024/RAW/FarmlandPloughed\_Fallow\_cell.tif"}\NormalTok{),}
 \AttributeTok{layer\_prefixes =} \FunctionTok{c}\NormalTok{(}\StringTok{"FarmlandPloughed\_Fallow"}\NormalTok{),}
 \AttributeTok{output\_dir   =} \StringTok{"./RasterGrids\_100m/2024/RAW/"}\NormalTok{,}
 \AttributeTok{n\_workers   =} \DecValTok{6}\NormalTok{,}
 \AttributeTok{radii     =} \FunctionTok{c}\NormalTok{(}\StringTok{"r500"}\NormalTok{),}
 \AttributeTok{radius\_mode  =} \StringTok{"sparse"}\NormalTok{,}
 \AttributeTok{extract\_fun  =} \StringTok{"mean"}\NormalTok{,}
 \AttributeTok{fill\_missing  =} \ConstantTok{TRUE}\NormalTok{,}
 \AttributeTok{IDW\_weight   =} \DecValTok{2}\NormalTok{,}
 \AttributeTok{future\_max\_size =} \DecValTok{40} \SpecialCharTok{*} \DecValTok{1024}\SpecialCharTok{\^{}}\DecValTok{3}\NormalTok{)}


\CommentTok{\# FarmlandPloughed\_Fallow\_r500.tif  egv\_251}
\NormalTok{slanis}\OtherTok{=}\FunctionTok{rast}\NormalTok{(}\StringTok{"./RasterGrids\_100m/2024/RAW/FarmlandPloughed\_Fallow\_r500.tif"}\NormalTok{)}
\FunctionTok{names}\NormalTok{(slanis)}\OtherTok{=}\StringTok{"egv\_251"}
\NormalTok{slanis2}\OtherTok{=}\FunctionTok{project}\NormalTok{(slanis,template100)}
\FunctionTok{writeRaster}\NormalTok{(slanis2,}
      \StringTok{"./RasterGrids\_100m/2024/RAW/FarmlandPloughed\_Fallow\_r500.tif"}\NormalTok{,}
      \AttributeTok{overwrite=}\ConstantTok{TRUE}\NormalTok{)}

\CommentTok{\# standardisation {-}{-}{-}{-}}
\ControlFlowTok{if}\NormalTok{(}\SpecialCharTok{!}\FunctionTok{require}\NormalTok{(terra)) \{}\FunctionTok{install.packages}\NormalTok{(}\StringTok{"terra"}\NormalTok{); }\FunctionTok{require}\NormalTok{(terra)\}}
\ControlFlowTok{if}\NormalTok{(}\SpecialCharTok{!}\FunctionTok{require}\NormalTok{(tidyverse)) \{}\FunctionTok{install.packages}\NormalTok{(}\StringTok{"tidyverse"}\NormalTok{); }\FunctionTok{require}\NormalTok{(tidyverse)\}}

\NormalTok{nosaukums}\OtherTok{=}\StringTok{"FarmlandPloughed\_Fallow\_r500.tif"}
\NormalTok{ielasisanas\_cels}\OtherTok{=}\FunctionTok{paste0}\NormalTok{(}\StringTok{"./RasterGrids\_100m/2024/RAW/"}\NormalTok{,nosaukums)}
\NormalTok{saglabasanas\_cels}\OtherTok{=}\FunctionTok{paste0}\NormalTok{(}\StringTok{"./RasterGrids\_100m/2024/Scaled/"}\NormalTok{,nosaukums)}
\NormalTok{slanis}\OtherTok{=}\FunctionTok{rast}\NormalTok{(ielasisanas\_cels)}
\NormalTok{videjais}\OtherTok{=}\FunctionTok{global}\NormalTok{(slanis,}\AttributeTok{fun=}\StringTok{"mean"}\NormalTok{,}\AttributeTok{na.rm=}\ConstantTok{TRUE}\NormalTok{)}
\NormalTok{centrets}\OtherTok{=}\NormalTok{slanis}\SpecialCharTok{{-}}\NormalTok{videjais[,}\DecValTok{1}\NormalTok{]}
\NormalTok{standartnovirze}\OtherTok{=}\NormalTok{terra}\SpecialCharTok{::}\FunctionTok{global}\NormalTok{(centrets,}\AttributeTok{fun=}\StringTok{"rms"}\NormalTok{,}\AttributeTok{na.rm=}\ConstantTok{TRUE}\NormalTok{)}
\NormalTok{merogots}\OtherTok{=}\NormalTok{centrets}\SpecialCharTok{/}\NormalTok{standartnovirze[,}\DecValTok{1}\NormalTok{]}
\FunctionTok{writeRaster}\NormalTok{(merogots,}
      \AttributeTok{filename=}\NormalTok{saglabasanas\_cels,}
      \AttributeTok{overwrite=}\ConstantTok{TRUE}\NormalTok{)}
\end{Highlighting}
\end{Shaded}

\section{FarmlandPloughed\_Fallow\_r1250}\label{ch06.252}

\textbf{filename:} \texttt{FarmlandPloughed\_Fallow\_r1250.tif}

\textbf{layername:} \texttt{egv\_252}

\textbf{English name:} Fractional cover of Fallow Land within the 1.25 km landscape

\textbf{Latvian name:} Papuvju platības īpatsvars 1,25 km ainavā

\textbf{Procedure:} The cover fraction within a radius of 1250 m around the analysis grid cell
is calculated as the area-weighted sum of the \hyperref[ch06.250]{analysis cells} inside
the buffer, using the workflow \texttt{egvtools::radius\_function()}. During the calculation of the landscape
metric, inverse distance weighted (power = 2) gap filling on the output is
applied to ensure no missing values at the edges. Then the layer is
rewritten to set its name. Finally, the layer is standardised by
subtracting the arithmetic mean and dividing by the root mean squared error.

\begin{Shaded}
\begin{Highlighting}[]
\CommentTok{\# libs {-}{-}{-}{-}}
\ControlFlowTok{if}\NormalTok{(}\SpecialCharTok{!}\FunctionTok{require}\NormalTok{(terra)) \{}\FunctionTok{install.packages}\NormalTok{(}\StringTok{"terra"}\NormalTok{); }\FunctionTok{require}\NormalTok{(terra)\}}
\ControlFlowTok{if}\NormalTok{(}\SpecialCharTok{!}\FunctionTok{require}\NormalTok{(egvtools)) \{remotes}\SpecialCharTok{::}\FunctionTok{install\_github}\NormalTok{(}\StringTok{"aavotins/egvtools"}\NormalTok{); }\FunctionTok{require}\NormalTok{(egvtools)\}}


\CommentTok{\# Templates {-}{-}{-}{-}{-}}
\NormalTok{template100}\OtherTok{=}\FunctionTok{rast}\NormalTok{(}\StringTok{"./Templates/TemplateRasters/LV100m\_10km.tif"}\NormalTok{)}

\CommentTok{\# radii {-}{-}{-}{-}}
\FunctionTok{radius\_function}\NormalTok{(}
 \AttributeTok{kvadrati\_path =} \StringTok{"./Templates/TemplateGrids/tiles/"}\NormalTok{,}
 \AttributeTok{radii\_path   =} \StringTok{"./Templates/TemplateGridPoints/tiles/"}\NormalTok{,}
 \AttributeTok{tikls100\_path =} \StringTok{"./Templates/TemplateGrids/tikls100\_sauzeme.parquet"}\NormalTok{,}
 \AttributeTok{template\_path =} \StringTok{"./Templates/TemplateRasters/LV100m\_10km.tif"}\NormalTok{,}
 \AttributeTok{input\_layers  =} \FunctionTok{c}\NormalTok{(}\StringTok{"./RasterGrids\_100m/2024/RAW/FarmlandPloughed\_Fallow\_cell.tif"}\NormalTok{),}
 \AttributeTok{layer\_prefixes =} \FunctionTok{c}\NormalTok{(}\StringTok{"FarmlandPloughed\_Fallow"}\NormalTok{),}
 \AttributeTok{output\_dir   =} \StringTok{"./RasterGrids\_100m/2024/RAW/"}\NormalTok{,}
 \AttributeTok{n\_workers   =} \DecValTok{6}\NormalTok{,}
 \AttributeTok{radii     =} \FunctionTok{c}\NormalTok{(}\StringTok{"r1250"}\NormalTok{),}
 \AttributeTok{radius\_mode  =} \StringTok{"sparse"}\NormalTok{,}
 \AttributeTok{extract\_fun  =} \StringTok{"mean"}\NormalTok{,}
 \AttributeTok{fill\_missing  =} \ConstantTok{TRUE}\NormalTok{,}
 \AttributeTok{IDW\_weight   =} \DecValTok{2}\NormalTok{,}
 \AttributeTok{future\_max\_size =} \DecValTok{40} \SpecialCharTok{*} \DecValTok{1024}\SpecialCharTok{\^{}}\DecValTok{3}\NormalTok{)}


\CommentTok{\# FarmlandPloughed\_Fallow\_r1250.tif egv\_252}
\NormalTok{slanis}\OtherTok{=}\FunctionTok{rast}\NormalTok{(}\StringTok{"./RasterGrids\_100m/2024/RAW/FarmlandPloughed\_Fallow\_r1250.tif"}\NormalTok{)}
\FunctionTok{names}\NormalTok{(slanis)}\OtherTok{=}\StringTok{"egv\_252"}
\NormalTok{slanis2}\OtherTok{=}\FunctionTok{project}\NormalTok{(slanis,template100)}
\FunctionTok{writeRaster}\NormalTok{(slanis2,}
      \StringTok{"./RasterGrids\_100m/2024/RAW/FarmlandPloughed\_Fallow\_r1250.tif"}\NormalTok{,}
      \AttributeTok{overwrite=}\ConstantTok{TRUE}\NormalTok{)}

\CommentTok{\# standardisation {-}{-}{-}{-}}
\ControlFlowTok{if}\NormalTok{(}\SpecialCharTok{!}\FunctionTok{require}\NormalTok{(terra)) \{}\FunctionTok{install.packages}\NormalTok{(}\StringTok{"terra"}\NormalTok{); }\FunctionTok{require}\NormalTok{(terra)\}}
\ControlFlowTok{if}\NormalTok{(}\SpecialCharTok{!}\FunctionTok{require}\NormalTok{(tidyverse)) \{}\FunctionTok{install.packages}\NormalTok{(}\StringTok{"tidyverse"}\NormalTok{); }\FunctionTok{require}\NormalTok{(tidyverse)\}}

\NormalTok{nosaukums}\OtherTok{=}\StringTok{"FarmlandPloughed\_Fallow\_r1250.tif"}
\NormalTok{ielasisanas\_cels}\OtherTok{=}\FunctionTok{paste0}\NormalTok{(}\StringTok{"./RasterGrids\_100m/2024/RAW/"}\NormalTok{,nosaukums)}
\NormalTok{saglabasanas\_cels}\OtherTok{=}\FunctionTok{paste0}\NormalTok{(}\StringTok{"./RasterGrids\_100m/2024/Scaled/"}\NormalTok{,nosaukums)}
\NormalTok{slanis}\OtherTok{=}\FunctionTok{rast}\NormalTok{(ielasisanas\_cels)}
\NormalTok{videjais}\OtherTok{=}\FunctionTok{global}\NormalTok{(slanis,}\AttributeTok{fun=}\StringTok{"mean"}\NormalTok{,}\AttributeTok{na.rm=}\ConstantTok{TRUE}\NormalTok{)}
\NormalTok{centrets}\OtherTok{=}\NormalTok{slanis}\SpecialCharTok{{-}}\NormalTok{videjais[,}\DecValTok{1}\NormalTok{]}
\NormalTok{standartnovirze}\OtherTok{=}\NormalTok{terra}\SpecialCharTok{::}\FunctionTok{global}\NormalTok{(centrets,}\AttributeTok{fun=}\StringTok{"rms"}\NormalTok{,}\AttributeTok{na.rm=}\ConstantTok{TRUE}\NormalTok{)}
\NormalTok{merogots}\OtherTok{=}\NormalTok{centrets}\SpecialCharTok{/}\NormalTok{standartnovirze[,}\DecValTok{1}\NormalTok{]}
\FunctionTok{writeRaster}\NormalTok{(merogots,}
      \AttributeTok{filename=}\NormalTok{saglabasanas\_cels,}
      \AttributeTok{overwrite=}\ConstantTok{TRUE}\NormalTok{)}
\end{Highlighting}
\end{Shaded}

\section{FarmlandPloughed\_Fallow\_r3000}\label{ch06.253}

\textbf{filename:} \texttt{FarmlandPloughed\_Fallow\_r3000.tif}

\textbf{layername:} \texttt{egv\_253}

\textbf{English name:} Fractional cover of Fallow Land within the 3 km landscape

\textbf{Latvian name:} Papuvju platības īpatsvars 3 km ainavā

\textbf{Procedure:} The cover fraction within a radius of 3000 m around the analysis grid cell
is calculated as the area-weighted sum of the \hyperref[ch06.250]{analysis cells} inside
the buffer, using the workflow \texttt{egvtools::radius\_function()}. During the calculation of the landscape
metric, inverse distance weighted (power = 2) gap filling on the output is
applied to ensure no missing values at the edges. Then the layer is
rewritten to set its name. Finally, the layer is standardised by
subtracting the arithmetic mean and dividing by the root mean squared error.

\begin{Shaded}
\begin{Highlighting}[]
\CommentTok{\# libs {-}{-}{-}{-}}
\ControlFlowTok{if}\NormalTok{(}\SpecialCharTok{!}\FunctionTok{require}\NormalTok{(terra)) \{}\FunctionTok{install.packages}\NormalTok{(}\StringTok{"terra"}\NormalTok{); }\FunctionTok{require}\NormalTok{(terra)\}}
\ControlFlowTok{if}\NormalTok{(}\SpecialCharTok{!}\FunctionTok{require}\NormalTok{(egvtools)) \{remotes}\SpecialCharTok{::}\FunctionTok{install\_github}\NormalTok{(}\StringTok{"aavotins/egvtools"}\NormalTok{); }\FunctionTok{require}\NormalTok{(egvtools)\}}


\CommentTok{\# Templates {-}{-}{-}{-}{-}}
\NormalTok{template100}\OtherTok{=}\FunctionTok{rast}\NormalTok{(}\StringTok{"./Templates/TemplateRasters/LV100m\_10km.tif"}\NormalTok{)}

\CommentTok{\# radii {-}{-}{-}{-}}
\FunctionTok{radius\_function}\NormalTok{(}
 \AttributeTok{kvadrati\_path =} \StringTok{"./Templates/TemplateGrids/tiles/"}\NormalTok{,}
 \AttributeTok{radii\_path   =} \StringTok{"./Templates/TemplateGridPoints/tiles/"}\NormalTok{,}
 \AttributeTok{tikls100\_path =} \StringTok{"./Templates/TemplateGrids/tikls100\_sauzeme.parquet"}\NormalTok{,}
 \AttributeTok{template\_path =} \StringTok{"./Templates/TemplateRasters/LV100m\_10km.tif"}\NormalTok{,}
 \AttributeTok{input\_layers  =} \FunctionTok{c}\NormalTok{(}\StringTok{"./RasterGrids\_100m/2024/RAW/FarmlandPloughed\_Fallow\_cell.tif"}\NormalTok{),}
 \AttributeTok{layer\_prefixes =} \FunctionTok{c}\NormalTok{(}\StringTok{"FarmlandPloughed\_Fallow"}\NormalTok{),}
 \AttributeTok{output\_dir   =} \StringTok{"./RasterGrids\_100m/2024/RAW/"}\NormalTok{,}
 \AttributeTok{n\_workers   =} \DecValTok{6}\NormalTok{,}
 \AttributeTok{radii     =} \FunctionTok{c}\NormalTok{(}\StringTok{"r3000"}\NormalTok{),}
 \AttributeTok{radius\_mode  =} \StringTok{"sparse"}\NormalTok{,}
 \AttributeTok{extract\_fun  =} \StringTok{"mean"}\NormalTok{,}
 \AttributeTok{fill\_missing  =} \ConstantTok{TRUE}\NormalTok{,}
 \AttributeTok{IDW\_weight   =} \DecValTok{2}\NormalTok{,}
 \AttributeTok{future\_max\_size =} \DecValTok{40} \SpecialCharTok{*} \DecValTok{1024}\SpecialCharTok{\^{}}\DecValTok{3}\NormalTok{)}


\CommentTok{\# FarmlandPloughed\_Fallow\_r3000.tif egv\_253}
\NormalTok{slanis}\OtherTok{=}\FunctionTok{rast}\NormalTok{(}\StringTok{"./RasterGrids\_100m/2024/RAW/FarmlandPloughed\_Fallow\_r3000.tif"}\NormalTok{)}
\FunctionTok{names}\NormalTok{(slanis)}\OtherTok{=}\StringTok{"egv\_253"}
\NormalTok{slanis2}\OtherTok{=}\FunctionTok{project}\NormalTok{(slanis,template100)}
\FunctionTok{writeRaster}\NormalTok{(slanis2,}
      \StringTok{"./RasterGrids\_100m/2024/RAW/FarmlandPloughed\_Fallow\_r3000.tif"}\NormalTok{,}
      \AttributeTok{overwrite=}\ConstantTok{TRUE}\NormalTok{)}

\CommentTok{\# standardisation {-}{-}{-}{-}}
\ControlFlowTok{if}\NormalTok{(}\SpecialCharTok{!}\FunctionTok{require}\NormalTok{(terra)) \{}\FunctionTok{install.packages}\NormalTok{(}\StringTok{"terra"}\NormalTok{); }\FunctionTok{require}\NormalTok{(terra)\}}
\ControlFlowTok{if}\NormalTok{(}\SpecialCharTok{!}\FunctionTok{require}\NormalTok{(tidyverse)) \{}\FunctionTok{install.packages}\NormalTok{(}\StringTok{"tidyverse"}\NormalTok{); }\FunctionTok{require}\NormalTok{(tidyverse)\}}

\NormalTok{nosaukums}\OtherTok{=}\StringTok{"FarmlandPloughed\_Fallow\_r3000.tif"}
\NormalTok{ielasisanas\_cels}\OtherTok{=}\FunctionTok{paste0}\NormalTok{(}\StringTok{"./RasterGrids\_100m/2024/RAW/"}\NormalTok{,nosaukums)}
\NormalTok{saglabasanas\_cels}\OtherTok{=}\FunctionTok{paste0}\NormalTok{(}\StringTok{"./RasterGrids\_100m/2024/Scaled/"}\NormalTok{,nosaukums)}
\NormalTok{slanis}\OtherTok{=}\FunctionTok{rast}\NormalTok{(ielasisanas\_cels)}
\NormalTok{videjais}\OtherTok{=}\FunctionTok{global}\NormalTok{(slanis,}\AttributeTok{fun=}\StringTok{"mean"}\NormalTok{,}\AttributeTok{na.rm=}\ConstantTok{TRUE}\NormalTok{)}
\NormalTok{centrets}\OtherTok{=}\NormalTok{slanis}\SpecialCharTok{{-}}\NormalTok{videjais[,}\DecValTok{1}\NormalTok{]}
\NormalTok{standartnovirze}\OtherTok{=}\NormalTok{terra}\SpecialCharTok{::}\FunctionTok{global}\NormalTok{(centrets,}\AttributeTok{fun=}\StringTok{"rms"}\NormalTok{,}\AttributeTok{na.rm=}\ConstantTok{TRUE}\NormalTok{)}
\NormalTok{merogots}\OtherTok{=}\NormalTok{centrets}\SpecialCharTok{/}\NormalTok{standartnovirze[,}\DecValTok{1}\NormalTok{]}
\FunctionTok{writeRaster}\NormalTok{(merogots,}
      \AttributeTok{filename=}\NormalTok{saglabasanas\_cels,}
      \AttributeTok{overwrite=}\ConstantTok{TRUE}\NormalTok{)}
\end{Highlighting}
\end{Shaded}

\section{FarmlandPloughed\_Fallow\_r10000}\label{ch06.254}

\textbf{filename:} \texttt{FarmlandPloughed\_Fallow\_r10000.tif}

\textbf{layername:} \texttt{egv\_254}

\textbf{English name:} Fractional cover of Fallow Land within the 10 km landscape

\textbf{Latvian name:} Papuvju platības īpatsvars 10 km ainavā

\textbf{Procedure:} The cover fraction within a radius of 10000 m around the analysis grid cell
is calculated as the area-weighted sum of the \hyperref[ch06.250]{analysis cells} inside
the buffer, using the workflow \texttt{egvtools::radius\_function()}. During the calculation of the landscape
metric, inverse distance weighted (power = 2) gap filling on the output is
applied to ensure no missing values at the edges. Then the layer is
rewritten to set its name. Finally, the layer is standardised by
subtracting the arithmetic mean and dividing by the root mean squared error.

\begin{Shaded}
\begin{Highlighting}[]
\CommentTok{\# libs {-}{-}{-}{-}}
\ControlFlowTok{if}\NormalTok{(}\SpecialCharTok{!}\FunctionTok{require}\NormalTok{(terra)) \{}\FunctionTok{install.packages}\NormalTok{(}\StringTok{"terra"}\NormalTok{); }\FunctionTok{require}\NormalTok{(terra)\}}
\ControlFlowTok{if}\NormalTok{(}\SpecialCharTok{!}\FunctionTok{require}\NormalTok{(egvtools)) \{remotes}\SpecialCharTok{::}\FunctionTok{install\_github}\NormalTok{(}\StringTok{"aavotins/egvtools"}\NormalTok{); }\FunctionTok{require}\NormalTok{(egvtools)\}}


\CommentTok{\# Templates {-}{-}{-}{-}{-}}
\NormalTok{template100}\OtherTok{=}\FunctionTok{rast}\NormalTok{(}\StringTok{"./Templates/TemplateRasters/LV100m\_10km.tif"}\NormalTok{)}

\CommentTok{\# radii {-}{-}{-}{-}}
\FunctionTok{radius\_function}\NormalTok{(}
 \AttributeTok{kvadrati\_path =} \StringTok{"./Templates/TemplateGrids/tiles/"}\NormalTok{,}
 \AttributeTok{radii\_path   =} \StringTok{"./Templates/TemplateGridPoints/tiles/"}\NormalTok{,}
 \AttributeTok{tikls100\_path =} \StringTok{"./Templates/TemplateGrids/tikls100\_sauzeme.parquet"}\NormalTok{,}
 \AttributeTok{template\_path =} \StringTok{"./Templates/TemplateRasters/LV100m\_10km.tif"}\NormalTok{,}
 \AttributeTok{input\_layers  =} \FunctionTok{c}\NormalTok{(}\StringTok{"./RasterGrids\_100m/2024/RAW/FarmlandPloughed\_Fallow\_cell.tif"}\NormalTok{),}
 \AttributeTok{layer\_prefixes =} \FunctionTok{c}\NormalTok{(}\StringTok{"FarmlandPloughed\_Fallow"}\NormalTok{),}
 \AttributeTok{output\_dir   =} \StringTok{"./RasterGrids\_100m/2024/RAW/"}\NormalTok{,}
 \AttributeTok{n\_workers   =} \DecValTok{6}\NormalTok{,}
 \AttributeTok{radii     =} \FunctionTok{c}\NormalTok{(}\StringTok{"r10000"}\NormalTok{),}
 \AttributeTok{radius\_mode  =} \StringTok{"sparse"}\NormalTok{,}
 \AttributeTok{extract\_fun  =} \StringTok{"mean"}\NormalTok{,}
 \AttributeTok{fill\_missing  =} \ConstantTok{TRUE}\NormalTok{,}
 \AttributeTok{IDW\_weight   =} \DecValTok{2}\NormalTok{,}
 \AttributeTok{future\_max\_size =} \DecValTok{40} \SpecialCharTok{*} \DecValTok{1024}\SpecialCharTok{\^{}}\DecValTok{3}\NormalTok{)}


\CommentTok{\# FarmlandPloughed\_Fallow\_r10000.tif    egv\_254}
\NormalTok{slanis}\OtherTok{=}\FunctionTok{rast}\NormalTok{(}\StringTok{"./RasterGrids\_100m/2024/RAW/FarmlandPloughed\_Fallow\_r10000.tif"}\NormalTok{)}
\FunctionTok{names}\NormalTok{(slanis)}\OtherTok{=}\StringTok{"egv\_254"}
\NormalTok{slanis2}\OtherTok{=}\FunctionTok{project}\NormalTok{(slanis,template100)}
\FunctionTok{writeRaster}\NormalTok{(slanis2,}
      \StringTok{"./RasterGrids\_100m/2024/RAW/FarmlandPloughed\_Fallow\_r10000.tif"}\NormalTok{,}
      \AttributeTok{overwrite=}\ConstantTok{TRUE}\NormalTok{)}

\CommentTok{\# standardisation {-}{-}{-}{-}}
\ControlFlowTok{if}\NormalTok{(}\SpecialCharTok{!}\FunctionTok{require}\NormalTok{(terra)) \{}\FunctionTok{install.packages}\NormalTok{(}\StringTok{"terra"}\NormalTok{); }\FunctionTok{require}\NormalTok{(terra)\}}
\ControlFlowTok{if}\NormalTok{(}\SpecialCharTok{!}\FunctionTok{require}\NormalTok{(tidyverse)) \{}\FunctionTok{install.packages}\NormalTok{(}\StringTok{"tidyverse"}\NormalTok{); }\FunctionTok{require}\NormalTok{(tidyverse)\}}

\NormalTok{nosaukums}\OtherTok{=}\StringTok{"FarmlandPloughed\_Fallow\_r10000.tif"}
\NormalTok{ielasisanas\_cels}\OtherTok{=}\FunctionTok{paste0}\NormalTok{(}\StringTok{"./RasterGrids\_100m/2024/RAW/"}\NormalTok{,nosaukums)}
\NormalTok{saglabasanas\_cels}\OtherTok{=}\FunctionTok{paste0}\NormalTok{(}\StringTok{"./RasterGrids\_100m/2024/Scaled/"}\NormalTok{,nosaukums)}
\NormalTok{slanis}\OtherTok{=}\FunctionTok{rast}\NormalTok{(ielasisanas\_cels)}
\NormalTok{videjais}\OtherTok{=}\FunctionTok{global}\NormalTok{(slanis,}\AttributeTok{fun=}\StringTok{"mean"}\NormalTok{,}\AttributeTok{na.rm=}\ConstantTok{TRUE}\NormalTok{)}
\NormalTok{centrets}\OtherTok{=}\NormalTok{slanis}\SpecialCharTok{{-}}\NormalTok{videjais[,}\DecValTok{1}\NormalTok{]}
\NormalTok{standartnovirze}\OtherTok{=}\NormalTok{terra}\SpecialCharTok{::}\FunctionTok{global}\NormalTok{(centrets,}\AttributeTok{fun=}\StringTok{"rms"}\NormalTok{,}\AttributeTok{na.rm=}\ConstantTok{TRUE}\NormalTok{)}
\NormalTok{merogots}\OtherTok{=}\NormalTok{centrets}\SpecialCharTok{/}\NormalTok{standartnovirze[,}\DecValTok{1}\NormalTok{]}
\FunctionTok{writeRaster}\NormalTok{(merogots,}
      \AttributeTok{filename=}\NormalTok{saglabasanas\_cels,}
      \AttributeTok{overwrite=}\ConstantTok{TRUE}\NormalTok{)}
\end{Highlighting}
\end{Shaded}

\section{FarmlandSubsidies\_BiologicalSubsidies\_cell}\label{ch06.255}

\textbf{filename:} \texttt{FarmlandSubsidies\_BiologicalSubsidies\_cell.tif}

\textbf{layername:} \texttt{egv\_255}

\textbf{English name:} Fractional cover of Farmland receiving Subsidies for
Biological Agriculture within the analysis cell (1 ha)

\textbf{Latvian name:} Bioloģiskās lauksaimniecības atbalstam pieteikto
lauksaimniecības platību īpatsvars analīzes šūnā (1 ha)

\textbf{Procedure:} First, agricultural parcels declared as receiving subsidies for
biological agriculture are selected from the \hyperref[Ch04.02]{Rural Support Service's information
on declared fields}. Geometries are then rasterised to input
resolution, ensuring value 1 at the polygon locations and value 0 elsewhere.
Rasterisation is
performed using the workflow \texttt{egvtools::polygon2input()}. Once rasterised, the
layer is aggregated to EGV resolution using the workflow \texttt{egvtools::input2egv()},
which calculates the arithmetic mean and thus
results in a cover fraction. During
aggregation, inverse distance weighted (power = 2) gap filling on the output is
applied to ensure no missing values at the edges. Finally, the layer is
standardised by subtracting the arithmetic mean and dividing by the root mean squared
error.

\begin{Shaded}
\begin{Highlighting}[]
\CommentTok{\# libs {-}{-}{-}{-}}
\ControlFlowTok{if}\NormalTok{(}\SpecialCharTok{!}\FunctionTok{require}\NormalTok{(egvtools)) \{remotes}\SpecialCharTok{::}\FunctionTok{install\_github}\NormalTok{(}\StringTok{"aavotins/egvtools"}\NormalTok{); }\FunctionTok{require}\NormalTok{(egvtools)\}}
\ControlFlowTok{if}\NormalTok{(}\SpecialCharTok{!}\FunctionTok{require}\NormalTok{(terra)) \{}\FunctionTok{install.packages}\NormalTok{(}\StringTok{"terra"}\NormalTok{); }\FunctionTok{require}\NormalTok{(terra)\}}
\ControlFlowTok{if}\NormalTok{(}\SpecialCharTok{!}\FunctionTok{require}\NormalTok{(sf)) \{}\FunctionTok{install.packages}\NormalTok{(}\StringTok{"sf"}\NormalTok{); }\FunctionTok{require}\NormalTok{(sf)\}}
\ControlFlowTok{if}\NormalTok{(}\SpecialCharTok{!}\FunctionTok{require}\NormalTok{(tidyverse)) \{}\FunctionTok{install.packages}\NormalTok{(}\StringTok{"tidyverse"}\NormalTok{); }\FunctionTok{require}\NormalTok{(tidyverse)\}}
\ControlFlowTok{if}\NormalTok{(}\SpecialCharTok{!}\FunctionTok{require}\NormalTok{(sfarrow)) \{}\FunctionTok{install.packages}\NormalTok{(}\StringTok{"sfarrow"}\NormalTok{); }\FunctionTok{require}\NormalTok{(sfarrow)\}}
\ControlFlowTok{if}\NormalTok{(}\SpecialCharTok{!}\FunctionTok{require}\NormalTok{(readxl)) \{}\FunctionTok{install.packages}\NormalTok{(}\StringTok{"readxl"}\NormalTok{); }\FunctionTok{require}\NormalTok{(readxl)\}}
\ControlFlowTok{if}\NormalTok{(}\SpecialCharTok{!}\FunctionTok{require}\NormalTok{(raster)) \{}\FunctionTok{install.packages}\NormalTok{(}\StringTok{"raster"}\NormalTok{); }\FunctionTok{require}\NormalTok{(raster)\}}
\ControlFlowTok{if}\NormalTok{(}\SpecialCharTok{!}\FunctionTok{require}\NormalTok{(fasterize)) \{}\FunctionTok{install.packages}\NormalTok{(}\StringTok{"fasterize"}\NormalTok{); }\FunctionTok{require}\NormalTok{(fasterize)\}}

\CommentTok{\# templates {-}{-}{-}{-}}
\NormalTok{template100}\OtherTok{=}\FunctionTok{rast}\NormalTok{(}\StringTok{"./Templates/TemplateRasters/LV100m\_10km.tif"}\NormalTok{)}
\NormalTok{template10}\OtherTok{=}\FunctionTok{rast}\NormalTok{(}\StringTok{"./Templates/TemplateRasters/LV10m\_10km.tif"}\NormalTok{)}
\NormalTok{rastrs10}\OtherTok{=}\FunctionTok{raster}\NormalTok{(template10)}

\NormalTok{nulls10}\OtherTok{=}\FunctionTok{rast}\NormalTok{(}\StringTok{"./Templates/TemplateRasters/nulls\_LV10m\_10km.tif"}\NormalTok{)}
\NormalTok{nulls100}\OtherTok{=}\FunctionTok{rast}\NormalTok{(}\StringTok{"./Templates/TemplateRasters/nulls\_LV100m\_10km.tif"}\NormalTok{)}

\CommentTok{\# codes {-}{-}{-}{-}}
\NormalTok{kodi}\OtherTok{=}\FunctionTok{read\_excel}\NormalTok{(}\StringTok{"./Geodata/2024/LAD/KulturuKodi\_2024.xlsx"}\NormalTok{)}
\NormalTok{kodi}\SpecialCharTok{$}\NormalTok{kods}\OtherTok{=}\FunctionTok{as.character}\NormalTok{(kodi}\SpecialCharTok{$}\NormalTok{kods)}
\CommentTok{\# LAD {-}{-}{-}{-}}
\NormalTok{lad}\OtherTok{=}\NormalTok{sfarrow}\SpecialCharTok{::}\FunctionTok{st\_read\_parquet}\NormalTok{(}\StringTok{"./Geodata/2024/LAD/Lauki\_2024.parquet"}\NormalTok{)}
\NormalTok{lad}\SpecialCharTok{$}\NormalTok{yes}\OtherTok{=}\DecValTok{1}
\NormalTok{lad}\OtherTok{=}\NormalTok{lad }\SpecialCharTok{\%\textgreater{}\%} 
 \FunctionTok{left\_join}\NormalTok{(kodi,}\AttributeTok{by=}\FunctionTok{c}\NormalTok{(}\StringTok{"PRODUCT\_CODE"}\OtherTok{=}\StringTok{"kods"}\NormalTok{))}

\CommentTok{\# simple landscape {-}{-}{-}{-}}
\NormalTok{simple\_landscape}\OtherTok{=}\FunctionTok{rast}\NormalTok{(}\StringTok{"RasterGrids\_10m/2024/Ainava\_vienk\_mask.tif"}\NormalTok{)}


\CommentTok{\# FarmlandSubsidies\_BiologicalSubsidies\_cell.tif    egv\_255 {-}{-}{-}{-}}
\NormalTok{dati}\OtherTok{=}\NormalTok{lad }\SpecialCharTok{\%\textgreater{}\%} 
 \FunctionTok{filter}\NormalTok{(}\FunctionTok{str\_detect}\NormalTok{(AID\_FORMS,}\StringTok{"BLA"}\NormalTok{))}
\FunctionTok{table}\NormalTok{(dati}\SpecialCharTok{$}\NormalTok{AID\_FORMS,}\AttributeTok{useNA=}\StringTok{"always"}\NormalTok{)}

\NormalTok{p2i\_rez}\OtherTok{=}\NormalTok{egvtools}\SpecialCharTok{::}\FunctionTok{polygon2input}\NormalTok{(}\AttributeTok{vector\_data =}\NormalTok{ dati,}
                \AttributeTok{template\_path =} \StringTok{"./Templates/TemplateRasters/LV10m\_10km.tif"}\NormalTok{,}
                \AttributeTok{out\_path =} \StringTok{"./RasterGrids\_10m/2024/"}\NormalTok{,}
                \AttributeTok{file\_name =} \StringTok{"FarmlandSubsidies\_BiologicalSubsidies\_input.tif"}\NormalTok{,}
                \AttributeTok{value\_field =} \StringTok{"yes"}\NormalTok{,}
                \AttributeTok{prepare=}\ConstantTok{FALSE}\NormalTok{,}
                \AttributeTok{background\_raster =} \StringTok{"./Templates/TemplateRasters/nulls\_LV10m\_10km.tif"}\NormalTok{,}
                \AttributeTok{plot\_result =} \ConstantTok{TRUE}\NormalTok{)}
\NormalTok{p2i\_rez}
\NormalTok{i2e\_rez}\OtherTok{=}\NormalTok{egvtools}\SpecialCharTok{::}\FunctionTok{input2egv}\NormalTok{(}\AttributeTok{input=}\FunctionTok{paste0}\NormalTok{(}\StringTok{"./RasterGrids\_10m/2024/"}\NormalTok{,}
                     \StringTok{"FarmlandSubsidies\_BiologicalSubsidies\_input.tif"}\NormalTok{),}
              \AttributeTok{egv\_template=} \StringTok{"./Templates/TemplateRasters/LV100m\_10km.tif"}\NormalTok{,}
              \AttributeTok{summary\_function =} \StringTok{"average"}\NormalTok{,}
              \AttributeTok{missing\_job =} \StringTok{"FillOutput"}\NormalTok{,}
              \AttributeTok{outlocation =} \StringTok{"./RasterGrids\_100m/2024/RAW/"}\NormalTok{,}
              \AttributeTok{outfilename =} \StringTok{"FarmlandSubsidies\_BiologicalSubsidies\_cell.tif"}\NormalTok{,}
              \AttributeTok{layername =} \StringTok{"egv\_255"}\NormalTok{,}
              \AttributeTok{idw\_weight =} \DecValTok{2}\NormalTok{,}
              \AttributeTok{plot\_gaps =} \ConstantTok{FALSE}\NormalTok{,}\AttributeTok{plot\_final =} \ConstantTok{TRUE}\NormalTok{)}
\NormalTok{i2e\_rez}
\FunctionTok{rm}\NormalTok{(p2i\_rez)}
\FunctionTok{rm}\NormalTok{(i2e\_rez)}
\FunctionTok{rm}\NormalTok{(dati)}
\FunctionTok{unlink}\NormalTok{(}\StringTok{"./RasterGrids\_10m/2024/FarmlandSubsidies\_BiologicalSubsidies\_input.tif"}\NormalTok{)}

\CommentTok{\# standardisation {-}{-}{-}{-}}
\ControlFlowTok{if}\NormalTok{(}\SpecialCharTok{!}\FunctionTok{require}\NormalTok{(terra)) \{}\FunctionTok{install.packages}\NormalTok{(}\StringTok{"terra"}\NormalTok{); }\FunctionTok{require}\NormalTok{(terra)\}}
\ControlFlowTok{if}\NormalTok{(}\SpecialCharTok{!}\FunctionTok{require}\NormalTok{(tidyverse)) \{}\FunctionTok{install.packages}\NormalTok{(}\StringTok{"tidyverse"}\NormalTok{); }\FunctionTok{require}\NormalTok{(tidyverse)\}}

\NormalTok{nosaukums}\OtherTok{=}\StringTok{"FarmlandSubsidies\_BiologicalSubsidies\_cell.tif"}
\NormalTok{ielasisanas\_cels}\OtherTok{=}\FunctionTok{paste0}\NormalTok{(}\StringTok{"./RasterGrids\_100m/2024/RAW/"}\NormalTok{,nosaukums)}
\NormalTok{saglabasanas\_cels}\OtherTok{=}\FunctionTok{paste0}\NormalTok{(}\StringTok{"./RasterGrids\_100m/2024/Scaled/"}\NormalTok{,nosaukums)}
\NormalTok{slanis}\OtherTok{=}\FunctionTok{rast}\NormalTok{(ielasisanas\_cels)}
\NormalTok{videjais}\OtherTok{=}\FunctionTok{global}\NormalTok{(slanis,}\AttributeTok{fun=}\StringTok{"mean"}\NormalTok{,}\AttributeTok{na.rm=}\ConstantTok{TRUE}\NormalTok{)}
\NormalTok{centrets}\OtherTok{=}\NormalTok{slanis}\SpecialCharTok{{-}}\NormalTok{videjais[,}\DecValTok{1}\NormalTok{]}
\NormalTok{standartnovirze}\OtherTok{=}\NormalTok{terra}\SpecialCharTok{::}\FunctionTok{global}\NormalTok{(centrets,}\AttributeTok{fun=}\StringTok{"rms"}\NormalTok{,}\AttributeTok{na.rm=}\ConstantTok{TRUE}\NormalTok{)}
\NormalTok{merogots}\OtherTok{=}\NormalTok{centrets}\SpecialCharTok{/}\NormalTok{standartnovirze[,}\DecValTok{1}\NormalTok{]}
\FunctionTok{writeRaster}\NormalTok{(merogots,}
      \AttributeTok{filename=}\NormalTok{saglabasanas\_cels,}
      \AttributeTok{overwrite=}\ConstantTok{TRUE}\NormalTok{)}
\end{Highlighting}
\end{Shaded}

\section{FarmlandSubsidies\_BiologicalSubsidies\_r500}\label{ch06.256}

\textbf{filename:} \texttt{FarmlandSubsidies\_BiologicalSubsidies\_r500.tif}

\textbf{layername:} \texttt{egv\_256}

\textbf{English name:} Fractional cover of Farmland receiving Subsidies for
Biological Agriculture within the 0.5 km landscape

\textbf{Latvian name:} Bioloģiskās lauksaimniecības atbalstam pieteikto
lauksaimniecības platību īpatsvars 0,5 km ainavā

\textbf{Procedure:} The cover fraction within a radius of 500 m around the analysis grid cell is
calculated as the area-weighted sum of the \hyperref[ch06.255]{analysis cells} inside the
buffer, using the workflow \texttt{egvtools::radius\_function()}. During the calculation of the landscape metric,
inverse distance weighted (power = 2) gap filling on the output is applied
to ensure no missing values at the edges. Then the layer is rewritten to set
its name. Finally, the layer is standardised by subtracting the arithmetic
mean and dividing by the root mean squared error.

\begin{Shaded}
\begin{Highlighting}[]
\CommentTok{\# libs {-}{-}{-}{-}}
\ControlFlowTok{if}\NormalTok{(}\SpecialCharTok{!}\FunctionTok{require}\NormalTok{(terra)) \{}\FunctionTok{install.packages}\NormalTok{(}\StringTok{"terra"}\NormalTok{); }\FunctionTok{require}\NormalTok{(terra)\}}
\ControlFlowTok{if}\NormalTok{(}\SpecialCharTok{!}\FunctionTok{require}\NormalTok{(egvtools)) \{remotes}\SpecialCharTok{::}\FunctionTok{install\_github}\NormalTok{(}\StringTok{"aavotins/egvtools"}\NormalTok{); }\FunctionTok{require}\NormalTok{(egvtools)\}}


\CommentTok{\# Templates {-}{-}{-}{-}{-}}
\NormalTok{template100}\OtherTok{=}\FunctionTok{rast}\NormalTok{(}\StringTok{"./Templates/TemplateRasters/LV100m\_10km.tif"}\NormalTok{)}

\CommentTok{\# radii {-}{-}{-}{-}}
\FunctionTok{radius\_function}\NormalTok{(}
 \AttributeTok{kvadrati\_path =} \StringTok{"./Templates/TemplateGrids/tiles/"}\NormalTok{,}
 \AttributeTok{radii\_path   =} \StringTok{"./Templates/TemplateGridPoints/tiles/"}\NormalTok{,}
 \AttributeTok{tikls100\_path =} \StringTok{"./Templates/TemplateGrids/tikls100\_sauzeme.parquet"}\NormalTok{,}
 \AttributeTok{template\_path =} \StringTok{"./Templates/TemplateRasters/LV100m\_10km.tif"}\NormalTok{,}
 \AttributeTok{input\_layers  =} \FunctionTok{c}\NormalTok{(}\StringTok{"./RasterGrids\_100m/2024/RAW/FarmlandSubsidies\_BiologicalSubsidies\_cell.tif"}\NormalTok{),}
 \AttributeTok{layer\_prefixes =} \FunctionTok{c}\NormalTok{(}\StringTok{"FarmlandSubsidies\_BiologicalSubsidies"}\NormalTok{),}
 \AttributeTok{output\_dir   =} \StringTok{"./RasterGrids\_100m/2024/RAW/"}\NormalTok{,}
 \AttributeTok{n\_workers   =} \DecValTok{6}\NormalTok{,}
 \AttributeTok{radii     =} \FunctionTok{c}\NormalTok{(}\StringTok{"r500"}\NormalTok{),}
 \AttributeTok{radius\_mode  =} \StringTok{"sparse"}\NormalTok{,}
 \AttributeTok{extract\_fun  =} \StringTok{"mean"}\NormalTok{,}
 \AttributeTok{fill\_missing  =} \ConstantTok{TRUE}\NormalTok{,}
 \AttributeTok{IDW\_weight   =} \DecValTok{2}\NormalTok{,}
 \AttributeTok{future\_max\_size =} \DecValTok{40} \SpecialCharTok{*} \DecValTok{1024}\SpecialCharTok{\^{}}\DecValTok{3}\NormalTok{)}


\CommentTok{\# FarmlandSubsidies\_BiologicalSubsidies\_r500.tif    egv\_256}
\NormalTok{slanis}\OtherTok{=}\FunctionTok{rast}\NormalTok{(}\StringTok{"./RasterGrids\_100m/2024/RAW/FarmlandSubsidies\_BiologicalSubsidies\_r500.tif"}\NormalTok{)}
\FunctionTok{names}\NormalTok{(slanis)}\OtherTok{=}\StringTok{"egv\_256"}
\NormalTok{slanis2}\OtherTok{=}\FunctionTok{project}\NormalTok{(slanis,template100)}
\FunctionTok{writeRaster}\NormalTok{(slanis2,}
      \StringTok{"./RasterGrids\_100m/2024/RAW/FarmlandSubsidies\_BiologicalSubsidies\_r500.tif"}\NormalTok{,}
      \AttributeTok{overwrite=}\ConstantTok{TRUE}\NormalTok{)}

\CommentTok{\# standardisation {-}{-}{-}{-}}
\ControlFlowTok{if}\NormalTok{(}\SpecialCharTok{!}\FunctionTok{require}\NormalTok{(terra)) \{}\FunctionTok{install.packages}\NormalTok{(}\StringTok{"terra"}\NormalTok{); }\FunctionTok{require}\NormalTok{(terra)\}}
\ControlFlowTok{if}\NormalTok{(}\SpecialCharTok{!}\FunctionTok{require}\NormalTok{(tidyverse)) \{}\FunctionTok{install.packages}\NormalTok{(}\StringTok{"tidyverse"}\NormalTok{); }\FunctionTok{require}\NormalTok{(tidyverse)\}}

\NormalTok{nosaukums}\OtherTok{=}\StringTok{"FarmlandSubsidies\_BiologicalSubsidies\_r500.tif"}
\NormalTok{ielasisanas\_cels}\OtherTok{=}\FunctionTok{paste0}\NormalTok{(}\StringTok{"./RasterGrids\_100m/2024/RAW/"}\NormalTok{,nosaukums)}
\NormalTok{saglabasanas\_cels}\OtherTok{=}\FunctionTok{paste0}\NormalTok{(}\StringTok{"./RasterGrids\_100m/2024/Scaled/"}\NormalTok{,nosaukums)}
\NormalTok{slanis}\OtherTok{=}\FunctionTok{rast}\NormalTok{(ielasisanas\_cels)}
\NormalTok{videjais}\OtherTok{=}\FunctionTok{global}\NormalTok{(slanis,}\AttributeTok{fun=}\StringTok{"mean"}\NormalTok{,}\AttributeTok{na.rm=}\ConstantTok{TRUE}\NormalTok{)}
\NormalTok{centrets}\OtherTok{=}\NormalTok{slanis}\SpecialCharTok{{-}}\NormalTok{videjais[,}\DecValTok{1}\NormalTok{]}
\NormalTok{standartnovirze}\OtherTok{=}\NormalTok{terra}\SpecialCharTok{::}\FunctionTok{global}\NormalTok{(centrets,}\AttributeTok{fun=}\StringTok{"rms"}\NormalTok{,}\AttributeTok{na.rm=}\ConstantTok{TRUE}\NormalTok{)}
\NormalTok{merogots}\OtherTok{=}\NormalTok{centrets}\SpecialCharTok{/}\NormalTok{standartnovirze[,}\DecValTok{1}\NormalTok{]}
\FunctionTok{writeRaster}\NormalTok{(merogots,}
      \AttributeTok{filename=}\NormalTok{saglabasanas\_cels,}
      \AttributeTok{overwrite=}\ConstantTok{TRUE}\NormalTok{)}
\end{Highlighting}
\end{Shaded}

\section{FarmlandSubsidies\_BiologicalSubsidies\_r1250}\label{ch06.257}

\textbf{filename:} \texttt{FarmlandSubsidies\_BiologicalSubsidies\_r1250.tif}

\textbf{layername:} \texttt{egv\_257}

\textbf{English name:} Fractional cover of Farmland receiving Subsidies for
Biological Agriculture within the 1.25 km landscape

\textbf{Latvian name:} Bioloģiskās lauksaimniecības atbalstam pieteikto
lauksaimniecības platību īpatsvars 1,25 km ainavā

\textbf{Procedure:} The cover fraction within a radius of 1250 m around the analysis grid cell
is calculated as the area-weighted sum of the \hyperref[ch06.255]{analysis cells} inside
the buffer, using the workflow \texttt{egvtools::radius\_function()}. During the calculation of the landscape
metric, inverse distance weighted (power = 2) gap filling on the output is
applied to ensure no missing values at the edges. Then the layer is
rewritten to set its name. Finally, the layer is standardised by
subtracting the arithmetic mean and dividing by the root mean squared error.

\begin{Shaded}
\begin{Highlighting}[]
\CommentTok{\# libs {-}{-}{-}{-}}
\ControlFlowTok{if}\NormalTok{(}\SpecialCharTok{!}\FunctionTok{require}\NormalTok{(terra)) \{}\FunctionTok{install.packages}\NormalTok{(}\StringTok{"terra"}\NormalTok{); }\FunctionTok{require}\NormalTok{(terra)\}}
\ControlFlowTok{if}\NormalTok{(}\SpecialCharTok{!}\FunctionTok{require}\NormalTok{(egvtools)) \{remotes}\SpecialCharTok{::}\FunctionTok{install\_github}\NormalTok{(}\StringTok{"aavotins/egvtools"}\NormalTok{); }\FunctionTok{require}\NormalTok{(egvtools)\}}


\CommentTok{\# Templates {-}{-}{-}{-}{-}}
\NormalTok{template100}\OtherTok{=}\FunctionTok{rast}\NormalTok{(}\StringTok{"./Templates/TemplateRasters/LV100m\_10km.tif"}\NormalTok{)}

\CommentTok{\# radii {-}{-}{-}{-}}
\FunctionTok{radius\_function}\NormalTok{(}
 \AttributeTok{kvadrati\_path =} \StringTok{"./Templates/TemplateGrids/tiles/"}\NormalTok{,}
 \AttributeTok{radii\_path   =} \StringTok{"./Templates/TemplateGridPoints/tiles/"}\NormalTok{,}
 \AttributeTok{tikls100\_path =} \StringTok{"./Templates/TemplateGrids/tikls100\_sauzeme.parquet"}\NormalTok{,}
 \AttributeTok{template\_path =} \StringTok{"./Templates/TemplateRasters/LV100m\_10km.tif"}\NormalTok{,}
 \AttributeTok{input\_layers  =} \FunctionTok{c}\NormalTok{(}\StringTok{"./RasterGrids\_100m/2024/RAW/FarmlandSubsidies\_BiologicalSubsidies\_cell.tif"}\NormalTok{),}
 \AttributeTok{layer\_prefixes =} \FunctionTok{c}\NormalTok{(}\StringTok{"FarmlandSubsidies\_BiologicalSubsidies"}\NormalTok{),}
 \AttributeTok{output\_dir   =} \StringTok{"./RasterGrids\_100m/2024/RAW/"}\NormalTok{,}
 \AttributeTok{n\_workers   =} \DecValTok{6}\NormalTok{,}
 \AttributeTok{radii     =} \FunctionTok{c}\NormalTok{(}\StringTok{"r1250"}\NormalTok{),}
 \AttributeTok{radius\_mode  =} \StringTok{"sparse"}\NormalTok{,}
 \AttributeTok{extract\_fun  =} \StringTok{"mean"}\NormalTok{,}
 \AttributeTok{fill\_missing  =} \ConstantTok{TRUE}\NormalTok{,}
 \AttributeTok{IDW\_weight   =} \DecValTok{2}\NormalTok{,}
 \AttributeTok{future\_max\_size =} \DecValTok{40} \SpecialCharTok{*} \DecValTok{1024}\SpecialCharTok{\^{}}\DecValTok{3}\NormalTok{)}


\CommentTok{\# FarmlandSubsidies\_BiologicalSubsidies\_r1250.tif   egv\_257}
\NormalTok{slanis}\OtherTok{=}\FunctionTok{rast}\NormalTok{(}\StringTok{"./RasterGrids\_100m/2024/RAW/FarmlandSubsidies\_BiologicalSubsidies\_r1250.tif"}\NormalTok{)}
\FunctionTok{names}\NormalTok{(slanis)}\OtherTok{=}\StringTok{"egv\_257"}
\NormalTok{slanis2}\OtherTok{=}\FunctionTok{project}\NormalTok{(slanis,template100)}
\FunctionTok{writeRaster}\NormalTok{(slanis2,}
      \StringTok{"./RasterGrids\_100m/2024/RAW/FarmlandSubsidies\_BiologicalSubsidies\_r1250.tif"}\NormalTok{,}
      \AttributeTok{overwrite=}\ConstantTok{TRUE}\NormalTok{)}

\CommentTok{\# standardisation {-}{-}{-}{-}}
\ControlFlowTok{if}\NormalTok{(}\SpecialCharTok{!}\FunctionTok{require}\NormalTok{(terra)) \{}\FunctionTok{install.packages}\NormalTok{(}\StringTok{"terra"}\NormalTok{); }\FunctionTok{require}\NormalTok{(terra)\}}
\ControlFlowTok{if}\NormalTok{(}\SpecialCharTok{!}\FunctionTok{require}\NormalTok{(tidyverse)) \{}\FunctionTok{install.packages}\NormalTok{(}\StringTok{"tidyverse"}\NormalTok{); }\FunctionTok{require}\NormalTok{(tidyverse)\}}

\NormalTok{nosaukums}\OtherTok{=}\StringTok{"FarmlandSubsidies\_BiologicalSubsidies\_r1250.tif"}
\NormalTok{ielasisanas\_cels}\OtherTok{=}\FunctionTok{paste0}\NormalTok{(}\StringTok{"./RasterGrids\_100m/2024/RAW/"}\NormalTok{,nosaukums)}
\NormalTok{saglabasanas\_cels}\OtherTok{=}\FunctionTok{paste0}\NormalTok{(}\StringTok{"./RasterGrids\_100m/2024/Scaled/"}\NormalTok{,nosaukums)}
\NormalTok{slanis}\OtherTok{=}\FunctionTok{rast}\NormalTok{(ielasisanas\_cels)}
\NormalTok{videjais}\OtherTok{=}\FunctionTok{global}\NormalTok{(slanis,}\AttributeTok{fun=}\StringTok{"mean"}\NormalTok{,}\AttributeTok{na.rm=}\ConstantTok{TRUE}\NormalTok{)}
\NormalTok{centrets}\OtherTok{=}\NormalTok{slanis}\SpecialCharTok{{-}}\NormalTok{videjais[,}\DecValTok{1}\NormalTok{]}
\NormalTok{standartnovirze}\OtherTok{=}\NormalTok{terra}\SpecialCharTok{::}\FunctionTok{global}\NormalTok{(centrets,}\AttributeTok{fun=}\StringTok{"rms"}\NormalTok{,}\AttributeTok{na.rm=}\ConstantTok{TRUE}\NormalTok{)}
\NormalTok{merogots}\OtherTok{=}\NormalTok{centrets}\SpecialCharTok{/}\NormalTok{standartnovirze[,}\DecValTok{1}\NormalTok{]}
\FunctionTok{writeRaster}\NormalTok{(merogots,}
      \AttributeTok{filename=}\NormalTok{saglabasanas\_cels,}
      \AttributeTok{overwrite=}\ConstantTok{TRUE}\NormalTok{)}
\end{Highlighting}
\end{Shaded}

\section{FarmlandSubsidies\_BiologicalSubsidies\_r3000}\label{ch06.258}

\textbf{filename:} \texttt{FarmlandSubsidies\_BiologicalSubsidies\_r3000.tif}

\textbf{layername:} \texttt{egv\_258}

\textbf{English name:} Fractional cover of Farmland receiving Subsidies for
Biological Agriculture within the 3 km landscape

\textbf{Latvian name:} Bioloģiskās lauksaimniecības atbalstam pieteikto
lauksaimniecības platību īpatsvars 3 km ainavā

\textbf{Procedure:} The cover fraction within a radius of 3000 m around the analysis grid cell
is calculated as the area-weighted sum of the \hyperref[ch06.255]{analysis cells} inside
the buffer, using the workflow \texttt{egvtools::radius\_function()}. During the calculation of the landscape
metric, inverse distance weighted (power = 2) gap filling on the output is
applied to ensure no missing values at the edges. Then the layer is
rewritten to set its name. Finally, the layer is standardised by
subtracting the arithmetic mean and dividing by the root mean squared error.

\begin{Shaded}
\begin{Highlighting}[]
\CommentTok{\# libs {-}{-}{-}{-}}
\ControlFlowTok{if}\NormalTok{(}\SpecialCharTok{!}\FunctionTok{require}\NormalTok{(terra)) \{}\FunctionTok{install.packages}\NormalTok{(}\StringTok{"terra"}\NormalTok{); }\FunctionTok{require}\NormalTok{(terra)\}}
\ControlFlowTok{if}\NormalTok{(}\SpecialCharTok{!}\FunctionTok{require}\NormalTok{(egvtools)) \{remotes}\SpecialCharTok{::}\FunctionTok{install\_github}\NormalTok{(}\StringTok{"aavotins/egvtools"}\NormalTok{); }\FunctionTok{require}\NormalTok{(egvtools)\}}


\CommentTok{\# Templates {-}{-}{-}{-}{-}}
\NormalTok{template100}\OtherTok{=}\FunctionTok{rast}\NormalTok{(}\StringTok{"./Templates/TemplateRasters/LV100m\_10km.tif"}\NormalTok{)}

\CommentTok{\# radii {-}{-}{-}{-}}
\FunctionTok{radius\_function}\NormalTok{(}
 \AttributeTok{kvadrati\_path =} \StringTok{"./Templates/TemplateGrids/tiles/"}\NormalTok{,}
 \AttributeTok{radii\_path   =} \StringTok{"./Templates/TemplateGridPoints/tiles/"}\NormalTok{,}
 \AttributeTok{tikls100\_path =} \StringTok{"./Templates/TemplateGrids/tikls100\_sauzeme.parquet"}\NormalTok{,}
 \AttributeTok{template\_path =} \StringTok{"./Templates/TemplateRasters/LV100m\_10km.tif"}\NormalTok{,}
 \AttributeTok{input\_layers  =} \FunctionTok{c}\NormalTok{(}\StringTok{"./RasterGrids\_100m/2024/RAW/FarmlandSubsidies\_BiologicalSubsidies\_cell.tif"}\NormalTok{),}
 \AttributeTok{layer\_prefixes =} \FunctionTok{c}\NormalTok{(}\StringTok{"FarmlandSubsidies\_BiologicalSubsidies"}\NormalTok{),}
 \AttributeTok{output\_dir   =} \StringTok{"./RasterGrids\_100m/2024/RAW/"}\NormalTok{,}
 \AttributeTok{n\_workers   =} \DecValTok{6}\NormalTok{,}
 \AttributeTok{radii     =} \FunctionTok{c}\NormalTok{(}\StringTok{"r3000"}\NormalTok{),}
 \AttributeTok{radius\_mode  =} \StringTok{"sparse"}\NormalTok{,}
 \AttributeTok{extract\_fun  =} \StringTok{"mean"}\NormalTok{,}
 \AttributeTok{fill\_missing  =} \ConstantTok{TRUE}\NormalTok{,}
 \AttributeTok{IDW\_weight   =} \DecValTok{2}\NormalTok{,}
 \AttributeTok{future\_max\_size =} \DecValTok{40} \SpecialCharTok{*} \DecValTok{1024}\SpecialCharTok{\^{}}\DecValTok{3}\NormalTok{)}


\CommentTok{\# FarmlandSubsidies\_BiologicalSubsidies\_r3000.tif   egv\_258}
\NormalTok{slanis}\OtherTok{=}\FunctionTok{rast}\NormalTok{(}\StringTok{"./RasterGrids\_100m/2024/RAW/FarmlandSubsidies\_BiologicalSubsidies\_r3000.tif"}\NormalTok{)}
\FunctionTok{names}\NormalTok{(slanis)}\OtherTok{=}\StringTok{"egv\_258"}
\NormalTok{slanis2}\OtherTok{=}\FunctionTok{project}\NormalTok{(slanis,template100)}
\FunctionTok{writeRaster}\NormalTok{(slanis2,}
      \StringTok{"./RasterGrids\_100m/2024/RAW/FarmlandSubsidies\_BiologicalSubsidies\_r3000.tif"}\NormalTok{,}
      \AttributeTok{overwrite=}\ConstantTok{TRUE}\NormalTok{)}

\CommentTok{\# standardisation {-}{-}{-}{-}}
\ControlFlowTok{if}\NormalTok{(}\SpecialCharTok{!}\FunctionTok{require}\NormalTok{(terra)) \{}\FunctionTok{install.packages}\NormalTok{(}\StringTok{"terra"}\NormalTok{); }\FunctionTok{require}\NormalTok{(terra)\}}
\ControlFlowTok{if}\NormalTok{(}\SpecialCharTok{!}\FunctionTok{require}\NormalTok{(tidyverse)) \{}\FunctionTok{install.packages}\NormalTok{(}\StringTok{"tidyverse"}\NormalTok{); }\FunctionTok{require}\NormalTok{(tidyverse)\}}

\NormalTok{nosaukums}\OtherTok{=}\StringTok{"FarmlandSubsidies\_BiologicalSubsidies\_r3000.tif"}
\NormalTok{ielasisanas\_cels}\OtherTok{=}\FunctionTok{paste0}\NormalTok{(}\StringTok{"./RasterGrids\_100m/2024/RAW/"}\NormalTok{,nosaukums)}
\NormalTok{saglabasanas\_cels}\OtherTok{=}\FunctionTok{paste0}\NormalTok{(}\StringTok{"./RasterGrids\_100m/2024/Scaled/"}\NormalTok{,nosaukums)}
\NormalTok{slanis}\OtherTok{=}\FunctionTok{rast}\NormalTok{(ielasisanas\_cels)}
\NormalTok{videjais}\OtherTok{=}\FunctionTok{global}\NormalTok{(slanis,}\AttributeTok{fun=}\StringTok{"mean"}\NormalTok{,}\AttributeTok{na.rm=}\ConstantTok{TRUE}\NormalTok{)}
\NormalTok{centrets}\OtherTok{=}\NormalTok{slanis}\SpecialCharTok{{-}}\NormalTok{videjais[,}\DecValTok{1}\NormalTok{]}
\NormalTok{standartnovirze}\OtherTok{=}\NormalTok{terra}\SpecialCharTok{::}\FunctionTok{global}\NormalTok{(centrets,}\AttributeTok{fun=}\StringTok{"rms"}\NormalTok{,}\AttributeTok{na.rm=}\ConstantTok{TRUE}\NormalTok{)}
\NormalTok{merogots}\OtherTok{=}\NormalTok{centrets}\SpecialCharTok{/}\NormalTok{standartnovirze[,}\DecValTok{1}\NormalTok{]}
\FunctionTok{writeRaster}\NormalTok{(merogots,}
      \AttributeTok{filename=}\NormalTok{saglabasanas\_cels,}
      \AttributeTok{overwrite=}\ConstantTok{TRUE}\NormalTok{)}
\end{Highlighting}
\end{Shaded}

\section{FarmlandSubsidies\_BiologicalSubsidies\_r10000}\label{ch06.259}

\textbf{filename:} \texttt{FarmlandSubsidies\_BiologicalSubsidies\_r10000.tif}

\textbf{layername:} \texttt{egv\_259}

\textbf{English name:} Fractional cover of Farmland receiving Subsidies for
Biological Agriculture within the 10 km landscape

\textbf{Latvian name:} Bioloģiskās lauksaimniecības atbalstam pieteikto
lauksaimniecības platību īpatsvars 10 km ainavā

\textbf{Procedure:} The cover fraction within a radius of 10000 m around the analysis grid cell
is calculated as the area-weighted sum of the \hyperref[ch06.255]{analysis cells} inside
the buffer, using the workflow \texttt{egvtools::radius\_function()}. During the calculation of the landscape
metric, inverse distance weighted (power = 2) gap filling on the output is
applied to ensure no missing values at the edges. Then the layer is
rewritten to set its name. Finally, the layer is standardised by
subtracting the arithmetic mean and dividing by the root mean squared error.

\begin{Shaded}
\begin{Highlighting}[]
\CommentTok{\# libs {-}{-}{-}{-}}
\ControlFlowTok{if}\NormalTok{(}\SpecialCharTok{!}\FunctionTok{require}\NormalTok{(terra)) \{}\FunctionTok{install.packages}\NormalTok{(}\StringTok{"terra"}\NormalTok{); }\FunctionTok{require}\NormalTok{(terra)\}}
\ControlFlowTok{if}\NormalTok{(}\SpecialCharTok{!}\FunctionTok{require}\NormalTok{(egvtools)) \{remotes}\SpecialCharTok{::}\FunctionTok{install\_github}\NormalTok{(}\StringTok{"aavotins/egvtools"}\NormalTok{); }\FunctionTok{require}\NormalTok{(egvtools)\}}


\CommentTok{\# Templates {-}{-}{-}{-}{-}}
\NormalTok{template100}\OtherTok{=}\FunctionTok{rast}\NormalTok{(}\StringTok{"./Templates/TemplateRasters/LV100m\_10km.tif"}\NormalTok{)}

\CommentTok{\# radii {-}{-}{-}{-}}
\FunctionTok{radius\_function}\NormalTok{(}
 \AttributeTok{kvadrati\_path =} \StringTok{"./Templates/TemplateGrids/tiles/"}\NormalTok{,}
 \AttributeTok{radii\_path   =} \StringTok{"./Templates/TemplateGridPoints/tiles/"}\NormalTok{,}
 \AttributeTok{tikls100\_path =} \StringTok{"./Templates/TemplateGrids/tikls100\_sauzeme.parquet"}\NormalTok{,}
 \AttributeTok{template\_path =} \StringTok{"./Templates/TemplateRasters/LV100m\_10km.tif"}\NormalTok{,}
 \AttributeTok{input\_layers  =} \FunctionTok{c}\NormalTok{(}\StringTok{"./RasterGrids\_100m/2024/RAW/FarmlandSubsidies\_BiologicalSubsidies\_cell.tif"}\NormalTok{),}
 \AttributeTok{layer\_prefixes =} \FunctionTok{c}\NormalTok{(}\StringTok{"FarmlandSubsidies\_BiologicalSubsidies"}\NormalTok{),}
 \AttributeTok{output\_dir   =} \StringTok{"./RasterGrids\_100m/2024/RAW/"}\NormalTok{,}
 \AttributeTok{n\_workers   =} \DecValTok{6}\NormalTok{,}
 \AttributeTok{radii     =} \FunctionTok{c}\NormalTok{(}\StringTok{"r10000"}\NormalTok{),}
 \AttributeTok{radius\_mode  =} \StringTok{"sparse"}\NormalTok{,}
 \AttributeTok{extract\_fun  =} \StringTok{"mean"}\NormalTok{,}
 \AttributeTok{fill\_missing  =} \ConstantTok{TRUE}\NormalTok{,}
 \AttributeTok{IDW\_weight   =} \DecValTok{2}\NormalTok{,}
 \AttributeTok{future\_max\_size =} \DecValTok{40} \SpecialCharTok{*} \DecValTok{1024}\SpecialCharTok{\^{}}\DecValTok{3}\NormalTok{)}


\CommentTok{\# FarmlandSubsidies\_BiologicalSubsidies\_r10000.tif  egv\_259}
\NormalTok{slanis}\OtherTok{=}\FunctionTok{rast}\NormalTok{(}\StringTok{"./RasterGrids\_100m/2024/RAW/FarmlandSubsidies\_BiologicalSubsidies\_r10000.tif"}\NormalTok{)}
\FunctionTok{names}\NormalTok{(slanis)}\OtherTok{=}\StringTok{"egv\_259"}
\NormalTok{slanis2}\OtherTok{=}\FunctionTok{project}\NormalTok{(slanis,template100)}
\FunctionTok{writeRaster}\NormalTok{(slanis2,}
      \StringTok{"./RasterGrids\_100m/2024/RAW/FarmlandSubsidies\_BiologicalSubsidies\_r10000.tif"}\NormalTok{,}
      \AttributeTok{overwrite=}\ConstantTok{TRUE}\NormalTok{)}

\CommentTok{\# standardisation {-}{-}{-}{-}}
\ControlFlowTok{if}\NormalTok{(}\SpecialCharTok{!}\FunctionTok{require}\NormalTok{(terra)) \{}\FunctionTok{install.packages}\NormalTok{(}\StringTok{"terra"}\NormalTok{); }\FunctionTok{require}\NormalTok{(terra)\}}
\ControlFlowTok{if}\NormalTok{(}\SpecialCharTok{!}\FunctionTok{require}\NormalTok{(tidyverse)) \{}\FunctionTok{install.packages}\NormalTok{(}\StringTok{"tidyverse"}\NormalTok{); }\FunctionTok{require}\NormalTok{(tidyverse)\}}

\NormalTok{nosaukums}\OtherTok{=}\StringTok{"FarmlandSubsidies\_BiologicalSubsidies\_r10000.tif"}
\NormalTok{ielasisanas\_cels}\OtherTok{=}\FunctionTok{paste0}\NormalTok{(}\StringTok{"./RasterGrids\_100m/2024/RAW/"}\NormalTok{,nosaukums)}
\NormalTok{saglabasanas\_cels}\OtherTok{=}\FunctionTok{paste0}\NormalTok{(}\StringTok{"./RasterGrids\_100m/2024/Scaled/"}\NormalTok{,nosaukums)}
\NormalTok{slanis}\OtherTok{=}\FunctionTok{rast}\NormalTok{(ielasisanas\_cels)}
\NormalTok{videjais}\OtherTok{=}\FunctionTok{global}\NormalTok{(slanis,}\AttributeTok{fun=}\StringTok{"mean"}\NormalTok{,}\AttributeTok{na.rm=}\ConstantTok{TRUE}\NormalTok{)}
\NormalTok{centrets}\OtherTok{=}\NormalTok{slanis}\SpecialCharTok{{-}}\NormalTok{videjais[,}\DecValTok{1}\NormalTok{]}
\NormalTok{standartnovirze}\OtherTok{=}\NormalTok{terra}\SpecialCharTok{::}\FunctionTok{global}\NormalTok{(centrets,}\AttributeTok{fun=}\StringTok{"rms"}\NormalTok{,}\AttributeTok{na.rm=}\ConstantTok{TRUE}\NormalTok{)}
\NormalTok{merogots}\OtherTok{=}\NormalTok{centrets}\SpecialCharTok{/}\NormalTok{standartnovirze[,}\DecValTok{1}\NormalTok{]}
\FunctionTok{writeRaster}\NormalTok{(merogots,}
      \AttributeTok{filename=}\NormalTok{saglabasanas\_cels,}
      \AttributeTok{overwrite=}\ConstantTok{TRUE}\NormalTok{)}
\end{Highlighting}
\end{Shaded}

\section{FarmlandTrees\_PermanentCrops\_cell}\label{ch06.260}

\textbf{filename:} \texttt{FarmlandTrees\_PermanentCrops\_cell.tif}

\textbf{layername:} \texttt{egv\_260}

\textbf{English name:} Fractional cover of Permanent Crops within the analysis cell
(1 ha)

\textbf{Latvian name:} Ilggadīgo kultūraugu platības īpatsvars analīzes šūnā (1 ha)

\textbf{Procedure:} First, agricultural parcels declared as permanent crops are
selected from the \hyperref[Ch04.02]{Rural Support Service's information on declared
fields}. Geometries are then rasterised to input resolution, ensuring
value 1 at the polygon locations and value 0 elsewhere. Rasterisation is
performed using the workflow \texttt{egvtools::polygon2input()}. Once rasterised, the
layer is aggregated to EGV resolution using the workflow \texttt{egvtools::input2egv()},
which calculates the arithmetic mean and thus
results in a cover fraction. During aggregation, inverse
distance weighted (power = 2) gap filling on the output is applied to
ensure no missing values at the edges. Finally, the layer is standardised
by subtracting the arithmetic mean and dividing by the root mean squared error.

\begin{Shaded}
\begin{Highlighting}[]
\CommentTok{\# libs {-}{-}{-}{-}}
\ControlFlowTok{if}\NormalTok{(}\SpecialCharTok{!}\FunctionTok{require}\NormalTok{(egvtools)) \{remotes}\SpecialCharTok{::}\FunctionTok{install\_github}\NormalTok{(}\StringTok{"aavotins/egvtools"}\NormalTok{); }\FunctionTok{require}\NormalTok{(egvtools)\}}
\ControlFlowTok{if}\NormalTok{(}\SpecialCharTok{!}\FunctionTok{require}\NormalTok{(terra)) \{}\FunctionTok{install.packages}\NormalTok{(}\StringTok{"terra"}\NormalTok{); }\FunctionTok{require}\NormalTok{(terra)\}}
\ControlFlowTok{if}\NormalTok{(}\SpecialCharTok{!}\FunctionTok{require}\NormalTok{(sf)) \{}\FunctionTok{install.packages}\NormalTok{(}\StringTok{"sf"}\NormalTok{); }\FunctionTok{require}\NormalTok{(sf)\}}
\ControlFlowTok{if}\NormalTok{(}\SpecialCharTok{!}\FunctionTok{require}\NormalTok{(tidyverse)) \{}\FunctionTok{install.packages}\NormalTok{(}\StringTok{"tidyverse"}\NormalTok{); }\FunctionTok{require}\NormalTok{(tidyverse)\}}
\ControlFlowTok{if}\NormalTok{(}\SpecialCharTok{!}\FunctionTok{require}\NormalTok{(sfarrow)) \{}\FunctionTok{install.packages}\NormalTok{(}\StringTok{"sfarrow"}\NormalTok{); }\FunctionTok{require}\NormalTok{(sfarrow)\}}
\ControlFlowTok{if}\NormalTok{(}\SpecialCharTok{!}\FunctionTok{require}\NormalTok{(readxl)) \{}\FunctionTok{install.packages}\NormalTok{(}\StringTok{"readxl"}\NormalTok{); }\FunctionTok{require}\NormalTok{(readxl)\}}
\ControlFlowTok{if}\NormalTok{(}\SpecialCharTok{!}\FunctionTok{require}\NormalTok{(raster)) \{}\FunctionTok{install.packages}\NormalTok{(}\StringTok{"raster"}\NormalTok{); }\FunctionTok{require}\NormalTok{(raster)\}}
\ControlFlowTok{if}\NormalTok{(}\SpecialCharTok{!}\FunctionTok{require}\NormalTok{(fasterize)) \{}\FunctionTok{install.packages}\NormalTok{(}\StringTok{"fasterize"}\NormalTok{); }\FunctionTok{require}\NormalTok{(fasterize)\}}

\CommentTok{\# templates {-}{-}{-}{-}}
\NormalTok{template100}\OtherTok{=}\FunctionTok{rast}\NormalTok{(}\StringTok{"./Templates/TemplateRasters/LV100m\_10km.tif"}\NormalTok{)}
\NormalTok{template10}\OtherTok{=}\FunctionTok{rast}\NormalTok{(}\StringTok{"./Templates/TemplateRasters/LV10m\_10km.tif"}\NormalTok{)}
\NormalTok{rastrs10}\OtherTok{=}\FunctionTok{raster}\NormalTok{(template10)}

\NormalTok{nulls10}\OtherTok{=}\FunctionTok{rast}\NormalTok{(}\StringTok{"./Templates/TemplateRasters/nulls\_LV10m\_10km.tif"}\NormalTok{)}
\NormalTok{nulls100}\OtherTok{=}\FunctionTok{rast}\NormalTok{(}\StringTok{"./Templates/TemplateRasters/nulls\_LV100m\_10km.tif"}\NormalTok{)}

\CommentTok{\# codes {-}{-}{-}{-}}
\NormalTok{kodi}\OtherTok{=}\FunctionTok{read\_excel}\NormalTok{(}\StringTok{"./Geodata/2024/LAD/KulturuKodi\_2024.xlsx"}\NormalTok{)}
\NormalTok{kodi}\SpecialCharTok{$}\NormalTok{kods}\OtherTok{=}\FunctionTok{as.character}\NormalTok{(kodi}\SpecialCharTok{$}\NormalTok{kods)}
\CommentTok{\# LAD {-}{-}{-}{-}}
\NormalTok{lad}\OtherTok{=}\NormalTok{sfarrow}\SpecialCharTok{::}\FunctionTok{st\_read\_parquet}\NormalTok{(}\StringTok{"./Geodata/2024/LAD/Lauki\_2024.parquet"}\NormalTok{)}
\NormalTok{lad}\SpecialCharTok{$}\NormalTok{yes}\OtherTok{=}\DecValTok{1}
\NormalTok{lad}\OtherTok{=}\NormalTok{lad }\SpecialCharTok{\%\textgreater{}\%} 
 \FunctionTok{left\_join}\NormalTok{(kodi,}\AttributeTok{by=}\FunctionTok{c}\NormalTok{(}\StringTok{"PRODUCT\_CODE"}\OtherTok{=}\StringTok{"kods"}\NormalTok{))}

\CommentTok{\# simple landscape {-}{-}{-}{-}}
\NormalTok{simple\_landscape}\OtherTok{=}\FunctionTok{rast}\NormalTok{(}\StringTok{"RasterGrids\_10m/2024/Ainava\_vienk\_mask.tif"}\NormalTok{)}


\CommentTok{\# FarmlandTrees\_PermanentCrops\_cell.tif egv\_260 {-}{-}{-}{-}}
\NormalTok{dati}\OtherTok{=}\NormalTok{lad }\SpecialCharTok{\%\textgreater{}\%} 
 \FunctionTok{filter}\NormalTok{(SDM\_grupa\_sakums }\SpecialCharTok{==} \StringTok{"augļudārzi"}\NormalTok{)}
\FunctionTok{table}\NormalTok{(dati}\SpecialCharTok{$}\NormalTok{SDM\_grupa\_sakums,}\AttributeTok{useNA=}\StringTok{"always"}\NormalTok{)}
\NormalTok{dati}\OtherTok{=}\NormalTok{dati }\SpecialCharTok{\%\textgreater{}\%} 
\NormalTok{ dplyr}\SpecialCharTok{::}\FunctionTok{select}\NormalTok{(yes)}

\NormalTok{topo}\OtherTok{=}\NormalTok{sfarrow}\SpecialCharTok{::}\FunctionTok{st\_read\_parquet}\NormalTok{(}\StringTok{"./Geodata/2024/TopographicMap/LandusA\_COMB.parquet"}\NormalTok{)}
\NormalTok{dati\_topo}\OtherTok{=}\NormalTok{ topo }\SpecialCharTok{\%\textgreater{}\%} 
 \FunctionTok{filter}\NormalTok{(FNAME }\SpecialCharTok{\%in\%} \FunctionTok{c}\NormalTok{(}\StringTok{"poligons\_Augludarzs"}\NormalTok{,}\StringTok{"poligons\_Augļudārzs"}\NormalTok{,}
           \StringTok{"poligons\_Ogulājs"}\NormalTok{,}\StringTok{"poligons\_Ogulajs"}\NormalTok{)) }\SpecialCharTok{\%\textgreater{}\%} 
 \FunctionTok{mutate}\NormalTok{(}\AttributeTok{yes=}\DecValTok{1}\NormalTok{) }\SpecialCharTok{\%\textgreater{}\%} 
\NormalTok{ dplyr}\SpecialCharTok{::}\FunctionTok{select}\NormalTok{(yes)}
\NormalTok{abidati}\OtherTok{=}\FunctionTok{rbind}\NormalTok{(dati,dati\_topo)}

\NormalTok{p2i\_rez}\OtherTok{=}\NormalTok{egvtools}\SpecialCharTok{::}\FunctionTok{polygon2input}\NormalTok{(}\AttributeTok{vector\_data =}\NormalTok{ abidati,}
                \AttributeTok{template\_path =} \StringTok{"./Templates/TemplateRasters/LV10m\_10km.tif"}\NormalTok{,}
                \AttributeTok{out\_path =} \StringTok{"./RasterGrids\_10m/2024/"}\NormalTok{,}
                \AttributeTok{file\_name =} \StringTok{"FarmlandTrees\_PermanentCrops\_input.tif"}\NormalTok{,}
                \AttributeTok{value\_field =} \StringTok{"yes"}\NormalTok{,}
                \AttributeTok{prepare=}\ConstantTok{FALSE}\NormalTok{,}
                \AttributeTok{background\_raster =} \StringTok{"./Templates/TemplateRasters/nulls\_LV10m\_10km.tif"}\NormalTok{,}
                \AttributeTok{plot\_result =} \ConstantTok{TRUE}\NormalTok{)}
\NormalTok{p2i\_rez}
\NormalTok{i2e\_rez}\OtherTok{=}\NormalTok{egvtools}\SpecialCharTok{::}\FunctionTok{input2egv}\NormalTok{(}\AttributeTok{input=}\FunctionTok{paste0}\NormalTok{(}\StringTok{"./RasterGrids\_10m/2024/"}\NormalTok{,}
                     \StringTok{"FarmlandTrees\_PermanentCrops\_input.tif"}\NormalTok{),}
              \AttributeTok{egv\_template=} \StringTok{"./Templates/TemplateRasters/LV100m\_10km.tif"}\NormalTok{,}
              \AttributeTok{summary\_function =} \StringTok{"average"}\NormalTok{,}
              \AttributeTok{missing\_job =} \StringTok{"FillOutput"}\NormalTok{,}
              \AttributeTok{outlocation =} \StringTok{"./RasterGrids\_100m/2024/RAW/"}\NormalTok{,}
              \AttributeTok{outfilename =} \StringTok{"FarmlandTrees\_PermanentCrops\_cell.tif"}\NormalTok{,}
              \AttributeTok{layername =} \StringTok{"egv\_260"}\NormalTok{,}
              \AttributeTok{idw\_weight =} \DecValTok{2}\NormalTok{,}
              \AttributeTok{plot\_gaps =} \ConstantTok{FALSE}\NormalTok{,}\AttributeTok{plot\_final =} \ConstantTok{TRUE}\NormalTok{)}
\NormalTok{i2e\_rez}
\FunctionTok{rm}\NormalTok{(p2i\_rez)}
\FunctionTok{rm}\NormalTok{(i2e\_rez)}
\FunctionTok{rm}\NormalTok{(dati)}
\FunctionTok{rm}\NormalTok{(topo)}
\FunctionTok{rm}\NormalTok{(dati\_topo)}
\FunctionTok{rm}\NormalTok{(abidati)}
\FunctionTok{unlink}\NormalTok{(}\StringTok{"./RasterGrids\_10m/2024/FarmlandTrees\_PermanentCrops\_input.tif"}\NormalTok{)}

\CommentTok{\# standardisation {-}{-}{-}{-}}
\ControlFlowTok{if}\NormalTok{(}\SpecialCharTok{!}\FunctionTok{require}\NormalTok{(terra)) \{}\FunctionTok{install.packages}\NormalTok{(}\StringTok{"terra"}\NormalTok{); }\FunctionTok{require}\NormalTok{(terra)\}}
\ControlFlowTok{if}\NormalTok{(}\SpecialCharTok{!}\FunctionTok{require}\NormalTok{(tidyverse)) \{}\FunctionTok{install.packages}\NormalTok{(}\StringTok{"tidyverse"}\NormalTok{); }\FunctionTok{require}\NormalTok{(tidyverse)\}}

\NormalTok{nosaukums}\OtherTok{=}\StringTok{"FarmlandTrees\_PermanentCrops\_cell.tif"}
\NormalTok{ielasisanas\_cels}\OtherTok{=}\FunctionTok{paste0}\NormalTok{(}\StringTok{"./RasterGrids\_100m/2024/RAW/"}\NormalTok{,nosaukums)}
\NormalTok{saglabasanas\_cels}\OtherTok{=}\FunctionTok{paste0}\NormalTok{(}\StringTok{"./RasterGrids\_100m/2024/Scaled/"}\NormalTok{,nosaukums)}
\NormalTok{slanis}\OtherTok{=}\FunctionTok{rast}\NormalTok{(ielasisanas\_cels)}
\NormalTok{videjais}\OtherTok{=}\FunctionTok{global}\NormalTok{(slanis,}\AttributeTok{fun=}\StringTok{"mean"}\NormalTok{,}\AttributeTok{na.rm=}\ConstantTok{TRUE}\NormalTok{)}
\NormalTok{centrets}\OtherTok{=}\NormalTok{slanis}\SpecialCharTok{{-}}\NormalTok{videjais[,}\DecValTok{1}\NormalTok{]}
\NormalTok{standartnovirze}\OtherTok{=}\NormalTok{terra}\SpecialCharTok{::}\FunctionTok{global}\NormalTok{(centrets,}\AttributeTok{fun=}\StringTok{"rms"}\NormalTok{,}\AttributeTok{na.rm=}\ConstantTok{TRUE}\NormalTok{)}
\NormalTok{merogots}\OtherTok{=}\NormalTok{centrets}\SpecialCharTok{/}\NormalTok{standartnovirze[,}\DecValTok{1}\NormalTok{]}
\FunctionTok{writeRaster}\NormalTok{(merogots,}
      \AttributeTok{filename=}\NormalTok{saglabasanas\_cels,}
      \AttributeTok{overwrite=}\ConstantTok{TRUE}\NormalTok{)}
\end{Highlighting}
\end{Shaded}

\section{FarmlandTrees\_PermanentCrops\_r500}\label{ch06.261}

\textbf{filename:} \texttt{FarmlandTrees\_PermanentCrops\_r500.tif}

\textbf{layername:} \texttt{egv\_261}

\textbf{English name:} Fractional cover of Permanent Crops within the 0.5 km
landscape

\textbf{Latvian name:} Ilggadīgo kultūraugu platības īpatsvars 0,5 km ainavā

\textbf{Procedure:} The cover fraction within a radius of 500 m around the analysis grid cell is
calculated as the area-weighted sum of the \hyperref[ch06.260]{analysis cells} inside the
buffer, using the workflow \texttt{egvtools::radius\_function()}. During the calculation of the landscape metric,
inverse distance weighted (power = 2) gap filling on the output is applied
to ensure no missing values at the edges. Then the layer is rewritten to set
its name. Finally, the layer is standardised by subtracting the arithmetic
mean and dividing by the root mean squared error.

\begin{Shaded}
\begin{Highlighting}[]
\CommentTok{\# libs {-}{-}{-}{-}}
\ControlFlowTok{if}\NormalTok{(}\SpecialCharTok{!}\FunctionTok{require}\NormalTok{(terra)) \{}\FunctionTok{install.packages}\NormalTok{(}\StringTok{"terra"}\NormalTok{); }\FunctionTok{require}\NormalTok{(terra)\}}
\ControlFlowTok{if}\NormalTok{(}\SpecialCharTok{!}\FunctionTok{require}\NormalTok{(egvtools)) \{remotes}\SpecialCharTok{::}\FunctionTok{install\_github}\NormalTok{(}\StringTok{"aavotins/egvtools"}\NormalTok{); }\FunctionTok{require}\NormalTok{(egvtools)\}}


\CommentTok{\# Templates {-}{-}{-}{-}{-}}
\NormalTok{template100}\OtherTok{=}\FunctionTok{rast}\NormalTok{(}\StringTok{"./Templates/TemplateRasters/LV100m\_10km.tif"}\NormalTok{)}

\CommentTok{\# radii {-}{-}{-}{-}}
\FunctionTok{radius\_function}\NormalTok{(}
 \AttributeTok{kvadrati\_path =} \StringTok{"./Templates/TemplateGrids/tiles/"}\NormalTok{,}
 \AttributeTok{radii\_path   =} \StringTok{"./Templates/TemplateGridPoints/tiles/"}\NormalTok{,}
 \AttributeTok{tikls100\_path =} \StringTok{"./Templates/TemplateGrids/tikls100\_sauzeme.parquet"}\NormalTok{,}
 \AttributeTok{template\_path =} \StringTok{"./Templates/TemplateRasters/LV100m\_10km.tif"}\NormalTok{,}
 \AttributeTok{input\_layers  =} \FunctionTok{c}\NormalTok{(}\StringTok{"./RasterGrids\_100m/2024/RAW/FarmlandTrees\_PermanentCrops\_cell.tif"}\NormalTok{),}
 \AttributeTok{layer\_prefixes =} \FunctionTok{c}\NormalTok{(}\StringTok{"FarmlandTrees\_PermanentCrops"}\NormalTok{),}
 \AttributeTok{output\_dir   =} \StringTok{"./RasterGrids\_100m/2024/RAW/"}\NormalTok{,}
 \AttributeTok{n\_workers   =} \DecValTok{6}\NormalTok{,}
 \AttributeTok{radii     =} \FunctionTok{c}\NormalTok{(}\StringTok{"r500"}\NormalTok{),}
 \AttributeTok{radius\_mode  =} \StringTok{"sparse"}\NormalTok{,}
 \AttributeTok{extract\_fun  =} \StringTok{"mean"}\NormalTok{,}
 \AttributeTok{fill\_missing  =} \ConstantTok{TRUE}\NormalTok{,}
 \AttributeTok{IDW\_weight   =} \DecValTok{2}\NormalTok{,}
 \AttributeTok{future\_max\_size =} \DecValTok{40} \SpecialCharTok{*} \DecValTok{1024}\SpecialCharTok{\^{}}\DecValTok{3}\NormalTok{)}


\CommentTok{\# FarmlandTrees\_PermanentCrops\_r500.tif egv\_261}
\NormalTok{slanis}\OtherTok{=}\FunctionTok{rast}\NormalTok{(}\StringTok{"./RasterGrids\_100m/2024/RAW/FarmlandTrees\_PermanentCrops\_r500.tif"}\NormalTok{)}
\FunctionTok{names}\NormalTok{(slanis)}\OtherTok{=}\StringTok{"egv\_261"}
\NormalTok{slanis2}\OtherTok{=}\FunctionTok{project}\NormalTok{(slanis,template100)}
\FunctionTok{writeRaster}\NormalTok{(slanis2,}
      \StringTok{"./RasterGrids\_100m/2024/RAW/FarmlandTrees\_PermanentCrops\_r500.tif"}\NormalTok{,}
      \AttributeTok{overwrite=}\ConstantTok{TRUE}\NormalTok{)}

\CommentTok{\# standardisation {-}{-}{-}{-}}
\ControlFlowTok{if}\NormalTok{(}\SpecialCharTok{!}\FunctionTok{require}\NormalTok{(terra)) \{}\FunctionTok{install.packages}\NormalTok{(}\StringTok{"terra"}\NormalTok{); }\FunctionTok{require}\NormalTok{(terra)\}}
\ControlFlowTok{if}\NormalTok{(}\SpecialCharTok{!}\FunctionTok{require}\NormalTok{(tidyverse)) \{}\FunctionTok{install.packages}\NormalTok{(}\StringTok{"tidyverse"}\NormalTok{); }\FunctionTok{require}\NormalTok{(tidyverse)\}}

\NormalTok{nosaukums}\OtherTok{=}\StringTok{"FarmlandTrees\_PermanentCrops\_r500.tif"}
\NormalTok{ielasisanas\_cels}\OtherTok{=}\FunctionTok{paste0}\NormalTok{(}\StringTok{"./RasterGrids\_100m/2024/RAW/"}\NormalTok{,nosaukums)}
\NormalTok{saglabasanas\_cels}\OtherTok{=}\FunctionTok{paste0}\NormalTok{(}\StringTok{"./RasterGrids\_100m/2024/Scaled/"}\NormalTok{,nosaukums)}
\NormalTok{slanis}\OtherTok{=}\FunctionTok{rast}\NormalTok{(ielasisanas\_cels)}
\NormalTok{videjais}\OtherTok{=}\FunctionTok{global}\NormalTok{(slanis,}\AttributeTok{fun=}\StringTok{"mean"}\NormalTok{,}\AttributeTok{na.rm=}\ConstantTok{TRUE}\NormalTok{)}
\NormalTok{centrets}\OtherTok{=}\NormalTok{slanis}\SpecialCharTok{{-}}\NormalTok{videjais[,}\DecValTok{1}\NormalTok{]}
\NormalTok{standartnovirze}\OtherTok{=}\NormalTok{terra}\SpecialCharTok{::}\FunctionTok{global}\NormalTok{(centrets,}\AttributeTok{fun=}\StringTok{"rms"}\NormalTok{,}\AttributeTok{na.rm=}\ConstantTok{TRUE}\NormalTok{)}
\NormalTok{merogots}\OtherTok{=}\NormalTok{centrets}\SpecialCharTok{/}\NormalTok{standartnovirze[,}\DecValTok{1}\NormalTok{]}
\FunctionTok{writeRaster}\NormalTok{(merogots,}
      \AttributeTok{filename=}\NormalTok{saglabasanas\_cels,}
      \AttributeTok{overwrite=}\ConstantTok{TRUE}\NormalTok{)}
\end{Highlighting}
\end{Shaded}

\section{FarmlandTrees\_PermanentCrops\_r1250}\label{ch06.262}

\textbf{filename:} \texttt{FarmlandTrees\_PermanentCrops\_r1250.tif}

\textbf{layername:} \texttt{egv\_262}

\textbf{English name:} Fractional cover of Permanent Crops within the 1.25 km
landscape

\textbf{Latvian name:} Ilggadīgo kultūraugu platības īpatsvars 1,25 km ainavā

\textbf{Procedure:} The cover fraction within a radius of 1250 m around the analysis grid cell
is calculated as the area-weighted sum of the \hyperref[ch06.260]{analysis cells} inside
the buffer, using the workflow \texttt{egvtools::radius\_function()}. During the calculation of the landscape
metric, inverse distance weighted (power = 2) gap filling on the output is
applied to ensure no missing values at the edges. Then the layer is
rewritten to set its name. Finally, the layer is standardised by
subtracting the arithmetic mean and dividing by the root mean squared error.

\begin{Shaded}
\begin{Highlighting}[]
\CommentTok{\# libs {-}{-}{-}{-}}
\ControlFlowTok{if}\NormalTok{(}\SpecialCharTok{!}\FunctionTok{require}\NormalTok{(terra)) \{}\FunctionTok{install.packages}\NormalTok{(}\StringTok{"terra"}\NormalTok{); }\FunctionTok{require}\NormalTok{(terra)\}}
\ControlFlowTok{if}\NormalTok{(}\SpecialCharTok{!}\FunctionTok{require}\NormalTok{(egvtools)) \{remotes}\SpecialCharTok{::}\FunctionTok{install\_github}\NormalTok{(}\StringTok{"aavotins/egvtools"}\NormalTok{); }\FunctionTok{require}\NormalTok{(egvtools)\}}


\CommentTok{\# Templates {-}{-}{-}{-}{-}}
\NormalTok{template100}\OtherTok{=}\FunctionTok{rast}\NormalTok{(}\StringTok{"./Templates/TemplateRasters/LV100m\_10km.tif"}\NormalTok{)}

\CommentTok{\# radii {-}{-}{-}{-}}
\FunctionTok{radius\_function}\NormalTok{(}
 \AttributeTok{kvadrati\_path =} \StringTok{"./Templates/TemplateGrids/tiles/"}\NormalTok{,}
 \AttributeTok{radii\_path   =} \StringTok{"./Templates/TemplateGridPoints/tiles/"}\NormalTok{,}
 \AttributeTok{tikls100\_path =} \StringTok{"./Templates/TemplateGrids/tikls100\_sauzeme.parquet"}\NormalTok{,}
 \AttributeTok{template\_path =} \StringTok{"./Templates/TemplateRasters/LV100m\_10km.tif"}\NormalTok{,}
 \AttributeTok{input\_layers  =} \FunctionTok{c}\NormalTok{(}\StringTok{"./RasterGrids\_100m/2024/RAW/FarmlandTrees\_PermanentCrops\_cell.tif"}\NormalTok{),}
 \AttributeTok{layer\_prefixes =} \FunctionTok{c}\NormalTok{(}\StringTok{"FarmlandTrees\_PermanentCrops"}\NormalTok{),}
 \AttributeTok{output\_dir   =} \StringTok{"./RasterGrids\_100m/2024/RAW/"}\NormalTok{,}
 \AttributeTok{n\_workers   =} \DecValTok{6}\NormalTok{,}
 \AttributeTok{radii     =} \FunctionTok{c}\NormalTok{(}\StringTok{"r1250"}\NormalTok{),}
 \AttributeTok{radius\_mode  =} \StringTok{"sparse"}\NormalTok{,}
 \AttributeTok{extract\_fun  =} \StringTok{"mean"}\NormalTok{,}
 \AttributeTok{fill\_missing  =} \ConstantTok{TRUE}\NormalTok{,}
 \AttributeTok{IDW\_weight   =} \DecValTok{2}\NormalTok{,}
 \AttributeTok{future\_max\_size =} \DecValTok{40} \SpecialCharTok{*} \DecValTok{1024}\SpecialCharTok{\^{}}\DecValTok{3}\NormalTok{)}


\CommentTok{\# FarmlandTrees\_PermanentCrops\_r1250.tif    egv\_262}
\NormalTok{slanis}\OtherTok{=}\FunctionTok{rast}\NormalTok{(}\StringTok{"./RasterGrids\_100m/2024/RAW/FarmlandTrees\_PermanentCrops\_r1250.tif"}\NormalTok{)}
\FunctionTok{names}\NormalTok{(slanis)}\OtherTok{=}\StringTok{"egv\_262"}
\NormalTok{slanis2}\OtherTok{=}\FunctionTok{project}\NormalTok{(slanis,template100)}
\FunctionTok{writeRaster}\NormalTok{(slanis2,}
      \StringTok{"./RasterGrids\_100m/2024/RAW/FarmlandTrees\_PermanentCrops\_r1250.tif"}\NormalTok{,}
      \AttributeTok{overwrite=}\ConstantTok{TRUE}\NormalTok{)}

\CommentTok{\# standardisation {-}{-}{-}{-}}
\ControlFlowTok{if}\NormalTok{(}\SpecialCharTok{!}\FunctionTok{require}\NormalTok{(terra)) \{}\FunctionTok{install.packages}\NormalTok{(}\StringTok{"terra"}\NormalTok{); }\FunctionTok{require}\NormalTok{(terra)\}}
\ControlFlowTok{if}\NormalTok{(}\SpecialCharTok{!}\FunctionTok{require}\NormalTok{(tidyverse)) \{}\FunctionTok{install.packages}\NormalTok{(}\StringTok{"tidyverse"}\NormalTok{); }\FunctionTok{require}\NormalTok{(tidyverse)\}}

\NormalTok{nosaukums}\OtherTok{=}\StringTok{"FarmlandTrees\_PermanentCrops\_r1250.tif"}
\NormalTok{ielasisanas\_cels}\OtherTok{=}\FunctionTok{paste0}\NormalTok{(}\StringTok{"./RasterGrids\_100m/2024/RAW/"}\NormalTok{,nosaukums)}
\NormalTok{saglabasanas\_cels}\OtherTok{=}\FunctionTok{paste0}\NormalTok{(}\StringTok{"./RasterGrids\_100m/2024/Scaled/"}\NormalTok{,nosaukums)}
\NormalTok{slanis}\OtherTok{=}\FunctionTok{rast}\NormalTok{(ielasisanas\_cels)}
\NormalTok{videjais}\OtherTok{=}\FunctionTok{global}\NormalTok{(slanis,}\AttributeTok{fun=}\StringTok{"mean"}\NormalTok{,}\AttributeTok{na.rm=}\ConstantTok{TRUE}\NormalTok{)}
\NormalTok{centrets}\OtherTok{=}\NormalTok{slanis}\SpecialCharTok{{-}}\NormalTok{videjais[,}\DecValTok{1}\NormalTok{]}
\NormalTok{standartnovirze}\OtherTok{=}\NormalTok{terra}\SpecialCharTok{::}\FunctionTok{global}\NormalTok{(centrets,}\AttributeTok{fun=}\StringTok{"rms"}\NormalTok{,}\AttributeTok{na.rm=}\ConstantTok{TRUE}\NormalTok{)}
\NormalTok{merogots}\OtherTok{=}\NormalTok{centrets}\SpecialCharTok{/}\NormalTok{standartnovirze[,}\DecValTok{1}\NormalTok{]}
\FunctionTok{writeRaster}\NormalTok{(merogots,}
      \AttributeTok{filename=}\NormalTok{saglabasanas\_cels,}
      \AttributeTok{overwrite=}\ConstantTok{TRUE}\NormalTok{)}
\end{Highlighting}
\end{Shaded}

\section{FarmlandTrees\_PermanentCrops\_r3000}\label{ch06.263}

\textbf{filename:} \texttt{FarmlandTrees\_PermanentCrops\_r3000.tif}

\textbf{layername:} \texttt{egv\_263}

\textbf{English name:} Fractional cover of Permanent Crops within the 3 km landscape

\textbf{Latvian name:} Ilggadīgo kultūraugu platības īpatsvars 3 km ainavā

\textbf{Procedure:} The cover fraction within a radius of 3000 m around the analysis grid cell
is calculated as the area-weighted sum of the \hyperref[ch06.260]{analysis cells} inside
the buffer, using the workflow \texttt{egvtools::radius\_function()}. During the calculation of the landscape
metric, inverse distance weighted (power = 2) gap filling on the output is
applied to ensure no missing values at the edges. Then the layer is
rewritten to set its name. Finally, the layer is standardised by
subtracting the arithmetic mean and dividing by the root mean squared error.

\begin{Shaded}
\begin{Highlighting}[]
\CommentTok{\# libs {-}{-}{-}{-}}
\ControlFlowTok{if}\NormalTok{(}\SpecialCharTok{!}\FunctionTok{require}\NormalTok{(terra)) \{}\FunctionTok{install.packages}\NormalTok{(}\StringTok{"terra"}\NormalTok{); }\FunctionTok{require}\NormalTok{(terra)\}}
\ControlFlowTok{if}\NormalTok{(}\SpecialCharTok{!}\FunctionTok{require}\NormalTok{(egvtools)) \{remotes}\SpecialCharTok{::}\FunctionTok{install\_github}\NormalTok{(}\StringTok{"aavotins/egvtools"}\NormalTok{); }\FunctionTok{require}\NormalTok{(egvtools)\}}


\CommentTok{\# Templates {-}{-}{-}{-}{-}}
\NormalTok{template100}\OtherTok{=}\FunctionTok{rast}\NormalTok{(}\StringTok{"./Templates/TemplateRasters/LV100m\_10km.tif"}\NormalTok{)}

\CommentTok{\# radii {-}{-}{-}{-}}
\FunctionTok{radius\_function}\NormalTok{(}
 \AttributeTok{kvadrati\_path =} \StringTok{"./Templates/TemplateGrids/tiles/"}\NormalTok{,}
 \AttributeTok{radii\_path   =} \StringTok{"./Templates/TemplateGridPoints/tiles/"}\NormalTok{,}
 \AttributeTok{tikls100\_path =} \StringTok{"./Templates/TemplateGrids/tikls100\_sauzeme.parquet"}\NormalTok{,}
 \AttributeTok{template\_path =} \StringTok{"./Templates/TemplateRasters/LV100m\_10km.tif"}\NormalTok{,}
 \AttributeTok{input\_layers  =} \FunctionTok{c}\NormalTok{(}\StringTok{"./RasterGrids\_100m/2024/RAW/FarmlandTrees\_PermanentCrops\_cell.tif"}\NormalTok{),}
 \AttributeTok{layer\_prefixes =} \FunctionTok{c}\NormalTok{(}\StringTok{"FarmlandTrees\_PermanentCrops"}\NormalTok{),}
 \AttributeTok{output\_dir   =} \StringTok{"./RasterGrids\_100m/2024/RAW/"}\NormalTok{,}
 \AttributeTok{n\_workers   =} \DecValTok{6}\NormalTok{,}
 \AttributeTok{radii     =} \FunctionTok{c}\NormalTok{(}\StringTok{"r3000"}\NormalTok{),}
 \AttributeTok{radius\_mode  =} \StringTok{"sparse"}\NormalTok{,}
 \AttributeTok{extract\_fun  =} \StringTok{"mean"}\NormalTok{,}
 \AttributeTok{fill\_missing  =} \ConstantTok{TRUE}\NormalTok{,}
 \AttributeTok{IDW\_weight   =} \DecValTok{2}\NormalTok{,}
 \AttributeTok{future\_max\_size =} \DecValTok{40} \SpecialCharTok{*} \DecValTok{1024}\SpecialCharTok{\^{}}\DecValTok{3}\NormalTok{)}


\CommentTok{\# FarmlandTrees\_PermanentCrops\_r3000.tif    egv\_263}
\NormalTok{slanis}\OtherTok{=}\FunctionTok{rast}\NormalTok{(}\StringTok{"./RasterGrids\_100m/2024/RAW/FarmlandTrees\_PermanentCrops\_r3000.tif"}\NormalTok{)}
\FunctionTok{names}\NormalTok{(slanis)}\OtherTok{=}\StringTok{"egv\_263"}
\NormalTok{slanis2}\OtherTok{=}\FunctionTok{project}\NormalTok{(slanis,template100)}
\FunctionTok{writeRaster}\NormalTok{(slanis2,}
      \StringTok{"./RasterGrids\_100m/2024/RAW/FarmlandTrees\_PermanentCrops\_r3000.tif"}\NormalTok{,}
      \AttributeTok{overwrite=}\ConstantTok{TRUE}\NormalTok{)}

\CommentTok{\# standardisation {-}{-}{-}{-}}
\ControlFlowTok{if}\NormalTok{(}\SpecialCharTok{!}\FunctionTok{require}\NormalTok{(terra)) \{}\FunctionTok{install.packages}\NormalTok{(}\StringTok{"terra"}\NormalTok{); }\FunctionTok{require}\NormalTok{(terra)\}}
\ControlFlowTok{if}\NormalTok{(}\SpecialCharTok{!}\FunctionTok{require}\NormalTok{(tidyverse)) \{}\FunctionTok{install.packages}\NormalTok{(}\StringTok{"tidyverse"}\NormalTok{); }\FunctionTok{require}\NormalTok{(tidyverse)\}}

\NormalTok{nosaukums}\OtherTok{=}\StringTok{"FarmlandTrees\_PermanentCrops\_r3000.tif"}
\NormalTok{ielasisanas\_cels}\OtherTok{=}\FunctionTok{paste0}\NormalTok{(}\StringTok{"./RasterGrids\_100m/2024/RAW/"}\NormalTok{,nosaukums)}
\NormalTok{saglabasanas\_cels}\OtherTok{=}\FunctionTok{paste0}\NormalTok{(}\StringTok{"./RasterGrids\_100m/2024/Scaled/"}\NormalTok{,nosaukums)}
\NormalTok{slanis}\OtherTok{=}\FunctionTok{rast}\NormalTok{(ielasisanas\_cels)}
\NormalTok{videjais}\OtherTok{=}\FunctionTok{global}\NormalTok{(slanis,}\AttributeTok{fun=}\StringTok{"mean"}\NormalTok{,}\AttributeTok{na.rm=}\ConstantTok{TRUE}\NormalTok{)}
\NormalTok{centrets}\OtherTok{=}\NormalTok{slanis}\SpecialCharTok{{-}}\NormalTok{videjais[,}\DecValTok{1}\NormalTok{]}
\NormalTok{standartnovirze}\OtherTok{=}\NormalTok{terra}\SpecialCharTok{::}\FunctionTok{global}\NormalTok{(centrets,}\AttributeTok{fun=}\StringTok{"rms"}\NormalTok{,}\AttributeTok{na.rm=}\ConstantTok{TRUE}\NormalTok{)}
\NormalTok{merogots}\OtherTok{=}\NormalTok{centrets}\SpecialCharTok{/}\NormalTok{standartnovirze[,}\DecValTok{1}\NormalTok{]}
\FunctionTok{writeRaster}\NormalTok{(merogots,}
      \AttributeTok{filename=}\NormalTok{saglabasanas\_cels,}
      \AttributeTok{overwrite=}\ConstantTok{TRUE}\NormalTok{)}
\end{Highlighting}
\end{Shaded}

\section{FarmlandTrees\_PermanentCrops\_r10000}\label{ch06.264}

\textbf{filename:} \texttt{FarmlandTrees\_PermanentCrops\_r10000.tif}

\textbf{layername:} \texttt{egv\_264}

\textbf{English name:} Fractional cover of Permanent Crops within the 10 km landscape

\textbf{Latvian name:} Ilggadīgo kultūraugu platības īpatsvars 10 km ainavā

\textbf{Procedure:} The cover fraction within a radius of 10000 m around the analysis grid cell
is calculated as the area-weighted sum of the \hyperref[ch06.260]{analysis cells} inside
the buffer, using the workflow \texttt{egvtools::radius\_function()}. During the calculation of the landscape
metric, inverse distance weighted (power = 2) gap filling on the output is
applied to ensure no missing values at the edges. Then the layer is
rewritten to set its name. Finally, the layer is standardised by
subtracting the arithmetic mean and dividing by the root mean squared error.

\begin{Shaded}
\begin{Highlighting}[]
\CommentTok{\# libs {-}{-}{-}{-}}
\ControlFlowTok{if}\NormalTok{(}\SpecialCharTok{!}\FunctionTok{require}\NormalTok{(terra)) \{}\FunctionTok{install.packages}\NormalTok{(}\StringTok{"terra"}\NormalTok{); }\FunctionTok{require}\NormalTok{(terra)\}}
\ControlFlowTok{if}\NormalTok{(}\SpecialCharTok{!}\FunctionTok{require}\NormalTok{(egvtools)) \{remotes}\SpecialCharTok{::}\FunctionTok{install\_github}\NormalTok{(}\StringTok{"aavotins/egvtools"}\NormalTok{); }\FunctionTok{require}\NormalTok{(egvtools)\}}


\CommentTok{\# Templates {-}{-}{-}{-}{-}}
\NormalTok{template100}\OtherTok{=}\FunctionTok{rast}\NormalTok{(}\StringTok{"./Templates/TemplateRasters/LV100m\_10km.tif"}\NormalTok{)}

\CommentTok{\# radii {-}{-}{-}{-}}
\FunctionTok{radius\_function}\NormalTok{(}
 \AttributeTok{kvadrati\_path =} \StringTok{"./Templates/TemplateGrids/tiles/"}\NormalTok{,}
 \AttributeTok{radii\_path   =} \StringTok{"./Templates/TemplateGridPoints/tiles/"}\NormalTok{,}
 \AttributeTok{tikls100\_path =} \StringTok{"./Templates/TemplateGrids/tikls100\_sauzeme.parquet"}\NormalTok{,}
 \AttributeTok{template\_path =} \StringTok{"./Templates/TemplateRasters/LV100m\_10km.tif"}\NormalTok{,}
 \AttributeTok{input\_layers  =} \FunctionTok{c}\NormalTok{(}\StringTok{"./RasterGrids\_100m/2024/RAW/FarmlandTrees\_PermanentCrops\_cell.tif"}\NormalTok{),}
 \AttributeTok{layer\_prefixes =} \FunctionTok{c}\NormalTok{(}\StringTok{"FarmlandTrees\_PermanentCrops"}\NormalTok{),}
 \AttributeTok{output\_dir   =} \StringTok{"./RasterGrids\_100m/2024/RAW/"}\NormalTok{,}
 \AttributeTok{n\_workers   =} \DecValTok{6}\NormalTok{,}
 \AttributeTok{radii     =} \FunctionTok{c}\NormalTok{(}\StringTok{"r10000"}\NormalTok{),}
 \AttributeTok{radius\_mode  =} \StringTok{"sparse"}\NormalTok{,}
 \AttributeTok{extract\_fun  =} \StringTok{"mean"}\NormalTok{,}
 \AttributeTok{fill\_missing  =} \ConstantTok{TRUE}\NormalTok{,}
 \AttributeTok{IDW\_weight   =} \DecValTok{2}\NormalTok{,}
 \AttributeTok{future\_max\_size =} \DecValTok{40} \SpecialCharTok{*} \DecValTok{1024}\SpecialCharTok{\^{}}\DecValTok{3}\NormalTok{)}


\CommentTok{\# FarmlandTrees\_PermanentCrops\_r10000.tif   egv\_264}
\NormalTok{slanis}\OtherTok{=}\FunctionTok{rast}\NormalTok{(}\StringTok{"./RasterGrids\_100m/2024/RAW/FarmlandTrees\_PermanentCrops\_r10000.tif"}\NormalTok{)}
\FunctionTok{names}\NormalTok{(slanis)}\OtherTok{=}\StringTok{"egv\_264"}
\NormalTok{slanis2}\OtherTok{=}\FunctionTok{project}\NormalTok{(slanis,template100)}
\FunctionTok{writeRaster}\NormalTok{(slanis2,}
      \StringTok{"./RasterGrids\_100m/2024/RAW/FarmlandTrees\_PermanentCrops\_r10000.tif"}\NormalTok{,}
      \AttributeTok{overwrite=}\ConstantTok{TRUE}\NormalTok{)}

\CommentTok{\# standardisation {-}{-}{-}{-}}
\ControlFlowTok{if}\NormalTok{(}\SpecialCharTok{!}\FunctionTok{require}\NormalTok{(terra)) \{}\FunctionTok{install.packages}\NormalTok{(}\StringTok{"terra"}\NormalTok{); }\FunctionTok{require}\NormalTok{(terra)\}}
\ControlFlowTok{if}\NormalTok{(}\SpecialCharTok{!}\FunctionTok{require}\NormalTok{(tidyverse)) \{}\FunctionTok{install.packages}\NormalTok{(}\StringTok{"tidyverse"}\NormalTok{); }\FunctionTok{require}\NormalTok{(tidyverse)\}}

\NormalTok{nosaukums}\OtherTok{=}\StringTok{"FarmlandTrees\_PermanentCrops\_r10000.tif"}
\NormalTok{ielasisanas\_cels}\OtherTok{=}\FunctionTok{paste0}\NormalTok{(}\StringTok{"./RasterGrids\_100m/2024/RAW/"}\NormalTok{,nosaukums)}
\NormalTok{saglabasanas\_cels}\OtherTok{=}\FunctionTok{paste0}\NormalTok{(}\StringTok{"./RasterGrids\_100m/2024/Scaled/"}\NormalTok{,nosaukums)}
\NormalTok{slanis}\OtherTok{=}\FunctionTok{rast}\NormalTok{(ielasisanas\_cels)}
\NormalTok{videjais}\OtherTok{=}\FunctionTok{global}\NormalTok{(slanis,}\AttributeTok{fun=}\StringTok{"mean"}\NormalTok{,}\AttributeTok{na.rm=}\ConstantTok{TRUE}\NormalTok{)}
\NormalTok{centrets}\OtherTok{=}\NormalTok{slanis}\SpecialCharTok{{-}}\NormalTok{videjais[,}\DecValTok{1}\NormalTok{]}
\NormalTok{standartnovirze}\OtherTok{=}\NormalTok{terra}\SpecialCharTok{::}\FunctionTok{global}\NormalTok{(centrets,}\AttributeTok{fun=}\StringTok{"rms"}\NormalTok{,}\AttributeTok{na.rm=}\ConstantTok{TRUE}\NormalTok{)}
\NormalTok{merogots}\OtherTok{=}\NormalTok{centrets}\SpecialCharTok{/}\NormalTok{standartnovirze[,}\DecValTok{1}\NormalTok{]}
\FunctionTok{writeRaster}\NormalTok{(merogots,}
      \AttributeTok{filename=}\NormalTok{saglabasanas\_cels,}
      \AttributeTok{overwrite=}\ConstantTok{TRUE}\NormalTok{)}
\end{Highlighting}
\end{Shaded}

\section{FarmlandTrees\_ShortRotationCoppice\_cell}\label{ch06.265}

\textbf{filename:} \texttt{FarmlandTrees\_ShortRotationCoppice\_cell.tif}

\textbf{layername:} \texttt{egv\_265}

\textbf{English name:} Fractional cover of Short-rotation Coppice and Other Woody
Energy Crops within the analysis cell (1 ha)

\textbf{Latvian name:} Īscirtmeta atvasāju un enerģijai audzētu kokaugu platības
īpatsvars analīzes šūnā (1 ha)

\textbf{Procedure:} First, agricultural parcels declared as short rotation coppice
are selected from the \hyperref[Ch04.02]{Rural Support Service's information on declared
fields}. Geometries are then rasterised to input resolution, ensuring
value 1 at the polygon locations and value 0 elsewhere. Rasterisation is
performed using the workflow \texttt{egvtools::polygon2input()}. Once rasterised, the
layer is aggregated to EGV resolution using the workflow \texttt{egvtools::input2egv()},
which calculates the arithmetic mean and thus
results in a cover fraction. During aggregation, inverse
distance weighted (power = 2) gap filling on the output is applied to
ensure no missing values at the edges. Finally, the layer is standardised
by subtracting the arithmetic mean and dividing by the root mean squared error.

\begin{Shaded}
\begin{Highlighting}[]
\CommentTok{\# libs {-}{-}{-}{-}}
\ControlFlowTok{if}\NormalTok{(}\SpecialCharTok{!}\FunctionTok{require}\NormalTok{(egvtools)) \{remotes}\SpecialCharTok{::}\FunctionTok{install\_github}\NormalTok{(}\StringTok{"aavotins/egvtools"}\NormalTok{); }\FunctionTok{require}\NormalTok{(egvtools)\}}
\ControlFlowTok{if}\NormalTok{(}\SpecialCharTok{!}\FunctionTok{require}\NormalTok{(terra)) \{}\FunctionTok{install.packages}\NormalTok{(}\StringTok{"terra"}\NormalTok{); }\FunctionTok{require}\NormalTok{(terra)\}}
\ControlFlowTok{if}\NormalTok{(}\SpecialCharTok{!}\FunctionTok{require}\NormalTok{(sf)) \{}\FunctionTok{install.packages}\NormalTok{(}\StringTok{"sf"}\NormalTok{); }\FunctionTok{require}\NormalTok{(sf)\}}
\ControlFlowTok{if}\NormalTok{(}\SpecialCharTok{!}\FunctionTok{require}\NormalTok{(tidyverse)) \{}\FunctionTok{install.packages}\NormalTok{(}\StringTok{"tidyverse"}\NormalTok{); }\FunctionTok{require}\NormalTok{(tidyverse)\}}
\ControlFlowTok{if}\NormalTok{(}\SpecialCharTok{!}\FunctionTok{require}\NormalTok{(sfarrow)) \{}\FunctionTok{install.packages}\NormalTok{(}\StringTok{"sfarrow"}\NormalTok{); }\FunctionTok{require}\NormalTok{(sfarrow)\}}
\ControlFlowTok{if}\NormalTok{(}\SpecialCharTok{!}\FunctionTok{require}\NormalTok{(readxl)) \{}\FunctionTok{install.packages}\NormalTok{(}\StringTok{"readxl"}\NormalTok{); }\FunctionTok{require}\NormalTok{(readxl)\}}
\ControlFlowTok{if}\NormalTok{(}\SpecialCharTok{!}\FunctionTok{require}\NormalTok{(raster)) \{}\FunctionTok{install.packages}\NormalTok{(}\StringTok{"raster"}\NormalTok{); }\FunctionTok{require}\NormalTok{(raster)\}}
\ControlFlowTok{if}\NormalTok{(}\SpecialCharTok{!}\FunctionTok{require}\NormalTok{(fasterize)) \{}\FunctionTok{install.packages}\NormalTok{(}\StringTok{"fasterize"}\NormalTok{); }\FunctionTok{require}\NormalTok{(fasterize)\}}

\CommentTok{\# templates {-}{-}{-}{-}}
\NormalTok{template100}\OtherTok{=}\FunctionTok{rast}\NormalTok{(}\StringTok{"./Templates/TemplateRasters/LV100m\_10km.tif"}\NormalTok{)}
\NormalTok{template10}\OtherTok{=}\FunctionTok{rast}\NormalTok{(}\StringTok{"./Templates/TemplateRasters/LV10m\_10km.tif"}\NormalTok{)}
\NormalTok{rastrs10}\OtherTok{=}\FunctionTok{raster}\NormalTok{(template10)}

\NormalTok{nulls10}\OtherTok{=}\FunctionTok{rast}\NormalTok{(}\StringTok{"./Templates/TemplateRasters/nulls\_LV10m\_10km.tif"}\NormalTok{)}
\NormalTok{nulls100}\OtherTok{=}\FunctionTok{rast}\NormalTok{(}\StringTok{"./Templates/TemplateRasters/nulls\_LV100m\_10km.tif"}\NormalTok{)}

\CommentTok{\# codes {-}{-}{-}{-}}
\NormalTok{kodi}\OtherTok{=}\FunctionTok{read\_excel}\NormalTok{(}\StringTok{"./Geodata/2024/LAD/KulturuKodi\_2024.xlsx"}\NormalTok{)}
\NormalTok{kodi}\SpecialCharTok{$}\NormalTok{kods}\OtherTok{=}\FunctionTok{as.character}\NormalTok{(kodi}\SpecialCharTok{$}\NormalTok{kods)}
\CommentTok{\# LAD {-}{-}{-}{-}}
\NormalTok{lad}\OtherTok{=}\NormalTok{sfarrow}\SpecialCharTok{::}\FunctionTok{st\_read\_parquet}\NormalTok{(}\StringTok{"./Geodata/2024/LAD/Lauki\_2024.parquet"}\NormalTok{)}
\NormalTok{lad}\SpecialCharTok{$}\NormalTok{yes}\OtherTok{=}\DecValTok{1}
\NormalTok{lad}\OtherTok{=}\NormalTok{lad }\SpecialCharTok{\%\textgreater{}\%} 
 \FunctionTok{left\_join}\NormalTok{(kodi,}\AttributeTok{by=}\FunctionTok{c}\NormalTok{(}\StringTok{"PRODUCT\_CODE"}\OtherTok{=}\StringTok{"kods"}\NormalTok{))}

\CommentTok{\# simple landscape {-}{-}{-}{-}}
\NormalTok{simple\_landscape}\OtherTok{=}\FunctionTok{rast}\NormalTok{(}\StringTok{"RasterGrids\_10m/2024/Ainava\_vienk\_mask.tif"}\NormalTok{)}


\CommentTok{\# FarmlandTrees\_ShortRotationCoppice\_cell.tif   egv\_265 {-}{-}{-}{-}}
\NormalTok{dati}\OtherTok{=}\NormalTok{lad }\SpecialCharTok{\%\textgreater{}\%} 
 \FunctionTok{filter}\NormalTok{(SDM\_grupa\_sakums }\SpecialCharTok{==} \StringTok{"krūmveida ilggadīgie stādījumi"}\NormalTok{)}
\FunctionTok{table}\NormalTok{(dati}\SpecialCharTok{$}\NormalTok{SDM\_grupa\_sakums,}\AttributeTok{useNA=}\StringTok{"always"}\NormalTok{)}

\NormalTok{p2i\_rez}\OtherTok{=}\NormalTok{egvtools}\SpecialCharTok{::}\FunctionTok{polygon2input}\NormalTok{(}\AttributeTok{vector\_data =}\NormalTok{ dati,}
                \AttributeTok{template\_path =} \StringTok{"./Templates/TemplateRasters/LV10m\_10km.tif"}\NormalTok{,}
                \AttributeTok{out\_path =} \StringTok{"./RasterGrids\_10m/2024/"}\NormalTok{,}
                \AttributeTok{file\_name =} \StringTok{"FarmlandTrees\_ShortRotationCoppice\_input.tif"}\NormalTok{,}
                \AttributeTok{value\_field =} \StringTok{"yes"}\NormalTok{,}
                \AttributeTok{prepare=}\ConstantTok{FALSE}\NormalTok{,}
                \AttributeTok{background\_raster =} \StringTok{"./Templates/TemplateRasters/nulls\_LV10m\_10km.tif"}\NormalTok{,}
                \AttributeTok{plot\_result =} \ConstantTok{TRUE}\NormalTok{)}
\NormalTok{p2i\_rez}
\NormalTok{i2e\_rez}\OtherTok{=}\NormalTok{egvtools}\SpecialCharTok{::}\FunctionTok{input2egv}\NormalTok{(}\AttributeTok{input=}\FunctionTok{paste0}\NormalTok{(}\StringTok{"./RasterGrids\_10m/2024/"}\NormalTok{,}
                     \StringTok{"FarmlandTrees\_ShortRotationCoppice\_input.tif"}\NormalTok{),}
              \AttributeTok{egv\_template=} \StringTok{"./Templates/TemplateRasters/LV100m\_10km.tif"}\NormalTok{,}
              \AttributeTok{summary\_function =} \StringTok{"average"}\NormalTok{,}
              \AttributeTok{missing\_job =} \StringTok{"FillOutput"}\NormalTok{,}
              \AttributeTok{outlocation =} \StringTok{"./RasterGrids\_100m/2024/RAW/"}\NormalTok{,}
              \AttributeTok{outfilename =} \StringTok{"FarmlandTrees\_ShortRotationCoppice\_cell.tif"}\NormalTok{,}
              \AttributeTok{layername =} \StringTok{"egv\_265"}\NormalTok{,}
              \AttributeTok{idw\_weight =} \DecValTok{2}\NormalTok{,}
              \AttributeTok{plot\_gaps =} \ConstantTok{FALSE}\NormalTok{,}\AttributeTok{plot\_final =} \ConstantTok{TRUE}\NormalTok{)}
\NormalTok{i2e\_rez}
\FunctionTok{rm}\NormalTok{(p2i\_rez)}
\FunctionTok{rm}\NormalTok{(i2e\_rez)}
\FunctionTok{rm}\NormalTok{(dati)}
\FunctionTok{unlink}\NormalTok{(}\StringTok{"./RasterGrids\_10m/2024/FarmlandTrees\_ShortRotationCoppice\_input.tif"}\NormalTok{)}

\CommentTok{\# standardisation {-}{-}{-}{-}}
\ControlFlowTok{if}\NormalTok{(}\SpecialCharTok{!}\FunctionTok{require}\NormalTok{(terra)) \{}\FunctionTok{install.packages}\NormalTok{(}\StringTok{"terra"}\NormalTok{); }\FunctionTok{require}\NormalTok{(terra)\}}
\ControlFlowTok{if}\NormalTok{(}\SpecialCharTok{!}\FunctionTok{require}\NormalTok{(tidyverse)) \{}\FunctionTok{install.packages}\NormalTok{(}\StringTok{"tidyverse"}\NormalTok{); }\FunctionTok{require}\NormalTok{(tidyverse)\}}

\NormalTok{nosaukums}\OtherTok{=}\StringTok{"FarmlandTrees\_ShortRotationCoppice\_cell.tif"}
\NormalTok{ielasisanas\_cels}\OtherTok{=}\FunctionTok{paste0}\NormalTok{(}\StringTok{"./RasterGrids\_100m/2024/RAW/"}\NormalTok{,nosaukums)}
\NormalTok{saglabasanas\_cels}\OtherTok{=}\FunctionTok{paste0}\NormalTok{(}\StringTok{"./RasterGrids\_100m/2024/Scaled/"}\NormalTok{,nosaukums)}
\NormalTok{slanis}\OtherTok{=}\FunctionTok{rast}\NormalTok{(ielasisanas\_cels)}
\NormalTok{videjais}\OtherTok{=}\FunctionTok{global}\NormalTok{(slanis,}\AttributeTok{fun=}\StringTok{"mean"}\NormalTok{,}\AttributeTok{na.rm=}\ConstantTok{TRUE}\NormalTok{)}
\NormalTok{centrets}\OtherTok{=}\NormalTok{slanis}\SpecialCharTok{{-}}\NormalTok{videjais[,}\DecValTok{1}\NormalTok{]}
\NormalTok{standartnovirze}\OtherTok{=}\NormalTok{terra}\SpecialCharTok{::}\FunctionTok{global}\NormalTok{(centrets,}\AttributeTok{fun=}\StringTok{"rms"}\NormalTok{,}\AttributeTok{na.rm=}\ConstantTok{TRUE}\NormalTok{)}
\NormalTok{merogots}\OtherTok{=}\NormalTok{centrets}\SpecialCharTok{/}\NormalTok{standartnovirze[,}\DecValTok{1}\NormalTok{]}
\FunctionTok{writeRaster}\NormalTok{(merogots,}
      \AttributeTok{filename=}\NormalTok{saglabasanas\_cels,}
      \AttributeTok{overwrite=}\ConstantTok{TRUE}\NormalTok{)}
\end{Highlighting}
\end{Shaded}

\section{FarmlandTrees\_ShortRotationCoppice\_r500}\label{ch06.266}

\textbf{filename:} \texttt{FarmlandTrees\_ShortRotationCoppice\_r500.tif}

\textbf{layername:} \texttt{egv\_266}

\textbf{English name:} Fractional cover of Short-rotation Coppice and Other Woody
Energy Crops within the 0.5 km landscape

\textbf{Latvian name:} Īscirtmeta atvasāju un enerģijai audzētu kokaugu platības
īpatsvars 0,5 km ainavā

\textbf{Procedure:} The cover fraction within a radius of 500 m around the analysis grid cell is
calculated as the area-weighted sum of the \hyperref[ch06.265]{analysis cells} inside the
buffer, using the workflow \texttt{egvtools::radius\_function()}. During the calculation of the landscape metric,
inverse distance weighted (power = 2) gap filling on the output is applied
to ensure no missing values at the edges. Then the layer is rewritten to set
its name. Finally, the layer is standardised by subtracting the arithmetic
mean and dividing by the root mean squared error.

\begin{Shaded}
\begin{Highlighting}[]
\CommentTok{\# libs {-}{-}{-}{-}}
\ControlFlowTok{if}\NormalTok{(}\SpecialCharTok{!}\FunctionTok{require}\NormalTok{(terra)) \{}\FunctionTok{install.packages}\NormalTok{(}\StringTok{"terra"}\NormalTok{); }\FunctionTok{require}\NormalTok{(terra)\}}
\ControlFlowTok{if}\NormalTok{(}\SpecialCharTok{!}\FunctionTok{require}\NormalTok{(egvtools)) \{remotes}\SpecialCharTok{::}\FunctionTok{install\_github}\NormalTok{(}\StringTok{"aavotins/egvtools"}\NormalTok{); }\FunctionTok{require}\NormalTok{(egvtools)\}}


\CommentTok{\# Templates {-}{-}{-}{-}{-}}
\NormalTok{template100}\OtherTok{=}\FunctionTok{rast}\NormalTok{(}\StringTok{"./Templates/TemplateRasters/LV100m\_10km.tif"}\NormalTok{)}

\CommentTok{\# radii {-}{-}{-}{-}}
\FunctionTok{radius\_function}\NormalTok{(}
 \AttributeTok{kvadrati\_path =} \StringTok{"./Templates/TemplateGrids/tiles/"}\NormalTok{,}
 \AttributeTok{radii\_path   =} \StringTok{"./Templates/TemplateGridPoints/tiles/"}\NormalTok{,}
 \AttributeTok{tikls100\_path =} \StringTok{"./Templates/TemplateGrids/tikls100\_sauzeme.parquet"}\NormalTok{,}
 \AttributeTok{template\_path =} \StringTok{"./Templates/TemplateRasters/LV100m\_10km.tif"}\NormalTok{,}
 \AttributeTok{input\_layers  =} \FunctionTok{c}\NormalTok{(}\StringTok{"./RasterGrids\_100m/2024/RAW/FarmlandTrees\_ShortRotationCoppice\_cell.tif"}\NormalTok{),}
 \AttributeTok{layer\_prefixes =} \FunctionTok{c}\NormalTok{(}\StringTok{"FarmlandTrees\_ShortRotationCoppice"}\NormalTok{),}
 \AttributeTok{output\_dir   =} \StringTok{"./RasterGrids\_100m/2024/RAW/"}\NormalTok{,}
 \AttributeTok{n\_workers   =} \DecValTok{6}\NormalTok{,}
 \AttributeTok{radii     =} \FunctionTok{c}\NormalTok{(}\StringTok{"r500"}\NormalTok{),}
 \AttributeTok{radius\_mode  =} \StringTok{"sparse"}\NormalTok{,}
 \AttributeTok{extract\_fun  =} \StringTok{"mean"}\NormalTok{,}
 \AttributeTok{fill\_missing  =} \ConstantTok{TRUE}\NormalTok{,}
 \AttributeTok{IDW\_weight   =} \DecValTok{2}\NormalTok{,}
 \AttributeTok{future\_max\_size =} \DecValTok{40} \SpecialCharTok{*} \DecValTok{1024}\SpecialCharTok{\^{}}\DecValTok{3}\NormalTok{)}


\CommentTok{\# FarmlandTrees\_ShortRotationCoppice\_r500.tif   egv\_266}
\NormalTok{slanis}\OtherTok{=}\FunctionTok{rast}\NormalTok{(}\StringTok{"./RasterGrids\_100m/2024/RAW/FarmlandTrees\_ShortRotationCoppice\_r500.tif"}\NormalTok{)}
\FunctionTok{names}\NormalTok{(slanis)}\OtherTok{=}\StringTok{"egv\_266"}
\NormalTok{slanis2}\OtherTok{=}\FunctionTok{project}\NormalTok{(slanis,template100)}
\FunctionTok{writeRaster}\NormalTok{(slanis2,}
      \StringTok{"./RasterGrids\_100m/2024/RAW/FarmlandTrees\_ShortRotationCoppice\_r500.tif"}\NormalTok{,}
      \AttributeTok{overwrite=}\ConstantTok{TRUE}\NormalTok{)}

\CommentTok{\# standardisation {-}{-}{-}{-}}
\ControlFlowTok{if}\NormalTok{(}\SpecialCharTok{!}\FunctionTok{require}\NormalTok{(terra)) \{}\FunctionTok{install.packages}\NormalTok{(}\StringTok{"terra"}\NormalTok{); }\FunctionTok{require}\NormalTok{(terra)\}}
\ControlFlowTok{if}\NormalTok{(}\SpecialCharTok{!}\FunctionTok{require}\NormalTok{(tidyverse)) \{}\FunctionTok{install.packages}\NormalTok{(}\StringTok{"tidyverse"}\NormalTok{); }\FunctionTok{require}\NormalTok{(tidyverse)\}}

\NormalTok{nosaukums}\OtherTok{=}\StringTok{"FarmlandTrees\_ShortRotationCoppice\_r500.tif"}
\NormalTok{ielasisanas\_cels}\OtherTok{=}\FunctionTok{paste0}\NormalTok{(}\StringTok{"./RasterGrids\_100m/2024/RAW/"}\NormalTok{,nosaukums)}
\NormalTok{saglabasanas\_cels}\OtherTok{=}\FunctionTok{paste0}\NormalTok{(}\StringTok{"./RasterGrids\_100m/2024/Scaled/"}\NormalTok{,nosaukums)}
\NormalTok{slanis}\OtherTok{=}\FunctionTok{rast}\NormalTok{(ielasisanas\_cels)}
\NormalTok{videjais}\OtherTok{=}\FunctionTok{global}\NormalTok{(slanis,}\AttributeTok{fun=}\StringTok{"mean"}\NormalTok{,}\AttributeTok{na.rm=}\ConstantTok{TRUE}\NormalTok{)}
\NormalTok{centrets}\OtherTok{=}\NormalTok{slanis}\SpecialCharTok{{-}}\NormalTok{videjais[,}\DecValTok{1}\NormalTok{]}
\NormalTok{standartnovirze}\OtherTok{=}\NormalTok{terra}\SpecialCharTok{::}\FunctionTok{global}\NormalTok{(centrets,}\AttributeTok{fun=}\StringTok{"rms"}\NormalTok{,}\AttributeTok{na.rm=}\ConstantTok{TRUE}\NormalTok{)}
\NormalTok{merogots}\OtherTok{=}\NormalTok{centrets}\SpecialCharTok{/}\NormalTok{standartnovirze[,}\DecValTok{1}\NormalTok{]}
\FunctionTok{writeRaster}\NormalTok{(merogots,}
      \AttributeTok{filename=}\NormalTok{saglabasanas\_cels,}
      \AttributeTok{overwrite=}\ConstantTok{TRUE}\NormalTok{)}
\end{Highlighting}
\end{Shaded}

\section{FarmlandTrees\_ShortRotationCoppice\_r1250}\label{ch06.267}

\textbf{filename:} \texttt{FarmlandTrees\_ShortRotationCoppice\_r1250.tif}

\textbf{layername:} \texttt{egv\_267}

\textbf{English name:} Fractional cover of Short-rotation Coppice and Other Woody
Energy Crops within the 1.25 km landscape

\textbf{Latvian name:} Īscirtmeta atvasāju un enerģijai audzētu kokaugu platības
īpatsvars 1,25 km ainavā

\textbf{Procedure:} The cover fraction within a radius of 1250 m around the analysis grid cell
is calculated as the area-weighted sum of the \hyperref[ch06.265]{analysis cells} inside
the buffer, using the workflow \texttt{egvtools::radius\_function()}. During the calculation of the landscape
metric, inverse distance weighted (power = 2) gap filling on the output is
applied to ensure no missing values at the edges. Then the layer is
rewritten to set its name. Finally, the layer is standardised by
subtracting the arithmetic mean and dividing by the root mean squared error.

\begin{Shaded}
\begin{Highlighting}[]
\CommentTok{\# libs {-}{-}{-}{-}}
\ControlFlowTok{if}\NormalTok{(}\SpecialCharTok{!}\FunctionTok{require}\NormalTok{(terra)) \{}\FunctionTok{install.packages}\NormalTok{(}\StringTok{"terra"}\NormalTok{); }\FunctionTok{require}\NormalTok{(terra)\}}
\ControlFlowTok{if}\NormalTok{(}\SpecialCharTok{!}\FunctionTok{require}\NormalTok{(egvtools)) \{remotes}\SpecialCharTok{::}\FunctionTok{install\_github}\NormalTok{(}\StringTok{"aavotins/egvtools"}\NormalTok{); }\FunctionTok{require}\NormalTok{(egvtools)\}}


\CommentTok{\# Templates {-}{-}{-}{-}{-}}
\NormalTok{template100}\OtherTok{=}\FunctionTok{rast}\NormalTok{(}\StringTok{"./Templates/TemplateRasters/LV100m\_10km.tif"}\NormalTok{)}

\CommentTok{\# radii {-}{-}{-}{-}}
\FunctionTok{radius\_function}\NormalTok{(}
 \AttributeTok{kvadrati\_path =} \StringTok{"./Templates/TemplateGrids/tiles/"}\NormalTok{,}
 \AttributeTok{radii\_path   =} \StringTok{"./Templates/TemplateGridPoints/tiles/"}\NormalTok{,}
 \AttributeTok{tikls100\_path =} \StringTok{"./Templates/TemplateGrids/tikls100\_sauzeme.parquet"}\NormalTok{,}
 \AttributeTok{template\_path =} \StringTok{"./Templates/TemplateRasters/LV100m\_10km.tif"}\NormalTok{,}
 \AttributeTok{input\_layers  =} \FunctionTok{c}\NormalTok{(}\StringTok{"./RasterGrids\_100m/2024/RAW/FarmlandTrees\_ShortRotationCoppice\_cell.tif"}\NormalTok{),}
 \AttributeTok{layer\_prefixes =} \FunctionTok{c}\NormalTok{(}\StringTok{"FarmlandTrees\_ShortRotationCoppice"}\NormalTok{),}
 \AttributeTok{output\_dir   =} \StringTok{"./RasterGrids\_100m/2024/RAW/"}\NormalTok{,}
 \AttributeTok{n\_workers   =} \DecValTok{6}\NormalTok{,}
 \AttributeTok{radii     =} \FunctionTok{c}\NormalTok{(}\StringTok{"r1250"}\NormalTok{),}
 \AttributeTok{radius\_mode  =} \StringTok{"sparse"}\NormalTok{,}
 \AttributeTok{extract\_fun  =} \StringTok{"mean"}\NormalTok{,}
 \AttributeTok{fill\_missing  =} \ConstantTok{TRUE}\NormalTok{,}
 \AttributeTok{IDW\_weight   =} \DecValTok{2}\NormalTok{,}
 \AttributeTok{future\_max\_size =} \DecValTok{40} \SpecialCharTok{*} \DecValTok{1024}\SpecialCharTok{\^{}}\DecValTok{3}\NormalTok{)}


\CommentTok{\# FarmlandTrees\_ShortRotationCoppice\_r1250.tif  egv\_267}
\NormalTok{slanis}\OtherTok{=}\FunctionTok{rast}\NormalTok{(}\StringTok{"./RasterGrids\_100m/2024/RAW/FarmlandTrees\_ShortRotationCoppice\_r1250.tif"}\NormalTok{)}
\FunctionTok{names}\NormalTok{(slanis)}\OtherTok{=}\StringTok{"egv\_267"}
\NormalTok{slanis2}\OtherTok{=}\FunctionTok{project}\NormalTok{(slanis,template100)}
\FunctionTok{writeRaster}\NormalTok{(slanis2,}
      \StringTok{"./RasterGrids\_100m/2024/RAW/FarmlandTrees\_ShortRotationCoppice\_r1250.tif"}\NormalTok{,}
      \AttributeTok{overwrite=}\ConstantTok{TRUE}\NormalTok{)}

\CommentTok{\# standardisation {-}{-}{-}{-}}
\ControlFlowTok{if}\NormalTok{(}\SpecialCharTok{!}\FunctionTok{require}\NormalTok{(terra)) \{}\FunctionTok{install.packages}\NormalTok{(}\StringTok{"terra"}\NormalTok{); }\FunctionTok{require}\NormalTok{(terra)\}}
\ControlFlowTok{if}\NormalTok{(}\SpecialCharTok{!}\FunctionTok{require}\NormalTok{(tidyverse)) \{}\FunctionTok{install.packages}\NormalTok{(}\StringTok{"tidyverse"}\NormalTok{); }\FunctionTok{require}\NormalTok{(tidyverse)\}}

\NormalTok{nosaukums}\OtherTok{=}\StringTok{"FarmlandTrees\_ShortRotationCoppice\_r1250.tif"}
\NormalTok{ielasisanas\_cels}\OtherTok{=}\FunctionTok{paste0}\NormalTok{(}\StringTok{"./RasterGrids\_100m/2024/RAW/"}\NormalTok{,nosaukums)}
\NormalTok{saglabasanas\_cels}\OtherTok{=}\FunctionTok{paste0}\NormalTok{(}\StringTok{"./RasterGrids\_100m/2024/Scaled/"}\NormalTok{,nosaukums)}
\NormalTok{slanis}\OtherTok{=}\FunctionTok{rast}\NormalTok{(ielasisanas\_cels)}
\NormalTok{videjais}\OtherTok{=}\FunctionTok{global}\NormalTok{(slanis,}\AttributeTok{fun=}\StringTok{"mean"}\NormalTok{,}\AttributeTok{na.rm=}\ConstantTok{TRUE}\NormalTok{)}
\NormalTok{centrets}\OtherTok{=}\NormalTok{slanis}\SpecialCharTok{{-}}\NormalTok{videjais[,}\DecValTok{1}\NormalTok{]}
\NormalTok{standartnovirze}\OtherTok{=}\NormalTok{terra}\SpecialCharTok{::}\FunctionTok{global}\NormalTok{(centrets,}\AttributeTok{fun=}\StringTok{"rms"}\NormalTok{,}\AttributeTok{na.rm=}\ConstantTok{TRUE}\NormalTok{)}
\NormalTok{merogots}\OtherTok{=}\NormalTok{centrets}\SpecialCharTok{/}\NormalTok{standartnovirze[,}\DecValTok{1}\NormalTok{]}
\FunctionTok{writeRaster}\NormalTok{(merogots,}
      \AttributeTok{filename=}\NormalTok{saglabasanas\_cels,}
      \AttributeTok{overwrite=}\ConstantTok{TRUE}\NormalTok{)}
\end{Highlighting}
\end{Shaded}

\section{FarmlandTrees\_ShortRotationCoppice\_r3000}\label{ch06.268}

\textbf{filename:} \texttt{FarmlandTrees\_ShortRotationCoppice\_r3000.tif}

\textbf{layername:} \texttt{egv\_268}

\textbf{English name:} Fractional cover of Short-rotation Coppice and Other Woody
Energy Crops within the 3 km landscape

\textbf{Latvian name:} Īscirtmeta atvasāju un enerģijai audzētu kokaugu platības
īpatsvars 3 km ainavā

\textbf{Procedure:} The cover fraction within a radius of 3000 m around the analysis grid cell
is calculated as the area-weighted sum of the \hyperref[ch06.265]{analysis cells} inside
the buffer, using the workflow \texttt{egvtools::radius\_function()}. During the calculation of the landscape
metric, inverse distance weighted (power = 2) gap filling on the output is
applied to ensure no missing values at the edges. Then the layer is
rewritten to set its name. Finally, the layer is standardised by
subtracting the arithmetic mean and dividing by the root mean squared error.

\begin{Shaded}
\begin{Highlighting}[]
\CommentTok{\# libs {-}{-}{-}{-}}
\ControlFlowTok{if}\NormalTok{(}\SpecialCharTok{!}\FunctionTok{require}\NormalTok{(terra)) \{}\FunctionTok{install.packages}\NormalTok{(}\StringTok{"terra"}\NormalTok{); }\FunctionTok{require}\NormalTok{(terra)\}}
\ControlFlowTok{if}\NormalTok{(}\SpecialCharTok{!}\FunctionTok{require}\NormalTok{(egvtools)) \{remotes}\SpecialCharTok{::}\FunctionTok{install\_github}\NormalTok{(}\StringTok{"aavotins/egvtools"}\NormalTok{); }\FunctionTok{require}\NormalTok{(egvtools)\}}


\CommentTok{\# Templates {-}{-}{-}{-}{-}}
\NormalTok{template100}\OtherTok{=}\FunctionTok{rast}\NormalTok{(}\StringTok{"./Templates/TemplateRasters/LV100m\_10km.tif"}\NormalTok{)}

\CommentTok{\# radii {-}{-}{-}{-}}
\FunctionTok{radius\_function}\NormalTok{(}
 \AttributeTok{kvadrati\_path =} \StringTok{"./Templates/TemplateGrids/tiles/"}\NormalTok{,}
 \AttributeTok{radii\_path   =} \StringTok{"./Templates/TemplateGridPoints/tiles/"}\NormalTok{,}
 \AttributeTok{tikls100\_path =} \StringTok{"./Templates/TemplateGrids/tikls100\_sauzeme.parquet"}\NormalTok{,}
 \AttributeTok{template\_path =} \StringTok{"./Templates/TemplateRasters/LV100m\_10km.tif"}\NormalTok{,}
 \AttributeTok{input\_layers  =} \FunctionTok{c}\NormalTok{(}\StringTok{"./RasterGrids\_100m/2024/RAW/FarmlandTrees\_ShortRotationCoppice\_cell.tif"}\NormalTok{),}
 \AttributeTok{layer\_prefixes =} \FunctionTok{c}\NormalTok{(}\StringTok{"FarmlandTrees\_ShortRotationCoppice"}\NormalTok{),}
 \AttributeTok{output\_dir   =} \StringTok{"./RasterGrids\_100m/2024/RAW/"}\NormalTok{,}
 \AttributeTok{n\_workers   =} \DecValTok{6}\NormalTok{,}
 \AttributeTok{radii     =} \FunctionTok{c}\NormalTok{(}\StringTok{"r3000"}\NormalTok{),}
 \AttributeTok{radius\_mode  =} \StringTok{"sparse"}\NormalTok{,}
 \AttributeTok{extract\_fun  =} \StringTok{"mean"}\NormalTok{,}
 \AttributeTok{fill\_missing  =} \ConstantTok{TRUE}\NormalTok{,}
 \AttributeTok{IDW\_weight   =} \DecValTok{2}\NormalTok{,}
 \AttributeTok{future\_max\_size =} \DecValTok{40} \SpecialCharTok{*} \DecValTok{1024}\SpecialCharTok{\^{}}\DecValTok{3}\NormalTok{)}


\CommentTok{\# FarmlandTrees\_ShortRotationCoppice\_r3000.tif  egv\_268}
\NormalTok{slanis}\OtherTok{=}\FunctionTok{rast}\NormalTok{(}\StringTok{"./RasterGrids\_100m/2024/RAW/FarmlandTrees\_ShortRotationCoppice\_r3000.tif"}\NormalTok{)}
\FunctionTok{names}\NormalTok{(slanis)}\OtherTok{=}\StringTok{"egv\_268"}
\NormalTok{slanis2}\OtherTok{=}\FunctionTok{project}\NormalTok{(slanis,template100)}
\FunctionTok{writeRaster}\NormalTok{(slanis2,}
      \StringTok{"./RasterGrids\_100m/2024/RAW/FarmlandTrees\_ShortRotationCoppice\_r3000.tif"}\NormalTok{,}
      \AttributeTok{overwrite=}\ConstantTok{TRUE}\NormalTok{)}

\CommentTok{\# standardisation {-}{-}{-}{-}}
\ControlFlowTok{if}\NormalTok{(}\SpecialCharTok{!}\FunctionTok{require}\NormalTok{(terra)) \{}\FunctionTok{install.packages}\NormalTok{(}\StringTok{"terra"}\NormalTok{); }\FunctionTok{require}\NormalTok{(terra)\}}
\ControlFlowTok{if}\NormalTok{(}\SpecialCharTok{!}\FunctionTok{require}\NormalTok{(tidyverse)) \{}\FunctionTok{install.packages}\NormalTok{(}\StringTok{"tidyverse"}\NormalTok{); }\FunctionTok{require}\NormalTok{(tidyverse)\}}
\NormalTok{nosaukums}\OtherTok{=}\StringTok{"FarmlandTrees\_ShortRotationCoppice\_r3000.tif"}
\NormalTok{ielasisanas\_cels}\OtherTok{=}\FunctionTok{paste0}\NormalTok{(}\StringTok{"./RasterGrids\_100m/2024/RAW/"}\NormalTok{,nosaukums)}
\NormalTok{saglabasanas\_cels}\OtherTok{=}\FunctionTok{paste0}\NormalTok{(}\StringTok{"./RasterGrids\_100m/2024/Scaled/"}\NormalTok{,nosaukums)}
\NormalTok{slanis}\OtherTok{=}\FunctionTok{rast}\NormalTok{(ielasisanas\_cels)}
\NormalTok{videjais}\OtherTok{=}\FunctionTok{global}\NormalTok{(slanis,}\AttributeTok{fun=}\StringTok{"mean"}\NormalTok{,}\AttributeTok{na.rm=}\ConstantTok{TRUE}\NormalTok{)}
\NormalTok{centrets}\OtherTok{=}\NormalTok{slanis}\SpecialCharTok{{-}}\NormalTok{videjais[,}\DecValTok{1}\NormalTok{]}
\NormalTok{standartnovirze}\OtherTok{=}\NormalTok{terra}\SpecialCharTok{::}\FunctionTok{global}\NormalTok{(centrets,}\AttributeTok{fun=}\StringTok{"rms"}\NormalTok{,}\AttributeTok{na.rm=}\ConstantTok{TRUE}\NormalTok{)}
\NormalTok{merogots}\OtherTok{=}\NormalTok{centrets}\SpecialCharTok{/}\NormalTok{standartnovirze[,}\DecValTok{1}\NormalTok{]}
\FunctionTok{writeRaster}\NormalTok{(merogots,}
      \AttributeTok{filename=}\NormalTok{saglabasanas\_cels,}
      \AttributeTok{overwrite=}\ConstantTok{TRUE}\NormalTok{)}
\end{Highlighting}
\end{Shaded}

\section{FarmlandTrees\_ShortRotationCoppice\_r10000}\label{ch06.269}

\textbf{filename:} \texttt{FarmlandTrees\_ShortRotationCoppice\_r10000.tif}

\textbf{layername:} \texttt{egv\_269}

\textbf{English name:} Fractional cover of Short-rotation Coppice and Other Woody
Energy Crops within the 10 km landscape

\textbf{Latvian name:} Īscirtmeta atvasāju un enerģijai audzētu kokaugu platības
īpatsvars 10 km ainavā

\textbf{Procedure:} The cover fraction within a radius of 10000 m around the analysis grid cell
is calculated as the area-weighted sum of the \hyperref[ch06.265]{analysis cells} inside
the buffer, using the workflow \texttt{egvtools::radius\_function()}. During the calculation of the landscape
metric, inverse distance weighted (power = 2) gap filling on the output is
applied to ensure no missing values at the edges. Then the layer is
rewritten to set its name. Finally, the layer is standardised by
subtracting the arithmetic mean and dividing by the root mean squared error.

\begin{Shaded}
\begin{Highlighting}[]
\CommentTok{\# libs {-}{-}{-}{-}}
\ControlFlowTok{if}\NormalTok{(}\SpecialCharTok{!}\FunctionTok{require}\NormalTok{(terra)) \{}\FunctionTok{install.packages}\NormalTok{(}\StringTok{"terra"}\NormalTok{); }\FunctionTok{require}\NormalTok{(terra)\}}
\ControlFlowTok{if}\NormalTok{(}\SpecialCharTok{!}\FunctionTok{require}\NormalTok{(egvtools)) \{remotes}\SpecialCharTok{::}\FunctionTok{install\_github}\NormalTok{(}\StringTok{"aavotins/egvtools"}\NormalTok{); }\FunctionTok{require}\NormalTok{(egvtools)\}}


\CommentTok{\# Templates {-}{-}{-}{-}{-}}
\NormalTok{template100}\OtherTok{=}\FunctionTok{rast}\NormalTok{(}\StringTok{"./Templates/TemplateRasters/LV100m\_10km.tif"}\NormalTok{)}

\CommentTok{\# radii {-}{-}{-}{-}}
\FunctionTok{radius\_function}\NormalTok{(}
 \AttributeTok{kvadrati\_path =} \StringTok{"./Templates/TemplateGrids/tiles/"}\NormalTok{,}
 \AttributeTok{radii\_path   =} \StringTok{"./Templates/TemplateGridPoints/tiles/"}\NormalTok{,}
 \AttributeTok{tikls100\_path =} \StringTok{"./Templates/TemplateGrids/tikls100\_sauzeme.parquet"}\NormalTok{,}
 \AttributeTok{template\_path =} \StringTok{"./Templates/TemplateRasters/LV100m\_10km.tif"}\NormalTok{,}
 \AttributeTok{input\_layers  =} \FunctionTok{c}\NormalTok{(}\StringTok{"./RasterGrids\_100m/2024/RAW/FarmlandTrees\_ShortRotationCoppice\_cell.tif"}\NormalTok{),}
 \AttributeTok{layer\_prefixes =} \FunctionTok{c}\NormalTok{(}\StringTok{"FarmlandTrees\_ShortRotationCoppice"}\NormalTok{),}
 \AttributeTok{output\_dir   =} \StringTok{"./RasterGrids\_100m/2024/RAW/"}\NormalTok{,}
 \AttributeTok{n\_workers   =} \DecValTok{6}\NormalTok{,}
 \AttributeTok{radii     =} \FunctionTok{c}\NormalTok{(}\StringTok{"r10000"}\NormalTok{),}
 \AttributeTok{radius\_mode  =} \StringTok{"sparse"}\NormalTok{,}
 \AttributeTok{extract\_fun  =} \StringTok{"mean"}\NormalTok{,}
 \AttributeTok{fill\_missing  =} \ConstantTok{TRUE}\NormalTok{,}
 \AttributeTok{IDW\_weight   =} \DecValTok{2}\NormalTok{,}
 \AttributeTok{future\_max\_size =} \DecValTok{40} \SpecialCharTok{*} \DecValTok{1024}\SpecialCharTok{\^{}}\DecValTok{3}\NormalTok{)}


\CommentTok{\# FarmlandTrees\_ShortRotationCoppice\_r10000.tif egv\_269}
\NormalTok{slanis}\OtherTok{=}\FunctionTok{rast}\NormalTok{(}\StringTok{"./RasterGrids\_100m/2024/RAW/FarmlandTrees\_ShortRotationCoppice\_r10000.tif"}\NormalTok{)}
\FunctionTok{names}\NormalTok{(slanis)}\OtherTok{=}\StringTok{"egv\_269"}
\NormalTok{slanis2}\OtherTok{=}\FunctionTok{project}\NormalTok{(slanis,template100)}
\FunctionTok{writeRaster}\NormalTok{(slanis2,}
      \StringTok{"./RasterGrids\_100m/2024/RAW/FarmlandTrees\_ShortRotationCoppice\_r10000.tif"}\NormalTok{,}
      \AttributeTok{overwrite=}\ConstantTok{TRUE}\NormalTok{)}

\CommentTok{\# standardisation {-}{-}{-}{-}}
\ControlFlowTok{if}\NormalTok{(}\SpecialCharTok{!}\FunctionTok{require}\NormalTok{(terra)) \{}\FunctionTok{install.packages}\NormalTok{(}\StringTok{"terra"}\NormalTok{); }\FunctionTok{require}\NormalTok{(terra)\}}
\ControlFlowTok{if}\NormalTok{(}\SpecialCharTok{!}\FunctionTok{require}\NormalTok{(tidyverse)) \{}\FunctionTok{install.packages}\NormalTok{(}\StringTok{"tidyverse"}\NormalTok{); }\FunctionTok{require}\NormalTok{(tidyverse)\}}

\NormalTok{nosaukums}\OtherTok{=}\StringTok{"FarmlandTrees\_ShortRotationCoppice\_r10000.tif"}
\NormalTok{ielasisanas\_cels}\OtherTok{=}\FunctionTok{paste0}\NormalTok{(}\StringTok{"./RasterGrids\_100m/2024/RAW/"}\NormalTok{,nosaukums)}
\NormalTok{saglabasanas\_cels}\OtherTok{=}\FunctionTok{paste0}\NormalTok{(}\StringTok{"./RasterGrids\_100m/2024/Scaled/"}\NormalTok{,nosaukums)}
\NormalTok{slanis}\OtherTok{=}\FunctionTok{rast}\NormalTok{(ielasisanas\_cels)}
\NormalTok{videjais}\OtherTok{=}\FunctionTok{global}\NormalTok{(slanis,}\AttributeTok{fun=}\StringTok{"mean"}\NormalTok{,}\AttributeTok{na.rm=}\ConstantTok{TRUE}\NormalTok{)}
\NormalTok{centrets}\OtherTok{=}\NormalTok{slanis}\SpecialCharTok{{-}}\NormalTok{videjais[,}\DecValTok{1}\NormalTok{]}
\NormalTok{standartnovirze}\OtherTok{=}\NormalTok{terra}\SpecialCharTok{::}\FunctionTok{global}\NormalTok{(centrets,}\AttributeTok{fun=}\StringTok{"rms"}\NormalTok{,}\AttributeTok{na.rm=}\ConstantTok{TRUE}\NormalTok{)}
\NormalTok{merogots}\OtherTok{=}\NormalTok{centrets}\SpecialCharTok{/}\NormalTok{standartnovirze[,}\DecValTok{1}\NormalTok{]}
\FunctionTok{writeRaster}\NormalTok{(merogots,}
      \AttributeTok{filename=}\NormalTok{saglabasanas\_cels,}
      \AttributeTok{overwrite=}\ConstantTok{TRUE}\NormalTok{)}
\end{Highlighting}
\end{Shaded}

\section{ForestsAge\_ClearcutsLowStands\_cell}\label{ch06.270}

\textbf{filename:} \texttt{ForestsAge\_ClearcutsLowStands\_cell.tif}

\textbf{layername:} \texttt{egv\_270}

\textbf{English name:} Fractional cover of Clearcuts and Forest Stands lower than 5 m within
the analysis cell (1 ha)

\textbf{Latvian name:} Izcirtumu un mežaudžu līdz 5 m augstumam platības īpatsvars
analīzes šūnā (1 ha)

\textbf{Procedure:} To prepare this
EGV, stands in land category 10 with a height of less than 5 m are selected from the \hyperref[Ch04.01]{State
Forest Service's State Forest Registry} and rasterised. After
rasterisation, this layer is covered by a clearcut mask. The resulting layer
is then aggregated to EGV resolution using the workflow \texttt{egvtools::input2egv()}, which
calculates the arithmetic mean to determine the cover fraction. During
aggregation, inverse distance weighted (power = 2) gap filling on the output is
applied to ensure no missing values at the edges. Finally, the layer is
standardised by subtracting the arithmetic mean and dividing by the root mean squared
error.

\begin{Shaded}
\begin{Highlighting}[]
\CommentTok{\# libs {-}{-}{-}{-}}
\ControlFlowTok{if}\NormalTok{(}\SpecialCharTok{!}\FunctionTok{require}\NormalTok{(egvtools)) \{remotes}\SpecialCharTok{::}\FunctionTok{install\_github}\NormalTok{(}\StringTok{"aavotins/egvtools"}\NormalTok{); }\FunctionTok{require}\NormalTok{(egvtools)\}}
\ControlFlowTok{if}\NormalTok{(}\SpecialCharTok{!}\FunctionTok{require}\NormalTok{(terra)) \{}\FunctionTok{install.packages}\NormalTok{(}\StringTok{"terra"}\NormalTok{); }\FunctionTok{require}\NormalTok{(terra)\}}
\ControlFlowTok{if}\NormalTok{(}\SpecialCharTok{!}\FunctionTok{require}\NormalTok{(sf)) \{}\FunctionTok{install.packages}\NormalTok{(}\StringTok{"sf"}\NormalTok{); }\FunctionTok{require}\NormalTok{(sf)\}}
\ControlFlowTok{if}\NormalTok{(}\SpecialCharTok{!}\FunctionTok{require}\NormalTok{(tidyverse)) \{}\FunctionTok{install.packages}\NormalTok{(}\StringTok{"tidyverse"}\NormalTok{); }\FunctionTok{require}\NormalTok{(tidyverse)\}}
\ControlFlowTok{if}\NormalTok{(}\SpecialCharTok{!}\FunctionTok{require}\NormalTok{(sfarrow)) \{}\FunctionTok{install.packages}\NormalTok{(}\StringTok{"sfarrow"}\NormalTok{); }\FunctionTok{require}\NormalTok{(sfarrow)\}}
\ControlFlowTok{if}\NormalTok{(}\SpecialCharTok{!}\FunctionTok{require}\NormalTok{(readxl)) \{}\FunctionTok{install.packages}\NormalTok{(}\StringTok{"readxl"}\NormalTok{); }\FunctionTok{require}\NormalTok{(readxl)\}}
\ControlFlowTok{if}\NormalTok{(}\SpecialCharTok{!}\FunctionTok{require}\NormalTok{(raster)) \{}\FunctionTok{install.packages}\NormalTok{(}\StringTok{"raster"}\NormalTok{); }\FunctionTok{require}\NormalTok{(raster)\}}
\ControlFlowTok{if}\NormalTok{(}\SpecialCharTok{!}\FunctionTok{require}\NormalTok{(fasterize)) \{}\FunctionTok{install.packages}\NormalTok{(}\StringTok{"fasterize"}\NormalTok{); }\FunctionTok{require}\NormalTok{(fasterize)\}}

\CommentTok{\# templates {-}{-}{-}{-}}
\NormalTok{template100}\OtherTok{=}\FunctionTok{rast}\NormalTok{(}\StringTok{"./Templates/TemplateRasters/LV100m\_10km.tif"}\NormalTok{)}
\NormalTok{template10}\OtherTok{=}\FunctionTok{rast}\NormalTok{(}\StringTok{"./Templates/TemplateRasters/LV10m\_10km.tif"}\NormalTok{)}
\NormalTok{rastrs10}\OtherTok{=}\FunctionTok{raster}\NormalTok{(template10)}

\NormalTok{nulls10}\OtherTok{=}\FunctionTok{rast}\NormalTok{(}\StringTok{"./Templates/TemplateRasters/nulls\_LV10m\_10km.tif"}\NormalTok{)}
\NormalTok{nulls100}\OtherTok{=}\FunctionTok{rast}\NormalTok{(}\StringTok{"./Templates/TemplateRasters/nulls\_LV100m\_10km.tif"}\NormalTok{)}


\CommentTok{\# simple landscape {-}{-}{-}{-}}
\NormalTok{simple\_landscape}\OtherTok{=}\FunctionTok{rast}\NormalTok{(}\StringTok{"RasterGrids\_10m/2024/Ainava\_vienk\_mask.tif"}\NormalTok{)}

\CommentTok{\# mvr {-}{-}{-}{-}}
\NormalTok{mvr}\OtherTok{=}\FunctionTok{st\_read\_parquet}\NormalTok{(}\StringTok{"./Geodata/2024/MVR/nogabali\_2024janv.parquet"}\NormalTok{)}
\NormalTok{mvr}\SpecialCharTok{$}\NormalTok{yes}\OtherTok{=}\DecValTok{1}


\CommentTok{\# ForestsAge\_ClearcutsLowStands\_cell.tif    egv\_270 {-}{-}{-}{-}}
\NormalTok{zemas\_audzes}\OtherTok{=}\NormalTok{mvr }\SpecialCharTok{\%\textgreater{}\%} 
 \FunctionTok{filter}\NormalTok{(zkat}\SpecialCharTok{==}\StringTok{"10"}\NormalTok{) }\SpecialCharTok{\%\textgreater{}\%} 
 \FunctionTok{filter}\NormalTok{(h10}\SpecialCharTok{\textless{}}\DecValTok{5}\NormalTok{) }\SpecialCharTok{\%\textgreater{}\%} 
\NormalTok{ dplyr}\SpecialCharTok{::}\FunctionTok{select}\NormalTok{(yes)}
\NormalTok{r\_zemasaudzes}\OtherTok{=}\FunctionTok{fasterize}\NormalTok{(zemas\_audzes,rastrs10,}\AttributeTok{field=}\StringTok{"yes"}\NormalTok{)}
\NormalTok{t\_zemasaudzes}\OtherTok{=}\FunctionTok{rast}\NormalTok{(r\_zemasaudzes)}
\FunctionTok{plot}\NormalTok{(t\_zemasaudzes)}

\NormalTok{cleacuts\_low}\OtherTok{=}\FunctionTok{cover}\NormalTok{(t\_zemasaudzes,clearcut\_mask)}
\FunctionTok{plot}\NormalTok{(cleacuts\_low)}

\NormalTok{i2e\_rez}\OtherTok{=}\NormalTok{egvtools}\SpecialCharTok{::}\FunctionTok{input2egv}\NormalTok{(}\AttributeTok{input=}\NormalTok{cleacuts\_low,}
              \AttributeTok{egv\_template=} \StringTok{"./Templates/TemplateRasters/LV100m\_10km.tif"}\NormalTok{,}
              \AttributeTok{summary\_function =} \StringTok{"average"}\NormalTok{,}
              \AttributeTok{missing\_job =} \StringTok{"FillOutput"}\NormalTok{,}
              \AttributeTok{outlocation =} \StringTok{"./RasterGrids\_100m/2024/RAW/"}\NormalTok{,}
              \AttributeTok{outfilename =} \StringTok{"ForestsAge\_ClearcutsLowStands\_cell.tif"}\NormalTok{,}
              \AttributeTok{layername =} \StringTok{"egv\_270"}\NormalTok{,}
              \AttributeTok{idw\_weight =} \DecValTok{2}\NormalTok{,}
              \AttributeTok{plot\_gaps =} \ConstantTok{FALSE}\NormalTok{,}\AttributeTok{plot\_final =} \ConstantTok{TRUE}\NormalTok{)}
\NormalTok{i2e\_rez}
\FunctionTok{rm}\NormalTok{(i2e\_rez)}
\FunctionTok{rm}\NormalTok{(zemas\_audzes)}
\FunctionTok{rm}\NormalTok{(r\_zemasaudzes)}
\FunctionTok{rm}\NormalTok{(t\_zemasaudzes)}
\FunctionTok{rm}\NormalTok{(cleacuts\_low)}

\CommentTok{\# standardisation {-}{-}{-}{-}}
\ControlFlowTok{if}\NormalTok{(}\SpecialCharTok{!}\FunctionTok{require}\NormalTok{(terra)) \{}\FunctionTok{install.packages}\NormalTok{(}\StringTok{"terra"}\NormalTok{); }\FunctionTok{require}\NormalTok{(terra)\}}
\ControlFlowTok{if}\NormalTok{(}\SpecialCharTok{!}\FunctionTok{require}\NormalTok{(tidyverse)) \{}\FunctionTok{install.packages}\NormalTok{(}\StringTok{"tidyverse"}\NormalTok{); }\FunctionTok{require}\NormalTok{(tidyverse)\}}

\NormalTok{nosaukums}\OtherTok{=}\StringTok{"ForestsAge\_ClearcutsLowStands\_cell.tif"}
\NormalTok{ielasisanas\_cels}\OtherTok{=}\FunctionTok{paste0}\NormalTok{(}\StringTok{"./RasterGrids\_100m/2024/RAW/"}\NormalTok{,nosaukums)}
\NormalTok{saglabasanas\_cels}\OtherTok{=}\FunctionTok{paste0}\NormalTok{(}\StringTok{"./RasterGrids\_100m/2024/Scaled/"}\NormalTok{,nosaukums)}
\NormalTok{slanis}\OtherTok{=}\FunctionTok{rast}\NormalTok{(ielasisanas\_cels)}
\NormalTok{videjais}\OtherTok{=}\FunctionTok{global}\NormalTok{(slanis,}\AttributeTok{fun=}\StringTok{"mean"}\NormalTok{,}\AttributeTok{na.rm=}\ConstantTok{TRUE}\NormalTok{)}
\NormalTok{centrets}\OtherTok{=}\NormalTok{slanis}\SpecialCharTok{{-}}\NormalTok{videjais[,}\DecValTok{1}\NormalTok{]}
\NormalTok{standartnovirze}\OtherTok{=}\NormalTok{terra}\SpecialCharTok{::}\FunctionTok{global}\NormalTok{(centrets,}\AttributeTok{fun=}\StringTok{"rms"}\NormalTok{,}\AttributeTok{na.rm=}\ConstantTok{TRUE}\NormalTok{)}
\NormalTok{merogots}\OtherTok{=}\NormalTok{centrets}\SpecialCharTok{/}\NormalTok{standartnovirze[,}\DecValTok{1}\NormalTok{]}
\FunctionTok{writeRaster}\NormalTok{(merogots,}
      \AttributeTok{filename=}\NormalTok{saglabasanas\_cels,}
      \AttributeTok{overwrite=}\ConstantTok{TRUE}\NormalTok{)}
\end{Highlighting}
\end{Shaded}

\section{ForestsAge\_ClearcutsLowStands\_r500}\label{ch06.271}

\textbf{filename:} \texttt{ForestsAge\_ClearcutsLowStands\_r500.tif}

\textbf{layername:} \texttt{egv\_271}

\textbf{English name:} Fractional cover of Clearcuts and Forest Stands lower than 5 m within
the 0.5 km landscape

\textbf{Latvian name:} Izcirtumu un mežaudžu līdz 5 m augstumam platības īpatsvars
0,5 km ainavā

\textbf{Procedure:} The cover fraction within a radius of 500 m around the analysis grid cell is
calculated as the area-weighted sum of the \hyperref[ch06.270]{analysis cells} inside the
buffer, using the workflow \texttt{egvtools::radius\_function()}. During the calculation of the landscape metric,
inverse distance weighted (power = 2) gap filling on the output is applied
to ensure no missing values at the edges. Then the layer is rewritten to set
its name. Finally, the layer is standardised by subtracting the arithmetic
mean and dividing by the root mean squared error.

\begin{Shaded}
\begin{Highlighting}[]
\CommentTok{\# libs {-}{-}{-}{-}}
\ControlFlowTok{if}\NormalTok{(}\SpecialCharTok{!}\FunctionTok{require}\NormalTok{(terra)) \{}\FunctionTok{install.packages}\NormalTok{(}\StringTok{"terra"}\NormalTok{); }\FunctionTok{require}\NormalTok{(terra)\}}
\ControlFlowTok{if}\NormalTok{(}\SpecialCharTok{!}\FunctionTok{require}\NormalTok{(egvtools)) \{remotes}\SpecialCharTok{::}\FunctionTok{install\_github}\NormalTok{(}\StringTok{"aavotins/egvtools"}\NormalTok{); }\FunctionTok{require}\NormalTok{(egvtools)\}}


\CommentTok{\# Templates {-}{-}{-}{-}{-}}
\NormalTok{template100}\OtherTok{=}\FunctionTok{rast}\NormalTok{(}\StringTok{"./Templates/TemplateRasters/LV100m\_10km.tif"}\NormalTok{)}

\CommentTok{\# radii {-}{-}{-}{-}}
\FunctionTok{radius\_function}\NormalTok{(}
 \AttributeTok{kvadrati\_path =} \StringTok{"./Templates/TemplateGrids/tiles/"}\NormalTok{,}
 \AttributeTok{radii\_path   =} \StringTok{"./Templates/TemplateGridPoints/tiles/"}\NormalTok{,}
 \AttributeTok{tikls100\_path =} \StringTok{"./Templates/TemplateGrids/tikls100\_sauzeme.parquet"}\NormalTok{,}
 \AttributeTok{template\_path =} \StringTok{"./Templates/TemplateRasters/LV100m\_10km.tif"}\NormalTok{,}
 \AttributeTok{input\_layers  =} \FunctionTok{c}\NormalTok{(}\StringTok{"./RasterGrids\_100m/2024/RAW/ForestsAge\_ClearcutsLowStands\_cell.tif"}\NormalTok{),}
 \AttributeTok{layer\_prefixes =} \FunctionTok{c}\NormalTok{(}\StringTok{"ForestsAge\_ClearcutsLowStands"}\NormalTok{),}
 \AttributeTok{output\_dir   =} \StringTok{"./RasterGrids\_100m/2024/RAW/"}\NormalTok{,}
 \AttributeTok{n\_workers   =} \DecValTok{6}\NormalTok{,}
 \AttributeTok{radii     =} \FunctionTok{c}\NormalTok{(}\StringTok{"r500"}\NormalTok{),}
 \AttributeTok{radius\_mode  =} \StringTok{"sparse"}\NormalTok{,}
 \AttributeTok{extract\_fun  =} \StringTok{"mean"}\NormalTok{,}
 \AttributeTok{fill\_missing  =} \ConstantTok{TRUE}\NormalTok{,}
 \AttributeTok{IDW\_weight   =} \DecValTok{2}\NormalTok{,}
 \AttributeTok{future\_max\_size =} \DecValTok{40} \SpecialCharTok{*} \DecValTok{1024}\SpecialCharTok{\^{}}\DecValTok{3}\NormalTok{)}


\CommentTok{\# ForestsAge\_ClearcutsLowStands\_r500.tif    egv\_271}
\NormalTok{slanis}\OtherTok{=}\FunctionTok{rast}\NormalTok{(}\StringTok{"./RasterGrids\_100m/2024/RAW/ForestsAge\_ClearcutsLowStands\_r500.tif"}\NormalTok{)}
\FunctionTok{names}\NormalTok{(slanis)}\OtherTok{=}\StringTok{"egv\_271"}
\NormalTok{slanis2}\OtherTok{=}\FunctionTok{project}\NormalTok{(slanis,template100)}
\FunctionTok{writeRaster}\NormalTok{(slanis2,}
      \StringTok{"./RasterGrids\_100m/2024/RAW/ForestsAge\_ClearcutsLowStands\_r500.tif"}\NormalTok{,}
      \AttributeTok{overwrite=}\ConstantTok{TRUE}\NormalTok{)}

\CommentTok{\# standardisation {-}{-}{-}{-}}
\ControlFlowTok{if}\NormalTok{(}\SpecialCharTok{!}\FunctionTok{require}\NormalTok{(terra)) \{}\FunctionTok{install.packages}\NormalTok{(}\StringTok{"terra"}\NormalTok{); }\FunctionTok{require}\NormalTok{(terra)\}}
\ControlFlowTok{if}\NormalTok{(}\SpecialCharTok{!}\FunctionTok{require}\NormalTok{(tidyverse)) \{}\FunctionTok{install.packages}\NormalTok{(}\StringTok{"tidyverse"}\NormalTok{); }\FunctionTok{require}\NormalTok{(tidyverse)\}}

\NormalTok{nosaukums}\OtherTok{=}\StringTok{"ForestsAge\_ClearcutsLowStands\_r500.tif"}
\NormalTok{ielasisanas\_cels}\OtherTok{=}\FunctionTok{paste0}\NormalTok{(}\StringTok{"./RasterGrids\_100m/2024/RAW/"}\NormalTok{,nosaukums)}
\NormalTok{saglabasanas\_cels}\OtherTok{=}\FunctionTok{paste0}\NormalTok{(}\StringTok{"./RasterGrids\_100m/2024/Scaled/"}\NormalTok{,nosaukums)}
\NormalTok{slanis}\OtherTok{=}\FunctionTok{rast}\NormalTok{(ielasisanas\_cels)}
\NormalTok{videjais}\OtherTok{=}\FunctionTok{global}\NormalTok{(slanis,}\AttributeTok{fun=}\StringTok{"mean"}\NormalTok{,}\AttributeTok{na.rm=}\ConstantTok{TRUE}\NormalTok{)}
\NormalTok{centrets}\OtherTok{=}\NormalTok{slanis}\SpecialCharTok{{-}}\NormalTok{videjais[,}\DecValTok{1}\NormalTok{]}
\NormalTok{standartnovirze}\OtherTok{=}\NormalTok{terra}\SpecialCharTok{::}\FunctionTok{global}\NormalTok{(centrets,}\AttributeTok{fun=}\StringTok{"rms"}\NormalTok{,}\AttributeTok{na.rm=}\ConstantTok{TRUE}\NormalTok{)}
\NormalTok{merogots}\OtherTok{=}\NormalTok{centrets}\SpecialCharTok{/}\NormalTok{standartnovirze[,}\DecValTok{1}\NormalTok{]}
\FunctionTok{writeRaster}\NormalTok{(merogots,}
      \AttributeTok{filename=}\NormalTok{saglabasanas\_cels,}
      \AttributeTok{overwrite=}\ConstantTok{TRUE}\NormalTok{)}
\end{Highlighting}
\end{Shaded}

\section{ForestsAge\_ClearcutsLowStands\_r1250}\label{ch06.272}

\textbf{filename:} \texttt{ForestsAge\_ClearcutsLowStands\_r1250.tif}

\textbf{layername:} \texttt{egv\_272}

\textbf{English name:} Fractional cover of Clearcuts and Forest Stands lower than 5 m within
the 1.25 km landscape

\textbf{Latvian name:} Izcirtumu un mežaudžu līdz 5 m augstumam platības īpatsvars
1,25 km ainavā

\textbf{Procedure:} The cover fraction within a radius of 1250 m around the analysis grid cell
is calculated as the area-weighted sum of the \hyperref[ch06.270]{analysis cells} inside
the buffer, using the workflow \texttt{egvtools::radius\_function()}. During the calculation of the landscape
metric, inverse distance weighted (power = 2) gap filling on the output is
applied to ensure no missing values at the edges. Then the layer is
rewritten to set its name. Finally, the layer is standardised by
subtracting the arithmetic mean and dividing by the root mean squared error.

\begin{Shaded}
\begin{Highlighting}[]
\CommentTok{\# libs {-}{-}{-}{-}}
\ControlFlowTok{if}\NormalTok{(}\SpecialCharTok{!}\FunctionTok{require}\NormalTok{(terra)) \{}\FunctionTok{install.packages}\NormalTok{(}\StringTok{"terra"}\NormalTok{); }\FunctionTok{require}\NormalTok{(terra)\}}
\ControlFlowTok{if}\NormalTok{(}\SpecialCharTok{!}\FunctionTok{require}\NormalTok{(egvtools)) \{remotes}\SpecialCharTok{::}\FunctionTok{install\_github}\NormalTok{(}\StringTok{"aavotins/egvtools"}\NormalTok{); }\FunctionTok{require}\NormalTok{(egvtools)\}}


\CommentTok{\# Templates {-}{-}{-}{-}{-}}
\NormalTok{template100}\OtherTok{=}\FunctionTok{rast}\NormalTok{(}\StringTok{"./Templates/TemplateRasters/LV100m\_10km.tif"}\NormalTok{)}

\CommentTok{\# radii {-}{-}{-}{-}}
\FunctionTok{radius\_function}\NormalTok{(}
 \AttributeTok{kvadrati\_path =} \StringTok{"./Templates/TemplateGrids/tiles/"}\NormalTok{,}
 \AttributeTok{radii\_path   =} \StringTok{"./Templates/TemplateGridPoints/tiles/"}\NormalTok{,}
 \AttributeTok{tikls100\_path =} \StringTok{"./Templates/TemplateGrids/tikls100\_sauzeme.parquet"}\NormalTok{,}
 \AttributeTok{template\_path =} \StringTok{"./Templates/TemplateRasters/LV100m\_10km.tif"}\NormalTok{,}
 \AttributeTok{input\_layers  =} \FunctionTok{c}\NormalTok{(}\StringTok{"./RasterGrids\_100m/2024/RAW/ForestsAge\_ClearcutsLowStands\_cell.tif"}\NormalTok{),}
 \AttributeTok{layer\_prefixes =} \FunctionTok{c}\NormalTok{(}\StringTok{"ForestsAge\_ClearcutsLowStands"}\NormalTok{),}
 \AttributeTok{output\_dir   =} \StringTok{"./RasterGrids\_100m/2024/RAW/"}\NormalTok{,}
 \AttributeTok{n\_workers   =} \DecValTok{6}\NormalTok{,}
 \AttributeTok{radii     =} \FunctionTok{c}\NormalTok{(}\StringTok{"r1250"}\NormalTok{),}
 \AttributeTok{radius\_mode  =} \StringTok{"sparse"}\NormalTok{,}
 \AttributeTok{extract\_fun  =} \StringTok{"mean"}\NormalTok{,}
 \AttributeTok{fill\_missing  =} \ConstantTok{TRUE}\NormalTok{,}
 \AttributeTok{IDW\_weight   =} \DecValTok{2}\NormalTok{,}
 \AttributeTok{future\_max\_size =} \DecValTok{40} \SpecialCharTok{*} \DecValTok{1024}\SpecialCharTok{\^{}}\DecValTok{3}\NormalTok{)}


\CommentTok{\# ForestsAge\_ClearcutsLowStands\_r1250.tif   egv\_272}
\NormalTok{slanis}\OtherTok{=}\FunctionTok{rast}\NormalTok{(}\StringTok{"./RasterGrids\_100m/2024/RAW/ForestsAge\_ClearcutsLowStands\_r1250.tif"}\NormalTok{)}
\FunctionTok{names}\NormalTok{(slanis)}\OtherTok{=}\StringTok{"egv\_272"}
\NormalTok{slanis2}\OtherTok{=}\FunctionTok{project}\NormalTok{(slanis,template100)}
\FunctionTok{writeRaster}\NormalTok{(slanis2,}
      \StringTok{"./RasterGrids\_100m/2024/RAW/ForestsAge\_ClearcutsLowStands\_r1250.tif"}\NormalTok{,}
      \AttributeTok{overwrite=}\ConstantTok{TRUE}\NormalTok{)}

\CommentTok{\# standardisation {-}{-}{-}{-}}
\ControlFlowTok{if}\NormalTok{(}\SpecialCharTok{!}\FunctionTok{require}\NormalTok{(terra)) \{}\FunctionTok{install.packages}\NormalTok{(}\StringTok{"terra"}\NormalTok{); }\FunctionTok{require}\NormalTok{(terra)\}}
\ControlFlowTok{if}\NormalTok{(}\SpecialCharTok{!}\FunctionTok{require}\NormalTok{(tidyverse)) \{}\FunctionTok{install.packages}\NormalTok{(}\StringTok{"tidyverse"}\NormalTok{); }\FunctionTok{require}\NormalTok{(tidyverse)\}}

\NormalTok{nosaukums}\OtherTok{=}\StringTok{"ForestsAge\_ClearcutsLowStands\_r1250.tif"}
\NormalTok{ielasisanas\_cels}\OtherTok{=}\FunctionTok{paste0}\NormalTok{(}\StringTok{"./RasterGrids\_100m/2024/RAW/"}\NormalTok{,nosaukums)}
\NormalTok{saglabasanas\_cels}\OtherTok{=}\FunctionTok{paste0}\NormalTok{(}\StringTok{"./RasterGrids\_100m/2024/Scaled/"}\NormalTok{,nosaukums)}
\NormalTok{slanis}\OtherTok{=}\FunctionTok{rast}\NormalTok{(ielasisanas\_cels)}
\NormalTok{videjais}\OtherTok{=}\FunctionTok{global}\NormalTok{(slanis,}\AttributeTok{fun=}\StringTok{"mean"}\NormalTok{,}\AttributeTok{na.rm=}\ConstantTok{TRUE}\NormalTok{)}
\NormalTok{centrets}\OtherTok{=}\NormalTok{slanis}\SpecialCharTok{{-}}\NormalTok{videjais[,}\DecValTok{1}\NormalTok{]}
\NormalTok{standartnovirze}\OtherTok{=}\NormalTok{terra}\SpecialCharTok{::}\FunctionTok{global}\NormalTok{(centrets,}\AttributeTok{fun=}\StringTok{"rms"}\NormalTok{,}\AttributeTok{na.rm=}\ConstantTok{TRUE}\NormalTok{)}
\NormalTok{merogots}\OtherTok{=}\NormalTok{centrets}\SpecialCharTok{/}\NormalTok{standartnovirze[,}\DecValTok{1}\NormalTok{]}
\FunctionTok{writeRaster}\NormalTok{(merogots,}
      \AttributeTok{filename=}\NormalTok{saglabasanas\_cels,}
      \AttributeTok{overwrite=}\ConstantTok{TRUE}\NormalTok{)}
\end{Highlighting}
\end{Shaded}

\section{ForestsAge\_ClearcutsLowStands\_r3000}\label{ch06.273}

\textbf{filename:} \texttt{ForestsAge\_ClearcutsLowStands\_r3000.tif}

\textbf{layername:} \texttt{egv\_273}

\textbf{English name:} Fractional cover of Clearcuts and Forest Stands lower than 5 m within
the 3 km landscape

\textbf{Latvian name:} Izcirtumu un mežaudžu līdz 5 m augstumam platības īpatsvars 3
km ainavā

\textbf{Procedure:} The cover fraction within a radius of 3000 m around the analysis grid cell
is calculated as the area-weighted sum of the \hyperref[ch06.270]{analysis cells} inside
the buffer, using the workflow \texttt{egvtools::radius\_function()}. During the calculation of the landscape
metric, inverse distance weighted (power = 2) gap filling on the output is
applied to ensure no missing values at the edges. Then the layer is
rewritten to set its name. Finally, the layer is standardised by
subtracting the arithmetic mean and dividing by the root mean squared error.

\begin{Shaded}
\begin{Highlighting}[]
\CommentTok{\# libs {-}{-}{-}{-}}
\ControlFlowTok{if}\NormalTok{(}\SpecialCharTok{!}\FunctionTok{require}\NormalTok{(terra)) \{}\FunctionTok{install.packages}\NormalTok{(}\StringTok{"terra"}\NormalTok{); }\FunctionTok{require}\NormalTok{(terra)\}}
\ControlFlowTok{if}\NormalTok{(}\SpecialCharTok{!}\FunctionTok{require}\NormalTok{(egvtools)) \{remotes}\SpecialCharTok{::}\FunctionTok{install\_github}\NormalTok{(}\StringTok{"aavotins/egvtools"}\NormalTok{); }\FunctionTok{require}\NormalTok{(egvtools)\}}


\CommentTok{\# Templates {-}{-}{-}{-}{-}}
\NormalTok{template100}\OtherTok{=}\FunctionTok{rast}\NormalTok{(}\StringTok{"./Templates/TemplateRasters/LV100m\_10km.tif"}\NormalTok{)}

\CommentTok{\# radii {-}{-}{-}{-}}
\FunctionTok{radius\_function}\NormalTok{(}
 \AttributeTok{kvadrati\_path =} \StringTok{"./Templates/TemplateGrids/tiles/"}\NormalTok{,}
 \AttributeTok{radii\_path   =} \StringTok{"./Templates/TemplateGridPoints/tiles/"}\NormalTok{,}
 \AttributeTok{tikls100\_path =} \StringTok{"./Templates/TemplateGrids/tikls100\_sauzeme.parquet"}\NormalTok{,}
 \AttributeTok{template\_path =} \StringTok{"./Templates/TemplateRasters/LV100m\_10km.tif"}\NormalTok{,}
 \AttributeTok{input\_layers  =} \FunctionTok{c}\NormalTok{(}\StringTok{"./RasterGrids\_100m/2024/RAW/ForestsAge\_ClearcutsLowStands\_cell.tif"}\NormalTok{),}
 \AttributeTok{layer\_prefixes =} \FunctionTok{c}\NormalTok{(}\StringTok{"ForestsAge\_ClearcutsLowStands"}\NormalTok{),}
 \AttributeTok{output\_dir   =} \StringTok{"./RasterGrids\_100m/2024/RAW/"}\NormalTok{,}
 \AttributeTok{n\_workers   =} \DecValTok{6}\NormalTok{,}
 \AttributeTok{radii     =} \FunctionTok{c}\NormalTok{(}\StringTok{"r3000"}\NormalTok{),}
 \AttributeTok{radius\_mode  =} \StringTok{"sparse"}\NormalTok{,}
 \AttributeTok{extract\_fun  =} \StringTok{"mean"}\NormalTok{,}
 \AttributeTok{fill\_missing  =} \ConstantTok{TRUE}\NormalTok{,}
 \AttributeTok{IDW\_weight   =} \DecValTok{2}\NormalTok{,}
 \AttributeTok{future\_max\_size =} \DecValTok{40} \SpecialCharTok{*} \DecValTok{1024}\SpecialCharTok{\^{}}\DecValTok{3}\NormalTok{)}


\CommentTok{\# ForestsAge\_ClearcutsLowStands\_r3000.tif   egv\_273}
\NormalTok{slanis}\OtherTok{=}\FunctionTok{rast}\NormalTok{(}\StringTok{"./RasterGrids\_100m/2024/RAW/ForestsAge\_ClearcutsLowStands\_r3000.tif"}\NormalTok{)}
\FunctionTok{names}\NormalTok{(slanis)}\OtherTok{=}\StringTok{"egv\_273"}
\NormalTok{slanis2}\OtherTok{=}\FunctionTok{project}\NormalTok{(slanis,template100)}
\FunctionTok{writeRaster}\NormalTok{(slanis2,}
      \StringTok{"./RasterGrids\_100m/2024/RAW/ForestsAge\_ClearcutsLowStands\_r3000.tif"}\NormalTok{,}
      \AttributeTok{overwrite=}\ConstantTok{TRUE}\NormalTok{)}

\CommentTok{\# standardisation {-}{-}{-}{-}}
\ControlFlowTok{if}\NormalTok{(}\SpecialCharTok{!}\FunctionTok{require}\NormalTok{(terra)) \{}\FunctionTok{install.packages}\NormalTok{(}\StringTok{"terra"}\NormalTok{); }\FunctionTok{require}\NormalTok{(terra)\}}
\ControlFlowTok{if}\NormalTok{(}\SpecialCharTok{!}\FunctionTok{require}\NormalTok{(tidyverse)) \{}\FunctionTok{install.packages}\NormalTok{(}\StringTok{"tidyverse"}\NormalTok{); }\FunctionTok{require}\NormalTok{(tidyverse)\}}

\NormalTok{nosaukums}\OtherTok{=}\StringTok{"ForestsAge\_ClearcutsLowStands\_r3000.tif"}
\NormalTok{ielasisanas\_cels}\OtherTok{=}\FunctionTok{paste0}\NormalTok{(}\StringTok{"./RasterGrids\_100m/2024/RAW/"}\NormalTok{,nosaukums)}
\NormalTok{saglabasanas\_cels}\OtherTok{=}\FunctionTok{paste0}\NormalTok{(}\StringTok{"./RasterGrids\_100m/2024/Scaled/"}\NormalTok{,nosaukums)}
\NormalTok{slanis}\OtherTok{=}\FunctionTok{rast}\NormalTok{(ielasisanas\_cels)}
\NormalTok{videjais}\OtherTok{=}\FunctionTok{global}\NormalTok{(slanis,}\AttributeTok{fun=}\StringTok{"mean"}\NormalTok{,}\AttributeTok{na.rm=}\ConstantTok{TRUE}\NormalTok{)}
\NormalTok{centrets}\OtherTok{=}\NormalTok{slanis}\SpecialCharTok{{-}}\NormalTok{videjais[,}\DecValTok{1}\NormalTok{]}
\NormalTok{standartnovirze}\OtherTok{=}\NormalTok{terra}\SpecialCharTok{::}\FunctionTok{global}\NormalTok{(centrets,}\AttributeTok{fun=}\StringTok{"rms"}\NormalTok{,}\AttributeTok{na.rm=}\ConstantTok{TRUE}\NormalTok{)}
\NormalTok{merogots}\OtherTok{=}\NormalTok{centrets}\SpecialCharTok{/}\NormalTok{standartnovirze[,}\DecValTok{1}\NormalTok{]}
\FunctionTok{writeRaster}\NormalTok{(merogots,}
      \AttributeTok{filename=}\NormalTok{saglabasanas\_cels,}
      \AttributeTok{overwrite=}\ConstantTok{TRUE}\NormalTok{)}
\end{Highlighting}
\end{Shaded}

\section{ForestsAge\_ClearcutsLowStands\_r10000}\label{ch06.274}

\textbf{filename:} \texttt{ForestsAge\_ClearcutsLowStands\_r10000.tif}

\textbf{layername:} \texttt{egv\_274}

\textbf{English name:} Fractional cover of Clearcuts and Forest Stands lower than 5 m within
the 10 km landscape

\textbf{Latvian name:} Izcirtumu un mežaudžu līdz 5 m augstumam platības īpatsvars 10
km ainavā

\textbf{Procedure:} The cover fraction within a radius of 10000 m around the analysis grid cell
is calculated as the area-weighted sum of the \hyperref[ch06.270]{analysis cells} inside
the buffer, using the workflow \texttt{egvtools::radius\_function()}. During the calculation of the landscape
metric, inverse distance weighted (power = 2) gap filling on the output is
applied to ensure no missing values at the edges. Then the layer is
rewritten to set its name. Finally, the layer is standardised by
subtracting the arithmetic mean and dividing by the root mean squared error.

\begin{Shaded}
\begin{Highlighting}[]
\CommentTok{\# libs {-}{-}{-}{-}}
\ControlFlowTok{if}\NormalTok{(}\SpecialCharTok{!}\FunctionTok{require}\NormalTok{(terra)) \{}\FunctionTok{install.packages}\NormalTok{(}\StringTok{"terra"}\NormalTok{); }\FunctionTok{require}\NormalTok{(terra)\}}
\ControlFlowTok{if}\NormalTok{(}\SpecialCharTok{!}\FunctionTok{require}\NormalTok{(egvtools)) \{remotes}\SpecialCharTok{::}\FunctionTok{install\_github}\NormalTok{(}\StringTok{"aavotins/egvtools"}\NormalTok{); }\FunctionTok{require}\NormalTok{(egvtools)\}}


\CommentTok{\# Templates {-}{-}{-}{-}{-}}
\NormalTok{template100}\OtherTok{=}\FunctionTok{rast}\NormalTok{(}\StringTok{"./Templates/TemplateRasters/LV100m\_10km.tif"}\NormalTok{)}

\CommentTok{\# radii {-}{-}{-}{-}}
\FunctionTok{radius\_function}\NormalTok{(}
 \AttributeTok{kvadrati\_path =} \StringTok{"./Templates/TemplateGrids/tiles/"}\NormalTok{,}
 \AttributeTok{radii\_path   =} \StringTok{"./Templates/TemplateGridPoints/tiles/"}\NormalTok{,}
 \AttributeTok{tikls100\_path =} \StringTok{"./Templates/TemplateGrids/tikls100\_sauzeme.parquet"}\NormalTok{,}
 \AttributeTok{template\_path =} \StringTok{"./Templates/TemplateRasters/LV100m\_10km.tif"}\NormalTok{,}
 \AttributeTok{input\_layers  =} \FunctionTok{c}\NormalTok{(}\StringTok{"./RasterGrids\_100m/2024/RAW/ForestsAge\_ClearcutsLowStands\_cell.tif"}\NormalTok{),}
 \AttributeTok{layer\_prefixes =} \FunctionTok{c}\NormalTok{(}\StringTok{"ForestsAge\_ClearcutsLowStands"}\NormalTok{),}
 \AttributeTok{output\_dir   =} \StringTok{"./RasterGrids\_100m/2024/RAW/"}\NormalTok{,}
 \AttributeTok{n\_workers   =} \DecValTok{6}\NormalTok{,}
 \AttributeTok{radii     =} \FunctionTok{c}\NormalTok{(}\StringTok{"r10000"}\NormalTok{),}
 \AttributeTok{radius\_mode  =} \StringTok{"sparse"}\NormalTok{,}
 \AttributeTok{extract\_fun  =} \StringTok{"mean"}\NormalTok{,}
 \AttributeTok{fill\_missing  =} \ConstantTok{TRUE}\NormalTok{,}
 \AttributeTok{IDW\_weight   =} \DecValTok{2}\NormalTok{,}
 \AttributeTok{future\_max\_size =} \DecValTok{40} \SpecialCharTok{*} \DecValTok{1024}\SpecialCharTok{\^{}}\DecValTok{3}\NormalTok{)}


\CommentTok{\# ForestsAge\_ClearcutsLowStands\_r10000.tif  egv\_274}
\NormalTok{slanis}\OtherTok{=}\FunctionTok{rast}\NormalTok{(}\StringTok{"./RasterGrids\_100m/2024/RAW/ForestsAge\_ClearcutsLowStands\_r10000.tif"}\NormalTok{)}
\FunctionTok{names}\NormalTok{(slanis)}\OtherTok{=}\StringTok{"egv\_274"}
\NormalTok{slanis2}\OtherTok{=}\FunctionTok{project}\NormalTok{(slanis,template100)}
\FunctionTok{writeRaster}\NormalTok{(slanis2,}
      \StringTok{"./RasterGrids\_100m/2024/RAW/ForestsAge\_ClearcutsLowStands\_r10000.tif"}\NormalTok{,}
      \AttributeTok{overwrite=}\ConstantTok{TRUE}\NormalTok{)}

\CommentTok{\# standardisation {-}{-}{-}{-}}
\ControlFlowTok{if}\NormalTok{(}\SpecialCharTok{!}\FunctionTok{require}\NormalTok{(terra)) \{}\FunctionTok{install.packages}\NormalTok{(}\StringTok{"terra"}\NormalTok{); }\FunctionTok{require}\NormalTok{(terra)\}}
\ControlFlowTok{if}\NormalTok{(}\SpecialCharTok{!}\FunctionTok{require}\NormalTok{(tidyverse)) \{}\FunctionTok{install.packages}\NormalTok{(}\StringTok{"tidyverse"}\NormalTok{); }\FunctionTok{require}\NormalTok{(tidyverse)\}}

\NormalTok{nosaukums}\OtherTok{=}\StringTok{"ForestsAge\_ClearcutsLowStands\_r10000.tif"}
\NormalTok{ielasisanas\_cels}\OtherTok{=}\FunctionTok{paste0}\NormalTok{(}\StringTok{"./RasterGrids\_100m/2024/RAW/"}\NormalTok{,nosaukums)}
\NormalTok{saglabasanas\_cels}\OtherTok{=}\FunctionTok{paste0}\NormalTok{(}\StringTok{"./RasterGrids\_100m/2024/Scaled/"}\NormalTok{,nosaukums)}
\NormalTok{slanis}\OtherTok{=}\FunctionTok{rast}\NormalTok{(ielasisanas\_cels)}
\NormalTok{videjais}\OtherTok{=}\FunctionTok{global}\NormalTok{(slanis,}\AttributeTok{fun=}\StringTok{"mean"}\NormalTok{,}\AttributeTok{na.rm=}\ConstantTok{TRUE}\NormalTok{)}
\NormalTok{centrets}\OtherTok{=}\NormalTok{slanis}\SpecialCharTok{{-}}\NormalTok{videjais[,}\DecValTok{1}\NormalTok{]}
\NormalTok{standartnovirze}\OtherTok{=}\NormalTok{terra}\SpecialCharTok{::}\FunctionTok{global}\NormalTok{(centrets,}\AttributeTok{fun=}\StringTok{"rms"}\NormalTok{,}\AttributeTok{na.rm=}\ConstantTok{TRUE}\NormalTok{)}
\NormalTok{merogots}\OtherTok{=}\NormalTok{centrets}\SpecialCharTok{/}\NormalTok{standartnovirze[,}\DecValTok{1}\NormalTok{]}
\FunctionTok{writeRaster}\NormalTok{(merogots,}
      \AttributeTok{filename=}\NormalTok{saglabasanas\_cels,}
      \AttributeTok{overwrite=}\ConstantTok{TRUE}\NormalTok{)}
\end{Highlighting}
\end{Shaded}

\section{ForestsAge\_Middle\_cell}\label{ch06.275}

\textbf{filename:} \texttt{ForestsAge\_Middle\_cell.tif}

\textbf{layername:} \texttt{egv\_275}

\textbf{English name:} Fractional cover of Middle-Aged Forest Stands within the analysis
cell (1 ha)

\textbf{Latvian name:} Vidēja vecuma un briestaudžu platības īpatsvars analīzes šūnā
(1 ha)

\textbf{Procedure:} Most EGVs describing forests are spatially restricted to areas outside
of clearcuts and dead stands. This mask is created using a combination of
the \hyperref[Ch04.01]{State Forest Service's
State Forest Registry} land category 12 and 14, and \hyperref[Ch04.09]{The
Global Forest Watch} pixels classified as lost tree canopy cover since
2020 (raster layer matching input, presence = 1, absence = 0).

To prepare this
EGV, stands in land category 10 and age groups two and three are selected from
\hyperref[Ch04.01]{State Forest Service's State Forest Registry} and rasterised.
Rasterisation is performed using the workflow \texttt{egvtools::polygon2input()} (presence = 1,
absence = 0), restricting presence locations only outside the clearcut mask. The resulting layer
is then aggregated to EGV resolution using the workflow \texttt{egvtools::input2egv()}, which
calculates the arithmetic mean to determine the cover fraction. During
aggregation, inverse distance weighted (power = 2) gap filling on the output is
applied to ensure no missing values at the edges. Finally, the layer is
standardised by subtracting the arithmetic mean and dividing by the root mean squared
error.

\begin{Shaded}
\begin{Highlighting}[]
\CommentTok{\# libs {-}{-}{-}{-}}
\ControlFlowTok{if}\NormalTok{(}\SpecialCharTok{!}\FunctionTok{require}\NormalTok{(egvtools)) \{remotes}\SpecialCharTok{::}\FunctionTok{install\_github}\NormalTok{(}\StringTok{"aavotins/egvtools"}\NormalTok{); }\FunctionTok{require}\NormalTok{(egvtools)\}}
\ControlFlowTok{if}\NormalTok{(}\SpecialCharTok{!}\FunctionTok{require}\NormalTok{(terra)) \{}\FunctionTok{install.packages}\NormalTok{(}\StringTok{"terra"}\NormalTok{); }\FunctionTok{require}\NormalTok{(terra)\}}
\ControlFlowTok{if}\NormalTok{(}\SpecialCharTok{!}\FunctionTok{require}\NormalTok{(sf)) \{}\FunctionTok{install.packages}\NormalTok{(}\StringTok{"sf"}\NormalTok{); }\FunctionTok{require}\NormalTok{(sf)\}}
\ControlFlowTok{if}\NormalTok{(}\SpecialCharTok{!}\FunctionTok{require}\NormalTok{(tidyverse)) \{}\FunctionTok{install.packages}\NormalTok{(}\StringTok{"tidyverse"}\NormalTok{); }\FunctionTok{require}\NormalTok{(tidyverse)\}}
\ControlFlowTok{if}\NormalTok{(}\SpecialCharTok{!}\FunctionTok{require}\NormalTok{(sfarrow)) \{}\FunctionTok{install.packages}\NormalTok{(}\StringTok{"sfarrow"}\NormalTok{); }\FunctionTok{require}\NormalTok{(sfarrow)\}}
\ControlFlowTok{if}\NormalTok{(}\SpecialCharTok{!}\FunctionTok{require}\NormalTok{(readxl)) \{}\FunctionTok{install.packages}\NormalTok{(}\StringTok{"readxl"}\NormalTok{); }\FunctionTok{require}\NormalTok{(readxl)\}}
\ControlFlowTok{if}\NormalTok{(}\SpecialCharTok{!}\FunctionTok{require}\NormalTok{(raster)) \{}\FunctionTok{install.packages}\NormalTok{(}\StringTok{"raster"}\NormalTok{); }\FunctionTok{require}\NormalTok{(raster)\}}
\ControlFlowTok{if}\NormalTok{(}\SpecialCharTok{!}\FunctionTok{require}\NormalTok{(fasterize)) \{}\FunctionTok{install.packages}\NormalTok{(}\StringTok{"fasterize"}\NormalTok{); }\FunctionTok{require}\NormalTok{(fasterize)\}}

\CommentTok{\# templates {-}{-}{-}{-}}
\NormalTok{template100}\OtherTok{=}\FunctionTok{rast}\NormalTok{(}\StringTok{"./Templates/TemplateRasters/LV100m\_10km.tif"}\NormalTok{)}
\NormalTok{template10}\OtherTok{=}\FunctionTok{rast}\NormalTok{(}\StringTok{"./Templates/TemplateRasters/LV10m\_10km.tif"}\NormalTok{)}
\NormalTok{rastrs10}\OtherTok{=}\FunctionTok{raster}\NormalTok{(template10)}

\NormalTok{nulls10}\OtherTok{=}\FunctionTok{rast}\NormalTok{(}\StringTok{"./Templates/TemplateRasters/nulls\_LV10m\_10km.tif"}\NormalTok{)}
\NormalTok{nulls100}\OtherTok{=}\FunctionTok{rast}\NormalTok{(}\StringTok{"./Templates/TemplateRasters/nulls\_LV100m\_10km.tif"}\NormalTok{)}


\CommentTok{\# simple landscape {-}{-}{-}{-}}
\NormalTok{simple\_landscape}\OtherTok{=}\FunctionTok{rast}\NormalTok{(}\StringTok{"RasterGrids\_10m/2024/Ainava\_vienk\_mask.tif"}\NormalTok{)}

\CommentTok{\# mvr {-}{-}{-}{-}}
\NormalTok{mvr}\OtherTok{=}\FunctionTok{st\_read\_parquet}\NormalTok{(}\StringTok{"./Geodata/2024/MVR/nogabali\_2024janv.parquet"}\NormalTok{)}
\NormalTok{mvr}\SpecialCharTok{$}\NormalTok{yes}\OtherTok{=}\DecValTok{1}

\CommentTok{\# clear cut mask {-}{-}{-}{-}}
\NormalTok{izcirtumi}\OtherTok{=}\NormalTok{mvr }\SpecialCharTok{\%\textgreater{}\%} 
 \FunctionTok{filter}\NormalTok{(zkat }\SpecialCharTok{\%in\%} \FunctionTok{c}\NormalTok{(}\StringTok{"12"}\NormalTok{,}\StringTok{"14"}\NormalTok{)) }\SpecialCharTok{\%\textgreater{}\%} 
\NormalTok{ dplyr}\SpecialCharTok{::}\FunctionTok{select}\NormalTok{(yes)}
\NormalTok{r\_izcirtumi\_mvr}\OtherTok{=}\FunctionTok{fasterize}\NormalTok{(izcirtumi,rastrs10,}\AttributeTok{field=}\StringTok{"yes"}\NormalTok{)}
\NormalTok{t\_izcirtumi\_mvr}\OtherTok{=}\FunctionTok{rast}\NormalTok{(r\_izcirtumi\_mvr)}
\FunctionTok{plot}\NormalTok{(t\_izcirtumi\_mvr)}

\NormalTok{tcl}\OtherTok{=}\FunctionTok{rast}\NormalTok{(}\StringTok{"./Geodata/2024/Trees/GFW/TreeCoverLoss\_v1\_12.tif"}\NormalTok{)}
\NormalTok{tcl2}\OtherTok{=}\FunctionTok{ifel}\NormalTok{(tcl}\SpecialCharTok{\textless{}}\DecValTok{20}\NormalTok{,}\DecValTok{0}\NormalTok{,}\DecValTok{1}\NormalTok{)}
\NormalTok{tclX}\OtherTok{=}\FunctionTok{cover}\NormalTok{(tcl2,nulls10)}
\FunctionTok{plot}\NormalTok{(tclX)}

\NormalTok{clearcut\_mask}\OtherTok{=}\FunctionTok{cover}\NormalTok{(t\_izcirtumi\_mvr,tclX,}
          \AttributeTok{filename=}\StringTok{"./RasterGrids\_10m/2024/Mask\_clearcuts.tif"}\NormalTok{,}
          \AttributeTok{overwrite=}\ConstantTok{TRUE}\NormalTok{)}
\FunctionTok{plot}\NormalTok{(clearcut\_mask)}

\FunctionTok{rm}\NormalTok{(izcirtumi)}
\FunctionTok{rm}\NormalTok{(r\_izcirtumi\_mvr)}
\FunctionTok{rm}\NormalTok{(t\_izcirtumi\_mvr)}
\FunctionTok{rm}\NormalTok{(tcl)}
\FunctionTok{rm}\NormalTok{(tcl2)}
\FunctionTok{rm}\NormalTok{(tclX)}

\CommentTok{\# ForestsAge\_Middle\_cell.tif    egv\_275 {-}{-}{-}{-}}
\NormalTok{videjas\_audzes}\OtherTok{=}\NormalTok{mvr }\SpecialCharTok{\%\textgreater{}\%} 
 \FunctionTok{filter}\NormalTok{(zkat}\SpecialCharTok{==}\StringTok{"10"}\NormalTok{) }\SpecialCharTok{\%\textgreater{}\%} 
 \CommentTok{\#filter(h10\textgreater{}=5) \%\textgreater{}\% }
 \FunctionTok{filter}\NormalTok{(vgr }\SpecialCharTok{\%in\%} \FunctionTok{c}\NormalTok{(}\StringTok{"2"}\NormalTok{,}\StringTok{"3"}\NormalTok{)) }\SpecialCharTok{\%\textgreater{}\%} 
\NormalTok{ dplyr}\SpecialCharTok{::}\FunctionTok{select}\NormalTok{(yes)}

\NormalTok{p2i\_rez}\OtherTok{=}\NormalTok{egvtools}\SpecialCharTok{::}\FunctionTok{polygon2input}\NormalTok{(}\AttributeTok{vector\_data =}\NormalTok{ videjas\_audzes,}
                \AttributeTok{template\_path =} \StringTok{"./Templates/TemplateRasters/LV10m\_10km.tif"}\NormalTok{,}
                \AttributeTok{out\_path =} \StringTok{"./RasterGrids\_10m/2024/"}\NormalTok{,}
                \AttributeTok{file\_name =} \StringTok{"ForestsAge\_Middle\_input.tif"}\NormalTok{,}
                \AttributeTok{value\_field =} \StringTok{"yes"}\NormalTok{,}
                \AttributeTok{restrict\_to =}\NormalTok{ clearcut\_mask,}
                \AttributeTok{restrict\_values =} \DecValTok{0}\NormalTok{,}
                \AttributeTok{prepare=}\ConstantTok{FALSE}\NormalTok{,}
                \AttributeTok{background\_raster =} \StringTok{"./Templates/TemplateRasters/nulls\_LV10m\_10km.tif"}\NormalTok{,}
                \AttributeTok{plot\_result =} \ConstantTok{TRUE}\NormalTok{)}
\NormalTok{p2i\_rez}
\NormalTok{i2e\_rez}\OtherTok{=}\NormalTok{egvtools}\SpecialCharTok{::}\FunctionTok{input2egv}\NormalTok{(}\AttributeTok{input=}\FunctionTok{paste0}\NormalTok{(}\StringTok{"./RasterGrids\_10m/2024/"}\NormalTok{,}
                     \StringTok{"ForestsAge\_Middle\_input.tif"}\NormalTok{),}
              \AttributeTok{egv\_template=} \StringTok{"./Templates/TemplateRasters/LV100m\_10km.tif"}\NormalTok{,}
              \AttributeTok{summary\_function =} \StringTok{"average"}\NormalTok{,}
              \AttributeTok{missing\_job =} \StringTok{"FillOutput"}\NormalTok{,}
              \AttributeTok{outlocation =} \StringTok{"./RasterGrids\_100m/2024/RAW/"}\NormalTok{,}
              \AttributeTok{outfilename =} \StringTok{"ForestsAge\_Middle\_cell.tif"}\NormalTok{,}
              \AttributeTok{layername =} \StringTok{"egv\_275"}\NormalTok{,}
              \AttributeTok{idw\_weight =} \DecValTok{2}\NormalTok{,}
              \AttributeTok{plot\_gaps =} \ConstantTok{FALSE}\NormalTok{,}\AttributeTok{plot\_final =} \ConstantTok{TRUE}\NormalTok{)}
\NormalTok{i2e\_rez}
\FunctionTok{rm}\NormalTok{(videjas\_audzes)}
\FunctionTok{rm}\NormalTok{(p2i\_rez)}
\FunctionTok{rm}\NormalTok{(i2e\_rez)}
\FunctionTok{unlink}\NormalTok{(}\StringTok{"./RasterGrids\_10m/2024/ForestsAge\_Middle\_input.tif"}\NormalTok{)}

\CommentTok{\# standardisation {-}{-}{-}{-}}
\ControlFlowTok{if}\NormalTok{(}\SpecialCharTok{!}\FunctionTok{require}\NormalTok{(terra)) \{}\FunctionTok{install.packages}\NormalTok{(}\StringTok{"terra"}\NormalTok{); }\FunctionTok{require}\NormalTok{(terra)\}}
\ControlFlowTok{if}\NormalTok{(}\SpecialCharTok{!}\FunctionTok{require}\NormalTok{(tidyverse)) \{}\FunctionTok{install.packages}\NormalTok{(}\StringTok{"tidyverse"}\NormalTok{); }\FunctionTok{require}\NormalTok{(tidyverse)\}}

\NormalTok{nosaukums}\OtherTok{=}\StringTok{"ForestsAge\_Middle\_cell.tif"}
\NormalTok{ielasisanas\_cels}\OtherTok{=}\FunctionTok{paste0}\NormalTok{(}\StringTok{"./RasterGrids\_100m/2024/RAW/"}\NormalTok{,nosaukums)}
\NormalTok{saglabasanas\_cels}\OtherTok{=}\FunctionTok{paste0}\NormalTok{(}\StringTok{"./RasterGrids\_100m/2024/Scaled/"}\NormalTok{,nosaukums)}
\NormalTok{slanis}\OtherTok{=}\FunctionTok{rast}\NormalTok{(ielasisanas\_cels)}
\NormalTok{videjais}\OtherTok{=}\FunctionTok{global}\NormalTok{(slanis,}\AttributeTok{fun=}\StringTok{"mean"}\NormalTok{,}\AttributeTok{na.rm=}\ConstantTok{TRUE}\NormalTok{)}
\NormalTok{centrets}\OtherTok{=}\NormalTok{slanis}\SpecialCharTok{{-}}\NormalTok{videjais[,}\DecValTok{1}\NormalTok{]}
\NormalTok{standartnovirze}\OtherTok{=}\NormalTok{terra}\SpecialCharTok{::}\FunctionTok{global}\NormalTok{(centrets,}\AttributeTok{fun=}\StringTok{"rms"}\NormalTok{,}\AttributeTok{na.rm=}\ConstantTok{TRUE}\NormalTok{)}
\NormalTok{merogots}\OtherTok{=}\NormalTok{centrets}\SpecialCharTok{/}\NormalTok{standartnovirze[,}\DecValTok{1}\NormalTok{]}
\FunctionTok{writeRaster}\NormalTok{(merogots,}
      \AttributeTok{filename=}\NormalTok{saglabasanas\_cels,}
      \AttributeTok{overwrite=}\ConstantTok{TRUE}\NormalTok{)}
\end{Highlighting}
\end{Shaded}

\section{ForestsAge\_Middle\_r500}\label{ch06.276}

\textbf{filename:} \texttt{ForestsAge\_Middle\_r500.tif}

\textbf{layername:} \texttt{egv\_276}

\textbf{English name:} Fractional cover of Middle-Aged Forest Stands within the 0.5 km
landscape

\textbf{Latvian name:} Vidēja vecuma un briestaudžu platības īpatsvars 0,5 km ainavā

\textbf{Procedure:} The cover fraction within a radius of 500 m around the analysis grid cell is
calculated as the area-weighted sum of the \hyperref[ch06.275]{analysis cells} inside the
buffer, using the workflow \texttt{egvtools::radius\_function()}. During the calculation of the landscape metric,
inverse distance weighted (power = 2) gap filling on the output is applied
to ensure no missing values at the edges. Then the layer is rewritten to set
its name. Finally, the layer is standardised by subtracting the arithmetic
mean and dividing by the root mean squared error.

\begin{Shaded}
\begin{Highlighting}[]
\CommentTok{\# libs {-}{-}{-}{-}}
\ControlFlowTok{if}\NormalTok{(}\SpecialCharTok{!}\FunctionTok{require}\NormalTok{(terra)) \{}\FunctionTok{install.packages}\NormalTok{(}\StringTok{"terra"}\NormalTok{); }\FunctionTok{require}\NormalTok{(terra)\}}
\ControlFlowTok{if}\NormalTok{(}\SpecialCharTok{!}\FunctionTok{require}\NormalTok{(egvtools)) \{remotes}\SpecialCharTok{::}\FunctionTok{install\_github}\NormalTok{(}\StringTok{"aavotins/egvtools"}\NormalTok{); }\FunctionTok{require}\NormalTok{(egvtools)\}}


\CommentTok{\# Templates {-}{-}{-}{-}{-}}
\NormalTok{template100}\OtherTok{=}\FunctionTok{rast}\NormalTok{(}\StringTok{"./Templates/TemplateRasters/LV100m\_10km.tif"}\NormalTok{)}

\CommentTok{\# radii {-}{-}{-}{-}}
\FunctionTok{radius\_function}\NormalTok{(}
 \AttributeTok{kvadrati\_path =} \StringTok{"./Templates/TemplateGrids/tiles/"}\NormalTok{,}
 \AttributeTok{radii\_path   =} \StringTok{"./Templates/TemplateGridPoints/tiles/"}\NormalTok{,}
 \AttributeTok{tikls100\_path =} \StringTok{"./Templates/TemplateGrids/tikls100\_sauzeme.parquet"}\NormalTok{,}
 \AttributeTok{template\_path =} \StringTok{"./Templates/TemplateRasters/LV100m\_10km.tif"}\NormalTok{,}
 \AttributeTok{input\_layers  =} \FunctionTok{c}\NormalTok{(}\StringTok{"./RasterGrids\_100m/2024/RAW/ForestsAge\_Middle\_cell.tif"}\NormalTok{),}
 \AttributeTok{layer\_prefixes =} \FunctionTok{c}\NormalTok{(}\StringTok{"ForestsAge\_Middle"}\NormalTok{),}
 \AttributeTok{output\_dir   =} \StringTok{"./RasterGrids\_100m/2024/RAW/"}\NormalTok{,}
 \AttributeTok{n\_workers   =} \DecValTok{6}\NormalTok{,}
 \AttributeTok{radii     =} \FunctionTok{c}\NormalTok{(}\StringTok{"r500"}\NormalTok{),}
 \AttributeTok{radius\_mode  =} \StringTok{"sparse"}\NormalTok{,}
 \AttributeTok{extract\_fun  =} \StringTok{"mean"}\NormalTok{,}
 \AttributeTok{fill\_missing  =} \ConstantTok{TRUE}\NormalTok{,}
 \AttributeTok{IDW\_weight   =} \DecValTok{2}\NormalTok{,}
 \AttributeTok{future\_max\_size =} \DecValTok{40} \SpecialCharTok{*} \DecValTok{1024}\SpecialCharTok{\^{}}\DecValTok{3}\NormalTok{)}


\CommentTok{\# ForestsAge\_Middle\_r500.tif    egv\_276}
\NormalTok{slanis}\OtherTok{=}\FunctionTok{rast}\NormalTok{(}\StringTok{"./RasterGrids\_100m/2024/RAW/ForestsAge\_Middle\_r500.tif"}\NormalTok{)}
\FunctionTok{names}\NormalTok{(slanis)}\OtherTok{=}\StringTok{"egv\_276"}
\NormalTok{slanis2}\OtherTok{=}\FunctionTok{project}\NormalTok{(slanis,template100)}
\FunctionTok{writeRaster}\NormalTok{(slanis2,}
      \StringTok{"./RasterGrids\_100m/2024/RAW/ForestsAge\_Middle\_r500.tif"}\NormalTok{,}
      \AttributeTok{overwrite=}\ConstantTok{TRUE}\NormalTok{)}

\CommentTok{\# standardisation {-}{-}{-}{-}}
\ControlFlowTok{if}\NormalTok{(}\SpecialCharTok{!}\FunctionTok{require}\NormalTok{(terra)) \{}\FunctionTok{install.packages}\NormalTok{(}\StringTok{"terra"}\NormalTok{); }\FunctionTok{require}\NormalTok{(terra)\}}
\ControlFlowTok{if}\NormalTok{(}\SpecialCharTok{!}\FunctionTok{require}\NormalTok{(tidyverse)) \{}\FunctionTok{install.packages}\NormalTok{(}\StringTok{"tidyverse"}\NormalTok{); }\FunctionTok{require}\NormalTok{(tidyverse)\}}

\NormalTok{nosaukums}\OtherTok{=}\StringTok{"ForestsAge\_Middle\_r500.tif"}
\NormalTok{ielasisanas\_cels}\OtherTok{=}\FunctionTok{paste0}\NormalTok{(}\StringTok{"./RasterGrids\_100m/2024/RAW/"}\NormalTok{,nosaukums)}
\NormalTok{saglabasanas\_cels}\OtherTok{=}\FunctionTok{paste0}\NormalTok{(}\StringTok{"./RasterGrids\_100m/2024/Scaled/"}\NormalTok{,nosaukums)}
\NormalTok{slanis}\OtherTok{=}\FunctionTok{rast}\NormalTok{(ielasisanas\_cels)}
\NormalTok{videjais}\OtherTok{=}\FunctionTok{global}\NormalTok{(slanis,}\AttributeTok{fun=}\StringTok{"mean"}\NormalTok{,}\AttributeTok{na.rm=}\ConstantTok{TRUE}\NormalTok{)}
\NormalTok{centrets}\OtherTok{=}\NormalTok{slanis}\SpecialCharTok{{-}}\NormalTok{videjais[,}\DecValTok{1}\NormalTok{]}
\NormalTok{standartnovirze}\OtherTok{=}\NormalTok{terra}\SpecialCharTok{::}\FunctionTok{global}\NormalTok{(centrets,}\AttributeTok{fun=}\StringTok{"rms"}\NormalTok{,}\AttributeTok{na.rm=}\ConstantTok{TRUE}\NormalTok{)}
\NormalTok{merogots}\OtherTok{=}\NormalTok{centrets}\SpecialCharTok{/}\NormalTok{standartnovirze[,}\DecValTok{1}\NormalTok{]}
\FunctionTok{writeRaster}\NormalTok{(merogots,}
      \AttributeTok{filename=}\NormalTok{saglabasanas\_cels,}
      \AttributeTok{overwrite=}\ConstantTok{TRUE}\NormalTok{)}
\end{Highlighting}
\end{Shaded}

\section{ForestsAge\_Middle\_r1250}\label{ch06.277}

\textbf{filename:} \texttt{ForestsAge\_Middle\_r1250.tif}

\textbf{layername:} \texttt{egv\_277}

\textbf{English name:} Fractional cover of Middle-Aged Forest Stands within the 1.25 km
landscape

\textbf{Latvian name:} Vidēja vecuma un briestaudžu platības īpatsvars 1,25 km ainavā

\textbf{Procedure:} The cover fraction within a radius of 1250 m around the analysis grid cell
is calculated as the area-weighted sum of the \hyperref[ch06.275]{analysis cells} inside
the buffer, using the workflow \texttt{egvtools::radius\_function()}. During the calculation of the landscape
metric, inverse distance weighted (power = 2) gap filling on the output is
applied to ensure no missing values at the edges. Then the layer is
rewritten to set its name. Finally, the layer is standardised by
subtracting the arithmetic mean and dividing by the root mean squared error.

\begin{Shaded}
\begin{Highlighting}[]
\CommentTok{\# libs {-}{-}{-}{-}}
\ControlFlowTok{if}\NormalTok{(}\SpecialCharTok{!}\FunctionTok{require}\NormalTok{(terra)) \{}\FunctionTok{install.packages}\NormalTok{(}\StringTok{"terra"}\NormalTok{); }\FunctionTok{require}\NormalTok{(terra)\}}
\ControlFlowTok{if}\NormalTok{(}\SpecialCharTok{!}\FunctionTok{require}\NormalTok{(egvtools)) \{remotes}\SpecialCharTok{::}\FunctionTok{install\_github}\NormalTok{(}\StringTok{"aavotins/egvtools"}\NormalTok{); }\FunctionTok{require}\NormalTok{(egvtools)\}}


\CommentTok{\# Templates {-}{-}{-}{-}{-}}
\NormalTok{template100}\OtherTok{=}\FunctionTok{rast}\NormalTok{(}\StringTok{"./Templates/TemplateRasters/LV100m\_10km.tif"}\NormalTok{)}

\CommentTok{\# radii {-}{-}{-}{-}}
\FunctionTok{radius\_function}\NormalTok{(}
 \AttributeTok{kvadrati\_path =} \StringTok{"./Templates/TemplateGrids/tiles/"}\NormalTok{,}
 \AttributeTok{radii\_path   =} \StringTok{"./Templates/TemplateGridPoints/tiles/"}\NormalTok{,}
 \AttributeTok{tikls100\_path =} \StringTok{"./Templates/TemplateGrids/tikls100\_sauzeme.parquet"}\NormalTok{,}
 \AttributeTok{template\_path =} \StringTok{"./Templates/TemplateRasters/LV100m\_10km.tif"}\NormalTok{,}
 \AttributeTok{input\_layers  =} \FunctionTok{c}\NormalTok{(}\StringTok{"./RasterGrids\_100m/2024/RAW/ForestsAge\_Middle\_cell.tif"}\NormalTok{),}
 \AttributeTok{layer\_prefixes =} \FunctionTok{c}\NormalTok{(}\StringTok{"ForestsAge\_Middle"}\NormalTok{),}
 \AttributeTok{output\_dir   =} \StringTok{"./RasterGrids\_100m/2024/RAW/"}\NormalTok{,}
 \AttributeTok{n\_workers   =} \DecValTok{6}\NormalTok{,}
 \AttributeTok{radii     =} \FunctionTok{c}\NormalTok{(}\StringTok{"r1250"}\NormalTok{),}
 \AttributeTok{radius\_mode  =} \StringTok{"sparse"}\NormalTok{,}
 \AttributeTok{extract\_fun  =} \StringTok{"mean"}\NormalTok{,}
 \AttributeTok{fill\_missing  =} \ConstantTok{TRUE}\NormalTok{,}
 \AttributeTok{IDW\_weight   =} \DecValTok{2}\NormalTok{,}
 \AttributeTok{future\_max\_size =} \DecValTok{40} \SpecialCharTok{*} \DecValTok{1024}\SpecialCharTok{\^{}}\DecValTok{3}\NormalTok{)}


\CommentTok{\# ForestsAge\_Middle\_r1250.tif   egv\_277}
\NormalTok{slanis}\OtherTok{=}\FunctionTok{rast}\NormalTok{(}\StringTok{"./RasterGrids\_100m/2024/RAW/ForestsAge\_Middle\_r1250.tif"}\NormalTok{)}
\FunctionTok{names}\NormalTok{(slanis)}\OtherTok{=}\StringTok{"egv\_277"}
\NormalTok{slanis2}\OtherTok{=}\FunctionTok{project}\NormalTok{(slanis,template100)}
\FunctionTok{writeRaster}\NormalTok{(slanis2,}
      \StringTok{"./RasterGrids\_100m/2024/RAW/ForestsAge\_Middle\_r1250.tif"}\NormalTok{,}
      \AttributeTok{overwrite=}\ConstantTok{TRUE}\NormalTok{)}

\CommentTok{\# standardisation {-}{-}{-}{-}}
\ControlFlowTok{if}\NormalTok{(}\SpecialCharTok{!}\FunctionTok{require}\NormalTok{(terra)) \{}\FunctionTok{install.packages}\NormalTok{(}\StringTok{"terra"}\NormalTok{); }\FunctionTok{require}\NormalTok{(terra)\}}
\ControlFlowTok{if}\NormalTok{(}\SpecialCharTok{!}\FunctionTok{require}\NormalTok{(tidyverse)) \{}\FunctionTok{install.packages}\NormalTok{(}\StringTok{"tidyverse"}\NormalTok{); }\FunctionTok{require}\NormalTok{(tidyverse)\}}

\NormalTok{nosaukums}\OtherTok{=}\StringTok{"ForestsAge\_Middle\_r1250.tif"}
\NormalTok{ielasisanas\_cels}\OtherTok{=}\FunctionTok{paste0}\NormalTok{(}\StringTok{"./RasterGrids\_100m/2024/RAW/"}\NormalTok{,nosaukums)}
\NormalTok{saglabasanas\_cels}\OtherTok{=}\FunctionTok{paste0}\NormalTok{(}\StringTok{"./RasterGrids\_100m/2024/Scaled/"}\NormalTok{,nosaukums)}
\NormalTok{slanis}\OtherTok{=}\FunctionTok{rast}\NormalTok{(ielasisanas\_cels)}
\NormalTok{videjais}\OtherTok{=}\FunctionTok{global}\NormalTok{(slanis,}\AttributeTok{fun=}\StringTok{"mean"}\NormalTok{,}\AttributeTok{na.rm=}\ConstantTok{TRUE}\NormalTok{)}
\NormalTok{centrets}\OtherTok{=}\NormalTok{slanis}\SpecialCharTok{{-}}\NormalTok{videjais[,}\DecValTok{1}\NormalTok{]}
\NormalTok{standartnovirze}\OtherTok{=}\NormalTok{terra}\SpecialCharTok{::}\FunctionTok{global}\NormalTok{(centrets,}\AttributeTok{fun=}\StringTok{"rms"}\NormalTok{,}\AttributeTok{na.rm=}\ConstantTok{TRUE}\NormalTok{)}
\NormalTok{merogots}\OtherTok{=}\NormalTok{centrets}\SpecialCharTok{/}\NormalTok{standartnovirze[,}\DecValTok{1}\NormalTok{]}
\FunctionTok{writeRaster}\NormalTok{(merogots,}
      \AttributeTok{filename=}\NormalTok{saglabasanas\_cels,}
      \AttributeTok{overwrite=}\ConstantTok{TRUE}\NormalTok{)}
\end{Highlighting}
\end{Shaded}

\section{ForestsAge\_Middle\_r3000}\label{ch06.278}

\textbf{filename:} \texttt{ForestsAge\_Middle\_r3000.tif}

\textbf{layername:} \texttt{egv\_278}

\textbf{English name:} Fractional cover of Middle-Aged Forest Stands within the 3 km
landscape

\textbf{Latvian name:} Vidēja vecuma un briestaudžu platības īpatsvars 3 km ainavā

\textbf{Procedure:} The cover fraction within a radius of 3000 m around the analysis grid cell
is calculated as the area-weighted sum of the \hyperref[ch06.275]{analysis cells} inside
the buffer, using the workflow \texttt{egvtools::radius\_function()}. During the calculation of the landscape
metric, inverse distance weighted (power = 2) gap filling on the output is
applied to ensure no missing values at the edges. Then the layer is
rewritten to set its name. Finally, the layer is standardised by
subtracting the arithmetic mean and dividing by the root mean squared error.

\begin{Shaded}
\begin{Highlighting}[]
\CommentTok{\# libs {-}{-}{-}{-}}
\ControlFlowTok{if}\NormalTok{(}\SpecialCharTok{!}\FunctionTok{require}\NormalTok{(terra)) \{}\FunctionTok{install.packages}\NormalTok{(}\StringTok{"terra"}\NormalTok{); }\FunctionTok{require}\NormalTok{(terra)\}}
\ControlFlowTok{if}\NormalTok{(}\SpecialCharTok{!}\FunctionTok{require}\NormalTok{(egvtools)) \{remotes}\SpecialCharTok{::}\FunctionTok{install\_github}\NormalTok{(}\StringTok{"aavotins/egvtools"}\NormalTok{); }\FunctionTok{require}\NormalTok{(egvtools)\}}


\CommentTok{\# Templates {-}{-}{-}{-}{-}}
\NormalTok{template100}\OtherTok{=}\FunctionTok{rast}\NormalTok{(}\StringTok{"./Templates/TemplateRasters/LV100m\_10km.tif"}\NormalTok{)}

\CommentTok{\# radii {-}{-}{-}{-}}
\FunctionTok{radius\_function}\NormalTok{(}
 \AttributeTok{kvadrati\_path =} \StringTok{"./Templates/TemplateGrids/tiles/"}\NormalTok{,}
 \AttributeTok{radii\_path   =} \StringTok{"./Templates/TemplateGridPoints/tiles/"}\NormalTok{,}
 \AttributeTok{tikls100\_path =} \StringTok{"./Templates/TemplateGrids/tikls100\_sauzeme.parquet"}\NormalTok{,}
 \AttributeTok{template\_path =} \StringTok{"./Templates/TemplateRasters/LV100m\_10km.tif"}\NormalTok{,}
 \AttributeTok{input\_layers  =} \FunctionTok{c}\NormalTok{(}\StringTok{"./RasterGrids\_100m/2024/RAW/ForestsAge\_Middle\_cell.tif"}\NormalTok{),}
 \AttributeTok{layer\_prefixes =} \FunctionTok{c}\NormalTok{(}\StringTok{"ForestsAge\_Middle"}\NormalTok{),}
 \AttributeTok{output\_dir   =} \StringTok{"./RasterGrids\_100m/2024/RAW/"}\NormalTok{,}
 \AttributeTok{n\_workers   =} \DecValTok{6}\NormalTok{,}
 \AttributeTok{radii     =} \FunctionTok{c}\NormalTok{(}\StringTok{"r3000"}\NormalTok{),}
 \AttributeTok{radius\_mode  =} \StringTok{"sparse"}\NormalTok{,}
 \AttributeTok{extract\_fun  =} \StringTok{"mean"}\NormalTok{,}
 \AttributeTok{fill\_missing  =} \ConstantTok{TRUE}\NormalTok{,}
 \AttributeTok{IDW\_weight   =} \DecValTok{2}\NormalTok{,}
 \AttributeTok{future\_max\_size =} \DecValTok{40} \SpecialCharTok{*} \DecValTok{1024}\SpecialCharTok{\^{}}\DecValTok{3}\NormalTok{)}


\CommentTok{\# ForestsAge\_Middle\_r3000.tif   egv\_278}
\NormalTok{slanis}\OtherTok{=}\FunctionTok{rast}\NormalTok{(}\StringTok{"./RasterGrids\_100m/2024/RAW/ForestsAge\_Middle\_r3000.tif"}\NormalTok{)}
\FunctionTok{names}\NormalTok{(slanis)}\OtherTok{=}\StringTok{"egv\_278"}
\NormalTok{slanis2}\OtherTok{=}\FunctionTok{project}\NormalTok{(slanis,template100)}
\FunctionTok{writeRaster}\NormalTok{(slanis2,}
      \StringTok{"./RasterGrids\_100m/2024/RAW/ForestsAge\_Middle\_r3000.tif"}\NormalTok{,}
      \AttributeTok{overwrite=}\ConstantTok{TRUE}\NormalTok{)}

\CommentTok{\# standardisation {-}{-}{-}{-}}
\ControlFlowTok{if}\NormalTok{(}\SpecialCharTok{!}\FunctionTok{require}\NormalTok{(terra)) \{}\FunctionTok{install.packages}\NormalTok{(}\StringTok{"terra"}\NormalTok{); }\FunctionTok{require}\NormalTok{(terra)\}}
\ControlFlowTok{if}\NormalTok{(}\SpecialCharTok{!}\FunctionTok{require}\NormalTok{(tidyverse)) \{}\FunctionTok{install.packages}\NormalTok{(}\StringTok{"tidyverse"}\NormalTok{); }\FunctionTok{require}\NormalTok{(tidyverse)\}}

\NormalTok{nosaukums}\OtherTok{=}\StringTok{"ForestsAge\_Middle\_r3000.tif"}
\NormalTok{ielasisanas\_cels}\OtherTok{=}\FunctionTok{paste0}\NormalTok{(}\StringTok{"./RasterGrids\_100m/2024/RAW/"}\NormalTok{,nosaukums)}
\NormalTok{saglabasanas\_cels}\OtherTok{=}\FunctionTok{paste0}\NormalTok{(}\StringTok{"./RasterGrids\_100m/2024/Scaled/"}\NormalTok{,nosaukums)}
\NormalTok{slanis}\OtherTok{=}\FunctionTok{rast}\NormalTok{(ielasisanas\_cels)}
\NormalTok{videjais}\OtherTok{=}\FunctionTok{global}\NormalTok{(slanis,}\AttributeTok{fun=}\StringTok{"mean"}\NormalTok{,}\AttributeTok{na.rm=}\ConstantTok{TRUE}\NormalTok{)}
\NormalTok{centrets}\OtherTok{=}\NormalTok{slanis}\SpecialCharTok{{-}}\NormalTok{videjais[,}\DecValTok{1}\NormalTok{]}
\NormalTok{standartnovirze}\OtherTok{=}\NormalTok{terra}\SpecialCharTok{::}\FunctionTok{global}\NormalTok{(centrets,}\AttributeTok{fun=}\StringTok{"rms"}\NormalTok{,}\AttributeTok{na.rm=}\ConstantTok{TRUE}\NormalTok{)}
\NormalTok{merogots}\OtherTok{=}\NormalTok{centrets}\SpecialCharTok{/}\NormalTok{standartnovirze[,}\DecValTok{1}\NormalTok{]}
\FunctionTok{writeRaster}\NormalTok{(merogots,}
      \AttributeTok{filename=}\NormalTok{saglabasanas\_cels,}
      \AttributeTok{overwrite=}\ConstantTok{TRUE}\NormalTok{)}
\end{Highlighting}
\end{Shaded}

\section{ForestsAge\_Middle\_r10000}\label{ch06.279}

\textbf{filename:} \texttt{ForestsAge\_Middle\_r10000.tif}

\textbf{layername:} \texttt{egv\_279}

\textbf{English name:} Fractional cover of Middle-Aged Forest Stands within the 10 km
landscape

\textbf{Latvian name:} Vidēja vecuma un briestaudžu platības īpatsvars 10 km ainavā

\textbf{Procedure:} The cover fraction within a radius of 10000 m around the analysis grid cell
is calculated as the area-weighted sum of the \hyperref[ch06.275]{analysis cells} inside
the buffer, using the workflow \texttt{egvtools::radius\_function()}. During the calculation of the landscape
metric, inverse distance weighted (power = 2) gap filling on the output is
applied to ensure no missing values at the edges. Then the layer is
rewritten to set its name. Finally, the layer is standardised by
subtracting the arithmetic mean and dividing by the root mean squared error.

\begin{Shaded}
\begin{Highlighting}[]
\CommentTok{\# libs {-}{-}{-}{-}}
\ControlFlowTok{if}\NormalTok{(}\SpecialCharTok{!}\FunctionTok{require}\NormalTok{(terra)) \{}\FunctionTok{install.packages}\NormalTok{(}\StringTok{"terra"}\NormalTok{); }\FunctionTok{require}\NormalTok{(terra)\}}
\ControlFlowTok{if}\NormalTok{(}\SpecialCharTok{!}\FunctionTok{require}\NormalTok{(egvtools)) \{remotes}\SpecialCharTok{::}\FunctionTok{install\_github}\NormalTok{(}\StringTok{"aavotins/egvtools"}\NormalTok{); }\FunctionTok{require}\NormalTok{(egvtools)\}}


\CommentTok{\# Templates {-}{-}{-}{-}{-}}
\NormalTok{template100}\OtherTok{=}\FunctionTok{rast}\NormalTok{(}\StringTok{"./Templates/TemplateRasters/LV100m\_10km.tif"}\NormalTok{)}

\CommentTok{\# radii {-}{-}{-}{-}}
\FunctionTok{radius\_function}\NormalTok{(}
 \AttributeTok{kvadrati\_path =} \StringTok{"./Templates/TemplateGrids/tiles/"}\NormalTok{,}
 \AttributeTok{radii\_path   =} \StringTok{"./Templates/TemplateGridPoints/tiles/"}\NormalTok{,}
 \AttributeTok{tikls100\_path =} \StringTok{"./Templates/TemplateGrids/tikls100\_sauzeme.parquet"}\NormalTok{,}
 \AttributeTok{template\_path =} \StringTok{"./Templates/TemplateRasters/LV100m\_10km.tif"}\NormalTok{,}
 \AttributeTok{input\_layers  =} \FunctionTok{c}\NormalTok{(}\StringTok{"./RasterGrids\_100m/2024/RAW/ForestsAge\_Middle\_cell.tif"}\NormalTok{),}
 \AttributeTok{layer\_prefixes =} \FunctionTok{c}\NormalTok{(}\StringTok{"ForestsAge\_Middle"}\NormalTok{),}
 \AttributeTok{output\_dir   =} \StringTok{"./RasterGrids\_100m/2024/RAW/"}\NormalTok{,}
 \AttributeTok{n\_workers   =} \DecValTok{6}\NormalTok{,}
 \AttributeTok{radii     =} \FunctionTok{c}\NormalTok{(}\StringTok{"r10000"}\NormalTok{),}
 \AttributeTok{radius\_mode  =} \StringTok{"sparse"}\NormalTok{,}
 \AttributeTok{extract\_fun  =} \StringTok{"mean"}\NormalTok{,}
 \AttributeTok{fill\_missing  =} \ConstantTok{TRUE}\NormalTok{,}
 \AttributeTok{IDW\_weight   =} \DecValTok{2}\NormalTok{,}
 \AttributeTok{future\_max\_size =} \DecValTok{40} \SpecialCharTok{*} \DecValTok{1024}\SpecialCharTok{\^{}}\DecValTok{3}\NormalTok{)}


\CommentTok{\# ForestsAge\_Middle\_r10000.tif  egv\_279}
\NormalTok{slanis}\OtherTok{=}\FunctionTok{rast}\NormalTok{(}\StringTok{"./RasterGrids\_100m/2024/RAW/ForestsAge\_Middle\_r10000.tif"}\NormalTok{)}
\FunctionTok{names}\NormalTok{(slanis)}\OtherTok{=}\StringTok{"egv\_279"}
\NormalTok{slanis2}\OtherTok{=}\FunctionTok{project}\NormalTok{(slanis,template100)}
\FunctionTok{writeRaster}\NormalTok{(slanis2,}
      \StringTok{"./RasterGrids\_100m/2024/RAW/ForestsAge\_Middle\_r10000.tif"}\NormalTok{,}
      \AttributeTok{overwrite=}\ConstantTok{TRUE}\NormalTok{)}

\CommentTok{\# standardisation {-}{-}{-}{-}}
\ControlFlowTok{if}\NormalTok{(}\SpecialCharTok{!}\FunctionTok{require}\NormalTok{(terra)) \{}\FunctionTok{install.packages}\NormalTok{(}\StringTok{"terra"}\NormalTok{); }\FunctionTok{require}\NormalTok{(terra)\}}
\ControlFlowTok{if}\NormalTok{(}\SpecialCharTok{!}\FunctionTok{require}\NormalTok{(tidyverse)) \{}\FunctionTok{install.packages}\NormalTok{(}\StringTok{"tidyverse"}\NormalTok{); }\FunctionTok{require}\NormalTok{(tidyverse)\}}

\NormalTok{nosaukums}\OtherTok{=}\StringTok{"ForestsAge\_Middle\_r10000.tif"}
\NormalTok{ielasisanas\_cels}\OtherTok{=}\FunctionTok{paste0}\NormalTok{(}\StringTok{"./RasterGrids\_100m/2024/RAW/"}\NormalTok{,nosaukums)}
\NormalTok{saglabasanas\_cels}\OtherTok{=}\FunctionTok{paste0}\NormalTok{(}\StringTok{"./RasterGrids\_100m/2024/Scaled/"}\NormalTok{,nosaukums)}
\NormalTok{slanis}\OtherTok{=}\FunctionTok{rast}\NormalTok{(ielasisanas\_cels)}
\NormalTok{videjais}\OtherTok{=}\FunctionTok{global}\NormalTok{(slanis,}\AttributeTok{fun=}\StringTok{"mean"}\NormalTok{,}\AttributeTok{na.rm=}\ConstantTok{TRUE}\NormalTok{)}
\NormalTok{centrets}\OtherTok{=}\NormalTok{slanis}\SpecialCharTok{{-}}\NormalTok{videjais[,}\DecValTok{1}\NormalTok{]}
\NormalTok{standartnovirze}\OtherTok{=}\NormalTok{terra}\SpecialCharTok{::}\FunctionTok{global}\NormalTok{(centrets,}\AttributeTok{fun=}\StringTok{"rms"}\NormalTok{,}\AttributeTok{na.rm=}\ConstantTok{TRUE}\NormalTok{)}
\NormalTok{merogots}\OtherTok{=}\NormalTok{centrets}\SpecialCharTok{/}\NormalTok{standartnovirze[,}\DecValTok{1}\NormalTok{]}
\FunctionTok{writeRaster}\NormalTok{(merogots,}
      \AttributeTok{filename=}\NormalTok{saglabasanas\_cels,}
      \AttributeTok{overwrite=}\ConstantTok{TRUE}\NormalTok{)}
\end{Highlighting}
\end{Shaded}

\section{ForestsAge\_Old\_cell}\label{ch06.280}

\textbf{filename:} \texttt{ForestsAge\_Old\_cell.tif}

\textbf{layername:} \texttt{egv\_280}

\textbf{English name:} Fractional cover of Old (over rotation age) Forest Stands within the
analysis cell (1 ha)

\textbf{Latvian name:} Vecu (kopš cirtmeta) mežaudžu platības īpatsvars analīzes šūnā (1
ha)

\textbf{Procedure:} Most EGVs describing forests are spatially restricted to areas outside
of clearcuts and dead stands. This mask is created using a combination of
the \hyperref[Ch04.01]{State Forest Service's
State Forest Registry} land category 12 and 14, and \hyperref[Ch04.09]{The
Global Forest Watch} pixels classified as lost tree canopy cover since
2020 (raster layer matching input, presence = 1, absence = 0).

To prepare this
EGV, stands in land category 10 and age groups four and five are selected from
\hyperref[Ch04.01]{State Forest Service's State Forest Registry} and rasterised.
Rasterisation is performed using the workflow \texttt{egvtools::polygon2input()} (presence = 1,
absence = 0), restricting presence locations only outside the clearcut mask. The resulting layer
is then aggregated to EGV resolution using the workflow \texttt{egvtools::input2egv()}, which
calculates the arithmetic mean to determine the cover fraction. During
aggregation, inverse distance weighted (power = 2) gap filling on the output is
applied to ensure no missing values at the edges. Finally, the layer is
standardised by subtracting the arithmetic mean and dividing by the root mean squared
error.

\begin{Shaded}
\begin{Highlighting}[]
\CommentTok{\# libs {-}{-}{-}{-}}
\ControlFlowTok{if}\NormalTok{(}\SpecialCharTok{!}\FunctionTok{require}\NormalTok{(egvtools)) \{remotes}\SpecialCharTok{::}\FunctionTok{install\_github}\NormalTok{(}\StringTok{"aavotins/egvtools"}\NormalTok{); }\FunctionTok{require}\NormalTok{(egvtools)\}}
\ControlFlowTok{if}\NormalTok{(}\SpecialCharTok{!}\FunctionTok{require}\NormalTok{(terra)) \{}\FunctionTok{install.packages}\NormalTok{(}\StringTok{"terra"}\NormalTok{); }\FunctionTok{require}\NormalTok{(terra)\}}
\ControlFlowTok{if}\NormalTok{(}\SpecialCharTok{!}\FunctionTok{require}\NormalTok{(sf)) \{}\FunctionTok{install.packages}\NormalTok{(}\StringTok{"sf"}\NormalTok{); }\FunctionTok{require}\NormalTok{(sf)\}}
\ControlFlowTok{if}\NormalTok{(}\SpecialCharTok{!}\FunctionTok{require}\NormalTok{(tidyverse)) \{}\FunctionTok{install.packages}\NormalTok{(}\StringTok{"tidyverse"}\NormalTok{); }\FunctionTok{require}\NormalTok{(tidyverse)\}}
\ControlFlowTok{if}\NormalTok{(}\SpecialCharTok{!}\FunctionTok{require}\NormalTok{(sfarrow)) \{}\FunctionTok{install.packages}\NormalTok{(}\StringTok{"sfarrow"}\NormalTok{); }\FunctionTok{require}\NormalTok{(sfarrow)\}}
\ControlFlowTok{if}\NormalTok{(}\SpecialCharTok{!}\FunctionTok{require}\NormalTok{(readxl)) \{}\FunctionTok{install.packages}\NormalTok{(}\StringTok{"readxl"}\NormalTok{); }\FunctionTok{require}\NormalTok{(readxl)\}}
\ControlFlowTok{if}\NormalTok{(}\SpecialCharTok{!}\FunctionTok{require}\NormalTok{(raster)) \{}\FunctionTok{install.packages}\NormalTok{(}\StringTok{"raster"}\NormalTok{); }\FunctionTok{require}\NormalTok{(raster)\}}
\ControlFlowTok{if}\NormalTok{(}\SpecialCharTok{!}\FunctionTok{require}\NormalTok{(fasterize)) \{}\FunctionTok{install.packages}\NormalTok{(}\StringTok{"fasterize"}\NormalTok{); }\FunctionTok{require}\NormalTok{(fasterize)\}}

\CommentTok{\# templates {-}{-}{-}{-}}
\NormalTok{template100}\OtherTok{=}\FunctionTok{rast}\NormalTok{(}\StringTok{"./Templates/TemplateRasters/LV100m\_10km.tif"}\NormalTok{)}
\NormalTok{template10}\OtherTok{=}\FunctionTok{rast}\NormalTok{(}\StringTok{"./Templates/TemplateRasters/LV10m\_10km.tif"}\NormalTok{)}
\NormalTok{rastrs10}\OtherTok{=}\FunctionTok{raster}\NormalTok{(template10)}

\NormalTok{nulls10}\OtherTok{=}\FunctionTok{rast}\NormalTok{(}\StringTok{"./Templates/TemplateRasters/nulls\_LV10m\_10km.tif"}\NormalTok{)}
\NormalTok{nulls100}\OtherTok{=}\FunctionTok{rast}\NormalTok{(}\StringTok{"./Templates/TemplateRasters/nulls\_LV100m\_10km.tif"}\NormalTok{)}


\CommentTok{\# simple landscape {-}{-}{-}{-}}
\NormalTok{simple\_landscape}\OtherTok{=}\FunctionTok{rast}\NormalTok{(}\StringTok{"RasterGrids\_10m/2024/Ainava\_vienk\_mask.tif"}\NormalTok{)}

\CommentTok{\# mvr {-}{-}{-}{-}}
\NormalTok{mvr}\OtherTok{=}\FunctionTok{st\_read\_parquet}\NormalTok{(}\StringTok{"./Geodata/2024/MVR/nogabali\_2024janv.parquet"}\NormalTok{)}
\NormalTok{mvr}\SpecialCharTok{$}\NormalTok{yes}\OtherTok{=}\DecValTok{1}

\CommentTok{\# clear cut mask {-}{-}{-}{-}}
\NormalTok{izcirtumi}\OtherTok{=}\NormalTok{mvr }\SpecialCharTok{\%\textgreater{}\%} 
 \FunctionTok{filter}\NormalTok{(zkat }\SpecialCharTok{\%in\%} \FunctionTok{c}\NormalTok{(}\StringTok{"12"}\NormalTok{,}\StringTok{"14"}\NormalTok{)) }\SpecialCharTok{\%\textgreater{}\%} 
\NormalTok{ dplyr}\SpecialCharTok{::}\FunctionTok{select}\NormalTok{(yes)}
\NormalTok{r\_izcirtumi\_mvr}\OtherTok{=}\FunctionTok{fasterize}\NormalTok{(izcirtumi,rastrs10,}\AttributeTok{field=}\StringTok{"yes"}\NormalTok{)}
\NormalTok{t\_izcirtumi\_mvr}\OtherTok{=}\FunctionTok{rast}\NormalTok{(r\_izcirtumi\_mvr)}
\FunctionTok{plot}\NormalTok{(t\_izcirtumi\_mvr)}

\NormalTok{tcl}\OtherTok{=}\FunctionTok{rast}\NormalTok{(}\StringTok{"./Geodata/2024/Trees/GFW/TreeCoverLoss\_v1\_12.tif"}\NormalTok{)}
\NormalTok{tcl2}\OtherTok{=}\FunctionTok{ifel}\NormalTok{(tcl}\SpecialCharTok{\textless{}}\DecValTok{20}\NormalTok{,}\DecValTok{0}\NormalTok{,}\DecValTok{1}\NormalTok{)}
\NormalTok{tclX}\OtherTok{=}\FunctionTok{cover}\NormalTok{(tcl2,nulls10)}
\FunctionTok{plot}\NormalTok{(tclX)}

\NormalTok{clearcut\_mask}\OtherTok{=}\FunctionTok{cover}\NormalTok{(t\_izcirtumi\_mvr,tclX,}
          \AttributeTok{filename=}\StringTok{"./RasterGrids\_10m/2024/Mask\_clearcuts.tif"}\NormalTok{,}
          \AttributeTok{overwrite=}\ConstantTok{TRUE}\NormalTok{)}
\FunctionTok{plot}\NormalTok{(clearcut\_mask)}

\FunctionTok{rm}\NormalTok{(izcirtumi)}
\FunctionTok{rm}\NormalTok{(r\_izcirtumi\_mvr)}
\FunctionTok{rm}\NormalTok{(t\_izcirtumi\_mvr)}
\FunctionTok{rm}\NormalTok{(tcl)}
\FunctionTok{rm}\NormalTok{(tcl2)}
\FunctionTok{rm}\NormalTok{(tclX)}

\CommentTok{\# ForestsAge\_Old\_cell.tif   egv\_280 {-}{-}{-}{-}}
\NormalTok{vecas}\OtherTok{=}\NormalTok{mvr }\SpecialCharTok{\%\textgreater{}\%} 
 \FunctionTok{filter}\NormalTok{(zkat}\SpecialCharTok{==}\StringTok{"10"}\NormalTok{) }\SpecialCharTok{\%\textgreater{}\%} 
 \CommentTok{\#filter(h10\textgreater{}=5) \%\textgreater{}\% }
 \FunctionTok{filter}\NormalTok{(vgr }\SpecialCharTok{\%in\%} \FunctionTok{c}\NormalTok{(}\StringTok{"4"}\NormalTok{,}\StringTok{"5"}\NormalTok{)) }\SpecialCharTok{\%\textgreater{}\%} 
\NormalTok{ dplyr}\SpecialCharTok{::}\FunctionTok{select}\NormalTok{(yes)}

\NormalTok{p2i\_rez}\OtherTok{=}\NormalTok{egvtools}\SpecialCharTok{::}\FunctionTok{polygon2input}\NormalTok{(}\AttributeTok{vector\_data =}\NormalTok{ vecas,}
                \AttributeTok{template\_path =} \StringTok{"./Templates/TemplateRasters/LV10m\_10km.tif"}\NormalTok{,}
                \AttributeTok{out\_path =} \StringTok{"./RasterGrids\_10m/2024/"}\NormalTok{,}
                \AttributeTok{file\_name =} \StringTok{"ForestsAge\_Old\_input.tif"}\NormalTok{,}
                \AttributeTok{value\_field =} \StringTok{"yes"}\NormalTok{,}
                \AttributeTok{restrict\_to =}\NormalTok{ clearcut\_mask,}
                \AttributeTok{restrict\_values =} \DecValTok{0}\NormalTok{,}
                \AttributeTok{prepare=}\ConstantTok{FALSE}\NormalTok{,}
                \AttributeTok{background\_raster =} \StringTok{"./Templates/TemplateRasters/nulls\_LV10m\_10km.tif"}\NormalTok{,}
                \AttributeTok{plot\_result =} \ConstantTok{TRUE}\NormalTok{)}
\NormalTok{p2i\_rez}
\NormalTok{i2e\_rez}\OtherTok{=}\NormalTok{egvtools}\SpecialCharTok{::}\FunctionTok{input2egv}\NormalTok{(}\AttributeTok{input=}\FunctionTok{paste0}\NormalTok{(}\StringTok{"./RasterGrids\_10m/2024/"}\NormalTok{,}
                     \StringTok{"ForestsAge\_Old\_input.tif"}\NormalTok{),}
              \AttributeTok{egv\_template=} \StringTok{"./Templates/TemplateRasters/LV100m\_10km.tif"}\NormalTok{,}
              \AttributeTok{summary\_function =} \StringTok{"average"}\NormalTok{,}
              \AttributeTok{missing\_job =} \StringTok{"FillOutput"}\NormalTok{,}
              \AttributeTok{outlocation =} \StringTok{"./RasterGrids\_100m/2024/RAW/"}\NormalTok{,}
              \AttributeTok{outfilename =} \StringTok{"ForestsAge\_Old\_cell.tif"}\NormalTok{,}
              \AttributeTok{layername =} \StringTok{"egv\_280"}\NormalTok{,}
              \AttributeTok{idw\_weight =} \DecValTok{2}\NormalTok{,}
              \AttributeTok{plot\_gaps =} \ConstantTok{FALSE}\NormalTok{,}\AttributeTok{plot\_final =} \ConstantTok{TRUE}\NormalTok{)}
\NormalTok{i2e\_rez}
\FunctionTok{rm}\NormalTok{(vecas)}
\FunctionTok{rm}\NormalTok{(p2i\_rez)}
\FunctionTok{rm}\NormalTok{(i2e\_rez)}
\FunctionTok{unlink}\NormalTok{(}\StringTok{"./RasterGrids\_10m/2024/ForestsAge\_Old\_input.tif"}\NormalTok{)}

\CommentTok{\# standardisation {-}{-}{-}{-}}
\ControlFlowTok{if}\NormalTok{(}\SpecialCharTok{!}\FunctionTok{require}\NormalTok{(terra)) \{}\FunctionTok{install.packages}\NormalTok{(}\StringTok{"terra"}\NormalTok{); }\FunctionTok{require}\NormalTok{(terra)\}}
\ControlFlowTok{if}\NormalTok{(}\SpecialCharTok{!}\FunctionTok{require}\NormalTok{(tidyverse)) \{}\FunctionTok{install.packages}\NormalTok{(}\StringTok{"tidyverse"}\NormalTok{); }\FunctionTok{require}\NormalTok{(tidyverse)\}}

\NormalTok{nosaukums}\OtherTok{=}\StringTok{"ForestsAge\_Old\_cell.tif"}
\NormalTok{ielasisanas\_cels}\OtherTok{=}\FunctionTok{paste0}\NormalTok{(}\StringTok{"./RasterGrids\_100m/2024/RAW/"}\NormalTok{,nosaukums)}
\NormalTok{saglabasanas\_cels}\OtherTok{=}\FunctionTok{paste0}\NormalTok{(}\StringTok{"./RasterGrids\_100m/2024/Scaled/"}\NormalTok{,nosaukums)}
\NormalTok{slanis}\OtherTok{=}\FunctionTok{rast}\NormalTok{(ielasisanas\_cels)}
\NormalTok{videjais}\OtherTok{=}\FunctionTok{global}\NormalTok{(slanis,}\AttributeTok{fun=}\StringTok{"mean"}\NormalTok{,}\AttributeTok{na.rm=}\ConstantTok{TRUE}\NormalTok{)}
\NormalTok{centrets}\OtherTok{=}\NormalTok{slanis}\SpecialCharTok{{-}}\NormalTok{videjais[,}\DecValTok{1}\NormalTok{]}
\NormalTok{standartnovirze}\OtherTok{=}\NormalTok{terra}\SpecialCharTok{::}\FunctionTok{global}\NormalTok{(centrets,}\AttributeTok{fun=}\StringTok{"rms"}\NormalTok{,}\AttributeTok{na.rm=}\ConstantTok{TRUE}\NormalTok{)}
\NormalTok{merogots}\OtherTok{=}\NormalTok{centrets}\SpecialCharTok{/}\NormalTok{standartnovirze[,}\DecValTok{1}\NormalTok{]}
\FunctionTok{writeRaster}\NormalTok{(merogots,}
      \AttributeTok{filename=}\NormalTok{saglabasanas\_cels,}
      \AttributeTok{overwrite=}\ConstantTok{TRUE}\NormalTok{)}
\end{Highlighting}
\end{Shaded}

\section{ForestsAge\_Old\_r500}\label{ch06.281}

\textbf{filename:} \texttt{ForestsAge\_Old\_r500.tif}

\textbf{layername:} \texttt{egv\_281}

\textbf{English name:} Fractional cover of Old (over rotation age) Forest Stands within the
0.5 km landscape

\textbf{Latvian name:} Vecu (kopš cirtmeta) mežaudžu platības īpatsvars 0,5 km ainavā

\textbf{Procedure:} The cover fraction within a radius of 500 m around the analysis grid cell is
calculated as the area-weighted sum of the \hyperref[ch06.280]{analysis cells} inside the
buffer, using the workflow \texttt{egvtools::radius\_function()}. During the calculation of the landscape metric,
inverse distance weighted (power = 2) gap filling on the output is applied
to ensure no missing values at the edges. Then the layer is rewritten to set
its name. Finally, the layer is standardised by subtracting the arithmetic
mean and dividing by the root mean squared error.

\begin{Shaded}
\begin{Highlighting}[]
\CommentTok{\# libs {-}{-}{-}{-}}
\ControlFlowTok{if}\NormalTok{(}\SpecialCharTok{!}\FunctionTok{require}\NormalTok{(terra)) \{}\FunctionTok{install.packages}\NormalTok{(}\StringTok{"terra"}\NormalTok{); }\FunctionTok{require}\NormalTok{(terra)\}}
\ControlFlowTok{if}\NormalTok{(}\SpecialCharTok{!}\FunctionTok{require}\NormalTok{(egvtools)) \{remotes}\SpecialCharTok{::}\FunctionTok{install\_github}\NormalTok{(}\StringTok{"aavotins/egvtools"}\NormalTok{); }\FunctionTok{require}\NormalTok{(egvtools)\}}


\CommentTok{\# Templates {-}{-}{-}{-}{-}}
\NormalTok{template100}\OtherTok{=}\FunctionTok{rast}\NormalTok{(}\StringTok{"./Templates/TemplateRasters/LV100m\_10km.tif"}\NormalTok{)}

\CommentTok{\# radii {-}{-}{-}{-}}
\FunctionTok{radius\_function}\NormalTok{(}
 \AttributeTok{kvadrati\_path =} \StringTok{"./Templates/TemplateGrids/tiles/"}\NormalTok{,}
 \AttributeTok{radii\_path   =} \StringTok{"./Templates/TemplateGridPoints/tiles/"}\NormalTok{,}
 \AttributeTok{tikls100\_path =} \StringTok{"./Templates/TemplateGrids/tikls100\_sauzeme.parquet"}\NormalTok{,}
 \AttributeTok{template\_path =} \StringTok{"./Templates/TemplateRasters/LV100m\_10km.tif"}\NormalTok{,}
 \AttributeTok{input\_layers  =} \FunctionTok{c}\NormalTok{(}\StringTok{"./RasterGrids\_100m/2024/RAW/ForestsAge\_Old\_cell.tif"}\NormalTok{),}
 \AttributeTok{layer\_prefixes =} \FunctionTok{c}\NormalTok{(}\StringTok{"ForestsAge\_Old"}\NormalTok{),}
 \AttributeTok{output\_dir   =} \StringTok{"./RasterGrids\_100m/2024/RAW/"}\NormalTok{,}
 \AttributeTok{n\_workers   =} \DecValTok{6}\NormalTok{,}
 \AttributeTok{radii     =} \FunctionTok{c}\NormalTok{(}\StringTok{"r500"}\NormalTok{),}
 \AttributeTok{radius\_mode  =} \StringTok{"sparse"}\NormalTok{,}
 \AttributeTok{extract\_fun  =} \StringTok{"mean"}\NormalTok{,}
 \AttributeTok{fill\_missing  =} \ConstantTok{TRUE}\NormalTok{,}
 \AttributeTok{IDW\_weight   =} \DecValTok{2}\NormalTok{,}
 \AttributeTok{future\_max\_size =} \DecValTok{40} \SpecialCharTok{*} \DecValTok{1024}\SpecialCharTok{\^{}}\DecValTok{3}\NormalTok{)}


\CommentTok{\# ForestsAge\_Old\_r500.tif   egv\_281}
\NormalTok{slanis}\OtherTok{=}\FunctionTok{rast}\NormalTok{(}\StringTok{"./RasterGrids\_100m/2024/RAW/ForestsAge\_Old\_r500.tif"}\NormalTok{)}
\FunctionTok{names}\NormalTok{(slanis)}\OtherTok{=}\StringTok{"egv\_281"}
\NormalTok{slanis2}\OtherTok{=}\FunctionTok{project}\NormalTok{(slanis,template100)}
\FunctionTok{writeRaster}\NormalTok{(slanis2,}
      \StringTok{"./RasterGrids\_100m/2024/RAW/ForestsAge\_Old\_r500.tif"}\NormalTok{,}
      \AttributeTok{overwrite=}\ConstantTok{TRUE}\NormalTok{)}

\CommentTok{\# standardisation {-}{-}{-}{-}}
\ControlFlowTok{if}\NormalTok{(}\SpecialCharTok{!}\FunctionTok{require}\NormalTok{(terra)) \{}\FunctionTok{install.packages}\NormalTok{(}\StringTok{"terra"}\NormalTok{); }\FunctionTok{require}\NormalTok{(terra)\}}
\ControlFlowTok{if}\NormalTok{(}\SpecialCharTok{!}\FunctionTok{require}\NormalTok{(tidyverse)) \{}\FunctionTok{install.packages}\NormalTok{(}\StringTok{"tidyverse"}\NormalTok{); }\FunctionTok{require}\NormalTok{(tidyverse)\}}

\NormalTok{nosaukums}\OtherTok{=}\StringTok{"ForestsAge\_Old\_r500.tif"}
\NormalTok{ielasisanas\_cels}\OtherTok{=}\FunctionTok{paste0}\NormalTok{(}\StringTok{"./RasterGrids\_100m/2024/RAW/"}\NormalTok{,nosaukums)}
\NormalTok{saglabasanas\_cels}\OtherTok{=}\FunctionTok{paste0}\NormalTok{(}\StringTok{"./RasterGrids\_100m/2024/Scaled/"}\NormalTok{,nosaukums)}
\NormalTok{slanis}\OtherTok{=}\FunctionTok{rast}\NormalTok{(ielasisanas\_cels)}
\NormalTok{videjais}\OtherTok{=}\FunctionTok{global}\NormalTok{(slanis,}\AttributeTok{fun=}\StringTok{"mean"}\NormalTok{,}\AttributeTok{na.rm=}\ConstantTok{TRUE}\NormalTok{)}
\NormalTok{centrets}\OtherTok{=}\NormalTok{slanis}\SpecialCharTok{{-}}\NormalTok{videjais[,}\DecValTok{1}\NormalTok{]}
\NormalTok{standartnovirze}\OtherTok{=}\NormalTok{terra}\SpecialCharTok{::}\FunctionTok{global}\NormalTok{(centrets,}\AttributeTok{fun=}\StringTok{"rms"}\NormalTok{,}\AttributeTok{na.rm=}\ConstantTok{TRUE}\NormalTok{)}
\NormalTok{merogots}\OtherTok{=}\NormalTok{centrets}\SpecialCharTok{/}\NormalTok{standartnovirze[,}\DecValTok{1}\NormalTok{]}
\FunctionTok{writeRaster}\NormalTok{(merogots,}
      \AttributeTok{filename=}\NormalTok{saglabasanas\_cels,}
      \AttributeTok{overwrite=}\ConstantTok{TRUE}\NormalTok{)}
\end{Highlighting}
\end{Shaded}

\section{ForestsAge\_Old\_r1250}\label{ch06.282}

\textbf{filename:} \texttt{ForestsAge\_Old\_r1250.tif}

\textbf{layername:} \texttt{egv\_282}

\textbf{English name:} Fractional cover of Old (over rotation age) Forest Stands within the
1.25 km landscape

\textbf{Latvian name:} Vecu (kopš cirtmeta) mežaudžu platības īpatsvars 1,25 km ainavā

\textbf{Procedure:} The cover fraction within a radius of 1250 m around the analysis grid cell
is calculated as the area-weighted sum of the \hyperref[ch06.280]{analysis cells} inside
the buffer, using the workflow \texttt{egvtools::radius\_function()}. During the calculation of the landscape
metric, inverse distance weighted (power = 2) gap filling on the output is
applied to ensure no missing values at the edges. Then the layer is
rewritten to set its name. Finally, the layer is standardised by
subtracting the arithmetic mean and dividing by the root mean squared error.

\begin{Shaded}
\begin{Highlighting}[]
\CommentTok{\# libs {-}{-}{-}{-}}
\ControlFlowTok{if}\NormalTok{(}\SpecialCharTok{!}\FunctionTok{require}\NormalTok{(terra)) \{}\FunctionTok{install.packages}\NormalTok{(}\StringTok{"terra"}\NormalTok{); }\FunctionTok{require}\NormalTok{(terra)\}}
\ControlFlowTok{if}\NormalTok{(}\SpecialCharTok{!}\FunctionTok{require}\NormalTok{(egvtools)) \{remotes}\SpecialCharTok{::}\FunctionTok{install\_github}\NormalTok{(}\StringTok{"aavotins/egvtools"}\NormalTok{); }\FunctionTok{require}\NormalTok{(egvtools)\}}


\CommentTok{\# Templates {-}{-}{-}{-}{-}}
\NormalTok{template100}\OtherTok{=}\FunctionTok{rast}\NormalTok{(}\StringTok{"./Templates/TemplateRasters/LV100m\_10km.tif"}\NormalTok{)}

\CommentTok{\# radii {-}{-}{-}{-}}
\FunctionTok{radius\_function}\NormalTok{(}
 \AttributeTok{kvadrati\_path =} \StringTok{"./Templates/TemplateGrids/tiles/"}\NormalTok{,}
 \AttributeTok{radii\_path   =} \StringTok{"./Templates/TemplateGridPoints/tiles/"}\NormalTok{,}
 \AttributeTok{tikls100\_path =} \StringTok{"./Templates/TemplateGrids/tikls100\_sauzeme.parquet"}\NormalTok{,}
 \AttributeTok{template\_path =} \StringTok{"./Templates/TemplateRasters/LV100m\_10km.tif"}\NormalTok{,}
 \AttributeTok{input\_layers  =} \FunctionTok{c}\NormalTok{(}\StringTok{"./RasterGrids\_100m/2024/RAW/ForestsAge\_Old\_cell.tif"}\NormalTok{),}
 \AttributeTok{layer\_prefixes =} \FunctionTok{c}\NormalTok{(}\StringTok{"ForestsAge\_Old"}\NormalTok{),}
 \AttributeTok{output\_dir   =} \StringTok{"./RasterGrids\_100m/2024/RAW/"}\NormalTok{,}
 \AttributeTok{n\_workers   =} \DecValTok{6}\NormalTok{,}
 \AttributeTok{radii     =} \FunctionTok{c}\NormalTok{(}\StringTok{"r1250"}\NormalTok{),}
 \AttributeTok{radius\_mode  =} \StringTok{"sparse"}\NormalTok{,}
 \AttributeTok{extract\_fun  =} \StringTok{"mean"}\NormalTok{,}
 \AttributeTok{fill\_missing  =} \ConstantTok{TRUE}\NormalTok{,}
 \AttributeTok{IDW\_weight   =} \DecValTok{2}\NormalTok{,}
 \AttributeTok{future\_max\_size =} \DecValTok{40} \SpecialCharTok{*} \DecValTok{1024}\SpecialCharTok{\^{}}\DecValTok{3}\NormalTok{)}


\CommentTok{\# ForestsAge\_Old\_r1250.tif  egv\_282}
\NormalTok{slanis}\OtherTok{=}\FunctionTok{rast}\NormalTok{(}\StringTok{"./RasterGrids\_100m/2024/RAW/ForestsAge\_Old\_r1250.tif"}\NormalTok{)}
\FunctionTok{names}\NormalTok{(slanis)}\OtherTok{=}\StringTok{"egv\_282"}
\NormalTok{slanis2}\OtherTok{=}\FunctionTok{project}\NormalTok{(slanis,template100)}
\FunctionTok{writeRaster}\NormalTok{(slanis2,}
      \StringTok{"./RasterGrids\_100m/2024/RAW/ForestsAge\_Old\_r1250.tif"}\NormalTok{,}
      \AttributeTok{overwrite=}\ConstantTok{TRUE}\NormalTok{)}

\CommentTok{\# standardisation {-}{-}{-}{-}}
\ControlFlowTok{if}\NormalTok{(}\SpecialCharTok{!}\FunctionTok{require}\NormalTok{(terra)) \{}\FunctionTok{install.packages}\NormalTok{(}\StringTok{"terra"}\NormalTok{); }\FunctionTok{require}\NormalTok{(terra)\}}
\ControlFlowTok{if}\NormalTok{(}\SpecialCharTok{!}\FunctionTok{require}\NormalTok{(tidyverse)) \{}\FunctionTok{install.packages}\NormalTok{(}\StringTok{"tidyverse"}\NormalTok{); }\FunctionTok{require}\NormalTok{(tidyverse)\}}

\NormalTok{nosaukums}\OtherTok{=}\StringTok{"ForestsAge\_Old\_r1250.tif"}
\NormalTok{ielasisanas\_cels}\OtherTok{=}\FunctionTok{paste0}\NormalTok{(}\StringTok{"./RasterGrids\_100m/2024/RAW/"}\NormalTok{,nosaukums)}
\NormalTok{saglabasanas\_cels}\OtherTok{=}\FunctionTok{paste0}\NormalTok{(}\StringTok{"./RasterGrids\_100m/2024/Scaled/"}\NormalTok{,nosaukums)}
\NormalTok{slanis}\OtherTok{=}\FunctionTok{rast}\NormalTok{(ielasisanas\_cels)}
\NormalTok{videjais}\OtherTok{=}\FunctionTok{global}\NormalTok{(slanis,}\AttributeTok{fun=}\StringTok{"mean"}\NormalTok{,}\AttributeTok{na.rm=}\ConstantTok{TRUE}\NormalTok{)}
\NormalTok{centrets}\OtherTok{=}\NormalTok{slanis}\SpecialCharTok{{-}}\NormalTok{videjais[,}\DecValTok{1}\NormalTok{]}
\NormalTok{standartnovirze}\OtherTok{=}\NormalTok{terra}\SpecialCharTok{::}\FunctionTok{global}\NormalTok{(centrets,}\AttributeTok{fun=}\StringTok{"rms"}\NormalTok{,}\AttributeTok{na.rm=}\ConstantTok{TRUE}\NormalTok{)}
\NormalTok{merogots}\OtherTok{=}\NormalTok{centrets}\SpecialCharTok{/}\NormalTok{standartnovirze[,}\DecValTok{1}\NormalTok{]}
\FunctionTok{writeRaster}\NormalTok{(merogots,}
      \AttributeTok{filename=}\NormalTok{saglabasanas\_cels,}
      \AttributeTok{overwrite=}\ConstantTok{TRUE}\NormalTok{)}
\end{Highlighting}
\end{Shaded}

\section{ForestsAge\_Old\_r3000}\label{ch06.283}

\textbf{filename:} \texttt{ForestsAge\_Old\_r3000.tif}

\textbf{layername:} \texttt{egv\_283}

\textbf{English name:} Fractional cover of Old (over rotation age) Forest Stands within the
3 km landscape

\textbf{Latvian name:} Vecu (kopš cirtmeta) mežaudžu platības īpatsvars 3 km ainavā

\textbf{Procedure:} The cover fraction within a radius of 3000 m around the analysis grid cell
is calculated as the area-weighted sum of the \hyperref[ch06.280]{analysis cells} inside
the buffer, using the workflow \texttt{egvtools::radius\_function()}. During the calculation of the landscape
metric, inverse distance weighted (power = 2) gap filling on the output is
applied to ensure no missing values at the edges. Then the layer is
rewritten to set its name. Finally, the layer is standardised by
subtracting the arithmetic mean and dividing by the root mean squared error.

\begin{Shaded}
\begin{Highlighting}[]
\CommentTok{\# libs {-}{-}{-}{-}}
\ControlFlowTok{if}\NormalTok{(}\SpecialCharTok{!}\FunctionTok{require}\NormalTok{(terra)) \{}\FunctionTok{install.packages}\NormalTok{(}\StringTok{"terra"}\NormalTok{); }\FunctionTok{require}\NormalTok{(terra)\}}
\ControlFlowTok{if}\NormalTok{(}\SpecialCharTok{!}\FunctionTok{require}\NormalTok{(egvtools)) \{remotes}\SpecialCharTok{::}\FunctionTok{install\_github}\NormalTok{(}\StringTok{"aavotins/egvtools"}\NormalTok{); }\FunctionTok{require}\NormalTok{(egvtools)\}}


\CommentTok{\# Templates {-}{-}{-}{-}{-}}
\NormalTok{template100}\OtherTok{=}\FunctionTok{rast}\NormalTok{(}\StringTok{"./Templates/TemplateRasters/LV100m\_10km.tif"}\NormalTok{)}

\CommentTok{\# radii {-}{-}{-}{-}}
\FunctionTok{radius\_function}\NormalTok{(}
 \AttributeTok{kvadrati\_path =} \StringTok{"./Templates/TemplateGrids/tiles/"}\NormalTok{,}
 \AttributeTok{radii\_path   =} \StringTok{"./Templates/TemplateGridPoints/tiles/"}\NormalTok{,}
 \AttributeTok{tikls100\_path =} \StringTok{"./Templates/TemplateGrids/tikls100\_sauzeme.parquet"}\NormalTok{,}
 \AttributeTok{template\_path =} \StringTok{"./Templates/TemplateRasters/LV100m\_10km.tif"}\NormalTok{,}
 \AttributeTok{input\_layers  =} \FunctionTok{c}\NormalTok{(}\StringTok{"./RasterGrids\_100m/2024/RAW/ForestsAge\_Old\_cell.tif"}\NormalTok{),}
 \AttributeTok{layer\_prefixes =} \FunctionTok{c}\NormalTok{(}\StringTok{"ForestsAge\_Old"}\NormalTok{),}
 \AttributeTok{output\_dir   =} \StringTok{"./RasterGrids\_100m/2024/RAW/"}\NormalTok{,}
 \AttributeTok{n\_workers   =} \DecValTok{6}\NormalTok{,}
 \AttributeTok{radii     =} \FunctionTok{c}\NormalTok{(}\StringTok{"r3000"}\NormalTok{),}
 \AttributeTok{radius\_mode  =} \StringTok{"sparse"}\NormalTok{,}
 \AttributeTok{extract\_fun  =} \StringTok{"mean"}\NormalTok{,}
 \AttributeTok{fill\_missing  =} \ConstantTok{TRUE}\NormalTok{,}
 \AttributeTok{IDW\_weight   =} \DecValTok{2}\NormalTok{,}
 \AttributeTok{future\_max\_size =} \DecValTok{40} \SpecialCharTok{*} \DecValTok{1024}\SpecialCharTok{\^{}}\DecValTok{3}\NormalTok{)}


\CommentTok{\# ForestsAge\_Old\_r3000.tif  egv\_283}
\NormalTok{slanis}\OtherTok{=}\FunctionTok{rast}\NormalTok{(}\StringTok{"./RasterGrids\_100m/2024/RAW/ForestsAge\_Old\_r3000.tif"}\NormalTok{)}
\FunctionTok{names}\NormalTok{(slanis)}\OtherTok{=}\StringTok{"egv\_283"}
\NormalTok{slanis2}\OtherTok{=}\FunctionTok{project}\NormalTok{(slanis,template100)}
\FunctionTok{writeRaster}\NormalTok{(slanis2,}
      \StringTok{"./RasterGrids\_100m/2024/RAW/ForestsAge\_Old\_r3000.tif"}\NormalTok{,}
      \AttributeTok{overwrite=}\ConstantTok{TRUE}\NormalTok{)}

\CommentTok{\# standardisation {-}{-}{-}{-}}
\ControlFlowTok{if}\NormalTok{(}\SpecialCharTok{!}\FunctionTok{require}\NormalTok{(terra)) \{}\FunctionTok{install.packages}\NormalTok{(}\StringTok{"terra"}\NormalTok{); }\FunctionTok{require}\NormalTok{(terra)\}}
\ControlFlowTok{if}\NormalTok{(}\SpecialCharTok{!}\FunctionTok{require}\NormalTok{(tidyverse)) \{}\FunctionTok{install.packages}\NormalTok{(}\StringTok{"tidyverse"}\NormalTok{); }\FunctionTok{require}\NormalTok{(tidyverse)\}}

\NormalTok{nosaukums}\OtherTok{=}\StringTok{"ForestsAge\_Old\_r3000.tif"}
\NormalTok{ielasisanas\_cels}\OtherTok{=}\FunctionTok{paste0}\NormalTok{(}\StringTok{"./RasterGrids\_100m/2024/RAW/"}\NormalTok{,nosaukums)}
\NormalTok{saglabasanas\_cels}\OtherTok{=}\FunctionTok{paste0}\NormalTok{(}\StringTok{"./RasterGrids\_100m/2024/Scaled/"}\NormalTok{,nosaukums)}
\NormalTok{slanis}\OtherTok{=}\FunctionTok{rast}\NormalTok{(ielasisanas\_cels)}
\NormalTok{videjais}\OtherTok{=}\FunctionTok{global}\NormalTok{(slanis,}\AttributeTok{fun=}\StringTok{"mean"}\NormalTok{,}\AttributeTok{na.rm=}\ConstantTok{TRUE}\NormalTok{)}
\NormalTok{centrets}\OtherTok{=}\NormalTok{slanis}\SpecialCharTok{{-}}\NormalTok{videjais[,}\DecValTok{1}\NormalTok{]}
\NormalTok{standartnovirze}\OtherTok{=}\NormalTok{terra}\SpecialCharTok{::}\FunctionTok{global}\NormalTok{(centrets,}\AttributeTok{fun=}\StringTok{"rms"}\NormalTok{,}\AttributeTok{na.rm=}\ConstantTok{TRUE}\NormalTok{)}
\NormalTok{merogots}\OtherTok{=}\NormalTok{centrets}\SpecialCharTok{/}\NormalTok{standartnovirze[,}\DecValTok{1}\NormalTok{]}
\FunctionTok{writeRaster}\NormalTok{(merogots,}
      \AttributeTok{filename=}\NormalTok{saglabasanas\_cels,}
      \AttributeTok{overwrite=}\ConstantTok{TRUE}\NormalTok{)}
\end{Highlighting}
\end{Shaded}

\section{ForestsAge\_Old\_r10000}\label{ch06.284}

\textbf{filename:} \texttt{ForestsAge\_Old\_r10000.tif}

\textbf{layername:} \texttt{egv\_284}

\textbf{English name:} Fractional cover of Old (over rotation age) Forest Stands within the
10 km landscape

\textbf{Latvian name:} Vecu (kopš cirtmeta) mežaudžu platības īpatsvars 10 km ainavā

\textbf{Procedure:} The cover fraction within a radius of 10000 m around the analysis grid cell
is calculated as the area-weighted sum of the \hyperref[ch06.280]{analysis cells} inside
the buffer, using the workflow \texttt{egvtools::radius\_function()}. During the calculation of the landscape
metric, inverse distance weighted (power = 2) gap filling on the output is
applied to ensure no missing values at the edges. Then the layer is
rewritten to set its name. Finally, the layer is standardised by
subtracting the arithmetic mean and dividing by the root mean squared error.

\begin{Shaded}
\begin{Highlighting}[]
\CommentTok{\# libs {-}{-}{-}{-}}
\ControlFlowTok{if}\NormalTok{(}\SpecialCharTok{!}\FunctionTok{require}\NormalTok{(terra)) \{}\FunctionTok{install.packages}\NormalTok{(}\StringTok{"terra"}\NormalTok{); }\FunctionTok{require}\NormalTok{(terra)\}}
\ControlFlowTok{if}\NormalTok{(}\SpecialCharTok{!}\FunctionTok{require}\NormalTok{(egvtools)) \{remotes}\SpecialCharTok{::}\FunctionTok{install\_github}\NormalTok{(}\StringTok{"aavotins/egvtools"}\NormalTok{); }\FunctionTok{require}\NormalTok{(egvtools)\}}


\CommentTok{\# Templates {-}{-}{-}{-}{-}}
\NormalTok{template100}\OtherTok{=}\FunctionTok{rast}\NormalTok{(}\StringTok{"./Templates/TemplateRasters/LV100m\_10km.tif"}\NormalTok{)}

\CommentTok{\# radii {-}{-}{-}{-}}
\FunctionTok{radius\_function}\NormalTok{(}
 \AttributeTok{kvadrati\_path =} \StringTok{"./Templates/TemplateGrids/tiles/"}\NormalTok{,}
 \AttributeTok{radii\_path   =} \StringTok{"./Templates/TemplateGridPoints/tiles/"}\NormalTok{,}
 \AttributeTok{tikls100\_path =} \StringTok{"./Templates/TemplateGrids/tikls100\_sauzeme.parquet"}\NormalTok{,}
 \AttributeTok{template\_path =} \StringTok{"./Templates/TemplateRasters/LV100m\_10km.tif"}\NormalTok{,}
 \AttributeTok{input\_layers  =} \FunctionTok{c}\NormalTok{(}\StringTok{"./RasterGrids\_100m/2024/RAW/ForestsAge\_Old\_cell.tif"}\NormalTok{),}
 \AttributeTok{layer\_prefixes =} \FunctionTok{c}\NormalTok{(}\StringTok{"ForestsAge\_Old"}\NormalTok{),}
 \AttributeTok{output\_dir   =} \StringTok{"./RasterGrids\_100m/2024/RAW/"}\NormalTok{,}
 \AttributeTok{n\_workers   =} \DecValTok{6}\NormalTok{,}
 \AttributeTok{radii     =} \FunctionTok{c}\NormalTok{(}\StringTok{"r10000"}\NormalTok{),}
 \AttributeTok{radius\_mode  =} \StringTok{"sparse"}\NormalTok{,}
 \AttributeTok{extract\_fun  =} \StringTok{"mean"}\NormalTok{,}
 \AttributeTok{fill\_missing  =} \ConstantTok{TRUE}\NormalTok{,}
 \AttributeTok{IDW\_weight   =} \DecValTok{2}\NormalTok{,}
 \AttributeTok{future\_max\_size =} \DecValTok{40} \SpecialCharTok{*} \DecValTok{1024}\SpecialCharTok{\^{}}\DecValTok{3}\NormalTok{)}


\CommentTok{\# ForestsAge\_Old\_r10000.tif egv\_284}
\NormalTok{slanis}\OtherTok{=}\FunctionTok{rast}\NormalTok{(}\StringTok{"./RasterGrids\_100m/2024/RAW/ForestsAge\_Old\_r10000.tif"}\NormalTok{)}
\FunctionTok{names}\NormalTok{(slanis)}\OtherTok{=}\StringTok{"egv\_284"}
\NormalTok{slanis2}\OtherTok{=}\FunctionTok{project}\NormalTok{(slanis,template100)}
\FunctionTok{writeRaster}\NormalTok{(slanis2,}
      \StringTok{"./RasterGrids\_100m/2024/RAW/ForestsAge\_Old\_r10000.tif"}\NormalTok{,}
      \AttributeTok{overwrite=}\ConstantTok{TRUE}\NormalTok{)}

\CommentTok{\# standardisation {-}{-}{-}{-}}
\ControlFlowTok{if}\NormalTok{(}\SpecialCharTok{!}\FunctionTok{require}\NormalTok{(terra)) \{}\FunctionTok{install.packages}\NormalTok{(}\StringTok{"terra"}\NormalTok{); }\FunctionTok{require}\NormalTok{(terra)\}}
\ControlFlowTok{if}\NormalTok{(}\SpecialCharTok{!}\FunctionTok{require}\NormalTok{(tidyverse)) \{}\FunctionTok{install.packages}\NormalTok{(}\StringTok{"tidyverse"}\NormalTok{); }\FunctionTok{require}\NormalTok{(tidyverse)\}}

\NormalTok{nosaukums}\OtherTok{=}\StringTok{"ForestsAge\_Old\_r10000.tif"}
\NormalTok{ielasisanas\_cels}\OtherTok{=}\FunctionTok{paste0}\NormalTok{(}\StringTok{"./RasterGrids\_100m/2024/RAW/"}\NormalTok{,nosaukums)}
\NormalTok{saglabasanas\_cels}\OtherTok{=}\FunctionTok{paste0}\NormalTok{(}\StringTok{"./RasterGrids\_100m/2024/Scaled/"}\NormalTok{,nosaukums)}
\NormalTok{slanis}\OtherTok{=}\FunctionTok{rast}\NormalTok{(ielasisanas\_cels)}
\NormalTok{videjais}\OtherTok{=}\FunctionTok{global}\NormalTok{(slanis,}\AttributeTok{fun=}\StringTok{"mean"}\NormalTok{,}\AttributeTok{na.rm=}\ConstantTok{TRUE}\NormalTok{)}
\NormalTok{centrets}\OtherTok{=}\NormalTok{slanis}\SpecialCharTok{{-}}\NormalTok{videjais[,}\DecValTok{1}\NormalTok{]}
\NormalTok{standartnovirze}\OtherTok{=}\NormalTok{terra}\SpecialCharTok{::}\FunctionTok{global}\NormalTok{(centrets,}\AttributeTok{fun=}\StringTok{"rms"}\NormalTok{,}\AttributeTok{na.rm=}\ConstantTok{TRUE}\NormalTok{)}
\NormalTok{merogots}\OtherTok{=}\NormalTok{centrets}\SpecialCharTok{/}\NormalTok{standartnovirze[,}\DecValTok{1}\NormalTok{]}
\FunctionTok{writeRaster}\NormalTok{(merogots,}
      \AttributeTok{filename=}\NormalTok{saglabasanas\_cels,}
      \AttributeTok{overwrite=}\ConstantTok{TRUE}\NormalTok{)}
\end{Highlighting}
\end{Shaded}

\section{ForestsAge\_YoungTallStandsShrubs\_cell}\label{ch06.285}

\textbf{filename:} \texttt{ForestsAge\_YoungTallStandsShrubs\_cell.tif}

\textbf{layername:} \texttt{egv\_285}

\textbf{English name:} Fractional cover of Shrubs, Young Forest Stands (at least 5 m tall)
within the analysis cell (1 ha)

\textbf{Latvian name:} Krūmāju un jaunaudžu (no 5 m augstuma) platības īpatsvars
analīzes šūnā (1 ha)

\textbf{Procedure:} Most EGVs describing forests are spatially restricted to areas outside
of clearcuts and dead stands. This mask is created using a combination of
the \hyperref[Ch04.01]{State Forest Service's
State Forest Registry} land category 12 and 14, and \hyperref[Ch04.09]{The
Global Forest Watch} pixels classified as lost tree canopy cover since
2020 (raster layer matching input, presence = 1, absence = 0).

To prepare this
EGV, stands in land category 10 and age group 1 with height above 5 m are
selected from the \hyperref[Ch04.01]{State Forest Service's State Forest Registry} and
rasterised (presence = 1, NA otherwise). This layer is then combined with the category
620 from the \hyperref[Ch05.03]{Landscape classification} (presence = 1, 0 otherwise).
Values in pixels matching the clearcut mask are set to 0. The resulting layer
is then aggregated to EGV resolution using the workflow \texttt{egvtools::input2egv()}, which
calculates the arithmetic mean to determine the cover fraction. During
aggregation, inverse distance weighted (power = 2) gap filling on the output is
applied to ensure no missing values at the edges. Finally, the layer is
standardised by subtracting the arithmetic mean and dividing by the root mean squared
error.

\begin{Shaded}
\begin{Highlighting}[]
\CommentTok{\# libs {-}{-}{-}{-}}
\ControlFlowTok{if}\NormalTok{(}\SpecialCharTok{!}\FunctionTok{require}\NormalTok{(egvtools)) \{remotes}\SpecialCharTok{::}\FunctionTok{install\_github}\NormalTok{(}\StringTok{"aavotins/egvtools"}\NormalTok{); }\FunctionTok{require}\NormalTok{(egvtools)\}}
\ControlFlowTok{if}\NormalTok{(}\SpecialCharTok{!}\FunctionTok{require}\NormalTok{(terra)) \{}\FunctionTok{install.packages}\NormalTok{(}\StringTok{"terra"}\NormalTok{); }\FunctionTok{require}\NormalTok{(terra)\}}
\ControlFlowTok{if}\NormalTok{(}\SpecialCharTok{!}\FunctionTok{require}\NormalTok{(sf)) \{}\FunctionTok{install.packages}\NormalTok{(}\StringTok{"sf"}\NormalTok{); }\FunctionTok{require}\NormalTok{(sf)\}}
\ControlFlowTok{if}\NormalTok{(}\SpecialCharTok{!}\FunctionTok{require}\NormalTok{(tidyverse)) \{}\FunctionTok{install.packages}\NormalTok{(}\StringTok{"tidyverse"}\NormalTok{); }\FunctionTok{require}\NormalTok{(tidyverse)\}}
\ControlFlowTok{if}\NormalTok{(}\SpecialCharTok{!}\FunctionTok{require}\NormalTok{(sfarrow)) \{}\FunctionTok{install.packages}\NormalTok{(}\StringTok{"sfarrow"}\NormalTok{); }\FunctionTok{require}\NormalTok{(sfarrow)\}}
\ControlFlowTok{if}\NormalTok{(}\SpecialCharTok{!}\FunctionTok{require}\NormalTok{(readxl)) \{}\FunctionTok{install.packages}\NormalTok{(}\StringTok{"readxl"}\NormalTok{); }\FunctionTok{require}\NormalTok{(readxl)\}}
\ControlFlowTok{if}\NormalTok{(}\SpecialCharTok{!}\FunctionTok{require}\NormalTok{(raster)) \{}\FunctionTok{install.packages}\NormalTok{(}\StringTok{"raster"}\NormalTok{); }\FunctionTok{require}\NormalTok{(raster)\}}
\ControlFlowTok{if}\NormalTok{(}\SpecialCharTok{!}\FunctionTok{require}\NormalTok{(fasterize)) \{}\FunctionTok{install.packages}\NormalTok{(}\StringTok{"fasterize"}\NormalTok{); }\FunctionTok{require}\NormalTok{(fasterize)\}}

\CommentTok{\# templates {-}{-}{-}{-}}
\NormalTok{template100}\OtherTok{=}\FunctionTok{rast}\NormalTok{(}\StringTok{"./Templates/TemplateRasters/LV100m\_10km.tif"}\NormalTok{)}
\NormalTok{template10}\OtherTok{=}\FunctionTok{rast}\NormalTok{(}\StringTok{"./Templates/TemplateRasters/LV10m\_10km.tif"}\NormalTok{)}
\NormalTok{rastrs10}\OtherTok{=}\FunctionTok{raster}\NormalTok{(template10)}

\NormalTok{nulls10}\OtherTok{=}\FunctionTok{rast}\NormalTok{(}\StringTok{"./Templates/TemplateRasters/nulls\_LV10m\_10km.tif"}\NormalTok{)}
\NormalTok{nulls100}\OtherTok{=}\FunctionTok{rast}\NormalTok{(}\StringTok{"./Templates/TemplateRasters/nulls\_LV100m\_10km.tif"}\NormalTok{)}


\CommentTok{\# simple landscape {-}{-}{-}{-}}
\NormalTok{simple\_landscape}\OtherTok{=}\FunctionTok{rast}\NormalTok{(}\StringTok{"RasterGrids\_10m/2024/Ainava\_vienk\_mask.tif"}\NormalTok{)}

\CommentTok{\# mvr {-}{-}{-}{-}}
\NormalTok{mvr}\OtherTok{=}\FunctionTok{st\_read\_parquet}\NormalTok{(}\StringTok{"./Geodata/2024/MVR/nogabali\_2024janv.parquet"}\NormalTok{)}
\NormalTok{mvr}\SpecialCharTok{$}\NormalTok{yes}\OtherTok{=}\DecValTok{1}

\CommentTok{\# clear cut mask {-}{-}{-}{-}}
\NormalTok{izcirtumi}\OtherTok{=}\NormalTok{mvr }\SpecialCharTok{\%\textgreater{}\%} 
 \FunctionTok{filter}\NormalTok{(zkat }\SpecialCharTok{\%in\%} \FunctionTok{c}\NormalTok{(}\StringTok{"12"}\NormalTok{,}\StringTok{"14"}\NormalTok{)) }\SpecialCharTok{\%\textgreater{}\%} 
\NormalTok{ dplyr}\SpecialCharTok{::}\FunctionTok{select}\NormalTok{(yes)}
\NormalTok{r\_izcirtumi\_mvr}\OtherTok{=}\FunctionTok{fasterize}\NormalTok{(izcirtumi,rastrs10,}\AttributeTok{field=}\StringTok{"yes"}\NormalTok{)}
\NormalTok{t\_izcirtumi\_mvr}\OtherTok{=}\FunctionTok{rast}\NormalTok{(r\_izcirtumi\_mvr)}
\FunctionTok{plot}\NormalTok{(t\_izcirtumi\_mvr)}

\NormalTok{tcl}\OtherTok{=}\FunctionTok{rast}\NormalTok{(}\StringTok{"./Geodata/2024/Trees/GFW/TreeCoverLoss\_v1\_12.tif"}\NormalTok{)}
\NormalTok{tcl2}\OtherTok{=}\FunctionTok{ifel}\NormalTok{(tcl}\SpecialCharTok{\textless{}}\DecValTok{20}\NormalTok{,}\DecValTok{0}\NormalTok{,}\DecValTok{1}\NormalTok{)}
\NormalTok{tclX}\OtherTok{=}\FunctionTok{cover}\NormalTok{(tcl2,nulls10)}
\FunctionTok{plot}\NormalTok{(tclX)}

\NormalTok{clearcut\_mask}\OtherTok{=}\FunctionTok{cover}\NormalTok{(t\_izcirtumi\_mvr,tclX,}
          \AttributeTok{filename=}\StringTok{"./RasterGrids\_10m/2024/Mask\_clearcuts.tif"}\NormalTok{,}
          \AttributeTok{overwrite=}\ConstantTok{TRUE}\NormalTok{)}
\FunctionTok{plot}\NormalTok{(clearcut\_mask)}

\FunctionTok{rm}\NormalTok{(izcirtumi)}
\FunctionTok{rm}\NormalTok{(r\_izcirtumi\_mvr)}
\FunctionTok{rm}\NormalTok{(t\_izcirtumi\_mvr)}
\FunctionTok{rm}\NormalTok{(tcl)}
\FunctionTok{rm}\NormalTok{(tcl2)}
\FunctionTok{rm}\NormalTok{(tclX)}

\CommentTok{\# ForestsAge\_YoungTallStandsShrubs\_cell.tif egv\_285 {-}{-}{-}{-}}
\NormalTok{jaunasaugstas}\OtherTok{=}\NormalTok{mvr }\SpecialCharTok{\%\textgreater{}\%} 
 \FunctionTok{filter}\NormalTok{(zkat}\SpecialCharTok{==}\StringTok{"10"}\NormalTok{) }\SpecialCharTok{\%\textgreater{}\%} 
 \FunctionTok{filter}\NormalTok{(h10}\SpecialCharTok{\textgreater{}=}\DecValTok{5}\NormalTok{) }\SpecialCharTok{\%\textgreater{}\%} 
 \FunctionTok{filter}\NormalTok{(vgr }\SpecialCharTok{\%in\%} \FunctionTok{c}\NormalTok{(}\StringTok{"1"}\NormalTok{)) }\SpecialCharTok{\%\textgreater{}\%} 
\NormalTok{ dplyr}\SpecialCharTok{::}\FunctionTok{select}\NormalTok{(yes)}
\NormalTok{r\_jaunasaugstas}\OtherTok{=}\FunctionTok{fasterize}\NormalTok{(jaunasaugstas,rastrs10,}\AttributeTok{field=}\StringTok{"yes"}\NormalTok{)}
\NormalTok{t\_jaunasaugstas}\OtherTok{=}\FunctionTok{rast}\NormalTok{(r\_jaunasaugstas)}
\FunctionTok{plot}\NormalTok{(t\_jaunasaugstas)}

\NormalTok{shrubs}\OtherTok{=}\FunctionTok{ifel}\NormalTok{(simple\_landscape}\SpecialCharTok{==}\DecValTok{620}\NormalTok{,}\DecValTok{1}\NormalTok{,}\DecValTok{0}\NormalTok{)}

\NormalTok{younshrubs}\OtherTok{=}\FunctionTok{cover}\NormalTok{(t\_jaunasaugstas,shrubs)}
\FunctionTok{plot}\NormalTok{(younshrubs)}

\NormalTok{younshrubs2}\OtherTok{=}\FunctionTok{ifel}\NormalTok{(younshrubs}\SpecialCharTok{==}\DecValTok{1}\SpecialCharTok{\&}\NormalTok{clearcut\_mask}\SpecialCharTok{==}\DecValTok{0}\NormalTok{,}\DecValTok{1}\NormalTok{,}\DecValTok{0}\NormalTok{)}
\FunctionTok{plot}\NormalTok{(younshrubs2)}

\NormalTok{i2e\_rez}\OtherTok{=}\NormalTok{egvtools}\SpecialCharTok{::}\FunctionTok{input2egv}\NormalTok{(}\AttributeTok{input=}\NormalTok{younshrubs2,}
              \AttributeTok{egv\_template=} \StringTok{"./Templates/TemplateRasters/LV100m\_10km.tif"}\NormalTok{,}
              \AttributeTok{summary\_function =} \StringTok{"average"}\NormalTok{,}
              \AttributeTok{missing\_job =} \StringTok{"FillOutput"}\NormalTok{,}
              \AttributeTok{outlocation =} \StringTok{"./RasterGrids\_100m/2024/RAW/"}\NormalTok{,}
              \AttributeTok{outfilename =} \StringTok{"ForestsAge\_YoungTallStandsShrubs\_cell.tif"}\NormalTok{,}
              \AttributeTok{layername =} \StringTok{"egv\_285"}\NormalTok{,}
              \AttributeTok{idw\_weight =} \DecValTok{2}\NormalTok{,}
              \AttributeTok{plot\_gaps =} \ConstantTok{FALSE}\NormalTok{,}\AttributeTok{plot\_final =} \ConstantTok{TRUE}\NormalTok{)}
\NormalTok{i2e\_rez}
\FunctionTok{rm}\NormalTok{(jaunasaugstas)}
\FunctionTok{rm}\NormalTok{(r\_jaunasaugstas)}
\FunctionTok{rm}\NormalTok{(t\_jaunasaugstas)}
\FunctionTok{rm}\NormalTok{(shrubs)}
\FunctionTok{rm}\NormalTok{(younshrubs)}
\FunctionTok{rm}\NormalTok{(younshrubs2)}
\FunctionTok{rm}\NormalTok{(i2e\_rez)}

\CommentTok{\# standardisation {-}{-}{-}{-}}
\ControlFlowTok{if}\NormalTok{(}\SpecialCharTok{!}\FunctionTok{require}\NormalTok{(terra)) \{}\FunctionTok{install.packages}\NormalTok{(}\StringTok{"terra"}\NormalTok{); }\FunctionTok{require}\NormalTok{(terra)\}}
\ControlFlowTok{if}\NormalTok{(}\SpecialCharTok{!}\FunctionTok{require}\NormalTok{(tidyverse)) \{}\FunctionTok{install.packages}\NormalTok{(}\StringTok{"tidyverse"}\NormalTok{); }\FunctionTok{require}\NormalTok{(tidyverse)\}}

\NormalTok{nosaukums}\OtherTok{=}\StringTok{"ForestsAge\_YoungTallStandsShrubs\_cell.tif"}
\NormalTok{ielasisanas\_cels}\OtherTok{=}\FunctionTok{paste0}\NormalTok{(}\StringTok{"./RasterGrids\_100m/2024/RAW/"}\NormalTok{,nosaukums)}
\NormalTok{saglabasanas\_cels}\OtherTok{=}\FunctionTok{paste0}\NormalTok{(}\StringTok{"./RasterGrids\_100m/2024/Scaled/"}\NormalTok{,nosaukums)}
\NormalTok{slanis}\OtherTok{=}\FunctionTok{rast}\NormalTok{(ielasisanas\_cels)}
\NormalTok{videjais}\OtherTok{=}\FunctionTok{global}\NormalTok{(slanis,}\AttributeTok{fun=}\StringTok{"mean"}\NormalTok{,}\AttributeTok{na.rm=}\ConstantTok{TRUE}\NormalTok{)}
\NormalTok{centrets}\OtherTok{=}\NormalTok{slanis}\SpecialCharTok{{-}}\NormalTok{videjais[,}\DecValTok{1}\NormalTok{]}
\NormalTok{standartnovirze}\OtherTok{=}\NormalTok{terra}\SpecialCharTok{::}\FunctionTok{global}\NormalTok{(centrets,}\AttributeTok{fun=}\StringTok{"rms"}\NormalTok{,}\AttributeTok{na.rm=}\ConstantTok{TRUE}\NormalTok{)}
\NormalTok{merogots}\OtherTok{=}\NormalTok{centrets}\SpecialCharTok{/}\NormalTok{standartnovirze[,}\DecValTok{1}\NormalTok{]}
\FunctionTok{writeRaster}\NormalTok{(merogots,}
      \AttributeTok{filename=}\NormalTok{saglabasanas\_cels,}
      \AttributeTok{overwrite=}\ConstantTok{TRUE}\NormalTok{)}
\end{Highlighting}
\end{Shaded}

\section{ForestsAge\_YoungTallStandsShrubs\_r500}\label{ch06.286}

\textbf{filename:} \texttt{ForestsAge\_YoungTallStandsShrubs\_r500.tif}

\textbf{layername:} \texttt{egv\_286}

\textbf{English name:} Fractional cover of Shrubs, Young Forest Stands (at least 5 m tall)
within the 0.5 km landscape

\textbf{Latvian name:} Krūmāju un jaunaudžu (no 5 m augstuma) platības īpatsvars 0,5
km ainavā

\textbf{Procedure:} The cover fraction within a radius of 500 m around the analysis grid cell is
calculated as the area-weighted sum of the \hyperref[ch06.285]{analysis cells} inside the
buffer, using the workflow \texttt{egvtools::radius\_function()}. During the calculation of the landscape metric,
inverse distance weighted (power = 2) gap filling on the output is applied
to ensure no missing values at the edges. Then the layer is rewritten to set
its name. Finally, the layer is standardised by subtracting the arithmetic
mean and dividing by the root mean squared error.

\begin{Shaded}
\begin{Highlighting}[]
\CommentTok{\# libs {-}{-}{-}{-}}
\ControlFlowTok{if}\NormalTok{(}\SpecialCharTok{!}\FunctionTok{require}\NormalTok{(terra)) \{}\FunctionTok{install.packages}\NormalTok{(}\StringTok{"terra"}\NormalTok{); }\FunctionTok{require}\NormalTok{(terra)\}}
\ControlFlowTok{if}\NormalTok{(}\SpecialCharTok{!}\FunctionTok{require}\NormalTok{(egvtools)) \{remotes}\SpecialCharTok{::}\FunctionTok{install\_github}\NormalTok{(}\StringTok{"aavotins/egvtools"}\NormalTok{); }\FunctionTok{require}\NormalTok{(egvtools)\}}


\CommentTok{\# Templates {-}{-}{-}{-}{-}}
\NormalTok{template100}\OtherTok{=}\FunctionTok{rast}\NormalTok{(}\StringTok{"./Templates/TemplateRasters/LV100m\_10km.tif"}\NormalTok{)}

\CommentTok{\# radii {-}{-}{-}{-}}
\FunctionTok{radius\_function}\NormalTok{(}
 \AttributeTok{kvadrati\_path =} \StringTok{"./Templates/TemplateGrids/tiles/"}\NormalTok{,}
 \AttributeTok{radii\_path   =} \StringTok{"./Templates/TemplateGridPoints/tiles/"}\NormalTok{,}
 \AttributeTok{tikls100\_path =} \StringTok{"./Templates/TemplateGrids/tikls100\_sauzeme.parquet"}\NormalTok{,}
 \AttributeTok{template\_path =} \StringTok{"./Templates/TemplateRasters/LV100m\_10km.tif"}\NormalTok{,}
 \AttributeTok{input\_layers  =} \FunctionTok{c}\NormalTok{(}\StringTok{"./RasterGrids\_100m/2024/RAW/ForestsAge\_YoungTallStandsShrubs\_cell.tif"}\NormalTok{),}
 \AttributeTok{layer\_prefixes =} \FunctionTok{c}\NormalTok{(}\StringTok{"ForestsAge\_YoungTallStandsShrubs"}\NormalTok{),}
 \AttributeTok{output\_dir   =} \StringTok{"./RasterGrids\_100m/2024/RAW/"}\NormalTok{,}
 \AttributeTok{n\_workers   =} \DecValTok{6}\NormalTok{,}
 \AttributeTok{radii     =} \FunctionTok{c}\NormalTok{(}\StringTok{"r500"}\NormalTok{),}
 \AttributeTok{radius\_mode  =} \StringTok{"sparse"}\NormalTok{,}
 \AttributeTok{extract\_fun  =} \StringTok{"mean"}\NormalTok{,}
 \AttributeTok{fill\_missing  =} \ConstantTok{TRUE}\NormalTok{,}
 \AttributeTok{IDW\_weight   =} \DecValTok{2}\NormalTok{,}
 \AttributeTok{future\_max\_size =} \DecValTok{40} \SpecialCharTok{*} \DecValTok{1024}\SpecialCharTok{\^{}}\DecValTok{3}\NormalTok{)}


\CommentTok{\# ForestsAge\_YoungTallStandsShrubs\_r500.tif egv\_286}
\NormalTok{slanis}\OtherTok{=}\FunctionTok{rast}\NormalTok{(}\StringTok{"./RasterGrids\_100m/2024/RAW/ForestsAge\_YoungTallStandsShrubs\_r500.tif"}\NormalTok{)}
\FunctionTok{names}\NormalTok{(slanis)}\OtherTok{=}\StringTok{"egv\_286"}
\NormalTok{slanis2}\OtherTok{=}\FunctionTok{project}\NormalTok{(slanis,template100)}
\FunctionTok{writeRaster}\NormalTok{(slanis2,}
      \StringTok{"./RasterGrids\_100m/2024/RAW/ForestsAge\_YoungTallStandsShrubs\_r500.tif"}\NormalTok{,}
      \AttributeTok{overwrite=}\ConstantTok{TRUE}\NormalTok{)}

\CommentTok{\# standardisation {-}{-}{-}{-}}
\ControlFlowTok{if}\NormalTok{(}\SpecialCharTok{!}\FunctionTok{require}\NormalTok{(terra)) \{}\FunctionTok{install.packages}\NormalTok{(}\StringTok{"terra"}\NormalTok{); }\FunctionTok{require}\NormalTok{(terra)\}}
\ControlFlowTok{if}\NormalTok{(}\SpecialCharTok{!}\FunctionTok{require}\NormalTok{(tidyverse)) \{}\FunctionTok{install.packages}\NormalTok{(}\StringTok{"tidyverse"}\NormalTok{); }\FunctionTok{require}\NormalTok{(tidyverse)\}}

\NormalTok{nosaukums}\OtherTok{=}\StringTok{"ForestsAge\_YoungTallStandsShrubs\_r500.tif"}
\NormalTok{ielasisanas\_cels}\OtherTok{=}\FunctionTok{paste0}\NormalTok{(}\StringTok{"./RasterGrids\_100m/2024/RAW/"}\NormalTok{,nosaukums)}
\NormalTok{saglabasanas\_cels}\OtherTok{=}\FunctionTok{paste0}\NormalTok{(}\StringTok{"./RasterGrids\_100m/2024/Scaled/"}\NormalTok{,nosaukums)}
\NormalTok{slanis}\OtherTok{=}\FunctionTok{rast}\NormalTok{(ielasisanas\_cels)}
\NormalTok{videjais}\OtherTok{=}\FunctionTok{global}\NormalTok{(slanis,}\AttributeTok{fun=}\StringTok{"mean"}\NormalTok{,}\AttributeTok{na.rm=}\ConstantTok{TRUE}\NormalTok{)}
\NormalTok{centrets}\OtherTok{=}\NormalTok{slanis}\SpecialCharTok{{-}}\NormalTok{videjais[,}\DecValTok{1}\NormalTok{]}
\NormalTok{standartnovirze}\OtherTok{=}\NormalTok{terra}\SpecialCharTok{::}\FunctionTok{global}\NormalTok{(centrets,}\AttributeTok{fun=}\StringTok{"rms"}\NormalTok{,}\AttributeTok{na.rm=}\ConstantTok{TRUE}\NormalTok{)}
\NormalTok{merogots}\OtherTok{=}\NormalTok{centrets}\SpecialCharTok{/}\NormalTok{standartnovirze[,}\DecValTok{1}\NormalTok{]}
\FunctionTok{writeRaster}\NormalTok{(merogots,}
      \AttributeTok{filename=}\NormalTok{saglabasanas\_cels,}
      \AttributeTok{overwrite=}\ConstantTok{TRUE}\NormalTok{)}
\end{Highlighting}
\end{Shaded}

\section{ForestsAge\_YoungTallStandsShrubs\_r1250}\label{ch06.287}

\textbf{filename:} \texttt{ForestsAge\_YoungTallStandsShrubs\_r1250.tif}

\textbf{layername:} \texttt{egv\_287}

\textbf{English name:} Fractional cover of Shrubs, Young Forest Stands (at least 5 m tall)
within the 1.25 km landscape

\textbf{Latvian name:} Krūmāju un jaunaudžu (no 5 m augstuma) platības īpatsvars 1,25
km ainavā

\textbf{Procedure:} The cover fraction within a radius of 1250 m around the analysis grid cell
is calculated as the area-weighted sum of the \hyperref[ch06.285]{analysis cells} inside
the buffer, using the workflow \texttt{egvtools::radius\_function()}. During the calculation of the landscape
metric, inverse distance weighted (power = 2) gap filling on the output is
applied to ensure no missing values at the edges. Then the layer is
rewritten to set its name. Finally, the layer is standardised by
subtracting the arithmetic mean and dividing by the root mean squared error.

\begin{Shaded}
\begin{Highlighting}[]
\CommentTok{\# libs {-}{-}{-}{-}}
\ControlFlowTok{if}\NormalTok{(}\SpecialCharTok{!}\FunctionTok{require}\NormalTok{(terra)) \{}\FunctionTok{install.packages}\NormalTok{(}\StringTok{"terra"}\NormalTok{); }\FunctionTok{require}\NormalTok{(terra)\}}
\ControlFlowTok{if}\NormalTok{(}\SpecialCharTok{!}\FunctionTok{require}\NormalTok{(egvtools)) \{remotes}\SpecialCharTok{::}\FunctionTok{install\_github}\NormalTok{(}\StringTok{"aavotins/egvtools"}\NormalTok{); }\FunctionTok{require}\NormalTok{(egvtools)\}}


\CommentTok{\# Templates {-}{-}{-}{-}{-}}
\NormalTok{template100}\OtherTok{=}\FunctionTok{rast}\NormalTok{(}\StringTok{"./Templates/TemplateRasters/LV100m\_10km.tif"}\NormalTok{)}

\CommentTok{\# radii {-}{-}{-}{-}}
\FunctionTok{radius\_function}\NormalTok{(}
 \AttributeTok{kvadrati\_path =} \StringTok{"./Templates/TemplateGrids/tiles/"}\NormalTok{,}
 \AttributeTok{radii\_path   =} \StringTok{"./Templates/TemplateGridPoints/tiles/"}\NormalTok{,}
 \AttributeTok{tikls100\_path =} \StringTok{"./Templates/TemplateGrids/tikls100\_sauzeme.parquet"}\NormalTok{,}
 \AttributeTok{template\_path =} \StringTok{"./Templates/TemplateRasters/LV100m\_10km.tif"}\NormalTok{,}
 \AttributeTok{input\_layers  =} \FunctionTok{c}\NormalTok{(}\StringTok{"./RasterGrids\_100m/2024/RAW/ForestsAge\_YoungTallStandsShrubs\_cell.tif"}\NormalTok{),}
 \AttributeTok{layer\_prefixes =} \FunctionTok{c}\NormalTok{(}\StringTok{"ForestsAge\_YoungTallStandsShrubs"}\NormalTok{),}
 \AttributeTok{output\_dir   =} \StringTok{"./RasterGrids\_100m/2024/RAW/"}\NormalTok{,}
 \AttributeTok{n\_workers   =} \DecValTok{6}\NormalTok{,}
 \AttributeTok{radii     =} \FunctionTok{c}\NormalTok{(}\StringTok{"r1250"}\NormalTok{),}
 \AttributeTok{radius\_mode  =} \StringTok{"sparse"}\NormalTok{,}
 \AttributeTok{extract\_fun  =} \StringTok{"mean"}\NormalTok{,}
 \AttributeTok{fill\_missing  =} \ConstantTok{TRUE}\NormalTok{,}
 \AttributeTok{IDW\_weight   =} \DecValTok{2}\NormalTok{,}
 \AttributeTok{future\_max\_size =} \DecValTok{40} \SpecialCharTok{*} \DecValTok{1024}\SpecialCharTok{\^{}}\DecValTok{3}\NormalTok{)}


\CommentTok{\# ForestsAge\_YoungTallStandsShrubs\_r1250.tif    egv\_287}
\NormalTok{slanis}\OtherTok{=}\FunctionTok{rast}\NormalTok{(}\StringTok{"./RasterGrids\_100m/2024/RAW/ForestsAge\_YoungTallStandsShrubs\_r1250.tif"}\NormalTok{)}
\FunctionTok{names}\NormalTok{(slanis)}\OtherTok{=}\StringTok{"egv\_287"}
\NormalTok{slanis2}\OtherTok{=}\FunctionTok{project}\NormalTok{(slanis,template100)}
\FunctionTok{writeRaster}\NormalTok{(slanis2,}
      \StringTok{"./RasterGrids\_100m/2024/RAW/ForestsAge\_YoungTallStandsShrubs\_r1250.tif"}\NormalTok{,}
      \AttributeTok{overwrite=}\ConstantTok{TRUE}\NormalTok{)}

\CommentTok{\# standardisation {-}{-}{-}{-}}
\ControlFlowTok{if}\NormalTok{(}\SpecialCharTok{!}\FunctionTok{require}\NormalTok{(terra)) \{}\FunctionTok{install.packages}\NormalTok{(}\StringTok{"terra"}\NormalTok{); }\FunctionTok{require}\NormalTok{(terra)\}}
\ControlFlowTok{if}\NormalTok{(}\SpecialCharTok{!}\FunctionTok{require}\NormalTok{(tidyverse)) \{}\FunctionTok{install.packages}\NormalTok{(}\StringTok{"tidyverse"}\NormalTok{); }\FunctionTok{require}\NormalTok{(tidyverse)\}}

\NormalTok{nosaukums}\OtherTok{=}\StringTok{"ForestsAge\_YoungTallStandsShrubs\_r1250.tif"}
\NormalTok{ielasisanas\_cels}\OtherTok{=}\FunctionTok{paste0}\NormalTok{(}\StringTok{"./RasterGrids\_100m/2024/RAW/"}\NormalTok{,nosaukums)}
\NormalTok{saglabasanas\_cels}\OtherTok{=}\FunctionTok{paste0}\NormalTok{(}\StringTok{"./RasterGrids\_100m/2024/Scaled/"}\NormalTok{,nosaukums)}
\NormalTok{slanis}\OtherTok{=}\FunctionTok{rast}\NormalTok{(ielasisanas\_cels)}
\NormalTok{videjais}\OtherTok{=}\FunctionTok{global}\NormalTok{(slanis,}\AttributeTok{fun=}\StringTok{"mean"}\NormalTok{,}\AttributeTok{na.rm=}\ConstantTok{TRUE}\NormalTok{)}
\NormalTok{centrets}\OtherTok{=}\NormalTok{slanis}\SpecialCharTok{{-}}\NormalTok{videjais[,}\DecValTok{1}\NormalTok{]}
\NormalTok{standartnovirze}\OtherTok{=}\NormalTok{terra}\SpecialCharTok{::}\FunctionTok{global}\NormalTok{(centrets,}\AttributeTok{fun=}\StringTok{"rms"}\NormalTok{,}\AttributeTok{na.rm=}\ConstantTok{TRUE}\NormalTok{)}
\NormalTok{merogots}\OtherTok{=}\NormalTok{centrets}\SpecialCharTok{/}\NormalTok{standartnovirze[,}\DecValTok{1}\NormalTok{]}
\FunctionTok{writeRaster}\NormalTok{(merogots,}
      \AttributeTok{filename=}\NormalTok{saglabasanas\_cels,}
      \AttributeTok{overwrite=}\ConstantTok{TRUE}\NormalTok{)}
\end{Highlighting}
\end{Shaded}

\section{ForestsAge\_YoungTallStandsShrubs\_r3000}\label{ch06.288}

\textbf{filename:} \texttt{ForestsAge\_YoungTallStandsShrubs\_r3000.tif}

\textbf{layername:} \texttt{egv\_288}

\textbf{English name:} Fractional cover of Shrubs, Young Forest Stands (at least 5 m tall)
within the 3 km landscape

\textbf{Latvian name:} Krūmāju un jaunaudžu (no 5 m augstuma) platības īpatsvars 3 km
ainavā

\textbf{Procedure:} The cover fraction within a radius of 3000 m around the analysis grid cell
is calculated as the area-weighted sum of the \hyperref[ch06.285]{analysis cells} inside
the buffer, using the workflow \texttt{egvtools::radius\_function()}. During the calculation of the landscape
metric, inverse distance weighted (power = 2) gap filling on the output is
applied to ensure no missing values at the edges. Then the layer is
rewritten to set its name. Finally, the layer is standardised by
subtracting the arithmetic mean and dividing by the root mean squared error.

\begin{Shaded}
\begin{Highlighting}[]
\CommentTok{\# libs {-}{-}{-}{-}}
\ControlFlowTok{if}\NormalTok{(}\SpecialCharTok{!}\FunctionTok{require}\NormalTok{(terra)) \{}\FunctionTok{install.packages}\NormalTok{(}\StringTok{"terra"}\NormalTok{); }\FunctionTok{require}\NormalTok{(terra)\}}
\ControlFlowTok{if}\NormalTok{(}\SpecialCharTok{!}\FunctionTok{require}\NormalTok{(egvtools)) \{remotes}\SpecialCharTok{::}\FunctionTok{install\_github}\NormalTok{(}\StringTok{"aavotins/egvtools"}\NormalTok{); }\FunctionTok{require}\NormalTok{(egvtools)\}}


\CommentTok{\# Templates {-}{-}{-}{-}{-}}
\NormalTok{template100}\OtherTok{=}\FunctionTok{rast}\NormalTok{(}\StringTok{"./Templates/TemplateRasters/LV100m\_10km.tif"}\NormalTok{)}

\CommentTok{\# radii {-}{-}{-}{-}}
\FunctionTok{radius\_function}\NormalTok{(}
 \AttributeTok{kvadrati\_path =} \StringTok{"./Templates/TemplateGrids/tiles/"}\NormalTok{,}
 \AttributeTok{radii\_path   =} \StringTok{"./Templates/TemplateGridPoints/tiles/"}\NormalTok{,}
 \AttributeTok{tikls100\_path =} \StringTok{"./Templates/TemplateGrids/tikls100\_sauzeme.parquet"}\NormalTok{,}
 \AttributeTok{template\_path =} \StringTok{"./Templates/TemplateRasters/LV100m\_10km.tif"}\NormalTok{,}
 \AttributeTok{input\_layers  =} \FunctionTok{c}\NormalTok{(}\StringTok{"./RasterGrids\_100m/2024/RAW/ForestsAge\_YoungTallStandsShrubs\_cell.tif"}\NormalTok{),}
 \AttributeTok{layer\_prefixes =} \FunctionTok{c}\NormalTok{(}\StringTok{"ForestsAge\_YoungTallStandsShrubs"}\NormalTok{),}
 \AttributeTok{output\_dir   =} \StringTok{"./RasterGrids\_100m/2024/RAW/"}\NormalTok{,}
 \AttributeTok{n\_workers   =} \DecValTok{6}\NormalTok{,}
 \AttributeTok{radii     =} \FunctionTok{c}\NormalTok{(}\StringTok{"r3000"}\NormalTok{),}
 \AttributeTok{radius\_mode  =} \StringTok{"sparse"}\NormalTok{,}
 \AttributeTok{extract\_fun  =} \StringTok{"mean"}\NormalTok{,}
 \AttributeTok{fill\_missing  =} \ConstantTok{TRUE}\NormalTok{,}
 \AttributeTok{IDW\_weight   =} \DecValTok{2}\NormalTok{,}
 \AttributeTok{future\_max\_size =} \DecValTok{40} \SpecialCharTok{*} \DecValTok{1024}\SpecialCharTok{\^{}}\DecValTok{3}\NormalTok{)}


\CommentTok{\# ForestsAge\_YoungTallStandsShrubs\_r3000.tif    egv\_288}
\NormalTok{slanis}\OtherTok{=}\FunctionTok{rast}\NormalTok{(}\StringTok{"./RasterGrids\_100m/2024/RAW/ForestsAge\_YoungTallStandsShrubs\_r3000.tif"}\NormalTok{)}
\FunctionTok{names}\NormalTok{(slanis)}\OtherTok{=}\StringTok{"egv\_288"}
\NormalTok{slanis2}\OtherTok{=}\FunctionTok{project}\NormalTok{(slanis,template100)}
\FunctionTok{writeRaster}\NormalTok{(slanis2,}
      \StringTok{"./RasterGrids\_100m/2024/RAW/ForestsAge\_YoungTallStandsShrubs\_r3000.tif"}\NormalTok{,}
      \AttributeTok{overwrite=}\ConstantTok{TRUE}\NormalTok{)}

\CommentTok{\# standardisation {-}{-}{-}{-}}
\ControlFlowTok{if}\NormalTok{(}\SpecialCharTok{!}\FunctionTok{require}\NormalTok{(terra)) \{}\FunctionTok{install.packages}\NormalTok{(}\StringTok{"terra"}\NormalTok{); }\FunctionTok{require}\NormalTok{(terra)\}}
\ControlFlowTok{if}\NormalTok{(}\SpecialCharTok{!}\FunctionTok{require}\NormalTok{(tidyverse)) \{}\FunctionTok{install.packages}\NormalTok{(}\StringTok{"tidyverse"}\NormalTok{); }\FunctionTok{require}\NormalTok{(tidyverse)\}}

\NormalTok{nosaukums}\OtherTok{=}\StringTok{"ForestsAge\_YoungTallStandsShrubs\_r3000.tif"}
\NormalTok{ielasisanas\_cels}\OtherTok{=}\FunctionTok{paste0}\NormalTok{(}\StringTok{"./RasterGrids\_100m/2024/RAW/"}\NormalTok{,nosaukums)}
\NormalTok{saglabasanas\_cels}\OtherTok{=}\FunctionTok{paste0}\NormalTok{(}\StringTok{"./RasterGrids\_100m/2024/Scaled/"}\NormalTok{,nosaukums)}
\NormalTok{slanis}\OtherTok{=}\FunctionTok{rast}\NormalTok{(ielasisanas\_cels)}
\NormalTok{videjais}\OtherTok{=}\FunctionTok{global}\NormalTok{(slanis,}\AttributeTok{fun=}\StringTok{"mean"}\NormalTok{,}\AttributeTok{na.rm=}\ConstantTok{TRUE}\NormalTok{)}
\NormalTok{centrets}\OtherTok{=}\NormalTok{slanis}\SpecialCharTok{{-}}\NormalTok{videjais[,}\DecValTok{1}\NormalTok{]}
\NormalTok{standartnovirze}\OtherTok{=}\NormalTok{terra}\SpecialCharTok{::}\FunctionTok{global}\NormalTok{(centrets,}\AttributeTok{fun=}\StringTok{"rms"}\NormalTok{,}\AttributeTok{na.rm=}\ConstantTok{TRUE}\NormalTok{)}
\NormalTok{merogots}\OtherTok{=}\NormalTok{centrets}\SpecialCharTok{/}\NormalTok{standartnovirze[,}\DecValTok{1}\NormalTok{]}
\FunctionTok{writeRaster}\NormalTok{(merogots,}
      \AttributeTok{filename=}\NormalTok{saglabasanas\_cels,}
      \AttributeTok{overwrite=}\ConstantTok{TRUE}\NormalTok{)}
\end{Highlighting}
\end{Shaded}

\section{ForestsAge\_YoungTallStandsShrubs\_r10000}\label{ch06.289}

\textbf{filename:} \texttt{ForestsAge\_YoungTallStandsShrubs\_r10000.tif}

\textbf{layername:} \texttt{egv\_289}

\textbf{English name:} Fractional cover of Shrubs, Young Forest Stands (at least 5 m tall)
within the 10 km landscape

\textbf{Latvian name:} Krūmāju un jaunaudžu (no 5 m augstuma) platības īpatsvars 10
km ainavā

\textbf{Procedure:} The cover fraction within a radius of 10000 m around the analysis grid cell
is calculated as the area-weighted sum of the \hyperref[ch06.285]{analysis cells} inside
the buffer, using the workflow \texttt{egvtools::radius\_function()}. During the calculation of the landscape
metric, inverse distance weighted (power = 2) gap filling on the output is
applied to ensure no missing values at the edges. Then the layer is
rewritten to set its name. Finally, the layer is standardised by
subtracting the arithmetic mean and dividing by the root mean squared error.

\begin{Shaded}
\begin{Highlighting}[]
\CommentTok{\# libs {-}{-}{-}{-}}
\ControlFlowTok{if}\NormalTok{(}\SpecialCharTok{!}\FunctionTok{require}\NormalTok{(terra)) \{}\FunctionTok{install.packages}\NormalTok{(}\StringTok{"terra"}\NormalTok{); }\FunctionTok{require}\NormalTok{(terra)\}}
\ControlFlowTok{if}\NormalTok{(}\SpecialCharTok{!}\FunctionTok{require}\NormalTok{(egvtools)) \{remotes}\SpecialCharTok{::}\FunctionTok{install\_github}\NormalTok{(}\StringTok{"aavotins/egvtools"}\NormalTok{); }\FunctionTok{require}\NormalTok{(egvtools)\}}


\CommentTok{\# Templates {-}{-}{-}{-}{-}}
\NormalTok{template100}\OtherTok{=}\FunctionTok{rast}\NormalTok{(}\StringTok{"./Templates/TemplateRasters/LV100m\_10km.tif"}\NormalTok{)}

\CommentTok{\# radii {-}{-}{-}{-}}
\FunctionTok{radius\_function}\NormalTok{(}
 \AttributeTok{kvadrati\_path =} \StringTok{"./Templates/TemplateGrids/tiles/"}\NormalTok{,}
 \AttributeTok{radii\_path   =} \StringTok{"./Templates/TemplateGridPoints/tiles/"}\NormalTok{,}
 \AttributeTok{tikls100\_path =} \StringTok{"./Templates/TemplateGrids/tikls100\_sauzeme.parquet"}\NormalTok{,}
 \AttributeTok{template\_path =} \StringTok{"./Templates/TemplateRasters/LV100m\_10km.tif"}\NormalTok{,}
 \AttributeTok{input\_layers  =} \FunctionTok{c}\NormalTok{(}\StringTok{"./RasterGrids\_100m/2024/RAW/ForestsAge\_YoungTallStandsShrubs\_cell.tif"}\NormalTok{),}
 \AttributeTok{layer\_prefixes =} \FunctionTok{c}\NormalTok{(}\StringTok{"ForestsAge\_YoungTallStandsShrubs"}\NormalTok{),}
 \AttributeTok{output\_dir   =} \StringTok{"./RasterGrids\_100m/2024/RAW/"}\NormalTok{,}
 \AttributeTok{n\_workers   =} \DecValTok{6}\NormalTok{,}
 \AttributeTok{radii     =} \FunctionTok{c}\NormalTok{(}\StringTok{"r10000"}\NormalTok{),}
 \AttributeTok{radius\_mode  =} \StringTok{"sparse"}\NormalTok{,}
 \AttributeTok{extract\_fun  =} \StringTok{"mean"}\NormalTok{,}
 \AttributeTok{fill\_missing  =} \ConstantTok{TRUE}\NormalTok{,}
 \AttributeTok{IDW\_weight   =} \DecValTok{2}\NormalTok{,}
 \AttributeTok{future\_max\_size =} \DecValTok{40} \SpecialCharTok{*} \DecValTok{1024}\SpecialCharTok{\^{}}\DecValTok{3}\NormalTok{)}


\CommentTok{\# ForestsAge\_YoungTallStandsShrubs\_r10000.tif   egv\_289}
\NormalTok{slanis}\OtherTok{=}\FunctionTok{rast}\NormalTok{(}\StringTok{"./RasterGrids\_100m/2024/RAW/ForestsAge\_YoungTallStandsShrubs\_r10000.tif"}\NormalTok{)}
\FunctionTok{names}\NormalTok{(slanis)}\OtherTok{=}\StringTok{"egv\_289"}
\NormalTok{slanis2}\OtherTok{=}\FunctionTok{project}\NormalTok{(slanis,template100)}
\FunctionTok{writeRaster}\NormalTok{(slanis2,}
      \StringTok{"./RasterGrids\_100m/2024/RAW/ForestsAge\_YoungTallStandsShrubs\_r10000.tif"}\NormalTok{,}
      \AttributeTok{overwrite=}\ConstantTok{TRUE}\NormalTok{)}

\CommentTok{\# standardisation {-}{-}{-}{-}}
\ControlFlowTok{if}\NormalTok{(}\SpecialCharTok{!}\FunctionTok{require}\NormalTok{(terra)) \{}\FunctionTok{install.packages}\NormalTok{(}\StringTok{"terra"}\NormalTok{); }\FunctionTok{require}\NormalTok{(terra)\}}
\ControlFlowTok{if}\NormalTok{(}\SpecialCharTok{!}\FunctionTok{require}\NormalTok{(tidyverse)) \{}\FunctionTok{install.packages}\NormalTok{(}\StringTok{"tidyverse"}\NormalTok{); }\FunctionTok{require}\NormalTok{(tidyverse)\}}

\NormalTok{nosaukums}\OtherTok{=}\StringTok{"ForestsAge\_YoungTallStandsShrubs\_r10000.tif"}
\NormalTok{ielasisanas\_cels}\OtherTok{=}\FunctionTok{paste0}\NormalTok{(}\StringTok{"./RasterGrids\_100m/2024/RAW/"}\NormalTok{,nosaukums)}
\NormalTok{saglabasanas\_cels}\OtherTok{=}\FunctionTok{paste0}\NormalTok{(}\StringTok{"./RasterGrids\_100m/2024/Scaled/"}\NormalTok{,nosaukums)}
\NormalTok{slanis}\OtherTok{=}\FunctionTok{rast}\NormalTok{(ielasisanas\_cels)}
\NormalTok{videjais}\OtherTok{=}\FunctionTok{global}\NormalTok{(slanis,}\AttributeTok{fun=}\StringTok{"mean"}\NormalTok{,}\AttributeTok{na.rm=}\ConstantTok{TRUE}\NormalTok{)}
\NormalTok{centrets}\OtherTok{=}\NormalTok{slanis}\SpecialCharTok{{-}}\NormalTok{videjais[,}\DecValTok{1}\NormalTok{]}
\NormalTok{standartnovirze}\OtherTok{=}\NormalTok{terra}\SpecialCharTok{::}\FunctionTok{global}\NormalTok{(centrets,}\AttributeTok{fun=}\StringTok{"rms"}\NormalTok{,}\AttributeTok{na.rm=}\ConstantTok{TRUE}\NormalTok{)}
\NormalTok{merogots}\OtherTok{=}\NormalTok{centrets}\SpecialCharTok{/}\NormalTok{standartnovirze[,}\DecValTok{1}\NormalTok{]}
\FunctionTok{writeRaster}\NormalTok{(merogots,}
      \AttributeTok{filename=}\NormalTok{saglabasanas\_cels,}
      \AttributeTok{overwrite=}\ConstantTok{TRUE}\NormalTok{)}
\end{Highlighting}
\end{Shaded}

\section{ForestsQuant\_AgeProp-average\_cell}\label{ch06.290}

\textbf{filename:} \texttt{ForestsQuant\_AgeProp-average\_cell.tif}

\textbf{layername:} \texttt{egv\_290}

\textbf{English name:} Average stand age relative to rotation age within the analysis
cell (1 ha)

\textbf{Latvian name:} Mežaudzes vecuma attiecība pret cirtmetu, vidējais analīzes
šūnā (1 ha)

\textbf{Procedure:} Most EGVs describing forests are spatially restricted to areas outside
of clearcuts and dead stands. This mask is created using a combination of
the \hyperref[Ch04.01]{State Forest Service's
State Forest Registry} land category 12 and 14, and \hyperref[Ch04.09]{The
Global Forest Watch} pixels classified as lost tree canopy cover since
2020 (raster layer matching input, presence = 1, absence = 0).

To prepare this EGV, every forest stand had assigned \href{https://likumi.lv/ta/id/2825\#p9}{legal rotation
age}, based on dominant tree species and bonity
class as registered in the \hyperref[Ch04.01]{State Forest Service's State Forest
Registry}. We assumed 35 years as the rotation age for grey alder. The registered age of the
dominant tree group is then divided by the stand specific legal rotation age.
This attribute has some extreme
values. We chose to limit them to the nearest integer showing only minimal
accumulation in histogram.

\includegraphics[width=0.8\linewidth]{./Figures/Histogramms/hist_egv290}

Resulting values at polygon geometries are rasterised with the workflow
\texttt{egvtools::polygon2input()}, restricting to pixels outside the clearcut mask. No
background values are assigned during rasterisation. The resulting layer is
then aggregated to EGV resolution using the workflow \texttt{egvtools::input2egv()} by calculating
arithmetic mean. After the aggregation, cells with no forest information are
filled with value 0. Finally, the layer is standardised by subtracting
the arithmetic mean and dividing by the root mean squared error.

\begin{Shaded}
\begin{Highlighting}[]
\CommentTok{\# libs {-}{-}{-}{-}}
\ControlFlowTok{if}\NormalTok{(}\SpecialCharTok{!}\FunctionTok{require}\NormalTok{(egvtools)) \{remotes}\SpecialCharTok{::}\FunctionTok{install\_github}\NormalTok{(}\StringTok{"aavotins/egvtools"}\NormalTok{); }\FunctionTok{require}\NormalTok{(egvtools)\}}
\ControlFlowTok{if}\NormalTok{(}\SpecialCharTok{!}\FunctionTok{require}\NormalTok{(terra)) \{}\FunctionTok{install.packages}\NormalTok{(}\StringTok{"terra"}\NormalTok{); }\FunctionTok{require}\NormalTok{(terra)\}}
\ControlFlowTok{if}\NormalTok{(}\SpecialCharTok{!}\FunctionTok{require}\NormalTok{(sf)) \{}\FunctionTok{install.packages}\NormalTok{(}\StringTok{"sf"}\NormalTok{); }\FunctionTok{require}\NormalTok{(sf)\}}
\ControlFlowTok{if}\NormalTok{(}\SpecialCharTok{!}\FunctionTok{require}\NormalTok{(tidyverse)) \{}\FunctionTok{install.packages}\NormalTok{(}\StringTok{"tidyverse"}\NormalTok{); }\FunctionTok{require}\NormalTok{(tidyverse)\}}
\ControlFlowTok{if}\NormalTok{(}\SpecialCharTok{!}\FunctionTok{require}\NormalTok{(sfarrow)) \{}\FunctionTok{install.packages}\NormalTok{(}\StringTok{"sfarrow"}\NormalTok{); }\FunctionTok{require}\NormalTok{(sfarrow)\}}
\ControlFlowTok{if}\NormalTok{(}\SpecialCharTok{!}\FunctionTok{require}\NormalTok{(readxl)) \{}\FunctionTok{install.packages}\NormalTok{(}\StringTok{"readxl"}\NormalTok{); }\FunctionTok{require}\NormalTok{(readxl)\}}
\ControlFlowTok{if}\NormalTok{(}\SpecialCharTok{!}\FunctionTok{require}\NormalTok{(raster)) \{}\FunctionTok{install.packages}\NormalTok{(}\StringTok{"raster"}\NormalTok{); }\FunctionTok{require}\NormalTok{(raster)\}}
\ControlFlowTok{if}\NormalTok{(}\SpecialCharTok{!}\FunctionTok{require}\NormalTok{(fasterize)) \{}\FunctionTok{install.packages}\NormalTok{(}\StringTok{"fasterize"}\NormalTok{); }\FunctionTok{require}\NormalTok{(fasterize)\}}

\CommentTok{\# templates {-}{-}{-}{-}}
\NormalTok{template100}\OtherTok{=}\FunctionTok{rast}\NormalTok{(}\StringTok{"./Templates/TemplateRasters/LV100m\_10km.tif"}\NormalTok{)}
\NormalTok{template10}\OtherTok{=}\FunctionTok{rast}\NormalTok{(}\StringTok{"./Templates/TemplateRasters/LV10m\_10km.tif"}\NormalTok{)}
\NormalTok{rastrs10}\OtherTok{=}\FunctionTok{raster}\NormalTok{(template10)}

\NormalTok{nulls10}\OtherTok{=}\FunctionTok{rast}\NormalTok{(}\StringTok{"./Templates/TemplateRasters/nulls\_LV10m\_10km.tif"}\NormalTok{)}
\NormalTok{nulls100}\OtherTok{=}\FunctionTok{rast}\NormalTok{(}\StringTok{"./Templates/TemplateRasters/nulls\_LV100m\_10km.tif"}\NormalTok{)}


\CommentTok{\# simple landscape {-}{-}{-}{-}}
\NormalTok{simple\_landscape}\OtherTok{=}\FunctionTok{rast}\NormalTok{(}\StringTok{"RasterGrids\_10m/2024/Ainava\_vienk\_mask.tif"}\NormalTok{)}

\CommentTok{\# mvr {-}{-}{-}{-}}
\NormalTok{mvr}\OtherTok{=}\FunctionTok{st\_read\_parquet}\NormalTok{(}\StringTok{"./Geodata/2024/MVR/nogabali\_2024janv.parquet"}\NormalTok{)}
\NormalTok{mvr}\SpecialCharTok{$}\NormalTok{yes}\OtherTok{=}\DecValTok{1}

\CommentTok{\# clear cut mask {-}{-}{-}{-}}
\NormalTok{izcirtumi}\OtherTok{=}\NormalTok{mvr }\SpecialCharTok{\%\textgreater{}\%} 
 \FunctionTok{filter}\NormalTok{(zkat }\SpecialCharTok{\%in\%} \FunctionTok{c}\NormalTok{(}\StringTok{"12"}\NormalTok{,}\StringTok{"14"}\NormalTok{)) }\SpecialCharTok{\%\textgreater{}\%} 
\NormalTok{ dplyr}\SpecialCharTok{::}\FunctionTok{select}\NormalTok{(yes)}
\NormalTok{r\_izcirtumi\_mvr}\OtherTok{=}\FunctionTok{fasterize}\NormalTok{(izcirtumi,rastrs10,}\AttributeTok{field=}\StringTok{"yes"}\NormalTok{)}
\NormalTok{t\_izcirtumi\_mvr}\OtherTok{=}\FunctionTok{rast}\NormalTok{(r\_izcirtumi\_mvr)}
\FunctionTok{plot}\NormalTok{(t\_izcirtumi\_mvr)}

\NormalTok{tcl}\OtherTok{=}\FunctionTok{rast}\NormalTok{(}\StringTok{"./Geodata/2024/Trees/GFW/TreeCoverLoss\_v1\_12.tif"}\NormalTok{)}
\NormalTok{tcl2}\OtherTok{=}\FunctionTok{ifel}\NormalTok{(tcl}\SpecialCharTok{\textless{}}\DecValTok{20}\NormalTok{,}\DecValTok{0}\NormalTok{,}\DecValTok{1}\NormalTok{)}
\NormalTok{tclX}\OtherTok{=}\FunctionTok{cover}\NormalTok{(tcl2,nulls10)}
\FunctionTok{plot}\NormalTok{(tclX)}

\NormalTok{clearcut\_mask}\OtherTok{=}\FunctionTok{cover}\NormalTok{(t\_izcirtumi\_mvr,tclX,}
          \AttributeTok{filename=}\StringTok{"./RasterGrids\_10m/2024/Mask\_clearcuts.tif"}\NormalTok{,}
          \AttributeTok{overwrite=}\ConstantTok{TRUE}\NormalTok{)}
\FunctionTok{plot}\NormalTok{(clearcut\_mask)}

\FunctionTok{rm}\NormalTok{(izcirtumi)}
\FunctionTok{rm}\NormalTok{(r\_izcirtumi\_mvr)}
\FunctionTok{rm}\NormalTok{(t\_izcirtumi\_mvr)}
\FunctionTok{rm}\NormalTok{(tcl)}
\FunctionTok{rm}\NormalTok{(tcl2)}
\FunctionTok{rm}\NormalTok{(tclX)}

\CommentTok{\# ForestsQuant\_AgeProp{-}average\_cell.tif egv\_290 {-}{-}{-}{-}}

\CommentTok{\#Forest law https://likumi.lv/ta/id/2825\#p9}
\NormalTok{ozoli}\OtherTok{=}\FunctionTok{c}\NormalTok{(}\StringTok{"10"}\NormalTok{,}\StringTok{"61"}\NormalTok{)}
\NormalTok{priedes\_lapegles}\OtherTok{=}\FunctionTok{c}\NormalTok{(}\StringTok{"1"}\NormalTok{,}\StringTok{"13"}\NormalTok{,}\StringTok{"14"}\NormalTok{,}\StringTok{"22"}\NormalTok{)}
\NormalTok{eolgvk}\OtherTok{=}\FunctionTok{c}\NormalTok{(}\StringTok{"3"}\NormalTok{,}\StringTok{"15"}\NormalTok{,}\StringTok{"23"}\NormalTok{,}\StringTok{"11"}\NormalTok{,}\StringTok{"64"}\NormalTok{,}\StringTok{"12"}\NormalTok{,}\StringTok{"62"}\NormalTok{,}\StringTok{"16"}\NormalTok{,}\StringTok{"65"}\NormalTok{,}\StringTok{"24"}\NormalTok{,}\StringTok{"63"}\NormalTok{)}
\NormalTok{berzi}\OtherTok{=}\FunctionTok{c}\NormalTok{(}\StringTok{"4"}\NormalTok{)}
\NormalTok{melnalksni}\OtherTok{=}\FunctionTok{c}\NormalTok{(}\StringTok{"6"}\NormalTok{)}
\NormalTok{apses}\OtherTok{=}\FunctionTok{c}\NormalTok{(}\StringTok{"8"}\NormalTok{,}\StringTok{"19"}\NormalTok{,}\StringTok{"68"}\NormalTok{)}

\NormalTok{bonA}\OtherTok{=}\FunctionTok{c}\NormalTok{(}\StringTok{"0"}\NormalTok{,}\StringTok{"1"}\NormalTok{)}
\NormalTok{bonB}\OtherTok{=}\FunctionTok{c}\NormalTok{(}\StringTok{"2"}\NormalTok{,}\StringTok{"3"}\NormalTok{)}
\NormalTok{bonC}\OtherTok{=}\FunctionTok{c}\NormalTok{(}\StringTok{"4"}\NormalTok{,}\StringTok{"5"}\NormalTok{,}\StringTok{"6"}\NormalTok{)}
\NormalTok{bonAB}\OtherTok{=}\FunctionTok{c}\NormalTok{(}\StringTok{"0"}\NormalTok{,}\StringTok{"1"}\NormalTok{,}\StringTok{"2"}\NormalTok{,}\StringTok{"3"}\NormalTok{)}

\NormalTok{nogabali}\OtherTok{=}\NormalTok{mvr }\SpecialCharTok{\%\textgreater{}\%} 
 \FunctionTok{mutate}\NormalTok{(}\AttributeTok{cirtmets=}\FunctionTok{ifelse}\NormalTok{((s10 }\SpecialCharTok{\%in\%}\NormalTok{ ozoli)}\SpecialCharTok{\&}\NormalTok{(bon }\SpecialCharTok{\%in\%}\NormalTok{ bonA),}\DecValTok{101}\NormalTok{,}
             \FunctionTok{ifelse}\NormalTok{((s10 }\SpecialCharTok{\%in\%}\NormalTok{ ozoli),}\DecValTok{121}\NormalTok{,}\ConstantTok{NA}\NormalTok{))) }\SpecialCharTok{\%\textgreater{}\%} 
 \FunctionTok{mutate}\NormalTok{(}\AttributeTok{cirtmets=}\FunctionTok{ifelse}\NormalTok{((s10 }\SpecialCharTok{\%in\%}\NormalTok{ priedes\_lapegles)}\SpecialCharTok{\&}\NormalTok{(bon }\SpecialCharTok{\%in\%}\NormalTok{ bonAB),}\DecValTok{101}\NormalTok{,}
             \FunctionTok{ifelse}\NormalTok{((s10 }\SpecialCharTok{\%in\%}\NormalTok{ priedes\_lapegles),}\DecValTok{121}\NormalTok{,cirtmets))) }\SpecialCharTok{\%\textgreater{}\%} 
 \FunctionTok{mutate}\NormalTok{(}\AttributeTok{cirtmets=}\FunctionTok{ifelse}\NormalTok{((s10 }\SpecialCharTok{\%in\%}\NormalTok{ eolgvk),}\DecValTok{81}\NormalTok{,cirtmets)) }\SpecialCharTok{\%\textgreater{}\%} 
 \FunctionTok{mutate}\NormalTok{(}\AttributeTok{cirtmets=}\FunctionTok{ifelse}\NormalTok{((s10 }\SpecialCharTok{\%in\%}\NormalTok{ berzi)}\SpecialCharTok{\&}\NormalTok{(bon }\SpecialCharTok{\%in\%}\NormalTok{ bonAB),}\DecValTok{71}\NormalTok{,}
             \FunctionTok{ifelse}\NormalTok{((s10 }\SpecialCharTok{\%in\%}\NormalTok{ berzi),}\DecValTok{51}\NormalTok{,cirtmets))) }\SpecialCharTok{\%\textgreater{}\%} 
 \FunctionTok{mutate}\NormalTok{(}\AttributeTok{cirtmets=}\FunctionTok{ifelse}\NormalTok{((s10 }\SpecialCharTok{\%in\%}\NormalTok{ melnalksni),}\DecValTok{71}\NormalTok{,cirtmets)) }\SpecialCharTok{\%\textgreater{}\%} 
 \FunctionTok{mutate}\NormalTok{(}\AttributeTok{cirtmets=}\FunctionTok{ifelse}\NormalTok{((s10 }\SpecialCharTok{\%in\%}\NormalTok{ apses),}\DecValTok{41}\NormalTok{,cirtmets))  }\SpecialCharTok{\%\textgreater{}\%} 
 \FunctionTok{mutate}\NormalTok{(}\AttributeTok{cirtmets=}\FunctionTok{ifelse}\NormalTok{(}\FunctionTok{is.na}\NormalTok{(cirtmets)}\SpecialCharTok{\&}\NormalTok{zkat}\SpecialCharTok{==}\StringTok{"10"}\NormalTok{,}\DecValTok{35}\NormalTok{,cirtmets)) }\SpecialCharTok{\%\textgreater{}\%} 
 \FunctionTok{mutate}\NormalTok{(}\AttributeTok{nogvec=}\NormalTok{a10}\SpecialCharTok{/}\NormalTok{cirtmets) }\SpecialCharTok{\%\textgreater{}\%} 
 \FunctionTok{mutate}\NormalTok{(}\AttributeTok{nogvec2=}\FunctionTok{ifelse}\NormalTok{(nogvec}\SpecialCharTok{\textgreater{}}\DecValTok{3}\NormalTok{,}\DecValTok{3}\NormalTok{,nogvec)) }\SpecialCharTok{\%\textgreater{}\%} 
 \FunctionTok{filter}\NormalTok{(}\SpecialCharTok{!}\FunctionTok{is.na}\NormalTok{(nogvec2))}

\FunctionTok{par}\NormalTok{(}\AttributeTok{mfrow=}\FunctionTok{c}\NormalTok{(}\DecValTok{1}\NormalTok{,}\DecValTok{2}\NormalTok{))}
\FunctionTok{options}\NormalTok{(}\AttributeTok{scipen=}\DecValTok{999}\NormalTok{)}
\FunctionTok{hist}\NormalTok{(nogabali}\SpecialCharTok{$}\NormalTok{nogvec,}\AttributeTok{main=}\StringTok{"Original"}\NormalTok{,}\AttributeTok{xlab=}\StringTok{"Relative age"}\NormalTok{)}
\FunctionTok{hist}\NormalTok{(nogabali}\SpecialCharTok{$}\NormalTok{nogvec2,}\AttributeTok{main=}\StringTok{"Limited"}\NormalTok{,}\AttributeTok{xlab=}\StringTok{"Relative age"}\NormalTok{)}
\FunctionTok{par}\NormalTok{(}\AttributeTok{mfrow=}\FunctionTok{c}\NormalTok{(}\DecValTok{1}\NormalTok{,}\DecValTok{1}\NormalTok{))}
\FunctionTok{options}\NormalTok{(}\AttributeTok{scipen=}\DecValTok{0}\NormalTok{)}
\CommentTok{\# 700*400}

\NormalTok{p2i\_rez}\OtherTok{=}\FunctionTok{polygon2input}\NormalTok{(}\AttributeTok{vector\_data=}\NormalTok{nogabali,}
         \AttributeTok{template\_path =} \StringTok{"./Templates/TemplateRasters/LV10m\_10km.tif"}\NormalTok{,}
         \AttributeTok{out\_path =} \StringTok{"./RasterGrids\_10m/2024/"}\NormalTok{,}
         \AttributeTok{file\_name =} \StringTok{"ForestQuant\_AgeProp.tif"}\NormalTok{,}
         \AttributeTok{value\_field =} \StringTok{"nogvec2"}\NormalTok{,}
         \AttributeTok{fun=}\StringTok{"max"}\NormalTok{,}
         \AttributeTok{prepare=}\ConstantTok{FALSE}\NormalTok{,}
         \AttributeTok{restrict\_to =}\NormalTok{ clearcut\_mask,}
         \AttributeTok{restrict\_values =} \DecValTok{0}\NormalTok{,}
         \AttributeTok{plot\_result=}\ConstantTok{TRUE}\NormalTok{)}
\NormalTok{p2i\_rez}
\NormalTok{i2e\_rez}\OtherTok{=}\FunctionTok{input2egv}\NormalTok{(}\AttributeTok{input=}\StringTok{"./RasterGrids\_10m/2024/ForestQuant\_AgeProp.tif"}\NormalTok{,}
         \AttributeTok{egv\_template =} \StringTok{"./Templates/TemplateRasters/LV100m\_10km.tif"}\NormalTok{,}
         \AttributeTok{summary\_function =} \StringTok{"average"}\NormalTok{,}
         \AttributeTok{missing\_job =} \StringTok{"CoverOutput"}\NormalTok{,}
         \AttributeTok{output\_bg =} \StringTok{"./Templates/TemplateRasters/nulls\_LV100m\_10km.tif"}\NormalTok{,}
         \AttributeTok{outlocation =} \StringTok{"./RasterGrids\_100m/2024/RAW/"}\NormalTok{,}
         \AttributeTok{outfilename =} \StringTok{"ForestsQuant\_AgeProp{-}average\_cell.tif"}\NormalTok{,}
         \AttributeTok{layername =} \StringTok{"egv\_290"}\NormalTok{,}
         \AttributeTok{plot\_final=}\ConstantTok{TRUE}\NormalTok{)}
\NormalTok{i2e\_rez}
\FunctionTok{rm}\NormalTok{(nogabali)}
\FunctionTok{rm}\NormalTok{(p2i\_rez)}
\FunctionTok{rm}\NormalTok{(i2e\_rez)}
\FunctionTok{unlink}\NormalTok{(}\StringTok{"./RasterGrids\_10m/2024/ForestQuant\_AgeProp.tif"}\NormalTok{)}

\CommentTok{\# standardisation {-}{-}{-}{-}}
\ControlFlowTok{if}\NormalTok{(}\SpecialCharTok{!}\FunctionTok{require}\NormalTok{(terra)) \{}\FunctionTok{install.packages}\NormalTok{(}\StringTok{"terra"}\NormalTok{); }\FunctionTok{require}\NormalTok{(terra)\}}
\ControlFlowTok{if}\NormalTok{(}\SpecialCharTok{!}\FunctionTok{require}\NormalTok{(tidyverse)) \{}\FunctionTok{install.packages}\NormalTok{(}\StringTok{"tidyverse"}\NormalTok{); }\FunctionTok{require}\NormalTok{(tidyverse)\}}

\NormalTok{nosaukums}\OtherTok{=}\StringTok{"ForestsQuant\_AgeProp{-}average\_cell.tif"}
\NormalTok{ielasisanas\_cels}\OtherTok{=}\FunctionTok{paste0}\NormalTok{(}\StringTok{"./RasterGrids\_100m/2024/RAW/"}\NormalTok{,nosaukums)}
\NormalTok{saglabasanas\_cels}\OtherTok{=}\FunctionTok{paste0}\NormalTok{(}\StringTok{"./RasterGrids\_100m/2024/Scaled/"}\NormalTok{,nosaukums)}
\NormalTok{slanis}\OtherTok{=}\FunctionTok{rast}\NormalTok{(ielasisanas\_cels)}
\NormalTok{videjais}\OtherTok{=}\FunctionTok{global}\NormalTok{(slanis,}\AttributeTok{fun=}\StringTok{"mean"}\NormalTok{,}\AttributeTok{na.rm=}\ConstantTok{TRUE}\NormalTok{)}
\NormalTok{centrets}\OtherTok{=}\NormalTok{slanis}\SpecialCharTok{{-}}\NormalTok{videjais[,}\DecValTok{1}\NormalTok{]}
\NormalTok{standartnovirze}\OtherTok{=}\NormalTok{terra}\SpecialCharTok{::}\FunctionTok{global}\NormalTok{(centrets,}\AttributeTok{fun=}\StringTok{"rms"}\NormalTok{,}\AttributeTok{na.rm=}\ConstantTok{TRUE}\NormalTok{)}
\NormalTok{merogots}\OtherTok{=}\NormalTok{centrets}\SpecialCharTok{/}\NormalTok{standartnovirze[,}\DecValTok{1}\NormalTok{]}
\FunctionTok{writeRaster}\NormalTok{(merogots,}
      \AttributeTok{filename=}\NormalTok{saglabasanas\_cels,}
      \AttributeTok{overwrite=}\ConstantTok{TRUE}\NormalTok{)}
\end{Highlighting}
\end{Shaded}

\section{ForestsQuant\_DominantDiameter-max\_cell}\label{ch06.291}

\textbf{filename:} \texttt{ForestsQuant\_DominantDiameter-max\_cell.tif}

\textbf{layername:} \texttt{egv\_291}

\textbf{English name:} Dominant tree trunk diameter, maximum within the analysis cell
(1 ha)

\textbf{Latvian name:} Koku stumbra diametrs, valdaudzes maksimālais analīzes šūnā (1
ha)

\textbf{Procedure:} Most of forests describing EGVs are spatially restricted outside
clearcuts and dead stands. Mask for this is created from the \hyperref[Ch04.01]{State Forest Service's
State Forest Registry} land category 12 and 14 combined with \hyperref[Ch04.09]{The
Global Forest Watch} pixels classified as lost tree canopy cover since
2020 (raster layer matching input, presence = 1, absence = 0).

This EGV is prepared based on the information of dominant tree species per
inventoried forest stand - \hyperref[Ch04.01]{State Forest Service's State Forest
Registry}. This attribute has some extreme
values. We chose to limit them to the nearest integer showing only minimal
accumulation in histogram.

\includegraphics[width=0.8\linewidth]{./Figures/Histogramms/hist_egv291}

Resulting values at polygon geometries are rasterised with the workflow
\texttt{egvtools::polygon2input()}, restricting to pixels outside the clearcut mask. No
background values are assigned during rasterisation. The resulting layer is
then aggregated to EGV resolution using the workflow \texttt{egvtools::input2egv()} by calculating
maximum value. After the aggregation, cells with no forest information are
filled with value 0. Finally, the layer is standardised by subtracting
the arithmetic mean and dividing by the root mean squared error.

\begin{Shaded}
\begin{Highlighting}[]
\CommentTok{\# libs {-}{-}{-}{-}}
\ControlFlowTok{if}\NormalTok{(}\SpecialCharTok{!}\FunctionTok{require}\NormalTok{(egvtools)) \{remotes}\SpecialCharTok{::}\FunctionTok{install\_github}\NormalTok{(}\StringTok{"aavotins/egvtools"}\NormalTok{); }\FunctionTok{require}\NormalTok{(egvtools)\}}
\ControlFlowTok{if}\NormalTok{(}\SpecialCharTok{!}\FunctionTok{require}\NormalTok{(terra)) \{}\FunctionTok{install.packages}\NormalTok{(}\StringTok{"terra"}\NormalTok{); }\FunctionTok{require}\NormalTok{(terra)\}}
\ControlFlowTok{if}\NormalTok{(}\SpecialCharTok{!}\FunctionTok{require}\NormalTok{(sf)) \{}\FunctionTok{install.packages}\NormalTok{(}\StringTok{"sf"}\NormalTok{); }\FunctionTok{require}\NormalTok{(sf)\}}
\ControlFlowTok{if}\NormalTok{(}\SpecialCharTok{!}\FunctionTok{require}\NormalTok{(tidyverse)) \{}\FunctionTok{install.packages}\NormalTok{(}\StringTok{"tidyverse"}\NormalTok{); }\FunctionTok{require}\NormalTok{(tidyverse)\}}
\ControlFlowTok{if}\NormalTok{(}\SpecialCharTok{!}\FunctionTok{require}\NormalTok{(sfarrow)) \{}\FunctionTok{install.packages}\NormalTok{(}\StringTok{"sfarrow"}\NormalTok{); }\FunctionTok{require}\NormalTok{(sfarrow)\}}
\ControlFlowTok{if}\NormalTok{(}\SpecialCharTok{!}\FunctionTok{require}\NormalTok{(readxl)) \{}\FunctionTok{install.packages}\NormalTok{(}\StringTok{"readxl"}\NormalTok{); }\FunctionTok{require}\NormalTok{(readxl)\}}
\ControlFlowTok{if}\NormalTok{(}\SpecialCharTok{!}\FunctionTok{require}\NormalTok{(raster)) \{}\FunctionTok{install.packages}\NormalTok{(}\StringTok{"raster"}\NormalTok{); }\FunctionTok{require}\NormalTok{(raster)\}}
\ControlFlowTok{if}\NormalTok{(}\SpecialCharTok{!}\FunctionTok{require}\NormalTok{(fasterize)) \{}\FunctionTok{install.packages}\NormalTok{(}\StringTok{"fasterize"}\NormalTok{); }\FunctionTok{require}\NormalTok{(fasterize)\}}

\CommentTok{\# templates {-}{-}{-}{-}}
\NormalTok{template100}\OtherTok{=}\FunctionTok{rast}\NormalTok{(}\StringTok{"./Templates/TemplateRasters/LV100m\_10km.tif"}\NormalTok{)}
\NormalTok{template10}\OtherTok{=}\FunctionTok{rast}\NormalTok{(}\StringTok{"./Templates/TemplateRasters/LV10m\_10km.tif"}\NormalTok{)}
\NormalTok{rastrs10}\OtherTok{=}\FunctionTok{raster}\NormalTok{(template10)}

\NormalTok{nulls10}\OtherTok{=}\FunctionTok{rast}\NormalTok{(}\StringTok{"./Templates/TemplateRasters/nulls\_LV10m\_10km.tif"}\NormalTok{)}
\NormalTok{nulls100}\OtherTok{=}\FunctionTok{rast}\NormalTok{(}\StringTok{"./Templates/TemplateRasters/nulls\_LV100m\_10km.tif"}\NormalTok{)}


\CommentTok{\# simple landscape {-}{-}{-}{-}}
\NormalTok{simple\_landscape}\OtherTok{=}\FunctionTok{rast}\NormalTok{(}\StringTok{"RasterGrids\_10m/2024/Ainava\_vienk\_mask.tif"}\NormalTok{)}

\CommentTok{\# mvr {-}{-}{-}{-}}
\NormalTok{mvr}\OtherTok{=}\FunctionTok{st\_read\_parquet}\NormalTok{(}\StringTok{"./Geodata/2024/MVR/nogabali\_2024janv.parquet"}\NormalTok{)}
\NormalTok{mvr}\SpecialCharTok{$}\NormalTok{yes}\OtherTok{=}\DecValTok{1}

\CommentTok{\# clear cut mask {-}{-}{-}{-}}
\NormalTok{izcirtumi}\OtherTok{=}\NormalTok{mvr }\SpecialCharTok{\%\textgreater{}\%} 
 \FunctionTok{filter}\NormalTok{(zkat }\SpecialCharTok{\%in\%} \FunctionTok{c}\NormalTok{(}\StringTok{"12"}\NormalTok{,}\StringTok{"14"}\NormalTok{)) }\SpecialCharTok{\%\textgreater{}\%} 
\NormalTok{ dplyr}\SpecialCharTok{::}\FunctionTok{select}\NormalTok{(yes)}
\NormalTok{r\_izcirtumi\_mvr}\OtherTok{=}\FunctionTok{fasterize}\NormalTok{(izcirtumi,rastrs10,}\AttributeTok{field=}\StringTok{"yes"}\NormalTok{)}
\NormalTok{t\_izcirtumi\_mvr}\OtherTok{=}\FunctionTok{rast}\NormalTok{(r\_izcirtumi\_mvr)}
\FunctionTok{plot}\NormalTok{(t\_izcirtumi\_mvr)}

\NormalTok{tcl}\OtherTok{=}\FunctionTok{rast}\NormalTok{(}\StringTok{"./Geodata/2024/Trees/GFW/TreeCoverLoss\_v1\_12.tif"}\NormalTok{)}
\NormalTok{tcl2}\OtherTok{=}\FunctionTok{ifel}\NormalTok{(tcl}\SpecialCharTok{\textless{}}\DecValTok{20}\NormalTok{,}\DecValTok{0}\NormalTok{,}\DecValTok{1}\NormalTok{)}
\NormalTok{tclX}\OtherTok{=}\FunctionTok{cover}\NormalTok{(tcl2,nulls10)}
\FunctionTok{plot}\NormalTok{(tclX)}

\NormalTok{clearcut\_mask}\OtherTok{=}\FunctionTok{cover}\NormalTok{(t\_izcirtumi\_mvr,tclX,}
          \AttributeTok{filename=}\StringTok{"./RasterGrids\_10m/2024/Mask\_clearcuts.tif"}\NormalTok{,}
          \AttributeTok{overwrite=}\ConstantTok{TRUE}\NormalTok{)}
\FunctionTok{plot}\NormalTok{(clearcut\_mask)}

\FunctionTok{rm}\NormalTok{(izcirtumi)}
\FunctionTok{rm}\NormalTok{(r\_izcirtumi\_mvr)}
\FunctionTok{rm}\NormalTok{(t\_izcirtumi\_mvr)}
\FunctionTok{rm}\NormalTok{(tcl)}
\FunctionTok{rm}\NormalTok{(tcl2)}
\FunctionTok{rm}\NormalTok{(tclX)}

\CommentTok{\# ForestsQuant\_DominantDiameter{-}max\_cell.tif    egv\_291 {-}{-}{-}{-}}
\NormalTok{nogabali}\OtherTok{=}\NormalTok{mvr }\SpecialCharTok{\%\textgreater{}\%} 
 \FunctionTok{mutate}\NormalTok{(}\AttributeTok{valddiam=}\FunctionTok{ifelse}\NormalTok{(d10}\SpecialCharTok{\textgreater{}}\DecValTok{70}\NormalTok{,}\DecValTok{70}\NormalTok{,d10)) }\SpecialCharTok{\%\textgreater{}\%} 
 \FunctionTok{filter}\NormalTok{(}\SpecialCharTok{!}\FunctionTok{is.na}\NormalTok{(valddiam))}

\FunctionTok{par}\NormalTok{(}\AttributeTok{mfrow=}\FunctionTok{c}\NormalTok{(}\DecValTok{1}\NormalTok{,}\DecValTok{2}\NormalTok{))}
\FunctionTok{options}\NormalTok{(}\AttributeTok{scipen=}\DecValTok{999}\NormalTok{)}
\FunctionTok{hist}\NormalTok{(nogabali}\SpecialCharTok{$}\NormalTok{d10,}\AttributeTok{main=}\StringTok{"Original"}\NormalTok{,}\AttributeTok{xlab=}\StringTok{"Dominant diameter"}\NormalTok{)}
\FunctionTok{hist}\NormalTok{(nogabali}\SpecialCharTok{$}\NormalTok{valddiam,}\AttributeTok{main=}\StringTok{"Limited"}\NormalTok{,}\AttributeTok{xlab=}\StringTok{"Dominant diameter"}\NormalTok{)}
\FunctionTok{par}\NormalTok{(}\AttributeTok{mfrow=}\FunctionTok{c}\NormalTok{(}\DecValTok{1}\NormalTok{,}\DecValTok{1}\NormalTok{))}
\FunctionTok{options}\NormalTok{(}\AttributeTok{scipen=}\DecValTok{0}\NormalTok{)}

\NormalTok{p2i\_rez}\OtherTok{=}\FunctionTok{polygon2input}\NormalTok{(}\AttributeTok{vector\_data=}\NormalTok{nogabali,}
           \AttributeTok{template\_path =} \StringTok{"./Templates/TemplateRasters/LV10m\_10km.tif"}\NormalTok{,}
           \AttributeTok{out\_path =} \StringTok{"./RasterGrids\_10m/2024/"}\NormalTok{,}
           \AttributeTok{file\_name =} \StringTok{"ForestQuant\_DominantDiameter.tif"}\NormalTok{,}
           \AttributeTok{value\_field =} \StringTok{"valddiam"}\NormalTok{,}
           \AttributeTok{fun=}\StringTok{"max"}\NormalTok{,}
           \AttributeTok{prepare=}\ConstantTok{FALSE}\NormalTok{,}
           \AttributeTok{restrict\_to =}\NormalTok{ clearcut\_mask,}
           \AttributeTok{restrict\_values =} \DecValTok{0}\NormalTok{,}
           \AttributeTok{plot\_result=}\ConstantTok{TRUE}\NormalTok{,}
           \AttributeTok{overwrite=}\ConstantTok{TRUE}\NormalTok{)}
\NormalTok{p2i\_rez}
\NormalTok{i2e\_rez}\OtherTok{=}\FunctionTok{input2egv}\NormalTok{(}\AttributeTok{input=}\StringTok{"./RasterGrids\_10m/2024/ForestQuant\_DominantDiameter.tif"}\NormalTok{,}
         \AttributeTok{egv\_template =} \StringTok{"./Templates/TemplateRasters/LV100m\_10km.tif"}\NormalTok{,}
         \AttributeTok{summary\_function =} \StringTok{"max"}\NormalTok{,}
         \AttributeTok{missing\_job =} \StringTok{"CoverOutput"}\NormalTok{,}
         \AttributeTok{output\_bg =} \StringTok{"./Templates/TemplateRasters/nulls\_LV100m\_10km.tif"}\NormalTok{,}
         \AttributeTok{outlocation =} \StringTok{"./RasterGrids\_100m/2024/RAW/"}\NormalTok{,}
         \AttributeTok{outfilename =} \StringTok{"ForestsQuant\_DominantDiameter{-}max\_cell.tif"}\NormalTok{,}
         \AttributeTok{layername =} \StringTok{"egv\_291"}\NormalTok{,}
         \AttributeTok{plot\_final=}\ConstantTok{TRUE}\NormalTok{)}
\NormalTok{i2e\_rez}
\FunctionTok{rm}\NormalTok{(p2i\_rez)}
\FunctionTok{rm}\NormalTok{(nogabali)}
\FunctionTok{rm}\NormalTok{(i2e\_rez)}
\FunctionTok{unlink}\NormalTok{(}\StringTok{"./RasterGrids\_10m/2024/ForestQuant\_DominantDiameter.tif"}\NormalTok{)}

\CommentTok{\# standardisation {-}{-}{-}{-}}
\ControlFlowTok{if}\NormalTok{(}\SpecialCharTok{!}\FunctionTok{require}\NormalTok{(terra)) \{}\FunctionTok{install.packages}\NormalTok{(}\StringTok{"terra"}\NormalTok{); }\FunctionTok{require}\NormalTok{(terra)\}}
\ControlFlowTok{if}\NormalTok{(}\SpecialCharTok{!}\FunctionTok{require}\NormalTok{(tidyverse)) \{}\FunctionTok{install.packages}\NormalTok{(}\StringTok{"tidyverse"}\NormalTok{); }\FunctionTok{require}\NormalTok{(tidyverse)\}}

\NormalTok{nosaukums}\OtherTok{=}\StringTok{"ForestsQuant\_DominantDiameter{-}max\_cell.tif"}
\NormalTok{ielasisanas\_cels}\OtherTok{=}\FunctionTok{paste0}\NormalTok{(}\StringTok{"./RasterGrids\_100m/2024/RAW/"}\NormalTok{,nosaukums)}
\NormalTok{saglabasanas\_cels}\OtherTok{=}\FunctionTok{paste0}\NormalTok{(}\StringTok{"./RasterGrids\_100m/2024/Scaled/"}\NormalTok{,nosaukums)}
\NormalTok{slanis}\OtherTok{=}\FunctionTok{rast}\NormalTok{(ielasisanas\_cels)}
\NormalTok{videjais}\OtherTok{=}\FunctionTok{global}\NormalTok{(slanis,}\AttributeTok{fun=}\StringTok{"mean"}\NormalTok{,}\AttributeTok{na.rm=}\ConstantTok{TRUE}\NormalTok{)}
\NormalTok{centrets}\OtherTok{=}\NormalTok{slanis}\SpecialCharTok{{-}}\NormalTok{videjais[,}\DecValTok{1}\NormalTok{]}
\NormalTok{standartnovirze}\OtherTok{=}\NormalTok{terra}\SpecialCharTok{::}\FunctionTok{global}\NormalTok{(centrets,}\AttributeTok{fun=}\StringTok{"rms"}\NormalTok{,}\AttributeTok{na.rm=}\ConstantTok{TRUE}\NormalTok{)}
\NormalTok{merogots}\OtherTok{=}\NormalTok{centrets}\SpecialCharTok{/}\NormalTok{standartnovirze[,}\DecValTok{1}\NormalTok{]}
\FunctionTok{writeRaster}\NormalTok{(merogots,}
      \AttributeTok{filename=}\NormalTok{saglabasanas\_cels,}
      \AttributeTok{overwrite=}\ConstantTok{TRUE}\NormalTok{)}
\end{Highlighting}
\end{Shaded}

\section{ForestsQuant\_LargestDiameter-max\_cell}\label{ch06.292}

\textbf{filename:} \texttt{ForestsQuant\_LargestDiameter-max\_cell.tif}

\textbf{layername:} \texttt{egv\_292}

\textbf{English name:} Largest tree trunk diameter within the analysis cell (1 ha)

\textbf{Latvian name:} Lielākais koka stumbra diametrs analīzes šūnā (1 ha)

\textbf{Procedure:} Most EGVs describing forests are spatially restricted to areas outside
of clearcuts and dead stands. This mask is created using a combination of
the \hyperref[Ch04.01]{State Forest Service's
State Forest Registry} land category 12 and 14, and \hyperref[Ch04.09]{The
Global Forest Watch} pixels classified as lost tree canopy cover since
2020 (raster layer matching input, presence = 1, absence = 0).

This EGV is prepared based on the information of the largest tree diameter per
inventoried forest stand - \hyperref[Ch04.01]{State Forest Service's State Forest
Registry}. This attribute has some extreme
values. We chose to limit them to the nearest integer showing only minimal
accumulation in histogram.

\includegraphics[width=0.8\linewidth]{./Figures/Histogramms/hist_egv292}

Resulting values at polygon geometries are rasterised with the workflow
\texttt{egvtools::polygon2input()}, restricting to pixels outside the clearcut mask. No
background values are assigned during rasterisation. The resulting layer is
then aggregated to EGV resolution using the workflow \texttt{egvtools::input2egv()} by calculating
maximum value. After the aggregation, cells with no forest information are
filled with value 0. Finally, the layer is standardised by subtracting
the arithmetic mean and dividing by the root mean squared error.

\begin{Shaded}
\begin{Highlighting}[]
\CommentTok{\# libs {-}{-}{-}{-}}
\ControlFlowTok{if}\NormalTok{(}\SpecialCharTok{!}\FunctionTok{require}\NormalTok{(egvtools)) \{remotes}\SpecialCharTok{::}\FunctionTok{install\_github}\NormalTok{(}\StringTok{"aavotins/egvtools"}\NormalTok{); }\FunctionTok{require}\NormalTok{(egvtools)\}}
\ControlFlowTok{if}\NormalTok{(}\SpecialCharTok{!}\FunctionTok{require}\NormalTok{(terra)) \{}\FunctionTok{install.packages}\NormalTok{(}\StringTok{"terra"}\NormalTok{); }\FunctionTok{require}\NormalTok{(terra)\}}
\ControlFlowTok{if}\NormalTok{(}\SpecialCharTok{!}\FunctionTok{require}\NormalTok{(sf)) \{}\FunctionTok{install.packages}\NormalTok{(}\StringTok{"sf"}\NormalTok{); }\FunctionTok{require}\NormalTok{(sf)\}}
\ControlFlowTok{if}\NormalTok{(}\SpecialCharTok{!}\FunctionTok{require}\NormalTok{(tidyverse)) \{}\FunctionTok{install.packages}\NormalTok{(}\StringTok{"tidyverse"}\NormalTok{); }\FunctionTok{require}\NormalTok{(tidyverse)\}}
\ControlFlowTok{if}\NormalTok{(}\SpecialCharTok{!}\FunctionTok{require}\NormalTok{(sfarrow)) \{}\FunctionTok{install.packages}\NormalTok{(}\StringTok{"sfarrow"}\NormalTok{); }\FunctionTok{require}\NormalTok{(sfarrow)\}}
\ControlFlowTok{if}\NormalTok{(}\SpecialCharTok{!}\FunctionTok{require}\NormalTok{(readxl)) \{}\FunctionTok{install.packages}\NormalTok{(}\StringTok{"readxl"}\NormalTok{); }\FunctionTok{require}\NormalTok{(readxl)\}}
\ControlFlowTok{if}\NormalTok{(}\SpecialCharTok{!}\FunctionTok{require}\NormalTok{(raster)) \{}\FunctionTok{install.packages}\NormalTok{(}\StringTok{"raster"}\NormalTok{); }\FunctionTok{require}\NormalTok{(raster)\}}
\ControlFlowTok{if}\NormalTok{(}\SpecialCharTok{!}\FunctionTok{require}\NormalTok{(fasterize)) \{}\FunctionTok{install.packages}\NormalTok{(}\StringTok{"fasterize"}\NormalTok{); }\FunctionTok{require}\NormalTok{(fasterize)\}}

\CommentTok{\# templates {-}{-}{-}{-}}
\NormalTok{template100}\OtherTok{=}\FunctionTok{rast}\NormalTok{(}\StringTok{"./Templates/TemplateRasters/LV100m\_10km.tif"}\NormalTok{)}
\NormalTok{template10}\OtherTok{=}\FunctionTok{rast}\NormalTok{(}\StringTok{"./Templates/TemplateRasters/LV10m\_10km.tif"}\NormalTok{)}
\NormalTok{rastrs10}\OtherTok{=}\FunctionTok{raster}\NormalTok{(template10)}

\NormalTok{nulls10}\OtherTok{=}\FunctionTok{rast}\NormalTok{(}\StringTok{"./Templates/TemplateRasters/nulls\_LV10m\_10km.tif"}\NormalTok{)}
\NormalTok{nulls100}\OtherTok{=}\FunctionTok{rast}\NormalTok{(}\StringTok{"./Templates/TemplateRasters/nulls\_LV100m\_10km.tif"}\NormalTok{)}


\CommentTok{\# simple landscape {-}{-}{-}{-}}
\NormalTok{simple\_landscape}\OtherTok{=}\FunctionTok{rast}\NormalTok{(}\StringTok{"RasterGrids\_10m/2024/Ainava\_vienk\_mask.tif"}\NormalTok{)}

\CommentTok{\# mvr {-}{-}{-}{-}}
\NormalTok{mvr}\OtherTok{=}\FunctionTok{st\_read\_parquet}\NormalTok{(}\StringTok{"./Geodata/2024/MVR/nogabali\_2024janv.parquet"}\NormalTok{)}
\NormalTok{mvr}\SpecialCharTok{$}\NormalTok{yes}\OtherTok{=}\DecValTok{1}

\CommentTok{\# clear cut mask {-}{-}{-}{-}}
\NormalTok{izcirtumi}\OtherTok{=}\NormalTok{mvr }\SpecialCharTok{\%\textgreater{}\%} 
 \FunctionTok{filter}\NormalTok{(zkat }\SpecialCharTok{\%in\%} \FunctionTok{c}\NormalTok{(}\StringTok{"12"}\NormalTok{,}\StringTok{"14"}\NormalTok{)) }\SpecialCharTok{\%\textgreater{}\%} 
\NormalTok{ dplyr}\SpecialCharTok{::}\FunctionTok{select}\NormalTok{(yes)}
\NormalTok{r\_izcirtumi\_mvr}\OtherTok{=}\FunctionTok{fasterize}\NormalTok{(izcirtumi,rastrs10,}\AttributeTok{field=}\StringTok{"yes"}\NormalTok{)}
\NormalTok{t\_izcirtumi\_mvr}\OtherTok{=}\FunctionTok{rast}\NormalTok{(r\_izcirtumi\_mvr)}
\FunctionTok{plot}\NormalTok{(t\_izcirtumi\_mvr)}

\NormalTok{tcl}\OtherTok{=}\FunctionTok{rast}\NormalTok{(}\StringTok{"./Geodata/2024/Trees/GFW/TreeCoverLoss\_v1\_12.tif"}\NormalTok{)}
\NormalTok{tcl2}\OtherTok{=}\FunctionTok{ifel}\NormalTok{(tcl}\SpecialCharTok{\textless{}}\DecValTok{20}\NormalTok{,}\DecValTok{0}\NormalTok{,}\DecValTok{1}\NormalTok{)}
\NormalTok{tclX}\OtherTok{=}\FunctionTok{cover}\NormalTok{(tcl2,nulls10)}
\FunctionTok{plot}\NormalTok{(tclX)}

\NormalTok{clearcut\_mask}\OtherTok{=}\FunctionTok{cover}\NormalTok{(t\_izcirtumi\_mvr,tclX,}
          \AttributeTok{filename=}\StringTok{"./RasterGrids\_10m/2024/Mask\_clearcuts.tif"}\NormalTok{,}
          \AttributeTok{overwrite=}\ConstantTok{TRUE}\NormalTok{)}
\FunctionTok{plot}\NormalTok{(clearcut\_mask)}

\FunctionTok{rm}\NormalTok{(izcirtumi)}
\FunctionTok{rm}\NormalTok{(r\_izcirtumi\_mvr)}
\FunctionTok{rm}\NormalTok{(t\_izcirtumi\_mvr)}
\FunctionTok{rm}\NormalTok{(tcl)}
\FunctionTok{rm}\NormalTok{(tcl2)}
\FunctionTok{rm}\NormalTok{(tclX)}

\CommentTok{\# ForestsQuant\_LargestDiameter{-}max\_cell.tif egv\_292 {-}{-}{-}{-}}
\NormalTok{nogabali}\OtherTok{=}\NormalTok{mvr }\SpecialCharTok{\%\textgreater{}\%}
 \FunctionTok{rowwise}\NormalTok{() }\SpecialCharTok{\%\textgreater{}\%} 
 \FunctionTok{mutate}\NormalTok{(}\AttributeTok{maxDiam=}\FunctionTok{max}\NormalTok{(}\FunctionTok{c}\NormalTok{(d10,d11,d12,d13,d14,d22,d23,d24),}\AttributeTok{na.rm=}\ConstantTok{TRUE}\NormalTok{)) }\SpecialCharTok{\%\textgreater{}\%} 
 \FunctionTok{ungroup}\NormalTok{() }\SpecialCharTok{\%\textgreater{}\%} 
 \FunctionTok{mutate}\NormalTok{(}\AttributeTok{maxDiam2=}\FunctionTok{ifelse}\NormalTok{(maxDiam}\SpecialCharTok{\textgreater{}}\DecValTok{100}\NormalTok{,}\DecValTok{100}\NormalTok{,maxDiam)) }\SpecialCharTok{\%\textgreater{}\%} 
 \FunctionTok{filter}\NormalTok{(}\SpecialCharTok{!}\FunctionTok{is.na}\NormalTok{(maxDiam2))}

\FunctionTok{par}\NormalTok{(}\AttributeTok{mfrow=}\FunctionTok{c}\NormalTok{(}\DecValTok{1}\NormalTok{,}\DecValTok{2}\NormalTok{))}
\FunctionTok{options}\NormalTok{(}\AttributeTok{scipen=}\DecValTok{999}\NormalTok{)}
\FunctionTok{hist}\NormalTok{(nogabali}\SpecialCharTok{$}\NormalTok{maxDiam,}\AttributeTok{main=}\StringTok{"Original"}\NormalTok{,}\AttributeTok{xlab=}\StringTok{"Largest diameter"}\NormalTok{)}
\FunctionTok{hist}\NormalTok{(nogabali}\SpecialCharTok{$}\NormalTok{maxDiam2,}\AttributeTok{main=}\StringTok{"Limited"}\NormalTok{,}\AttributeTok{xlab=}\StringTok{"Largest diameter"}\NormalTok{)}
\FunctionTok{par}\NormalTok{(}\AttributeTok{mfrow=}\FunctionTok{c}\NormalTok{(}\DecValTok{1}\NormalTok{,}\DecValTok{1}\NormalTok{))}
\FunctionTok{options}\NormalTok{(}\AttributeTok{scipen=}\DecValTok{0}\NormalTok{)}

\NormalTok{p2i\_rez}\OtherTok{=}\FunctionTok{polygon2input}\NormalTok{(}\AttributeTok{vector\_data=}\NormalTok{nogabali,}
           \AttributeTok{template\_path =} \StringTok{"./Templates/TemplateRasters/LV10m\_10km.tif"}\NormalTok{,}
           \AttributeTok{out\_path =} \StringTok{"./RasterGrids\_10m/2024/"}\NormalTok{,}
           \AttributeTok{file\_name =} \StringTok{"ForestsQuant\_LargestDiameter.tif"}\NormalTok{,}
           \AttributeTok{value\_field =} \StringTok{"maxDiam2"}\NormalTok{,}
           \AttributeTok{fun=}\StringTok{"max"}\NormalTok{,}
           \AttributeTok{prepare=}\ConstantTok{FALSE}\NormalTok{,}
           \AttributeTok{restrict\_to =}\NormalTok{ clearcut\_mask,}
           \AttributeTok{restrict\_values =} \DecValTok{0}\NormalTok{,}
           \AttributeTok{plot\_result=}\ConstantTok{TRUE}\NormalTok{,}
           \AttributeTok{overwrite=}\ConstantTok{TRUE}\NormalTok{)}
\NormalTok{p2i\_rez}
\NormalTok{i2e\_rez}\OtherTok{=}\FunctionTok{input2egv}\NormalTok{(}\AttributeTok{input=}\StringTok{"./RasterGrids\_10m/2024/ForestsQuant\_LargestDiameter.tif"}\NormalTok{,}
         \AttributeTok{egv\_template =} \StringTok{"./Templates/TemplateRasters/LV100m\_10km.tif"}\NormalTok{,}
         \AttributeTok{summary\_function =} \StringTok{"max"}\NormalTok{,}
         \AttributeTok{missing\_job =} \StringTok{"CoverOutput"}\NormalTok{,}
         \AttributeTok{output\_bg =} \StringTok{"./Templates/TemplateRasters/nulls\_LV100m\_10km.tif"}\NormalTok{,}
         \AttributeTok{outlocation =} \StringTok{"./RasterGrids\_100m/2024/RAW/"}\NormalTok{,}
         \AttributeTok{outfilename =} \StringTok{"ForestsQuant\_LargestDiameter{-}max\_cell.tif"}\NormalTok{,}
         \AttributeTok{layername =} \StringTok{"egv\_292"}\NormalTok{,}
         \AttributeTok{plot\_final=}\ConstantTok{TRUE}\NormalTok{)}
\NormalTok{i2e\_rez}
\FunctionTok{rm}\NormalTok{(p2i\_rez)}
\FunctionTok{rm}\NormalTok{(nogabali)}
\FunctionTok{rm}\NormalTok{(i2e\_rez)}
\FunctionTok{unlink}\NormalTok{(}\StringTok{"./RasterGrids\_10m/2024/ForestsQuant\_LargestDiameter.tif"}\NormalTok{)}

\CommentTok{\# standardisation {-}{-}{-}{-}}
\ControlFlowTok{if}\NormalTok{(}\SpecialCharTok{!}\FunctionTok{require}\NormalTok{(terra)) \{}\FunctionTok{install.packages}\NormalTok{(}\StringTok{"terra"}\NormalTok{); }\FunctionTok{require}\NormalTok{(terra)\}}
\ControlFlowTok{if}\NormalTok{(}\SpecialCharTok{!}\FunctionTok{require}\NormalTok{(tidyverse)) \{}\FunctionTok{install.packages}\NormalTok{(}\StringTok{"tidyverse"}\NormalTok{); }\FunctionTok{require}\NormalTok{(tidyverse)\}}

\NormalTok{nosaukums}\OtherTok{=}\StringTok{"ForestsQuant\_LargestDiameter{-}max\_cell.tif"}
\NormalTok{ielasisanas\_cels}\OtherTok{=}\FunctionTok{paste0}\NormalTok{(}\StringTok{"./RasterGrids\_100m/2024/RAW/"}\NormalTok{,nosaukums)}
\NormalTok{saglabasanas\_cels}\OtherTok{=}\FunctionTok{paste0}\NormalTok{(}\StringTok{"./RasterGrids\_100m/2024/Scaled/"}\NormalTok{,nosaukums)}
\NormalTok{slanis}\OtherTok{=}\FunctionTok{rast}\NormalTok{(ielasisanas\_cels)}
\NormalTok{videjais}\OtherTok{=}\FunctionTok{global}\NormalTok{(slanis,}\AttributeTok{fun=}\StringTok{"mean"}\NormalTok{,}\AttributeTok{na.rm=}\ConstantTok{TRUE}\NormalTok{)}
\NormalTok{centrets}\OtherTok{=}\NormalTok{slanis}\SpecialCharTok{{-}}\NormalTok{videjais[,}\DecValTok{1}\NormalTok{]}
\NormalTok{standartnovirze}\OtherTok{=}\NormalTok{terra}\SpecialCharTok{::}\FunctionTok{global}\NormalTok{(centrets,}\AttributeTok{fun=}\StringTok{"rms"}\NormalTok{,}\AttributeTok{na.rm=}\ConstantTok{TRUE}\NormalTok{)}
\NormalTok{merogots}\OtherTok{=}\NormalTok{centrets}\SpecialCharTok{/}\NormalTok{standartnovirze[,}\DecValTok{1}\NormalTok{]}
\FunctionTok{writeRaster}\NormalTok{(merogots,}
      \AttributeTok{filename=}\NormalTok{saglabasanas\_cels,}
      \AttributeTok{overwrite=}\ConstantTok{TRUE}\NormalTok{)}
\end{Highlighting}
\end{Shaded}

\section{ForestsQuant\_TimeSinceDisturbance-average\_cell}\label{ch06.293}

\textbf{filename:} \texttt{ForestsQuant\_TimeSinceDisturbance-average\_cell.tif}

\textbf{layername:} \texttt{egv\_293}

\textbf{English name:} Time since last disturbance affecting tree growing within the
analysis cell (1 ha)

\textbf{Latvian name:} Laiks kopš pēdējā ar koku augšanu saistītā traucējuma analīzes
šūnā (1 ha)

\textbf{Procedure:} This EGV is prepared primarily based on the information of the forestry related
disturbances as registered per inventoried forest stand - \hyperref[Ch04.01]{State Forest
Service's State Forest Registry}. The register, however, includes obvious
errors - values later than 2024 and earlier than 1500 that are set to NA. Remaining
values are subtracted from 2024. In stands with no disturbance registered,
the age of dominant tree group is used to calculate minimum difference (age of
time since disturbance) from the year 2024. This attribute has some extreme
values. We chose to limit them to the nearest integer showing only minimal
accumulation in histogram.

\includegraphics[width=0.8\linewidth]{./Figures/Histogramms/hist_egv293}

Resulting values at polygon geometries are rasterised (presence only). This
raster layer is then overlaid with reclassified year of tree cover loss
(reclassified to difference from the year 2024) and per pixel minimum value is
retained. As not all the forests or tree covered areas are inventoried, classes
from the \hyperref[Ch05.03]{Landscape classification} are used to impute assumption of time
since tree growing disturbance - for class 620 we assume five years, whereas
for classes 630 and 640 - 50 years and 0 otherwise. Per pixel minimum layer is
then overlaid the assumed time since disturbance layer. The resulting layer is
then aggregated to EGV resolution using the workflow \texttt{egvtools::input2egv()} by calculating
arithmetic mean value. After the aggregation, inverse distance weighted (power =
2) gap filling is applied to avoid possible gaps at the edges. Finally, the layer
is standardised by subtracting the arithmetic mean and dividing by the root
mean squared error.

\begin{Shaded}
\begin{Highlighting}[]
\CommentTok{\# libs {-}{-}{-}{-}}
\ControlFlowTok{if}\NormalTok{(}\SpecialCharTok{!}\FunctionTok{require}\NormalTok{(egvtools)) \{remotes}\SpecialCharTok{::}\FunctionTok{install\_github}\NormalTok{(}\StringTok{"aavotins/egvtools"}\NormalTok{); }\FunctionTok{require}\NormalTok{(egvtools)\}}
\ControlFlowTok{if}\NormalTok{(}\SpecialCharTok{!}\FunctionTok{require}\NormalTok{(terra)) \{}\FunctionTok{install.packages}\NormalTok{(}\StringTok{"terra"}\NormalTok{); }\FunctionTok{require}\NormalTok{(terra)\}}
\ControlFlowTok{if}\NormalTok{(}\SpecialCharTok{!}\FunctionTok{require}\NormalTok{(sf)) \{}\FunctionTok{install.packages}\NormalTok{(}\StringTok{"sf"}\NormalTok{); }\FunctionTok{require}\NormalTok{(sf)\}}
\ControlFlowTok{if}\NormalTok{(}\SpecialCharTok{!}\FunctionTok{require}\NormalTok{(tidyverse)) \{}\FunctionTok{install.packages}\NormalTok{(}\StringTok{"tidyverse"}\NormalTok{); }\FunctionTok{require}\NormalTok{(tidyverse)\}}
\ControlFlowTok{if}\NormalTok{(}\SpecialCharTok{!}\FunctionTok{require}\NormalTok{(sfarrow)) \{}\FunctionTok{install.packages}\NormalTok{(}\StringTok{"sfarrow"}\NormalTok{); }\FunctionTok{require}\NormalTok{(sfarrow)\}}
\ControlFlowTok{if}\NormalTok{(}\SpecialCharTok{!}\FunctionTok{require}\NormalTok{(readxl)) \{}\FunctionTok{install.packages}\NormalTok{(}\StringTok{"readxl"}\NormalTok{); }\FunctionTok{require}\NormalTok{(readxl)\}}
\ControlFlowTok{if}\NormalTok{(}\SpecialCharTok{!}\FunctionTok{require}\NormalTok{(raster)) \{}\FunctionTok{install.packages}\NormalTok{(}\StringTok{"raster"}\NormalTok{); }\FunctionTok{require}\NormalTok{(raster)\}}
\ControlFlowTok{if}\NormalTok{(}\SpecialCharTok{!}\FunctionTok{require}\NormalTok{(fasterize)) \{}\FunctionTok{install.packages}\NormalTok{(}\StringTok{"fasterize"}\NormalTok{); }\FunctionTok{require}\NormalTok{(fasterize)\}}

\CommentTok{\# templates {-}{-}{-}{-}}
\NormalTok{template100}\OtherTok{=}\FunctionTok{rast}\NormalTok{(}\StringTok{"./Templates/TemplateRasters/LV100m\_10km.tif"}\NormalTok{)}
\NormalTok{template10}\OtherTok{=}\FunctionTok{rast}\NormalTok{(}\StringTok{"./Templates/TemplateRasters/LV10m\_10km.tif"}\NormalTok{)}
\NormalTok{rastrs10}\OtherTok{=}\FunctionTok{raster}\NormalTok{(template10)}

\NormalTok{nulls10}\OtherTok{=}\FunctionTok{rast}\NormalTok{(}\StringTok{"./Templates/TemplateRasters/nulls\_LV10m\_10km.tif"}\NormalTok{)}
\NormalTok{nulls100}\OtherTok{=}\FunctionTok{rast}\NormalTok{(}\StringTok{"./Templates/TemplateRasters/nulls\_LV100m\_10km.tif"}\NormalTok{)}


\CommentTok{\# simple landscape {-}{-}{-}{-}}
\NormalTok{simple\_landscape}\OtherTok{=}\FunctionTok{rast}\NormalTok{(}\StringTok{"RasterGrids\_10m/2024/Ainava\_vienk\_mask.tif"}\NormalTok{)}

\CommentTok{\# mvr {-}{-}{-}{-}}
\NormalTok{mvr}\OtherTok{=}\FunctionTok{st\_read\_parquet}\NormalTok{(}\StringTok{"./Geodata/2024/MVR/nogabali\_2024janv.parquet"}\NormalTok{)}
\NormalTok{mvr}\SpecialCharTok{$}\NormalTok{yes}\OtherTok{=}\DecValTok{1}

\CommentTok{\# clear cut mask {-}{-}{-}{-}}
\NormalTok{izcirtumi}\OtherTok{=}\NormalTok{mvr }\SpecialCharTok{\%\textgreater{}\%} 
 \FunctionTok{filter}\NormalTok{(zkat }\SpecialCharTok{\%in\%} \FunctionTok{c}\NormalTok{(}\StringTok{"12"}\NormalTok{,}\StringTok{"14"}\NormalTok{)) }\SpecialCharTok{\%\textgreater{}\%} 
\NormalTok{ dplyr}\SpecialCharTok{::}\FunctionTok{select}\NormalTok{(yes)}
\NormalTok{r\_izcirtumi\_mvr}\OtherTok{=}\FunctionTok{fasterize}\NormalTok{(izcirtumi,rastrs10,}\AttributeTok{field=}\StringTok{"yes"}\NormalTok{)}
\NormalTok{t\_izcirtumi\_mvr}\OtherTok{=}\FunctionTok{rast}\NormalTok{(r\_izcirtumi\_mvr)}
\FunctionTok{plot}\NormalTok{(t\_izcirtumi\_mvr)}

\NormalTok{tcl}\OtherTok{=}\FunctionTok{rast}\NormalTok{(}\StringTok{"./Geodata/2024/Trees/GFW/TreeCoverLoss\_v1\_12.tif"}\NormalTok{)}
\NormalTok{tcl2}\OtherTok{=}\FunctionTok{ifel}\NormalTok{(tcl}\SpecialCharTok{\textless{}}\DecValTok{20}\NormalTok{,}\DecValTok{0}\NormalTok{,}\DecValTok{1}\NormalTok{)}
\NormalTok{tclX}\OtherTok{=}\FunctionTok{cover}\NormalTok{(tcl2,nulls10)}
\FunctionTok{plot}\NormalTok{(tclX)}

\NormalTok{clearcut\_mask}\OtherTok{=}\FunctionTok{cover}\NormalTok{(t\_izcirtumi\_mvr,tclX,}
          \AttributeTok{filename=}\StringTok{"./RasterGrids\_10m/2024/Mask\_clearcuts.tif"}\NormalTok{,}
          \AttributeTok{overwrite=}\ConstantTok{TRUE}\NormalTok{)}
\FunctionTok{plot}\NormalTok{(clearcut\_mask)}

\FunctionTok{rm}\NormalTok{(izcirtumi)}
\FunctionTok{rm}\NormalTok{(r\_izcirtumi\_mvr)}
\FunctionTok{rm}\NormalTok{(t\_izcirtumi\_mvr)}
\FunctionTok{rm}\NormalTok{(tcl)}
\FunctionTok{rm}\NormalTok{(tcl2)}
\FunctionTok{rm}\NormalTok{(tclX)}

\CommentTok{\# ForestsQuant\_TimeSinceDisturbance{-}average\_cell.tif    egv\_293 {-}{-}{-}{-}}
\NormalTok{nogabali}\OtherTok{=}\NormalTok{mvr }\SpecialCharTok{\%\textgreater{}\%} 
 \FunctionTok{mutate}\NormalTok{(}\AttributeTok{new\_PDG=}\FunctionTok{ifelse}\NormalTok{(p\_darbg}\SpecialCharTok{\textgreater{}}\DecValTok{2024}\NormalTok{,}\ConstantTok{NA}\NormalTok{,}
            \FunctionTok{ifelse}\NormalTok{(p\_darbv }\SpecialCharTok{\%in\%} \FunctionTok{c}\NormalTok{(}\StringTok{"1"}\NormalTok{,}\StringTok{"4"}\NormalTok{,}\StringTok{"5"}\NormalTok{,}\StringTok{"6"}\NormalTok{,}\StringTok{"7"}\NormalTok{,}\StringTok{"10"}\NormalTok{,}\StringTok{"11"}\NormalTok{),p\_darbg,}\ConstantTok{NA}\NormalTok{)),}
     \AttributeTok{new\_PDG2=}\FunctionTok{ifelse}\NormalTok{(new\_PDG}\SpecialCharTok{\textless{}}\DecValTok{1500}\NormalTok{,}\ConstantTok{NA}\NormalTok{,new\_PDG),}
     \AttributeTok{new\_PCG=}\FunctionTok{ifelse}\NormalTok{(p\_cirg}\SpecialCharTok{\textgreater{}}\DecValTok{2024}\NormalTok{,}\ConstantTok{NA}\NormalTok{,p\_cirg),}
     \AttributeTok{new\_PCG2=}\FunctionTok{ifelse}\NormalTok{(new\_PCG}\SpecialCharTok{\textless{}}\DecValTok{1500}\NormalTok{,}\ConstantTok{NA}\NormalTok{,new\_PCG),}
     \AttributeTok{vecumam=}\FunctionTok{ifelse}\NormalTok{(a10}\SpecialCharTok{==}\DecValTok{0}\NormalTok{,}\ConstantTok{NA}\NormalTok{,a10),}
     \AttributeTok{new\_PCG3=}\DecValTok{2024}\SpecialCharTok{{-}}\NormalTok{new\_PCG2,}
     \AttributeTok{new\_PDG3=}\DecValTok{2024}\SpecialCharTok{{-}}\NormalTok{new\_PDG2) }\SpecialCharTok{\%\textgreater{}\%} 
 \FunctionTok{rowwise}\NormalTok{() }\SpecialCharTok{\%\textgreater{}\%} 
 \FunctionTok{mutate}\NormalTok{(}\AttributeTok{Laikam=}\FunctionTok{min}\NormalTok{(}\FunctionTok{c}\NormalTok{(vecumam,new\_PDG3,new\_PCG3),}\AttributeTok{na.rm=}\ConstantTok{TRUE}\NormalTok{)) }\SpecialCharTok{\%\textgreater{}\%} 
 \FunctionTok{ungroup}\NormalTok{() }\SpecialCharTok{\%\textgreater{}\%} 
 \FunctionTok{mutate}\NormalTok{(}\AttributeTok{KopsTraucejuma=}\FunctionTok{ifelse}\NormalTok{(}\FunctionTok{is.infinite}\NormalTok{(Laikam),}\ConstantTok{NA}\NormalTok{,Laikam)) }\SpecialCharTok{\%\textgreater{}\%} 
 \FunctionTok{mutate}\NormalTok{(}\AttributeTok{KopsTraucejuma2=}\FunctionTok{ifelse}\NormalTok{(KopsTraucejuma}\SpecialCharTok{\textgreater{}}\DecValTok{200}\NormalTok{,}\DecValTok{200}\NormalTok{,KopsTraucejuma)) }\SpecialCharTok{\%\textgreater{}\%} 
 \FunctionTok{filter}\NormalTok{(}\SpecialCharTok{!}\FunctionTok{is.na}\NormalTok{(KopsTraucejuma2))}

\FunctionTok{par}\NormalTok{(}\AttributeTok{mfrow=}\FunctionTok{c}\NormalTok{(}\DecValTok{1}\NormalTok{,}\DecValTok{2}\NormalTok{))}
\FunctionTok{options}\NormalTok{(}\AttributeTok{scipen=}\DecValTok{999}\NormalTok{)}
\FunctionTok{hist}\NormalTok{(nogabali}\SpecialCharTok{$}\NormalTok{KopsTraucejuma,}\AttributeTok{main=}\StringTok{"Original"}\NormalTok{,}\AttributeTok{xlab=}\StringTok{"Time since disturbance"}\NormalTok{)}
\FunctionTok{hist}\NormalTok{(nogabali}\SpecialCharTok{$}\NormalTok{KopsTraucejuma2,}\AttributeTok{main=}\StringTok{"Limited"}\NormalTok{,}\AttributeTok{xlab=}\StringTok{"Time since disturbance"}\NormalTok{)}
\FunctionTok{par}\NormalTok{(}\AttributeTok{mfrow=}\FunctionTok{c}\NormalTok{(}\DecValTok{1}\NormalTok{,}\DecValTok{1}\NormalTok{))}
\FunctionTok{options}\NormalTok{(}\AttributeTok{scipen=}\DecValTok{0}\NormalTok{)}

\NormalTok{mvr\_trauclaiks}\OtherTok{=}\NormalTok{fasterize}\SpecialCharTok{::}\FunctionTok{fasterize}\NormalTok{(nogabali,rastrs10,}\AttributeTok{field=}\StringTok{"KopsTraucejuma2"}\NormalTok{,}\AttributeTok{fun =} \StringTok{"min"}\NormalTok{)}
\NormalTok{t\_MVRtrauclaiks}\OtherTok{=}\FunctionTok{rast}\NormalTok{(mvr\_trauclaiks)}
\FunctionTok{plot}\NormalTok{(t\_MVRtrauclaiks)}

\NormalTok{gfw}\OtherTok{=}\FunctionTok{rast}\NormalTok{(}\StringTok{"./Geodata/2024/Trees/GFW/TreeCoverLoss\_v1\_12.tif"}\NormalTok{)}
\FunctionTok{plot}\NormalTok{(gfw)}
\NormalTok{gfw2}\OtherTok{=}\FunctionTok{ifel}\NormalTok{(gfw}\SpecialCharTok{\textgreater{}=}\DecValTok{0}\NormalTok{,}\DecValTok{24}\SpecialCharTok{{-}}\NormalTok{gfw,}\ConstantTok{NA}\NormalTok{)}
\FunctionTok{plot}\NormalTok{(gfw2)}


\CommentTok{\# No ainavas:}
\DocumentationTok{\#\# Mežaudzes un koki = 50}
\DocumentationTok{\#\# Krūmāji un parki = 5}
\DocumentationTok{\#\# pārējais = 0}
\NormalTok{aizpildisanai}\OtherTok{=}\FunctionTok{ifel}\NormalTok{(simple\_landscape}\SpecialCharTok{==}\DecValTok{630}\SpecialCharTok{|}\NormalTok{simple\_landscape}\SpecialCharTok{==}\DecValTok{640}\NormalTok{,}\DecValTok{50}\NormalTok{,}
          \FunctionTok{ifel}\NormalTok{(simple\_landscape}\SpecialCharTok{==}\DecValTok{620}\NormalTok{,}\DecValTok{5}\NormalTok{,}\DecValTok{0}\NormalTok{))}
\FunctionTok{freq}\NormalTok{(aizpildisanai)}

\NormalTok{trauclaiks1}\OtherTok{=}\NormalTok{terra}\SpecialCharTok{::}\FunctionTok{app}\NormalTok{(}\FunctionTok{c}\NormalTok{(gfw2,t\_MVRtrauclaiks),}\AttributeTok{fun=}\StringTok{"min"}\NormalTok{,}\AttributeTok{na.rm=}\ConstantTok{TRUE}\NormalTok{)}
\FunctionTok{plot}\NormalTok{(trauclaiks1)}
\NormalTok{trauclaiks2}\OtherTok{=}\FunctionTok{cover}\NormalTok{(trauclaiks1,aizpildisanai)}
\FunctionTok{plot}\NormalTok{(trauclaiks2)}

\NormalTok{i2e\_rez}\OtherTok{=}\FunctionTok{input2egv}\NormalTok{(}\AttributeTok{input=}\NormalTok{trauclaiks2,}
         \AttributeTok{egv\_template =} \StringTok{"./Templates/TemplateRasters/LV100m\_10km.tif"}\NormalTok{,}
         \AttributeTok{summary\_function =} \StringTok{"average"}\NormalTok{,}
         \AttributeTok{missing\_job =} \StringTok{"FillOutput"}\NormalTok{,}
         \AttributeTok{idw\_weight =} \DecValTok{2}\NormalTok{,}
         \AttributeTok{outlocation =} \StringTok{"./RasterGrids\_100m/2024/RAW/"}\NormalTok{,}
         \AttributeTok{outfilename =} \StringTok{"ForestsQuant\_TimeSinceDisturbance{-}average\_cell.tif"}\NormalTok{,}
         \AttributeTok{layername =} \StringTok{"egv\_293"}\NormalTok{,}
         \AttributeTok{plot\_final=}\ConstantTok{TRUE}\NormalTok{)}
\NormalTok{i2e\_rez}
\FunctionTok{rm}\NormalTok{(nogabali)}
\FunctionTok{rm}\NormalTok{(mvr\_trauclaiks)}
\FunctionTok{rm}\NormalTok{(t\_MVRtrauclaiks)}
\FunctionTok{rm}\NormalTok{(gfw)}
\FunctionTok{rm}\NormalTok{(gfw2)}
\FunctionTok{rm}\NormalTok{(aizpildisanai)}
\FunctionTok{rm}\NormalTok{(trauclaiks1)}
\FunctionTok{rm}\NormalTok{(trauclaiks2)}
\FunctionTok{rm}\NormalTok{(i2e\_rez)}

\CommentTok{\# standardisation {-}{-}{-}{-}}
\ControlFlowTok{if}\NormalTok{(}\SpecialCharTok{!}\FunctionTok{require}\NormalTok{(terra)) \{}\FunctionTok{install.packages}\NormalTok{(}\StringTok{"terra"}\NormalTok{); }\FunctionTok{require}\NormalTok{(terra)\}}
\ControlFlowTok{if}\NormalTok{(}\SpecialCharTok{!}\FunctionTok{require}\NormalTok{(tidyverse)) \{}\FunctionTok{install.packages}\NormalTok{(}\StringTok{"tidyverse"}\NormalTok{); }\FunctionTok{require}\NormalTok{(tidyverse)\}}

\NormalTok{nosaukums}\OtherTok{=}\StringTok{"ForestsQuant\_TimeSinceDisturbance{-}average\_cell.tif"}
\NormalTok{ielasisanas\_cels}\OtherTok{=}\FunctionTok{paste0}\NormalTok{(}\StringTok{"./RasterGrids\_100m/2024/RAW/"}\NormalTok{,nosaukums)}
\NormalTok{saglabasanas\_cels}\OtherTok{=}\FunctionTok{paste0}\NormalTok{(}\StringTok{"./RasterGrids\_100m/2024/Scaled/"}\NormalTok{,nosaukums)}
\NormalTok{slanis}\OtherTok{=}\FunctionTok{rast}\NormalTok{(ielasisanas\_cels)}
\NormalTok{videjais}\OtherTok{=}\FunctionTok{global}\NormalTok{(slanis,}\AttributeTok{fun=}\StringTok{"mean"}\NormalTok{,}\AttributeTok{na.rm=}\ConstantTok{TRUE}\NormalTok{)}
\NormalTok{centrets}\OtherTok{=}\NormalTok{slanis}\SpecialCharTok{{-}}\NormalTok{videjais[,}\DecValTok{1}\NormalTok{]}
\NormalTok{standartnovirze}\OtherTok{=}\NormalTok{terra}\SpecialCharTok{::}\FunctionTok{global}\NormalTok{(centrets,}\AttributeTok{fun=}\StringTok{"rms"}\NormalTok{,}\AttributeTok{na.rm=}\ConstantTok{TRUE}\NormalTok{)}
\NormalTok{merogots}\OtherTok{=}\NormalTok{centrets}\SpecialCharTok{/}\NormalTok{standartnovirze[,}\DecValTok{1}\NormalTok{]}
\FunctionTok{writeRaster}\NormalTok{(merogots,}
      \AttributeTok{filename=}\NormalTok{saglabasanas\_cels,}
      \AttributeTok{overwrite=}\ConstantTok{TRUE}\NormalTok{)}
\end{Highlighting}
\end{Shaded}

\section{ForestsQuant\_VolumeAspen-sum\_cell}\label{ch06.294}

\textbf{filename:} \texttt{ForestsQuant\_VolumeAspen-sum\_cell.tif}

\textbf{layername:} \texttt{egv\_294}

\textbf{English name:} Timber volume of Aspens, Poplars within the analysis cell (1
ha)

\textbf{Latvian name:} Apšu, papeļu krāja analīzes šūnā (1 ha)

\textbf{Procedure:} Most EGVs describing forests are spatially restricted to areas outside
of clearcuts and dead stands. This mask is created using a combination of
the \hyperref[Ch04.01]{State Forest Service's
State Forest Registry} land category 12 and 14, and \hyperref[Ch04.09]{The
Global Forest Watch} pixels classified as lost tree canopy cover since
2020 (raster layer matching input, presence = 1, absence = 0).

This EGV is prepared based on the information of timber volume of aspen
(species codes: 8, 19, 68; see tree species codes in \hyperref[Ch01]{Terminology and
acronyms}) in the inventoried forest stands - \hyperref[Ch04.01]{State Forest Service's
State Forest Registry}. This attribute has some extreme
values. We chose to limit them to the nearest integer showing only minimal
accumulation in histogram.

\includegraphics[width=0.8\linewidth]{./Figures/Histogramms/hist_egv294}

Resulting values at polygon geometries are rasterised with the workflow
\texttt{egvtools::polygon2input()}, restricting to pixels outside the clearcut mask. No
background values are assigned during rasterisation. The resulting layer is
then aggregated to EGV resolution using the workflow \texttt{egvtools::input2egv()} by calculating
sum of pixel values. After the aggregation, cells with no forest information
are filled with value 0. Finally, the layer is standardised by subtracting
the arithmetic mean and dividing by the root mean squared error.

\begin{Shaded}
\begin{Highlighting}[]
\CommentTok{\# libs {-}{-}{-}{-}}
\ControlFlowTok{if}\NormalTok{(}\SpecialCharTok{!}\FunctionTok{require}\NormalTok{(egvtools)) \{remotes}\SpecialCharTok{::}\FunctionTok{install\_github}\NormalTok{(}\StringTok{"aavotins/egvtools"}\NormalTok{); }\FunctionTok{require}\NormalTok{(egvtools)\}}
\ControlFlowTok{if}\NormalTok{(}\SpecialCharTok{!}\FunctionTok{require}\NormalTok{(terra)) \{}\FunctionTok{install.packages}\NormalTok{(}\StringTok{"terra"}\NormalTok{); }\FunctionTok{require}\NormalTok{(terra)\}}
\ControlFlowTok{if}\NormalTok{(}\SpecialCharTok{!}\FunctionTok{require}\NormalTok{(sf)) \{}\FunctionTok{install.packages}\NormalTok{(}\StringTok{"sf"}\NormalTok{); }\FunctionTok{require}\NormalTok{(sf)\}}
\ControlFlowTok{if}\NormalTok{(}\SpecialCharTok{!}\FunctionTok{require}\NormalTok{(tidyverse)) \{}\FunctionTok{install.packages}\NormalTok{(}\StringTok{"tidyverse"}\NormalTok{); }\FunctionTok{require}\NormalTok{(tidyverse)\}}
\ControlFlowTok{if}\NormalTok{(}\SpecialCharTok{!}\FunctionTok{require}\NormalTok{(sfarrow)) \{}\FunctionTok{install.packages}\NormalTok{(}\StringTok{"sfarrow"}\NormalTok{); }\FunctionTok{require}\NormalTok{(sfarrow)\}}
\ControlFlowTok{if}\NormalTok{(}\SpecialCharTok{!}\FunctionTok{require}\NormalTok{(readxl)) \{}\FunctionTok{install.packages}\NormalTok{(}\StringTok{"readxl"}\NormalTok{); }\FunctionTok{require}\NormalTok{(readxl)\}}
\ControlFlowTok{if}\NormalTok{(}\SpecialCharTok{!}\FunctionTok{require}\NormalTok{(raster)) \{}\FunctionTok{install.packages}\NormalTok{(}\StringTok{"raster"}\NormalTok{); }\FunctionTok{require}\NormalTok{(raster)\}}
\ControlFlowTok{if}\NormalTok{(}\SpecialCharTok{!}\FunctionTok{require}\NormalTok{(fasterize)) \{}\FunctionTok{install.packages}\NormalTok{(}\StringTok{"fasterize"}\NormalTok{); }\FunctionTok{require}\NormalTok{(fasterize)\}}

\CommentTok{\# templates {-}{-}{-}{-}}
\NormalTok{template100}\OtherTok{=}\FunctionTok{rast}\NormalTok{(}\StringTok{"./Templates/TemplateRasters/LV100m\_10km.tif"}\NormalTok{)}
\NormalTok{template10}\OtherTok{=}\FunctionTok{rast}\NormalTok{(}\StringTok{"./Templates/TemplateRasters/LV10m\_10km.tif"}\NormalTok{)}
\NormalTok{rastrs10}\OtherTok{=}\FunctionTok{raster}\NormalTok{(template10)}

\NormalTok{nulls10}\OtherTok{=}\FunctionTok{rast}\NormalTok{(}\StringTok{"./Templates/TemplateRasters/nulls\_LV10m\_10km.tif"}\NormalTok{)}
\NormalTok{nulls100}\OtherTok{=}\FunctionTok{rast}\NormalTok{(}\StringTok{"./Templates/TemplateRasters/nulls\_LV100m\_10km.tif"}\NormalTok{)}


\CommentTok{\# simple landscape {-}{-}{-}{-}}
\NormalTok{simple\_landscape}\OtherTok{=}\FunctionTok{rast}\NormalTok{(}\StringTok{"RasterGrids\_10m/2024/Ainava\_vienk\_mask.tif"}\NormalTok{)}

\CommentTok{\# mvr {-}{-}{-}{-}}
\NormalTok{mvr}\OtherTok{=}\FunctionTok{st\_read\_parquet}\NormalTok{(}\StringTok{"./Geodata/2024/MVR/nogabali\_2024janv.parquet"}\NormalTok{)}
\NormalTok{mvr}\SpecialCharTok{$}\NormalTok{yes}\OtherTok{=}\DecValTok{1}

\CommentTok{\# clear cut mask {-}{-}{-}{-}}
\NormalTok{izcirtumi}\OtherTok{=}\NormalTok{mvr }\SpecialCharTok{\%\textgreater{}\%} 
 \FunctionTok{filter}\NormalTok{(zkat }\SpecialCharTok{\%in\%} \FunctionTok{c}\NormalTok{(}\StringTok{"12"}\NormalTok{,}\StringTok{"14"}\NormalTok{)) }\SpecialCharTok{\%\textgreater{}\%} 
\NormalTok{ dplyr}\SpecialCharTok{::}\FunctionTok{select}\NormalTok{(yes)}
\NormalTok{r\_izcirtumi\_mvr}\OtherTok{=}\FunctionTok{fasterize}\NormalTok{(izcirtumi,rastrs10,}\AttributeTok{field=}\StringTok{"yes"}\NormalTok{)}
\NormalTok{t\_izcirtumi\_mvr}\OtherTok{=}\FunctionTok{rast}\NormalTok{(r\_izcirtumi\_mvr)}
\FunctionTok{plot}\NormalTok{(t\_izcirtumi\_mvr)}

\NormalTok{tcl}\OtherTok{=}\FunctionTok{rast}\NormalTok{(}\StringTok{"./Geodata/2024/Trees/GFW/TreeCoverLoss\_v1\_12.tif"}\NormalTok{)}
\NormalTok{tcl2}\OtherTok{=}\FunctionTok{ifel}\NormalTok{(tcl}\SpecialCharTok{\textless{}}\DecValTok{20}\NormalTok{,}\DecValTok{0}\NormalTok{,}\DecValTok{1}\NormalTok{)}
\NormalTok{tclX}\OtherTok{=}\FunctionTok{cover}\NormalTok{(tcl2,nulls10)}
\FunctionTok{plot}\NormalTok{(tclX)}

\NormalTok{clearcut\_mask}\OtherTok{=}\FunctionTok{cover}\NormalTok{(t\_izcirtumi\_mvr,tclX,}
          \AttributeTok{filename=}\StringTok{"./RasterGrids\_10m/2024/Mask\_clearcuts.tif"}\NormalTok{,}
          \AttributeTok{overwrite=}\ConstantTok{TRUE}\NormalTok{)}
\FunctionTok{plot}\NormalTok{(clearcut\_mask)}

\FunctionTok{rm}\NormalTok{(izcirtumi)}
\FunctionTok{rm}\NormalTok{(r\_izcirtumi\_mvr)}
\FunctionTok{rm}\NormalTok{(t\_izcirtumi\_mvr)}
\FunctionTok{rm}\NormalTok{(tcl)}
\FunctionTok{rm}\NormalTok{(tcl2)}
\FunctionTok{rm}\NormalTok{(tclX)}

\CommentTok{\# ForestsQuant\_VolumeAspen{-}sum\_cell.tif egv\_294 {-}{-}{-}{-}}

\NormalTok{apses}\OtherTok{=}\FunctionTok{c}\NormalTok{(}\StringTok{"8"}\NormalTok{,}\StringTok{"19"}\NormalTok{,}\StringTok{"68"}\NormalTok{)}
\NormalTok{nogabali}\OtherTok{=}\NormalTok{mvr }\SpecialCharTok{\%\textgreater{}\%} 
 \FunctionTok{mutate}\NormalTok{(}\AttributeTok{ApsuKraja=}\FunctionTok{ifelse}\NormalTok{(s10 }\SpecialCharTok{\%in\%}\NormalTok{ apses, v10, }\DecValTok{0}\NormalTok{)}\SpecialCharTok{+}\FunctionTok{ifelse}\NormalTok{(s11 }\SpecialCharTok{\%in\%}\NormalTok{ apses,v11,}\DecValTok{0}\NormalTok{)}\SpecialCharTok{+}
      \FunctionTok{ifelse}\NormalTok{(s12 }\SpecialCharTok{\%in\%}\NormalTok{ apses, v12,}\DecValTok{0}\NormalTok{)}\SpecialCharTok{+}\FunctionTok{ifelse}\NormalTok{(s13 }\SpecialCharTok{\%in\%}\NormalTok{ apses,v13,}\DecValTok{0}\NormalTok{)}\SpecialCharTok{+}
      \FunctionTok{ifelse}\NormalTok{(s14 }\SpecialCharTok{\%in\%}\NormalTok{ apses, v14,}\DecValTok{0}\NormalTok{)) }\SpecialCharTok{\%\textgreater{}\%} 
 \FunctionTok{mutate}\NormalTok{(}\AttributeTok{ApsuKraja2=}\NormalTok{ApsuKraja}\SpecialCharTok{/}\DecValTok{10000}\SpecialCharTok{*}\DecValTok{10}\SpecialCharTok{*}\DecValTok{10}\NormalTok{) }\SpecialCharTok{\%\textgreater{}\%} 
 \FunctionTok{mutate}\NormalTok{(}\AttributeTok{ApsuKraja3=}\FunctionTok{ifelse}\NormalTok{(ApsuKraja2}\SpecialCharTok{\textgreater{}}\DecValTok{5}\NormalTok{,}\DecValTok{5}\NormalTok{,ApsuKraja2)) }\SpecialCharTok{\%\textgreater{}\%} 
 \FunctionTok{filter}\NormalTok{(}\SpecialCharTok{!}\FunctionTok{is.na}\NormalTok{(ApsuKraja2))}


\FunctionTok{par}\NormalTok{(}\AttributeTok{mfrow=}\FunctionTok{c}\NormalTok{(}\DecValTok{1}\NormalTok{,}\DecValTok{2}\NormalTok{))}
\FunctionTok{options}\NormalTok{(}\AttributeTok{scipen=}\DecValTok{999}\NormalTok{)}
\FunctionTok{hist}\NormalTok{(nogabali}\SpecialCharTok{$}\NormalTok{ApsuKraja2,}\AttributeTok{main=}\StringTok{"Original"}\NormalTok{,}\AttributeTok{xlab=}\StringTok{"Aspen volume"}\NormalTok{)}
\FunctionTok{hist}\NormalTok{(nogabali}\SpecialCharTok{$}\NormalTok{ApsuKraja3,}\AttributeTok{main=}\StringTok{"Limited"}\NormalTok{,}\AttributeTok{xlab=}\StringTok{"Aspen volume"}\NormalTok{)}
\FunctionTok{par}\NormalTok{(}\AttributeTok{mfrow=}\FunctionTok{c}\NormalTok{(}\DecValTok{1}\NormalTok{,}\DecValTok{1}\NormalTok{))}
\FunctionTok{options}\NormalTok{(}\AttributeTok{scipen=}\DecValTok{0}\NormalTok{)}

\NormalTok{p2i\_rez}\OtherTok{=}\FunctionTok{polygon2input}\NormalTok{(}\AttributeTok{vector\_data=}\NormalTok{nogabali,}
           \AttributeTok{template\_path =} \StringTok{"./Templates/TemplateRasters/LV10m\_10km.tif"}\NormalTok{,}
           \AttributeTok{out\_path =} \StringTok{"./RasterGrids\_10m/2024/"}\NormalTok{,}
           \AttributeTok{file\_name =} \StringTok{"ForestsQuant\_VolumeAspen.tif"}\NormalTok{,}
           \AttributeTok{value\_field =} \StringTok{"ApsuKraja3"}\NormalTok{,}
           \AttributeTok{fun=}\StringTok{"max"}\NormalTok{,}
           \AttributeTok{prepare=}\ConstantTok{FALSE}\NormalTok{,}
           \AttributeTok{restrict\_to =}\NormalTok{ clearcut\_mask,}
           \AttributeTok{restrict\_values =} \DecValTok{0}\NormalTok{,}
           \AttributeTok{plot\_result=}\ConstantTok{TRUE}\NormalTok{,}
           \AttributeTok{overwrite=}\ConstantTok{TRUE}\NormalTok{)}
\NormalTok{p2i\_rez}
\NormalTok{i2e\_rez}\OtherTok{=}\FunctionTok{input2egv}\NormalTok{(}\AttributeTok{input=}\StringTok{"./RasterGrids\_10m/2024/ForestsQuant\_VolumeAspen.tif"}\NormalTok{,}
         \AttributeTok{egv\_template =} \StringTok{"./Templates/TemplateRasters/LV100m\_10km.tif"}\NormalTok{,}
         \AttributeTok{summary\_function =} \StringTok{"sum"}\NormalTok{,}
         \AttributeTok{missing\_job =} \StringTok{"CoverOutput"}\NormalTok{,}
         \AttributeTok{output\_bg =} \StringTok{"./Templates/TemplateRasters/nulls\_LV100m\_10km.tif"}\NormalTok{,}
         \AttributeTok{outlocation =} \StringTok{"./RasterGrids\_100m/2024/RAW/"}\NormalTok{,}
         \AttributeTok{outfilename =} \StringTok{"ForestsQuant\_VolumeAspen{-}sum\_cell.tif"}\NormalTok{,}
         \AttributeTok{layername =} \StringTok{"egv\_294"}\NormalTok{,}
         \AttributeTok{plot\_final=}\ConstantTok{TRUE}\NormalTok{)}
\NormalTok{i2e\_rez}
\FunctionTok{rm}\NormalTok{(p2i\_rez)}
\FunctionTok{rm}\NormalTok{(nogabali)}
\FunctionTok{rm}\NormalTok{(apses)}
\FunctionTok{rm}\NormalTok{(i2e\_rez)}
\FunctionTok{unlink}\NormalTok{(}\StringTok{"./RasterGrids\_10m/2024/ForestsQuant\_VolumeAspen.tif"}\NormalTok{)}

\CommentTok{\# standardisation {-}{-}{-}{-}}
\ControlFlowTok{if}\NormalTok{(}\SpecialCharTok{!}\FunctionTok{require}\NormalTok{(terra)) \{}\FunctionTok{install.packages}\NormalTok{(}\StringTok{"terra"}\NormalTok{); }\FunctionTok{require}\NormalTok{(terra)\}}
\ControlFlowTok{if}\NormalTok{(}\SpecialCharTok{!}\FunctionTok{require}\NormalTok{(tidyverse)) \{}\FunctionTok{install.packages}\NormalTok{(}\StringTok{"tidyverse"}\NormalTok{); }\FunctionTok{require}\NormalTok{(tidyverse)\}}

\NormalTok{nosaukums}\OtherTok{=}\StringTok{"ForestsQuant\_VolumeAspen{-}sum\_cell.tif"}
\NormalTok{ielasisanas\_cels}\OtherTok{=}\FunctionTok{paste0}\NormalTok{(}\StringTok{"./RasterGrids\_100m/2024/RAW/"}\NormalTok{,nosaukums)}
\NormalTok{saglabasanas\_cels}\OtherTok{=}\FunctionTok{paste0}\NormalTok{(}\StringTok{"./RasterGrids\_100m/2024/Scaled/"}\NormalTok{,nosaukums)}
\NormalTok{slanis}\OtherTok{=}\FunctionTok{rast}\NormalTok{(ielasisanas\_cels)}
\NormalTok{videjais}\OtherTok{=}\FunctionTok{global}\NormalTok{(slanis,}\AttributeTok{fun=}\StringTok{"mean"}\NormalTok{,}\AttributeTok{na.rm=}\ConstantTok{TRUE}\NormalTok{)}
\NormalTok{centrets}\OtherTok{=}\NormalTok{slanis}\SpecialCharTok{{-}}\NormalTok{videjais[,}\DecValTok{1}\NormalTok{]}
\NormalTok{standartnovirze}\OtherTok{=}\NormalTok{terra}\SpecialCharTok{::}\FunctionTok{global}\NormalTok{(centrets,}\AttributeTok{fun=}\StringTok{"rms"}\NormalTok{,}\AttributeTok{na.rm=}\ConstantTok{TRUE}\NormalTok{)}
\NormalTok{merogots}\OtherTok{=}\NormalTok{centrets}\SpecialCharTok{/}\NormalTok{standartnovirze[,}\DecValTok{1}\NormalTok{]}
\FunctionTok{writeRaster}\NormalTok{(merogots,}
      \AttributeTok{filename=}\NormalTok{saglabasanas\_cels,}
      \AttributeTok{overwrite=}\ConstantTok{TRUE}\NormalTok{)}
\end{Highlighting}
\end{Shaded}

\section{ForestsQuant\_VolumeBirch-sum\_cell}\label{ch06.295}

\textbf{filename:} \texttt{ForestsQuant\_VolumeBirch-sum\_cell.tif}

\textbf{layername:} \texttt{egv\_295}

\textbf{English name:} Timber volume of Birches within the analysis cell (1 ha)

\textbf{Latvian name:} Bērzu krāja analīzes šūnā (1 ha)

\textbf{Procedure:} Most EGVs describing forests are spatially restricted to areas outside
of clearcuts and dead stands. This mask is created using a combination of
the \hyperref[Ch04.01]{State Forest Service's
State Forest Registry} land category 12 and 14, and \hyperref[Ch04.09]{The
Global Forest Watch} pixels classified as lost tree canopy cover since
2020 (raster layer matching input, presence = 1, absence = 0).

This EGV is prepared based on the information of timber volume of birch
(species code: 4; see tree species codes in \hyperref[Ch01]{Terminology and acronyms})
in the inventoried forest stands - \hyperref[Ch04.01]{State Forest Service's State Forest
Registry}. This attribute has some extreme
values. We chose to limit them to the nearest integer showing only minimal
accumulation in histogram.

\includegraphics[width=0.8\linewidth]{./Figures/Histogramms/hist_egv295}

Resulting values at polygon geometries are rasterised with the workflow
\texttt{egvtools::polygon2input()}, restricting to pixels outside the clearcut mask. No
background values are assigned during rasterisation. The resulting layer is
then aggregated to EGV resolution using the workflow \texttt{egvtools::input2egv()} by calculating
sum of pixel values. After the aggregation, cells with no forest information
are filled with value 0. Finally, the layer is standardised by subtracting
the arithmetic mean and dividing by the root mean squared error.

\begin{Shaded}
\begin{Highlighting}[]
\CommentTok{\# libs {-}{-}{-}{-}}
\ControlFlowTok{if}\NormalTok{(}\SpecialCharTok{!}\FunctionTok{require}\NormalTok{(egvtools)) \{remotes}\SpecialCharTok{::}\FunctionTok{install\_github}\NormalTok{(}\StringTok{"aavotins/egvtools"}\NormalTok{); }\FunctionTok{require}\NormalTok{(egvtools)\}}
\ControlFlowTok{if}\NormalTok{(}\SpecialCharTok{!}\FunctionTok{require}\NormalTok{(terra)) \{}\FunctionTok{install.packages}\NormalTok{(}\StringTok{"terra"}\NormalTok{); }\FunctionTok{require}\NormalTok{(terra)\}}
\ControlFlowTok{if}\NormalTok{(}\SpecialCharTok{!}\FunctionTok{require}\NormalTok{(sf)) \{}\FunctionTok{install.packages}\NormalTok{(}\StringTok{"sf"}\NormalTok{); }\FunctionTok{require}\NormalTok{(sf)\}}
\ControlFlowTok{if}\NormalTok{(}\SpecialCharTok{!}\FunctionTok{require}\NormalTok{(tidyverse)) \{}\FunctionTok{install.packages}\NormalTok{(}\StringTok{"tidyverse"}\NormalTok{); }\FunctionTok{require}\NormalTok{(tidyverse)\}}
\ControlFlowTok{if}\NormalTok{(}\SpecialCharTok{!}\FunctionTok{require}\NormalTok{(sfarrow)) \{}\FunctionTok{install.packages}\NormalTok{(}\StringTok{"sfarrow"}\NormalTok{); }\FunctionTok{require}\NormalTok{(sfarrow)\}}
\ControlFlowTok{if}\NormalTok{(}\SpecialCharTok{!}\FunctionTok{require}\NormalTok{(readxl)) \{}\FunctionTok{install.packages}\NormalTok{(}\StringTok{"readxl"}\NormalTok{); }\FunctionTok{require}\NormalTok{(readxl)\}}
\ControlFlowTok{if}\NormalTok{(}\SpecialCharTok{!}\FunctionTok{require}\NormalTok{(raster)) \{}\FunctionTok{install.packages}\NormalTok{(}\StringTok{"raster"}\NormalTok{); }\FunctionTok{require}\NormalTok{(raster)\}}
\ControlFlowTok{if}\NormalTok{(}\SpecialCharTok{!}\FunctionTok{require}\NormalTok{(fasterize)) \{}\FunctionTok{install.packages}\NormalTok{(}\StringTok{"fasterize"}\NormalTok{); }\FunctionTok{require}\NormalTok{(fasterize)\}}

\CommentTok{\# templates {-}{-}{-}{-}}
\NormalTok{template100}\OtherTok{=}\FunctionTok{rast}\NormalTok{(}\StringTok{"./Templates/TemplateRasters/LV100m\_10km.tif"}\NormalTok{)}
\NormalTok{template10}\OtherTok{=}\FunctionTok{rast}\NormalTok{(}\StringTok{"./Templates/TemplateRasters/LV10m\_10km.tif"}\NormalTok{)}
\NormalTok{rastrs10}\OtherTok{=}\FunctionTok{raster}\NormalTok{(template10)}

\NormalTok{nulls10}\OtherTok{=}\FunctionTok{rast}\NormalTok{(}\StringTok{"./Templates/TemplateRasters/nulls\_LV10m\_10km.tif"}\NormalTok{)}
\NormalTok{nulls100}\OtherTok{=}\FunctionTok{rast}\NormalTok{(}\StringTok{"./Templates/TemplateRasters/nulls\_LV100m\_10km.tif"}\NormalTok{)}


\CommentTok{\# simple landscape {-}{-}{-}{-}}
\NormalTok{simple\_landscape}\OtherTok{=}\FunctionTok{rast}\NormalTok{(}\StringTok{"RasterGrids\_10m/2024/Ainava\_vienk\_mask.tif"}\NormalTok{)}

\CommentTok{\# mvr {-}{-}{-}{-}}
\NormalTok{mvr}\OtherTok{=}\FunctionTok{st\_read\_parquet}\NormalTok{(}\StringTok{"./Geodata/2024/MVR/nogabali\_2024janv.parquet"}\NormalTok{)}
\NormalTok{mvr}\SpecialCharTok{$}\NormalTok{yes}\OtherTok{=}\DecValTok{1}

\CommentTok{\# clear cut mask {-}{-}{-}{-}}
\NormalTok{izcirtumi}\OtherTok{=}\NormalTok{mvr }\SpecialCharTok{\%\textgreater{}\%} 
 \FunctionTok{filter}\NormalTok{(zkat }\SpecialCharTok{\%in\%} \FunctionTok{c}\NormalTok{(}\StringTok{"12"}\NormalTok{,}\StringTok{"14"}\NormalTok{)) }\SpecialCharTok{\%\textgreater{}\%} 
\NormalTok{ dplyr}\SpecialCharTok{::}\FunctionTok{select}\NormalTok{(yes)}
\NormalTok{r\_izcirtumi\_mvr}\OtherTok{=}\FunctionTok{fasterize}\NormalTok{(izcirtumi,rastrs10,}\AttributeTok{field=}\StringTok{"yes"}\NormalTok{)}
\NormalTok{t\_izcirtumi\_mvr}\OtherTok{=}\FunctionTok{rast}\NormalTok{(r\_izcirtumi\_mvr)}
\FunctionTok{plot}\NormalTok{(t\_izcirtumi\_mvr)}

\NormalTok{tcl}\OtherTok{=}\FunctionTok{rast}\NormalTok{(}\StringTok{"./Geodata/2024/Trees/GFW/TreeCoverLoss\_v1\_12.tif"}\NormalTok{)}
\NormalTok{tcl2}\OtherTok{=}\FunctionTok{ifel}\NormalTok{(tcl}\SpecialCharTok{\textless{}}\DecValTok{20}\NormalTok{,}\DecValTok{0}\NormalTok{,}\DecValTok{1}\NormalTok{)}
\NormalTok{tclX}\OtherTok{=}\FunctionTok{cover}\NormalTok{(tcl2,nulls10)}
\FunctionTok{plot}\NormalTok{(tclX)}

\NormalTok{clearcut\_mask}\OtherTok{=}\FunctionTok{cover}\NormalTok{(t\_izcirtumi\_mvr,tclX,}
          \AttributeTok{filename=}\StringTok{"./RasterGrids\_10m/2024/Mask\_clearcuts.tif"}\NormalTok{,}
          \AttributeTok{overwrite=}\ConstantTok{TRUE}\NormalTok{)}
\FunctionTok{plot}\NormalTok{(clearcut\_mask)}

\FunctionTok{rm}\NormalTok{(izcirtumi)}
\FunctionTok{rm}\NormalTok{(r\_izcirtumi\_mvr)}
\FunctionTok{rm}\NormalTok{(t\_izcirtumi\_mvr)}
\FunctionTok{rm}\NormalTok{(tcl)}
\FunctionTok{rm}\NormalTok{(tcl2)}
\FunctionTok{rm}\NormalTok{(tclX)}

\CommentTok{\# ForestsQuant\_VolumeBirch{-}sum\_cell.tif egv\_295 {-}{-}{-}{-}}

\NormalTok{berzi}\OtherTok{=}\FunctionTok{c}\NormalTok{(}\StringTok{"4"}\NormalTok{)}
\NormalTok{nogabali}\OtherTok{=}\NormalTok{mvr }\SpecialCharTok{\%\textgreater{}\%} 
 \FunctionTok{mutate}\NormalTok{(}\AttributeTok{BerzuKraja=}\FunctionTok{ifelse}\NormalTok{(s10 }\SpecialCharTok{\%in\%}\NormalTok{ berzi, v10, }\DecValTok{0}\NormalTok{)}\SpecialCharTok{+}\FunctionTok{ifelse}\NormalTok{(s11 }\SpecialCharTok{\%in\%}\NormalTok{ berzi,v11,}\DecValTok{0}\NormalTok{)}\SpecialCharTok{+}
      \FunctionTok{ifelse}\NormalTok{(s12 }\SpecialCharTok{\%in\%}\NormalTok{ berzi, v12,}\DecValTok{0}\NormalTok{)}\SpecialCharTok{+}\FunctionTok{ifelse}\NormalTok{(s13 }\SpecialCharTok{\%in\%}\NormalTok{ berzi,v13,}\DecValTok{0}\NormalTok{)}\SpecialCharTok{+}
      \FunctionTok{ifelse}\NormalTok{(s14 }\SpecialCharTok{\%in\%}\NormalTok{ berzi, v14,}\DecValTok{0}\NormalTok{)) }\SpecialCharTok{\%\textgreater{}\%} 
 \FunctionTok{mutate}\NormalTok{(}\AttributeTok{BerzuKraja2=}\NormalTok{BerzuKraja}\SpecialCharTok{/}\DecValTok{10000}\SpecialCharTok{*}\DecValTok{10}\SpecialCharTok{*}\DecValTok{10}\NormalTok{) }\SpecialCharTok{\%\textgreater{}\%} 
 \FunctionTok{mutate}\NormalTok{(}\AttributeTok{BerzuKraja3=}\FunctionTok{ifelse}\NormalTok{(BerzuKraja2}\SpecialCharTok{\textgreater{}}\DecValTok{5}\NormalTok{,}\DecValTok{5}\NormalTok{,BerzuKraja2)) }\SpecialCharTok{\%\textgreater{}\%} 
 \FunctionTok{filter}\NormalTok{(}\SpecialCharTok{!}\FunctionTok{is.na}\NormalTok{(BerzuKraja2))}

\FunctionTok{par}\NormalTok{(}\AttributeTok{mfrow=}\FunctionTok{c}\NormalTok{(}\DecValTok{1}\NormalTok{,}\DecValTok{2}\NormalTok{))}
\FunctionTok{options}\NormalTok{(}\AttributeTok{scipen=}\DecValTok{999}\NormalTok{)}
\FunctionTok{hist}\NormalTok{(nogabali}\SpecialCharTok{$}\NormalTok{BerzuKraja2,}\AttributeTok{main=}\StringTok{"Original"}\NormalTok{,}\AttributeTok{xlab=}\StringTok{"Birch volume"}\NormalTok{)}
\FunctionTok{hist}\NormalTok{(nogabali}\SpecialCharTok{$}\NormalTok{BerzuKraja3,}\AttributeTok{main=}\StringTok{"Limited"}\NormalTok{,}\AttributeTok{xlab=}\StringTok{"Birch volume"}\NormalTok{)}
\FunctionTok{par}\NormalTok{(}\AttributeTok{mfrow=}\FunctionTok{c}\NormalTok{(}\DecValTok{1}\NormalTok{,}\DecValTok{1}\NormalTok{))}
\FunctionTok{options}\NormalTok{(}\AttributeTok{scipen=}\DecValTok{0}\NormalTok{)}

\NormalTok{p2i\_rez}\OtherTok{=}\FunctionTok{polygon2input}\NormalTok{(}\AttributeTok{vector\_data=}\NormalTok{nogabali,}
           \AttributeTok{template\_path =} \StringTok{"./Templates/TemplateRasters/LV10m\_10km.tif"}\NormalTok{,}
           \AttributeTok{out\_path =} \StringTok{"./RasterGrids\_10m/2024/"}\NormalTok{,}
           \AttributeTok{file\_name =} \StringTok{"ForestsQuant\_VolumeBirch.tif"}\NormalTok{,}
           \AttributeTok{value\_field =} \StringTok{"BerzuKraja3"}\NormalTok{,}
           \AttributeTok{fun=}\StringTok{"max"}\NormalTok{,}
           \AttributeTok{prepare=}\ConstantTok{FALSE}\NormalTok{,}
           \AttributeTok{restrict\_to =}\NormalTok{ clearcut\_mask,}
           \AttributeTok{restrict\_values =} \DecValTok{0}\NormalTok{,}
           \AttributeTok{plot\_result=}\ConstantTok{TRUE}\NormalTok{,}
           \AttributeTok{overwrite=}\ConstantTok{TRUE}\NormalTok{)}
\NormalTok{p2i\_rez}
\NormalTok{i2e\_rez}\OtherTok{=}\FunctionTok{input2egv}\NormalTok{(}\AttributeTok{input=}\StringTok{"./RasterGrids\_10m/2024/ForestsQuant\_VolumeBirch.tif"}\NormalTok{,}
         \AttributeTok{egv\_template =} \StringTok{"./Templates/TemplateRasters/LV100m\_10km.tif"}\NormalTok{,}
         \AttributeTok{summary\_function =} \StringTok{"sum"}\NormalTok{,}
         \AttributeTok{missing\_job =} \StringTok{"CoverOutput"}\NormalTok{,}
         \AttributeTok{output\_bg =} \StringTok{"./Templates/TemplateRasters/nulls\_LV100m\_10km.tif"}\NormalTok{,}
         \AttributeTok{outlocation =} \StringTok{"./RasterGrids\_100m/2024/RAW/"}\NormalTok{,}
         \AttributeTok{outfilename =} \StringTok{"ForestsQuant\_VolumeBirch{-}sum\_cell.tif"}\NormalTok{,}
         \AttributeTok{layername =} \StringTok{"egv\_295"}\NormalTok{,}
         \AttributeTok{plot\_final=}\ConstantTok{TRUE}\NormalTok{)}
\NormalTok{i2e\_rez}
\FunctionTok{rm}\NormalTok{(p2i\_rez)}
\FunctionTok{rm}\NormalTok{(nogabali)}
\FunctionTok{rm}\NormalTok{(berzi)}
\FunctionTok{rm}\NormalTok{(i2e\_rez)}
\FunctionTok{unlink}\NormalTok{(}\StringTok{"./RasterGrids\_10m/2024/ForestsQuant\_VolumeBirch.tif"}\NormalTok{)}

\CommentTok{\# standardisation {-}{-}{-}{-}}
\ControlFlowTok{if}\NormalTok{(}\SpecialCharTok{!}\FunctionTok{require}\NormalTok{(terra)) \{}\FunctionTok{install.packages}\NormalTok{(}\StringTok{"terra"}\NormalTok{); }\FunctionTok{require}\NormalTok{(terra)\}}
\ControlFlowTok{if}\NormalTok{(}\SpecialCharTok{!}\FunctionTok{require}\NormalTok{(tidyverse)) \{}\FunctionTok{install.packages}\NormalTok{(}\StringTok{"tidyverse"}\NormalTok{); }\FunctionTok{require}\NormalTok{(tidyverse)\}}

\NormalTok{nosaukums}\OtherTok{=}\StringTok{"ForestsQuant\_VolumeBirch{-}sum\_cell.tif"}
\NormalTok{ielasisanas\_cels}\OtherTok{=}\FunctionTok{paste0}\NormalTok{(}\StringTok{"./RasterGrids\_100m/2024/RAW/"}\NormalTok{,nosaukums)}
\NormalTok{saglabasanas\_cels}\OtherTok{=}\FunctionTok{paste0}\NormalTok{(}\StringTok{"./RasterGrids\_100m/2024/Scaled/"}\NormalTok{,nosaukums)}
\NormalTok{slanis}\OtherTok{=}\FunctionTok{rast}\NormalTok{(ielasisanas\_cels)}
\NormalTok{videjais}\OtherTok{=}\FunctionTok{global}\NormalTok{(slanis,}\AttributeTok{fun=}\StringTok{"mean"}\NormalTok{,}\AttributeTok{na.rm=}\ConstantTok{TRUE}\NormalTok{)}
\NormalTok{centrets}\OtherTok{=}\NormalTok{slanis}\SpecialCharTok{{-}}\NormalTok{videjais[,}\DecValTok{1}\NormalTok{]}
\NormalTok{standartnovirze}\OtherTok{=}\NormalTok{terra}\SpecialCharTok{::}\FunctionTok{global}\NormalTok{(centrets,}\AttributeTok{fun=}\StringTok{"rms"}\NormalTok{,}\AttributeTok{na.rm=}\ConstantTok{TRUE}\NormalTok{)}
\NormalTok{merogots}\OtherTok{=}\NormalTok{centrets}\SpecialCharTok{/}\NormalTok{standartnovirze[,}\DecValTok{1}\NormalTok{]}
\FunctionTok{writeRaster}\NormalTok{(merogots,}
      \AttributeTok{filename=}\NormalTok{saglabasanas\_cels,}
      \AttributeTok{overwrite=}\ConstantTok{TRUE}\NormalTok{)}
\end{Highlighting}
\end{Shaded}

\section{ForestsQuant\_VolumeBlackAlder-sum\_cell}\label{ch06.296}

\textbf{filename:} \texttt{ForestsQuant\_VolumeBlackAlder-sum\_cell.tif}

\textbf{layername:} \texttt{egv\_296}

\textbf{English name:} Timber volume of Black Alder within the analysis cell (1 ha)

\textbf{Latvian name:} Melnalkšņu krāja analīzes šūnā (1 ha)

\textbf{Procedure:} Most EGVs describing forests are spatially restricted to areas outside
of clearcuts and dead stands. This mask is created using a combination of
the \hyperref[Ch04.01]{State Forest Service's
State Forest Registry} land category 12 and 14, and \hyperref[Ch04.09]{The
Global Forest Watch} pixels classified as lost tree canopy cover since
2020 (raster layer matching input, presence = 1, absence = 0).

This EGV is prepared based on the information of timber volume of black alder
(species code: 6; see tree species codes in \hyperref[Ch01]{Terminology and acronyms})
in the inventoried forest stands - \hyperref[Ch04.01]{State Forest Service's State Forest
Registry}. This attribute has some extreme
values. We chose to limit them to the nearest integer showing only minimal
accumulation in histogram.

\includegraphics[width=0.8\linewidth]{./Figures/Histogramms/hist_egv296}

Resulting values at polygon geometries are rasterised with the workflow
\texttt{egvtools::polygon2input()}, restricting to pixels outside the clearcut mask. No
background values are assigned during rasterisation. The resulting layer is
then aggregated to EGV resolution using the workflow \texttt{egvtools::input2egv()} by calculating
sum of pixel values. After the aggregation, cells with no forest information
are filled with value 0. Finally, the layer is standardised by subtracting
the arithmetic mean and dividing by the root mean squared error.

\begin{Shaded}
\begin{Highlighting}[]
\CommentTok{\# libs {-}{-}{-}{-}}
\ControlFlowTok{if}\NormalTok{(}\SpecialCharTok{!}\FunctionTok{require}\NormalTok{(egvtools)) \{remotes}\SpecialCharTok{::}\FunctionTok{install\_github}\NormalTok{(}\StringTok{"aavotins/egvtools"}\NormalTok{); }\FunctionTok{require}\NormalTok{(egvtools)\}}
\ControlFlowTok{if}\NormalTok{(}\SpecialCharTok{!}\FunctionTok{require}\NormalTok{(terra)) \{}\FunctionTok{install.packages}\NormalTok{(}\StringTok{"terra"}\NormalTok{); }\FunctionTok{require}\NormalTok{(terra)\}}
\ControlFlowTok{if}\NormalTok{(}\SpecialCharTok{!}\FunctionTok{require}\NormalTok{(sf)) \{}\FunctionTok{install.packages}\NormalTok{(}\StringTok{"sf"}\NormalTok{); }\FunctionTok{require}\NormalTok{(sf)\}}
\ControlFlowTok{if}\NormalTok{(}\SpecialCharTok{!}\FunctionTok{require}\NormalTok{(tidyverse)) \{}\FunctionTok{install.packages}\NormalTok{(}\StringTok{"tidyverse"}\NormalTok{); }\FunctionTok{require}\NormalTok{(tidyverse)\}}
\ControlFlowTok{if}\NormalTok{(}\SpecialCharTok{!}\FunctionTok{require}\NormalTok{(sfarrow)) \{}\FunctionTok{install.packages}\NormalTok{(}\StringTok{"sfarrow"}\NormalTok{); }\FunctionTok{require}\NormalTok{(sfarrow)\}}
\ControlFlowTok{if}\NormalTok{(}\SpecialCharTok{!}\FunctionTok{require}\NormalTok{(readxl)) \{}\FunctionTok{install.packages}\NormalTok{(}\StringTok{"readxl"}\NormalTok{); }\FunctionTok{require}\NormalTok{(readxl)\}}
\ControlFlowTok{if}\NormalTok{(}\SpecialCharTok{!}\FunctionTok{require}\NormalTok{(raster)) \{}\FunctionTok{install.packages}\NormalTok{(}\StringTok{"raster"}\NormalTok{); }\FunctionTok{require}\NormalTok{(raster)\}}
\ControlFlowTok{if}\NormalTok{(}\SpecialCharTok{!}\FunctionTok{require}\NormalTok{(fasterize)) \{}\FunctionTok{install.packages}\NormalTok{(}\StringTok{"fasterize"}\NormalTok{); }\FunctionTok{require}\NormalTok{(fasterize)\}}

\CommentTok{\# templates {-}{-}{-}{-}}
\NormalTok{template100}\OtherTok{=}\FunctionTok{rast}\NormalTok{(}\StringTok{"./Templates/TemplateRasters/LV100m\_10km.tif"}\NormalTok{)}
\NormalTok{template10}\OtherTok{=}\FunctionTok{rast}\NormalTok{(}\StringTok{"./Templates/TemplateRasters/LV10m\_10km.tif"}\NormalTok{)}
\NormalTok{rastrs10}\OtherTok{=}\FunctionTok{raster}\NormalTok{(template10)}

\NormalTok{nulls10}\OtherTok{=}\FunctionTok{rast}\NormalTok{(}\StringTok{"./Templates/TemplateRasters/nulls\_LV10m\_10km.tif"}\NormalTok{)}
\NormalTok{nulls100}\OtherTok{=}\FunctionTok{rast}\NormalTok{(}\StringTok{"./Templates/TemplateRasters/nulls\_LV100m\_10km.tif"}\NormalTok{)}


\CommentTok{\# simple landscape {-}{-}{-}{-}}
\NormalTok{simple\_landscape}\OtherTok{=}\FunctionTok{rast}\NormalTok{(}\StringTok{"RasterGrids\_10m/2024/Ainava\_vienk\_mask.tif"}\NormalTok{)}

\CommentTok{\# mvr {-}{-}{-}{-}}
\NormalTok{mvr}\OtherTok{=}\FunctionTok{st\_read\_parquet}\NormalTok{(}\StringTok{"./Geodata/2024/MVR/nogabali\_2024janv.parquet"}\NormalTok{)}
\NormalTok{mvr}\SpecialCharTok{$}\NormalTok{yes}\OtherTok{=}\DecValTok{1}

\CommentTok{\# clear cut mask {-}{-}{-}{-}}
\NormalTok{izcirtumi}\OtherTok{=}\NormalTok{mvr }\SpecialCharTok{\%\textgreater{}\%} 
 \FunctionTok{filter}\NormalTok{(zkat }\SpecialCharTok{\%in\%} \FunctionTok{c}\NormalTok{(}\StringTok{"12"}\NormalTok{,}\StringTok{"14"}\NormalTok{)) }\SpecialCharTok{\%\textgreater{}\%} 
\NormalTok{ dplyr}\SpecialCharTok{::}\FunctionTok{select}\NormalTok{(yes)}
\NormalTok{r\_izcirtumi\_mvr}\OtherTok{=}\FunctionTok{fasterize}\NormalTok{(izcirtumi,rastrs10,}\AttributeTok{field=}\StringTok{"yes"}\NormalTok{)}
\NormalTok{t\_izcirtumi\_mvr}\OtherTok{=}\FunctionTok{rast}\NormalTok{(r\_izcirtumi\_mvr)}
\FunctionTok{plot}\NormalTok{(t\_izcirtumi\_mvr)}

\NormalTok{tcl}\OtherTok{=}\FunctionTok{rast}\NormalTok{(}\StringTok{"./Geodata/2024/Trees/GFW/TreeCoverLoss\_v1\_12.tif"}\NormalTok{)}
\NormalTok{tcl2}\OtherTok{=}\FunctionTok{ifel}\NormalTok{(tcl}\SpecialCharTok{\textless{}}\DecValTok{20}\NormalTok{,}\DecValTok{0}\NormalTok{,}\DecValTok{1}\NormalTok{)}
\NormalTok{tclX}\OtherTok{=}\FunctionTok{cover}\NormalTok{(tcl2,nulls10)}
\FunctionTok{plot}\NormalTok{(tclX)}

\NormalTok{clearcut\_mask}\OtherTok{=}\FunctionTok{cover}\NormalTok{(t\_izcirtumi\_mvr,tclX,}
          \AttributeTok{filename=}\StringTok{"./RasterGrids\_10m/2024/Mask\_clearcuts.tif"}\NormalTok{,}
          \AttributeTok{overwrite=}\ConstantTok{TRUE}\NormalTok{)}
\FunctionTok{plot}\NormalTok{(clearcut\_mask)}

\FunctionTok{rm}\NormalTok{(izcirtumi)}
\FunctionTok{rm}\NormalTok{(r\_izcirtumi\_mvr)}
\FunctionTok{rm}\NormalTok{(t\_izcirtumi\_mvr)}
\FunctionTok{rm}\NormalTok{(tcl)}
\FunctionTok{rm}\NormalTok{(tcl2)}
\FunctionTok{rm}\NormalTok{(tclX)}

\CommentTok{\# ForestsQuant\_VolumeBlackAlder{-}sum\_cell.tif    egv\_296 {-}{-}{-}{-}}

\NormalTok{melnalksni}\OtherTok{=}\FunctionTok{c}\NormalTok{(}\StringTok{"6"}\NormalTok{)}
\NormalTok{nogabali}\OtherTok{=}\NormalTok{mvr }\SpecialCharTok{\%\textgreater{}\%} 
 \FunctionTok{mutate}\NormalTok{(}\AttributeTok{MeKraja=}\FunctionTok{ifelse}\NormalTok{(s10 }\SpecialCharTok{\%in\%}\NormalTok{ melnalksni, v10, }\DecValTok{0}\NormalTok{)}\SpecialCharTok{+}\FunctionTok{ifelse}\NormalTok{(s11 }\SpecialCharTok{\%in\%}\NormalTok{ melnalksni,v11,}\DecValTok{0}\NormalTok{)}\SpecialCharTok{+}
      \FunctionTok{ifelse}\NormalTok{(s12 }\SpecialCharTok{\%in\%}\NormalTok{ melnalksni, v12,}\DecValTok{0}\NormalTok{)}\SpecialCharTok{+}\FunctionTok{ifelse}\NormalTok{(s13 }\SpecialCharTok{\%in\%}\NormalTok{ melnalksni,v13,}\DecValTok{0}\NormalTok{)}\SpecialCharTok{+}
      \FunctionTok{ifelse}\NormalTok{(s14 }\SpecialCharTok{\%in\%}\NormalTok{ melnalksni, v14,}\DecValTok{0}\NormalTok{)) }\SpecialCharTok{\%\textgreater{}\%} 
 \FunctionTok{mutate}\NormalTok{(}\AttributeTok{MeKraja2=}\NormalTok{MeKraja}\SpecialCharTok{/}\DecValTok{10000}\SpecialCharTok{*}\DecValTok{10}\SpecialCharTok{*}\DecValTok{10}\NormalTok{) }\SpecialCharTok{\%\textgreater{}\%} 
 \FunctionTok{mutate}\NormalTok{(}\AttributeTok{MeKraja3=}\FunctionTok{ifelse}\NormalTok{(MeKraja2}\SpecialCharTok{\textgreater{}}\DecValTok{4}\NormalTok{,}\DecValTok{4}\NormalTok{,MeKraja2)) }\SpecialCharTok{\%\textgreater{}\%} 
 \FunctionTok{filter}\NormalTok{(}\SpecialCharTok{!}\FunctionTok{is.na}\NormalTok{(MeKraja2))}

\FunctionTok{par}\NormalTok{(}\AttributeTok{mfrow=}\FunctionTok{c}\NormalTok{(}\DecValTok{1}\NormalTok{,}\DecValTok{2}\NormalTok{))}
\FunctionTok{options}\NormalTok{(}\AttributeTok{scipen=}\DecValTok{999}\NormalTok{)}
\FunctionTok{hist}\NormalTok{(nogabali}\SpecialCharTok{$}\NormalTok{MeKraja2,}\AttributeTok{main=}\StringTok{"Original"}\NormalTok{,}\AttributeTok{xlab=}\StringTok{"Black alder volume"}\NormalTok{)}
\FunctionTok{hist}\NormalTok{(nogabali}\SpecialCharTok{$}\NormalTok{MeKraja3,}\AttributeTok{main=}\StringTok{"Limited"}\NormalTok{,}\AttributeTok{xlab=}\StringTok{"Black alder volume"}\NormalTok{)}
\FunctionTok{par}\NormalTok{(}\AttributeTok{mfrow=}\FunctionTok{c}\NormalTok{(}\DecValTok{1}\NormalTok{,}\DecValTok{1}\NormalTok{))}
\FunctionTok{options}\NormalTok{(}\AttributeTok{scipen=}\DecValTok{0}\NormalTok{)}

\NormalTok{p2i\_rez}\OtherTok{=}\FunctionTok{polygon2input}\NormalTok{(}\AttributeTok{vector\_data=}\NormalTok{nogabali,}
           \AttributeTok{template\_path =} \StringTok{"./Templates/TemplateRasters/LV10m\_10km.tif"}\NormalTok{,}
           \AttributeTok{out\_path =} \StringTok{"./RasterGrids\_10m/2024/"}\NormalTok{,}
           \AttributeTok{file\_name =} \StringTok{"ForestsQuant\_VolumeBlackAlder.tif"}\NormalTok{,}
           \AttributeTok{value\_field =} \StringTok{"MeKraja3"}\NormalTok{,}
           \AttributeTok{fun=}\StringTok{"max"}\NormalTok{,}
           \AttributeTok{prepare=}\ConstantTok{FALSE}\NormalTok{,}
           \AttributeTok{restrict\_to =}\NormalTok{ clearcut\_mask,}
           \AttributeTok{restrict\_values =} \DecValTok{0}\NormalTok{,}
           \AttributeTok{plot\_result=}\ConstantTok{TRUE}\NormalTok{,}
           \AttributeTok{overwrite=}\ConstantTok{TRUE}\NormalTok{)}
\NormalTok{p2i\_rez}
\NormalTok{i2e\_rez}\OtherTok{=}\FunctionTok{input2egv}\NormalTok{(}\AttributeTok{input=}\StringTok{"./RasterGrids\_10m/2024/ForestsQuant\_VolumeBlackAlder.tif"}\NormalTok{,}
         \AttributeTok{egv\_template =} \StringTok{"./Templates/TemplateRasters/LV100m\_10km.tif"}\NormalTok{,}
         \AttributeTok{summary\_function =} \StringTok{"sum"}\NormalTok{,}
         \AttributeTok{missing\_job =} \StringTok{"CoverOutput"}\NormalTok{,}
         \AttributeTok{output\_bg =} \StringTok{"./Templates/TemplateRasters/nulls\_LV100m\_10km.tif"}\NormalTok{,}
         \AttributeTok{outlocation =} \StringTok{"./RasterGrids\_100m/2024/RAW/"}\NormalTok{,}
         \AttributeTok{outfilename =} \StringTok{"ForestsQuant\_VolumeBlackAlder{-}sum\_cell.tif"}\NormalTok{,}
         \AttributeTok{layername =} \StringTok{"egv\_296"}\NormalTok{,}
         \AttributeTok{plot\_final=}\ConstantTok{TRUE}\NormalTok{)}
\NormalTok{i2e\_rez}
\FunctionTok{rm}\NormalTok{(p2i\_rez)}
\FunctionTok{rm}\NormalTok{(nogabali)}
\FunctionTok{rm}\NormalTok{(melnalksni)}
\FunctionTok{rm}\NormalTok{(i2e\_rez)}
\FunctionTok{unlink}\NormalTok{(}\StringTok{"./RasterGrids\_10m/2024/ForestsQuant\_VolumeBlackAlder.tif"}\NormalTok{)}

\CommentTok{\# standardisation {-}{-}{-}{-}}
\ControlFlowTok{if}\NormalTok{(}\SpecialCharTok{!}\FunctionTok{require}\NormalTok{(terra)) \{}\FunctionTok{install.packages}\NormalTok{(}\StringTok{"terra"}\NormalTok{); }\FunctionTok{require}\NormalTok{(terra)\}}
\ControlFlowTok{if}\NormalTok{(}\SpecialCharTok{!}\FunctionTok{require}\NormalTok{(tidyverse)) \{}\FunctionTok{install.packages}\NormalTok{(}\StringTok{"tidyverse"}\NormalTok{); }\FunctionTok{require}\NormalTok{(tidyverse)\}}

\NormalTok{nosaukums}\OtherTok{=}\StringTok{"ForestsQuant\_VolumeBlackAlder{-}sum\_cell.tif"}
\NormalTok{ielasisanas\_cels}\OtherTok{=}\FunctionTok{paste0}\NormalTok{(}\StringTok{"./RasterGrids\_100m/2024/RAW/"}\NormalTok{,nosaukums)}
\NormalTok{saglabasanas\_cels}\OtherTok{=}\FunctionTok{paste0}\NormalTok{(}\StringTok{"./RasterGrids\_100m/2024/Scaled/"}\NormalTok{,nosaukums)}
\NormalTok{slanis}\OtherTok{=}\FunctionTok{rast}\NormalTok{(ielasisanas\_cels)}
\NormalTok{videjais}\OtherTok{=}\FunctionTok{global}\NormalTok{(slanis,}\AttributeTok{fun=}\StringTok{"mean"}\NormalTok{,}\AttributeTok{na.rm=}\ConstantTok{TRUE}\NormalTok{)}
\NormalTok{centrets}\OtherTok{=}\NormalTok{slanis}\SpecialCharTok{{-}}\NormalTok{videjais[,}\DecValTok{1}\NormalTok{]}
\NormalTok{standartnovirze}\OtherTok{=}\NormalTok{terra}\SpecialCharTok{::}\FunctionTok{global}\NormalTok{(centrets,}\AttributeTok{fun=}\StringTok{"rms"}\NormalTok{,}\AttributeTok{na.rm=}\ConstantTok{TRUE}\NormalTok{)}
\NormalTok{merogots}\OtherTok{=}\NormalTok{centrets}\SpecialCharTok{/}\NormalTok{standartnovirze[,}\DecValTok{1}\NormalTok{]}
\FunctionTok{writeRaster}\NormalTok{(merogots,}
      \AttributeTok{filename=}\NormalTok{saglabasanas\_cels,}
      \AttributeTok{overwrite=}\ConstantTok{TRUE}\NormalTok{)}
\end{Highlighting}
\end{Shaded}

\section{ForestsQuant\_VolumeBorealDeciduousOther-sum\_cell}\label{ch06.297}

\textbf{filename:} \texttt{ForestsQuant\_VolumeBorealDeciduousOther-sum\_cell.tif}

\textbf{layername:} \texttt{egv\_297}

\textbf{English name:} Timber volume of Other Boreal Deciduous trees within the
analysis cell (1 ha)

\textbf{Latvian name:} Citu šaurlapju krāja analīzes šūnā (1 ha)

\textbf{Procedure:} Most EGVs describing forests are spatially restricted to areas outside
of clearcuts and dead stands. This mask is created using a combination of
the \hyperref[Ch04.01]{State Forest Service's
State Forest Registry} land category 12 and 14, and \hyperref[Ch04.09]{The
Global Forest Watch} pixels classified as lost tree canopy cover since
2020 (raster layer matching input, presence = 1, absence = 0).

This EGV is prepared based on the information of timber volume of Boreal
deciduous tree species not separately described with own EGVs (species codes: 9, 20, 21, 32,
35; see tree species codes in \hyperref[Ch01]{Terminology and acronyms}) in the
inventoried forest stands - \hyperref[Ch04.01]{State Forest Service's State Forest
Registry}. This attribute has some extreme
values. We chose to limit them to the nearest integer showing only minimal
accumulation in histogram.

\includegraphics[width=0.8\linewidth]{./Figures/Histogramms/hist_egv297}

Resulting values at polygon geometries are rasterised with the workflow
\texttt{egvtools::polygon2input()}, restricting to pixels outside the clearcut mask. No
background values are assigned during rasterisation. The resulting layer is
then aggregated to EGV resolution using the workflow \texttt{egvtools::input2egv()} by calculating
sum of pixel values. After the aggregation, cells with no forest information
are filled with value 0. Finally, the layer is standardised by subtracting
the arithmetic mean and dividing by the root mean squared error.

\begin{Shaded}
\begin{Highlighting}[]
\CommentTok{\# libs {-}{-}{-}{-}}
\ControlFlowTok{if}\NormalTok{(}\SpecialCharTok{!}\FunctionTok{require}\NormalTok{(egvtools)) \{remotes}\SpecialCharTok{::}\FunctionTok{install\_github}\NormalTok{(}\StringTok{"aavotins/egvtools"}\NormalTok{); }\FunctionTok{require}\NormalTok{(egvtools)\}}
\ControlFlowTok{if}\NormalTok{(}\SpecialCharTok{!}\FunctionTok{require}\NormalTok{(terra)) \{}\FunctionTok{install.packages}\NormalTok{(}\StringTok{"terra"}\NormalTok{); }\FunctionTok{require}\NormalTok{(terra)\}}
\ControlFlowTok{if}\NormalTok{(}\SpecialCharTok{!}\FunctionTok{require}\NormalTok{(sf)) \{}\FunctionTok{install.packages}\NormalTok{(}\StringTok{"sf"}\NormalTok{); }\FunctionTok{require}\NormalTok{(sf)\}}
\ControlFlowTok{if}\NormalTok{(}\SpecialCharTok{!}\FunctionTok{require}\NormalTok{(tidyverse)) \{}\FunctionTok{install.packages}\NormalTok{(}\StringTok{"tidyverse"}\NormalTok{); }\FunctionTok{require}\NormalTok{(tidyverse)\}}
\ControlFlowTok{if}\NormalTok{(}\SpecialCharTok{!}\FunctionTok{require}\NormalTok{(sfarrow)) \{}\FunctionTok{install.packages}\NormalTok{(}\StringTok{"sfarrow"}\NormalTok{); }\FunctionTok{require}\NormalTok{(sfarrow)\}}
\ControlFlowTok{if}\NormalTok{(}\SpecialCharTok{!}\FunctionTok{require}\NormalTok{(readxl)) \{}\FunctionTok{install.packages}\NormalTok{(}\StringTok{"readxl"}\NormalTok{); }\FunctionTok{require}\NormalTok{(readxl)\}}
\ControlFlowTok{if}\NormalTok{(}\SpecialCharTok{!}\FunctionTok{require}\NormalTok{(raster)) \{}\FunctionTok{install.packages}\NormalTok{(}\StringTok{"raster"}\NormalTok{); }\FunctionTok{require}\NormalTok{(raster)\}}
\ControlFlowTok{if}\NormalTok{(}\SpecialCharTok{!}\FunctionTok{require}\NormalTok{(fasterize)) \{}\FunctionTok{install.packages}\NormalTok{(}\StringTok{"fasterize"}\NormalTok{); }\FunctionTok{require}\NormalTok{(fasterize)\}}

\CommentTok{\# templates {-}{-}{-}{-}}
\NormalTok{template100}\OtherTok{=}\FunctionTok{rast}\NormalTok{(}\StringTok{"./Templates/TemplateRasters/LV100m\_10km.tif"}\NormalTok{)}
\NormalTok{template10}\OtherTok{=}\FunctionTok{rast}\NormalTok{(}\StringTok{"./Templates/TemplateRasters/LV10m\_10km.tif"}\NormalTok{)}
\NormalTok{rastrs10}\OtherTok{=}\FunctionTok{raster}\NormalTok{(template10)}

\NormalTok{nulls10}\OtherTok{=}\FunctionTok{rast}\NormalTok{(}\StringTok{"./Templates/TemplateRasters/nulls\_LV10m\_10km.tif"}\NormalTok{)}
\NormalTok{nulls100}\OtherTok{=}\FunctionTok{rast}\NormalTok{(}\StringTok{"./Templates/TemplateRasters/nulls\_LV100m\_10km.tif"}\NormalTok{)}


\CommentTok{\# simple landscape {-}{-}{-}{-}}
\NormalTok{simple\_landscape}\OtherTok{=}\FunctionTok{rast}\NormalTok{(}\StringTok{"RasterGrids\_10m/2024/Ainava\_vienk\_mask.tif"}\NormalTok{)}

\CommentTok{\# mvr {-}{-}{-}{-}}
\NormalTok{mvr}\OtherTok{=}\FunctionTok{st\_read\_parquet}\NormalTok{(}\StringTok{"./Geodata/2024/MVR/nogabali\_2024janv.parquet"}\NormalTok{)}
\NormalTok{mvr}\SpecialCharTok{$}\NormalTok{yes}\OtherTok{=}\DecValTok{1}

\CommentTok{\# clear cut mask {-}{-}{-}{-}}
\NormalTok{izcirtumi}\OtherTok{=}\NormalTok{mvr }\SpecialCharTok{\%\textgreater{}\%} 
 \FunctionTok{filter}\NormalTok{(zkat }\SpecialCharTok{\%in\%} \FunctionTok{c}\NormalTok{(}\StringTok{"12"}\NormalTok{,}\StringTok{"14"}\NormalTok{)) }\SpecialCharTok{\%\textgreater{}\%} 
\NormalTok{ dplyr}\SpecialCharTok{::}\FunctionTok{select}\NormalTok{(yes)}
\NormalTok{r\_izcirtumi\_mvr}\OtherTok{=}\FunctionTok{fasterize}\NormalTok{(izcirtumi,rastrs10,}\AttributeTok{field=}\StringTok{"yes"}\NormalTok{)}
\NormalTok{t\_izcirtumi\_mvr}\OtherTok{=}\FunctionTok{rast}\NormalTok{(r\_izcirtumi\_mvr)}
\FunctionTok{plot}\NormalTok{(t\_izcirtumi\_mvr)}

\NormalTok{tcl}\OtherTok{=}\FunctionTok{rast}\NormalTok{(}\StringTok{"./Geodata/2024/Trees/GFW/TreeCoverLoss\_v1\_12.tif"}\NormalTok{)}
\NormalTok{tcl2}\OtherTok{=}\FunctionTok{ifel}\NormalTok{(tcl}\SpecialCharTok{\textless{}}\DecValTok{20}\NormalTok{,}\DecValTok{0}\NormalTok{,}\DecValTok{1}\NormalTok{)}
\NormalTok{tclX}\OtherTok{=}\FunctionTok{cover}\NormalTok{(tcl2,nulls10)}
\FunctionTok{plot}\NormalTok{(tclX)}

\NormalTok{clearcut\_mask}\OtherTok{=}\FunctionTok{cover}\NormalTok{(t\_izcirtumi\_mvr,tclX,}
          \AttributeTok{filename=}\StringTok{"./RasterGrids\_10m/2024/Mask\_clearcuts.tif"}\NormalTok{,}
          \AttributeTok{overwrite=}\ConstantTok{TRUE}\NormalTok{)}
\FunctionTok{plot}\NormalTok{(clearcut\_mask)}

\FunctionTok{rm}\NormalTok{(izcirtumi)}
\FunctionTok{rm}\NormalTok{(r\_izcirtumi\_mvr)}
\FunctionTok{rm}\NormalTok{(t\_izcirtumi\_mvr)}
\FunctionTok{rm}\NormalTok{(tcl)}
\FunctionTok{rm}\NormalTok{(tcl2)}
\FunctionTok{rm}\NormalTok{(tclX)}

\CommentTok{\# ForestsQuant\_VolumeBorealDeciduousOther{-}sum\_cell.tif  egv\_297 {-}{-}{-}{-}}

\NormalTok{sl\_citi}\OtherTok{=}\FunctionTok{c}\NormalTok{(}\StringTok{"9"}\NormalTok{,}\StringTok{"20"}\NormalTok{,}\StringTok{"21"}\NormalTok{,}\StringTok{"32"}\NormalTok{,}\StringTok{"35"}\NormalTok{)}
\NormalTok{nogabali}\OtherTok{=}\NormalTok{mvr }\SpecialCharTok{\%\textgreater{}\%} 
 \FunctionTok{mutate}\NormalTok{(}\AttributeTok{SaurlapjuCKraja=}\FunctionTok{ifelse}\NormalTok{(s10 }\SpecialCharTok{\%in\%}\NormalTok{ sl\_citi, v10, }\DecValTok{0}\NormalTok{)}\SpecialCharTok{+}\FunctionTok{ifelse}\NormalTok{(s11 }\SpecialCharTok{\%in\%}\NormalTok{ sl\_citi,v11,}\DecValTok{0}\NormalTok{)}\SpecialCharTok{+}
      \FunctionTok{ifelse}\NormalTok{(s12 }\SpecialCharTok{\%in\%}\NormalTok{ sl\_citi, v12,}\DecValTok{0}\NormalTok{)}\SpecialCharTok{+}\FunctionTok{ifelse}\NormalTok{(s13 }\SpecialCharTok{\%in\%}\NormalTok{ sl\_citi,v13,}\DecValTok{0}\NormalTok{)}\SpecialCharTok{+}
      \FunctionTok{ifelse}\NormalTok{(s14 }\SpecialCharTok{\%in\%}\NormalTok{ sl\_citi, v14,}\DecValTok{0}\NormalTok{)) }\SpecialCharTok{\%\textgreater{}\%} 
 \FunctionTok{mutate}\NormalTok{(}\AttributeTok{SaurlapjuCKraja2=}\NormalTok{SaurlapjuCKraja}\SpecialCharTok{/}\DecValTok{10000}\SpecialCharTok{*}\DecValTok{10}\SpecialCharTok{*}\DecValTok{10}\NormalTok{) }\SpecialCharTok{\%\textgreater{}\%} 
 \FunctionTok{mutate}\NormalTok{(}\AttributeTok{SaurlapjuCKraja3=}\FunctionTok{ifelse}\NormalTok{(SaurlapjuCKraja2}\SpecialCharTok{\textgreater{}}\DecValTok{3}\NormalTok{,}\DecValTok{3}\NormalTok{,SaurlapjuCKraja2)) }\SpecialCharTok{\%\textgreater{}\%} 
 \FunctionTok{filter}\NormalTok{(}\SpecialCharTok{!}\FunctionTok{is.na}\NormalTok{(SaurlapjuCKraja2))}

\FunctionTok{par}\NormalTok{(}\AttributeTok{mfrow=}\FunctionTok{c}\NormalTok{(}\DecValTok{1}\NormalTok{,}\DecValTok{2}\NormalTok{))}
\FunctionTok{options}\NormalTok{(}\AttributeTok{scipen=}\DecValTok{999}\NormalTok{)}
\FunctionTok{hist}\NormalTok{(nogabali}\SpecialCharTok{$}\NormalTok{SaurlapjuCKraja2,}\AttributeTok{main=}\StringTok{"Original"}\NormalTok{,}\AttributeTok{xlab=}\StringTok{"Other Boreal deciduous volume"}\NormalTok{)}
\FunctionTok{hist}\NormalTok{(nogabali}\SpecialCharTok{$}\NormalTok{SaurlapjuCKraja3,}\AttributeTok{main=}\StringTok{"Limited"}\NormalTok{,}\AttributeTok{xlab=}\StringTok{"Other Boreal deciduous volume"}\NormalTok{)}
\FunctionTok{par}\NormalTok{(}\AttributeTok{mfrow=}\FunctionTok{c}\NormalTok{(}\DecValTok{1}\NormalTok{,}\DecValTok{1}\NormalTok{))}
\FunctionTok{options}\NormalTok{(}\AttributeTok{scipen=}\DecValTok{0}\NormalTok{)}

\NormalTok{p2i\_rez}\OtherTok{=}\FunctionTok{polygon2input}\NormalTok{(}\AttributeTok{vector\_data=}\NormalTok{nogabali,}
           \AttributeTok{template\_path =} \StringTok{"./Templates/TemplateRasters/LV10m\_10km.tif"}\NormalTok{,}
           \AttributeTok{out\_path =} \StringTok{"./RasterGrids\_10m/2024/"}\NormalTok{,}
           \AttributeTok{file\_name =} \StringTok{"ForestsQuant\_VolumeBorealDeciduousOther.tif"}\NormalTok{,}
           \AttributeTok{value\_field =} \StringTok{"SaurlapjuCKraja3"}\NormalTok{,}
           \AttributeTok{fun=}\StringTok{"max"}\NormalTok{,}
           \AttributeTok{prepare=}\ConstantTok{FALSE}\NormalTok{,}
           \AttributeTok{restrict\_to =}\NormalTok{ clearcut\_mask,}
           \AttributeTok{restrict\_values =} \DecValTok{0}\NormalTok{,}
           \AttributeTok{plot\_result=}\ConstantTok{TRUE}\NormalTok{,}
           \AttributeTok{overwrite=}\ConstantTok{TRUE}\NormalTok{)}
\NormalTok{p2i\_rez}
\NormalTok{i2e\_rez}\OtherTok{=}\FunctionTok{input2egv}\NormalTok{(}\AttributeTok{input=}\StringTok{"./RasterGrids\_10m/2024/ForestsQuant\_VolumeBorealDeciduousOther.tif"}\NormalTok{,}
         \AttributeTok{egv\_template =} \StringTok{"./Templates/TemplateRasters/LV100m\_10km.tif"}\NormalTok{,}
         \AttributeTok{summary\_function =} \StringTok{"sum"}\NormalTok{,}
         \AttributeTok{missing\_job =} \StringTok{"CoverOutput"}\NormalTok{,}
         \AttributeTok{output\_bg =} \StringTok{"./Templates/TemplateRasters/nulls\_LV100m\_10km.tif"}\NormalTok{,}
         \AttributeTok{outlocation =} \StringTok{"./RasterGrids\_100m/2024/RAW/"}\NormalTok{,}
         \AttributeTok{outfilename =} \StringTok{"ForestsQuant\_VolumeBorealDeciduousOther{-}sum\_cell.tif"}\NormalTok{,}
         \AttributeTok{layername =} \StringTok{"egv\_297"}\NormalTok{,}
         \AttributeTok{plot\_final=}\ConstantTok{TRUE}\NormalTok{)}
\NormalTok{i2e\_rez}
\FunctionTok{rm}\NormalTok{(p2i\_rez)}
\FunctionTok{rm}\NormalTok{(nogabali)}
\FunctionTok{rm}\NormalTok{(sl\_citi)}
\FunctionTok{rm}\NormalTok{(i2e\_rez)}
\FunctionTok{unlink}\NormalTok{(}\StringTok{"./RasterGrids\_10m/2024/ForestsQuant\_VolumeBorealDeciduousOther.tif"}\NormalTok{)}


\CommentTok{\# standardisation {-}{-}{-}{-}}
\ControlFlowTok{if}\NormalTok{(}\SpecialCharTok{!}\FunctionTok{require}\NormalTok{(terra)) \{}\FunctionTok{install.packages}\NormalTok{(}\StringTok{"terra"}\NormalTok{); }\FunctionTok{require}\NormalTok{(terra)\}}
\ControlFlowTok{if}\NormalTok{(}\SpecialCharTok{!}\FunctionTok{require}\NormalTok{(tidyverse)) \{}\FunctionTok{install.packages}\NormalTok{(}\StringTok{"tidyverse"}\NormalTok{); }\FunctionTok{require}\NormalTok{(tidyverse)\}}

\NormalTok{nosaukums}\OtherTok{=}\StringTok{"ForestsQuant\_VolumeBorealDeciduousOther{-}sum\_cell.tif"}
\NormalTok{ielasisanas\_cels}\OtherTok{=}\FunctionTok{paste0}\NormalTok{(}\StringTok{"./RasterGrids\_100m/2024/RAW/"}\NormalTok{,nosaukums)}
\NormalTok{saglabasanas\_cels}\OtherTok{=}\FunctionTok{paste0}\NormalTok{(}\StringTok{"./RasterGrids\_100m/2024/Scaled/"}\NormalTok{,nosaukums)}
\NormalTok{slanis}\OtherTok{=}\FunctionTok{rast}\NormalTok{(ielasisanas\_cels)}
\NormalTok{videjais}\OtherTok{=}\FunctionTok{global}\NormalTok{(slanis,}\AttributeTok{fun=}\StringTok{"mean"}\NormalTok{,}\AttributeTok{na.rm=}\ConstantTok{TRUE}\NormalTok{)}
\NormalTok{centrets}\OtherTok{=}\NormalTok{slanis}\SpecialCharTok{{-}}\NormalTok{videjais[,}\DecValTok{1}\NormalTok{]}
\NormalTok{standartnovirze}\OtherTok{=}\NormalTok{terra}\SpecialCharTok{::}\FunctionTok{global}\NormalTok{(centrets,}\AttributeTok{fun=}\StringTok{"rms"}\NormalTok{,}\AttributeTok{na.rm=}\ConstantTok{TRUE}\NormalTok{)}
\NormalTok{merogots}\OtherTok{=}\NormalTok{centrets}\SpecialCharTok{/}\NormalTok{standartnovirze[,}\DecValTok{1}\NormalTok{]}
\FunctionTok{writeRaster}\NormalTok{(merogots,}
      \AttributeTok{filename=}\NormalTok{saglabasanas\_cels,}
      \AttributeTok{overwrite=}\ConstantTok{TRUE}\NormalTok{)}
\end{Highlighting}
\end{Shaded}

\section{ForestsQuant\_VolumeBorealDeciduousTotal-sum\_cell}\label{ch06.298}

\textbf{filename:} \texttt{ForestsQuant\_VolumeBorealDeciduousTotal-sum\_cell.tif}

\textbf{layername:} \texttt{egv\_298}

\textbf{English name:} Timber volume of Boreal Deciduous trees within the analysis
cell (1 ha)

\textbf{Latvian name:} Šaurlapju krāja analīzes šūnā (1 ha)

\textbf{Procedure:} Most EGVs describing forests are spatially restricted to areas outside
of clearcuts and dead stands. This mask is created using a combination of
the \hyperref[Ch04.01]{State Forest Service's
State Forest Registry} land category 12 and 14, and \hyperref[Ch04.09]{The
Global Forest Watch} pixels classified as lost tree canopy cover since
2020 (raster layer matching input, presence = 1, absence = 0).

This EGV is prepared based on the information of timber volume of Boreal
deciduous tree species (species codes: 4, 6, 8, 9, 19, 20, 21, 32, 35, 68;
see tree species codes in \hyperref[Ch01]{Terminology and acronyms}) in the inventoried
forest stands - \hyperref[Ch04.01]{State Forest Service's State Forest Registry}.
This attribute has some extreme
values. We chose to limit them to the nearest integer showing only minimal
accumulation in histogram.

\includegraphics[width=0.8\linewidth]{./Figures/Histogramms/hist_egv298}

Resulting values at polygon geometries are rasterised with the workflow
\texttt{egvtools::polygon2input()}, restricting to pixels outside the clearcut mask. No
background values are assigned during rasterisation. The resulting layer is
then aggregated to EGV resolution using the workflow \texttt{egvtools::input2egv()} by calculating
sum of pixel values. After the aggregation, cells with no forest information
are filled with value 0. Finally, the layer is standardised by subtracting
the arithmetic mean and dividing by the root mean squared error.

\begin{Shaded}
\begin{Highlighting}[]
\CommentTok{\# libs {-}{-}{-}{-}}
\ControlFlowTok{if}\NormalTok{(}\SpecialCharTok{!}\FunctionTok{require}\NormalTok{(egvtools)) \{remotes}\SpecialCharTok{::}\FunctionTok{install\_github}\NormalTok{(}\StringTok{"aavotins/egvtools"}\NormalTok{); }\FunctionTok{require}\NormalTok{(egvtools)\}}
\ControlFlowTok{if}\NormalTok{(}\SpecialCharTok{!}\FunctionTok{require}\NormalTok{(terra)) \{}\FunctionTok{install.packages}\NormalTok{(}\StringTok{"terra"}\NormalTok{); }\FunctionTok{require}\NormalTok{(terra)\}}
\ControlFlowTok{if}\NormalTok{(}\SpecialCharTok{!}\FunctionTok{require}\NormalTok{(sf)) \{}\FunctionTok{install.packages}\NormalTok{(}\StringTok{"sf"}\NormalTok{); }\FunctionTok{require}\NormalTok{(sf)\}}
\ControlFlowTok{if}\NormalTok{(}\SpecialCharTok{!}\FunctionTok{require}\NormalTok{(tidyverse)) \{}\FunctionTok{install.packages}\NormalTok{(}\StringTok{"tidyverse"}\NormalTok{); }\FunctionTok{require}\NormalTok{(tidyverse)\}}
\ControlFlowTok{if}\NormalTok{(}\SpecialCharTok{!}\FunctionTok{require}\NormalTok{(sfarrow)) \{}\FunctionTok{install.packages}\NormalTok{(}\StringTok{"sfarrow"}\NormalTok{); }\FunctionTok{require}\NormalTok{(sfarrow)\}}
\ControlFlowTok{if}\NormalTok{(}\SpecialCharTok{!}\FunctionTok{require}\NormalTok{(readxl)) \{}\FunctionTok{install.packages}\NormalTok{(}\StringTok{"readxl"}\NormalTok{); }\FunctionTok{require}\NormalTok{(readxl)\}}
\ControlFlowTok{if}\NormalTok{(}\SpecialCharTok{!}\FunctionTok{require}\NormalTok{(raster)) \{}\FunctionTok{install.packages}\NormalTok{(}\StringTok{"raster"}\NormalTok{); }\FunctionTok{require}\NormalTok{(raster)\}}
\ControlFlowTok{if}\NormalTok{(}\SpecialCharTok{!}\FunctionTok{require}\NormalTok{(fasterize)) \{}\FunctionTok{install.packages}\NormalTok{(}\StringTok{"fasterize"}\NormalTok{); }\FunctionTok{require}\NormalTok{(fasterize)\}}

\CommentTok{\# templates {-}{-}{-}{-}}
\NormalTok{template100}\OtherTok{=}\FunctionTok{rast}\NormalTok{(}\StringTok{"./Templates/TemplateRasters/LV100m\_10km.tif"}\NormalTok{)}
\NormalTok{template10}\OtherTok{=}\FunctionTok{rast}\NormalTok{(}\StringTok{"./Templates/TemplateRasters/LV10m\_10km.tif"}\NormalTok{)}
\NormalTok{rastrs10}\OtherTok{=}\FunctionTok{raster}\NormalTok{(template10)}

\NormalTok{nulls10}\OtherTok{=}\FunctionTok{rast}\NormalTok{(}\StringTok{"./Templates/TemplateRasters/nulls\_LV10m\_10km.tif"}\NormalTok{)}
\NormalTok{nulls100}\OtherTok{=}\FunctionTok{rast}\NormalTok{(}\StringTok{"./Templates/TemplateRasters/nulls\_LV100m\_10km.tif"}\NormalTok{)}


\CommentTok{\# simple landscape {-}{-}{-}{-}}
\NormalTok{simple\_landscape}\OtherTok{=}\FunctionTok{rast}\NormalTok{(}\StringTok{"RasterGrids\_10m/2024/Ainava\_vienk\_mask.tif"}\NormalTok{)}

\CommentTok{\# mvr {-}{-}{-}{-}}
\NormalTok{mvr}\OtherTok{=}\FunctionTok{st\_read\_parquet}\NormalTok{(}\StringTok{"./Geodata/2024/MVR/nogabali\_2024janv.parquet"}\NormalTok{)}
\NormalTok{mvr}\SpecialCharTok{$}\NormalTok{yes}\OtherTok{=}\DecValTok{1}

\CommentTok{\# clear cut mask {-}{-}{-}{-}}
\NormalTok{izcirtumi}\OtherTok{=}\NormalTok{mvr }\SpecialCharTok{\%\textgreater{}\%} 
 \FunctionTok{filter}\NormalTok{(zkat }\SpecialCharTok{\%in\%} \FunctionTok{c}\NormalTok{(}\StringTok{"12"}\NormalTok{,}\StringTok{"14"}\NormalTok{)) }\SpecialCharTok{\%\textgreater{}\%} 
\NormalTok{ dplyr}\SpecialCharTok{::}\FunctionTok{select}\NormalTok{(yes)}
\NormalTok{r\_izcirtumi\_mvr}\OtherTok{=}\FunctionTok{fasterize}\NormalTok{(izcirtumi,rastrs10,}\AttributeTok{field=}\StringTok{"yes"}\NormalTok{)}
\NormalTok{t\_izcirtumi\_mvr}\OtherTok{=}\FunctionTok{rast}\NormalTok{(r\_izcirtumi\_mvr)}
\FunctionTok{plot}\NormalTok{(t\_izcirtumi\_mvr)}

\NormalTok{tcl}\OtherTok{=}\FunctionTok{rast}\NormalTok{(}\StringTok{"./Geodata/2024/Trees/GFW/TreeCoverLoss\_v1\_12.tif"}\NormalTok{)}
\NormalTok{tcl2}\OtherTok{=}\FunctionTok{ifel}\NormalTok{(tcl}\SpecialCharTok{\textless{}}\DecValTok{20}\NormalTok{,}\DecValTok{0}\NormalTok{,}\DecValTok{1}\NormalTok{)}
\NormalTok{tclX}\OtherTok{=}\FunctionTok{cover}\NormalTok{(tcl2,nulls10)}
\FunctionTok{plot}\NormalTok{(tclX)}

\NormalTok{clearcut\_mask}\OtherTok{=}\FunctionTok{cover}\NormalTok{(t\_izcirtumi\_mvr,tclX,}
          \AttributeTok{filename=}\StringTok{"./RasterGrids\_10m/2024/Mask\_clearcuts.tif"}\NormalTok{,}
          \AttributeTok{overwrite=}\ConstantTok{TRUE}\NormalTok{)}
\FunctionTok{plot}\NormalTok{(clearcut\_mask)}

\FunctionTok{rm}\NormalTok{(izcirtumi)}
\FunctionTok{rm}\NormalTok{(r\_izcirtumi\_mvr)}
\FunctionTok{rm}\NormalTok{(t\_izcirtumi\_mvr)}
\FunctionTok{rm}\NormalTok{(tcl)}
\FunctionTok{rm}\NormalTok{(tcl2)}
\FunctionTok{rm}\NormalTok{(tclX)}

\CommentTok{\# ForestsQuant\_VolumeBorealDeciduousTotal{-}sum\_cell.tif  egv\_298 {-}{-}{-}{-}}

\NormalTok{sl\_visi}\OtherTok{=}\FunctionTok{c}\NormalTok{(}\StringTok{"4"}\NormalTok{,}\StringTok{"6"}\NormalTok{,}\StringTok{"8"}\NormalTok{,}\StringTok{"9"}\NormalTok{,}\StringTok{"19"}\NormalTok{,}\StringTok{"20"}\NormalTok{,}\StringTok{"21"}\NormalTok{,}\StringTok{"32"}\NormalTok{,}\StringTok{"35"}\NormalTok{,}\StringTok{"68"}\NormalTok{)}
\NormalTok{nogabali}\OtherTok{=}\NormalTok{mvr }\SpecialCharTok{\%\textgreater{}\%} 
 \FunctionTok{mutate}\NormalTok{(}\AttributeTok{SaurlapjuVKraja=}\FunctionTok{ifelse}\NormalTok{(s10 }\SpecialCharTok{\%in\%}\NormalTok{ sl\_visi, v10, }\DecValTok{0}\NormalTok{)}\SpecialCharTok{+}\FunctionTok{ifelse}\NormalTok{(s11 }\SpecialCharTok{\%in\%}\NormalTok{ sl\_visi,v11,}\DecValTok{0}\NormalTok{)}\SpecialCharTok{+}
      \FunctionTok{ifelse}\NormalTok{(s12 }\SpecialCharTok{\%in\%}\NormalTok{ sl\_visi, v12,}\DecValTok{0}\NormalTok{)}\SpecialCharTok{+}\FunctionTok{ifelse}\NormalTok{(s13 }\SpecialCharTok{\%in\%}\NormalTok{ sl\_visi,v13,}\DecValTok{0}\NormalTok{)}\SpecialCharTok{+}
      \FunctionTok{ifelse}\NormalTok{(s14 }\SpecialCharTok{\%in\%}\NormalTok{ sl\_visi, v14,}\DecValTok{0}\NormalTok{)) }\SpecialCharTok{\%\textgreater{}\%} 
 \FunctionTok{mutate}\NormalTok{(}\AttributeTok{SaurlapjuVKraja2=}\NormalTok{SaurlapjuVKraja}\SpecialCharTok{/}\DecValTok{10000}\SpecialCharTok{*}\DecValTok{10}\SpecialCharTok{*}\DecValTok{10}\NormalTok{) }\SpecialCharTok{\%\textgreater{}\%} 
 \FunctionTok{mutate}\NormalTok{(}\AttributeTok{SaurlapjuVKraja3=}\FunctionTok{ifelse}\NormalTok{(SaurlapjuVKraja2}\SpecialCharTok{\textgreater{}}\DecValTok{6}\NormalTok{,}\DecValTok{6}\NormalTok{,SaurlapjuVKraja2)) }\SpecialCharTok{\%\textgreater{}\%} 
 \FunctionTok{filter}\NormalTok{(}\SpecialCharTok{!}\FunctionTok{is.na}\NormalTok{(SaurlapjuVKraja2))}

\FunctionTok{par}\NormalTok{(}\AttributeTok{mfrow=}\FunctionTok{c}\NormalTok{(}\DecValTok{1}\NormalTok{,}\DecValTok{2}\NormalTok{))}
\FunctionTok{options}\NormalTok{(}\AttributeTok{scipen=}\DecValTok{999}\NormalTok{)}
\FunctionTok{hist}\NormalTok{(nogabali}\SpecialCharTok{$}\NormalTok{SaurlapjuVKraja2,}\AttributeTok{main=}\StringTok{"Original"}\NormalTok{,}\AttributeTok{xlab=}\StringTok{"All Boreal deciduous volume"}\NormalTok{)}
\FunctionTok{hist}\NormalTok{(nogabali}\SpecialCharTok{$}\NormalTok{SaurlapjuVKraja3,}\AttributeTok{main=}\StringTok{"Limited"}\NormalTok{,}\AttributeTok{xlab=}\StringTok{"All Boreal deciduous volume"}\NormalTok{)}
\FunctionTok{par}\NormalTok{(}\AttributeTok{mfrow=}\FunctionTok{c}\NormalTok{(}\DecValTok{1}\NormalTok{,}\DecValTok{1}\NormalTok{))}
\FunctionTok{options}\NormalTok{(}\AttributeTok{scipen=}\DecValTok{0}\NormalTok{)}

\NormalTok{p2i\_rez}\OtherTok{=}\FunctionTok{polygon2input}\NormalTok{(}\AttributeTok{vector\_data=}\NormalTok{nogabali,}
           \AttributeTok{template\_path =} \StringTok{"./Templates/TemplateRasters/LV10m\_10km.tif"}\NormalTok{,}
           \AttributeTok{out\_path =} \StringTok{"./RasterGrids\_10m/2024/"}\NormalTok{,}
           \AttributeTok{file\_name =} \StringTok{"ForestsQuant\_VolumeBorealDeciduousTotal.tif"}\NormalTok{,}
           \AttributeTok{value\_field =} \StringTok{"SaurlapjuVKraja3"}\NormalTok{,}
           \AttributeTok{fun=}\StringTok{"max"}\NormalTok{,}
           \AttributeTok{prepare=}\ConstantTok{FALSE}\NormalTok{,}
           \AttributeTok{restrict\_to =}\NormalTok{ clearcut\_mask,}
           \AttributeTok{restrict\_values =} \DecValTok{0}\NormalTok{,}
           \AttributeTok{plot\_result=}\ConstantTok{TRUE}\NormalTok{,}
           \AttributeTok{overwrite=}\ConstantTok{TRUE}\NormalTok{)}
\NormalTok{p2i\_rez}
\NormalTok{i2e\_rez}\OtherTok{=}\FunctionTok{input2egv}\NormalTok{(}\AttributeTok{input=}\StringTok{"./RasterGrids\_10m/2024/ForestsQuant\_VolumeBorealDeciduousTotal.tif"}\NormalTok{,}
         \AttributeTok{egv\_template =} \StringTok{"./Templates/TemplateRasters/LV100m\_10km.tif"}\NormalTok{,}
         \AttributeTok{summary\_function =} \StringTok{"sum"}\NormalTok{,}
         \AttributeTok{missing\_job =} \StringTok{"CoverOutput"}\NormalTok{,}
         \AttributeTok{output\_bg =} \StringTok{"./Templates/TemplateRasters/nulls\_LV100m\_10km.tif"}\NormalTok{,}
         \AttributeTok{outlocation =} \StringTok{"./RasterGrids\_100m/2024/RAW/"}\NormalTok{,}
         \AttributeTok{outfilename =} \StringTok{"ForestsQuant\_VolumeBorealDeciduousTotal{-}sum\_cell.tif"}\NormalTok{,}
         \AttributeTok{layername =} \StringTok{"egv\_298"}\NormalTok{,}
         \AttributeTok{plot\_final=}\ConstantTok{TRUE}\NormalTok{)}
\NormalTok{i2e\_rez}
\FunctionTok{rm}\NormalTok{(p2i\_rez)}
\FunctionTok{rm}\NormalTok{(nogabali)}
\FunctionTok{rm}\NormalTok{(sl\_visi)}
\FunctionTok{rm}\NormalTok{(i2e\_rez)}
\FunctionTok{unlink}\NormalTok{(}\StringTok{"./RasterGrids\_10m/2024/ForestsQuant\_VolumeBorealDeciduousTotal.tif"}\NormalTok{)}



\CommentTok{\# standardisation {-}{-}{-}{-}}
\ControlFlowTok{if}\NormalTok{(}\SpecialCharTok{!}\FunctionTok{require}\NormalTok{(terra)) \{}\FunctionTok{install.packages}\NormalTok{(}\StringTok{"terra"}\NormalTok{); }\FunctionTok{require}\NormalTok{(terra)\}}
\ControlFlowTok{if}\NormalTok{(}\SpecialCharTok{!}\FunctionTok{require}\NormalTok{(tidyverse)) \{}\FunctionTok{install.packages}\NormalTok{(}\StringTok{"tidyverse"}\NormalTok{); }\FunctionTok{require}\NormalTok{(tidyverse)\}}

\NormalTok{nosaukums}\OtherTok{=}\StringTok{"ForestsQuant\_VolumeBorealDeciduousTotal{-}sum\_cell.tif"}
\NormalTok{ielasisanas\_cels}\OtherTok{=}\FunctionTok{paste0}\NormalTok{(}\StringTok{"./RasterGrids\_100m/2024/RAW/"}\NormalTok{,nosaukums)}
\NormalTok{saglabasanas\_cels}\OtherTok{=}\FunctionTok{paste0}\NormalTok{(}\StringTok{"./RasterGrids\_100m/2024/Scaled/"}\NormalTok{,nosaukums)}
\NormalTok{slanis}\OtherTok{=}\FunctionTok{rast}\NormalTok{(ielasisanas\_cels)}
\NormalTok{videjais}\OtherTok{=}\FunctionTok{global}\NormalTok{(slanis,}\AttributeTok{fun=}\StringTok{"mean"}\NormalTok{,}\AttributeTok{na.rm=}\ConstantTok{TRUE}\NormalTok{)}
\NormalTok{centrets}\OtherTok{=}\NormalTok{slanis}\SpecialCharTok{{-}}\NormalTok{videjais[,}\DecValTok{1}\NormalTok{]}
\NormalTok{standartnovirze}\OtherTok{=}\NormalTok{terra}\SpecialCharTok{::}\FunctionTok{global}\NormalTok{(centrets,}\AttributeTok{fun=}\StringTok{"rms"}\NormalTok{,}\AttributeTok{na.rm=}\ConstantTok{TRUE}\NormalTok{)}
\NormalTok{merogots}\OtherTok{=}\NormalTok{centrets}\SpecialCharTok{/}\NormalTok{standartnovirze[,}\DecValTok{1}\NormalTok{]}
\FunctionTok{writeRaster}\NormalTok{(merogots,}
      \AttributeTok{filename=}\NormalTok{saglabasanas\_cels,}
      \AttributeTok{overwrite=}\ConstantTok{TRUE}\NormalTok{)}
\end{Highlighting}
\end{Shaded}

\section{ForestsQuant\_VolumeConiferous-sum\_cell}\label{ch06.299}

\textbf{filename:} \texttt{ForestsQuant\_VolumeConiferous-sum\_cell.tif}

\textbf{layername:} \texttt{egv\_299}

\textbf{English name:} Timber volume of Coniferous trees within the analysis cell (1
ha)

\textbf{Latvian name:} Skujkoku krāja analīzes šūnā (1 ha)

\textbf{Procedure:} Most EGVs describing forests are spatially restricted to areas outside
of clearcuts and dead stands. This mask is created using a combination of
the \hyperref[Ch04.01]{State Forest Service's
State Forest Registry} land category 12 and 14, and \hyperref[Ch04.09]{The
Global Forest Watch} pixels classified as lost tree canopy cover since
2020 (raster layer matching input, presence = 1, absence = 0).

This EGV is prepared based on the information of timber volume of coniferous
tree species (species codes: 1, 14, 22, 3, 13, 15, 23, 28; see tree species codes in
\hyperref[Ch01]{Terminology and acronyms}) in the inventoried forest stands - \hyperref[Ch04.01]{State
Forest Service's State Forest Registry}. This attribute has some extreme
values. We chose to limit them to the nearest integer showing only minimal
accumulation in histogram.

\includegraphics[width=0.8\linewidth]{./Figures/Histogramms/hist_egv299}

Resulting values at polygon geometries are rasterised with the workflow
\texttt{egvtools::polygon2input()}, restricting to pixels outside the clearcut mask. No
background values are assigned during rasterisation. The resulting layer is
then aggregated to EGV resolution using the workflow \texttt{egvtools::input2egv()} by calculating
sum of pixel values. After the aggregation, cells with no forest information
are filled with value 0. Finally, the layer is standardised by subtracting
the arithmetic mean and dividing by the root mean squared error.

\begin{Shaded}
\begin{Highlighting}[]
\CommentTok{\# libs {-}{-}{-}{-}}
\ControlFlowTok{if}\NormalTok{(}\SpecialCharTok{!}\FunctionTok{require}\NormalTok{(egvtools)) \{remotes}\SpecialCharTok{::}\FunctionTok{install\_github}\NormalTok{(}\StringTok{"aavotins/egvtools"}\NormalTok{); }\FunctionTok{require}\NormalTok{(egvtools)\}}
\ControlFlowTok{if}\NormalTok{(}\SpecialCharTok{!}\FunctionTok{require}\NormalTok{(terra)) \{}\FunctionTok{install.packages}\NormalTok{(}\StringTok{"terra"}\NormalTok{); }\FunctionTok{require}\NormalTok{(terra)\}}
\ControlFlowTok{if}\NormalTok{(}\SpecialCharTok{!}\FunctionTok{require}\NormalTok{(sf)) \{}\FunctionTok{install.packages}\NormalTok{(}\StringTok{"sf"}\NormalTok{); }\FunctionTok{require}\NormalTok{(sf)\}}
\ControlFlowTok{if}\NormalTok{(}\SpecialCharTok{!}\FunctionTok{require}\NormalTok{(tidyverse)) \{}\FunctionTok{install.packages}\NormalTok{(}\StringTok{"tidyverse"}\NormalTok{); }\FunctionTok{require}\NormalTok{(tidyverse)\}}
\ControlFlowTok{if}\NormalTok{(}\SpecialCharTok{!}\FunctionTok{require}\NormalTok{(sfarrow)) \{}\FunctionTok{install.packages}\NormalTok{(}\StringTok{"sfarrow"}\NormalTok{); }\FunctionTok{require}\NormalTok{(sfarrow)\}}
\ControlFlowTok{if}\NormalTok{(}\SpecialCharTok{!}\FunctionTok{require}\NormalTok{(readxl)) \{}\FunctionTok{install.packages}\NormalTok{(}\StringTok{"readxl"}\NormalTok{); }\FunctionTok{require}\NormalTok{(readxl)\}}
\ControlFlowTok{if}\NormalTok{(}\SpecialCharTok{!}\FunctionTok{require}\NormalTok{(raster)) \{}\FunctionTok{install.packages}\NormalTok{(}\StringTok{"raster"}\NormalTok{); }\FunctionTok{require}\NormalTok{(raster)\}}
\ControlFlowTok{if}\NormalTok{(}\SpecialCharTok{!}\FunctionTok{require}\NormalTok{(fasterize)) \{}\FunctionTok{install.packages}\NormalTok{(}\StringTok{"fasterize"}\NormalTok{); }\FunctionTok{require}\NormalTok{(fasterize)\}}

\CommentTok{\# templates {-}{-}{-}{-}}
\NormalTok{template100}\OtherTok{=}\FunctionTok{rast}\NormalTok{(}\StringTok{"./Templates/TemplateRasters/LV100m\_10km.tif"}\NormalTok{)}
\NormalTok{template10}\OtherTok{=}\FunctionTok{rast}\NormalTok{(}\StringTok{"./Templates/TemplateRasters/LV10m\_10km.tif"}\NormalTok{)}
\NormalTok{rastrs10}\OtherTok{=}\FunctionTok{raster}\NormalTok{(template10)}

\NormalTok{nulls10}\OtherTok{=}\FunctionTok{rast}\NormalTok{(}\StringTok{"./Templates/TemplateRasters/nulls\_LV10m\_10km.tif"}\NormalTok{)}
\NormalTok{nulls100}\OtherTok{=}\FunctionTok{rast}\NormalTok{(}\StringTok{"./Templates/TemplateRasters/nulls\_LV100m\_10km.tif"}\NormalTok{)}


\CommentTok{\# simple landscape {-}{-}{-}{-}}
\NormalTok{simple\_landscape}\OtherTok{=}\FunctionTok{rast}\NormalTok{(}\StringTok{"RasterGrids\_10m/2024/Ainava\_vienk\_mask.tif"}\NormalTok{)}

\CommentTok{\# mvr {-}{-}{-}{-}}
\NormalTok{mvr}\OtherTok{=}\FunctionTok{st\_read\_parquet}\NormalTok{(}\StringTok{"./Geodata/2024/MVR/nogabali\_2024janv.parquet"}\NormalTok{)}
\NormalTok{mvr}\SpecialCharTok{$}\NormalTok{yes}\OtherTok{=}\DecValTok{1}

\CommentTok{\# clear cut mask {-}{-}{-}{-}}
\NormalTok{izcirtumi}\OtherTok{=}\NormalTok{mvr }\SpecialCharTok{\%\textgreater{}\%} 
 \FunctionTok{filter}\NormalTok{(zkat }\SpecialCharTok{\%in\%} \FunctionTok{c}\NormalTok{(}\StringTok{"12"}\NormalTok{,}\StringTok{"14"}\NormalTok{)) }\SpecialCharTok{\%\textgreater{}\%} 
\NormalTok{ dplyr}\SpecialCharTok{::}\FunctionTok{select}\NormalTok{(yes)}
\NormalTok{r\_izcirtumi\_mvr}\OtherTok{=}\FunctionTok{fasterize}\NormalTok{(izcirtumi,rastrs10,}\AttributeTok{field=}\StringTok{"yes"}\NormalTok{)}
\NormalTok{t\_izcirtumi\_mvr}\OtherTok{=}\FunctionTok{rast}\NormalTok{(r\_izcirtumi\_mvr)}
\FunctionTok{plot}\NormalTok{(t\_izcirtumi\_mvr)}

\NormalTok{tcl}\OtherTok{=}\FunctionTok{rast}\NormalTok{(}\StringTok{"./Geodata/2024/Trees/GFW/TreeCoverLoss\_v1\_12.tif"}\NormalTok{)}
\NormalTok{tcl2}\OtherTok{=}\FunctionTok{ifel}\NormalTok{(tcl}\SpecialCharTok{\textless{}}\DecValTok{20}\NormalTok{,}\DecValTok{0}\NormalTok{,}\DecValTok{1}\NormalTok{)}
\NormalTok{tclX}\OtherTok{=}\FunctionTok{cover}\NormalTok{(tcl2,nulls10)}
\FunctionTok{plot}\NormalTok{(tclX)}

\NormalTok{clearcut\_mask}\OtherTok{=}\FunctionTok{cover}\NormalTok{(t\_izcirtumi\_mvr,tclX,}
          \AttributeTok{filename=}\StringTok{"./RasterGrids\_10m/2024/Mask\_clearcuts.tif"}\NormalTok{,}
          \AttributeTok{overwrite=}\ConstantTok{TRUE}\NormalTok{)}
\FunctionTok{plot}\NormalTok{(clearcut\_mask)}

\FunctionTok{rm}\NormalTok{(izcirtumi)}
\FunctionTok{rm}\NormalTok{(r\_izcirtumi\_mvr)}
\FunctionTok{rm}\NormalTok{(t\_izcirtumi\_mvr)}
\FunctionTok{rm}\NormalTok{(tcl)}
\FunctionTok{rm}\NormalTok{(tcl2)}
\FunctionTok{rm}\NormalTok{(tclX)}

\CommentTok{\# ForestsQuant\_VolumeConiferous{-}sum\_cell.tif    egv\_299 {-}{-}{-}{-}}

\NormalTok{skujkoki}\OtherTok{=}\FunctionTok{c}\NormalTok{(}\StringTok{"1"}\NormalTok{,}\StringTok{"14"}\NormalTok{,}\StringTok{"22"}\NormalTok{,}\StringTok{"3"}\NormalTok{,}\StringTok{"13"}\NormalTok{,}\StringTok{"15"}\NormalTok{,}\StringTok{"23"}\NormalTok{,}\StringTok{"28"}\NormalTok{)}
\NormalTok{nogabali}\OtherTok{=}\NormalTok{mvr }\SpecialCharTok{\%\textgreater{}\%} 
 \FunctionTok{mutate}\NormalTok{(}\AttributeTok{SkujkokuKraja=}\FunctionTok{ifelse}\NormalTok{(s10 }\SpecialCharTok{\%in\%}\NormalTok{ skujkoki, v10, }\DecValTok{0}\NormalTok{)}\SpecialCharTok{+}\FunctionTok{ifelse}\NormalTok{(s11 }\SpecialCharTok{\%in\%}\NormalTok{ skujkoki,v11,}\DecValTok{0}\NormalTok{)}\SpecialCharTok{+}
      \FunctionTok{ifelse}\NormalTok{(s12 }\SpecialCharTok{\%in\%}\NormalTok{ skujkoki, v12,}\DecValTok{0}\NormalTok{)}\SpecialCharTok{+}\FunctionTok{ifelse}\NormalTok{(s13 }\SpecialCharTok{\%in\%}\NormalTok{ skujkoki,v13,}\DecValTok{0}\NormalTok{)}\SpecialCharTok{+}
      \FunctionTok{ifelse}\NormalTok{(s14 }\SpecialCharTok{\%in\%}\NormalTok{ skujkoki, v14,}\DecValTok{0}\NormalTok{)) }\SpecialCharTok{\%\textgreater{}\%} 
 \FunctionTok{mutate}\NormalTok{(}\AttributeTok{SkujkokuKraja2=}\NormalTok{SkujkokuKraja}\SpecialCharTok{/}\DecValTok{10000}\SpecialCharTok{*}\DecValTok{10}\SpecialCharTok{*}\DecValTok{10}\NormalTok{) }\SpecialCharTok{\%\textgreater{}\%} 
 \FunctionTok{mutate}\NormalTok{(}\AttributeTok{SkujkokuKraja3=}\FunctionTok{ifelse}\NormalTok{(SkujkokuKraja2}\SpecialCharTok{\textgreater{}}\DecValTok{7}\NormalTok{,}\DecValTok{7}\NormalTok{,SkujkokuKraja2)) }\SpecialCharTok{\%\textgreater{}\%} 
 \FunctionTok{filter}\NormalTok{(}\SpecialCharTok{!}\FunctionTok{is.na}\NormalTok{(SkujkokuKraja2))}

\FunctionTok{par}\NormalTok{(}\AttributeTok{mfrow=}\FunctionTok{c}\NormalTok{(}\DecValTok{1}\NormalTok{,}\DecValTok{2}\NormalTok{))}
\FunctionTok{options}\NormalTok{(}\AttributeTok{scipen=}\DecValTok{999}\NormalTok{)}
\FunctionTok{hist}\NormalTok{(nogabali}\SpecialCharTok{$}\NormalTok{SkujkokuKraja2,}\AttributeTok{main=}\StringTok{"Original"}\NormalTok{,}\AttributeTok{xlab=}\StringTok{"Coniferous volume"}\NormalTok{)}
\FunctionTok{hist}\NormalTok{(nogabali}\SpecialCharTok{$}\NormalTok{SkujkokuKraja3,}\AttributeTok{main=}\StringTok{"Limited"}\NormalTok{,}\AttributeTok{xlab=}\StringTok{"Coniferous volume"}\NormalTok{)}
\FunctionTok{par}\NormalTok{(}\AttributeTok{mfrow=}\FunctionTok{c}\NormalTok{(}\DecValTok{1}\NormalTok{,}\DecValTok{1}\NormalTok{))}
\FunctionTok{options}\NormalTok{(}\AttributeTok{scipen=}\DecValTok{0}\NormalTok{)}

\NormalTok{p2i\_rez}\OtherTok{=}\FunctionTok{polygon2input}\NormalTok{(}\AttributeTok{vector\_data=}\NormalTok{nogabali,}
           \AttributeTok{template\_path =} \StringTok{"./Templates/TemplateRasters/LV10m\_10km.tif"}\NormalTok{,}
           \AttributeTok{out\_path =} \StringTok{"./RasterGrids\_10m/2024/"}\NormalTok{,}
           \AttributeTok{file\_name =} \StringTok{"ForestsQuant\_VolumeConiferous.tif"}\NormalTok{,}
           \AttributeTok{value\_field =} \StringTok{"SkujkokuKraja3"}\NormalTok{,}
           \AttributeTok{fun=}\StringTok{"max"}\NormalTok{,}
           \AttributeTok{prepare=}\ConstantTok{FALSE}\NormalTok{,}
           \AttributeTok{restrict\_to =}\NormalTok{ clearcut\_mask,}
           \AttributeTok{restrict\_values =} \DecValTok{0}\NormalTok{,}
           \AttributeTok{plot\_result=}\ConstantTok{TRUE}\NormalTok{,}
           \AttributeTok{overwrite=}\ConstantTok{TRUE}\NormalTok{)}
\NormalTok{p2i\_rez}
\NormalTok{i2e\_rez}\OtherTok{=}\FunctionTok{input2egv}\NormalTok{(}\AttributeTok{input=}\StringTok{"./RasterGrids\_10m/2024/ForestsQuant\_VolumeConiferous.tif"}\NormalTok{,}
         \AttributeTok{egv\_template =} \StringTok{"./Templates/TemplateRasters/LV100m\_10km.tif"}\NormalTok{,}
         \AttributeTok{summary\_function =} \StringTok{"sum"}\NormalTok{,}
         \AttributeTok{missing\_job =} \StringTok{"CoverOutput"}\NormalTok{,}
         \AttributeTok{output\_bg =} \StringTok{"./Templates/TemplateRasters/nulls\_LV100m\_10km.tif"}\NormalTok{,}
         \AttributeTok{outlocation =} \StringTok{"./RasterGrids\_100m/2024/RAW/"}\NormalTok{,}
         \AttributeTok{outfilename =} \StringTok{"ForestsQuant\_VolumeConiferous{-}sum\_cell.tif"}\NormalTok{,}
         \AttributeTok{layername =} \StringTok{"egv\_299"}\NormalTok{,}
         \AttributeTok{plot\_final=}\ConstantTok{TRUE}\NormalTok{)}
\NormalTok{i2e\_rez}
\FunctionTok{rm}\NormalTok{(p2i\_rez)}
\FunctionTok{rm}\NormalTok{(nogabali)}
\FunctionTok{rm}\NormalTok{(skujkoki)}
\FunctionTok{rm}\NormalTok{(i2e\_rez)}
\FunctionTok{unlink}\NormalTok{(}\StringTok{"./RasterGrids\_10m/2024/ForestsQuant\_VolumeConiferous.tif"}\NormalTok{)}


\CommentTok{\# standardisation {-}{-}{-}{-}}
\ControlFlowTok{if}\NormalTok{(}\SpecialCharTok{!}\FunctionTok{require}\NormalTok{(terra)) \{}\FunctionTok{install.packages}\NormalTok{(}\StringTok{"terra"}\NormalTok{); }\FunctionTok{require}\NormalTok{(terra)\}}
\ControlFlowTok{if}\NormalTok{(}\SpecialCharTok{!}\FunctionTok{require}\NormalTok{(tidyverse)) \{}\FunctionTok{install.packages}\NormalTok{(}\StringTok{"tidyverse"}\NormalTok{); }\FunctionTok{require}\NormalTok{(tidyverse)\}}

\NormalTok{nosaukums}\OtherTok{=}\StringTok{"ForestsQuant\_VolumeConiferous{-}sum\_cell.tif"}
\NormalTok{ielasisanas\_cels}\OtherTok{=}\FunctionTok{paste0}\NormalTok{(}\StringTok{"./RasterGrids\_100m/2024/RAW/"}\NormalTok{,nosaukums)}
\NormalTok{saglabasanas\_cels}\OtherTok{=}\FunctionTok{paste0}\NormalTok{(}\StringTok{"./RasterGrids\_100m/2024/Scaled/"}\NormalTok{,nosaukums)}
\NormalTok{slanis}\OtherTok{=}\FunctionTok{rast}\NormalTok{(ielasisanas\_cels)}
\NormalTok{videjais}\OtherTok{=}\FunctionTok{global}\NormalTok{(slanis,}\AttributeTok{fun=}\StringTok{"mean"}\NormalTok{,}\AttributeTok{na.rm=}\ConstantTok{TRUE}\NormalTok{)}
\NormalTok{centrets}\OtherTok{=}\NormalTok{slanis}\SpecialCharTok{{-}}\NormalTok{videjais[,}\DecValTok{1}\NormalTok{]}
\NormalTok{standartnovirze}\OtherTok{=}\NormalTok{terra}\SpecialCharTok{::}\FunctionTok{global}\NormalTok{(centrets,}\AttributeTok{fun=}\StringTok{"rms"}\NormalTok{,}\AttributeTok{na.rm=}\ConstantTok{TRUE}\NormalTok{)}
\NormalTok{merogots}\OtherTok{=}\NormalTok{centrets}\SpecialCharTok{/}\NormalTok{standartnovirze[,}\DecValTok{1}\NormalTok{]}
\FunctionTok{writeRaster}\NormalTok{(merogots,}
      \AttributeTok{filename=}\NormalTok{saglabasanas\_cels,}
      \AttributeTok{overwrite=}\ConstantTok{TRUE}\NormalTok{)}
\end{Highlighting}
\end{Shaded}

\section{ForestsQuant\_VolumeOak-sum\_cell}\label{ch06.300}

\textbf{filename:} \texttt{ForestsQuant\_VolumeOak-sum\_cell.tif}

\textbf{layername:} \texttt{egv\_300}

\textbf{English name:} Timber volume of Oaks within the analysis cell (1 ha)

\textbf{Latvian name:} Ozolu krāja analīzes šūnā (1 ha)

\textbf{Procedure:} Most EGVs describing forests are spatially restricted to areas outside
of clearcuts and dead stands. This mask is created using a combination of
the \hyperref[Ch04.01]{State Forest Service's
State Forest Registry} land category 12 and 14, and \hyperref[Ch04.09]{The
Global Forest Watch} pixels classified as lost tree canopy cover since
2020 (raster layer matching input, presence = 1, absence = 0).

This EGV is prepared based on the information of timber volume of oaks (species
codes: 10, 61; see tree species codes in \hyperref[Ch01]{Terminology and acronyms}) in
the inventoried forest stands - \hyperref[Ch04.01]{State Forest Service's State Forest
Registry}. This attribute has some extreme
values. We chose to limit them to the nearest integer showing only minimal
accumulation in histogram.

\includegraphics[width=0.8\linewidth]{./Figures/Histogramms/hist_egv300}

Resulting values at polygon geometries are rasterised with the workflow
\texttt{egvtools::polygon2input()}, restricting to pixels outside the clearcut mask. No
background values are assigned during rasterisation. The resulting layer is
then aggregated to EGV resolution using the workflow \texttt{egvtools::input2egv()} by calculating
sum of pixel values. After the aggregation, cells with no forest information
are filled with value 0. Finally, the layer is standardised by subtracting
the arithmetic mean and dividing by the root mean squared error.

\begin{Shaded}
\begin{Highlighting}[]
\CommentTok{\# libs {-}{-}{-}{-}}
\ControlFlowTok{if}\NormalTok{(}\SpecialCharTok{!}\FunctionTok{require}\NormalTok{(egvtools)) \{remotes}\SpecialCharTok{::}\FunctionTok{install\_github}\NormalTok{(}\StringTok{"aavotins/egvtools"}\NormalTok{); }\FunctionTok{require}\NormalTok{(egvtools)\}}
\ControlFlowTok{if}\NormalTok{(}\SpecialCharTok{!}\FunctionTok{require}\NormalTok{(terra)) \{}\FunctionTok{install.packages}\NormalTok{(}\StringTok{"terra"}\NormalTok{); }\FunctionTok{require}\NormalTok{(terra)\}}
\ControlFlowTok{if}\NormalTok{(}\SpecialCharTok{!}\FunctionTok{require}\NormalTok{(sf)) \{}\FunctionTok{install.packages}\NormalTok{(}\StringTok{"sf"}\NormalTok{); }\FunctionTok{require}\NormalTok{(sf)\}}
\ControlFlowTok{if}\NormalTok{(}\SpecialCharTok{!}\FunctionTok{require}\NormalTok{(tidyverse)) \{}\FunctionTok{install.packages}\NormalTok{(}\StringTok{"tidyverse"}\NormalTok{); }\FunctionTok{require}\NormalTok{(tidyverse)\}}
\ControlFlowTok{if}\NormalTok{(}\SpecialCharTok{!}\FunctionTok{require}\NormalTok{(sfarrow)) \{}\FunctionTok{install.packages}\NormalTok{(}\StringTok{"sfarrow"}\NormalTok{); }\FunctionTok{require}\NormalTok{(sfarrow)\}}
\ControlFlowTok{if}\NormalTok{(}\SpecialCharTok{!}\FunctionTok{require}\NormalTok{(readxl)) \{}\FunctionTok{install.packages}\NormalTok{(}\StringTok{"readxl"}\NormalTok{); }\FunctionTok{require}\NormalTok{(readxl)\}}
\ControlFlowTok{if}\NormalTok{(}\SpecialCharTok{!}\FunctionTok{require}\NormalTok{(raster)) \{}\FunctionTok{install.packages}\NormalTok{(}\StringTok{"raster"}\NormalTok{); }\FunctionTok{require}\NormalTok{(raster)\}}
\ControlFlowTok{if}\NormalTok{(}\SpecialCharTok{!}\FunctionTok{require}\NormalTok{(fasterize)) \{}\FunctionTok{install.packages}\NormalTok{(}\StringTok{"fasterize"}\NormalTok{); }\FunctionTok{require}\NormalTok{(fasterize)\}}

\CommentTok{\# templates {-}{-}{-}{-}}
\NormalTok{template100}\OtherTok{=}\FunctionTok{rast}\NormalTok{(}\StringTok{"./Templates/TemplateRasters/LV100m\_10km.tif"}\NormalTok{)}
\NormalTok{template10}\OtherTok{=}\FunctionTok{rast}\NormalTok{(}\StringTok{"./Templates/TemplateRasters/LV10m\_10km.tif"}\NormalTok{)}
\NormalTok{rastrs10}\OtherTok{=}\FunctionTok{raster}\NormalTok{(template10)}

\NormalTok{nulls10}\OtherTok{=}\FunctionTok{rast}\NormalTok{(}\StringTok{"./Templates/TemplateRasters/nulls\_LV10m\_10km.tif"}\NormalTok{)}
\NormalTok{nulls100}\OtherTok{=}\FunctionTok{rast}\NormalTok{(}\StringTok{"./Templates/TemplateRasters/nulls\_LV100m\_10km.tif"}\NormalTok{)}


\CommentTok{\# simple landscape {-}{-}{-}{-}}
\NormalTok{simple\_landscape}\OtherTok{=}\FunctionTok{rast}\NormalTok{(}\StringTok{"RasterGrids\_10m/2024/Ainava\_vienk\_mask.tif"}\NormalTok{)}

\CommentTok{\# mvr {-}{-}{-}{-}}
\NormalTok{mvr}\OtherTok{=}\FunctionTok{st\_read\_parquet}\NormalTok{(}\StringTok{"./Geodata/2024/MVR/nogabali\_2024janv.parquet"}\NormalTok{)}
\NormalTok{mvr}\SpecialCharTok{$}\NormalTok{yes}\OtherTok{=}\DecValTok{1}

\CommentTok{\# clear cut mask {-}{-}{-}{-}}
\NormalTok{izcirtumi}\OtherTok{=}\NormalTok{mvr }\SpecialCharTok{\%\textgreater{}\%} 
 \FunctionTok{filter}\NormalTok{(zkat }\SpecialCharTok{\%in\%} \FunctionTok{c}\NormalTok{(}\StringTok{"12"}\NormalTok{,}\StringTok{"14"}\NormalTok{)) }\SpecialCharTok{\%\textgreater{}\%} 
\NormalTok{ dplyr}\SpecialCharTok{::}\FunctionTok{select}\NormalTok{(yes)}
\NormalTok{r\_izcirtumi\_mvr}\OtherTok{=}\FunctionTok{fasterize}\NormalTok{(izcirtumi,rastrs10,}\AttributeTok{field=}\StringTok{"yes"}\NormalTok{)}
\NormalTok{t\_izcirtumi\_mvr}\OtherTok{=}\FunctionTok{rast}\NormalTok{(r\_izcirtumi\_mvr)}
\FunctionTok{plot}\NormalTok{(t\_izcirtumi\_mvr)}

\NormalTok{tcl}\OtherTok{=}\FunctionTok{rast}\NormalTok{(}\StringTok{"./Geodata/2024/Trees/GFW/TreeCoverLoss\_v1\_12.tif"}\NormalTok{)}
\NormalTok{tcl2}\OtherTok{=}\FunctionTok{ifel}\NormalTok{(tcl}\SpecialCharTok{\textless{}}\DecValTok{20}\NormalTok{,}\DecValTok{0}\NormalTok{,}\DecValTok{1}\NormalTok{)}
\NormalTok{tclX}\OtherTok{=}\FunctionTok{cover}\NormalTok{(tcl2,nulls10)}
\FunctionTok{plot}\NormalTok{(tclX)}

\NormalTok{clearcut\_mask}\OtherTok{=}\FunctionTok{cover}\NormalTok{(t\_izcirtumi\_mvr,tclX,}
          \AttributeTok{filename=}\StringTok{"./RasterGrids\_10m/2024/Mask\_clearcuts.tif"}\NormalTok{,}
          \AttributeTok{overwrite=}\ConstantTok{TRUE}\NormalTok{)}
\FunctionTok{plot}\NormalTok{(clearcut\_mask)}

\FunctionTok{rm}\NormalTok{(izcirtumi)}
\FunctionTok{rm}\NormalTok{(r\_izcirtumi\_mvr)}
\FunctionTok{rm}\NormalTok{(t\_izcirtumi\_mvr)}
\FunctionTok{rm}\NormalTok{(tcl)}
\FunctionTok{rm}\NormalTok{(tcl2)}
\FunctionTok{rm}\NormalTok{(tclX)}

\CommentTok{\# ForestsQuant\_VolumeOak{-}sum\_cell.tif   egv\_300 {-}{-}{-}{-}}
\NormalTok{ozoli}\OtherTok{=}\FunctionTok{c}\NormalTok{(}\StringTok{"10"}\NormalTok{,}\StringTok{"61"}\NormalTok{)}
\NormalTok{nogabali}\OtherTok{=}\NormalTok{mvr }\SpecialCharTok{\%\textgreater{}\%} 
 \FunctionTok{mutate}\NormalTok{(}\AttributeTok{OzoluKraja=}\FunctionTok{ifelse}\NormalTok{(s10 }\SpecialCharTok{\%in\%}\NormalTok{ ozoli, v10, }\DecValTok{0}\NormalTok{)}\SpecialCharTok{+}\FunctionTok{ifelse}\NormalTok{(s11 }\SpecialCharTok{\%in\%}\NormalTok{ ozoli,v11,}\DecValTok{0}\NormalTok{)}\SpecialCharTok{+}
      \FunctionTok{ifelse}\NormalTok{(s12 }\SpecialCharTok{\%in\%}\NormalTok{ ozoli, v12,}\DecValTok{0}\NormalTok{)}\SpecialCharTok{+}\FunctionTok{ifelse}\NormalTok{(s13 }\SpecialCharTok{\%in\%}\NormalTok{ ozoli,v13,}\DecValTok{0}\NormalTok{)}\SpecialCharTok{+}
      \FunctionTok{ifelse}\NormalTok{(s14 }\SpecialCharTok{\%in\%}\NormalTok{ ozoli, v14,}\DecValTok{0}\NormalTok{)) }\SpecialCharTok{\%\textgreater{}\%} 
 \FunctionTok{mutate}\NormalTok{(}\AttributeTok{OzoluKraja2=}\NormalTok{OzoluKraja}\SpecialCharTok{/}\DecValTok{10000}\SpecialCharTok{*}\DecValTok{10}\SpecialCharTok{*}\DecValTok{10}\NormalTok{) }\SpecialCharTok{\%\textgreater{}\%} 
 \FunctionTok{mutate}\NormalTok{(}\AttributeTok{OzoluKraja3=}\FunctionTok{ifelse}\NormalTok{(OzoluKraja2}\SpecialCharTok{\textgreater{}}\DecValTok{2}\NormalTok{,}\DecValTok{2}\NormalTok{,OzoluKraja2)) }\SpecialCharTok{\%\textgreater{}\%} 
 \FunctionTok{filter}\NormalTok{(}\SpecialCharTok{!}\FunctionTok{is.na}\NormalTok{(OzoluKraja2))}

\FunctionTok{par}\NormalTok{(}\AttributeTok{mfrow=}\FunctionTok{c}\NormalTok{(}\DecValTok{1}\NormalTok{,}\DecValTok{2}\NormalTok{))}
\FunctionTok{options}\NormalTok{(}\AttributeTok{scipen=}\DecValTok{999}\NormalTok{)}
\FunctionTok{hist}\NormalTok{(nogabali}\SpecialCharTok{$}\NormalTok{OzoluKraja2,}\AttributeTok{main=}\StringTok{"Original"}\NormalTok{,}\AttributeTok{xlab=}\StringTok{"Oak volume"}\NormalTok{)}
\FunctionTok{hist}\NormalTok{(nogabali}\SpecialCharTok{$}\NormalTok{OzoluKraja3,}\AttributeTok{main=}\StringTok{"Limited"}\NormalTok{,}\AttributeTok{xlab=}\StringTok{"Oak volume"}\NormalTok{)}
\FunctionTok{par}\NormalTok{(}\AttributeTok{mfrow=}\FunctionTok{c}\NormalTok{(}\DecValTok{1}\NormalTok{,}\DecValTok{1}\NormalTok{))}
\FunctionTok{options}\NormalTok{(}\AttributeTok{scipen=}\DecValTok{0}\NormalTok{)}

\NormalTok{p2i\_rez}\OtherTok{=}\FunctionTok{polygon2input}\NormalTok{(}\AttributeTok{vector\_data=}\NormalTok{nogabali,}
           \AttributeTok{template\_path =} \StringTok{"./Templates/TemplateRasters/LV10m\_10km.tif"}\NormalTok{,}
           \AttributeTok{out\_path =} \StringTok{"./RasterGrids\_10m/2024/"}\NormalTok{,}
           \AttributeTok{file\_name =} \StringTok{"ForestsQuant\_VolumeOak.tif"}\NormalTok{,}
           \AttributeTok{value\_field =} \StringTok{"OzoluKraja3"}\NormalTok{,}
           \AttributeTok{fun=}\StringTok{"max"}\NormalTok{,}
           \AttributeTok{prepare=}\ConstantTok{FALSE}\NormalTok{,}
           \AttributeTok{restrict\_to =}\NormalTok{ clearcut\_mask,}
           \AttributeTok{restrict\_values =} \DecValTok{0}\NormalTok{,}
           \AttributeTok{plot\_result=}\ConstantTok{TRUE}\NormalTok{,}
           \AttributeTok{overwrite=}\ConstantTok{TRUE}\NormalTok{)}
\NormalTok{p2i\_rez}
\NormalTok{i2e\_rez}\OtherTok{=}\FunctionTok{input2egv}\NormalTok{(}\AttributeTok{input=}\StringTok{"./RasterGrids\_10m/2024/ForestsQuant\_VolumeOak.tif"}\NormalTok{,}
         \AttributeTok{egv\_template =} \StringTok{"./Templates/TemplateRasters/LV100m\_10km.tif"}\NormalTok{,}
         \AttributeTok{summary\_function =} \StringTok{"sum"}\NormalTok{,}
         \AttributeTok{missing\_job =} \StringTok{"CoverOutput"}\NormalTok{,}
         \AttributeTok{output\_bg =} \StringTok{"./Templates/TemplateRasters/nulls\_LV100m\_10km.tif"}\NormalTok{,}
         \AttributeTok{outlocation =} \StringTok{"./RasterGrids\_100m/2024/RAW/"}\NormalTok{,}
         \AttributeTok{outfilename =} \StringTok{"ForestsQuant\_VolumeOak{-}sum\_cell.tif"}\NormalTok{,}
         \AttributeTok{layername =} \StringTok{"egv\_300"}\NormalTok{,}
         \AttributeTok{plot\_final=}\ConstantTok{TRUE}\NormalTok{)}
\NormalTok{i2e\_rez}
\FunctionTok{rm}\NormalTok{(p2i\_rez)}
\FunctionTok{rm}\NormalTok{(nogabali)}
\FunctionTok{rm}\NormalTok{(ozoli)}
\FunctionTok{rm}\NormalTok{(i2e\_rez)}
\FunctionTok{unlink}\NormalTok{(}\StringTok{"./RasterGrids\_10m/2024/ForestsQuant\_VolumeOak.tif"}\NormalTok{)}


\CommentTok{\# standardisation {-}{-}{-}{-}}
\ControlFlowTok{if}\NormalTok{(}\SpecialCharTok{!}\FunctionTok{require}\NormalTok{(terra)) \{}\FunctionTok{install.packages}\NormalTok{(}\StringTok{"terra"}\NormalTok{); }\FunctionTok{require}\NormalTok{(terra)\}}
\ControlFlowTok{if}\NormalTok{(}\SpecialCharTok{!}\FunctionTok{require}\NormalTok{(tidyverse)) \{}\FunctionTok{install.packages}\NormalTok{(}\StringTok{"tidyverse"}\NormalTok{); }\FunctionTok{require}\NormalTok{(tidyverse)\}}

\NormalTok{nosaukums}\OtherTok{=}\StringTok{"ForestsQuant\_VolumeOak{-}sum\_cell.tif"}
\NormalTok{ielasisanas\_cels}\OtherTok{=}\FunctionTok{paste0}\NormalTok{(}\StringTok{"./RasterGrids\_100m/2024/RAW/"}\NormalTok{,nosaukums)}
\NormalTok{saglabasanas\_cels}\OtherTok{=}\FunctionTok{paste0}\NormalTok{(}\StringTok{"./RasterGrids\_100m/2024/Scaled/"}\NormalTok{,nosaukums)}
\NormalTok{slanis}\OtherTok{=}\FunctionTok{rast}\NormalTok{(ielasisanas\_cels)}
\NormalTok{videjais}\OtherTok{=}\FunctionTok{global}\NormalTok{(slanis,}\AttributeTok{fun=}\StringTok{"mean"}\NormalTok{,}\AttributeTok{na.rm=}\ConstantTok{TRUE}\NormalTok{)}
\NormalTok{centrets}\OtherTok{=}\NormalTok{slanis}\SpecialCharTok{{-}}\NormalTok{videjais[,}\DecValTok{1}\NormalTok{]}
\NormalTok{standartnovirze}\OtherTok{=}\NormalTok{terra}\SpecialCharTok{::}\FunctionTok{global}\NormalTok{(centrets,}\AttributeTok{fun=}\StringTok{"rms"}\NormalTok{,}\AttributeTok{na.rm=}\ConstantTok{TRUE}\NormalTok{)}
\NormalTok{merogots}\OtherTok{=}\NormalTok{centrets}\SpecialCharTok{/}\NormalTok{standartnovirze[,}\DecValTok{1}\NormalTok{]}
\FunctionTok{writeRaster}\NormalTok{(merogots,}
      \AttributeTok{filename=}\NormalTok{saglabasanas\_cels,}
      \AttributeTok{overwrite=}\ConstantTok{TRUE}\NormalTok{)}
\end{Highlighting}
\end{Shaded}

\section{ForestsQuant\_VolumeOakMaple-sum\_cell}\label{ch06.301}

\textbf{filename:} \texttt{ForestsQuant\_VolumeOakMaple-sum\_cell.tif}

\textbf{layername:} \texttt{egv\_301}

\textbf{English name:} Timber volume of Oaks, Maples within the analysis cell (1 ha)

\textbf{Latvian name:} Ozolu, kļavu krāja analīzes šūnā (1 ha)

\textbf{Procedure:} Most EGVs describing forests are spatially restricted to areas outside
of clearcuts and dead stands. This mask is created using a combination of
the \hyperref[Ch04.01]{State Forest Service's
State Forest Registry} land category 12 and 14, and \hyperref[Ch04.09]{The
Global Forest Watch} pixels classified as lost tree canopy cover since
2020 (raster layer matching input, presence = 1, absence = 0).

This EGV is prepared based on the information of timber volume of oaks and
maples (species codes: 10, 61, 24, 63; see tree species codes in \hyperref[Ch01]{Terminology
and acronyms}) in the inventoried forest stands - \hyperref[Ch04.01]{State Forest Service's
State Forest Registry}. This attribute has some extreme
values. We chose to limit them to the nearest integer showing only minimal
accumulation in histogram.

\includegraphics[width=0.8\linewidth]{./Figures/Histogramms/hist_egv301}

Resulting values at polygon geometries are rasterised with the workflow
\texttt{egvtools::polygon2input()}, restricting to pixels outside the clearcut mask. No
background values are assigned during rasterisation. The resulting layer is
then aggregated to EGV resolution using the workflow \texttt{egvtools::input2egv()} by calculating
sum of pixel values. After the aggregation, cells with no forest information
are filled with value 0. Finally, the layer is standardised by subtracting
the arithmetic mean and dividing by the root mean squared error.

\begin{Shaded}
\begin{Highlighting}[]
\CommentTok{\# libs {-}{-}{-}{-}}
\ControlFlowTok{if}\NormalTok{(}\SpecialCharTok{!}\FunctionTok{require}\NormalTok{(egvtools)) \{remotes}\SpecialCharTok{::}\FunctionTok{install\_github}\NormalTok{(}\StringTok{"aavotins/egvtools"}\NormalTok{); }\FunctionTok{require}\NormalTok{(egvtools)\}}
\ControlFlowTok{if}\NormalTok{(}\SpecialCharTok{!}\FunctionTok{require}\NormalTok{(terra)) \{}\FunctionTok{install.packages}\NormalTok{(}\StringTok{"terra"}\NormalTok{); }\FunctionTok{require}\NormalTok{(terra)\}}
\ControlFlowTok{if}\NormalTok{(}\SpecialCharTok{!}\FunctionTok{require}\NormalTok{(sf)) \{}\FunctionTok{install.packages}\NormalTok{(}\StringTok{"sf"}\NormalTok{); }\FunctionTok{require}\NormalTok{(sf)\}}
\ControlFlowTok{if}\NormalTok{(}\SpecialCharTok{!}\FunctionTok{require}\NormalTok{(tidyverse)) \{}\FunctionTok{install.packages}\NormalTok{(}\StringTok{"tidyverse"}\NormalTok{); }\FunctionTok{require}\NormalTok{(tidyverse)\}}
\ControlFlowTok{if}\NormalTok{(}\SpecialCharTok{!}\FunctionTok{require}\NormalTok{(sfarrow)) \{}\FunctionTok{install.packages}\NormalTok{(}\StringTok{"sfarrow"}\NormalTok{); }\FunctionTok{require}\NormalTok{(sfarrow)\}}
\ControlFlowTok{if}\NormalTok{(}\SpecialCharTok{!}\FunctionTok{require}\NormalTok{(readxl)) \{}\FunctionTok{install.packages}\NormalTok{(}\StringTok{"readxl"}\NormalTok{); }\FunctionTok{require}\NormalTok{(readxl)\}}
\ControlFlowTok{if}\NormalTok{(}\SpecialCharTok{!}\FunctionTok{require}\NormalTok{(raster)) \{}\FunctionTok{install.packages}\NormalTok{(}\StringTok{"raster"}\NormalTok{); }\FunctionTok{require}\NormalTok{(raster)\}}
\ControlFlowTok{if}\NormalTok{(}\SpecialCharTok{!}\FunctionTok{require}\NormalTok{(fasterize)) \{}\FunctionTok{install.packages}\NormalTok{(}\StringTok{"fasterize"}\NormalTok{); }\FunctionTok{require}\NormalTok{(fasterize)\}}

\CommentTok{\# templates {-}{-}{-}{-}}
\NormalTok{template100}\OtherTok{=}\FunctionTok{rast}\NormalTok{(}\StringTok{"./Templates/TemplateRasters/LV100m\_10km.tif"}\NormalTok{)}
\NormalTok{template10}\OtherTok{=}\FunctionTok{rast}\NormalTok{(}\StringTok{"./Templates/TemplateRasters/LV10m\_10km.tif"}\NormalTok{)}
\NormalTok{rastrs10}\OtherTok{=}\FunctionTok{raster}\NormalTok{(template10)}

\NormalTok{nulls10}\OtherTok{=}\FunctionTok{rast}\NormalTok{(}\StringTok{"./Templates/TemplateRasters/nulls\_LV10m\_10km.tif"}\NormalTok{)}
\NormalTok{nulls100}\OtherTok{=}\FunctionTok{rast}\NormalTok{(}\StringTok{"./Templates/TemplateRasters/nulls\_LV100m\_10km.tif"}\NormalTok{)}


\CommentTok{\# simple landscape {-}{-}{-}{-}}
\NormalTok{simple\_landscape}\OtherTok{=}\FunctionTok{rast}\NormalTok{(}\StringTok{"RasterGrids\_10m/2024/Ainava\_vienk\_mask.tif"}\NormalTok{)}

\CommentTok{\# mvr {-}{-}{-}{-}}
\NormalTok{mvr}\OtherTok{=}\FunctionTok{st\_read\_parquet}\NormalTok{(}\StringTok{"./Geodata/2024/MVR/nogabali\_2024janv.parquet"}\NormalTok{)}
\NormalTok{mvr}\SpecialCharTok{$}\NormalTok{yes}\OtherTok{=}\DecValTok{1}

\CommentTok{\# clear cut mask {-}{-}{-}{-}}
\NormalTok{izcirtumi}\OtherTok{=}\NormalTok{mvr }\SpecialCharTok{\%\textgreater{}\%} 
 \FunctionTok{filter}\NormalTok{(zkat }\SpecialCharTok{\%in\%} \FunctionTok{c}\NormalTok{(}\StringTok{"12"}\NormalTok{,}\StringTok{"14"}\NormalTok{)) }\SpecialCharTok{\%\textgreater{}\%} 
\NormalTok{ dplyr}\SpecialCharTok{::}\FunctionTok{select}\NormalTok{(yes)}
\NormalTok{r\_izcirtumi\_mvr}\OtherTok{=}\FunctionTok{fasterize}\NormalTok{(izcirtumi,rastrs10,}\AttributeTok{field=}\StringTok{"yes"}\NormalTok{)}
\NormalTok{t\_izcirtumi\_mvr}\OtherTok{=}\FunctionTok{rast}\NormalTok{(r\_izcirtumi\_mvr)}
\FunctionTok{plot}\NormalTok{(t\_izcirtumi\_mvr)}

\NormalTok{tcl}\OtherTok{=}\FunctionTok{rast}\NormalTok{(}\StringTok{"./Geodata/2024/Trees/GFW/TreeCoverLoss\_v1\_12.tif"}\NormalTok{)}
\NormalTok{tcl2}\OtherTok{=}\FunctionTok{ifel}\NormalTok{(tcl}\SpecialCharTok{\textless{}}\DecValTok{20}\NormalTok{,}\DecValTok{0}\NormalTok{,}\DecValTok{1}\NormalTok{)}
\NormalTok{tclX}\OtherTok{=}\FunctionTok{cover}\NormalTok{(tcl2,nulls10)}
\FunctionTok{plot}\NormalTok{(tclX)}

\NormalTok{clearcut\_mask}\OtherTok{=}\FunctionTok{cover}\NormalTok{(t\_izcirtumi\_mvr,tclX,}
          \AttributeTok{filename=}\StringTok{"./RasterGrids\_10m/2024/Mask\_clearcuts.tif"}\NormalTok{,}
          \AttributeTok{overwrite=}\ConstantTok{TRUE}\NormalTok{)}
\FunctionTok{plot}\NormalTok{(clearcut\_mask)}

\FunctionTok{rm}\NormalTok{(izcirtumi)}
\FunctionTok{rm}\NormalTok{(r\_izcirtumi\_mvr)}
\FunctionTok{rm}\NormalTok{(t\_izcirtumi\_mvr)}
\FunctionTok{rm}\NormalTok{(tcl)}
\FunctionTok{rm}\NormalTok{(tcl2)}
\FunctionTok{rm}\NormalTok{(tclX)}

\CommentTok{\# ForestsQuant\_VolumeOakMaple{-}sum\_cell.tif  egv\_301 {-}{-}{-}{-}}
\NormalTok{ozolklavas}\OtherTok{=}\FunctionTok{c}\NormalTok{(}\StringTok{"10"}\NormalTok{,}\StringTok{"61"}\NormalTok{,}\StringTok{"24"}\NormalTok{,}\StringTok{"63"}\NormalTok{)}
\NormalTok{nogabali}\OtherTok{=}\NormalTok{mvr }\SpecialCharTok{\%\textgreater{}\%} 
 \FunctionTok{mutate}\NormalTok{(}\AttributeTok{OzolKlavuKraja=}\FunctionTok{ifelse}\NormalTok{(s10 }\SpecialCharTok{\%in\%}\NormalTok{ ozolklavas, v10, }\DecValTok{0}\NormalTok{)}\SpecialCharTok{+}\FunctionTok{ifelse}\NormalTok{(s11 }\SpecialCharTok{\%in\%}\NormalTok{ ozolklavas,v11,}\DecValTok{0}\NormalTok{)}\SpecialCharTok{+}
      \FunctionTok{ifelse}\NormalTok{(s12 }\SpecialCharTok{\%in\%}\NormalTok{ ozolklavas, v12,}\DecValTok{0}\NormalTok{)}\SpecialCharTok{+}\FunctionTok{ifelse}\NormalTok{(s13 }\SpecialCharTok{\%in\%}\NormalTok{ ozolklavas,v13,}\DecValTok{0}\NormalTok{)}\SpecialCharTok{+}
      \FunctionTok{ifelse}\NormalTok{(s14 }\SpecialCharTok{\%in\%}\NormalTok{ ozolklavas, v14,}\DecValTok{0}\NormalTok{)) }\SpecialCharTok{\%\textgreater{}\%} 
 \FunctionTok{mutate}\NormalTok{(}\AttributeTok{OzolKlavuKraja2=}\NormalTok{OzolKlavuKraja}\SpecialCharTok{/}\DecValTok{10000}\SpecialCharTok{*}\DecValTok{10}\SpecialCharTok{*}\DecValTok{10}\NormalTok{) }\SpecialCharTok{\%\textgreater{}\%} 
 \FunctionTok{mutate}\NormalTok{(}\AttributeTok{OzolKlavuKraja3=}\FunctionTok{ifelse}\NormalTok{(OzolKlavuKraja2}\SpecialCharTok{\textgreater{}}\DecValTok{3}\NormalTok{,}\DecValTok{3}\NormalTok{,OzolKlavuKraja2)) }\SpecialCharTok{\%\textgreater{}\%} 
 \FunctionTok{filter}\NormalTok{(}\SpecialCharTok{!}\FunctionTok{is.na}\NormalTok{(OzolKlavuKraja2))}

\FunctionTok{par}\NormalTok{(}\AttributeTok{mfrow=}\FunctionTok{c}\NormalTok{(}\DecValTok{1}\NormalTok{,}\DecValTok{2}\NormalTok{))}
\FunctionTok{options}\NormalTok{(}\AttributeTok{scipen=}\DecValTok{999}\NormalTok{)}
\FunctionTok{hist}\NormalTok{(nogabali}\SpecialCharTok{$}\NormalTok{OzolKlavuKraja2,}\AttributeTok{main=}\StringTok{"Original"}\NormalTok{,}\AttributeTok{xlab=}\StringTok{"Oak and maple volume"}\NormalTok{)}
\FunctionTok{hist}\NormalTok{(nogabali}\SpecialCharTok{$}\NormalTok{OzolKlavuKraja3,}\AttributeTok{main=}\StringTok{"Limited"}\NormalTok{,}\AttributeTok{xlab=}\StringTok{"Oak and maple volume"}\NormalTok{)}
\FunctionTok{par}\NormalTok{(}\AttributeTok{mfrow=}\FunctionTok{c}\NormalTok{(}\DecValTok{1}\NormalTok{,}\DecValTok{1}\NormalTok{))}
\FunctionTok{options}\NormalTok{(}\AttributeTok{scipen=}\DecValTok{0}\NormalTok{)}

\NormalTok{p2i\_rez}\OtherTok{=}\FunctionTok{polygon2input}\NormalTok{(}\AttributeTok{vector\_data=}\NormalTok{nogabali,}
           \AttributeTok{template\_path =} \StringTok{"./Templates/TemplateRasters/LV10m\_10km.tif"}\NormalTok{,}
           \AttributeTok{out\_path =} \StringTok{"./RasterGrids\_10m/2024/"}\NormalTok{,}
           \AttributeTok{file\_name =} \StringTok{"ForestsQuant\_VolumeOakMaple.tif"}\NormalTok{,}
           \AttributeTok{value\_field =} \StringTok{"OzolKlavuKraja3"}\NormalTok{,}
           \AttributeTok{fun=}\StringTok{"max"}\NormalTok{,}
           \AttributeTok{prepare=}\ConstantTok{FALSE}\NormalTok{,}
           \AttributeTok{restrict\_to =}\NormalTok{ clearcut\_mask,}
           \AttributeTok{restrict\_values =} \DecValTok{0}\NormalTok{,}
           \AttributeTok{plot\_result=}\ConstantTok{TRUE}\NormalTok{,}
           \AttributeTok{overwrite=}\ConstantTok{TRUE}\NormalTok{)}
\NormalTok{p2i\_rez}
\NormalTok{i2e\_rez}\OtherTok{=}\FunctionTok{input2egv}\NormalTok{(}\AttributeTok{input=}\StringTok{"./RasterGrids\_10m/2024/ForestsQuant\_VolumeOakMaple.tif"}\NormalTok{,}
         \AttributeTok{egv\_template =} \StringTok{"./Templates/TemplateRasters/LV100m\_10km.tif"}\NormalTok{,}
         \AttributeTok{summary\_function =} \StringTok{"sum"}\NormalTok{,}
         \AttributeTok{missing\_job =} \StringTok{"CoverOutput"}\NormalTok{,}
         \AttributeTok{output\_bg =} \StringTok{"./Templates/TemplateRasters/nulls\_LV100m\_10km.tif"}\NormalTok{,}
         \AttributeTok{outlocation =} \StringTok{"./RasterGrids\_100m/2024/RAW/"}\NormalTok{,}
         \AttributeTok{outfilename =} \StringTok{"ForestsQuant\_VolumeOakMaple{-}sum\_cell.tif"}\NormalTok{,}
         \AttributeTok{layername =} \StringTok{"egv\_301"}\NormalTok{,}
         \AttributeTok{plot\_final=}\ConstantTok{TRUE}\NormalTok{)}
\NormalTok{i2e\_rez}
\FunctionTok{rm}\NormalTok{(p2i\_rez)}
\FunctionTok{rm}\NormalTok{(nogabali)}
\FunctionTok{rm}\NormalTok{(ozolklavas)}
\FunctionTok{rm}\NormalTok{(i2e\_rez)}
\FunctionTok{unlink}\NormalTok{(}\StringTok{"./RasterGrids\_10m/2024/ForestsQuant\_VolumeOakMaple.tif"}\NormalTok{)}


\CommentTok{\# standardisation {-}{-}{-}{-}}
\ControlFlowTok{if}\NormalTok{(}\SpecialCharTok{!}\FunctionTok{require}\NormalTok{(terra)) \{}\FunctionTok{install.packages}\NormalTok{(}\StringTok{"terra"}\NormalTok{); }\FunctionTok{require}\NormalTok{(terra)\}}
\ControlFlowTok{if}\NormalTok{(}\SpecialCharTok{!}\FunctionTok{require}\NormalTok{(tidyverse)) \{}\FunctionTok{install.packages}\NormalTok{(}\StringTok{"tidyverse"}\NormalTok{); }\FunctionTok{require}\NormalTok{(tidyverse)\}}

\NormalTok{nosaukums}\OtherTok{=}\StringTok{"ForestsQuant\_VolumeOakMaple{-}sum\_cell.tif"}
\NormalTok{ielasisanas\_cels}\OtherTok{=}\FunctionTok{paste0}\NormalTok{(}\StringTok{"./RasterGrids\_100m/2024/RAW/"}\NormalTok{,nosaukums)}
\NormalTok{saglabasanas\_cels}\OtherTok{=}\FunctionTok{paste0}\NormalTok{(}\StringTok{"./RasterGrids\_100m/2024/Scaled/"}\NormalTok{,nosaukums)}
\NormalTok{slanis}\OtherTok{=}\FunctionTok{rast}\NormalTok{(ielasisanas\_cels)}
\NormalTok{videjais}\OtherTok{=}\FunctionTok{global}\NormalTok{(slanis,}\AttributeTok{fun=}\StringTok{"mean"}\NormalTok{,}\AttributeTok{na.rm=}\ConstantTok{TRUE}\NormalTok{)}
\NormalTok{centrets}\OtherTok{=}\NormalTok{slanis}\SpecialCharTok{{-}}\NormalTok{videjais[,}\DecValTok{1}\NormalTok{]}
\NormalTok{standartnovirze}\OtherTok{=}\NormalTok{terra}\SpecialCharTok{::}\FunctionTok{global}\NormalTok{(centrets,}\AttributeTok{fun=}\StringTok{"rms"}\NormalTok{,}\AttributeTok{na.rm=}\ConstantTok{TRUE}\NormalTok{)}
\NormalTok{merogots}\OtherTok{=}\NormalTok{centrets}\SpecialCharTok{/}\NormalTok{standartnovirze[,}\DecValTok{1}\NormalTok{]}
\FunctionTok{writeRaster}\NormalTok{(merogots,}
      \AttributeTok{filename=}\NormalTok{saglabasanas\_cels,}
      \AttributeTok{overwrite=}\ConstantTok{TRUE}\NormalTok{)}
\end{Highlighting}
\end{Shaded}

\section{ForestsQuant\_VolumePine-sum\_cell}\label{ch06.302}

\textbf{filename:} \texttt{ForestsQuant\_VolumePine-sum\_cell.tif}

\textbf{layername:} \texttt{egv\_302}

\textbf{English name:} Timber volume of Pines within the analysis cell (1 ha)

\textbf{Latvian name:} Priežu krāja analīzes šūnā (1 ha)

\textbf{Procedure:} Most EGVs describing forests are spatially restricted to areas outside
of clearcuts and dead stands. This mask is created using a combination of
the \hyperref[Ch04.01]{State Forest Service's
State Forest Registry} land category 12 and 14, and \hyperref[Ch04.09]{The
Global Forest Watch} pixels classified as lost tree canopy cover since
2020 (raster layer matching input, presence = 1, absence = 0).

This EGV is prepared based on the information of timber volume of pines
(species codes: 1, 14, 22; see tree species codes in \hyperref[Ch01]{Terminology and
acronyms}) in the inventoried forest stands - \hyperref[Ch04.01]{State Forest Service's
State Forest Registry}. This attribute has some extreme
values. We chose to limit them to the nearest integer showing only minimal
accumulation in histogram.

\includegraphics[width=0.8\linewidth]{./Figures/Histogramms/hist_egv302}

Resulting values at polygon geometries are rasterised with the workflow
\texttt{egvtools::polygon2input()}, restricting to pixels outside the clearcut mask. No
background values are assigned during rasterisation. The resulting layer is
then aggregated to EGV resolution using the workflow \texttt{egvtools::input2egv()} by calculating
sum of pixel values. After the aggregation, cells with no forest information
are filled with value 0. Finally, the layer is standardised by subtracting
the arithmetic mean and dividing by the root mean squared error.

\begin{Shaded}
\begin{Highlighting}[]
\CommentTok{\# libs {-}{-}{-}{-}}
\ControlFlowTok{if}\NormalTok{(}\SpecialCharTok{!}\FunctionTok{require}\NormalTok{(egvtools)) \{remotes}\SpecialCharTok{::}\FunctionTok{install\_github}\NormalTok{(}\StringTok{"aavotins/egvtools"}\NormalTok{); }\FunctionTok{require}\NormalTok{(egvtools)\}}
\ControlFlowTok{if}\NormalTok{(}\SpecialCharTok{!}\FunctionTok{require}\NormalTok{(terra)) \{}\FunctionTok{install.packages}\NormalTok{(}\StringTok{"terra"}\NormalTok{); }\FunctionTok{require}\NormalTok{(terra)\}}
\ControlFlowTok{if}\NormalTok{(}\SpecialCharTok{!}\FunctionTok{require}\NormalTok{(sf)) \{}\FunctionTok{install.packages}\NormalTok{(}\StringTok{"sf"}\NormalTok{); }\FunctionTok{require}\NormalTok{(sf)\}}
\ControlFlowTok{if}\NormalTok{(}\SpecialCharTok{!}\FunctionTok{require}\NormalTok{(tidyverse)) \{}\FunctionTok{install.packages}\NormalTok{(}\StringTok{"tidyverse"}\NormalTok{); }\FunctionTok{require}\NormalTok{(tidyverse)\}}
\ControlFlowTok{if}\NormalTok{(}\SpecialCharTok{!}\FunctionTok{require}\NormalTok{(sfarrow)) \{}\FunctionTok{install.packages}\NormalTok{(}\StringTok{"sfarrow"}\NormalTok{); }\FunctionTok{require}\NormalTok{(sfarrow)\}}
\ControlFlowTok{if}\NormalTok{(}\SpecialCharTok{!}\FunctionTok{require}\NormalTok{(readxl)) \{}\FunctionTok{install.packages}\NormalTok{(}\StringTok{"readxl"}\NormalTok{); }\FunctionTok{require}\NormalTok{(readxl)\}}
\ControlFlowTok{if}\NormalTok{(}\SpecialCharTok{!}\FunctionTok{require}\NormalTok{(raster)) \{}\FunctionTok{install.packages}\NormalTok{(}\StringTok{"raster"}\NormalTok{); }\FunctionTok{require}\NormalTok{(raster)\}}
\ControlFlowTok{if}\NormalTok{(}\SpecialCharTok{!}\FunctionTok{require}\NormalTok{(fasterize)) \{}\FunctionTok{install.packages}\NormalTok{(}\StringTok{"fasterize"}\NormalTok{); }\FunctionTok{require}\NormalTok{(fasterize)\}}

\CommentTok{\# templates {-}{-}{-}{-}}
\NormalTok{template100}\OtherTok{=}\FunctionTok{rast}\NormalTok{(}\StringTok{"./Templates/TemplateRasters/LV100m\_10km.tif"}\NormalTok{)}
\NormalTok{template10}\OtherTok{=}\FunctionTok{rast}\NormalTok{(}\StringTok{"./Templates/TemplateRasters/LV10m\_10km.tif"}\NormalTok{)}
\NormalTok{rastrs10}\OtherTok{=}\FunctionTok{raster}\NormalTok{(template10)}

\NormalTok{nulls10}\OtherTok{=}\FunctionTok{rast}\NormalTok{(}\StringTok{"./Templates/TemplateRasters/nulls\_LV10m\_10km.tif"}\NormalTok{)}
\NormalTok{nulls100}\OtherTok{=}\FunctionTok{rast}\NormalTok{(}\StringTok{"./Templates/TemplateRasters/nulls\_LV100m\_10km.tif"}\NormalTok{)}


\CommentTok{\# simple landscape {-}{-}{-}{-}}
\NormalTok{simple\_landscape}\OtherTok{=}\FunctionTok{rast}\NormalTok{(}\StringTok{"RasterGrids\_10m/2024/Ainava\_vienk\_mask.tif"}\NormalTok{)}

\CommentTok{\# mvr {-}{-}{-}{-}}
\NormalTok{mvr}\OtherTok{=}\FunctionTok{st\_read\_parquet}\NormalTok{(}\StringTok{"./Geodata/2024/MVR/nogabali\_2024janv.parquet"}\NormalTok{)}
\NormalTok{mvr}\SpecialCharTok{$}\NormalTok{yes}\OtherTok{=}\DecValTok{1}

\CommentTok{\# clear cut mask {-}{-}{-}{-}}
\NormalTok{izcirtumi}\OtherTok{=}\NormalTok{mvr }\SpecialCharTok{\%\textgreater{}\%} 
 \FunctionTok{filter}\NormalTok{(zkat }\SpecialCharTok{\%in\%} \FunctionTok{c}\NormalTok{(}\StringTok{"12"}\NormalTok{,}\StringTok{"14"}\NormalTok{)) }\SpecialCharTok{\%\textgreater{}\%} 
\NormalTok{ dplyr}\SpecialCharTok{::}\FunctionTok{select}\NormalTok{(yes)}
\NormalTok{r\_izcirtumi\_mvr}\OtherTok{=}\FunctionTok{fasterize}\NormalTok{(izcirtumi,rastrs10,}\AttributeTok{field=}\StringTok{"yes"}\NormalTok{)}
\NormalTok{t\_izcirtumi\_mvr}\OtherTok{=}\FunctionTok{rast}\NormalTok{(r\_izcirtumi\_mvr)}
\FunctionTok{plot}\NormalTok{(t\_izcirtumi\_mvr)}

\NormalTok{tcl}\OtherTok{=}\FunctionTok{rast}\NormalTok{(}\StringTok{"./Geodata/2024/Trees/GFW/TreeCoverLoss\_v1\_12.tif"}\NormalTok{)}
\NormalTok{tcl2}\OtherTok{=}\FunctionTok{ifel}\NormalTok{(tcl}\SpecialCharTok{\textless{}}\DecValTok{20}\NormalTok{,}\DecValTok{0}\NormalTok{,}\DecValTok{1}\NormalTok{)}
\NormalTok{tclX}\OtherTok{=}\FunctionTok{cover}\NormalTok{(tcl2,nulls10)}
\FunctionTok{plot}\NormalTok{(tclX)}

\NormalTok{clearcut\_mask}\OtherTok{=}\FunctionTok{cover}\NormalTok{(t\_izcirtumi\_mvr,tclX,}
          \AttributeTok{filename=}\StringTok{"./RasterGrids\_10m/2024/Mask\_clearcuts.tif"}\NormalTok{,}
          \AttributeTok{overwrite=}\ConstantTok{TRUE}\NormalTok{)}
\FunctionTok{plot}\NormalTok{(clearcut\_mask)}

\FunctionTok{rm}\NormalTok{(izcirtumi)}
\FunctionTok{rm}\NormalTok{(r\_izcirtumi\_mvr)}
\FunctionTok{rm}\NormalTok{(t\_izcirtumi\_mvr)}
\FunctionTok{rm}\NormalTok{(tcl)}
\FunctionTok{rm}\NormalTok{(tcl2)}
\FunctionTok{rm}\NormalTok{(tclX)}

\CommentTok{\# ForestsQuant\_VolumePine{-}sum\_cell.tif  egv\_302 {-}{-}{-}{-}}

\NormalTok{priedes}\OtherTok{=}\FunctionTok{c}\NormalTok{(}\StringTok{"1"}\NormalTok{,}\StringTok{"14"}\NormalTok{,}\StringTok{"22"}\NormalTok{)}
\NormalTok{nogabali}\OtherTok{=}\NormalTok{mvr }\SpecialCharTok{\%\textgreater{}\%} 
 \FunctionTok{mutate}\NormalTok{(}\AttributeTok{PriezuKraja=}\FunctionTok{ifelse}\NormalTok{(s10 }\SpecialCharTok{\%in\%}\NormalTok{ priedes, v10, }\DecValTok{0}\NormalTok{)}\SpecialCharTok{+}\FunctionTok{ifelse}\NormalTok{(s11 }\SpecialCharTok{\%in\%}\NormalTok{ priedes,v11,}\DecValTok{0}\NormalTok{)}\SpecialCharTok{+}
      \FunctionTok{ifelse}\NormalTok{(s12 }\SpecialCharTok{\%in\%}\NormalTok{ priedes, v12,}\DecValTok{0}\NormalTok{)}\SpecialCharTok{+}\FunctionTok{ifelse}\NormalTok{(s13 }\SpecialCharTok{\%in\%}\NormalTok{ priedes,v13,}\DecValTok{0}\NormalTok{)}\SpecialCharTok{+}
      \FunctionTok{ifelse}\NormalTok{(s14 }\SpecialCharTok{\%in\%}\NormalTok{ priedes, v14,}\DecValTok{0}\NormalTok{)) }\SpecialCharTok{\%\textgreater{}\%} 
 \FunctionTok{mutate}\NormalTok{(}\AttributeTok{PriezuKraja2=}\NormalTok{PriezuKraja}\SpecialCharTok{/}\DecValTok{10000}\SpecialCharTok{*}\DecValTok{10}\SpecialCharTok{*}\DecValTok{10}\NormalTok{) }\SpecialCharTok{\%\textgreater{}\%} 
 \FunctionTok{mutate}\NormalTok{(}\AttributeTok{PriezuKraja3=}\FunctionTok{ifelse}\NormalTok{(PriezuKraja2}\SpecialCharTok{\textgreater{}}\DecValTok{6}\NormalTok{,}\DecValTok{6}\NormalTok{,PriezuKraja2)) }\SpecialCharTok{\%\textgreater{}\%} 
 \FunctionTok{filter}\NormalTok{(}\SpecialCharTok{!}\FunctionTok{is.na}\NormalTok{(PriezuKraja2))}

\FunctionTok{par}\NormalTok{(}\AttributeTok{mfrow=}\FunctionTok{c}\NormalTok{(}\DecValTok{1}\NormalTok{,}\DecValTok{2}\NormalTok{))}
\FunctionTok{options}\NormalTok{(}\AttributeTok{scipen=}\DecValTok{999}\NormalTok{)}
\FunctionTok{hist}\NormalTok{(nogabali}\SpecialCharTok{$}\NormalTok{PriezuKraja2,}\AttributeTok{main=}\StringTok{"Original"}\NormalTok{,}\AttributeTok{xlab=}\StringTok{"Pine volume"}\NormalTok{)}
\FunctionTok{hist}\NormalTok{(nogabali}\SpecialCharTok{$}\NormalTok{PriezuKraja3,}\AttributeTok{main=}\StringTok{"Limited"}\NormalTok{,}\AttributeTok{xlab=}\StringTok{"Pine volume"}\NormalTok{)}
\FunctionTok{par}\NormalTok{(}\AttributeTok{mfrow=}\FunctionTok{c}\NormalTok{(}\DecValTok{1}\NormalTok{,}\DecValTok{1}\NormalTok{))}
\FunctionTok{options}\NormalTok{(}\AttributeTok{scipen=}\DecValTok{0}\NormalTok{)}

\NormalTok{p2i\_rez}\OtherTok{=}\FunctionTok{polygon2input}\NormalTok{(}\AttributeTok{vector\_data=}\NormalTok{nogabali,}
           \AttributeTok{template\_path =} \StringTok{"./Templates/TemplateRasters/LV10m\_10km.tif"}\NormalTok{,}
           \AttributeTok{out\_path =} \StringTok{"./RasterGrids\_10m/2024/"}\NormalTok{,}
           \AttributeTok{file\_name =} \StringTok{"ForestsQuant\_VolumePine.tif"}\NormalTok{,}
           \AttributeTok{value\_field =} \StringTok{"PriezuKraja3"}\NormalTok{,}
           \AttributeTok{fun=}\StringTok{"max"}\NormalTok{,}
           \AttributeTok{prepare=}\ConstantTok{FALSE}\NormalTok{,}
           \AttributeTok{restrict\_to =}\NormalTok{ clearcut\_mask,}
           \AttributeTok{restrict\_values =} \DecValTok{0}\NormalTok{,}
           \AttributeTok{plot\_result=}\ConstantTok{TRUE}\NormalTok{,}
           \AttributeTok{overwrite=}\ConstantTok{TRUE}\NormalTok{)}
\NormalTok{p2i\_rez}
\NormalTok{i2e\_rez}\OtherTok{=}\FunctionTok{input2egv}\NormalTok{(}\AttributeTok{input=}\StringTok{"./RasterGrids\_10m/2024/ForestsQuant\_VolumePine.tif"}\NormalTok{,}
         \AttributeTok{egv\_template =} \StringTok{"./Templates/TemplateRasters/LV100m\_10km.tif"}\NormalTok{,}
         \AttributeTok{summary\_function =} \StringTok{"sum"}\NormalTok{,}
         \AttributeTok{missing\_job =} \StringTok{"CoverOutput"}\NormalTok{,}
         \AttributeTok{output\_bg =} \StringTok{"./Templates/TemplateRasters/nulls\_LV100m\_10km.tif"}\NormalTok{,}
         \AttributeTok{outlocation =} \StringTok{"./RasterGrids\_100m/2024/RAW/"}\NormalTok{,}
         \AttributeTok{outfilename =} \StringTok{"ForestsQuant\_VolumePine{-}sum\_cell.tif"}\NormalTok{,}
         \AttributeTok{layername =} \StringTok{"egv\_302"}\NormalTok{,}
         \AttributeTok{plot\_final=}\ConstantTok{TRUE}\NormalTok{)}
\NormalTok{i2e\_rez}
\FunctionTok{rm}\NormalTok{(p2i\_rez)}
\FunctionTok{rm}\NormalTok{(nogabali)}
\FunctionTok{rm}\NormalTok{(priedes)}
\FunctionTok{rm}\NormalTok{(i2e\_rez)}
\FunctionTok{unlink}\NormalTok{(}\StringTok{"./RasterGrids\_10m/2024/ForestsQuant\_VolumePine.tif"}\NormalTok{)}


\CommentTok{\# standardisation {-}{-}{-}{-}}
\ControlFlowTok{if}\NormalTok{(}\SpecialCharTok{!}\FunctionTok{require}\NormalTok{(terra)) \{}\FunctionTok{install.packages}\NormalTok{(}\StringTok{"terra"}\NormalTok{); }\FunctionTok{require}\NormalTok{(terra)\}}
\ControlFlowTok{if}\NormalTok{(}\SpecialCharTok{!}\FunctionTok{require}\NormalTok{(tidyverse)) \{}\FunctionTok{install.packages}\NormalTok{(}\StringTok{"tidyverse"}\NormalTok{); }\FunctionTok{require}\NormalTok{(tidyverse)\}}

\NormalTok{nosaukums}\OtherTok{=}\StringTok{"ForestsQuant\_VolumePine{-}sum\_cell.tif"}
\NormalTok{ielasisanas\_cels}\OtherTok{=}\FunctionTok{paste0}\NormalTok{(}\StringTok{"./RasterGrids\_100m/2024/RAW/"}\NormalTok{,nosaukums)}
\NormalTok{saglabasanas\_cels}\OtherTok{=}\FunctionTok{paste0}\NormalTok{(}\StringTok{"./RasterGrids\_100m/2024/Scaled/"}\NormalTok{,nosaukums)}
\NormalTok{slanis}\OtherTok{=}\FunctionTok{rast}\NormalTok{(ielasisanas\_cels)}
\NormalTok{videjais}\OtherTok{=}\FunctionTok{global}\NormalTok{(slanis,}\AttributeTok{fun=}\StringTok{"mean"}\NormalTok{,}\AttributeTok{na.rm=}\ConstantTok{TRUE}\NormalTok{)}
\NormalTok{centrets}\OtherTok{=}\NormalTok{slanis}\SpecialCharTok{{-}}\NormalTok{videjais[,}\DecValTok{1}\NormalTok{]}
\NormalTok{standartnovirze}\OtherTok{=}\NormalTok{terra}\SpecialCharTok{::}\FunctionTok{global}\NormalTok{(centrets,}\AttributeTok{fun=}\StringTok{"rms"}\NormalTok{,}\AttributeTok{na.rm=}\ConstantTok{TRUE}\NormalTok{)}
\NormalTok{merogots}\OtherTok{=}\NormalTok{centrets}\SpecialCharTok{/}\NormalTok{standartnovirze[,}\DecValTok{1}\NormalTok{]}
\FunctionTok{writeRaster}\NormalTok{(merogots,}
      \AttributeTok{filename=}\NormalTok{saglabasanas\_cels,}
      \AttributeTok{overwrite=}\ConstantTok{TRUE}\NormalTok{)}
\end{Highlighting}
\end{Shaded}

\section{ForestsQuant\_VolumeSpruce-sum\_cell}\label{ch06.303}

\textbf{filename:} \texttt{ForestsQuant\_VolumeSpruce-sum\_cell.tif}

\textbf{layername:} \texttt{egv\_303}

\textbf{English name:} Timber volume of Spruces within the analysis cell (1 ha)

\textbf{Latvian name:} Egļu krāja analīzes šūnā (1 ha)

\textbf{Procedure:} Most EGVs describing forests are spatially restricted to areas outside
of clearcuts and dead stands. This mask is created using a combination of
the \hyperref[Ch04.01]{State Forest Service's
State Forest Registry} land category 12 and 14, and \hyperref[Ch04.09]{The
Global Forest Watch} pixels classified as lost tree canopy cover since
2020 (raster layer matching input, presence = 1, absence = 0).

This EGV is prepared based on the information of timber volume of spruces\\
(species codes: 3, 13, 15, 23, 28; see tree species codes in \hyperref[Ch01]{Terminology and
acronyms}) in the inventoried forest stands - \hyperref[Ch04.01]{State Forest Service's
State Forest Registry}. This attribute has some extreme
values. We chose to limit them to the nearest integer showing only minimal
accumulation in histogram.

\includegraphics[width=0.8\linewidth]{./Figures/Histogramms/hist_egv303}

Resulting values at polygon geometries are rasterised with the workflow
\texttt{egvtools::polygon2input()}, restricting to pixels outside the clearcut mask. No
background values are assigned during rasterisation. The resulting layer is
then aggregated to EGV resolution using the workflow \texttt{egvtools::input2egv()} by calculating
sum of pixel values. After the aggregation, cells with no forest information
are filled with value 0. Finally, the layer is standardised by subtracting
the arithmetic mean and dividing by the root mean squared error.

\begin{Shaded}
\begin{Highlighting}[]
\CommentTok{\# libs {-}{-}{-}{-}}
\ControlFlowTok{if}\NormalTok{(}\SpecialCharTok{!}\FunctionTok{require}\NormalTok{(egvtools)) \{remotes}\SpecialCharTok{::}\FunctionTok{install\_github}\NormalTok{(}\StringTok{"aavotins/egvtools"}\NormalTok{); }\FunctionTok{require}\NormalTok{(egvtools)\}}
\ControlFlowTok{if}\NormalTok{(}\SpecialCharTok{!}\FunctionTok{require}\NormalTok{(terra)) \{}\FunctionTok{install.packages}\NormalTok{(}\StringTok{"terra"}\NormalTok{); }\FunctionTok{require}\NormalTok{(terra)\}}
\ControlFlowTok{if}\NormalTok{(}\SpecialCharTok{!}\FunctionTok{require}\NormalTok{(sf)) \{}\FunctionTok{install.packages}\NormalTok{(}\StringTok{"sf"}\NormalTok{); }\FunctionTok{require}\NormalTok{(sf)\}}
\ControlFlowTok{if}\NormalTok{(}\SpecialCharTok{!}\FunctionTok{require}\NormalTok{(tidyverse)) \{}\FunctionTok{install.packages}\NormalTok{(}\StringTok{"tidyverse"}\NormalTok{); }\FunctionTok{require}\NormalTok{(tidyverse)\}}
\ControlFlowTok{if}\NormalTok{(}\SpecialCharTok{!}\FunctionTok{require}\NormalTok{(sfarrow)) \{}\FunctionTok{install.packages}\NormalTok{(}\StringTok{"sfarrow"}\NormalTok{); }\FunctionTok{require}\NormalTok{(sfarrow)\}}
\ControlFlowTok{if}\NormalTok{(}\SpecialCharTok{!}\FunctionTok{require}\NormalTok{(readxl)) \{}\FunctionTok{install.packages}\NormalTok{(}\StringTok{"readxl"}\NormalTok{); }\FunctionTok{require}\NormalTok{(readxl)\}}
\ControlFlowTok{if}\NormalTok{(}\SpecialCharTok{!}\FunctionTok{require}\NormalTok{(raster)) \{}\FunctionTok{install.packages}\NormalTok{(}\StringTok{"raster"}\NormalTok{); }\FunctionTok{require}\NormalTok{(raster)\}}
\ControlFlowTok{if}\NormalTok{(}\SpecialCharTok{!}\FunctionTok{require}\NormalTok{(fasterize)) \{}\FunctionTok{install.packages}\NormalTok{(}\StringTok{"fasterize"}\NormalTok{); }\FunctionTok{require}\NormalTok{(fasterize)\}}

\CommentTok{\# templates {-}{-}{-}{-}}
\NormalTok{template100}\OtherTok{=}\FunctionTok{rast}\NormalTok{(}\StringTok{"./Templates/TemplateRasters/LV100m\_10km.tif"}\NormalTok{)}
\NormalTok{template10}\OtherTok{=}\FunctionTok{rast}\NormalTok{(}\StringTok{"./Templates/TemplateRasters/LV10m\_10km.tif"}\NormalTok{)}
\NormalTok{rastrs10}\OtherTok{=}\FunctionTok{raster}\NormalTok{(template10)}

\NormalTok{nulls10}\OtherTok{=}\FunctionTok{rast}\NormalTok{(}\StringTok{"./Templates/TemplateRasters/nulls\_LV10m\_10km.tif"}\NormalTok{)}
\NormalTok{nulls100}\OtherTok{=}\FunctionTok{rast}\NormalTok{(}\StringTok{"./Templates/TemplateRasters/nulls\_LV100m\_10km.tif"}\NormalTok{)}


\CommentTok{\# simple landscape {-}{-}{-}{-}}
\NormalTok{simple\_landscape}\OtherTok{=}\FunctionTok{rast}\NormalTok{(}\StringTok{"RasterGrids\_10m/2024/Ainava\_vienk\_mask.tif"}\NormalTok{)}

\CommentTok{\# mvr {-}{-}{-}{-}}
\NormalTok{mvr}\OtherTok{=}\FunctionTok{st\_read\_parquet}\NormalTok{(}\StringTok{"./Geodata/2024/MVR/nogabali\_2024janv.parquet"}\NormalTok{)}
\NormalTok{mvr}\SpecialCharTok{$}\NormalTok{yes}\OtherTok{=}\DecValTok{1}

\CommentTok{\# clear cut mask {-}{-}{-}{-}}
\NormalTok{izcirtumi}\OtherTok{=}\NormalTok{mvr }\SpecialCharTok{\%\textgreater{}\%} 
 \FunctionTok{filter}\NormalTok{(zkat }\SpecialCharTok{\%in\%} \FunctionTok{c}\NormalTok{(}\StringTok{"12"}\NormalTok{,}\StringTok{"14"}\NormalTok{)) }\SpecialCharTok{\%\textgreater{}\%} 
\NormalTok{ dplyr}\SpecialCharTok{::}\FunctionTok{select}\NormalTok{(yes)}
\NormalTok{r\_izcirtumi\_mvr}\OtherTok{=}\FunctionTok{fasterize}\NormalTok{(izcirtumi,rastrs10,}\AttributeTok{field=}\StringTok{"yes"}\NormalTok{)}
\NormalTok{t\_izcirtumi\_mvr}\OtherTok{=}\FunctionTok{rast}\NormalTok{(r\_izcirtumi\_mvr)}
\FunctionTok{plot}\NormalTok{(t\_izcirtumi\_mvr)}

\NormalTok{tcl}\OtherTok{=}\FunctionTok{rast}\NormalTok{(}\StringTok{"./Geodata/2024/Trees/GFW/TreeCoverLoss\_v1\_12.tif"}\NormalTok{)}
\NormalTok{tcl2}\OtherTok{=}\FunctionTok{ifel}\NormalTok{(tcl}\SpecialCharTok{\textless{}}\DecValTok{20}\NormalTok{,}\DecValTok{0}\NormalTok{,}\DecValTok{1}\NormalTok{)}
\NormalTok{tclX}\OtherTok{=}\FunctionTok{cover}\NormalTok{(tcl2,nulls10)}
\FunctionTok{plot}\NormalTok{(tclX)}

\NormalTok{clearcut\_mask}\OtherTok{=}\FunctionTok{cover}\NormalTok{(t\_izcirtumi\_mvr,tclX,}
          \AttributeTok{filename=}\StringTok{"./RasterGrids\_10m/2024/Mask\_clearcuts.tif"}\NormalTok{,}
          \AttributeTok{overwrite=}\ConstantTok{TRUE}\NormalTok{)}
\FunctionTok{plot}\NormalTok{(clearcut\_mask)}

\FunctionTok{rm}\NormalTok{(izcirtumi)}
\FunctionTok{rm}\NormalTok{(r\_izcirtumi\_mvr)}
\FunctionTok{rm}\NormalTok{(t\_izcirtumi\_mvr)}
\FunctionTok{rm}\NormalTok{(tcl)}
\FunctionTok{rm}\NormalTok{(tcl2)}
\FunctionTok{rm}\NormalTok{(tclX)}

\CommentTok{\# ForestsQuant\_VolumeSpruce{-}sum\_cell.tif    egv\_303 {-}{-}{-}{-}}

\NormalTok{egles}\OtherTok{=}\FunctionTok{c}\NormalTok{(}\StringTok{"3"}\NormalTok{,}\StringTok{"13"}\NormalTok{,}\StringTok{"15"}\NormalTok{,}\StringTok{"23"}\NormalTok{,}\StringTok{"28"}\NormalTok{)}
\NormalTok{nogabali}\OtherTok{=}\NormalTok{mvr }\SpecialCharTok{\%\textgreater{}\%} 
 \FunctionTok{mutate}\NormalTok{(}\AttributeTok{EgluKraja=}\FunctionTok{ifelse}\NormalTok{(s10 }\SpecialCharTok{\%in\%}\NormalTok{ egles, v10, }\DecValTok{0}\NormalTok{)}\SpecialCharTok{+}\FunctionTok{ifelse}\NormalTok{(s11 }\SpecialCharTok{\%in\%}\NormalTok{ egles,v11,}\DecValTok{0}\NormalTok{)}\SpecialCharTok{+}
      \FunctionTok{ifelse}\NormalTok{(s12 }\SpecialCharTok{\%in\%}\NormalTok{ egles, v12,}\DecValTok{0}\NormalTok{)}\SpecialCharTok{+}\FunctionTok{ifelse}\NormalTok{(s13 }\SpecialCharTok{\%in\%}\NormalTok{ egles,v13,}\DecValTok{0}\NormalTok{)}\SpecialCharTok{+}
      \FunctionTok{ifelse}\NormalTok{(s14 }\SpecialCharTok{\%in\%}\NormalTok{ egles, v14,}\DecValTok{0}\NormalTok{)) }\SpecialCharTok{\%\textgreater{}\%} 
 \FunctionTok{mutate}\NormalTok{(}\AttributeTok{EgluKraja2=}\NormalTok{EgluKraja}\SpecialCharTok{/}\DecValTok{10000}\SpecialCharTok{*}\DecValTok{10}\SpecialCharTok{*}\DecValTok{10}\NormalTok{) }\SpecialCharTok{\%\textgreater{}\%} 
 \FunctionTok{mutate}\NormalTok{(}\AttributeTok{EgluKraja3=}\FunctionTok{ifelse}\NormalTok{(EgluKraja2}\SpecialCharTok{\textgreater{}}\DecValTok{5}\NormalTok{,}\DecValTok{5}\NormalTok{,EgluKraja2)) }\SpecialCharTok{\%\textgreater{}\%} 
 \FunctionTok{filter}\NormalTok{(}\SpecialCharTok{!}\FunctionTok{is.na}\NormalTok{(EgluKraja2))}

\FunctionTok{par}\NormalTok{(}\AttributeTok{mfrow=}\FunctionTok{c}\NormalTok{(}\DecValTok{1}\NormalTok{,}\DecValTok{2}\NormalTok{))}
\FunctionTok{options}\NormalTok{(}\AttributeTok{scipen=}\DecValTok{999}\NormalTok{)}
\FunctionTok{hist}\NormalTok{(nogabali}\SpecialCharTok{$}\NormalTok{EgluKraja2,}\AttributeTok{main=}\StringTok{"Original"}\NormalTok{,}\AttributeTok{xlab=}\StringTok{"Spruce volume"}\NormalTok{)}
\FunctionTok{hist}\NormalTok{(nogabali}\SpecialCharTok{$}\NormalTok{EgluKraja3,}\AttributeTok{main=}\StringTok{"Limited"}\NormalTok{,}\AttributeTok{xlab=}\StringTok{"Spruce volume"}\NormalTok{)}
\FunctionTok{par}\NormalTok{(}\AttributeTok{mfrow=}\FunctionTok{c}\NormalTok{(}\DecValTok{1}\NormalTok{,}\DecValTok{1}\NormalTok{))}
\FunctionTok{options}\NormalTok{(}\AttributeTok{scipen=}\DecValTok{0}\NormalTok{)}

\NormalTok{p2i\_rez}\OtherTok{=}\FunctionTok{polygon2input}\NormalTok{(}\AttributeTok{vector\_data=}\NormalTok{nogabali,}
           \AttributeTok{template\_path =} \StringTok{"./Templates/TemplateRasters/LV10m\_10km.tif"}\NormalTok{,}
           \AttributeTok{out\_path =} \StringTok{"./RasterGrids\_10m/2024/"}\NormalTok{,}
           \AttributeTok{file\_name =} \StringTok{"ForestsQuant\_VolumeSpruce.tif"}\NormalTok{,}
           \AttributeTok{value\_field =} \StringTok{"EgluKraja3"}\NormalTok{,}
           \AttributeTok{fun=}\StringTok{"max"}\NormalTok{,}
           \AttributeTok{prepare=}\ConstantTok{FALSE}\NormalTok{,}
           \AttributeTok{restrict\_to =}\NormalTok{ clearcut\_mask,}
           \AttributeTok{restrict\_values =} \DecValTok{0}\NormalTok{,}
           \AttributeTok{plot\_result=}\ConstantTok{TRUE}\NormalTok{,}
           \AttributeTok{overwrite=}\ConstantTok{TRUE}\NormalTok{)}
\NormalTok{p2i\_rez}
\NormalTok{i2e\_rez}\OtherTok{=}\FunctionTok{input2egv}\NormalTok{(}\AttributeTok{input=}\StringTok{"./RasterGrids\_10m/2024/ForestsQuant\_VolumeSpruce.tif"}\NormalTok{,}
         \AttributeTok{egv\_template =} \StringTok{"./Templates/TemplateRasters/LV100m\_10km.tif"}\NormalTok{,}
         \AttributeTok{summary\_function =} \StringTok{"sum"}\NormalTok{,}
         \AttributeTok{missing\_job =} \StringTok{"CoverOutput"}\NormalTok{,}
         \AttributeTok{output\_bg =} \StringTok{"./Templates/TemplateRasters/nulls\_LV100m\_10km.tif"}\NormalTok{,}
         \AttributeTok{outlocation =} \StringTok{"./RasterGrids\_100m/2024/RAW/"}\NormalTok{,}
         \AttributeTok{outfilename =} \StringTok{"ForestsQuant\_VolumeSpruce{-}sum\_cell.tif"}\NormalTok{,}
         \AttributeTok{layername =} \StringTok{"egv\_303"}\NormalTok{,}
         \AttributeTok{plot\_final=}\ConstantTok{TRUE}\NormalTok{)}
\NormalTok{i2e\_rez}
\FunctionTok{rm}\NormalTok{(p2i\_rez)}
\FunctionTok{rm}\NormalTok{(nogabali)}
\FunctionTok{rm}\NormalTok{(egles)}
\FunctionTok{rm}\NormalTok{(i2e\_rez)}
\FunctionTok{unlink}\NormalTok{(}\StringTok{"./RasterGrids\_10m/2024/ForestsQuant\_VolumeSpruce.tif"}\NormalTok{)}


\CommentTok{\# standardisation {-}{-}{-}{-}}
\ControlFlowTok{if}\NormalTok{(}\SpecialCharTok{!}\FunctionTok{require}\NormalTok{(terra)) \{}\FunctionTok{install.packages}\NormalTok{(}\StringTok{"terra"}\NormalTok{); }\FunctionTok{require}\NormalTok{(terra)\}}
\ControlFlowTok{if}\NormalTok{(}\SpecialCharTok{!}\FunctionTok{require}\NormalTok{(tidyverse)) \{}\FunctionTok{install.packages}\NormalTok{(}\StringTok{"tidyverse"}\NormalTok{); }\FunctionTok{require}\NormalTok{(tidyverse)\}}

\NormalTok{nosaukums}\OtherTok{=}\StringTok{"ForestsQuant\_VolumeSpruce{-}sum\_cell.tif"}
\NormalTok{ielasisanas\_cels}\OtherTok{=}\FunctionTok{paste0}\NormalTok{(}\StringTok{"./RasterGrids\_100m/2024/RAW/"}\NormalTok{,nosaukums)}
\NormalTok{saglabasanas\_cels}\OtherTok{=}\FunctionTok{paste0}\NormalTok{(}\StringTok{"./RasterGrids\_100m/2024/Scaled/"}\NormalTok{,nosaukums)}
\NormalTok{slanis}\OtherTok{=}\FunctionTok{rast}\NormalTok{(ielasisanas\_cels)}
\NormalTok{videjais}\OtherTok{=}\FunctionTok{global}\NormalTok{(slanis,}\AttributeTok{fun=}\StringTok{"mean"}\NormalTok{,}\AttributeTok{na.rm=}\ConstantTok{TRUE}\NormalTok{)}
\NormalTok{centrets}\OtherTok{=}\NormalTok{slanis}\SpecialCharTok{{-}}\NormalTok{videjais[,}\DecValTok{1}\NormalTok{]}
\NormalTok{standartnovirze}\OtherTok{=}\NormalTok{terra}\SpecialCharTok{::}\FunctionTok{global}\NormalTok{(centrets,}\AttributeTok{fun=}\StringTok{"rms"}\NormalTok{,}\AttributeTok{na.rm=}\ConstantTok{TRUE}\NormalTok{)}
\NormalTok{merogots}\OtherTok{=}\NormalTok{centrets}\SpecialCharTok{/}\NormalTok{standartnovirze[,}\DecValTok{1}\NormalTok{]}
\FunctionTok{writeRaster}\NormalTok{(merogots,}
      \AttributeTok{filename=}\NormalTok{saglabasanas\_cels,}
      \AttributeTok{overwrite=}\ConstantTok{TRUE}\NormalTok{)}
\end{Highlighting}
\end{Shaded}

\section{ForestsQuant\_VolumeTemperateDeciduousTotal-sum\_cell}\label{ch06.304}

\textbf{filename:} \texttt{ForestsQuant\_VolumeTemperateDeciduousTotal-sum\_cell.tif}

\textbf{layername:} \texttt{egv\_304}

\textbf{English name:} Timber volume of Temperate Deciduous trees within the analysis
cell (1 ha)

\textbf{Latvian name:} Platlapju krāja analīzes šūnā (1 ha)

\textbf{Procedure:} Most EGVs describing forests are spatially restricted to areas outside
of clearcuts and dead stands. This mask is created using a combination of
the \hyperref[Ch04.01]{State Forest Service's
State Forest Registry} land category 12 and 14, and \hyperref[Ch04.09]{The
Global Forest Watch} pixels classified as lost tree canopy cover since
2020 (raster layer matching input, presence = 1, absence = 0).

This EGV is prepared based on the information of timber volume of temperate
deciduous tree species
(species codes: 10, 11, 12, 16, 17, 18, 24, 25, 26, 27, 29, 50, 61, 62, 63, 64,
65, 66, 67, 69; see tree species codes in \hyperref[Ch01]{Terminology and acronyms}) in
the inventoried forest stands - \hyperref[Ch04.01]{State Forest Service's State Forest
Registry}. This attribute has some extreme
values. We chose to limit them to the nearest integer showing only minimal
accumulation in histogram.

\includegraphics[width=0.8\linewidth]{./Figures/Histogramms/hist_egv304}

Resulting values at polygon geometries are rasterised with the workflow
\texttt{egvtools::polygon2input()}, restricting to pixels outside the clearcut mask. No
background values are assigned during rasterisation. The resulting layer is
then aggregated to EGV resolution using the workflow \texttt{egvtools::input2egv()} by calculating
sum of pixel values. After the aggregation, cells with no forest information
are filled with value 0. Finally, the layer is standardised by subtracting
the arithmetic mean and dividing by the root mean squared error.

\begin{Shaded}
\begin{Highlighting}[]
\CommentTok{\# libs {-}{-}{-}{-}}
\ControlFlowTok{if}\NormalTok{(}\SpecialCharTok{!}\FunctionTok{require}\NormalTok{(egvtools)) \{remotes}\SpecialCharTok{::}\FunctionTok{install\_github}\NormalTok{(}\StringTok{"aavotins/egvtools"}\NormalTok{); }\FunctionTok{require}\NormalTok{(egvtools)\}}
\ControlFlowTok{if}\NormalTok{(}\SpecialCharTok{!}\FunctionTok{require}\NormalTok{(terra)) \{}\FunctionTok{install.packages}\NormalTok{(}\StringTok{"terra"}\NormalTok{); }\FunctionTok{require}\NormalTok{(terra)\}}
\ControlFlowTok{if}\NormalTok{(}\SpecialCharTok{!}\FunctionTok{require}\NormalTok{(sf)) \{}\FunctionTok{install.packages}\NormalTok{(}\StringTok{"sf"}\NormalTok{); }\FunctionTok{require}\NormalTok{(sf)\}}
\ControlFlowTok{if}\NormalTok{(}\SpecialCharTok{!}\FunctionTok{require}\NormalTok{(tidyverse)) \{}\FunctionTok{install.packages}\NormalTok{(}\StringTok{"tidyverse"}\NormalTok{); }\FunctionTok{require}\NormalTok{(tidyverse)\}}
\ControlFlowTok{if}\NormalTok{(}\SpecialCharTok{!}\FunctionTok{require}\NormalTok{(sfarrow)) \{}\FunctionTok{install.packages}\NormalTok{(}\StringTok{"sfarrow"}\NormalTok{); }\FunctionTok{require}\NormalTok{(sfarrow)\}}
\ControlFlowTok{if}\NormalTok{(}\SpecialCharTok{!}\FunctionTok{require}\NormalTok{(readxl)) \{}\FunctionTok{install.packages}\NormalTok{(}\StringTok{"readxl"}\NormalTok{); }\FunctionTok{require}\NormalTok{(readxl)\}}
\ControlFlowTok{if}\NormalTok{(}\SpecialCharTok{!}\FunctionTok{require}\NormalTok{(raster)) \{}\FunctionTok{install.packages}\NormalTok{(}\StringTok{"raster"}\NormalTok{); }\FunctionTok{require}\NormalTok{(raster)\}}
\ControlFlowTok{if}\NormalTok{(}\SpecialCharTok{!}\FunctionTok{require}\NormalTok{(fasterize)) \{}\FunctionTok{install.packages}\NormalTok{(}\StringTok{"fasterize"}\NormalTok{); }\FunctionTok{require}\NormalTok{(fasterize)\}}

\CommentTok{\# templates {-}{-}{-}{-}}
\NormalTok{template100}\OtherTok{=}\FunctionTok{rast}\NormalTok{(}\StringTok{"./Templates/TemplateRasters/LV100m\_10km.tif"}\NormalTok{)}
\NormalTok{template10}\OtherTok{=}\FunctionTok{rast}\NormalTok{(}\StringTok{"./Templates/TemplateRasters/LV10m\_10km.tif"}\NormalTok{)}
\NormalTok{rastrs10}\OtherTok{=}\FunctionTok{raster}\NormalTok{(template10)}

\NormalTok{nulls10}\OtherTok{=}\FunctionTok{rast}\NormalTok{(}\StringTok{"./Templates/TemplateRasters/nulls\_LV10m\_10km.tif"}\NormalTok{)}
\NormalTok{nulls100}\OtherTok{=}\FunctionTok{rast}\NormalTok{(}\StringTok{"./Templates/TemplateRasters/nulls\_LV100m\_10km.tif"}\NormalTok{)}


\CommentTok{\# simple landscape {-}{-}{-}{-}}
\NormalTok{simple\_landscape}\OtherTok{=}\FunctionTok{rast}\NormalTok{(}\StringTok{"RasterGrids\_10m/2024/Ainava\_vienk\_mask.tif"}\NormalTok{)}

\CommentTok{\# mvr {-}{-}{-}{-}}
\NormalTok{mvr}\OtherTok{=}\FunctionTok{st\_read\_parquet}\NormalTok{(}\StringTok{"./Geodata/2024/MVR/nogabali\_2024janv.parquet"}\NormalTok{)}
\NormalTok{mvr}\SpecialCharTok{$}\NormalTok{yes}\OtherTok{=}\DecValTok{1}

\CommentTok{\# clear cut mask {-}{-}{-}{-}}
\NormalTok{izcirtumi}\OtherTok{=}\NormalTok{mvr }\SpecialCharTok{\%\textgreater{}\%} 
 \FunctionTok{filter}\NormalTok{(zkat }\SpecialCharTok{\%in\%} \FunctionTok{c}\NormalTok{(}\StringTok{"12"}\NormalTok{,}\StringTok{"14"}\NormalTok{)) }\SpecialCharTok{\%\textgreater{}\%} 
\NormalTok{ dplyr}\SpecialCharTok{::}\FunctionTok{select}\NormalTok{(yes)}
\NormalTok{r\_izcirtumi\_mvr}\OtherTok{=}\FunctionTok{fasterize}\NormalTok{(izcirtumi,rastrs10,}\AttributeTok{field=}\StringTok{"yes"}\NormalTok{)}
\NormalTok{t\_izcirtumi\_mvr}\OtherTok{=}\FunctionTok{rast}\NormalTok{(r\_izcirtumi\_mvr)}
\FunctionTok{plot}\NormalTok{(t\_izcirtumi\_mvr)}

\NormalTok{tcl}\OtherTok{=}\FunctionTok{rast}\NormalTok{(}\StringTok{"./Geodata/2024/Trees/GFW/TreeCoverLoss\_v1\_12.tif"}\NormalTok{)}
\NormalTok{tcl2}\OtherTok{=}\FunctionTok{ifel}\NormalTok{(tcl}\SpecialCharTok{\textless{}}\DecValTok{20}\NormalTok{,}\DecValTok{0}\NormalTok{,}\DecValTok{1}\NormalTok{)}
\NormalTok{tclX}\OtherTok{=}\FunctionTok{cover}\NormalTok{(tcl2,nulls10)}
\FunctionTok{plot}\NormalTok{(tclX)}

\NormalTok{clearcut\_mask}\OtherTok{=}\FunctionTok{cover}\NormalTok{(t\_izcirtumi\_mvr,tclX,}
          \AttributeTok{filename=}\StringTok{"./RasterGrids\_10m/2024/Mask\_clearcuts.tif"}\NormalTok{,}
          \AttributeTok{overwrite=}\ConstantTok{TRUE}\NormalTok{)}
\FunctionTok{plot}\NormalTok{(clearcut\_mask)}

\FunctionTok{rm}\NormalTok{(izcirtumi)}
\FunctionTok{rm}\NormalTok{(r\_izcirtumi\_mvr)}
\FunctionTok{rm}\NormalTok{(t\_izcirtumi\_mvr)}
\FunctionTok{rm}\NormalTok{(tcl)}
\FunctionTok{rm}\NormalTok{(tcl2)}
\FunctionTok{rm}\NormalTok{(tclX)}

\CommentTok{\# ForestsQuant\_VolumeTemperateDeciduousTotal{-}sum\_cell.tif   egv\_304 {-}{-}{-}{-}}

\NormalTok{platlapji}\OtherTok{=}\FunctionTok{c}\NormalTok{(}\StringTok{"10"}\NormalTok{,}\StringTok{"11"}\NormalTok{,}\StringTok{"12"}\NormalTok{,}\StringTok{"16"}\NormalTok{,}\StringTok{"17"}\NormalTok{,}\StringTok{"18"}\NormalTok{,}\StringTok{"24"}\NormalTok{,}\StringTok{"25"}\NormalTok{,}\StringTok{"26"}\NormalTok{,}\StringTok{"27"}\NormalTok{,}\StringTok{"29"}\NormalTok{,}\StringTok{"50"}\NormalTok{,}
      \StringTok{"61"}\NormalTok{,}\StringTok{"62"}\NormalTok{,}\StringTok{"63"}\NormalTok{,}\StringTok{"64"}\NormalTok{,}\StringTok{"65"}\NormalTok{,}\StringTok{"66"}\NormalTok{,}\StringTok{"67"}\NormalTok{,}\StringTok{"69"}\NormalTok{)}
\NormalTok{nogabali}\OtherTok{=}\NormalTok{mvr }\SpecialCharTok{\%\textgreater{}\%} 
 \FunctionTok{mutate}\NormalTok{(}\AttributeTok{PlatKraja=}\FunctionTok{ifelse}\NormalTok{(s10 }\SpecialCharTok{\%in\%}\NormalTok{ platlapji, v10, }\DecValTok{0}\NormalTok{)}\SpecialCharTok{+}\FunctionTok{ifelse}\NormalTok{(s11 }\SpecialCharTok{\%in\%}\NormalTok{ platlapji,v11,}\DecValTok{0}\NormalTok{)}\SpecialCharTok{+}
      \FunctionTok{ifelse}\NormalTok{(s12 }\SpecialCharTok{\%in\%}\NormalTok{ platlapji, v12,}\DecValTok{0}\NormalTok{)}\SpecialCharTok{+}\FunctionTok{ifelse}\NormalTok{(s13 }\SpecialCharTok{\%in\%}\NormalTok{ platlapji,v13,}\DecValTok{0}\NormalTok{)}\SpecialCharTok{+}
      \FunctionTok{ifelse}\NormalTok{(s14 }\SpecialCharTok{\%in\%}\NormalTok{ platlapji, v14,}\DecValTok{0}\NormalTok{)) }\SpecialCharTok{\%\textgreater{}\%} 
 \FunctionTok{mutate}\NormalTok{(}\AttributeTok{PlatKraja2=}\NormalTok{PlatKraja}\SpecialCharTok{/}\DecValTok{10000}\SpecialCharTok{*}\DecValTok{10}\SpecialCharTok{*}\DecValTok{10}\NormalTok{) }\SpecialCharTok{\%\textgreater{}\%} 
 \FunctionTok{mutate}\NormalTok{(}\AttributeTok{PlatKraja3=}\FunctionTok{ifelse}\NormalTok{(PlatKraja2}\SpecialCharTok{\textgreater{}}\DecValTok{4}\NormalTok{,}\DecValTok{4}\NormalTok{,PlatKraja2)) }\SpecialCharTok{\%\textgreater{}\%} 
 \FunctionTok{filter}\NormalTok{(}\SpecialCharTok{!}\FunctionTok{is.na}\NormalTok{(PlatKraja2))}

\FunctionTok{par}\NormalTok{(}\AttributeTok{mfrow=}\FunctionTok{c}\NormalTok{(}\DecValTok{1}\NormalTok{,}\DecValTok{2}\NormalTok{))}
\FunctionTok{options}\NormalTok{(}\AttributeTok{scipen=}\DecValTok{999}\NormalTok{)}
\FunctionTok{hist}\NormalTok{(nogabali}\SpecialCharTok{$}\NormalTok{PlatKraja2,}\AttributeTok{main=}\StringTok{"Original"}\NormalTok{,}\AttributeTok{xlab=}\StringTok{"Total volume}\SpecialCharTok{\textbackslash{}n}\StringTok{temperate deciduous"}\NormalTok{)}
\FunctionTok{hist}\NormalTok{(nogabali}\SpecialCharTok{$}\NormalTok{PlatKraja3,}\AttributeTok{main=}\StringTok{"Limited"}\NormalTok{,}\AttributeTok{xlab=}\StringTok{"Total volume}\SpecialCharTok{\textbackslash{}n}\StringTok{temperate deciduous"}\NormalTok{)}
\FunctionTok{par}\NormalTok{(}\AttributeTok{mfrow=}\FunctionTok{c}\NormalTok{(}\DecValTok{1}\NormalTok{,}\DecValTok{1}\NormalTok{))}
\FunctionTok{options}\NormalTok{(}\AttributeTok{scipen=}\DecValTok{0}\NormalTok{)}

\NormalTok{p2i\_rez}\OtherTok{=}\FunctionTok{polygon2input}\NormalTok{(}\AttributeTok{vector\_data=}\NormalTok{nogabali,}
           \AttributeTok{template\_path =} \StringTok{"./Templates/TemplateRasters/LV10m\_10km.tif"}\NormalTok{,}
           \AttributeTok{out\_path =} \StringTok{"./RasterGrids\_10m/2024/"}\NormalTok{,}
           \AttributeTok{file\_name =} \StringTok{"ForestsQuant\_VolumeTemperateDeciduousTotal.tif"}\NormalTok{,}
           \AttributeTok{value\_field =} \StringTok{"PlatKraja3"}\NormalTok{,}
           \AttributeTok{fun=}\StringTok{"max"}\NormalTok{,}
           \AttributeTok{prepare=}\ConstantTok{FALSE}\NormalTok{,}
           \AttributeTok{restrict\_to =}\NormalTok{ clearcut\_mask,}
           \AttributeTok{restrict\_values =} \DecValTok{0}\NormalTok{,}
           \AttributeTok{plot\_result=}\ConstantTok{TRUE}\NormalTok{,}
           \AttributeTok{overwrite=}\ConstantTok{TRUE}\NormalTok{)}
\NormalTok{p2i\_rez}
\NormalTok{i2e\_rez}\OtherTok{=}\FunctionTok{input2egv}\NormalTok{(}\AttributeTok{input=}\StringTok{"./RasterGrids\_10m/2024/ForestsQuant\_VolumeTemperateDeciduousTotal.tif"}\NormalTok{,}
         \AttributeTok{egv\_template =} \StringTok{"./Templates/TemplateRasters/LV100m\_10km.tif"}\NormalTok{,}
         \AttributeTok{summary\_function =} \StringTok{"sum"}\NormalTok{,}
         \AttributeTok{missing\_job =} \StringTok{"CoverOutput"}\NormalTok{,}
         \AttributeTok{output\_bg =} \StringTok{"./Templates/TemplateRasters/nulls\_LV100m\_10km.tif"}\NormalTok{,}
         \AttributeTok{outlocation =} \StringTok{"./RasterGrids\_100m/2024/RAW/"}\NormalTok{,}
         \AttributeTok{outfilename =} \StringTok{"ForestsQuant\_VolumeTemperateDeciduousTotal{-}sum\_cell.tif"}\NormalTok{,}
         \AttributeTok{layername =} \StringTok{"egv\_304"}\NormalTok{,}
         \AttributeTok{plot\_final=}\ConstantTok{TRUE}\NormalTok{)}
\NormalTok{i2e\_rez}
\FunctionTok{rm}\NormalTok{(p2i\_rez)}
\FunctionTok{rm}\NormalTok{(nogabali)}
\FunctionTok{rm}\NormalTok{(platlapji)}
\FunctionTok{rm}\NormalTok{(i2e\_rez)}
\FunctionTok{unlink}\NormalTok{(}\StringTok{"./RasterGrids\_10m/2024/ForestsQuant\_VolumeTemperateDeciduousTotal.tif"}\NormalTok{)}



\CommentTok{\# standardisation {-}{-}{-}{-}}
\ControlFlowTok{if}\NormalTok{(}\SpecialCharTok{!}\FunctionTok{require}\NormalTok{(terra)) \{}\FunctionTok{install.packages}\NormalTok{(}\StringTok{"terra"}\NormalTok{); }\FunctionTok{require}\NormalTok{(terra)\}}
\ControlFlowTok{if}\NormalTok{(}\SpecialCharTok{!}\FunctionTok{require}\NormalTok{(tidyverse)) \{}\FunctionTok{install.packages}\NormalTok{(}\StringTok{"tidyverse"}\NormalTok{); }\FunctionTok{require}\NormalTok{(tidyverse)\}}

\NormalTok{nosaukums}\OtherTok{=}\StringTok{"ForestsQuant\_VolumeTemperateDeciduousTotal{-}sum\_cell.tif"}
\NormalTok{ielasisanas\_cels}\OtherTok{=}\FunctionTok{paste0}\NormalTok{(}\StringTok{"./RasterGrids\_100m/2024/RAW/"}\NormalTok{,nosaukums)}
\NormalTok{saglabasanas\_cels}\OtherTok{=}\FunctionTok{paste0}\NormalTok{(}\StringTok{"./RasterGrids\_100m/2024/Scaled/"}\NormalTok{,nosaukums)}
\NormalTok{slanis}\OtherTok{=}\FunctionTok{rast}\NormalTok{(ielasisanas\_cels)}
\NormalTok{videjais}\OtherTok{=}\FunctionTok{global}\NormalTok{(slanis,}\AttributeTok{fun=}\StringTok{"mean"}\NormalTok{,}\AttributeTok{na.rm=}\ConstantTok{TRUE}\NormalTok{)}
\NormalTok{centrets}\OtherTok{=}\NormalTok{slanis}\SpecialCharTok{{-}}\NormalTok{videjais[,}\DecValTok{1}\NormalTok{]}
\NormalTok{standartnovirze}\OtherTok{=}\NormalTok{terra}\SpecialCharTok{::}\FunctionTok{global}\NormalTok{(centrets,}\AttributeTok{fun=}\StringTok{"rms"}\NormalTok{,}\AttributeTok{na.rm=}\ConstantTok{TRUE}\NormalTok{)}
\NormalTok{merogots}\OtherTok{=}\NormalTok{centrets}\SpecialCharTok{/}\NormalTok{standartnovirze[,}\DecValTok{1}\NormalTok{]}
\FunctionTok{writeRaster}\NormalTok{(merogots,}
      \AttributeTok{filename=}\NormalTok{saglabasanas\_cels,}
      \AttributeTok{overwrite=}\ConstantTok{TRUE}\NormalTok{)}
\end{Highlighting}
\end{Shaded}

\section{ForestsQuant\_VolumeTemperateWithoutOak-sum\_cell}\label{ch06.305}

\textbf{filename:} \texttt{ForestsQuant\_VolumeTemperateWithoutOak-sum\_cell.tif}

\textbf{layername:} \texttt{egv\_305}

\textbf{English name:} Timber volume of Temperate Deciduous trees (without oaks)
within the analysis cell (1 ha)

\textbf{Latvian name:} Platlapju (bez ozoliem) krāja analīzes šūnā (1 ha)

\textbf{Procedure:} Most EGVs describing forests are spatially restricted to areas outside
of clearcuts and dead stands. This mask is created using a combination of
the \hyperref[Ch04.01]{State Forest Service's
State Forest Registry} land category 12 and 14, and \hyperref[Ch04.09]{The
Global Forest Watch} pixels classified as lost tree canopy cover since
2020 (raster layer matching input, presence = 1, absence = 0).

This EGV is prepared based on the information of timber volume of temperate
deciduous tree species except oaks
(species codes: 11, 12, 16, 17, 18, 24, 25, 26, 27, 29, 50, 62, 63, 64, 65, 66,
67, 69; see tree species codes in \hyperref[Ch01]{Terminology and acronyms}) in the
inventoried forest stands - \hyperref[Ch04.01]{State Forest Service's State Forest
Registry}. This attribute has some extreme
values. We chose to limit them to the nearest integer showing only minimal
accumulation in histogram.

\includegraphics[width=0.8\linewidth]{./Figures/Histogramms/hist_egv305}

Resulting values at polygon geometries are rasterised with the workflow
\texttt{egvtools::polygon2input()}, restricting to pixels outside the clearcut mask. No
background values are assigned during rasterisation. The resulting layer is
then aggregated to EGV resolution using the workflow \texttt{egvtools::input2egv()} by calculating
sum of pixel values. After the aggregation, cells with no forest information
are filled with value 0. Finally, the layer is standardised by subtracting
the arithmetic mean and dividing by the root mean squared error.

\begin{Shaded}
\begin{Highlighting}[]
\CommentTok{\# libs {-}{-}{-}{-}}
\ControlFlowTok{if}\NormalTok{(}\SpecialCharTok{!}\FunctionTok{require}\NormalTok{(egvtools)) \{remotes}\SpecialCharTok{::}\FunctionTok{install\_github}\NormalTok{(}\StringTok{"aavotins/egvtools"}\NormalTok{); }\FunctionTok{require}\NormalTok{(egvtools)\}}
\ControlFlowTok{if}\NormalTok{(}\SpecialCharTok{!}\FunctionTok{require}\NormalTok{(terra)) \{}\FunctionTok{install.packages}\NormalTok{(}\StringTok{"terra"}\NormalTok{); }\FunctionTok{require}\NormalTok{(terra)\}}
\ControlFlowTok{if}\NormalTok{(}\SpecialCharTok{!}\FunctionTok{require}\NormalTok{(sf)) \{}\FunctionTok{install.packages}\NormalTok{(}\StringTok{"sf"}\NormalTok{); }\FunctionTok{require}\NormalTok{(sf)\}}
\ControlFlowTok{if}\NormalTok{(}\SpecialCharTok{!}\FunctionTok{require}\NormalTok{(tidyverse)) \{}\FunctionTok{install.packages}\NormalTok{(}\StringTok{"tidyverse"}\NormalTok{); }\FunctionTok{require}\NormalTok{(tidyverse)\}}
\ControlFlowTok{if}\NormalTok{(}\SpecialCharTok{!}\FunctionTok{require}\NormalTok{(sfarrow)) \{}\FunctionTok{install.packages}\NormalTok{(}\StringTok{"sfarrow"}\NormalTok{); }\FunctionTok{require}\NormalTok{(sfarrow)\}}
\ControlFlowTok{if}\NormalTok{(}\SpecialCharTok{!}\FunctionTok{require}\NormalTok{(readxl)) \{}\FunctionTok{install.packages}\NormalTok{(}\StringTok{"readxl"}\NormalTok{); }\FunctionTok{require}\NormalTok{(readxl)\}}
\ControlFlowTok{if}\NormalTok{(}\SpecialCharTok{!}\FunctionTok{require}\NormalTok{(raster)) \{}\FunctionTok{install.packages}\NormalTok{(}\StringTok{"raster"}\NormalTok{); }\FunctionTok{require}\NormalTok{(raster)\}}
\ControlFlowTok{if}\NormalTok{(}\SpecialCharTok{!}\FunctionTok{require}\NormalTok{(fasterize)) \{}\FunctionTok{install.packages}\NormalTok{(}\StringTok{"fasterize"}\NormalTok{); }\FunctionTok{require}\NormalTok{(fasterize)\}}

\CommentTok{\# templates {-}{-}{-}{-}}
\NormalTok{template100}\OtherTok{=}\FunctionTok{rast}\NormalTok{(}\StringTok{"./Templates/TemplateRasters/LV100m\_10km.tif"}\NormalTok{)}
\NormalTok{template10}\OtherTok{=}\FunctionTok{rast}\NormalTok{(}\StringTok{"./Templates/TemplateRasters/LV10m\_10km.tif"}\NormalTok{)}
\NormalTok{rastrs10}\OtherTok{=}\FunctionTok{raster}\NormalTok{(template10)}

\NormalTok{nulls10}\OtherTok{=}\FunctionTok{rast}\NormalTok{(}\StringTok{"./Templates/TemplateRasters/nulls\_LV10m\_10km.tif"}\NormalTok{)}
\NormalTok{nulls100}\OtherTok{=}\FunctionTok{rast}\NormalTok{(}\StringTok{"./Templates/TemplateRasters/nulls\_LV100m\_10km.tif"}\NormalTok{)}


\CommentTok{\# simple landscape {-}{-}{-}{-}}
\NormalTok{simple\_landscape}\OtherTok{=}\FunctionTok{rast}\NormalTok{(}\StringTok{"RasterGrids\_10m/2024/Ainava\_vienk\_mask.tif"}\NormalTok{)}

\CommentTok{\# mvr {-}{-}{-}{-}}
\NormalTok{mvr}\OtherTok{=}\FunctionTok{st\_read\_parquet}\NormalTok{(}\StringTok{"./Geodata/2024/MVR/nogabali\_2024janv.parquet"}\NormalTok{)}
\NormalTok{mvr}\SpecialCharTok{$}\NormalTok{yes}\OtherTok{=}\DecValTok{1}

\CommentTok{\# clear cut mask {-}{-}{-}{-}}
\NormalTok{izcirtumi}\OtherTok{=}\NormalTok{mvr }\SpecialCharTok{\%\textgreater{}\%} 
 \FunctionTok{filter}\NormalTok{(zkat }\SpecialCharTok{\%in\%} \FunctionTok{c}\NormalTok{(}\StringTok{"12"}\NormalTok{,}\StringTok{"14"}\NormalTok{)) }\SpecialCharTok{\%\textgreater{}\%} 
\NormalTok{ dplyr}\SpecialCharTok{::}\FunctionTok{select}\NormalTok{(yes)}
\NormalTok{r\_izcirtumi\_mvr}\OtherTok{=}\FunctionTok{fasterize}\NormalTok{(izcirtumi,rastrs10,}\AttributeTok{field=}\StringTok{"yes"}\NormalTok{)}
\NormalTok{t\_izcirtumi\_mvr}\OtherTok{=}\FunctionTok{rast}\NormalTok{(r\_izcirtumi\_mvr)}
\FunctionTok{plot}\NormalTok{(t\_izcirtumi\_mvr)}

\NormalTok{tcl}\OtherTok{=}\FunctionTok{rast}\NormalTok{(}\StringTok{"./Geodata/2024/Trees/GFW/TreeCoverLoss\_v1\_12.tif"}\NormalTok{)}
\NormalTok{tcl2}\OtherTok{=}\FunctionTok{ifel}\NormalTok{(tcl}\SpecialCharTok{\textless{}}\DecValTok{20}\NormalTok{,}\DecValTok{0}\NormalTok{,}\DecValTok{1}\NormalTok{)}
\NormalTok{tclX}\OtherTok{=}\FunctionTok{cover}\NormalTok{(tcl2,nulls10)}
\FunctionTok{plot}\NormalTok{(tclX)}

\NormalTok{clearcut\_mask}\OtherTok{=}\FunctionTok{cover}\NormalTok{(t\_izcirtumi\_mvr,tclX,}
          \AttributeTok{filename=}\StringTok{"./RasterGrids\_10m/2024/Mask\_clearcuts.tif"}\NormalTok{,}
          \AttributeTok{overwrite=}\ConstantTok{TRUE}\NormalTok{)}
\FunctionTok{plot}\NormalTok{(clearcut\_mask)}

\FunctionTok{rm}\NormalTok{(izcirtumi)}
\FunctionTok{rm}\NormalTok{(r\_izcirtumi\_mvr)}
\FunctionTok{rm}\NormalTok{(t\_izcirtumi\_mvr)}
\FunctionTok{rm}\NormalTok{(tcl)}
\FunctionTok{rm}\NormalTok{(tcl2)}
\FunctionTok{rm}\NormalTok{(tclX)}

\CommentTok{\# ForestsQuant\_VolumeTemperateWithoutOak{-}sum\_cell.tif   egv\_305 {-}{-}{-}{-}}
\NormalTok{neozoli}\OtherTok{=}\FunctionTok{c}\NormalTok{(}\StringTok{"11"}\NormalTok{,}\StringTok{"12"}\NormalTok{,}\StringTok{"16"}\NormalTok{,}\StringTok{"17"}\NormalTok{,}\StringTok{"18"}\NormalTok{,}\StringTok{"24"}\NormalTok{,}\StringTok{"25"}\NormalTok{,}\StringTok{"26"}\NormalTok{,}\StringTok{"27"}\NormalTok{,}\StringTok{"29"}\NormalTok{,}\StringTok{"50"}\NormalTok{,}
     \StringTok{"62"}\NormalTok{,}\StringTok{"63"}\NormalTok{,}\StringTok{"64"}\NormalTok{,}\StringTok{"65"}\NormalTok{,}\StringTok{"66"}\NormalTok{,}\StringTok{"67"}\NormalTok{,}\StringTok{"69"}\NormalTok{)}
\NormalTok{nogabali}\OtherTok{=}\NormalTok{mvr }\SpecialCharTok{\%\textgreater{}\%} 
 \FunctionTok{mutate}\NormalTok{(}\AttributeTok{BezOzoluKraja=}\FunctionTok{ifelse}\NormalTok{(s10 }\SpecialCharTok{\%in\%}\NormalTok{ neozoli, v10, }\DecValTok{0}\NormalTok{)}\SpecialCharTok{+}\FunctionTok{ifelse}\NormalTok{(s11 }\SpecialCharTok{\%in\%}\NormalTok{ neozoli,v11,}\DecValTok{0}\NormalTok{)}\SpecialCharTok{+}
      \FunctionTok{ifelse}\NormalTok{(s12 }\SpecialCharTok{\%in\%}\NormalTok{ neozoli, v12,}\DecValTok{0}\NormalTok{)}\SpecialCharTok{+}\FunctionTok{ifelse}\NormalTok{(s13 }\SpecialCharTok{\%in\%}\NormalTok{ neozoli,v13,}\DecValTok{0}\NormalTok{)}\SpecialCharTok{+}
      \FunctionTok{ifelse}\NormalTok{(s14 }\SpecialCharTok{\%in\%}\NormalTok{ neozoli, v14,}\DecValTok{0}\NormalTok{)) }\SpecialCharTok{\%\textgreater{}\%} 
 \FunctionTok{mutate}\NormalTok{(}\AttributeTok{BezOzoluKraja2=}\NormalTok{BezOzoluKraja}\SpecialCharTok{/}\DecValTok{10000}\SpecialCharTok{*}\DecValTok{10}\SpecialCharTok{*}\DecValTok{10}\NormalTok{) }\SpecialCharTok{\%\textgreater{}\%} 
 \FunctionTok{mutate}\NormalTok{(}\AttributeTok{BezOzoluKraja3=}\FunctionTok{ifelse}\NormalTok{(BezOzoluKraja2}\SpecialCharTok{\textgreater{}}\DecValTok{3}\NormalTok{,}\DecValTok{3}\NormalTok{,BezOzoluKraja2)) }\SpecialCharTok{\%\textgreater{}\%} 
 \FunctionTok{filter}\NormalTok{(}\SpecialCharTok{!}\FunctionTok{is.na}\NormalTok{(BezOzoluKraja2))}

\FunctionTok{par}\NormalTok{(}\AttributeTok{mfrow=}\FunctionTok{c}\NormalTok{(}\DecValTok{1}\NormalTok{,}\DecValTok{2}\NormalTok{))}
\FunctionTok{options}\NormalTok{(}\AttributeTok{scipen=}\DecValTok{999}\NormalTok{)}
\FunctionTok{hist}\NormalTok{(nogabali}\SpecialCharTok{$}\NormalTok{BezOzoluKraja2,}\AttributeTok{main=}\StringTok{"Original"}\NormalTok{,}\AttributeTok{xlab=}\StringTok{"Temperate deciduous volume}\SpecialCharTok{\textbackslash{}n}\StringTok{ without oak"}\NormalTok{)}
\FunctionTok{hist}\NormalTok{(nogabali}\SpecialCharTok{$}\NormalTok{BezOzoluKraja3,}\AttributeTok{main=}\StringTok{"Limited"}\NormalTok{,}\AttributeTok{xlab=}\StringTok{"Temperate deciduous volume}\SpecialCharTok{\textbackslash{}n}\StringTok{without oak"}\NormalTok{)}
\FunctionTok{par}\NormalTok{(}\AttributeTok{mfrow=}\FunctionTok{c}\NormalTok{(}\DecValTok{1}\NormalTok{,}\DecValTok{1}\NormalTok{))}
\FunctionTok{options}\NormalTok{(}\AttributeTok{scipen=}\DecValTok{0}\NormalTok{)}

\NormalTok{p2i\_rez}\OtherTok{=}\FunctionTok{polygon2input}\NormalTok{(}\AttributeTok{vector\_data=}\NormalTok{nogabali,}
           \AttributeTok{template\_path =} \StringTok{"./Templates/TemplateRasters/LV10m\_10km.tif"}\NormalTok{,}
           \AttributeTok{out\_path =} \StringTok{"./RasterGrids\_10m/2024/"}\NormalTok{,}
           \AttributeTok{file\_name =} \StringTok{"ForestsQuant\_VolumeTemperateWithoutOak.tif"}\NormalTok{,}
           \AttributeTok{value\_field =} \StringTok{"BezOzoluKraja3"}\NormalTok{,}
           \AttributeTok{fun=}\StringTok{"max"}\NormalTok{,}
           \AttributeTok{prepare=}\ConstantTok{FALSE}\NormalTok{,}
           \AttributeTok{restrict\_to =}\NormalTok{ clearcut\_mask,}
           \AttributeTok{restrict\_values =} \DecValTok{0}\NormalTok{,}
           \AttributeTok{plot\_result=}\ConstantTok{TRUE}\NormalTok{,}
           \AttributeTok{overwrite=}\ConstantTok{TRUE}\NormalTok{)}
\NormalTok{p2i\_rez}
\NormalTok{i2e\_rez}\OtherTok{=}\FunctionTok{input2egv}\NormalTok{(}\AttributeTok{input=}\StringTok{"./RasterGrids\_10m/2024/ForestsQuant\_VolumeTemperateWithoutOak.tif"}\NormalTok{,}
         \AttributeTok{egv\_template =} \StringTok{"./Templates/TemplateRasters/LV100m\_10km.tif"}\NormalTok{,}
         \AttributeTok{summary\_function =} \StringTok{"sum"}\NormalTok{,}
         \AttributeTok{missing\_job =} \StringTok{"CoverOutput"}\NormalTok{,}
         \AttributeTok{output\_bg =} \StringTok{"./Templates/TemplateRasters/nulls\_LV100m\_10km.tif"}\NormalTok{,}
         \AttributeTok{outlocation =} \StringTok{"./RasterGrids\_100m/2024/RAW/"}\NormalTok{,}
         \AttributeTok{outfilename =} \StringTok{"ForestsQuant\_VolumeTemperateWithoutOak{-}sum\_cell.tif"}\NormalTok{,}
         \AttributeTok{layername =} \StringTok{"egv\_305"}\NormalTok{,}
         \AttributeTok{plot\_final=}\ConstantTok{TRUE}\NormalTok{)}
\NormalTok{i2e\_rez}
\FunctionTok{rm}\NormalTok{(p2i\_rez)}
\FunctionTok{rm}\NormalTok{(nogabali)}
\FunctionTok{rm}\NormalTok{(neozoli)}
\FunctionTok{rm}\NormalTok{(i2e\_rez)}
\FunctionTok{unlink}\NormalTok{(}\StringTok{"./RasterGrids\_10m/2024/ForestsQuant\_VolumeTemperateWithoutOak.tif"}\NormalTok{)}


\CommentTok{\# standardisation {-}{-}{-}{-}}
\ControlFlowTok{if}\NormalTok{(}\SpecialCharTok{!}\FunctionTok{require}\NormalTok{(terra)) \{}\FunctionTok{install.packages}\NormalTok{(}\StringTok{"terra"}\NormalTok{); }\FunctionTok{require}\NormalTok{(terra)\}}
\ControlFlowTok{if}\NormalTok{(}\SpecialCharTok{!}\FunctionTok{require}\NormalTok{(tidyverse)) \{}\FunctionTok{install.packages}\NormalTok{(}\StringTok{"tidyverse"}\NormalTok{); }\FunctionTok{require}\NormalTok{(tidyverse)\}}

\NormalTok{nosaukums}\OtherTok{=}\StringTok{"ForestsQuant\_VolumeTemperateWithoutOak{-}sum\_cell.tif"}
\NormalTok{ielasisanas\_cels}\OtherTok{=}\FunctionTok{paste0}\NormalTok{(}\StringTok{"./RasterGrids\_100m/2024/RAW/"}\NormalTok{,nosaukums)}
\NormalTok{saglabasanas\_cels}\OtherTok{=}\FunctionTok{paste0}\NormalTok{(}\StringTok{"./RasterGrids\_100m/2024/Scaled/"}\NormalTok{,nosaukums)}
\NormalTok{slanis}\OtherTok{=}\FunctionTok{rast}\NormalTok{(ielasisanas\_cels)}
\NormalTok{videjais}\OtherTok{=}\FunctionTok{global}\NormalTok{(slanis,}\AttributeTok{fun=}\StringTok{"mean"}\NormalTok{,}\AttributeTok{na.rm=}\ConstantTok{TRUE}\NormalTok{)}
\NormalTok{centrets}\OtherTok{=}\NormalTok{slanis}\SpecialCharTok{{-}}\NormalTok{videjais[,}\DecValTok{1}\NormalTok{]}
\NormalTok{standartnovirze}\OtherTok{=}\NormalTok{terra}\SpecialCharTok{::}\FunctionTok{global}\NormalTok{(centrets,}\AttributeTok{fun=}\StringTok{"rms"}\NormalTok{,}\AttributeTok{na.rm=}\ConstantTok{TRUE}\NormalTok{)}
\NormalTok{merogots}\OtherTok{=}\NormalTok{centrets}\SpecialCharTok{/}\NormalTok{standartnovirze[,}\DecValTok{1}\NormalTok{]}
\FunctionTok{writeRaster}\NormalTok{(merogots,}
      \AttributeTok{filename=}\NormalTok{saglabasanas\_cels,}
      \AttributeTok{overwrite=}\ConstantTok{TRUE}\NormalTok{)}
\end{Highlighting}
\end{Shaded}

\section{ForestsQuant\_VolumeTemperateWithoutOakMaple-sum\_cell}\label{ch06.306}

\textbf{filename:} \texttt{ForestsQuant\_VolumeTemperateWithoutOakMaple-sum\_cell.tif}

\textbf{layername:} \texttt{egv\_306}

\textbf{English name:} Timber volume of Temperate Deciduous trees (without oaks,
maples) within the analysis cell (1 ha)

\textbf{Latvian name:} Platlapju (bez ozoliem, kļavām) krāja analīzes šūnā (1 ha)

\textbf{Procedure:} Most EGVs describing forests are spatially restricted to areas outside
of clearcuts and dead stands. This mask is created using a combination of
the \hyperref[Ch04.01]{State Forest Service's
State Forest Registry} land category 12 and 14, and \hyperref[Ch04.09]{The
Global Forest Watch} pixels classified as lost tree canopy cover since
2020 (raster layer matching input, presence = 1, absence = 0).

This EGV is prepared based on the information of timber volume of teperate
deciduous trees except oaks and maples
(species codes: 11, 12, 16, 17, 18, 25, 26, 27, 29, 50, 62, 64, 65, 66, 67, 69;
see tree species codes in \hyperref[Ch01]{Terminology and acronyms}) in the inventoried
forest stands - \hyperref[Ch04.01]{State Forest Service's State Forest Registry}.
This attribute has some extreme
values. We chose to limit them to the nearest integer showing only minimal
accumulation in histogram.

\includegraphics[width=0.8\linewidth]{./Figures/Histogramms/hist_egv306}

Resulting values at polygon geometries are rasterised with the workflow
\texttt{egvtools::polygon2input()}, restricting to pixels outside the clearcut mask. No
background values are assigned during rasterisation. The resulting layer is
then aggregated to EGV resolution using the workflow \texttt{egvtools::input2egv()} by calculating
sum of pixel values. After the aggregation, cells with no forest information
are filled with value 0. Finally, the layer is standardised by subtracting
the arithmetic mean and dividing by the root mean squared error.

\begin{Shaded}
\begin{Highlighting}[]
\CommentTok{\# libs {-}{-}{-}{-}}
\ControlFlowTok{if}\NormalTok{(}\SpecialCharTok{!}\FunctionTok{require}\NormalTok{(egvtools)) \{remotes}\SpecialCharTok{::}\FunctionTok{install\_github}\NormalTok{(}\StringTok{"aavotins/egvtools"}\NormalTok{); }\FunctionTok{require}\NormalTok{(egvtools)\}}
\ControlFlowTok{if}\NormalTok{(}\SpecialCharTok{!}\FunctionTok{require}\NormalTok{(terra)) \{}\FunctionTok{install.packages}\NormalTok{(}\StringTok{"terra"}\NormalTok{); }\FunctionTok{require}\NormalTok{(terra)\}}
\ControlFlowTok{if}\NormalTok{(}\SpecialCharTok{!}\FunctionTok{require}\NormalTok{(sf)) \{}\FunctionTok{install.packages}\NormalTok{(}\StringTok{"sf"}\NormalTok{); }\FunctionTok{require}\NormalTok{(sf)\}}
\ControlFlowTok{if}\NormalTok{(}\SpecialCharTok{!}\FunctionTok{require}\NormalTok{(tidyverse)) \{}\FunctionTok{install.packages}\NormalTok{(}\StringTok{"tidyverse"}\NormalTok{); }\FunctionTok{require}\NormalTok{(tidyverse)\}}
\ControlFlowTok{if}\NormalTok{(}\SpecialCharTok{!}\FunctionTok{require}\NormalTok{(sfarrow)) \{}\FunctionTok{install.packages}\NormalTok{(}\StringTok{"sfarrow"}\NormalTok{); }\FunctionTok{require}\NormalTok{(sfarrow)\}}
\ControlFlowTok{if}\NormalTok{(}\SpecialCharTok{!}\FunctionTok{require}\NormalTok{(readxl)) \{}\FunctionTok{install.packages}\NormalTok{(}\StringTok{"readxl"}\NormalTok{); }\FunctionTok{require}\NormalTok{(readxl)\}}
\ControlFlowTok{if}\NormalTok{(}\SpecialCharTok{!}\FunctionTok{require}\NormalTok{(raster)) \{}\FunctionTok{install.packages}\NormalTok{(}\StringTok{"raster"}\NormalTok{); }\FunctionTok{require}\NormalTok{(raster)\}}
\ControlFlowTok{if}\NormalTok{(}\SpecialCharTok{!}\FunctionTok{require}\NormalTok{(fasterize)) \{}\FunctionTok{install.packages}\NormalTok{(}\StringTok{"fasterize"}\NormalTok{); }\FunctionTok{require}\NormalTok{(fasterize)\}}

\CommentTok{\# templates {-}{-}{-}{-}}
\NormalTok{template100}\OtherTok{=}\FunctionTok{rast}\NormalTok{(}\StringTok{"./Templates/TemplateRasters/LV100m\_10km.tif"}\NormalTok{)}
\NormalTok{template10}\OtherTok{=}\FunctionTok{rast}\NormalTok{(}\StringTok{"./Templates/TemplateRasters/LV10m\_10km.tif"}\NormalTok{)}
\NormalTok{rastrs10}\OtherTok{=}\FunctionTok{raster}\NormalTok{(template10)}

\NormalTok{nulls10}\OtherTok{=}\FunctionTok{rast}\NormalTok{(}\StringTok{"./Templates/TemplateRasters/nulls\_LV10m\_10km.tif"}\NormalTok{)}
\NormalTok{nulls100}\OtherTok{=}\FunctionTok{rast}\NormalTok{(}\StringTok{"./Templates/TemplateRasters/nulls\_LV100m\_10km.tif"}\NormalTok{)}


\CommentTok{\# simple landscape {-}{-}{-}{-}}
\NormalTok{simple\_landscape}\OtherTok{=}\FunctionTok{rast}\NormalTok{(}\StringTok{"RasterGrids\_10m/2024/Ainava\_vienk\_mask.tif"}\NormalTok{)}

\CommentTok{\# mvr {-}{-}{-}{-}}
\NormalTok{mvr}\OtherTok{=}\FunctionTok{st\_read\_parquet}\NormalTok{(}\StringTok{"./Geodata/2024/MVR/nogabali\_2024janv.parquet"}\NormalTok{)}
\NormalTok{mvr}\SpecialCharTok{$}\NormalTok{yes}\OtherTok{=}\DecValTok{1}

\CommentTok{\# clear cut mask {-}{-}{-}{-}}
\NormalTok{izcirtumi}\OtherTok{=}\NormalTok{mvr }\SpecialCharTok{\%\textgreater{}\%} 
 \FunctionTok{filter}\NormalTok{(zkat }\SpecialCharTok{\%in\%} \FunctionTok{c}\NormalTok{(}\StringTok{"12"}\NormalTok{,}\StringTok{"14"}\NormalTok{)) }\SpecialCharTok{\%\textgreater{}\%} 
\NormalTok{ dplyr}\SpecialCharTok{::}\FunctionTok{select}\NormalTok{(yes)}
\NormalTok{r\_izcirtumi\_mvr}\OtherTok{=}\FunctionTok{fasterize}\NormalTok{(izcirtumi,rastrs10,}\AttributeTok{field=}\StringTok{"yes"}\NormalTok{)}
\NormalTok{t\_izcirtumi\_mvr}\OtherTok{=}\FunctionTok{rast}\NormalTok{(r\_izcirtumi\_mvr)}
\FunctionTok{plot}\NormalTok{(t\_izcirtumi\_mvr)}

\NormalTok{tcl}\OtherTok{=}\FunctionTok{rast}\NormalTok{(}\StringTok{"./Geodata/2024/Trees/GFW/TreeCoverLoss\_v1\_12.tif"}\NormalTok{)}
\NormalTok{tcl2}\OtherTok{=}\FunctionTok{ifel}\NormalTok{(tcl}\SpecialCharTok{\textless{}}\DecValTok{20}\NormalTok{,}\DecValTok{0}\NormalTok{,}\DecValTok{1}\NormalTok{)}
\NormalTok{tclX}\OtherTok{=}\FunctionTok{cover}\NormalTok{(tcl2,nulls10)}
\FunctionTok{plot}\NormalTok{(tclX)}

\NormalTok{clearcut\_mask}\OtherTok{=}\FunctionTok{cover}\NormalTok{(t\_izcirtumi\_mvr,tclX,}
          \AttributeTok{filename=}\StringTok{"./RasterGrids\_10m/2024/Mask\_clearcuts.tif"}\NormalTok{,}
          \AttributeTok{overwrite=}\ConstantTok{TRUE}\NormalTok{)}
\FunctionTok{plot}\NormalTok{(clearcut\_mask)}

\FunctionTok{rm}\NormalTok{(izcirtumi)}
\FunctionTok{rm}\NormalTok{(r\_izcirtumi\_mvr)}
\FunctionTok{rm}\NormalTok{(t\_izcirtumi\_mvr)}
\FunctionTok{rm}\NormalTok{(tcl)}
\FunctionTok{rm}\NormalTok{(tcl2)}
\FunctionTok{rm}\NormalTok{(tclX)}

\CommentTok{\# ForestsQuant\_VolumeTemperateWithoutOakMaple{-}sum\_cell.tif  egv\_306 {-}{-}{-}{-}}
\NormalTok{neozolklavas}\OtherTok{=}\FunctionTok{c}\NormalTok{(}\StringTok{"11"}\NormalTok{,}\StringTok{"12"}\NormalTok{,}\StringTok{"16"}\NormalTok{,}\StringTok{"17"}\NormalTok{,}\StringTok{"18"}\NormalTok{,}\StringTok{"25"}\NormalTok{,}\StringTok{"26"}\NormalTok{,}\StringTok{"27"}\NormalTok{,}\StringTok{"29"}\NormalTok{,}\StringTok{"50"}\NormalTok{,}
        \StringTok{"62"}\NormalTok{,}\StringTok{"64"}\NormalTok{,}\StringTok{"65"}\NormalTok{,}\StringTok{"66"}\NormalTok{,}\StringTok{"67"}\NormalTok{,}\StringTok{"69"}\NormalTok{)}
\NormalTok{nogabali}\OtherTok{=}\NormalTok{mvr }\SpecialCharTok{\%\textgreater{}\%} 
 \FunctionTok{mutate}\NormalTok{(}\AttributeTok{BezOzolKlavuKraja=}\FunctionTok{ifelse}\NormalTok{(s10 }\SpecialCharTok{\%in\%}\NormalTok{ neozolklavas, v10, }\DecValTok{0}\NormalTok{)}\SpecialCharTok{+}\FunctionTok{ifelse}\NormalTok{(s11 }\SpecialCharTok{\%in\%}\NormalTok{ neozolklavas,v11,}\DecValTok{0}\NormalTok{)}\SpecialCharTok{+}
      \FunctionTok{ifelse}\NormalTok{(s12 }\SpecialCharTok{\%in\%}\NormalTok{ neozolklavas, v12,}\DecValTok{0}\NormalTok{)}\SpecialCharTok{+}\FunctionTok{ifelse}\NormalTok{(s13 }\SpecialCharTok{\%in\%}\NormalTok{ neozolklavas,v13,}\DecValTok{0}\NormalTok{)}\SpecialCharTok{+}
      \FunctionTok{ifelse}\NormalTok{(s14 }\SpecialCharTok{\%in\%}\NormalTok{ neozolklavas, v14,}\DecValTok{0}\NormalTok{)) }\SpecialCharTok{\%\textgreater{}\%} 
 \FunctionTok{mutate}\NormalTok{(}\AttributeTok{BezOzolKlavuKraja2=}\NormalTok{BezOzolKlavuKraja}\SpecialCharTok{/}\DecValTok{10000}\SpecialCharTok{*}\DecValTok{10}\SpecialCharTok{*}\DecValTok{10}\NormalTok{) }\SpecialCharTok{\%\textgreater{}\%} 
 \FunctionTok{mutate}\NormalTok{(}\AttributeTok{BezOzolKlavuKraja3=}\FunctionTok{ifelse}\NormalTok{(BezOzolKlavuKraja2}\SpecialCharTok{\textgreater{}}\DecValTok{3}\NormalTok{,}\DecValTok{3}\NormalTok{,BezOzolKlavuKraja2)) }\SpecialCharTok{\%\textgreater{}\%} 
 \FunctionTok{filter}\NormalTok{(}\SpecialCharTok{!}\FunctionTok{is.na}\NormalTok{(BezOzolKlavuKraja2))}

\FunctionTok{par}\NormalTok{(}\AttributeTok{mfrow=}\FunctionTok{c}\NormalTok{(}\DecValTok{1}\NormalTok{,}\DecValTok{2}\NormalTok{))}
\FunctionTok{options}\NormalTok{(}\AttributeTok{scipen=}\DecValTok{999}\NormalTok{)}
\FunctionTok{hist}\NormalTok{(nogabali}\SpecialCharTok{$}\NormalTok{BezOzolKlavuKraja2,}\AttributeTok{main=}\StringTok{"Original"}\NormalTok{,}\AttributeTok{xlab=}\StringTok{"Temperate deciduous volume}\SpecialCharTok{\textbackslash{}n}\StringTok{ without oak and maple"}\NormalTok{)}
\FunctionTok{hist}\NormalTok{(nogabali}\SpecialCharTok{$}\NormalTok{BezOzolKlavuKraja3,}\AttributeTok{main=}\StringTok{"Limited"}\NormalTok{,}\AttributeTok{xlab=}\StringTok{"Temperate deciduous volume}\SpecialCharTok{\textbackslash{}n}\StringTok{without oak and maple"}\NormalTok{)}
\FunctionTok{par}\NormalTok{(}\AttributeTok{mfrow=}\FunctionTok{c}\NormalTok{(}\DecValTok{1}\NormalTok{,}\DecValTok{1}\NormalTok{))}
\FunctionTok{options}\NormalTok{(}\AttributeTok{scipen=}\DecValTok{0}\NormalTok{)}

\NormalTok{p2i\_rez}\OtherTok{=}\FunctionTok{polygon2input}\NormalTok{(}\AttributeTok{vector\_data=}\NormalTok{nogabali,}
           \AttributeTok{template\_path =} \StringTok{"./Templates/TemplateRasters/LV10m\_10km.tif"}\NormalTok{,}
           \AttributeTok{out\_path =} \StringTok{"./RasterGrids\_10m/2024/"}\NormalTok{,}
           \AttributeTok{file\_name =} \StringTok{"ForestsQuant\_VolumeTemperateWithoutOakMaple.tif"}\NormalTok{,}
           \AttributeTok{value\_field =} \StringTok{"BezOzolKlavuKraja3"}\NormalTok{,}
           \AttributeTok{fun=}\StringTok{"max"}\NormalTok{,}
           \AttributeTok{prepare=}\ConstantTok{FALSE}\NormalTok{,}
           \AttributeTok{restrict\_to =}\NormalTok{ clearcut\_mask,}
           \AttributeTok{restrict\_values =} \DecValTok{0}\NormalTok{,}
           \AttributeTok{plot\_result=}\ConstantTok{TRUE}\NormalTok{,}
           \AttributeTok{overwrite=}\ConstantTok{TRUE}\NormalTok{)}
\NormalTok{p2i\_rez}
\NormalTok{i2e\_rez}\OtherTok{=}\FunctionTok{input2egv}\NormalTok{(}\AttributeTok{input=}\StringTok{"./RasterGrids\_10m/2024/ForestsQuant\_VolumeTemperateWithoutOakMaple.tif"}\NormalTok{,}
         \AttributeTok{egv\_template =} \StringTok{"./Templates/TemplateRasters/LV100m\_10km.tif"}\NormalTok{,}
         \AttributeTok{summary\_function =} \StringTok{"sum"}\NormalTok{,}
         \AttributeTok{missing\_job =} \StringTok{"CoverOutput"}\NormalTok{,}
         \AttributeTok{output\_bg =} \StringTok{"./Templates/TemplateRasters/nulls\_LV100m\_10km.tif"}\NormalTok{,}
         \AttributeTok{outlocation =} \StringTok{"./RasterGrids\_100m/2024/RAW/"}\NormalTok{,}
         \AttributeTok{outfilename =} \StringTok{"ForestsQuant\_VolumeTemperateWithoutOakMaple{-}sum\_cell.tif"}\NormalTok{,}
         \AttributeTok{layername =} \StringTok{"egv\_306"}\NormalTok{,}
         \AttributeTok{plot\_final=}\ConstantTok{TRUE}\NormalTok{)}
\NormalTok{i2e\_rez}
\FunctionTok{rm}\NormalTok{(p2i\_rez)}
\FunctionTok{rm}\NormalTok{(nogabali)}
\FunctionTok{rm}\NormalTok{(neozolklavas)}
\FunctionTok{rm}\NormalTok{(i2e\_rez)}
\FunctionTok{unlink}\NormalTok{(}\StringTok{"./RasterGrids\_10m/2024/ForestsQuant\_VolumeTemperateWithoutOakMaple.tif"}\NormalTok{)}


\CommentTok{\# standardisation {-}{-}{-}{-}}
\ControlFlowTok{if}\NormalTok{(}\SpecialCharTok{!}\FunctionTok{require}\NormalTok{(terra)) \{}\FunctionTok{install.packages}\NormalTok{(}\StringTok{"terra"}\NormalTok{); }\FunctionTok{require}\NormalTok{(terra)\}}
\ControlFlowTok{if}\NormalTok{(}\SpecialCharTok{!}\FunctionTok{require}\NormalTok{(tidyverse)) \{}\FunctionTok{install.packages}\NormalTok{(}\StringTok{"tidyverse"}\NormalTok{); }\FunctionTok{require}\NormalTok{(tidyverse)\}}

\NormalTok{nosaukums}\OtherTok{=}\StringTok{"ForestsQuant\_VolumeTemperateWithoutOakMaple{-}sum\_cell.tif"}
\NormalTok{ielasisanas\_cels}\OtherTok{=}\FunctionTok{paste0}\NormalTok{(}\StringTok{"./RasterGrids\_100m/2024/RAW/"}\NormalTok{,nosaukums)}
\NormalTok{saglabasanas\_cels}\OtherTok{=}\FunctionTok{paste0}\NormalTok{(}\StringTok{"./RasterGrids\_100m/2024/Scaled/"}\NormalTok{,nosaukums)}
\NormalTok{slanis}\OtherTok{=}\FunctionTok{rast}\NormalTok{(ielasisanas\_cels)}
\NormalTok{videjais}\OtherTok{=}\FunctionTok{global}\NormalTok{(slanis,}\AttributeTok{fun=}\StringTok{"mean"}\NormalTok{,}\AttributeTok{na.rm=}\ConstantTok{TRUE}\NormalTok{)}
\NormalTok{centrets}\OtherTok{=}\NormalTok{slanis}\SpecialCharTok{{-}}\NormalTok{videjais[,}\DecValTok{1}\NormalTok{]}
\NormalTok{standartnovirze}\OtherTok{=}\NormalTok{terra}\SpecialCharTok{::}\FunctionTok{global}\NormalTok{(centrets,}\AttributeTok{fun=}\StringTok{"rms"}\NormalTok{,}\AttributeTok{na.rm=}\ConstantTok{TRUE}\NormalTok{)}
\NormalTok{merogots}\OtherTok{=}\NormalTok{centrets}\SpecialCharTok{/}\NormalTok{standartnovirze[,}\DecValTok{1}\NormalTok{]}
\FunctionTok{writeRaster}\NormalTok{(merogots,}
      \AttributeTok{filename=}\NormalTok{saglabasanas\_cels,}
      \AttributeTok{overwrite=}\ConstantTok{TRUE}\NormalTok{)}
\end{Highlighting}
\end{Shaded}

\section{ForestsQuant\_VolumeTotal-sum\_cell}\label{ch06.307}

\textbf{filename:} \texttt{ForestsQuant\_VolumeTotal-sum\_cell.tif}

\textbf{layername:} \texttt{egv\_307}

\textbf{English name:} Timber volume within the analysis cell (1 ha)

\textbf{Latvian name:} Kopējā krāja analīzes šūnā (1 ha)

\textbf{Procedure:} Most EGVs describing forests are spatially restricted to areas outside
of clearcuts and dead stands. This mask is created using a combination of
the \hyperref[Ch04.01]{State Forest Service's
State Forest Registry} land category 12 and 14, and \hyperref[Ch04.09]{The
Global Forest Watch} pixels classified as lost tree canopy cover since
2020 (raster layer matching input, presence = 1, absence = 0).

This EGV is prepared based on the information of timber volume in the
inventoried forest stands - \hyperref[Ch04.01]{State Forest Service's State Forest
Registry}. This attribute has some extreme
values. We chose to limit them to the nearest integer showing only minimal
accumulation in histogram.

\includegraphics[width=0.8\linewidth]{./Figures/Histogramms/hist_egv307}

Resulting values at polygon geometries are rasterised with the workflow
\texttt{egvtools::polygon2input()}, restricting to pixels outside the clearcut mask. No
background values are assigned during rasterisation. The resulting layer is
then aggregated to EGV resolution using the workflow \texttt{egvtools::input2egv()} by calculating
sum of pixel values. After the aggregation, cells with no forest information
are filled with value 0. Finally, the layer is standardised by subtracting
the arithmetic mean and dividing by the root mean squared error.

\begin{Shaded}
\begin{Highlighting}[]
\CommentTok{\# libs {-}{-}{-}{-}}
\ControlFlowTok{if}\NormalTok{(}\SpecialCharTok{!}\FunctionTok{require}\NormalTok{(egvtools)) \{remotes}\SpecialCharTok{::}\FunctionTok{install\_github}\NormalTok{(}\StringTok{"aavotins/egvtools"}\NormalTok{); }\FunctionTok{require}\NormalTok{(egvtools)\}}
\ControlFlowTok{if}\NormalTok{(}\SpecialCharTok{!}\FunctionTok{require}\NormalTok{(terra)) \{}\FunctionTok{install.packages}\NormalTok{(}\StringTok{"terra"}\NormalTok{); }\FunctionTok{require}\NormalTok{(terra)\}}
\ControlFlowTok{if}\NormalTok{(}\SpecialCharTok{!}\FunctionTok{require}\NormalTok{(sf)) \{}\FunctionTok{install.packages}\NormalTok{(}\StringTok{"sf"}\NormalTok{); }\FunctionTok{require}\NormalTok{(sf)\}}
\ControlFlowTok{if}\NormalTok{(}\SpecialCharTok{!}\FunctionTok{require}\NormalTok{(tidyverse)) \{}\FunctionTok{install.packages}\NormalTok{(}\StringTok{"tidyverse"}\NormalTok{); }\FunctionTok{require}\NormalTok{(tidyverse)\}}
\ControlFlowTok{if}\NormalTok{(}\SpecialCharTok{!}\FunctionTok{require}\NormalTok{(sfarrow)) \{}\FunctionTok{install.packages}\NormalTok{(}\StringTok{"sfarrow"}\NormalTok{); }\FunctionTok{require}\NormalTok{(sfarrow)\}}
\ControlFlowTok{if}\NormalTok{(}\SpecialCharTok{!}\FunctionTok{require}\NormalTok{(readxl)) \{}\FunctionTok{install.packages}\NormalTok{(}\StringTok{"readxl"}\NormalTok{); }\FunctionTok{require}\NormalTok{(readxl)\}}
\ControlFlowTok{if}\NormalTok{(}\SpecialCharTok{!}\FunctionTok{require}\NormalTok{(raster)) \{}\FunctionTok{install.packages}\NormalTok{(}\StringTok{"raster"}\NormalTok{); }\FunctionTok{require}\NormalTok{(raster)\}}
\ControlFlowTok{if}\NormalTok{(}\SpecialCharTok{!}\FunctionTok{require}\NormalTok{(fasterize)) \{}\FunctionTok{install.packages}\NormalTok{(}\StringTok{"fasterize"}\NormalTok{); }\FunctionTok{require}\NormalTok{(fasterize)\}}

\CommentTok{\# templates {-}{-}{-}{-}}
\NormalTok{template100}\OtherTok{=}\FunctionTok{rast}\NormalTok{(}\StringTok{"./Templates/TemplateRasters/LV100m\_10km.tif"}\NormalTok{)}
\NormalTok{template10}\OtherTok{=}\FunctionTok{rast}\NormalTok{(}\StringTok{"./Templates/TemplateRasters/LV10m\_10km.tif"}\NormalTok{)}
\NormalTok{rastrs10}\OtherTok{=}\FunctionTok{raster}\NormalTok{(template10)}

\NormalTok{nulls10}\OtherTok{=}\FunctionTok{rast}\NormalTok{(}\StringTok{"./Templates/TemplateRasters/nulls\_LV10m\_10km.tif"}\NormalTok{)}
\NormalTok{nulls100}\OtherTok{=}\FunctionTok{rast}\NormalTok{(}\StringTok{"./Templates/TemplateRasters/nulls\_LV100m\_10km.tif"}\NormalTok{)}


\CommentTok{\# simple landscape {-}{-}{-}{-}}
\NormalTok{simple\_landscape}\OtherTok{=}\FunctionTok{rast}\NormalTok{(}\StringTok{"RasterGrids\_10m/2024/Ainava\_vienk\_mask.tif"}\NormalTok{)}

\CommentTok{\# mvr {-}{-}{-}{-}}
\NormalTok{mvr}\OtherTok{=}\FunctionTok{st\_read\_parquet}\NormalTok{(}\StringTok{"./Geodata/2024/MVR/nogabali\_2024janv.parquet"}\NormalTok{)}
\NormalTok{mvr}\SpecialCharTok{$}\NormalTok{yes}\OtherTok{=}\DecValTok{1}

\CommentTok{\# clear cut mask {-}{-}{-}{-}}
\NormalTok{izcirtumi}\OtherTok{=}\NormalTok{mvr }\SpecialCharTok{\%\textgreater{}\%} 
 \FunctionTok{filter}\NormalTok{(zkat }\SpecialCharTok{\%in\%} \FunctionTok{c}\NormalTok{(}\StringTok{"12"}\NormalTok{,}\StringTok{"14"}\NormalTok{)) }\SpecialCharTok{\%\textgreater{}\%} 
\NormalTok{ dplyr}\SpecialCharTok{::}\FunctionTok{select}\NormalTok{(yes)}
\NormalTok{r\_izcirtumi\_mvr}\OtherTok{=}\FunctionTok{fasterize}\NormalTok{(izcirtumi,rastrs10,}\AttributeTok{field=}\StringTok{"yes"}\NormalTok{)}
\NormalTok{t\_izcirtumi\_mvr}\OtherTok{=}\FunctionTok{rast}\NormalTok{(r\_izcirtumi\_mvr)}
\FunctionTok{plot}\NormalTok{(t\_izcirtumi\_mvr)}

\NormalTok{tcl}\OtherTok{=}\FunctionTok{rast}\NormalTok{(}\StringTok{"./Geodata/2024/Trees/GFW/TreeCoverLoss\_v1\_12.tif"}\NormalTok{)}
\NormalTok{tcl2}\OtherTok{=}\FunctionTok{ifel}\NormalTok{(tcl}\SpecialCharTok{\textless{}}\DecValTok{20}\NormalTok{,}\DecValTok{0}\NormalTok{,}\DecValTok{1}\NormalTok{)}
\NormalTok{tclX}\OtherTok{=}\FunctionTok{cover}\NormalTok{(tcl2,nulls10)}
\FunctionTok{plot}\NormalTok{(tclX)}

\NormalTok{clearcut\_mask}\OtherTok{=}\FunctionTok{cover}\NormalTok{(t\_izcirtumi\_mvr,tclX,}
          \AttributeTok{filename=}\StringTok{"./RasterGrids\_10m/2024/Mask\_clearcuts.tif"}\NormalTok{,}
          \AttributeTok{overwrite=}\ConstantTok{TRUE}\NormalTok{)}
\FunctionTok{plot}\NormalTok{(clearcut\_mask)}

\FunctionTok{rm}\NormalTok{(izcirtumi)}
\FunctionTok{rm}\NormalTok{(r\_izcirtumi\_mvr)}
\FunctionTok{rm}\NormalTok{(t\_izcirtumi\_mvr)}
\FunctionTok{rm}\NormalTok{(tcl)}
\FunctionTok{rm}\NormalTok{(tcl2)}
\FunctionTok{rm}\NormalTok{(tclX)}

\CommentTok{\# ForestsQuant\_VolumeTotal{-}sum\_cell.tif egv\_307 {-}{-}{-}{-}}

\NormalTok{nogabali}\OtherTok{=}\NormalTok{mvr }\SpecialCharTok{\%\textgreater{}\%} 
 \FunctionTok{mutate}\NormalTok{(}\AttributeTok{KopejaKraja=}\NormalTok{v10}\SpecialCharTok{+}\NormalTok{v11}\SpecialCharTok{+}\NormalTok{v12}\SpecialCharTok{+}\NormalTok{v13}\SpecialCharTok{+}\NormalTok{v14) }\SpecialCharTok{\%\textgreater{}\%} 
 \FunctionTok{mutate}\NormalTok{(}\AttributeTok{KopejaKraja2=}\NormalTok{KopejaKraja}\SpecialCharTok{/}\DecValTok{10000}\SpecialCharTok{*}\DecValTok{10}\SpecialCharTok{*}\DecValTok{10}\NormalTok{) }\SpecialCharTok{\%\textgreater{}\%} 
 \FunctionTok{mutate}\NormalTok{(}\AttributeTok{KopejaKraja3=}\FunctionTok{ifelse}\NormalTok{(KopejaKraja2}\SpecialCharTok{\textgreater{}}\DecValTok{8}\NormalTok{,}\DecValTok{8}\NormalTok{,KopejaKraja2)) }\SpecialCharTok{\%\textgreater{}\%} 
 \FunctionTok{filter}\NormalTok{(}\SpecialCharTok{!}\FunctionTok{is.na}\NormalTok{(KopejaKraja2))}

\FunctionTok{par}\NormalTok{(}\AttributeTok{mfrow=}\FunctionTok{c}\NormalTok{(}\DecValTok{1}\NormalTok{,}\DecValTok{2}\NormalTok{))}
\FunctionTok{options}\NormalTok{(}\AttributeTok{scipen=}\DecValTok{999}\NormalTok{)}
\FunctionTok{hist}\NormalTok{(nogabali}\SpecialCharTok{$}\NormalTok{KopejaKraja2,}\AttributeTok{main=}\StringTok{"Original"}\NormalTok{,}\AttributeTok{xlab=}\StringTok{"Timber volume"}\NormalTok{)}
\FunctionTok{hist}\NormalTok{(nogabali}\SpecialCharTok{$}\NormalTok{KopejaKraja3,}\AttributeTok{main=}\StringTok{"Limited"}\NormalTok{,}\AttributeTok{xlab=}\StringTok{"Timber volume"}\NormalTok{)}
\FunctionTok{par}\NormalTok{(}\AttributeTok{mfrow=}\FunctionTok{c}\NormalTok{(}\DecValTok{1}\NormalTok{,}\DecValTok{1}\NormalTok{))}
\FunctionTok{options}\NormalTok{(}\AttributeTok{scipen=}\DecValTok{0}\NormalTok{)}

\NormalTok{p2i\_rez}\OtherTok{=}\FunctionTok{polygon2input}\NormalTok{(}\AttributeTok{vector\_data=}\NormalTok{nogabali,}
           \AttributeTok{template\_path =} \StringTok{"./Templates/TemplateRasters/LV10m\_10km.tif"}\NormalTok{,}
           \AttributeTok{out\_path =} \StringTok{"./RasterGrids\_10m/2024/"}\NormalTok{,}
           \AttributeTok{file\_name =} \StringTok{"ForestsQuant\_VolumeTotal.tif"}\NormalTok{,}
           \AttributeTok{value\_field =} \StringTok{"KopejaKraja3"}\NormalTok{,}
           \AttributeTok{fun=}\StringTok{"max"}\NormalTok{,}
           \AttributeTok{prepare=}\ConstantTok{FALSE}\NormalTok{,}
           \AttributeTok{restrict\_to =}\NormalTok{ clearcut\_mask,}
           \AttributeTok{restrict\_values =} \DecValTok{0}\NormalTok{,}
           \AttributeTok{plot\_result=}\ConstantTok{TRUE}\NormalTok{,}
           \AttributeTok{overwrite=}\ConstantTok{TRUE}\NormalTok{)}
\NormalTok{p2i\_rez}
\NormalTok{i2e\_rez}\OtherTok{=}\FunctionTok{input2egv}\NormalTok{(}\AttributeTok{input=}\StringTok{"./RasterGrids\_10m/2024/ForestsQuant\_VolumeTotal.tif"}\NormalTok{,}
         \AttributeTok{egv\_template =} \StringTok{"./Templates/TemplateRasters/LV100m\_10km.tif"}\NormalTok{,}
         \AttributeTok{summary\_function =} \StringTok{"sum"}\NormalTok{,}
         \AttributeTok{missing\_job =} \StringTok{"CoverOutput"}\NormalTok{,}
         \AttributeTok{output\_bg =} \StringTok{"./Templates/TemplateRasters/nulls\_LV100m\_10km.tif"}\NormalTok{,}
         \AttributeTok{outlocation =} \StringTok{"./RasterGrids\_100m/2024/RAW/"}\NormalTok{,}
         \AttributeTok{outfilename =} \StringTok{"ForestsQuant\_VolumeTotal{-}sum\_cell.tif"}\NormalTok{,}
         \AttributeTok{layername =} \StringTok{"egv\_307"}\NormalTok{,}
         \AttributeTok{plot\_final=}\ConstantTok{TRUE}\NormalTok{)}
\NormalTok{i2e\_rez}
\FunctionTok{rm}\NormalTok{(p2i\_rez)}
\FunctionTok{rm}\NormalTok{(nogabali)}
\FunctionTok{rm}\NormalTok{(i2e\_rez)}
\FunctionTok{unlink}\NormalTok{(}\StringTok{"./RasterGrids\_10m/2024/ForestsQuant\_VolumeTotal.tif"}\NormalTok{)}

\CommentTok{\# standardisation {-}{-}{-}{-}}
\ControlFlowTok{if}\NormalTok{(}\SpecialCharTok{!}\FunctionTok{require}\NormalTok{(terra)) \{}\FunctionTok{install.packages}\NormalTok{(}\StringTok{"terra"}\NormalTok{); }\FunctionTok{require}\NormalTok{(terra)\}}
\ControlFlowTok{if}\NormalTok{(}\SpecialCharTok{!}\FunctionTok{require}\NormalTok{(tidyverse)) \{}\FunctionTok{install.packages}\NormalTok{(}\StringTok{"tidyverse"}\NormalTok{); }\FunctionTok{require}\NormalTok{(tidyverse)\}}

\NormalTok{nosaukums}\OtherTok{=}\StringTok{"ForestsQuant\_VolumeTotal{-}sum\_cell.tif"}
\NormalTok{ielasisanas\_cels}\OtherTok{=}\FunctionTok{paste0}\NormalTok{(}\StringTok{"./RasterGrids\_100m/2024/RAW/"}\NormalTok{,nosaukums)}
\NormalTok{saglabasanas\_cels}\OtherTok{=}\FunctionTok{paste0}\NormalTok{(}\StringTok{"./RasterGrids\_100m/2024/Scaled/"}\NormalTok{,nosaukums)}
\NormalTok{slanis}\OtherTok{=}\FunctionTok{rast}\NormalTok{(ielasisanas\_cels)}
\NormalTok{videjais}\OtherTok{=}\FunctionTok{global}\NormalTok{(slanis,}\AttributeTok{fun=}\StringTok{"mean"}\NormalTok{,}\AttributeTok{na.rm=}\ConstantTok{TRUE}\NormalTok{)}
\NormalTok{centrets}\OtherTok{=}\NormalTok{slanis}\SpecialCharTok{{-}}\NormalTok{videjais[,}\DecValTok{1}\NormalTok{]}
\NormalTok{standartnovirze}\OtherTok{=}\NormalTok{terra}\SpecialCharTok{::}\FunctionTok{global}\NormalTok{(centrets,}\AttributeTok{fun=}\StringTok{"rms"}\NormalTok{,}\AttributeTok{na.rm=}\ConstantTok{TRUE}\NormalTok{)}
\NormalTok{merogots}\OtherTok{=}\NormalTok{centrets}\SpecialCharTok{/}\NormalTok{standartnovirze[,}\DecValTok{1}\NormalTok{]}
\FunctionTok{writeRaster}\NormalTok{(merogots,}
      \AttributeTok{filename=}\NormalTok{saglabasanas\_cels,}
      \AttributeTok{overwrite=}\ConstantTok{TRUE}\NormalTok{)}
\end{Highlighting}
\end{Shaded}

\section{ForestsSoil\_EutrophicDrained\_cell}\label{ch06.308}

\textbf{filename:} \texttt{ForestsSoil\_EutrophicDrained\_cell.tif}

\textbf{layername:} \texttt{egv\_308}

\textbf{English name:} Fractional cover of Drained Eutrophic Forests within the
analysis cell (1 ha)

\textbf{Latvian name:} Susinātu eitrofu mežu platības īpatsvars analīzes šūnā (1 ha)

\textbf{Procedure:} To prepare this EGV, forest stands with forest type equal to
``19'', ``21'', ``24'' or ``25'' are selected from the \hyperref[Ch04.01]{State Forest Service's State Forest
Registry} and rasterised. Rasterisation is performed using the
workflow \texttt{egvtools::polygon2input()} with background
covering (value 0). The resulting layer
is then aggregated to EGV resolution using the workflow \texttt{egvtools::input2egv()}, which
calculates the arithmetic mean to determine the cover fraction. During
aggregation, inverse distance weighted (power = 2) gap filling on the output is
applied to ensure no missing values at the edges. Finally, the layer is
standardised by subtracting the arithmetic mean and dividing by the root mean squared
error.

\begin{Shaded}
\begin{Highlighting}[]
\CommentTok{\# libs {-}{-}{-}{-}}
\ControlFlowTok{if}\NormalTok{(}\SpecialCharTok{!}\FunctionTok{require}\NormalTok{(egvtools)) \{remotes}\SpecialCharTok{::}\FunctionTok{install\_github}\NormalTok{(}\StringTok{"aavotins/egvtools"}\NormalTok{); }\FunctionTok{require}\NormalTok{(egvtools)\}}
\ControlFlowTok{if}\NormalTok{(}\SpecialCharTok{!}\FunctionTok{require}\NormalTok{(terra)) \{}\FunctionTok{install.packages}\NormalTok{(}\StringTok{"terra"}\NormalTok{); }\FunctionTok{require}\NormalTok{(terra)\}}
\ControlFlowTok{if}\NormalTok{(}\SpecialCharTok{!}\FunctionTok{require}\NormalTok{(sf)) \{}\FunctionTok{install.packages}\NormalTok{(}\StringTok{"sf"}\NormalTok{); }\FunctionTok{require}\NormalTok{(sf)\}}
\ControlFlowTok{if}\NormalTok{(}\SpecialCharTok{!}\FunctionTok{require}\NormalTok{(tidyverse)) \{}\FunctionTok{install.packages}\NormalTok{(}\StringTok{"tidyverse"}\NormalTok{); }\FunctionTok{require}\NormalTok{(tidyverse)\}}
\ControlFlowTok{if}\NormalTok{(}\SpecialCharTok{!}\FunctionTok{require}\NormalTok{(sfarrow)) \{}\FunctionTok{install.packages}\NormalTok{(}\StringTok{"sfarrow"}\NormalTok{); }\FunctionTok{require}\NormalTok{(sfarrow)\}}
\ControlFlowTok{if}\NormalTok{(}\SpecialCharTok{!}\FunctionTok{require}\NormalTok{(readxl)) \{}\FunctionTok{install.packages}\NormalTok{(}\StringTok{"readxl"}\NormalTok{); }\FunctionTok{require}\NormalTok{(readxl)\}}
\ControlFlowTok{if}\NormalTok{(}\SpecialCharTok{!}\FunctionTok{require}\NormalTok{(raster)) \{}\FunctionTok{install.packages}\NormalTok{(}\StringTok{"raster"}\NormalTok{); }\FunctionTok{require}\NormalTok{(raster)\}}
\ControlFlowTok{if}\NormalTok{(}\SpecialCharTok{!}\FunctionTok{require}\NormalTok{(fasterize)) \{}\FunctionTok{install.packages}\NormalTok{(}\StringTok{"fasterize"}\NormalTok{); }\FunctionTok{require}\NormalTok{(fasterize)\}}

\CommentTok{\# templates {-}{-}{-}{-}}
\NormalTok{template100}\OtherTok{=}\FunctionTok{rast}\NormalTok{(}\StringTok{"./Templates/TemplateRasters/LV100m\_10km.tif"}\NormalTok{)}
\NormalTok{template10}\OtherTok{=}\FunctionTok{rast}\NormalTok{(}\StringTok{"./Templates/TemplateRasters/LV10m\_10km.tif"}\NormalTok{)}
\NormalTok{rastrs10}\OtherTok{=}\FunctionTok{raster}\NormalTok{(template10)}

\NormalTok{nulls10}\OtherTok{=}\FunctionTok{rast}\NormalTok{(}\StringTok{"./Templates/TemplateRasters/nulls\_LV10m\_10km.tif"}\NormalTok{)}
\NormalTok{nulls100}\OtherTok{=}\FunctionTok{rast}\NormalTok{(}\StringTok{"./Templates/TemplateRasters/nulls\_LV100m\_10km.tif"}\NormalTok{)}


\CommentTok{\# simple landscape {-}{-}{-}{-}}
\NormalTok{simple\_landscape}\OtherTok{=}\FunctionTok{rast}\NormalTok{(}\StringTok{"RasterGrids\_10m/2024/Ainava\_vienk\_mask.tif"}\NormalTok{)}

\CommentTok{\# mvr {-}{-}{-}{-}}
\NormalTok{mvr}\OtherTok{=}\FunctionTok{st\_read\_parquet}\NormalTok{(}\StringTok{"./Geodata/2024/MVR/nogabali\_2024janv.parquet"}\NormalTok{)}
\NormalTok{mvr}\SpecialCharTok{$}\NormalTok{yes}\OtherTok{=}\DecValTok{1}


\CommentTok{\# ForestsSoil\_EutrophicDrained\_cell.tif egv\_308 {-}{-}{-}{-}}
\NormalTok{EutrophicDrained}\OtherTok{=}\NormalTok{mvr }\SpecialCharTok{\%\textgreater{}\%} 
 \FunctionTok{filter}\NormalTok{(mt }\SpecialCharTok{\%in\%} \FunctionTok{c}\NormalTok{(}\StringTok{"19"}\NormalTok{,}\StringTok{"21"}\NormalTok{,}\StringTok{"24"}\NormalTok{,}\StringTok{"25"}\NormalTok{))}
\NormalTok{p2i\_rez}\OtherTok{=}\NormalTok{egvtools}\SpecialCharTok{::}\FunctionTok{polygon2input}\NormalTok{(}\AttributeTok{vector\_data =}\NormalTok{ EutrophicDrained,}
                \AttributeTok{template\_path =} \StringTok{"./Templates/TemplateRasters/LV10m\_10km.tif"}\NormalTok{,}
                \AttributeTok{out\_path =} \StringTok{"./RasterGrids\_10m/2024/"}\NormalTok{,}
                \AttributeTok{file\_name =} \StringTok{"ForestsSoil\_EutrophicDrained\_input.tif"}\NormalTok{,}
                \AttributeTok{value\_field =} \StringTok{"yes"}\NormalTok{,}
                \AttributeTok{prepare=}\ConstantTok{FALSE}\NormalTok{,}
                \AttributeTok{background\_raster =} \StringTok{"./Templates/TemplateRasters/nulls\_LV10m\_10km.tif"}\NormalTok{,}
                \AttributeTok{plot\_result =} \ConstantTok{TRUE}\NormalTok{)}
\NormalTok{p2i\_rez}
\NormalTok{i2e\_rez}\OtherTok{=}\NormalTok{egvtools}\SpecialCharTok{::}\FunctionTok{input2egv}\NormalTok{(}\AttributeTok{input=}\FunctionTok{paste0}\NormalTok{(}\StringTok{"./RasterGrids\_10m/2024/"}\NormalTok{,}
                     \StringTok{"ForestsSoil\_EutrophicDrained\_input.tif"}\NormalTok{),}
              \AttributeTok{egv\_template=} \StringTok{"./Templates/TemplateRasters/LV100m\_10km.tif"}\NormalTok{,}
              \AttributeTok{summary\_function =} \StringTok{"average"}\NormalTok{,}
              \AttributeTok{missing\_job =} \StringTok{"FillOutput"}\NormalTok{,}
              \AttributeTok{outlocation =} \StringTok{"./RasterGrids\_100m/2024/RAW/"}\NormalTok{,}
              \AttributeTok{outfilename =} \StringTok{"ForestsSoil\_EutrophicDrained\_cell.tif"}\NormalTok{,}
              \AttributeTok{layername =} \StringTok{"egv\_308"}\NormalTok{,}
              \AttributeTok{idw\_weight =} \DecValTok{2}\NormalTok{,}
              \AttributeTok{plot\_gaps =} \ConstantTok{FALSE}\NormalTok{,}\AttributeTok{plot\_final =} \ConstantTok{TRUE}\NormalTok{)}
\NormalTok{i2e\_rez}
\FunctionTok{rm}\NormalTok{(EutrophicDrained)}
\FunctionTok{rm}\NormalTok{(p2i\_rez)}
\FunctionTok{rm}\NormalTok{(i2e\_rez)}
\FunctionTok{unlink}\NormalTok{(}\StringTok{"./RasterGrids\_10m/2024/ForestsSoil\_EutrophicDrained\_input.tif"}\NormalTok{)}

\CommentTok{\# standardisation {-}{-}{-}{-}}
\ControlFlowTok{if}\NormalTok{(}\SpecialCharTok{!}\FunctionTok{require}\NormalTok{(terra)) \{}\FunctionTok{install.packages}\NormalTok{(}\StringTok{"terra"}\NormalTok{); }\FunctionTok{require}\NormalTok{(terra)\}}
\ControlFlowTok{if}\NormalTok{(}\SpecialCharTok{!}\FunctionTok{require}\NormalTok{(tidyverse)) \{}\FunctionTok{install.packages}\NormalTok{(}\StringTok{"tidyverse"}\NormalTok{); }\FunctionTok{require}\NormalTok{(tidyverse)\}}

\NormalTok{nosaukums}\OtherTok{=}\StringTok{"ForestsSoil\_EutrophicDrained\_cell.tif"}
\NormalTok{ielasisanas\_cels}\OtherTok{=}\FunctionTok{paste0}\NormalTok{(}\StringTok{"./RasterGrids\_100m/2024/RAW/"}\NormalTok{,nosaukums)}
\NormalTok{saglabasanas\_cels}\OtherTok{=}\FunctionTok{paste0}\NormalTok{(}\StringTok{"./RasterGrids\_100m/2024/Scaled/"}\NormalTok{,nosaukums)}
\NormalTok{slanis}\OtherTok{=}\FunctionTok{rast}\NormalTok{(ielasisanas\_cels)}
\NormalTok{videjais}\OtherTok{=}\FunctionTok{global}\NormalTok{(slanis,}\AttributeTok{fun=}\StringTok{"mean"}\NormalTok{,}\AttributeTok{na.rm=}\ConstantTok{TRUE}\NormalTok{)}
\NormalTok{centrets}\OtherTok{=}\NormalTok{slanis}\SpecialCharTok{{-}}\NormalTok{videjais[,}\DecValTok{1}\NormalTok{]}
\NormalTok{standartnovirze}\OtherTok{=}\NormalTok{terra}\SpecialCharTok{::}\FunctionTok{global}\NormalTok{(centrets,}\AttributeTok{fun=}\StringTok{"rms"}\NormalTok{,}\AttributeTok{na.rm=}\ConstantTok{TRUE}\NormalTok{)}
\NormalTok{merogots}\OtherTok{=}\NormalTok{centrets}\SpecialCharTok{/}\NormalTok{standartnovirze[,}\DecValTok{1}\NormalTok{]}
\FunctionTok{writeRaster}\NormalTok{(merogots,}
      \AttributeTok{filename=}\NormalTok{saglabasanas\_cels,}
      \AttributeTok{overwrite=}\ConstantTok{TRUE}\NormalTok{)}
\end{Highlighting}
\end{Shaded}

\section{ForestsSoil\_EutrophicDrained\_r500}\label{ch06.309}

\textbf{filename:} \texttt{ForestsSoil\_EutrophicDrained\_r500.tif}

\textbf{layername:} \texttt{egv\_309}

\textbf{English name:} Fractional cover of Drained Eutrophic Forests within the 0.5
km landscape

\textbf{Latvian name:} Susinātu eitrofu mežu platības īpatsvars 0,5 km ainavā

\textbf{Procedure:} The cover fraction within a radius of 500 m around the analysis grid cell is
calculated as the area-weighted sum of the \hyperref[ch06.308]{analysis cells} inside the
buffer, using the workflow \texttt{egvtools::radius\_function()}. During the calculation of the landscape metric,
inverse distance weighted (power = 2) gap filling on the output is applied
to ensure no missing values at the edges. Then the layer is rewritten to set
its name. Finally, the layer is standardised by subtracting the arithmetic
mean and dividing by the root mean squared error.

\begin{Shaded}
\begin{Highlighting}[]
\CommentTok{\# libs {-}{-}{-}{-}}
\ControlFlowTok{if}\NormalTok{(}\SpecialCharTok{!}\FunctionTok{require}\NormalTok{(terra)) \{}\FunctionTok{install.packages}\NormalTok{(}\StringTok{"terra"}\NormalTok{); }\FunctionTok{require}\NormalTok{(terra)\}}
\ControlFlowTok{if}\NormalTok{(}\SpecialCharTok{!}\FunctionTok{require}\NormalTok{(egvtools)) \{remotes}\SpecialCharTok{::}\FunctionTok{install\_github}\NormalTok{(}\StringTok{"aavotins/egvtools"}\NormalTok{); }\FunctionTok{require}\NormalTok{(egvtools)\}}


\CommentTok{\# Templates {-}{-}{-}{-}{-}}
\NormalTok{template100}\OtherTok{=}\FunctionTok{rast}\NormalTok{(}\StringTok{"./Templates/TemplateRasters/LV100m\_10km.tif"}\NormalTok{)}

\CommentTok{\# radii {-}{-}{-}{-}}
\FunctionTok{radius\_function}\NormalTok{(}
 \AttributeTok{kvadrati\_path =} \StringTok{"./Templates/TemplateGrids/tiles/"}\NormalTok{,}
 \AttributeTok{radii\_path   =} \StringTok{"./Templates/TemplateGridPoints/tiles/"}\NormalTok{,}
 \AttributeTok{tikls100\_path =} \StringTok{"./Templates/TemplateGrids/tikls100\_sauzeme.parquet"}\NormalTok{,}
 \AttributeTok{template\_path =} \StringTok{"./Templates/TemplateRasters/LV100m\_10km.tif"}\NormalTok{,}
 \AttributeTok{input\_layers  =} \FunctionTok{c}\NormalTok{(}\StringTok{"./RasterGrids\_100m/2024/RAW/ForestsSoil\_EutrophicDrained\_cell.tif"}\NormalTok{),}
 \AttributeTok{layer\_prefixes =} \FunctionTok{c}\NormalTok{(}\StringTok{"ForestsSoil\_EutrophicDrained"}\NormalTok{),}
 \AttributeTok{output\_dir   =} \StringTok{"./RasterGrids\_100m/2024/RAW/"}\NormalTok{,}
 \AttributeTok{n\_workers   =} \DecValTok{6}\NormalTok{,}
 \AttributeTok{radii     =} \FunctionTok{c}\NormalTok{(}\StringTok{"r500"}\NormalTok{),}
 \AttributeTok{radius\_mode  =} \StringTok{"sparse"}\NormalTok{,}
 \AttributeTok{extract\_fun  =} \StringTok{"mean"}\NormalTok{,}
 \AttributeTok{fill\_missing  =} \ConstantTok{TRUE}\NormalTok{,}
 \AttributeTok{IDW\_weight   =} \DecValTok{2}\NormalTok{,}
 \AttributeTok{future\_max\_size =} \DecValTok{40} \SpecialCharTok{*} \DecValTok{1024}\SpecialCharTok{\^{}}\DecValTok{3}\NormalTok{)}


\CommentTok{\# ForestsSoil\_EutrophicDrained\_r500.tif egv\_309}
\NormalTok{slanis}\OtherTok{=}\FunctionTok{rast}\NormalTok{(}\StringTok{"./RasterGrids\_100m/2024/RAW/ForestsSoil\_EutrophicDrained\_r500.tif"}\NormalTok{)}
\FunctionTok{names}\NormalTok{(slanis)}\OtherTok{=}\StringTok{"egv\_309"}
\NormalTok{slanis2}\OtherTok{=}\FunctionTok{project}\NormalTok{(slanis,template100)}
\FunctionTok{writeRaster}\NormalTok{(slanis2,}
      \StringTok{"./RasterGrids\_100m/2024/RAW/ForestsSoil\_EutrophicDrained\_r500.tif"}\NormalTok{,}
      \AttributeTok{overwrite=}\ConstantTok{TRUE}\NormalTok{)}

\CommentTok{\# standardisation {-}{-}{-}{-}}
\ControlFlowTok{if}\NormalTok{(}\SpecialCharTok{!}\FunctionTok{require}\NormalTok{(terra)) \{}\FunctionTok{install.packages}\NormalTok{(}\StringTok{"terra"}\NormalTok{); }\FunctionTok{require}\NormalTok{(terra)\}}
\ControlFlowTok{if}\NormalTok{(}\SpecialCharTok{!}\FunctionTok{require}\NormalTok{(tidyverse)) \{}\FunctionTok{install.packages}\NormalTok{(}\StringTok{"tidyverse"}\NormalTok{); }\FunctionTok{require}\NormalTok{(tidyverse)\}}

\NormalTok{nosaukums}\OtherTok{=}\StringTok{"ForestsSoil\_EutrophicDrained\_r500.tif"}
\NormalTok{ielasisanas\_cels}\OtherTok{=}\FunctionTok{paste0}\NormalTok{(}\StringTok{"./RasterGrids\_100m/2024/RAW/"}\NormalTok{,nosaukums)}
\NormalTok{saglabasanas\_cels}\OtherTok{=}\FunctionTok{paste0}\NormalTok{(}\StringTok{"./RasterGrids\_100m/2024/Scaled/"}\NormalTok{,nosaukums)}
\NormalTok{slanis}\OtherTok{=}\FunctionTok{rast}\NormalTok{(ielasisanas\_cels)}
\NormalTok{videjais}\OtherTok{=}\FunctionTok{global}\NormalTok{(slanis,}\AttributeTok{fun=}\StringTok{"mean"}\NormalTok{,}\AttributeTok{na.rm=}\ConstantTok{TRUE}\NormalTok{)}
\NormalTok{centrets}\OtherTok{=}\NormalTok{slanis}\SpecialCharTok{{-}}\NormalTok{videjais[,}\DecValTok{1}\NormalTok{]}
\NormalTok{standartnovirze}\OtherTok{=}\NormalTok{terra}\SpecialCharTok{::}\FunctionTok{global}\NormalTok{(centrets,}\AttributeTok{fun=}\StringTok{"rms"}\NormalTok{,}\AttributeTok{na.rm=}\ConstantTok{TRUE}\NormalTok{)}
\NormalTok{merogots}\OtherTok{=}\NormalTok{centrets}\SpecialCharTok{/}\NormalTok{standartnovirze[,}\DecValTok{1}\NormalTok{]}
\FunctionTok{writeRaster}\NormalTok{(merogots,}
      \AttributeTok{filename=}\NormalTok{saglabasanas\_cels,}
      \AttributeTok{overwrite=}\ConstantTok{TRUE}\NormalTok{)}
\end{Highlighting}
\end{Shaded}

\section{ForestsSoil\_EutrophicDrained\_r1250}\label{ch06.310}

\textbf{filename:} \texttt{ForestsSoil\_EutrophicDrained\_r1250.tif}

\textbf{layername:} \texttt{egv\_310}

\textbf{English name:} Fractional cover of Drained Eutrophic Forests within the 1.25
km landscape

\textbf{Latvian name:} Susinātu eitrofu mežu platības īpatsvars 1,25 km ainavā

\textbf{Procedure:} The cover fraction within a radius of 1250 m around the analysis grid cell
is calculated as the area-weighted sum of the \hyperref[ch06.308]{analysis cells} inside
the buffer, using the workflow \texttt{egvtools::radius\_function()}. During the calculation of the landscape
metric, inverse distance weighted (power = 2) gap filling on the output is
applied to ensure no missing values at the edges. Then the layer is
rewritten to set its name. Finally, the layer is standardised by
subtracting the arithmetic mean and dividing by the root mean squared error.

\begin{Shaded}
\begin{Highlighting}[]
\CommentTok{\# libs {-}{-}{-}{-}}
\ControlFlowTok{if}\NormalTok{(}\SpecialCharTok{!}\FunctionTok{require}\NormalTok{(terra)) \{}\FunctionTok{install.packages}\NormalTok{(}\StringTok{"terra"}\NormalTok{); }\FunctionTok{require}\NormalTok{(terra)\}}
\ControlFlowTok{if}\NormalTok{(}\SpecialCharTok{!}\FunctionTok{require}\NormalTok{(egvtools)) \{remotes}\SpecialCharTok{::}\FunctionTok{install\_github}\NormalTok{(}\StringTok{"aavotins/egvtools"}\NormalTok{); }\FunctionTok{require}\NormalTok{(egvtools)\}}


\CommentTok{\# Templates {-}{-}{-}{-}{-}}
\NormalTok{template100}\OtherTok{=}\FunctionTok{rast}\NormalTok{(}\StringTok{"./Templates/TemplateRasters/LV100m\_10km.tif"}\NormalTok{)}

\CommentTok{\# radii {-}{-}{-}{-}}
\FunctionTok{radius\_function}\NormalTok{(}
 \AttributeTok{kvadrati\_path =} \StringTok{"./Templates/TemplateGrids/tiles/"}\NormalTok{,}
 \AttributeTok{radii\_path   =} \StringTok{"./Templates/TemplateGridPoints/tiles/"}\NormalTok{,}
 \AttributeTok{tikls100\_path =} \StringTok{"./Templates/TemplateGrids/tikls100\_sauzeme.parquet"}\NormalTok{,}
 \AttributeTok{template\_path =} \StringTok{"./Templates/TemplateRasters/LV100m\_10km.tif"}\NormalTok{,}
 \AttributeTok{input\_layers  =} \FunctionTok{c}\NormalTok{(}\StringTok{"./RasterGrids\_100m/2024/RAW/ForestsSoil\_EutrophicDrained\_cell.tif"}\NormalTok{),}
 \AttributeTok{layer\_prefixes =} \FunctionTok{c}\NormalTok{(}\StringTok{"ForestsSoil\_EutrophicDrained"}\NormalTok{),}
 \AttributeTok{output\_dir   =} \StringTok{"./RasterGrids\_100m/2024/RAW/"}\NormalTok{,}
 \AttributeTok{n\_workers   =} \DecValTok{6}\NormalTok{,}
 \AttributeTok{radii     =} \FunctionTok{c}\NormalTok{(}\StringTok{"r1250"}\NormalTok{),}
 \AttributeTok{radius\_mode  =} \StringTok{"sparse"}\NormalTok{,}
 \AttributeTok{extract\_fun  =} \StringTok{"mean"}\NormalTok{,}
 \AttributeTok{fill\_missing  =} \ConstantTok{TRUE}\NormalTok{,}
 \AttributeTok{IDW\_weight   =} \DecValTok{2}\NormalTok{,}
 \AttributeTok{future\_max\_size =} \DecValTok{40} \SpecialCharTok{*} \DecValTok{1024}\SpecialCharTok{\^{}}\DecValTok{3}\NormalTok{)}


\CommentTok{\# ForestsSoil\_EutrophicDrained\_r1250.tif    egv\_310}
\NormalTok{slanis}\OtherTok{=}\FunctionTok{rast}\NormalTok{(}\StringTok{"./RasterGrids\_100m/2024/RAW/ForestsSoil\_EutrophicDrained\_r1250.tif"}\NormalTok{)}
\FunctionTok{names}\NormalTok{(slanis)}\OtherTok{=}\StringTok{"egv\_310"}
\NormalTok{slanis2}\OtherTok{=}\FunctionTok{project}\NormalTok{(slanis,template100)}
\FunctionTok{writeRaster}\NormalTok{(slanis2,}
      \StringTok{"./RasterGrids\_100m/2024/RAW/ForestsSoil\_EutrophicDrained\_r1250.tif"}\NormalTok{,}
      \AttributeTok{overwrite=}\ConstantTok{TRUE}\NormalTok{)}

\CommentTok{\# standardisation {-}{-}{-}{-}}
\ControlFlowTok{if}\NormalTok{(}\SpecialCharTok{!}\FunctionTok{require}\NormalTok{(terra)) \{}\FunctionTok{install.packages}\NormalTok{(}\StringTok{"terra"}\NormalTok{); }\FunctionTok{require}\NormalTok{(terra)\}}
\ControlFlowTok{if}\NormalTok{(}\SpecialCharTok{!}\FunctionTok{require}\NormalTok{(tidyverse)) \{}\FunctionTok{install.packages}\NormalTok{(}\StringTok{"tidyverse"}\NormalTok{); }\FunctionTok{require}\NormalTok{(tidyverse)\}}

\NormalTok{nosaukums}\OtherTok{=}\StringTok{"ForestsSoil\_EutrophicDrained\_r1250.tif"}
\NormalTok{ielasisanas\_cels}\OtherTok{=}\FunctionTok{paste0}\NormalTok{(}\StringTok{"./RasterGrids\_100m/2024/RAW/"}\NormalTok{,nosaukums)}
\NormalTok{saglabasanas\_cels}\OtherTok{=}\FunctionTok{paste0}\NormalTok{(}\StringTok{"./RasterGrids\_100m/2024/Scaled/"}\NormalTok{,nosaukums)}
\NormalTok{slanis}\OtherTok{=}\FunctionTok{rast}\NormalTok{(ielasisanas\_cels)}
\NormalTok{videjais}\OtherTok{=}\FunctionTok{global}\NormalTok{(slanis,}\AttributeTok{fun=}\StringTok{"mean"}\NormalTok{,}\AttributeTok{na.rm=}\ConstantTok{TRUE}\NormalTok{)}
\NormalTok{centrets}\OtherTok{=}\NormalTok{slanis}\SpecialCharTok{{-}}\NormalTok{videjais[,}\DecValTok{1}\NormalTok{]}
\NormalTok{standartnovirze}\OtherTok{=}\NormalTok{terra}\SpecialCharTok{::}\FunctionTok{global}\NormalTok{(centrets,}\AttributeTok{fun=}\StringTok{"rms"}\NormalTok{,}\AttributeTok{na.rm=}\ConstantTok{TRUE}\NormalTok{)}
\NormalTok{merogots}\OtherTok{=}\NormalTok{centrets}\SpecialCharTok{/}\NormalTok{standartnovirze[,}\DecValTok{1}\NormalTok{]}
\FunctionTok{writeRaster}\NormalTok{(merogots,}
      \AttributeTok{filename=}\NormalTok{saglabasanas\_cels,}
      \AttributeTok{overwrite=}\ConstantTok{TRUE}\NormalTok{)}
\end{Highlighting}
\end{Shaded}

\section{ForestsSoil\_EutrophicDrained\_r3000}\label{ch06.311}

\textbf{filename:} \texttt{ForestsSoil\_EutrophicDrained\_r3000.tif}

\textbf{layername:} \texttt{egv\_311}

\textbf{English name:} Fractional cover of Drained Eutrophic Forests within the 3 km
landscape

\textbf{Latvian name:} Susinātu eitrofu mežu platības īpatsvars 3 km ainavā

\textbf{Procedure:} The cover fraction within a radius of 3000 m around the analysis grid cell
is calculated as the area-weighted sum of the \hyperref[ch06.308]{analysis cells} inside
the buffer, using the workflow \texttt{egvtools::radius\_function()}. During the calculation of the landscape
metric, inverse distance weighted (power = 2) gap filling on the output is
applied to ensure no missing values at the edges. Then the layer is
rewritten to set its name. Finally, the layer is standardised by
subtracting the arithmetic mean and dividing by the root mean squared error.

\begin{Shaded}
\begin{Highlighting}[]
\CommentTok{\# libs {-}{-}{-}{-}}
\ControlFlowTok{if}\NormalTok{(}\SpecialCharTok{!}\FunctionTok{require}\NormalTok{(terra)) \{}\FunctionTok{install.packages}\NormalTok{(}\StringTok{"terra"}\NormalTok{); }\FunctionTok{require}\NormalTok{(terra)\}}
\ControlFlowTok{if}\NormalTok{(}\SpecialCharTok{!}\FunctionTok{require}\NormalTok{(egvtools)) \{remotes}\SpecialCharTok{::}\FunctionTok{install\_github}\NormalTok{(}\StringTok{"aavotins/egvtools"}\NormalTok{); }\FunctionTok{require}\NormalTok{(egvtools)\}}


\CommentTok{\# Templates {-}{-}{-}{-}{-}}
\NormalTok{template100}\OtherTok{=}\FunctionTok{rast}\NormalTok{(}\StringTok{"./Templates/TemplateRasters/LV100m\_10km.tif"}\NormalTok{)}

\CommentTok{\# radii {-}{-}{-}{-}}
\FunctionTok{radius\_function}\NormalTok{(}
 \AttributeTok{kvadrati\_path =} \StringTok{"./Templates/TemplateGrids/tiles/"}\NormalTok{,}
 \AttributeTok{radii\_path   =} \StringTok{"./Templates/TemplateGridPoints/tiles/"}\NormalTok{,}
 \AttributeTok{tikls100\_path =} \StringTok{"./Templates/TemplateGrids/tikls100\_sauzeme.parquet"}\NormalTok{,}
 \AttributeTok{template\_path =} \StringTok{"./Templates/TemplateRasters/LV100m\_10km.tif"}\NormalTok{,}
 \AttributeTok{input\_layers  =} \FunctionTok{c}\NormalTok{(}\StringTok{"./RasterGrids\_100m/2024/RAW/ForestsSoil\_EutrophicDrained\_cell.tif"}\NormalTok{),}
 \AttributeTok{layer\_prefixes =} \FunctionTok{c}\NormalTok{(}\StringTok{"ForestsSoil\_EutrophicDrained"}\NormalTok{),}
 \AttributeTok{output\_dir   =} \StringTok{"./RasterGrids\_100m/2024/RAW/"}\NormalTok{,}
 \AttributeTok{n\_workers   =} \DecValTok{6}\NormalTok{,}
 \AttributeTok{radii     =} \FunctionTok{c}\NormalTok{(}\StringTok{"r3000"}\NormalTok{),}
 \AttributeTok{radius\_mode  =} \StringTok{"sparse"}\NormalTok{,}
 \AttributeTok{extract\_fun  =} \StringTok{"mean"}\NormalTok{,}
 \AttributeTok{fill\_missing  =} \ConstantTok{TRUE}\NormalTok{,}
 \AttributeTok{IDW\_weight   =} \DecValTok{2}\NormalTok{,}
 \AttributeTok{future\_max\_size =} \DecValTok{40} \SpecialCharTok{*} \DecValTok{1024}\SpecialCharTok{\^{}}\DecValTok{3}\NormalTok{)}


\CommentTok{\# ForestsSoil\_EutrophicDrained\_r3000.tif    egv\_311}
\NormalTok{slanis}\OtherTok{=}\FunctionTok{rast}\NormalTok{(}\StringTok{"./RasterGrids\_100m/2024/RAW/ForestsSoil\_EutrophicDrained\_r3000.tif"}\NormalTok{)}
\FunctionTok{names}\NormalTok{(slanis)}\OtherTok{=}\StringTok{"egv\_311"}
\NormalTok{slanis2}\OtherTok{=}\FunctionTok{project}\NormalTok{(slanis,template100)}
\FunctionTok{writeRaster}\NormalTok{(slanis2,}
      \StringTok{"./RasterGrids\_100m/2024/RAW/ForestsSoil\_EutrophicDrained\_r3000.tif"}\NormalTok{,}
      \AttributeTok{overwrite=}\ConstantTok{TRUE}\NormalTok{)}

\CommentTok{\# standardisation {-}{-}{-}{-}}
\ControlFlowTok{if}\NormalTok{(}\SpecialCharTok{!}\FunctionTok{require}\NormalTok{(terra)) \{}\FunctionTok{install.packages}\NormalTok{(}\StringTok{"terra"}\NormalTok{); }\FunctionTok{require}\NormalTok{(terra)\}}
\ControlFlowTok{if}\NormalTok{(}\SpecialCharTok{!}\FunctionTok{require}\NormalTok{(tidyverse)) \{}\FunctionTok{install.packages}\NormalTok{(}\StringTok{"tidyverse"}\NormalTok{); }\FunctionTok{require}\NormalTok{(tidyverse)\}}

\NormalTok{nosaukums}\OtherTok{=}\StringTok{"ForestsSoil\_EutrophicDrained\_r3000.tif"}
\NormalTok{ielasisanas\_cels}\OtherTok{=}\FunctionTok{paste0}\NormalTok{(}\StringTok{"./RasterGrids\_100m/2024/RAW/"}\NormalTok{,nosaukums)}
\NormalTok{saglabasanas\_cels}\OtherTok{=}\FunctionTok{paste0}\NormalTok{(}\StringTok{"./RasterGrids\_100m/2024/Scaled/"}\NormalTok{,nosaukums)}
\NormalTok{slanis}\OtherTok{=}\FunctionTok{rast}\NormalTok{(ielasisanas\_cels)}
\NormalTok{videjais}\OtherTok{=}\FunctionTok{global}\NormalTok{(slanis,}\AttributeTok{fun=}\StringTok{"mean"}\NormalTok{,}\AttributeTok{na.rm=}\ConstantTok{TRUE}\NormalTok{)}
\NormalTok{centrets}\OtherTok{=}\NormalTok{slanis}\SpecialCharTok{{-}}\NormalTok{videjais[,}\DecValTok{1}\NormalTok{]}
\NormalTok{standartnovirze}\OtherTok{=}\NormalTok{terra}\SpecialCharTok{::}\FunctionTok{global}\NormalTok{(centrets,}\AttributeTok{fun=}\StringTok{"rms"}\NormalTok{,}\AttributeTok{na.rm=}\ConstantTok{TRUE}\NormalTok{)}
\NormalTok{merogots}\OtherTok{=}\NormalTok{centrets}\SpecialCharTok{/}\NormalTok{standartnovirze[,}\DecValTok{1}\NormalTok{]}
\FunctionTok{writeRaster}\NormalTok{(merogots,}
      \AttributeTok{filename=}\NormalTok{saglabasanas\_cels,}
      \AttributeTok{overwrite=}\ConstantTok{TRUE}\NormalTok{)}
\end{Highlighting}
\end{Shaded}

\section{ForestsSoil\_EutrophicDrained\_r10000}\label{ch06.312}

\textbf{filename:} \texttt{ForestsSoil\_EutrophicDrained\_r10000.tif}

\textbf{layername:} \texttt{egv\_312}

\textbf{English name:} Fractional cover of Drained Eutrophic Forests within the 10 km
landscape

\textbf{Latvian name:} Susinātu eitrofu mežu platības īpatsvars 10 km ainavā

\textbf{Procedure:} The cover fraction within a radius of 10000 m around the analysis grid cell
is calculated as the area-weighted sum of the \hyperref[ch06.308]{analysis cells} inside
the buffer, using the workflow \texttt{egvtools::radius\_function()}. During the calculation of the landscape
metric, inverse distance weighted (power = 2) gap filling on the output is
applied to ensure no missing values at the edges. Then the layer is
rewritten to set its name. Finally, the layer is standardised by
subtracting the arithmetic mean and dividing by the root mean squared error.

\begin{Shaded}
\begin{Highlighting}[]
\CommentTok{\# libs {-}{-}{-}{-}}
\ControlFlowTok{if}\NormalTok{(}\SpecialCharTok{!}\FunctionTok{require}\NormalTok{(terra)) \{}\FunctionTok{install.packages}\NormalTok{(}\StringTok{"terra"}\NormalTok{); }\FunctionTok{require}\NormalTok{(terra)\}}
\ControlFlowTok{if}\NormalTok{(}\SpecialCharTok{!}\FunctionTok{require}\NormalTok{(egvtools)) \{remotes}\SpecialCharTok{::}\FunctionTok{install\_github}\NormalTok{(}\StringTok{"aavotins/egvtools"}\NormalTok{); }\FunctionTok{require}\NormalTok{(egvtools)\}}


\CommentTok{\# Templates {-}{-}{-}{-}{-}}
\NormalTok{template100}\OtherTok{=}\FunctionTok{rast}\NormalTok{(}\StringTok{"./Templates/TemplateRasters/LV100m\_10km.tif"}\NormalTok{)}

\CommentTok{\# radii {-}{-}{-}{-}}
\FunctionTok{radius\_function}\NormalTok{(}
 \AttributeTok{kvadrati\_path =} \StringTok{"./Templates/TemplateGrids/tiles/"}\NormalTok{,}
 \AttributeTok{radii\_path   =} \StringTok{"./Templates/TemplateGridPoints/tiles/"}\NormalTok{,}
 \AttributeTok{tikls100\_path =} \StringTok{"./Templates/TemplateGrids/tikls100\_sauzeme.parquet"}\NormalTok{,}
 \AttributeTok{template\_path =} \StringTok{"./Templates/TemplateRasters/LV100m\_10km.tif"}\NormalTok{,}
 \AttributeTok{input\_layers  =} \FunctionTok{c}\NormalTok{(}\StringTok{"./RasterGrids\_100m/2024/RAW/ForestsSoil\_EutrophicDrained\_cell.tif"}\NormalTok{),}
 \AttributeTok{layer\_prefixes =} \FunctionTok{c}\NormalTok{(}\StringTok{"ForestsSoil\_EutrophicDrained"}\NormalTok{),}
 \AttributeTok{output\_dir   =} \StringTok{"./RasterGrids\_100m/2024/RAW/"}\NormalTok{,}
 \AttributeTok{n\_workers   =} \DecValTok{6}\NormalTok{,}
 \AttributeTok{radii     =} \FunctionTok{c}\NormalTok{(}\StringTok{"r10000"}\NormalTok{),}
 \AttributeTok{radius\_mode  =} \StringTok{"sparse"}\NormalTok{,}
 \AttributeTok{extract\_fun  =} \StringTok{"mean"}\NormalTok{,}
 \AttributeTok{fill\_missing  =} \ConstantTok{TRUE}\NormalTok{,}
 \AttributeTok{IDW\_weight   =} \DecValTok{2}\NormalTok{,}
 \AttributeTok{future\_max\_size =} \DecValTok{40} \SpecialCharTok{*} \DecValTok{1024}\SpecialCharTok{\^{}}\DecValTok{3}\NormalTok{)}


\CommentTok{\# ForestsSoil\_EutrophicDrained\_r10000.tif   egv\_312}
\NormalTok{slanis}\OtherTok{=}\FunctionTok{rast}\NormalTok{(}\StringTok{"./RasterGrids\_100m/2024/RAW/ForestsSoil\_EutrophicDrained\_r10000.tif"}\NormalTok{)}
\FunctionTok{names}\NormalTok{(slanis)}\OtherTok{=}\StringTok{"egv\_312"}
\NormalTok{slanis2}\OtherTok{=}\FunctionTok{project}\NormalTok{(slanis,template100)}
\FunctionTok{writeRaster}\NormalTok{(slanis2,}
      \StringTok{"./RasterGrids\_100m/2024/RAW/ForestsSoil\_EutrophicDrained\_r10000.tif"}\NormalTok{,}
      \AttributeTok{overwrite=}\ConstantTok{TRUE}\NormalTok{)}

\CommentTok{\# standardisation {-}{-}{-}{-}}
\ControlFlowTok{if}\NormalTok{(}\SpecialCharTok{!}\FunctionTok{require}\NormalTok{(terra)) \{}\FunctionTok{install.packages}\NormalTok{(}\StringTok{"terra"}\NormalTok{); }\FunctionTok{require}\NormalTok{(terra)\}}
\ControlFlowTok{if}\NormalTok{(}\SpecialCharTok{!}\FunctionTok{require}\NormalTok{(tidyverse)) \{}\FunctionTok{install.packages}\NormalTok{(}\StringTok{"tidyverse"}\NormalTok{); }\FunctionTok{require}\NormalTok{(tidyverse)\}}

\NormalTok{nosaukums}\OtherTok{=}\StringTok{"ForestsSoil\_EutrophicDrained\_r10000.tif"}
\NormalTok{ielasisanas\_cels}\OtherTok{=}\FunctionTok{paste0}\NormalTok{(}\StringTok{"./RasterGrids\_100m/2024/RAW/"}\NormalTok{,nosaukums)}
\NormalTok{saglabasanas\_cels}\OtherTok{=}\FunctionTok{paste0}\NormalTok{(}\StringTok{"./RasterGrids\_100m/2024/Scaled/"}\NormalTok{,nosaukums)}
\NormalTok{slanis}\OtherTok{=}\FunctionTok{rast}\NormalTok{(ielasisanas\_cels)}
\NormalTok{videjais}\OtherTok{=}\FunctionTok{global}\NormalTok{(slanis,}\AttributeTok{fun=}\StringTok{"mean"}\NormalTok{,}\AttributeTok{na.rm=}\ConstantTok{TRUE}\NormalTok{)}
\NormalTok{centrets}\OtherTok{=}\NormalTok{slanis}\SpecialCharTok{{-}}\NormalTok{videjais[,}\DecValTok{1}\NormalTok{]}
\NormalTok{standartnovirze}\OtherTok{=}\NormalTok{terra}\SpecialCharTok{::}\FunctionTok{global}\NormalTok{(centrets,}\AttributeTok{fun=}\StringTok{"rms"}\NormalTok{,}\AttributeTok{na.rm=}\ConstantTok{TRUE}\NormalTok{)}
\NormalTok{merogots}\OtherTok{=}\NormalTok{centrets}\SpecialCharTok{/}\NormalTok{standartnovirze[,}\DecValTok{1}\NormalTok{]}
\FunctionTok{writeRaster}\NormalTok{(merogots,}
      \AttributeTok{filename=}\NormalTok{saglabasanas\_cels,}
      \AttributeTok{overwrite=}\ConstantTok{TRUE}\NormalTok{)}
\end{Highlighting}
\end{Shaded}

\section{ForestsSoil\_EutrophicMineral\_cell}\label{ch06.313}

\textbf{filename:} \texttt{ForestsSoil\_EutrophicMineral\_cell.tif}

\textbf{layername:} \texttt{egv\_313}

\textbf{English name:} Fractional cover of Eutrophic Forests on undrained Mineral
Soils within the analysis cell (1 ha)

\textbf{Latvian name:} Eitrofu mežu nesusinātās minerālaugsnēs platības īpatsvars
analīzes šūnā (1 ha)

\textbf{Procedure:} To prepare this EGV, forest stands with forest type equal to ``5'',
``6'', ``10'' or ``11'' are selected from the \hyperref[Ch04.01]{State Forest Service's State Forest
Registry} and rasterised. Rasterisation is performed using
the workflow \texttt{egvtools::polygon2input()} with background
covering (value 0). The resulting layer
is then aggregated to EGV resolution using the workflow \texttt{egvtools::input2egv()}, which
calculates the arithmetic mean to determine the cover fraction. During
aggregation, inverse distance weighted (power = 2) gap filling on the output is
applied to ensure no missing values at the edges. Finally, the layer is
standardised by subtracting the arithmetic mean and dividing by the root mean squared
error.

\begin{Shaded}
\begin{Highlighting}[]
\CommentTok{\# libs {-}{-}{-}{-}}
\ControlFlowTok{if}\NormalTok{(}\SpecialCharTok{!}\FunctionTok{require}\NormalTok{(egvtools)) \{remotes}\SpecialCharTok{::}\FunctionTok{install\_github}\NormalTok{(}\StringTok{"aavotins/egvtools"}\NormalTok{); }\FunctionTok{require}\NormalTok{(egvtools)\}}
\ControlFlowTok{if}\NormalTok{(}\SpecialCharTok{!}\FunctionTok{require}\NormalTok{(terra)) \{}\FunctionTok{install.packages}\NormalTok{(}\StringTok{"terra"}\NormalTok{); }\FunctionTok{require}\NormalTok{(terra)\}}
\ControlFlowTok{if}\NormalTok{(}\SpecialCharTok{!}\FunctionTok{require}\NormalTok{(sf)) \{}\FunctionTok{install.packages}\NormalTok{(}\StringTok{"sf"}\NormalTok{); }\FunctionTok{require}\NormalTok{(sf)\}}
\ControlFlowTok{if}\NormalTok{(}\SpecialCharTok{!}\FunctionTok{require}\NormalTok{(tidyverse)) \{}\FunctionTok{install.packages}\NormalTok{(}\StringTok{"tidyverse"}\NormalTok{); }\FunctionTok{require}\NormalTok{(tidyverse)\}}
\ControlFlowTok{if}\NormalTok{(}\SpecialCharTok{!}\FunctionTok{require}\NormalTok{(sfarrow)) \{}\FunctionTok{install.packages}\NormalTok{(}\StringTok{"sfarrow"}\NormalTok{); }\FunctionTok{require}\NormalTok{(sfarrow)\}}
\ControlFlowTok{if}\NormalTok{(}\SpecialCharTok{!}\FunctionTok{require}\NormalTok{(readxl)) \{}\FunctionTok{install.packages}\NormalTok{(}\StringTok{"readxl"}\NormalTok{); }\FunctionTok{require}\NormalTok{(readxl)\}}
\ControlFlowTok{if}\NormalTok{(}\SpecialCharTok{!}\FunctionTok{require}\NormalTok{(raster)) \{}\FunctionTok{install.packages}\NormalTok{(}\StringTok{"raster"}\NormalTok{); }\FunctionTok{require}\NormalTok{(raster)\}}
\ControlFlowTok{if}\NormalTok{(}\SpecialCharTok{!}\FunctionTok{require}\NormalTok{(fasterize)) \{}\FunctionTok{install.packages}\NormalTok{(}\StringTok{"fasterize"}\NormalTok{); }\FunctionTok{require}\NormalTok{(fasterize)\}}

\CommentTok{\# templates {-}{-}{-}{-}}
\NormalTok{template100}\OtherTok{=}\FunctionTok{rast}\NormalTok{(}\StringTok{"./Templates/TemplateRasters/LV100m\_10km.tif"}\NormalTok{)}
\NormalTok{template10}\OtherTok{=}\FunctionTok{rast}\NormalTok{(}\StringTok{"./Templates/TemplateRasters/LV10m\_10km.tif"}\NormalTok{)}
\NormalTok{rastrs10}\OtherTok{=}\FunctionTok{raster}\NormalTok{(template10)}

\NormalTok{nulls10}\OtherTok{=}\FunctionTok{rast}\NormalTok{(}\StringTok{"./Templates/TemplateRasters/nulls\_LV10m\_10km.tif"}\NormalTok{)}
\NormalTok{nulls100}\OtherTok{=}\FunctionTok{rast}\NormalTok{(}\StringTok{"./Templates/TemplateRasters/nulls\_LV100m\_10km.tif"}\NormalTok{)}


\CommentTok{\# simple landscape {-}{-}{-}{-}}
\NormalTok{simple\_landscape}\OtherTok{=}\FunctionTok{rast}\NormalTok{(}\StringTok{"RasterGrids\_10m/2024/Ainava\_vienk\_mask.tif"}\NormalTok{)}

\CommentTok{\# mvr {-}{-}{-}{-}}
\NormalTok{mvr}\OtherTok{=}\FunctionTok{st\_read\_parquet}\NormalTok{(}\StringTok{"./Geodata/2024/MVR/nogabali\_2024janv.parquet"}\NormalTok{)}
\NormalTok{mvr}\SpecialCharTok{$}\NormalTok{yes}\OtherTok{=}\DecValTok{1}


\CommentTok{\# ForestsSoil\_EutrophicMineral\_cell.tif egv\_313 {-}{-}{-}{-}}
\NormalTok{EutrophicMineral}\OtherTok{=}\NormalTok{mvr }\SpecialCharTok{\%\textgreater{}\%} 
 \FunctionTok{filter}\NormalTok{(mt }\SpecialCharTok{\%in\%} \FunctionTok{c}\NormalTok{(}\StringTok{"5"}\NormalTok{,}\StringTok{"6"}\NormalTok{,}\StringTok{"10"}\NormalTok{,}\StringTok{"11"}\NormalTok{))}
\NormalTok{p2i\_rez}\OtherTok{=}\NormalTok{egvtools}\SpecialCharTok{::}\FunctionTok{polygon2input}\NormalTok{(}\AttributeTok{vector\_data =}\NormalTok{ EutrophicMineral,}
                \AttributeTok{template\_path =} \StringTok{"./Templates/TemplateRasters/LV10m\_10km.tif"}\NormalTok{,}
                \AttributeTok{out\_path =} \StringTok{"./RasterGrids\_10m/2024/"}\NormalTok{,}
                \AttributeTok{file\_name =} \StringTok{"ForestsSoil\_EutrophicMineral\_input.tif"}\NormalTok{,}
                \AttributeTok{value\_field =} \StringTok{"yes"}\NormalTok{,}
                \AttributeTok{prepare=}\ConstantTok{FALSE}\NormalTok{,}
                \AttributeTok{background\_raster =} \StringTok{"./Templates/TemplateRasters/nulls\_LV10m\_10km.tif"}\NormalTok{,}
                \AttributeTok{plot\_result =} \ConstantTok{TRUE}\NormalTok{)}
\NormalTok{p2i\_rez}
\NormalTok{i2e\_rez}\OtherTok{=}\NormalTok{egvtools}\SpecialCharTok{::}\FunctionTok{input2egv}\NormalTok{(}\AttributeTok{input=}\FunctionTok{paste0}\NormalTok{(}\StringTok{"./RasterGrids\_10m/2024/"}\NormalTok{,}
                     \StringTok{"ForestsSoil\_EutrophicMineral\_input.tif"}\NormalTok{),}
              \AttributeTok{egv\_template=} \StringTok{"./Templates/TemplateRasters/LV100m\_10km.tif"}\NormalTok{,}
              \AttributeTok{summary\_function =} \StringTok{"average"}\NormalTok{,}
              \AttributeTok{missing\_job =} \StringTok{"FillOutput"}\NormalTok{,}
              \AttributeTok{outlocation =} \StringTok{"./RasterGrids\_100m/2024/RAW/"}\NormalTok{,}
              \AttributeTok{outfilename =} \StringTok{"ForestsSoil\_EutrophicMineral\_cell.tif"}\NormalTok{,}
              \AttributeTok{layername =} \StringTok{"egv\_313"}\NormalTok{,}
              \AttributeTok{idw\_weight =} \DecValTok{2}\NormalTok{,}
              \AttributeTok{plot\_gaps =} \ConstantTok{FALSE}\NormalTok{,}\AttributeTok{plot\_final =} \ConstantTok{TRUE}\NormalTok{)}
\NormalTok{i2e\_rez}
\FunctionTok{rm}\NormalTok{(EutrophicMineral)}
\FunctionTok{rm}\NormalTok{(p2i\_rez)}
\FunctionTok{rm}\NormalTok{(i2e\_rez)}
\FunctionTok{unlink}\NormalTok{(}\StringTok{"./RasterGrids\_10m/2024/ForestsSoil\_EutrophicMineral\_input.tif"}\NormalTok{)}

\CommentTok{\# standardisation {-}{-}{-}{-}}
\ControlFlowTok{if}\NormalTok{(}\SpecialCharTok{!}\FunctionTok{require}\NormalTok{(terra)) \{}\FunctionTok{install.packages}\NormalTok{(}\StringTok{"terra"}\NormalTok{); }\FunctionTok{require}\NormalTok{(terra)\}}
\ControlFlowTok{if}\NormalTok{(}\SpecialCharTok{!}\FunctionTok{require}\NormalTok{(tidyverse)) \{}\FunctionTok{install.packages}\NormalTok{(}\StringTok{"tidyverse"}\NormalTok{); }\FunctionTok{require}\NormalTok{(tidyverse)\}}

\NormalTok{nosaukums}\OtherTok{=}\StringTok{"ForestsSoil\_EutrophicMineral\_cell.tif"}
\NormalTok{ielasisanas\_cels}\OtherTok{=}\FunctionTok{paste0}\NormalTok{(}\StringTok{"./RasterGrids\_100m/2024/RAW/"}\NormalTok{,nosaukums)}
\NormalTok{saglabasanas\_cels}\OtherTok{=}\FunctionTok{paste0}\NormalTok{(}\StringTok{"./RasterGrids\_100m/2024/Scaled/"}\NormalTok{,nosaukums)}
\NormalTok{slanis}\OtherTok{=}\FunctionTok{rast}\NormalTok{(ielasisanas\_cels)}
\NormalTok{videjais}\OtherTok{=}\FunctionTok{global}\NormalTok{(slanis,}\AttributeTok{fun=}\StringTok{"mean"}\NormalTok{,}\AttributeTok{na.rm=}\ConstantTok{TRUE}\NormalTok{)}
\NormalTok{centrets}\OtherTok{=}\NormalTok{slanis}\SpecialCharTok{{-}}\NormalTok{videjais[,}\DecValTok{1}\NormalTok{]}
\NormalTok{standartnovirze}\OtherTok{=}\NormalTok{terra}\SpecialCharTok{::}\FunctionTok{global}\NormalTok{(centrets,}\AttributeTok{fun=}\StringTok{"rms"}\NormalTok{,}\AttributeTok{na.rm=}\ConstantTok{TRUE}\NormalTok{)}
\NormalTok{merogots}\OtherTok{=}\NormalTok{centrets}\SpecialCharTok{/}\NormalTok{standartnovirze[,}\DecValTok{1}\NormalTok{]}
\FunctionTok{writeRaster}\NormalTok{(merogots,}
      \AttributeTok{filename=}\NormalTok{saglabasanas\_cels,}
      \AttributeTok{overwrite=}\ConstantTok{TRUE}\NormalTok{)}
\end{Highlighting}
\end{Shaded}

\section{ForestsSoil\_EutrophicMineral\_r500}\label{ch06.314}

\textbf{filename:} \texttt{ForestsSoil\_EutrophicMineral\_r500.tif}

\textbf{layername:} \texttt{egv\_314}

\textbf{English name:} Fractional cover of Eutrophic Forests on undrained Mineral
Soils within the 0.5 km landscape

\textbf{Latvian name:} Eitrofu mežu nesusinātās minerālaugsnēs platības īpatsvars 0,5
km ainavā

\textbf{Procedure:} The cover fraction within a radius of 500 m around the analysis grid cell is
calculated as the area-weighted sum of the \hyperref[ch06.313]{analysis cells} inside the
buffer, using the workflow \texttt{egvtools::radius\_function()}. During the calculation of the landscape metric,
inverse distance weighted (power = 2) gap filling on the output is applied
to ensure no missing values at the edges. Then the layer is rewritten to set
its name. Finally, the layer is standardised by subtracting the arithmetic
mean and dividing by the root mean squared error.

\begin{Shaded}
\begin{Highlighting}[]
\CommentTok{\# libs {-}{-}{-}{-}}
\ControlFlowTok{if}\NormalTok{(}\SpecialCharTok{!}\FunctionTok{require}\NormalTok{(terra)) \{}\FunctionTok{install.packages}\NormalTok{(}\StringTok{"terra"}\NormalTok{); }\FunctionTok{require}\NormalTok{(terra)\}}
\ControlFlowTok{if}\NormalTok{(}\SpecialCharTok{!}\FunctionTok{require}\NormalTok{(egvtools)) \{remotes}\SpecialCharTok{::}\FunctionTok{install\_github}\NormalTok{(}\StringTok{"aavotins/egvtools"}\NormalTok{); }\FunctionTok{require}\NormalTok{(egvtools)\}}


\CommentTok{\# Templates {-}{-}{-}{-}{-}}
\NormalTok{template100}\OtherTok{=}\FunctionTok{rast}\NormalTok{(}\StringTok{"./Templates/TemplateRasters/LV100m\_10km.tif"}\NormalTok{)}

\CommentTok{\# radii {-}{-}{-}{-}}
\FunctionTok{radius\_function}\NormalTok{(}
 \AttributeTok{kvadrati\_path =} \StringTok{"./Templates/TemplateGrids/tiles/"}\NormalTok{,}
 \AttributeTok{radii\_path   =} \StringTok{"./Templates/TemplateGridPoints/tiles/"}\NormalTok{,}
 \AttributeTok{tikls100\_path =} \StringTok{"./Templates/TemplateGrids/tikls100\_sauzeme.parquet"}\NormalTok{,}
 \AttributeTok{template\_path =} \StringTok{"./Templates/TemplateRasters/LV100m\_10km.tif"}\NormalTok{,}
 \AttributeTok{input\_layers  =} \FunctionTok{c}\NormalTok{(}\StringTok{"./RasterGrids\_100m/2024/RAW/ForestsSoil\_EutrophicMineral\_cell.tif"}\NormalTok{),}
 \AttributeTok{layer\_prefixes =} \FunctionTok{c}\NormalTok{(}\StringTok{"ForestsSoil\_EutrophicMineral"}\NormalTok{),}
 \AttributeTok{output\_dir   =} \StringTok{"./RasterGrids\_100m/2024/RAW/"}\NormalTok{,}
 \AttributeTok{n\_workers   =} \DecValTok{6}\NormalTok{,}
 \AttributeTok{radii     =} \FunctionTok{c}\NormalTok{(}\StringTok{"r500"}\NormalTok{),}
 \AttributeTok{radius\_mode  =} \StringTok{"sparse"}\NormalTok{,}
 \AttributeTok{extract\_fun  =} \StringTok{"mean"}\NormalTok{,}
 \AttributeTok{fill\_missing  =} \ConstantTok{TRUE}\NormalTok{,}
 \AttributeTok{IDW\_weight   =} \DecValTok{2}\NormalTok{,}
 \AttributeTok{future\_max\_size =} \DecValTok{40} \SpecialCharTok{*} \DecValTok{1024}\SpecialCharTok{\^{}}\DecValTok{3}\NormalTok{)}


\CommentTok{\# ForestsSoil\_EutrophicMineral\_r500.tif egv\_314}
\NormalTok{slanis}\OtherTok{=}\FunctionTok{rast}\NormalTok{(}\StringTok{"./RasterGrids\_100m/2024/RAW/ForestsSoil\_EutrophicMineral\_r500.tif"}\NormalTok{)}
\FunctionTok{names}\NormalTok{(slanis)}\OtherTok{=}\StringTok{"egv\_314"}
\NormalTok{slanis2}\OtherTok{=}\FunctionTok{project}\NormalTok{(slanis,template100)}
\FunctionTok{writeRaster}\NormalTok{(slanis2,}
      \StringTok{"./RasterGrids\_100m/2024/RAW/ForestsSoil\_EutrophicMineral\_r500.tif"}\NormalTok{,}
      \AttributeTok{overwrite=}\ConstantTok{TRUE}\NormalTok{)}

\CommentTok{\# standardisation {-}{-}{-}{-}}
\ControlFlowTok{if}\NormalTok{(}\SpecialCharTok{!}\FunctionTok{require}\NormalTok{(terra)) \{}\FunctionTok{install.packages}\NormalTok{(}\StringTok{"terra"}\NormalTok{); }\FunctionTok{require}\NormalTok{(terra)\}}
\ControlFlowTok{if}\NormalTok{(}\SpecialCharTok{!}\FunctionTok{require}\NormalTok{(tidyverse)) \{}\FunctionTok{install.packages}\NormalTok{(}\StringTok{"tidyverse"}\NormalTok{); }\FunctionTok{require}\NormalTok{(tidyverse)\}}

\NormalTok{nosaukums}\OtherTok{=}\StringTok{"ForestsSoil\_EutrophicMineral\_r500.tif"}
\NormalTok{ielasisanas\_cels}\OtherTok{=}\FunctionTok{paste0}\NormalTok{(}\StringTok{"./RasterGrids\_100m/2024/RAW/"}\NormalTok{,nosaukums)}
\NormalTok{saglabasanas\_cels}\OtherTok{=}\FunctionTok{paste0}\NormalTok{(}\StringTok{"./RasterGrids\_100m/2024/Scaled/"}\NormalTok{,nosaukums)}
\NormalTok{slanis}\OtherTok{=}\FunctionTok{rast}\NormalTok{(ielasisanas\_cels)}
\NormalTok{videjais}\OtherTok{=}\FunctionTok{global}\NormalTok{(slanis,}\AttributeTok{fun=}\StringTok{"mean"}\NormalTok{,}\AttributeTok{na.rm=}\ConstantTok{TRUE}\NormalTok{)}
\NormalTok{centrets}\OtherTok{=}\NormalTok{slanis}\SpecialCharTok{{-}}\NormalTok{videjais[,}\DecValTok{1}\NormalTok{]}
\NormalTok{standartnovirze}\OtherTok{=}\NormalTok{terra}\SpecialCharTok{::}\FunctionTok{global}\NormalTok{(centrets,}\AttributeTok{fun=}\StringTok{"rms"}\NormalTok{,}\AttributeTok{na.rm=}\ConstantTok{TRUE}\NormalTok{)}
\NormalTok{merogots}\OtherTok{=}\NormalTok{centrets}\SpecialCharTok{/}\NormalTok{standartnovirze[,}\DecValTok{1}\NormalTok{]}
\FunctionTok{writeRaster}\NormalTok{(merogots,}
      \AttributeTok{filename=}\NormalTok{saglabasanas\_cels,}
      \AttributeTok{overwrite=}\ConstantTok{TRUE}\NormalTok{)}
\end{Highlighting}
\end{Shaded}

\section{ForestsSoil\_EutrophicMineral\_r1250}\label{ch06.315}

\textbf{filename:} \texttt{ForestsSoil\_EutrophicMineral\_r1250.tif}

\textbf{layername:} \texttt{egv\_315}

\textbf{English name:} Fractional cover of Eutrophic Forests on undrained Mineral
Soils within the 1.25 km landscape

\textbf{Latvian name:} Eitrofu mežu nesusinātās minerālaugsnēs platības īpatsvars
1,25 km ainavā

\textbf{Procedure:} The cover fraction within a radius of 1250 m around the analysis grid cell
is calculated as the area-weighted sum of the \hyperref[ch06.313]{analysis cells} inside
the buffer, using the workflow \texttt{egvtools::radius\_function()}. During the calculation of the landscape
metric, inverse distance weighted (power = 2) gap filling on the output is
applied to ensure no missing values at the edges. Then the layer is
rewritten to set its name. Finally, the layer is standardised by
subtracting the arithmetic mean and dividing by the root mean squared error.

\begin{Shaded}
\begin{Highlighting}[]
\CommentTok{\# libs {-}{-}{-}{-}}
\ControlFlowTok{if}\NormalTok{(}\SpecialCharTok{!}\FunctionTok{require}\NormalTok{(terra)) \{}\FunctionTok{install.packages}\NormalTok{(}\StringTok{"terra"}\NormalTok{); }\FunctionTok{require}\NormalTok{(terra)\}}
\ControlFlowTok{if}\NormalTok{(}\SpecialCharTok{!}\FunctionTok{require}\NormalTok{(egvtools)) \{remotes}\SpecialCharTok{::}\FunctionTok{install\_github}\NormalTok{(}\StringTok{"aavotins/egvtools"}\NormalTok{); }\FunctionTok{require}\NormalTok{(egvtools)\}}


\CommentTok{\# Templates {-}{-}{-}{-}{-}}
\NormalTok{template100}\OtherTok{=}\FunctionTok{rast}\NormalTok{(}\StringTok{"./Templates/TemplateRasters/LV100m\_10km.tif"}\NormalTok{)}

\CommentTok{\# radii {-}{-}{-}{-}}
\FunctionTok{radius\_function}\NormalTok{(}
 \AttributeTok{kvadrati\_path =} \StringTok{"./Templates/TemplateGrids/tiles/"}\NormalTok{,}
 \AttributeTok{radii\_path   =} \StringTok{"./Templates/TemplateGridPoints/tiles/"}\NormalTok{,}
 \AttributeTok{tikls100\_path =} \StringTok{"./Templates/TemplateGrids/tikls100\_sauzeme.parquet"}\NormalTok{,}
 \AttributeTok{template\_path =} \StringTok{"./Templates/TemplateRasters/LV100m\_10km.tif"}\NormalTok{,}
 \AttributeTok{input\_layers  =} \FunctionTok{c}\NormalTok{(}\StringTok{"./RasterGrids\_100m/2024/RAW/ForestsSoil\_EutrophicMineral\_cell.tif"}\NormalTok{),}
 \AttributeTok{layer\_prefixes =} \FunctionTok{c}\NormalTok{(}\StringTok{"ForestsSoil\_EutrophicMineral"}\NormalTok{),}
 \AttributeTok{output\_dir   =} \StringTok{"./RasterGrids\_100m/2024/RAW/"}\NormalTok{,}
 \AttributeTok{n\_workers   =} \DecValTok{6}\NormalTok{,}
 \AttributeTok{radii     =} \FunctionTok{c}\NormalTok{(}\StringTok{"r1250"}\NormalTok{),}
 \AttributeTok{radius\_mode  =} \StringTok{"sparse"}\NormalTok{,}
 \AttributeTok{extract\_fun  =} \StringTok{"mean"}\NormalTok{,}
 \AttributeTok{fill\_missing  =} \ConstantTok{TRUE}\NormalTok{,}
 \AttributeTok{IDW\_weight   =} \DecValTok{2}\NormalTok{,}
 \AttributeTok{future\_max\_size =} \DecValTok{40} \SpecialCharTok{*} \DecValTok{1024}\SpecialCharTok{\^{}}\DecValTok{3}\NormalTok{)}


\CommentTok{\# ForestsSoil\_EutrophicMineral\_r1250.tif    egv\_315}
\NormalTok{slanis}\OtherTok{=}\FunctionTok{rast}\NormalTok{(}\StringTok{"./RasterGrids\_100m/2024/RAW/ForestsSoil\_EutrophicMineral\_r1250.tif"}\NormalTok{)}
\FunctionTok{names}\NormalTok{(slanis)}\OtherTok{=}\StringTok{"egv\_315"}
\NormalTok{slanis2}\OtherTok{=}\FunctionTok{project}\NormalTok{(slanis,template100)}
\FunctionTok{writeRaster}\NormalTok{(slanis2,}
      \StringTok{"./RasterGrids\_100m/2024/RAW/ForestsSoil\_EutrophicMineral\_r1250.tif"}\NormalTok{,}
      \AttributeTok{overwrite=}\ConstantTok{TRUE}\NormalTok{)}

\CommentTok{\# standardisation {-}{-}{-}{-}}
\ControlFlowTok{if}\NormalTok{(}\SpecialCharTok{!}\FunctionTok{require}\NormalTok{(terra)) \{}\FunctionTok{install.packages}\NormalTok{(}\StringTok{"terra"}\NormalTok{); }\FunctionTok{require}\NormalTok{(terra)\}}
\ControlFlowTok{if}\NormalTok{(}\SpecialCharTok{!}\FunctionTok{require}\NormalTok{(tidyverse)) \{}\FunctionTok{install.packages}\NormalTok{(}\StringTok{"tidyverse"}\NormalTok{); }\FunctionTok{require}\NormalTok{(tidyverse)\}}

\NormalTok{nosaukums}\OtherTok{=}\StringTok{"ForestsSoil\_EutrophicMineral\_r1250.tif"}
\NormalTok{ielasisanas\_cels}\OtherTok{=}\FunctionTok{paste0}\NormalTok{(}\StringTok{"./RasterGrids\_100m/2024/RAW/"}\NormalTok{,nosaukums)}
\NormalTok{saglabasanas\_cels}\OtherTok{=}\FunctionTok{paste0}\NormalTok{(}\StringTok{"./RasterGrids\_100m/2024/Scaled/"}\NormalTok{,nosaukums)}
\NormalTok{slanis}\OtherTok{=}\FunctionTok{rast}\NormalTok{(ielasisanas\_cels)}
\NormalTok{videjais}\OtherTok{=}\FunctionTok{global}\NormalTok{(slanis,}\AttributeTok{fun=}\StringTok{"mean"}\NormalTok{,}\AttributeTok{na.rm=}\ConstantTok{TRUE}\NormalTok{)}
\NormalTok{centrets}\OtherTok{=}\NormalTok{slanis}\SpecialCharTok{{-}}\NormalTok{videjais[,}\DecValTok{1}\NormalTok{]}
\NormalTok{standartnovirze}\OtherTok{=}\NormalTok{terra}\SpecialCharTok{::}\FunctionTok{global}\NormalTok{(centrets,}\AttributeTok{fun=}\StringTok{"rms"}\NormalTok{,}\AttributeTok{na.rm=}\ConstantTok{TRUE}\NormalTok{)}
\NormalTok{merogots}\OtherTok{=}\NormalTok{centrets}\SpecialCharTok{/}\NormalTok{standartnovirze[,}\DecValTok{1}\NormalTok{]}
\FunctionTok{writeRaster}\NormalTok{(merogots,}
      \AttributeTok{filename=}\NormalTok{saglabasanas\_cels,}
      \AttributeTok{overwrite=}\ConstantTok{TRUE}\NormalTok{)}
\end{Highlighting}
\end{Shaded}

\section{ForestsSoil\_EutrophicMineral\_r3000}\label{ch06.316}

\textbf{filename:} \texttt{ForestsSoil\_EutrophicMineral\_r3000.tif}

\textbf{layername:} \texttt{egv\_316}

\textbf{English name:} Fractional cover of Eutrophic Forests on undrained Mineral
Soils within the 3 km landscape

\textbf{Latvian name:} Eitrofu mežu nesusinātās minerālaugsnēs platības īpatsvars 3
km ainavā

\textbf{Procedure:} The cover fraction within a radius of 3000 m around the analysis grid cell
is calculated as the area-weighted sum of the \hyperref[ch06.313]{analysis cells} inside
the buffer, using the workflow \texttt{egvtools::radius\_function()}. During the calculation of the landscape
metric, inverse distance weighted (power = 2) gap filling on the output is
applied to ensure no missing values at the edges. Then the layer is
rewritten to set its name. Finally, the layer is standardised by
subtracting the arithmetic mean and dividing by the root mean squared error.

\begin{Shaded}
\begin{Highlighting}[]
\CommentTok{\# libs {-}{-}{-}{-}}
\ControlFlowTok{if}\NormalTok{(}\SpecialCharTok{!}\FunctionTok{require}\NormalTok{(terra)) \{}\FunctionTok{install.packages}\NormalTok{(}\StringTok{"terra"}\NormalTok{); }\FunctionTok{require}\NormalTok{(terra)\}}
\ControlFlowTok{if}\NormalTok{(}\SpecialCharTok{!}\FunctionTok{require}\NormalTok{(egvtools)) \{remotes}\SpecialCharTok{::}\FunctionTok{install\_github}\NormalTok{(}\StringTok{"aavotins/egvtools"}\NormalTok{); }\FunctionTok{require}\NormalTok{(egvtools)\}}


\CommentTok{\# Templates {-}{-}{-}{-}{-}}
\NormalTok{template100}\OtherTok{=}\FunctionTok{rast}\NormalTok{(}\StringTok{"./Templates/TemplateRasters/LV100m\_10km.tif"}\NormalTok{)}

\CommentTok{\# radii {-}{-}{-}{-}}
\FunctionTok{radius\_function}\NormalTok{(}
 \AttributeTok{kvadrati\_path =} \StringTok{"./Templates/TemplateGrids/tiles/"}\NormalTok{,}
 \AttributeTok{radii\_path   =} \StringTok{"./Templates/TemplateGridPoints/tiles/"}\NormalTok{,}
 \AttributeTok{tikls100\_path =} \StringTok{"./Templates/TemplateGrids/tikls100\_sauzeme.parquet"}\NormalTok{,}
 \AttributeTok{template\_path =} \StringTok{"./Templates/TemplateRasters/LV100m\_10km.tif"}\NormalTok{,}
 \AttributeTok{input\_layers  =} \FunctionTok{c}\NormalTok{(}\StringTok{"./RasterGrids\_100m/2024/RAW/ForestsSoil\_EutrophicMineral\_cell.tif"}\NormalTok{),}
 \AttributeTok{layer\_prefixes =} \FunctionTok{c}\NormalTok{(}\StringTok{"ForestsSoil\_EutrophicMineral"}\NormalTok{),}
 \AttributeTok{output\_dir   =} \StringTok{"./RasterGrids\_100m/2024/RAW/"}\NormalTok{,}
 \AttributeTok{n\_workers   =} \DecValTok{6}\NormalTok{,}
 \AttributeTok{radii     =} \FunctionTok{c}\NormalTok{(}\StringTok{"r3000"}\NormalTok{),}
 \AttributeTok{radius\_mode  =} \StringTok{"sparse"}\NormalTok{,}
 \AttributeTok{extract\_fun  =} \StringTok{"mean"}\NormalTok{,}
 \AttributeTok{fill\_missing  =} \ConstantTok{TRUE}\NormalTok{,}
 \AttributeTok{IDW\_weight   =} \DecValTok{2}\NormalTok{,}
 \AttributeTok{future\_max\_size =} \DecValTok{40} \SpecialCharTok{*} \DecValTok{1024}\SpecialCharTok{\^{}}\DecValTok{3}\NormalTok{)}


\CommentTok{\# ForestsSoil\_EutrophicMineral\_r3000.tif    egv\_316}
\NormalTok{slanis}\OtherTok{=}\FunctionTok{rast}\NormalTok{(}\StringTok{"./RasterGrids\_100m/2024/RAW/ForestsSoil\_EutrophicMineral\_r3000.tif"}\NormalTok{)}
\FunctionTok{names}\NormalTok{(slanis)}\OtherTok{=}\StringTok{"egv\_316"}
\NormalTok{slanis2}\OtherTok{=}\FunctionTok{project}\NormalTok{(slanis,template100)}
\FunctionTok{writeRaster}\NormalTok{(slanis2,}
      \StringTok{"./RasterGrids\_100m/2024/RAW/ForestsSoil\_EutrophicMineral\_r3000.tif"}\NormalTok{,}
      \AttributeTok{overwrite=}\ConstantTok{TRUE}\NormalTok{)}

\CommentTok{\# standardisation {-}{-}{-}{-}}
\ControlFlowTok{if}\NormalTok{(}\SpecialCharTok{!}\FunctionTok{require}\NormalTok{(terra)) \{}\FunctionTok{install.packages}\NormalTok{(}\StringTok{"terra"}\NormalTok{); }\FunctionTok{require}\NormalTok{(terra)\}}
\ControlFlowTok{if}\NormalTok{(}\SpecialCharTok{!}\FunctionTok{require}\NormalTok{(tidyverse)) \{}\FunctionTok{install.packages}\NormalTok{(}\StringTok{"tidyverse"}\NormalTok{); }\FunctionTok{require}\NormalTok{(tidyverse)\}}

\NormalTok{nosaukums}\OtherTok{=}\StringTok{"ForestsSoil\_EutrophicMineral\_r3000.tif"}
\NormalTok{ielasisanas\_cels}\OtherTok{=}\FunctionTok{paste0}\NormalTok{(}\StringTok{"./RasterGrids\_100m/2024/RAW/"}\NormalTok{,nosaukums)}
\NormalTok{saglabasanas\_cels}\OtherTok{=}\FunctionTok{paste0}\NormalTok{(}\StringTok{"./RasterGrids\_100m/2024/Scaled/"}\NormalTok{,nosaukums)}
\NormalTok{slanis}\OtherTok{=}\FunctionTok{rast}\NormalTok{(ielasisanas\_cels)}
\NormalTok{videjais}\OtherTok{=}\FunctionTok{global}\NormalTok{(slanis,}\AttributeTok{fun=}\StringTok{"mean"}\NormalTok{,}\AttributeTok{na.rm=}\ConstantTok{TRUE}\NormalTok{)}
\NormalTok{centrets}\OtherTok{=}\NormalTok{slanis}\SpecialCharTok{{-}}\NormalTok{videjais[,}\DecValTok{1}\NormalTok{]}
\NormalTok{standartnovirze}\OtherTok{=}\NormalTok{terra}\SpecialCharTok{::}\FunctionTok{global}\NormalTok{(centrets,}\AttributeTok{fun=}\StringTok{"rms"}\NormalTok{,}\AttributeTok{na.rm=}\ConstantTok{TRUE}\NormalTok{)}
\NormalTok{merogots}\OtherTok{=}\NormalTok{centrets}\SpecialCharTok{/}\NormalTok{standartnovirze[,}\DecValTok{1}\NormalTok{]}
\FunctionTok{writeRaster}\NormalTok{(merogots,}
      \AttributeTok{filename=}\NormalTok{saglabasanas\_cels,}
      \AttributeTok{overwrite=}\ConstantTok{TRUE}\NormalTok{)}
\end{Highlighting}
\end{Shaded}

\section{ForestsSoil\_EutrophicMineral\_r10000}\label{ch06.317}

\textbf{filename:} \texttt{ForestsSoil\_EutrophicMineral\_r10000.tif}

\textbf{layername:} \texttt{egv\_317}

\textbf{English name:} Fractional cover of Eutrophic Forests on undrained Mineral
Soils within the 10 km landscape

\textbf{Latvian name:} Eitrofu mežu nesusinātās minerālaugsnēs platības īpatsvars 10
km ainavā

\textbf{Procedure:} The cover fraction within a radius of 10000 m around the analysis grid cell
is calculated as the area-weighted sum of the \hyperref[ch06.313]{analysis cells} inside
the buffer, using the workflow \texttt{egvtools::radius\_function()}. During the calculation of the landscape
metric, inverse distance weighted (power = 2) gap filling on the output is
applied to ensure no missing values at the edges. Then the layer is
rewritten to set its name. Finally, the layer is standardised by
subtracting the arithmetic mean and dividing by the root mean squared error.

\begin{Shaded}
\begin{Highlighting}[]
\CommentTok{\# libs {-}{-}{-}{-}}
\ControlFlowTok{if}\NormalTok{(}\SpecialCharTok{!}\FunctionTok{require}\NormalTok{(terra)) \{}\FunctionTok{install.packages}\NormalTok{(}\StringTok{"terra"}\NormalTok{); }\FunctionTok{require}\NormalTok{(terra)\}}
\ControlFlowTok{if}\NormalTok{(}\SpecialCharTok{!}\FunctionTok{require}\NormalTok{(egvtools)) \{remotes}\SpecialCharTok{::}\FunctionTok{install\_github}\NormalTok{(}\StringTok{"aavotins/egvtools"}\NormalTok{); }\FunctionTok{require}\NormalTok{(egvtools)\}}


\CommentTok{\# Templates {-}{-}{-}{-}{-}}
\NormalTok{template100}\OtherTok{=}\FunctionTok{rast}\NormalTok{(}\StringTok{"./Templates/TemplateRasters/LV100m\_10km.tif"}\NormalTok{)}

\CommentTok{\# radii {-}{-}{-}{-}}
\FunctionTok{radius\_function}\NormalTok{(}
 \AttributeTok{kvadrati\_path =} \StringTok{"./Templates/TemplateGrids/tiles/"}\NormalTok{,}
 \AttributeTok{radii\_path   =} \StringTok{"./Templates/TemplateGridPoints/tiles/"}\NormalTok{,}
 \AttributeTok{tikls100\_path =} \StringTok{"./Templates/TemplateGrids/tikls100\_sauzeme.parquet"}\NormalTok{,}
 \AttributeTok{template\_path =} \StringTok{"./Templates/TemplateRasters/LV100m\_10km.tif"}\NormalTok{,}
 \AttributeTok{input\_layers  =} \FunctionTok{c}\NormalTok{(}\StringTok{"./RasterGrids\_100m/2024/RAW/ForestsSoil\_EutrophicMineral\_cell.tif"}\NormalTok{),}
 \AttributeTok{layer\_prefixes =} \FunctionTok{c}\NormalTok{(}\StringTok{"ForestsSoil\_EutrophicMineral"}\NormalTok{),}
 \AttributeTok{output\_dir   =} \StringTok{"./RasterGrids\_100m/2024/RAW/"}\NormalTok{,}
 \AttributeTok{n\_workers   =} \DecValTok{6}\NormalTok{,}
 \AttributeTok{radii     =} \FunctionTok{c}\NormalTok{(}\StringTok{"r10000"}\NormalTok{),}
 \AttributeTok{radius\_mode  =} \StringTok{"sparse"}\NormalTok{,}
 \AttributeTok{extract\_fun  =} \StringTok{"mean"}\NormalTok{,}
 \AttributeTok{fill\_missing  =} \ConstantTok{TRUE}\NormalTok{,}
 \AttributeTok{IDW\_weight   =} \DecValTok{2}\NormalTok{,}
 \AttributeTok{future\_max\_size =} \DecValTok{40} \SpecialCharTok{*} \DecValTok{1024}\SpecialCharTok{\^{}}\DecValTok{3}\NormalTok{)}


\CommentTok{\# ForestsSoil\_EutrophicMineral\_r10000.tif   egv\_317}
\NormalTok{slanis}\OtherTok{=}\FunctionTok{rast}\NormalTok{(}\StringTok{"./RasterGrids\_100m/2024/RAW/ForestsSoil\_EutrophicMineral\_r10000.tif"}\NormalTok{)}
\FunctionTok{names}\NormalTok{(slanis)}\OtherTok{=}\StringTok{"egv\_317"}
\NormalTok{slanis2}\OtherTok{=}\FunctionTok{project}\NormalTok{(slanis,template100)}
\FunctionTok{writeRaster}\NormalTok{(slanis2,}
      \StringTok{"./RasterGrids\_100m/2024/RAW/ForestsSoil\_EutrophicMineral\_r10000.tif"}\NormalTok{,}
      \AttributeTok{overwrite=}\ConstantTok{TRUE}\NormalTok{)}

\CommentTok{\# standardisation {-}{-}{-}{-}}
\ControlFlowTok{if}\NormalTok{(}\SpecialCharTok{!}\FunctionTok{require}\NormalTok{(terra)) \{}\FunctionTok{install.packages}\NormalTok{(}\StringTok{"terra"}\NormalTok{); }\FunctionTok{require}\NormalTok{(terra)\}}
\ControlFlowTok{if}\NormalTok{(}\SpecialCharTok{!}\FunctionTok{require}\NormalTok{(tidyverse)) \{}\FunctionTok{install.packages}\NormalTok{(}\StringTok{"tidyverse"}\NormalTok{); }\FunctionTok{require}\NormalTok{(tidyverse)\}}

\NormalTok{nosaukums}\OtherTok{=}\StringTok{"ForestsSoil\_EutrophicMineral\_r10000.tif"}
\NormalTok{ielasisanas\_cels}\OtherTok{=}\FunctionTok{paste0}\NormalTok{(}\StringTok{"./RasterGrids\_100m/2024/RAW/"}\NormalTok{,nosaukums)}
\NormalTok{saglabasanas\_cels}\OtherTok{=}\FunctionTok{paste0}\NormalTok{(}\StringTok{"./RasterGrids\_100m/2024/Scaled/"}\NormalTok{,nosaukums)}
\NormalTok{slanis}\OtherTok{=}\FunctionTok{rast}\NormalTok{(ielasisanas\_cels)}
\NormalTok{videjais}\OtherTok{=}\FunctionTok{global}\NormalTok{(slanis,}\AttributeTok{fun=}\StringTok{"mean"}\NormalTok{,}\AttributeTok{na.rm=}\ConstantTok{TRUE}\NormalTok{)}
\NormalTok{centrets}\OtherTok{=}\NormalTok{slanis}\SpecialCharTok{{-}}\NormalTok{videjais[,}\DecValTok{1}\NormalTok{]}
\NormalTok{standartnovirze}\OtherTok{=}\NormalTok{terra}\SpecialCharTok{::}\FunctionTok{global}\NormalTok{(centrets,}\AttributeTok{fun=}\StringTok{"rms"}\NormalTok{,}\AttributeTok{na.rm=}\ConstantTok{TRUE}\NormalTok{)}
\NormalTok{merogots}\OtherTok{=}\NormalTok{centrets}\SpecialCharTok{/}\NormalTok{standartnovirze[,}\DecValTok{1}\NormalTok{]}
\FunctionTok{writeRaster}\NormalTok{(merogots,}
      \AttributeTok{filename=}\NormalTok{saglabasanas\_cels,}
      \AttributeTok{overwrite=}\ConstantTok{TRUE}\NormalTok{)}
\end{Highlighting}
\end{Shaded}

\section{ForestsSoil\_EutrophicOrganic\_cell}\label{ch06.318}

\textbf{filename:} \texttt{ForestsSoil\_EutrophicOrganic\_cell.tif}

\textbf{layername:} \texttt{egv\_318}

\textbf{English name:} Fractional cover of Eutrophic Forests on undrained Organic
Soils within the analysis cell (1 ha)

\textbf{Latvian name:} Eitrofu mežu nesusinātās organiskajās augsnēs platības
īpatsvars analīzes šūnā (1 ha)

\textbf{Procedure:} To prepare this EGV, forest stands with forest type equal to ``15''
or ``16'' are selected from the \hyperref[Ch04.01]{State Forest Service's State Forest
Registry} and rasterised. Rasterisation is performed using
the workflow \texttt{egvtools::polygon2input()} with background
covering (value 0). The resulting layer
is then aggregated to EGV resolution using the workflow \texttt{egvtools::input2egv()}, which
calculates the arithmetic mean to determine the cover fraction. During
aggregation, inverse distance weighted (power = 2) gap filling on the output is
applied to ensure no missing values at the edges. Finally, the layer is
standardised by subtracting the arithmetic mean and dividing by the root mean squared
error.

\begin{Shaded}
\begin{Highlighting}[]
\CommentTok{\# libs {-}{-}{-}{-}}
\ControlFlowTok{if}\NormalTok{(}\SpecialCharTok{!}\FunctionTok{require}\NormalTok{(egvtools)) \{remotes}\SpecialCharTok{::}\FunctionTok{install\_github}\NormalTok{(}\StringTok{"aavotins/egvtools"}\NormalTok{); }\FunctionTok{require}\NormalTok{(egvtools)\}}
\ControlFlowTok{if}\NormalTok{(}\SpecialCharTok{!}\FunctionTok{require}\NormalTok{(terra)) \{}\FunctionTok{install.packages}\NormalTok{(}\StringTok{"terra"}\NormalTok{); }\FunctionTok{require}\NormalTok{(terra)\}}
\ControlFlowTok{if}\NormalTok{(}\SpecialCharTok{!}\FunctionTok{require}\NormalTok{(sf)) \{}\FunctionTok{install.packages}\NormalTok{(}\StringTok{"sf"}\NormalTok{); }\FunctionTok{require}\NormalTok{(sf)\}}
\ControlFlowTok{if}\NormalTok{(}\SpecialCharTok{!}\FunctionTok{require}\NormalTok{(tidyverse)) \{}\FunctionTok{install.packages}\NormalTok{(}\StringTok{"tidyverse"}\NormalTok{); }\FunctionTok{require}\NormalTok{(tidyverse)\}}
\ControlFlowTok{if}\NormalTok{(}\SpecialCharTok{!}\FunctionTok{require}\NormalTok{(sfarrow)) \{}\FunctionTok{install.packages}\NormalTok{(}\StringTok{"sfarrow"}\NormalTok{); }\FunctionTok{require}\NormalTok{(sfarrow)\}}
\ControlFlowTok{if}\NormalTok{(}\SpecialCharTok{!}\FunctionTok{require}\NormalTok{(readxl)) \{}\FunctionTok{install.packages}\NormalTok{(}\StringTok{"readxl"}\NormalTok{); }\FunctionTok{require}\NormalTok{(readxl)\}}
\ControlFlowTok{if}\NormalTok{(}\SpecialCharTok{!}\FunctionTok{require}\NormalTok{(raster)) \{}\FunctionTok{install.packages}\NormalTok{(}\StringTok{"raster"}\NormalTok{); }\FunctionTok{require}\NormalTok{(raster)\}}
\ControlFlowTok{if}\NormalTok{(}\SpecialCharTok{!}\FunctionTok{require}\NormalTok{(fasterize)) \{}\FunctionTok{install.packages}\NormalTok{(}\StringTok{"fasterize"}\NormalTok{); }\FunctionTok{require}\NormalTok{(fasterize)\}}

\CommentTok{\# templates {-}{-}{-}{-}}
\NormalTok{template100}\OtherTok{=}\FunctionTok{rast}\NormalTok{(}\StringTok{"./Templates/TemplateRasters/LV100m\_10km.tif"}\NormalTok{)}
\NormalTok{template10}\OtherTok{=}\FunctionTok{rast}\NormalTok{(}\StringTok{"./Templates/TemplateRasters/LV10m\_10km.tif"}\NormalTok{)}
\NormalTok{rastrs10}\OtherTok{=}\FunctionTok{raster}\NormalTok{(template10)}

\NormalTok{nulls10}\OtherTok{=}\FunctionTok{rast}\NormalTok{(}\StringTok{"./Templates/TemplateRasters/nulls\_LV10m\_10km.tif"}\NormalTok{)}
\NormalTok{nulls100}\OtherTok{=}\FunctionTok{rast}\NormalTok{(}\StringTok{"./Templates/TemplateRasters/nulls\_LV100m\_10km.tif"}\NormalTok{)}


\CommentTok{\# simple landscape {-}{-}{-}{-}}
\NormalTok{simple\_landscape}\OtherTok{=}\FunctionTok{rast}\NormalTok{(}\StringTok{"RasterGrids\_10m/2024/Ainava\_vienk\_mask.tif"}\NormalTok{)}

\CommentTok{\# mvr {-}{-}{-}{-}}
\NormalTok{mvr}\OtherTok{=}\FunctionTok{st\_read\_parquet}\NormalTok{(}\StringTok{"./Geodata/2024/MVR/nogabali\_2024janv.parquet"}\NormalTok{)}
\NormalTok{mvr}\SpecialCharTok{$}\NormalTok{yes}\OtherTok{=}\DecValTok{1}


\CommentTok{\# ForestsSoil\_EutrophicOrganic\_cell.tif egv\_318 {-}{-}{-}{-}}
\NormalTok{EutrophicOrganic}\OtherTok{=}\NormalTok{mvr }\SpecialCharTok{\%\textgreater{}\%} 
 \FunctionTok{filter}\NormalTok{(mt }\SpecialCharTok{\%in\%} \FunctionTok{c}\NormalTok{(}\StringTok{"15"}\NormalTok{,}\StringTok{"16"}\NormalTok{))}
\NormalTok{p2i\_rez}\OtherTok{=}\NormalTok{egvtools}\SpecialCharTok{::}\FunctionTok{polygon2input}\NormalTok{(}\AttributeTok{vector\_data =}\NormalTok{ EutrophicOrganic,}
                \AttributeTok{template\_path =} \StringTok{"./Templates/TemplateRasters/LV10m\_10km.tif"}\NormalTok{,}
                \AttributeTok{out\_path =} \StringTok{"./RasterGrids\_10m/2024/"}\NormalTok{,}
                \AttributeTok{file\_name =} \StringTok{"ForestsSoil\_EutrophicOrganic\_input.tif"}\NormalTok{,}
                \AttributeTok{value\_field =} \StringTok{"yes"}\NormalTok{,}
                \AttributeTok{prepare=}\ConstantTok{FALSE}\NormalTok{,}
                \AttributeTok{background\_raster =} \StringTok{"./Templates/TemplateRasters/nulls\_LV10m\_10km.tif"}\NormalTok{,}
                \AttributeTok{plot\_result =} \ConstantTok{TRUE}\NormalTok{)}
\NormalTok{p2i\_rez}
\NormalTok{i2e\_rez}\OtherTok{=}\NormalTok{egvtools}\SpecialCharTok{::}\FunctionTok{input2egv}\NormalTok{(}\AttributeTok{input=}\FunctionTok{paste0}\NormalTok{(}\StringTok{"./RasterGrids\_10m/2024/"}\NormalTok{,}
                     \StringTok{"ForestsSoil\_EutrophicOrganic\_input.tif"}\NormalTok{),}
              \AttributeTok{egv\_template=} \StringTok{"./Templates/TemplateRasters/LV100m\_10km.tif"}\NormalTok{,}
              \AttributeTok{summary\_function =} \StringTok{"average"}\NormalTok{,}
              \AttributeTok{missing\_job =} \StringTok{"FillOutput"}\NormalTok{,}
              \AttributeTok{outlocation =} \StringTok{"./RasterGrids\_100m/2024/RAW/"}\NormalTok{,}
              \AttributeTok{outfilename =} \StringTok{"ForestsSoil\_EutrophicOrganic\_cell.tif"}\NormalTok{,}
              \AttributeTok{layername =} \StringTok{"egv\_318"}\NormalTok{,}
              \AttributeTok{idw\_weight =} \DecValTok{2}\NormalTok{,}
              \AttributeTok{plot\_gaps =} \ConstantTok{FALSE}\NormalTok{,}\AttributeTok{plot\_final =} \ConstantTok{TRUE}\NormalTok{)}
\NormalTok{i2e\_rez}
\FunctionTok{rm}\NormalTok{(EutrophicOrganic)}
\FunctionTok{rm}\NormalTok{(p2i\_rez)}
\FunctionTok{rm}\NormalTok{(i2e\_rez)}
\FunctionTok{unlink}\NormalTok{(}\StringTok{"./RasterGrids\_10m/2024/ForestsSoil\_EutrophicOrganic\_input.tif"}\NormalTok{)}

\CommentTok{\# standardisation {-}{-}{-}{-}}
\ControlFlowTok{if}\NormalTok{(}\SpecialCharTok{!}\FunctionTok{require}\NormalTok{(terra)) \{}\FunctionTok{install.packages}\NormalTok{(}\StringTok{"terra"}\NormalTok{); }\FunctionTok{require}\NormalTok{(terra)\}}
\ControlFlowTok{if}\NormalTok{(}\SpecialCharTok{!}\FunctionTok{require}\NormalTok{(tidyverse)) \{}\FunctionTok{install.packages}\NormalTok{(}\StringTok{"tidyverse"}\NormalTok{); }\FunctionTok{require}\NormalTok{(tidyverse)\}}

\NormalTok{nosaukums}\OtherTok{=}\StringTok{"ForestsSoil\_EutrophicOrganic\_cell.tif"}
\NormalTok{ielasisanas\_cels}\OtherTok{=}\FunctionTok{paste0}\NormalTok{(}\StringTok{"./RasterGrids\_100m/2024/RAW/"}\NormalTok{,nosaukums)}
\NormalTok{saglabasanas\_cels}\OtherTok{=}\FunctionTok{paste0}\NormalTok{(}\StringTok{"./RasterGrids\_100m/2024/Scaled/"}\NormalTok{,nosaukums)}
\NormalTok{slanis}\OtherTok{=}\FunctionTok{rast}\NormalTok{(ielasisanas\_cels)}
\NormalTok{videjais}\OtherTok{=}\FunctionTok{global}\NormalTok{(slanis,}\AttributeTok{fun=}\StringTok{"mean"}\NormalTok{,}\AttributeTok{na.rm=}\ConstantTok{TRUE}\NormalTok{)}
\NormalTok{centrets}\OtherTok{=}\NormalTok{slanis}\SpecialCharTok{{-}}\NormalTok{videjais[,}\DecValTok{1}\NormalTok{]}
\NormalTok{standartnovirze}\OtherTok{=}\NormalTok{terra}\SpecialCharTok{::}\FunctionTok{global}\NormalTok{(centrets,}\AttributeTok{fun=}\StringTok{"rms"}\NormalTok{,}\AttributeTok{na.rm=}\ConstantTok{TRUE}\NormalTok{)}
\NormalTok{merogots}\OtherTok{=}\NormalTok{centrets}\SpecialCharTok{/}\NormalTok{standartnovirze[,}\DecValTok{1}\NormalTok{]}
\FunctionTok{writeRaster}\NormalTok{(merogots,}
      \AttributeTok{filename=}\NormalTok{saglabasanas\_cels,}
      \AttributeTok{overwrite=}\ConstantTok{TRUE}\NormalTok{)}
\end{Highlighting}
\end{Shaded}

\section{ForestsSoil\_EutrophicOrganic\_r500}\label{ch06.319}

\textbf{filename:} \texttt{ForestsSoil\_EutrophicOrganic\_r500.tif}

\textbf{layername:} \texttt{egv\_319}

\textbf{English name:} Fractional cover of Eutrophic Forests on undrained Organic
Soils within the 0.5 km landscape

\textbf{Latvian name:} Eitrofu mežu nesusinātās organiskajās augsnēs platības
īpatsvars 0,5 km ainavā

\textbf{Procedure:} The cover fraction within a radius of 500 m around the analysis grid cell is
calculated as the area-weighted sum of the \hyperref[ch06.318]{analysis cells} inside the
buffer, using the workflow \texttt{egvtools::radius\_function()}. During the calculation of the landscape metric,
inverse distance weighted (power = 2) gap filling on the output is applied
to ensure no missing values at the edges. Then the layer is rewritten to set
its name. Finally, the layer is standardised by subtracting the arithmetic
mean and dividing by the root mean squared error.

\begin{Shaded}
\begin{Highlighting}[]
\CommentTok{\# libs {-}{-}{-}{-}}
\ControlFlowTok{if}\NormalTok{(}\SpecialCharTok{!}\FunctionTok{require}\NormalTok{(terra)) \{}\FunctionTok{install.packages}\NormalTok{(}\StringTok{"terra"}\NormalTok{); }\FunctionTok{require}\NormalTok{(terra)\}}
\ControlFlowTok{if}\NormalTok{(}\SpecialCharTok{!}\FunctionTok{require}\NormalTok{(egvtools)) \{remotes}\SpecialCharTok{::}\FunctionTok{install\_github}\NormalTok{(}\StringTok{"aavotins/egvtools"}\NormalTok{); }\FunctionTok{require}\NormalTok{(egvtools)\}}


\CommentTok{\# Templates {-}{-}{-}{-}{-}}
\NormalTok{template100}\OtherTok{=}\FunctionTok{rast}\NormalTok{(}\StringTok{"./Templates/TemplateRasters/LV100m\_10km.tif"}\NormalTok{)}

\CommentTok{\# radii {-}{-}{-}{-}}
\FunctionTok{radius\_function}\NormalTok{(}
 \AttributeTok{kvadrati\_path =} \StringTok{"./Templates/TemplateGrids/tiles/"}\NormalTok{,}
 \AttributeTok{radii\_path   =} \StringTok{"./Templates/TemplateGridPoints/tiles/"}\NormalTok{,}
 \AttributeTok{tikls100\_path =} \StringTok{"./Templates/TemplateGrids/tikls100\_sauzeme.parquet"}\NormalTok{,}
 \AttributeTok{template\_path =} \StringTok{"./Templates/TemplateRasters/LV100m\_10km.tif"}\NormalTok{,}
 \AttributeTok{input\_layers  =} \FunctionTok{c}\NormalTok{(}\StringTok{"./RasterGrids\_100m/2024/RAW/ForestsSoil\_EutrophicOrganic\_cell.tif"}\NormalTok{),}
 \AttributeTok{layer\_prefixes =} \FunctionTok{c}\NormalTok{(}\StringTok{"ForestsSoil\_EutrophicOrganic"}\NormalTok{),}
 \AttributeTok{output\_dir   =} \StringTok{"./RasterGrids\_100m/2024/RAW/"}\NormalTok{,}
 \AttributeTok{n\_workers   =} \DecValTok{6}\NormalTok{,}
 \AttributeTok{radii     =} \FunctionTok{c}\NormalTok{(}\StringTok{"r500"}\NormalTok{),}
 \AttributeTok{radius\_mode  =} \StringTok{"sparse"}\NormalTok{,}
 \AttributeTok{extract\_fun  =} \StringTok{"mean"}\NormalTok{,}
 \AttributeTok{fill\_missing  =} \ConstantTok{TRUE}\NormalTok{,}
 \AttributeTok{IDW\_weight   =} \DecValTok{2}\NormalTok{,}
 \AttributeTok{future\_max\_size =} \DecValTok{40} \SpecialCharTok{*} \DecValTok{1024}\SpecialCharTok{\^{}}\DecValTok{3}\NormalTok{)}


\CommentTok{\# ForestsSoil\_EutrophicOrganic\_r500.tif egv\_319}
\NormalTok{slanis}\OtherTok{=}\FunctionTok{rast}\NormalTok{(}\StringTok{"./RasterGrids\_100m/2024/RAW/ForestsSoil\_EutrophicOrganic\_r500.tif"}\NormalTok{)}
\FunctionTok{names}\NormalTok{(slanis)}\OtherTok{=}\StringTok{"egv\_319"}
\NormalTok{slanis2}\OtherTok{=}\FunctionTok{project}\NormalTok{(slanis,template100)}
\FunctionTok{writeRaster}\NormalTok{(slanis2,}
      \StringTok{"./RasterGrids\_100m/2024/RAW/ForestsSoil\_EutrophicOrganic\_r500.tif"}\NormalTok{,}
      \AttributeTok{overwrite=}\ConstantTok{TRUE}\NormalTok{)}

\CommentTok{\# standardisation {-}{-}{-}{-}}
\ControlFlowTok{if}\NormalTok{(}\SpecialCharTok{!}\FunctionTok{require}\NormalTok{(terra)) \{}\FunctionTok{install.packages}\NormalTok{(}\StringTok{"terra"}\NormalTok{); }\FunctionTok{require}\NormalTok{(terra)\}}
\ControlFlowTok{if}\NormalTok{(}\SpecialCharTok{!}\FunctionTok{require}\NormalTok{(tidyverse)) \{}\FunctionTok{install.packages}\NormalTok{(}\StringTok{"tidyverse"}\NormalTok{); }\FunctionTok{require}\NormalTok{(tidyverse)\}}

\NormalTok{nosaukums}\OtherTok{=}\StringTok{"ForestsSoil\_EutrophicOrganic\_r500.tif"}
\NormalTok{ielasisanas\_cels}\OtherTok{=}\FunctionTok{paste0}\NormalTok{(}\StringTok{"./RasterGrids\_100m/2024/RAW/"}\NormalTok{,nosaukums)}
\NormalTok{saglabasanas\_cels}\OtherTok{=}\FunctionTok{paste0}\NormalTok{(}\StringTok{"./RasterGrids\_100m/2024/Scaled/"}\NormalTok{,nosaukums)}
\NormalTok{slanis}\OtherTok{=}\FunctionTok{rast}\NormalTok{(ielasisanas\_cels)}
\NormalTok{videjais}\OtherTok{=}\FunctionTok{global}\NormalTok{(slanis,}\AttributeTok{fun=}\StringTok{"mean"}\NormalTok{,}\AttributeTok{na.rm=}\ConstantTok{TRUE}\NormalTok{)}
\NormalTok{centrets}\OtherTok{=}\NormalTok{slanis}\SpecialCharTok{{-}}\NormalTok{videjais[,}\DecValTok{1}\NormalTok{]}
\NormalTok{standartnovirze}\OtherTok{=}\NormalTok{terra}\SpecialCharTok{::}\FunctionTok{global}\NormalTok{(centrets,}\AttributeTok{fun=}\StringTok{"rms"}\NormalTok{,}\AttributeTok{na.rm=}\ConstantTok{TRUE}\NormalTok{)}
\NormalTok{merogots}\OtherTok{=}\NormalTok{centrets}\SpecialCharTok{/}\NormalTok{standartnovirze[,}\DecValTok{1}\NormalTok{]}
\FunctionTok{writeRaster}\NormalTok{(merogots,}
      \AttributeTok{filename=}\NormalTok{saglabasanas\_cels,}
      \AttributeTok{overwrite=}\ConstantTok{TRUE}\NormalTok{)}
\end{Highlighting}
\end{Shaded}

\section{ForestsSoil\_EutrophicOrganic\_r1250}\label{ch06.320}

\textbf{filename:} \texttt{ForestsSoil\_EutrophicOrganic\_r1250.tif}

\textbf{layername:} \texttt{egv\_320}

\textbf{English name:} Fractional cover of Eutrophic Forests on undrained Organic
Soils within the 1.25 km landscape

\textbf{Latvian name:} Eitrofu mežu nesusinātās organiskajās augsnēs platības
īpatsvars 1,25 km ainavā

\textbf{Procedure:} The cover fraction within a radius of 1250 m around the analysis grid cell
is calculated as the area-weighted sum of the \hyperref[ch06.318]{analysis cells} inside
the buffer, using the workflow \texttt{egvtools::radius\_function()}. During the calculation of the landscape
metric, inverse distance weighted (power = 2) gap filling on the output is
applied to ensure no missing values at the edges. Then the layer is
rewritten to set its name. Finally, the layer is standardised by
subtracting the arithmetic mean and dividing by the root mean squared error.

\begin{Shaded}
\begin{Highlighting}[]
\CommentTok{\# libs {-}{-}{-}{-}}
\ControlFlowTok{if}\NormalTok{(}\SpecialCharTok{!}\FunctionTok{require}\NormalTok{(terra)) \{}\FunctionTok{install.packages}\NormalTok{(}\StringTok{"terra"}\NormalTok{); }\FunctionTok{require}\NormalTok{(terra)\}}
\ControlFlowTok{if}\NormalTok{(}\SpecialCharTok{!}\FunctionTok{require}\NormalTok{(egvtools)) \{remotes}\SpecialCharTok{::}\FunctionTok{install\_github}\NormalTok{(}\StringTok{"aavotins/egvtools"}\NormalTok{); }\FunctionTok{require}\NormalTok{(egvtools)\}}


\CommentTok{\# Templates {-}{-}{-}{-}{-}}
\NormalTok{template100}\OtherTok{=}\FunctionTok{rast}\NormalTok{(}\StringTok{"./Templates/TemplateRasters/LV100m\_10km.tif"}\NormalTok{)}

\CommentTok{\# radii {-}{-}{-}{-}}
\FunctionTok{radius\_function}\NormalTok{(}
 \AttributeTok{kvadrati\_path =} \StringTok{"./Templates/TemplateGrids/tiles/"}\NormalTok{,}
 \AttributeTok{radii\_path   =} \StringTok{"./Templates/TemplateGridPoints/tiles/"}\NormalTok{,}
 \AttributeTok{tikls100\_path =} \StringTok{"./Templates/TemplateGrids/tikls100\_sauzeme.parquet"}\NormalTok{,}
 \AttributeTok{template\_path =} \StringTok{"./Templates/TemplateRasters/LV100m\_10km.tif"}\NormalTok{,}
 \AttributeTok{input\_layers  =} \FunctionTok{c}\NormalTok{(}\StringTok{"./RasterGrids\_100m/2024/RAW/ForestsSoil\_EutrophicOrganic\_cell.tif"}\NormalTok{),}
 \AttributeTok{layer\_prefixes =} \FunctionTok{c}\NormalTok{(}\StringTok{"ForestsSoil\_EutrophicOrganic"}\NormalTok{),}
 \AttributeTok{output\_dir   =} \StringTok{"./RasterGrids\_100m/2024/RAW/"}\NormalTok{,}
 \AttributeTok{n\_workers   =} \DecValTok{6}\NormalTok{,}
 \AttributeTok{radii     =} \FunctionTok{c}\NormalTok{(}\StringTok{"r1250"}\NormalTok{),}
 \AttributeTok{radius\_mode  =} \StringTok{"sparse"}\NormalTok{,}
 \AttributeTok{extract\_fun  =} \StringTok{"mean"}\NormalTok{,}
 \AttributeTok{fill\_missing  =} \ConstantTok{TRUE}\NormalTok{,}
 \AttributeTok{IDW\_weight   =} \DecValTok{2}\NormalTok{,}
 \AttributeTok{future\_max\_size =} \DecValTok{40} \SpecialCharTok{*} \DecValTok{1024}\SpecialCharTok{\^{}}\DecValTok{3}\NormalTok{)}


\CommentTok{\# ForestsSoil\_EutrophicOrganic\_r1250.tif    egv\_320}
\NormalTok{slanis}\OtherTok{=}\FunctionTok{rast}\NormalTok{(}\StringTok{"./RasterGrids\_100m/2024/RAW/ForestsSoil\_EutrophicOrganic\_r1250.tif"}\NormalTok{)}
\FunctionTok{names}\NormalTok{(slanis)}\OtherTok{=}\StringTok{"egv\_320"}
\NormalTok{slanis2}\OtherTok{=}\FunctionTok{project}\NormalTok{(slanis,template100)}
\FunctionTok{writeRaster}\NormalTok{(slanis2,}
      \StringTok{"./RasterGrids\_100m/2024/RAW/ForestsSoil\_EutrophicOrganic\_r1250.tif"}\NormalTok{,}
      \AttributeTok{overwrite=}\ConstantTok{TRUE}\NormalTok{)}

\CommentTok{\# standardisation {-}{-}{-}{-}}
\ControlFlowTok{if}\NormalTok{(}\SpecialCharTok{!}\FunctionTok{require}\NormalTok{(terra)) \{}\FunctionTok{install.packages}\NormalTok{(}\StringTok{"terra"}\NormalTok{); }\FunctionTok{require}\NormalTok{(terra)\}}
\ControlFlowTok{if}\NormalTok{(}\SpecialCharTok{!}\FunctionTok{require}\NormalTok{(tidyverse)) \{}\FunctionTok{install.packages}\NormalTok{(}\StringTok{"tidyverse"}\NormalTok{); }\FunctionTok{require}\NormalTok{(tidyverse)\}}

\NormalTok{nosaukums}\OtherTok{=}\StringTok{"ForestsSoil\_EutrophicOrganic\_r1250.tif"}
\NormalTok{ielasisanas\_cels}\OtherTok{=}\FunctionTok{paste0}\NormalTok{(}\StringTok{"./RasterGrids\_100m/2024/RAW/"}\NormalTok{,nosaukums)}
\NormalTok{saglabasanas\_cels}\OtherTok{=}\FunctionTok{paste0}\NormalTok{(}\StringTok{"./RasterGrids\_100m/2024/Scaled/"}\NormalTok{,nosaukums)}
\NormalTok{slanis}\OtherTok{=}\FunctionTok{rast}\NormalTok{(ielasisanas\_cels)}
\NormalTok{videjais}\OtherTok{=}\FunctionTok{global}\NormalTok{(slanis,}\AttributeTok{fun=}\StringTok{"mean"}\NormalTok{,}\AttributeTok{na.rm=}\ConstantTok{TRUE}\NormalTok{)}
\NormalTok{centrets}\OtherTok{=}\NormalTok{slanis}\SpecialCharTok{{-}}\NormalTok{videjais[,}\DecValTok{1}\NormalTok{]}
\NormalTok{standartnovirze}\OtherTok{=}\NormalTok{terra}\SpecialCharTok{::}\FunctionTok{global}\NormalTok{(centrets,}\AttributeTok{fun=}\StringTok{"rms"}\NormalTok{,}\AttributeTok{na.rm=}\ConstantTok{TRUE}\NormalTok{)}
\NormalTok{merogots}\OtherTok{=}\NormalTok{centrets}\SpecialCharTok{/}\NormalTok{standartnovirze[,}\DecValTok{1}\NormalTok{]}
\FunctionTok{writeRaster}\NormalTok{(merogots,}
      \AttributeTok{filename=}\NormalTok{saglabasanas\_cels,}
      \AttributeTok{overwrite=}\ConstantTok{TRUE}\NormalTok{)}
\end{Highlighting}
\end{Shaded}

\section{ForestsSoil\_EutrophicOrganic\_r3000}\label{ch06.321}

\textbf{filename:} \texttt{ForestsSoil\_EutrophicOrganic\_r3000.tif}

\textbf{layername:} \texttt{egv\_321}

\textbf{English name:} Fractional cover of Eutrophic Forests on undrained Organic
Soils within the 3 km landscape

\textbf{Latvian name:} Eitrofu mežu nesusinātās organiskajās augsnēs platības
īpatsvars 3 km ainavā

\textbf{Procedure:} The cover fraction within a radius of 3000 m around the analysis grid cell
is calculated as the area-weighted sum of the \hyperref[ch06.318]{analysis cells} inside
the buffer, using the workflow \texttt{egvtools::radius\_function()}. During the calculation of the landscape
metric, inverse distance weighted (power = 2) gap filling on the output is
applied to ensure no missing values at the edges. Then the layer is
rewritten to set its name. Finally, the layer is standardised by
subtracting the arithmetic mean and dividing by the root mean squared error.

\begin{Shaded}
\begin{Highlighting}[]
\CommentTok{\# libs {-}{-}{-}{-}}
\ControlFlowTok{if}\NormalTok{(}\SpecialCharTok{!}\FunctionTok{require}\NormalTok{(terra)) \{}\FunctionTok{install.packages}\NormalTok{(}\StringTok{"terra"}\NormalTok{); }\FunctionTok{require}\NormalTok{(terra)\}}
\ControlFlowTok{if}\NormalTok{(}\SpecialCharTok{!}\FunctionTok{require}\NormalTok{(egvtools)) \{remotes}\SpecialCharTok{::}\FunctionTok{install\_github}\NormalTok{(}\StringTok{"aavotins/egvtools"}\NormalTok{); }\FunctionTok{require}\NormalTok{(egvtools)\}}


\CommentTok{\# Templates {-}{-}{-}{-}{-}}
\NormalTok{template100}\OtherTok{=}\FunctionTok{rast}\NormalTok{(}\StringTok{"./Templates/TemplateRasters/LV100m\_10km.tif"}\NormalTok{)}

\CommentTok{\# radii {-}{-}{-}{-}}
\FunctionTok{radius\_function}\NormalTok{(}
 \AttributeTok{kvadrati\_path =} \StringTok{"./Templates/TemplateGrids/tiles/"}\NormalTok{,}
 \AttributeTok{radii\_path   =} \StringTok{"./Templates/TemplateGridPoints/tiles/"}\NormalTok{,}
 \AttributeTok{tikls100\_path =} \StringTok{"./Templates/TemplateGrids/tikls100\_sauzeme.parquet"}\NormalTok{,}
 \AttributeTok{template\_path =} \StringTok{"./Templates/TemplateRasters/LV100m\_10km.tif"}\NormalTok{,}
 \AttributeTok{input\_layers  =} \FunctionTok{c}\NormalTok{(}\StringTok{"./RasterGrids\_100m/2024/RAW/ForestsSoil\_EutrophicOrganic\_cell.tif"}\NormalTok{),}
 \AttributeTok{layer\_prefixes =} \FunctionTok{c}\NormalTok{(}\StringTok{"ForestsSoil\_EutrophicOrganic"}\NormalTok{),}
 \AttributeTok{output\_dir   =} \StringTok{"./RasterGrids\_100m/2024/RAW/"}\NormalTok{,}
 \AttributeTok{n\_workers   =} \DecValTok{6}\NormalTok{,}
 \AttributeTok{radii     =} \FunctionTok{c}\NormalTok{(}\StringTok{"r3000"}\NormalTok{),}
 \AttributeTok{radius\_mode  =} \StringTok{"sparse"}\NormalTok{,}
 \AttributeTok{extract\_fun  =} \StringTok{"mean"}\NormalTok{,}
 \AttributeTok{fill\_missing  =} \ConstantTok{TRUE}\NormalTok{,}
 \AttributeTok{IDW\_weight   =} \DecValTok{2}\NormalTok{,}
 \AttributeTok{future\_max\_size =} \DecValTok{40} \SpecialCharTok{*} \DecValTok{1024}\SpecialCharTok{\^{}}\DecValTok{3}\NormalTok{)}


\CommentTok{\# ForestsSoil\_EutrophicOrganic\_r3000.tif    egv\_321}
\NormalTok{slanis}\OtherTok{=}\FunctionTok{rast}\NormalTok{(}\StringTok{"./RasterGrids\_100m/2024/RAW/ForestsSoil\_EutrophicOrganic\_r3000.tif"}\NormalTok{)}
\FunctionTok{names}\NormalTok{(slanis)}\OtherTok{=}\StringTok{"egv\_321"}
\NormalTok{slanis2}\OtherTok{=}\FunctionTok{project}\NormalTok{(slanis,template100)}
\FunctionTok{writeRaster}\NormalTok{(slanis2,}
      \StringTok{"./RasterGrids\_100m/2024/RAW/ForestsSoil\_EutrophicOrganic\_r3000.tif"}\NormalTok{,}
      \AttributeTok{overwrite=}\ConstantTok{TRUE}\NormalTok{)}

\CommentTok{\# standardisation {-}{-}{-}{-}}
\ControlFlowTok{if}\NormalTok{(}\SpecialCharTok{!}\FunctionTok{require}\NormalTok{(terra)) \{}\FunctionTok{install.packages}\NormalTok{(}\StringTok{"terra"}\NormalTok{); }\FunctionTok{require}\NormalTok{(terra)\}}
\ControlFlowTok{if}\NormalTok{(}\SpecialCharTok{!}\FunctionTok{require}\NormalTok{(tidyverse)) \{}\FunctionTok{install.packages}\NormalTok{(}\StringTok{"tidyverse"}\NormalTok{); }\FunctionTok{require}\NormalTok{(tidyverse)\}}

\NormalTok{nosaukums}\OtherTok{=}\StringTok{"ForestsSoil\_EutrophicOrganic\_r3000.tif"}
\NormalTok{ielasisanas\_cels}\OtherTok{=}\FunctionTok{paste0}\NormalTok{(}\StringTok{"./RasterGrids\_100m/2024/RAW/"}\NormalTok{,nosaukums)}
\NormalTok{saglabasanas\_cels}\OtherTok{=}\FunctionTok{paste0}\NormalTok{(}\StringTok{"./RasterGrids\_100m/2024/Scaled/"}\NormalTok{,nosaukums)}
\NormalTok{slanis}\OtherTok{=}\FunctionTok{rast}\NormalTok{(ielasisanas\_cels)}
\NormalTok{videjais}\OtherTok{=}\FunctionTok{global}\NormalTok{(slanis,}\AttributeTok{fun=}\StringTok{"mean"}\NormalTok{,}\AttributeTok{na.rm=}\ConstantTok{TRUE}\NormalTok{)}
\NormalTok{centrets}\OtherTok{=}\NormalTok{slanis}\SpecialCharTok{{-}}\NormalTok{videjais[,}\DecValTok{1}\NormalTok{]}
\NormalTok{standartnovirze}\OtherTok{=}\NormalTok{terra}\SpecialCharTok{::}\FunctionTok{global}\NormalTok{(centrets,}\AttributeTok{fun=}\StringTok{"rms"}\NormalTok{,}\AttributeTok{na.rm=}\ConstantTok{TRUE}\NormalTok{)}
\NormalTok{merogots}\OtherTok{=}\NormalTok{centrets}\SpecialCharTok{/}\NormalTok{standartnovirze[,}\DecValTok{1}\NormalTok{]}
\FunctionTok{writeRaster}\NormalTok{(merogots,}
      \AttributeTok{filename=}\NormalTok{saglabasanas\_cels,}
      \AttributeTok{overwrite=}\ConstantTok{TRUE}\NormalTok{)}
\end{Highlighting}
\end{Shaded}

\section{ForestsSoil\_EutrophicOrganic\_r10000}\label{ch06.322}

\textbf{filename:} \texttt{ForestsSoil\_EutrophicOrganic\_r10000.tif}

\textbf{layername:} \texttt{egv\_322}

\textbf{English name:} Fractional cover of Eutrophic Forests on undrained Organic
Soils within the 10 km landscape

\textbf{Latvian name:} Eitrofu mežu nesusinātās organiskajās augsnēs platības
īpatsvars 10 km ainavā

\textbf{Procedure:} The cover fraction within a radius of 10000 m around the analysis grid cell
is calculated as the area-weighted sum of the \hyperref[ch06.318]{analysis cells} inside
the buffer, using the workflow \texttt{egvtools::radius\_function()}. During the calculation of the landscape
metric, inverse distance weighted (power = 2) gap filling on the output is
applied to ensure no missing values at the edges. Then the layer is
rewritten to set its name. Finally, the layer is standardised by
subtracting the arithmetic mean and dividing by the root mean squared error.

\begin{Shaded}
\begin{Highlighting}[]
\CommentTok{\# libs {-}{-}{-}{-}}
\ControlFlowTok{if}\NormalTok{(}\SpecialCharTok{!}\FunctionTok{require}\NormalTok{(terra)) \{}\FunctionTok{install.packages}\NormalTok{(}\StringTok{"terra"}\NormalTok{); }\FunctionTok{require}\NormalTok{(terra)\}}
\ControlFlowTok{if}\NormalTok{(}\SpecialCharTok{!}\FunctionTok{require}\NormalTok{(egvtools)) \{remotes}\SpecialCharTok{::}\FunctionTok{install\_github}\NormalTok{(}\StringTok{"aavotins/egvtools"}\NormalTok{); }\FunctionTok{require}\NormalTok{(egvtools)\}}


\CommentTok{\# Templates {-}{-}{-}{-}{-}}
\NormalTok{template100}\OtherTok{=}\FunctionTok{rast}\NormalTok{(}\StringTok{"./Templates/TemplateRasters/LV100m\_10km.tif"}\NormalTok{)}

\CommentTok{\# radii {-}{-}{-}{-}}
\FunctionTok{radius\_function}\NormalTok{(}
 \AttributeTok{kvadrati\_path =} \StringTok{"./Templates/TemplateGrids/tiles/"}\NormalTok{,}
 \AttributeTok{radii\_path   =} \StringTok{"./Templates/TemplateGridPoints/tiles/"}\NormalTok{,}
 \AttributeTok{tikls100\_path =} \StringTok{"./Templates/TemplateGrids/tikls100\_sauzeme.parquet"}\NormalTok{,}
 \AttributeTok{template\_path =} \StringTok{"./Templates/TemplateRasters/LV100m\_10km.tif"}\NormalTok{,}
 \AttributeTok{input\_layers  =} \FunctionTok{c}\NormalTok{(}\StringTok{"./RasterGrids\_100m/2024/RAW/ForestsSoil\_EutrophicOrganic\_cell.tif"}\NormalTok{),}
 \AttributeTok{layer\_prefixes =} \FunctionTok{c}\NormalTok{(}\StringTok{"ForestsSoil\_EutrophicOrganic"}\NormalTok{),}
 \AttributeTok{output\_dir   =} \StringTok{"./RasterGrids\_100m/2024/RAW/"}\NormalTok{,}
 \AttributeTok{n\_workers   =} \DecValTok{6}\NormalTok{,}
 \AttributeTok{radii     =} \FunctionTok{c}\NormalTok{(}\StringTok{"r10000"}\NormalTok{),}
 \AttributeTok{radius\_mode  =} \StringTok{"sparse"}\NormalTok{,}
 \AttributeTok{extract\_fun  =} \StringTok{"mean"}\NormalTok{,}
 \AttributeTok{fill\_missing  =} \ConstantTok{TRUE}\NormalTok{,}
 \AttributeTok{IDW\_weight   =} \DecValTok{2}\NormalTok{,}
 \AttributeTok{future\_max\_size =} \DecValTok{40} \SpecialCharTok{*} \DecValTok{1024}\SpecialCharTok{\^{}}\DecValTok{3}\NormalTok{)}


\CommentTok{\# ForestsSoil\_EutrophicOrganic\_r10000.tif   egv\_322}
\NormalTok{slanis}\OtherTok{=}\FunctionTok{rast}\NormalTok{(}\StringTok{"./RasterGrids\_100m/2024/RAW/ForestsSoil\_EutrophicOrganic\_r10000.tif"}\NormalTok{)}
\FunctionTok{names}\NormalTok{(slanis)}\OtherTok{=}\StringTok{"egv\_322"}
\NormalTok{slanis2}\OtherTok{=}\FunctionTok{project}\NormalTok{(slanis,template100)}
\FunctionTok{writeRaster}\NormalTok{(slanis2,}
      \StringTok{"./RasterGrids\_100m/2024/RAW/ForestsSoil\_EutrophicOrganic\_r10000.tif"}\NormalTok{,}
      \AttributeTok{overwrite=}\ConstantTok{TRUE}\NormalTok{)}

\CommentTok{\# standardisation {-}{-}{-}{-}}
\ControlFlowTok{if}\NormalTok{(}\SpecialCharTok{!}\FunctionTok{require}\NormalTok{(terra)) \{}\FunctionTok{install.packages}\NormalTok{(}\StringTok{"terra"}\NormalTok{); }\FunctionTok{require}\NormalTok{(terra)\}}
\ControlFlowTok{if}\NormalTok{(}\SpecialCharTok{!}\FunctionTok{require}\NormalTok{(tidyverse)) \{}\FunctionTok{install.packages}\NormalTok{(}\StringTok{"tidyverse"}\NormalTok{); }\FunctionTok{require}\NormalTok{(tidyverse)\}}

\NormalTok{nosaukums}\OtherTok{=}\StringTok{"ForestsSoil\_EutrophicOrganic\_r10000.tif"}
\NormalTok{ielasisanas\_cels}\OtherTok{=}\FunctionTok{paste0}\NormalTok{(}\StringTok{"./RasterGrids\_100m/2024/RAW/"}\NormalTok{,nosaukums)}
\NormalTok{saglabasanas\_cels}\OtherTok{=}\FunctionTok{paste0}\NormalTok{(}\StringTok{"./RasterGrids\_100m/2024/Scaled/"}\NormalTok{,nosaukums)}
\NormalTok{slanis}\OtherTok{=}\FunctionTok{rast}\NormalTok{(ielasisanas\_cels)}
\NormalTok{videjais}\OtherTok{=}\FunctionTok{global}\NormalTok{(slanis,}\AttributeTok{fun=}\StringTok{"mean"}\NormalTok{,}\AttributeTok{na.rm=}\ConstantTok{TRUE}\NormalTok{)}
\NormalTok{centrets}\OtherTok{=}\NormalTok{slanis}\SpecialCharTok{{-}}\NormalTok{videjais[,}\DecValTok{1}\NormalTok{]}
\NormalTok{standartnovirze}\OtherTok{=}\NormalTok{terra}\SpecialCharTok{::}\FunctionTok{global}\NormalTok{(centrets,}\AttributeTok{fun=}\StringTok{"rms"}\NormalTok{,}\AttributeTok{na.rm=}\ConstantTok{TRUE}\NormalTok{)}
\NormalTok{merogots}\OtherTok{=}\NormalTok{centrets}\SpecialCharTok{/}\NormalTok{standartnovirze[,}\DecValTok{1}\NormalTok{]}
\FunctionTok{writeRaster}\NormalTok{(merogots,}
      \AttributeTok{filename=}\NormalTok{saglabasanas\_cels,}
      \AttributeTok{overwrite=}\ConstantTok{TRUE}\NormalTok{)}
\end{Highlighting}
\end{Shaded}

\section{ForestsSoil\_MesotrophicMineral\_cell}\label{ch06.323}

\textbf{filename:} \texttt{ForestsSoil\_MesotrophicMineral\_cell.tif}

\textbf{layername:} \texttt{egv\_323}

\textbf{English name:} Fractional cover of Mesotrophic Forests on undrained Mineral
Soils within the analysis cell (1 ha)

\textbf{Latvian name:} Mezotrofu mežu nesusinātās minerālaugsnēs platības īpatsvars analīzes šūnā
(1 ha)

\textbf{Procedure:} To prepare this EGV, forest stands with forest type equal to ``4''
or ``9'' are selected from the \hyperref[Ch04.01]{State Forest Service's State Forest
Registry} and rasterised. Rasterisation is performed using the
workflow \texttt{egvtools::polygon2input()} with background
covering (value 0). The resulting layer
is then aggregated to EGV resolution using the workflow \texttt{egvtools::input2egv()}, which
calculates the arithmetic mean to determine the cover fraction. During
aggregation, inverse distance weighted (power = 2) gap filling on the output is
applied to ensure no missing values at the edges. Finally, the layer is
standardised by subtracting the arithmetic mean and dividing by the root mean squared
error.

\begin{Shaded}
\begin{Highlighting}[]
\CommentTok{\# libs {-}{-}{-}{-}}
\ControlFlowTok{if}\NormalTok{(}\SpecialCharTok{!}\FunctionTok{require}\NormalTok{(egvtools)) \{remotes}\SpecialCharTok{::}\FunctionTok{install\_github}\NormalTok{(}\StringTok{"aavotins/egvtools"}\NormalTok{); }\FunctionTok{require}\NormalTok{(egvtools)\}}
\ControlFlowTok{if}\NormalTok{(}\SpecialCharTok{!}\FunctionTok{require}\NormalTok{(terra)) \{}\FunctionTok{install.packages}\NormalTok{(}\StringTok{"terra"}\NormalTok{); }\FunctionTok{require}\NormalTok{(terra)\}}
\ControlFlowTok{if}\NormalTok{(}\SpecialCharTok{!}\FunctionTok{require}\NormalTok{(sf)) \{}\FunctionTok{install.packages}\NormalTok{(}\StringTok{"sf"}\NormalTok{); }\FunctionTok{require}\NormalTok{(sf)\}}
\ControlFlowTok{if}\NormalTok{(}\SpecialCharTok{!}\FunctionTok{require}\NormalTok{(tidyverse)) \{}\FunctionTok{install.packages}\NormalTok{(}\StringTok{"tidyverse"}\NormalTok{); }\FunctionTok{require}\NormalTok{(tidyverse)\}}
\ControlFlowTok{if}\NormalTok{(}\SpecialCharTok{!}\FunctionTok{require}\NormalTok{(sfarrow)) \{}\FunctionTok{install.packages}\NormalTok{(}\StringTok{"sfarrow"}\NormalTok{); }\FunctionTok{require}\NormalTok{(sfarrow)\}}
\ControlFlowTok{if}\NormalTok{(}\SpecialCharTok{!}\FunctionTok{require}\NormalTok{(readxl)) \{}\FunctionTok{install.packages}\NormalTok{(}\StringTok{"readxl"}\NormalTok{); }\FunctionTok{require}\NormalTok{(readxl)\}}
\ControlFlowTok{if}\NormalTok{(}\SpecialCharTok{!}\FunctionTok{require}\NormalTok{(raster)) \{}\FunctionTok{install.packages}\NormalTok{(}\StringTok{"raster"}\NormalTok{); }\FunctionTok{require}\NormalTok{(raster)\}}
\ControlFlowTok{if}\NormalTok{(}\SpecialCharTok{!}\FunctionTok{require}\NormalTok{(fasterize)) \{}\FunctionTok{install.packages}\NormalTok{(}\StringTok{"fasterize"}\NormalTok{); }\FunctionTok{require}\NormalTok{(fasterize)\}}

\CommentTok{\# templates {-}{-}{-}{-}}
\NormalTok{template100}\OtherTok{=}\FunctionTok{rast}\NormalTok{(}\StringTok{"./Templates/TemplateRasters/LV100m\_10km.tif"}\NormalTok{)}
\NormalTok{template10}\OtherTok{=}\FunctionTok{rast}\NormalTok{(}\StringTok{"./Templates/TemplateRasters/LV10m\_10km.tif"}\NormalTok{)}
\NormalTok{rastrs10}\OtherTok{=}\FunctionTok{raster}\NormalTok{(template10)}

\NormalTok{nulls10}\OtherTok{=}\FunctionTok{rast}\NormalTok{(}\StringTok{"./Templates/TemplateRasters/nulls\_LV10m\_10km.tif"}\NormalTok{)}
\NormalTok{nulls100}\OtherTok{=}\FunctionTok{rast}\NormalTok{(}\StringTok{"./Templates/TemplateRasters/nulls\_LV100m\_10km.tif"}\NormalTok{)}


\CommentTok{\# simple landscape {-}{-}{-}{-}}
\NormalTok{simple\_landscape}\OtherTok{=}\FunctionTok{rast}\NormalTok{(}\StringTok{"RasterGrids\_10m/2024/Ainava\_vienk\_mask.tif"}\NormalTok{)}

\CommentTok{\# mvr {-}{-}{-}{-}}
\NormalTok{mvr}\OtherTok{=}\FunctionTok{st\_read\_parquet}\NormalTok{(}\StringTok{"./Geodata/2024/MVR/nogabali\_2024janv.parquet"}\NormalTok{)}
\NormalTok{mvr}\SpecialCharTok{$}\NormalTok{yes}\OtherTok{=}\DecValTok{1}


\CommentTok{\# ForestsSoil\_MesotrophicMineral\_cell.tif   egv\_323 {-}{-}{-}{-}}
\NormalTok{MesotrophicMineral}\OtherTok{=}\NormalTok{mvr }\SpecialCharTok{\%\textgreater{}\%} 
 \FunctionTok{filter}\NormalTok{(mt }\SpecialCharTok{\%in\%} \FunctionTok{c}\NormalTok{(}\StringTok{"4"}\NormalTok{,}\StringTok{"9"}\NormalTok{))}
\NormalTok{p2i\_rez}\OtherTok{=}\NormalTok{egvtools}\SpecialCharTok{::}\FunctionTok{polygon2input}\NormalTok{(}\AttributeTok{vector\_data =}\NormalTok{ MesotrophicMineral,}
                \AttributeTok{template\_path =} \StringTok{"./Templates/TemplateRasters/LV10m\_10km.tif"}\NormalTok{,}
                \AttributeTok{out\_path =} \StringTok{"./RasterGrids\_10m/2024/"}\NormalTok{,}
                \AttributeTok{file\_name =} \StringTok{"ForestsSoil\_MesotrophicMineral\_input.tif"}\NormalTok{,}
                \AttributeTok{value\_field =} \StringTok{"yes"}\NormalTok{,}
                \AttributeTok{prepare=}\ConstantTok{FALSE}\NormalTok{,}
                \AttributeTok{background\_raster =} \StringTok{"./Templates/TemplateRasters/nulls\_LV10m\_10km.tif"}\NormalTok{,}
                \AttributeTok{plot\_result =} \ConstantTok{TRUE}\NormalTok{)}
\NormalTok{p2i\_rez}
\NormalTok{i2e\_rez}\OtherTok{=}\NormalTok{egvtools}\SpecialCharTok{::}\FunctionTok{input2egv}\NormalTok{(}\AttributeTok{input=}\FunctionTok{paste0}\NormalTok{(}\StringTok{"./RasterGrids\_10m/2024/"}\NormalTok{,}
                     \StringTok{"ForestsSoil\_MesotrophicMineral\_input.tif"}\NormalTok{),}
              \AttributeTok{egv\_template=} \StringTok{"./Templates/TemplateRasters/LV100m\_10km.tif"}\NormalTok{,}
              \AttributeTok{summary\_function =} \StringTok{"average"}\NormalTok{,}
              \AttributeTok{missing\_job =} \StringTok{"FillOutput"}\NormalTok{,}
              \AttributeTok{outlocation =} \StringTok{"./RasterGrids\_100m/2024/RAW/"}\NormalTok{,}
              \AttributeTok{outfilename =} \StringTok{"ForestsSoil\_MesotrophicMineral\_cell.tif"}\NormalTok{,}
              \AttributeTok{layername =} \StringTok{"egv\_323"}\NormalTok{,}
              \AttributeTok{idw\_weight =} \DecValTok{2}\NormalTok{,}
              \AttributeTok{plot\_gaps =} \ConstantTok{FALSE}\NormalTok{,}\AttributeTok{plot\_final =} \ConstantTok{TRUE}\NormalTok{)}
\NormalTok{i2e\_rez}
\FunctionTok{rm}\NormalTok{(MesotrophicMineral)}
\FunctionTok{rm}\NormalTok{(p2i\_rez)}
\FunctionTok{rm}\NormalTok{(i2e\_rez)}
\FunctionTok{unlink}\NormalTok{(}\StringTok{"./RasterGrids\_10m/2024/ForestsSoil\_MesotrophicMineral\_input.tif"}\NormalTok{)}

\CommentTok{\# standardisation {-}{-}{-}{-}}
\ControlFlowTok{if}\NormalTok{(}\SpecialCharTok{!}\FunctionTok{require}\NormalTok{(terra)) \{}\FunctionTok{install.packages}\NormalTok{(}\StringTok{"terra"}\NormalTok{); }\FunctionTok{require}\NormalTok{(terra)\}}
\ControlFlowTok{if}\NormalTok{(}\SpecialCharTok{!}\FunctionTok{require}\NormalTok{(tidyverse)) \{}\FunctionTok{install.packages}\NormalTok{(}\StringTok{"tidyverse"}\NormalTok{); }\FunctionTok{require}\NormalTok{(tidyverse)\}}

\NormalTok{nosaukums}\OtherTok{=}\StringTok{"ForestsSoil\_MesotrophicMineral\_cell.tif"}
\NormalTok{ielasisanas\_cels}\OtherTok{=}\FunctionTok{paste0}\NormalTok{(}\StringTok{"./RasterGrids\_100m/2024/RAW/"}\NormalTok{,nosaukums)}
\NormalTok{saglabasanas\_cels}\OtherTok{=}\FunctionTok{paste0}\NormalTok{(}\StringTok{"./RasterGrids\_100m/2024/Scaled/"}\NormalTok{,nosaukums)}
\NormalTok{slanis}\OtherTok{=}\FunctionTok{rast}\NormalTok{(ielasisanas\_cels)}
\NormalTok{videjais}\OtherTok{=}\FunctionTok{global}\NormalTok{(slanis,}\AttributeTok{fun=}\StringTok{"mean"}\NormalTok{,}\AttributeTok{na.rm=}\ConstantTok{TRUE}\NormalTok{)}
\NormalTok{centrets}\OtherTok{=}\NormalTok{slanis}\SpecialCharTok{{-}}\NormalTok{videjais[,}\DecValTok{1}\NormalTok{]}
\NormalTok{standartnovirze}\OtherTok{=}\NormalTok{terra}\SpecialCharTok{::}\FunctionTok{global}\NormalTok{(centrets,}\AttributeTok{fun=}\StringTok{"rms"}\NormalTok{,}\AttributeTok{na.rm=}\ConstantTok{TRUE}\NormalTok{)}
\NormalTok{merogots}\OtherTok{=}\NormalTok{centrets}\SpecialCharTok{/}\NormalTok{standartnovirze[,}\DecValTok{1}\NormalTok{]}
\FunctionTok{writeRaster}\NormalTok{(merogots,}
      \AttributeTok{filename=}\NormalTok{saglabasanas\_cels,}
      \AttributeTok{overwrite=}\ConstantTok{TRUE}\NormalTok{)}
\end{Highlighting}
\end{Shaded}

\section{ForestsSoil\_MesotrophicMineral\_r500}\label{ch06.324}

\textbf{filename:} \texttt{ForestsSoil\_MesotrophicMineral\_r500.tif}

\textbf{layername:} \texttt{egv\_324}

\textbf{English name:} Fractional cover of Mesotrophic Forests on undrained Mineral
Soils within the 0.5 km landscape

\textbf{Latvian name:} Mezotrofu mežu nesusinātās minerālaugsnēs platības īpatsvars
0,5 km ainavā

\textbf{Procedure:} The cover fraction within a radius of 500 m around the analysis grid cell is
calculated as the area-weighted sum of the \hyperref[ch06.323]{analysis cells} inside the
buffer, using the workflow \texttt{egvtools::radius\_function()}. During the calculation of the landscape metric,
inverse distance weighted (power = 2) gap filling on the output is applied
to ensure no missing values at the edges. Then the layer is rewritten to set
its name. Finally, the layer is standardised by subtracting the arithmetic
mean and dividing by the root mean squared error.

\begin{Shaded}
\begin{Highlighting}[]
\CommentTok{\# libs {-}{-}{-}{-}}
\ControlFlowTok{if}\NormalTok{(}\SpecialCharTok{!}\FunctionTok{require}\NormalTok{(terra)) \{}\FunctionTok{install.packages}\NormalTok{(}\StringTok{"terra"}\NormalTok{); }\FunctionTok{require}\NormalTok{(terra)\}}
\ControlFlowTok{if}\NormalTok{(}\SpecialCharTok{!}\FunctionTok{require}\NormalTok{(egvtools)) \{remotes}\SpecialCharTok{::}\FunctionTok{install\_github}\NormalTok{(}\StringTok{"aavotins/egvtools"}\NormalTok{); }\FunctionTok{require}\NormalTok{(egvtools)\}}


\CommentTok{\# Templates {-}{-}{-}{-}{-}}
\NormalTok{template100}\OtherTok{=}\FunctionTok{rast}\NormalTok{(}\StringTok{"./Templates/TemplateRasters/LV100m\_10km.tif"}\NormalTok{)}

\CommentTok{\# radii {-}{-}{-}{-}}
\FunctionTok{radius\_function}\NormalTok{(}
 \AttributeTok{kvadrati\_path =} \StringTok{"./Templates/TemplateGrids/tiles/"}\NormalTok{,}
 \AttributeTok{radii\_path   =} \StringTok{"./Templates/TemplateGridPoints/tiles/"}\NormalTok{,}
 \AttributeTok{tikls100\_path =} \StringTok{"./Templates/TemplateGrids/tikls100\_sauzeme.parquet"}\NormalTok{,}
 \AttributeTok{template\_path =} \StringTok{"./Templates/TemplateRasters/LV100m\_10km.tif"}\NormalTok{,}
 \AttributeTok{input\_layers  =} \FunctionTok{c}\NormalTok{(}\StringTok{"./RasterGrids\_100m/2024/RAW/ForestsSoil\_MesotrophicMineral\_cell.tif"}\NormalTok{),}
 \AttributeTok{layer\_prefixes =} \FunctionTok{c}\NormalTok{(}\StringTok{"ForestsSoil\_MesotrophicMineral"}\NormalTok{),}
 \AttributeTok{output\_dir   =} \StringTok{"./RasterGrids\_100m/2024/RAW/"}\NormalTok{,}
 \AttributeTok{n\_workers   =} \DecValTok{6}\NormalTok{,}
 \AttributeTok{radii     =} \FunctionTok{c}\NormalTok{(}\StringTok{"r500"}\NormalTok{),}
 \AttributeTok{radius\_mode  =} \StringTok{"sparse"}\NormalTok{,}
 \AttributeTok{extract\_fun  =} \StringTok{"mean"}\NormalTok{,}
 \AttributeTok{fill\_missing  =} \ConstantTok{TRUE}\NormalTok{,}
 \AttributeTok{IDW\_weight   =} \DecValTok{2}\NormalTok{,}
 \AttributeTok{future\_max\_size =} \DecValTok{40} \SpecialCharTok{*} \DecValTok{1024}\SpecialCharTok{\^{}}\DecValTok{3}\NormalTok{)}


\CommentTok{\# ForestsSoil\_MesotrophicMineral\_r500.tif   egv\_324}
\NormalTok{slanis}\OtherTok{=}\FunctionTok{rast}\NormalTok{(}\StringTok{"./RasterGrids\_100m/2024/RAW/ForestsSoil\_MesotrophicMineral\_r500.tif"}\NormalTok{)}
\FunctionTok{names}\NormalTok{(slanis)}\OtherTok{=}\StringTok{"egv\_324"}
\NormalTok{slanis2}\OtherTok{=}\FunctionTok{project}\NormalTok{(slanis,template100)}
\FunctionTok{writeRaster}\NormalTok{(slanis2,}
      \StringTok{"./RasterGrids\_100m/2024/RAW/ForestsSoil\_MesotrophicMineral\_r500.tif"}\NormalTok{,}
      \AttributeTok{overwrite=}\ConstantTok{TRUE}\NormalTok{)}

\CommentTok{\# standardisation {-}{-}{-}{-}}
\ControlFlowTok{if}\NormalTok{(}\SpecialCharTok{!}\FunctionTok{require}\NormalTok{(terra)) \{}\FunctionTok{install.packages}\NormalTok{(}\StringTok{"terra"}\NormalTok{); }\FunctionTok{require}\NormalTok{(terra)\}}
\ControlFlowTok{if}\NormalTok{(}\SpecialCharTok{!}\FunctionTok{require}\NormalTok{(tidyverse)) \{}\FunctionTok{install.packages}\NormalTok{(}\StringTok{"tidyverse"}\NormalTok{); }\FunctionTok{require}\NormalTok{(tidyverse)\}}

\NormalTok{nosaukums}\OtherTok{=}\StringTok{"ForestsSoil\_MesotrophicMineral\_r500.tif"}
\NormalTok{ielasisanas\_cels}\OtherTok{=}\FunctionTok{paste0}\NormalTok{(}\StringTok{"./RasterGrids\_100m/2024/RAW/"}\NormalTok{,nosaukums)}
\NormalTok{saglabasanas\_cels}\OtherTok{=}\FunctionTok{paste0}\NormalTok{(}\StringTok{"./RasterGrids\_100m/2024/Scaled/"}\NormalTok{,nosaukums)}
\NormalTok{slanis}\OtherTok{=}\FunctionTok{rast}\NormalTok{(ielasisanas\_cels)}
\NormalTok{videjais}\OtherTok{=}\FunctionTok{global}\NormalTok{(slanis,}\AttributeTok{fun=}\StringTok{"mean"}\NormalTok{,}\AttributeTok{na.rm=}\ConstantTok{TRUE}\NormalTok{)}
\NormalTok{centrets}\OtherTok{=}\NormalTok{slanis}\SpecialCharTok{{-}}\NormalTok{videjais[,}\DecValTok{1}\NormalTok{]}
\NormalTok{standartnovirze}\OtherTok{=}\NormalTok{terra}\SpecialCharTok{::}\FunctionTok{global}\NormalTok{(centrets,}\AttributeTok{fun=}\StringTok{"rms"}\NormalTok{,}\AttributeTok{na.rm=}\ConstantTok{TRUE}\NormalTok{)}
\NormalTok{merogots}\OtherTok{=}\NormalTok{centrets}\SpecialCharTok{/}\NormalTok{standartnovirze[,}\DecValTok{1}\NormalTok{]}
\FunctionTok{writeRaster}\NormalTok{(merogots,}
      \AttributeTok{filename=}\NormalTok{saglabasanas\_cels,}
      \AttributeTok{overwrite=}\ConstantTok{TRUE}\NormalTok{)}
\end{Highlighting}
\end{Shaded}

\section{ForestsSoil\_MesotrophicMineral\_r1250}\label{ch06.325}

\textbf{filename:} \texttt{ForestsSoil\_MesotrophicMineral\_r1250.tif}

\textbf{layername:} \texttt{egv\_325}

\textbf{English name:} Fractional cover of Mesotrophic Forests on undrained Mineral
Soils within the 1.25 km landscape

\textbf{Latvian name:} Mezotrofu mežu nesusinātās minerālaugsnēs platības īpatsvars
1,25 km ainavā

\textbf{Procedure:} The cover fraction within a radius of 1250 m around the analysis grid cell
is calculated as the area-weighted sum of the \hyperref[ch06.323]{analysis cells} inside
the buffer, using the workflow \texttt{egvtools::radius\_function()}. During the calculation of the landscape
metric, inverse distance weighted (power = 2) gap filling on the output is
applied to ensure no missing values at the edges. Then the layer is
rewritten to set its name. Finally, the layer is standardised by
subtracting the arithmetic mean and dividing by the root mean squared error.

\begin{Shaded}
\begin{Highlighting}[]
\CommentTok{\# libs {-}{-}{-}{-}}
\ControlFlowTok{if}\NormalTok{(}\SpecialCharTok{!}\FunctionTok{require}\NormalTok{(terra)) \{}\FunctionTok{install.packages}\NormalTok{(}\StringTok{"terra"}\NormalTok{); }\FunctionTok{require}\NormalTok{(terra)\}}
\ControlFlowTok{if}\NormalTok{(}\SpecialCharTok{!}\FunctionTok{require}\NormalTok{(egvtools)) \{remotes}\SpecialCharTok{::}\FunctionTok{install\_github}\NormalTok{(}\StringTok{"aavotins/egvtools"}\NormalTok{); }\FunctionTok{require}\NormalTok{(egvtools)\}}


\CommentTok{\# Templates {-}{-}{-}{-}{-}}
\NormalTok{template100}\OtherTok{=}\FunctionTok{rast}\NormalTok{(}\StringTok{"./Templates/TemplateRasters/LV100m\_10km.tif"}\NormalTok{)}

\CommentTok{\# radii {-}{-}{-}{-}}
\FunctionTok{radius\_function}\NormalTok{(}
 \AttributeTok{kvadrati\_path =} \StringTok{"./Templates/TemplateGrids/tiles/"}\NormalTok{,}
 \AttributeTok{radii\_path   =} \StringTok{"./Templates/TemplateGridPoints/tiles/"}\NormalTok{,}
 \AttributeTok{tikls100\_path =} \StringTok{"./Templates/TemplateGrids/tikls100\_sauzeme.parquet"}\NormalTok{,}
 \AttributeTok{template\_path =} \StringTok{"./Templates/TemplateRasters/LV100m\_10km.tif"}\NormalTok{,}
 \AttributeTok{input\_layers  =} \FunctionTok{c}\NormalTok{(}\StringTok{"./RasterGrids\_100m/2024/RAW/ForestsSoil\_MesotrophicMineral\_cell.tif"}\NormalTok{),}
 \AttributeTok{layer\_prefixes =} \FunctionTok{c}\NormalTok{(}\StringTok{"ForestsSoil\_MesotrophicMineral"}\NormalTok{),}
 \AttributeTok{output\_dir   =} \StringTok{"./RasterGrids\_100m/2024/RAW/"}\NormalTok{,}
 \AttributeTok{n\_workers   =} \DecValTok{6}\NormalTok{,}
 \AttributeTok{radii     =} \FunctionTok{c}\NormalTok{(}\StringTok{"r1250"}\NormalTok{),}
 \AttributeTok{radius\_mode  =} \StringTok{"sparse"}\NormalTok{,}
 \AttributeTok{extract\_fun  =} \StringTok{"mean"}\NormalTok{,}
 \AttributeTok{fill\_missing  =} \ConstantTok{TRUE}\NormalTok{,}
 \AttributeTok{IDW\_weight   =} \DecValTok{2}\NormalTok{,}
 \AttributeTok{future\_max\_size =} \DecValTok{40} \SpecialCharTok{*} \DecValTok{1024}\SpecialCharTok{\^{}}\DecValTok{3}\NormalTok{)}


\CommentTok{\# ForestsSoil\_MesotrophicMineral\_r1250.tif  egv\_325}
\NormalTok{slanis}\OtherTok{=}\FunctionTok{rast}\NormalTok{(}\StringTok{"./RasterGrids\_100m/2024/RAW/ForestsSoil\_MesotrophicMineral\_r1250.tif"}\NormalTok{)}
\FunctionTok{names}\NormalTok{(slanis)}\OtherTok{=}\StringTok{"egv\_325"}
\NormalTok{slanis2}\OtherTok{=}\FunctionTok{project}\NormalTok{(slanis,template100)}
\FunctionTok{writeRaster}\NormalTok{(slanis2,}
      \StringTok{"./RasterGrids\_100m/2024/RAW/ForestsSoil\_MesotrophicMineral\_r1250.tif"}\NormalTok{,}
      \AttributeTok{overwrite=}\ConstantTok{TRUE}\NormalTok{)}

\CommentTok{\# standardisation {-}{-}{-}{-}}
\ControlFlowTok{if}\NormalTok{(}\SpecialCharTok{!}\FunctionTok{require}\NormalTok{(terra)) \{}\FunctionTok{install.packages}\NormalTok{(}\StringTok{"terra"}\NormalTok{); }\FunctionTok{require}\NormalTok{(terra)\}}
\ControlFlowTok{if}\NormalTok{(}\SpecialCharTok{!}\FunctionTok{require}\NormalTok{(tidyverse)) \{}\FunctionTok{install.packages}\NormalTok{(}\StringTok{"tidyverse"}\NormalTok{); }\FunctionTok{require}\NormalTok{(tidyverse)\}}

\NormalTok{nosaukums}\OtherTok{=}\StringTok{"ForestsSoil\_MesotrophicMineral\_r1250.tif"}
\NormalTok{ielasisanas\_cels}\OtherTok{=}\FunctionTok{paste0}\NormalTok{(}\StringTok{"./RasterGrids\_100m/2024/RAW/"}\NormalTok{,nosaukums)}
\NormalTok{saglabasanas\_cels}\OtherTok{=}\FunctionTok{paste0}\NormalTok{(}\StringTok{"./RasterGrids\_100m/2024/Scaled/"}\NormalTok{,nosaukums)}
\NormalTok{slanis}\OtherTok{=}\FunctionTok{rast}\NormalTok{(ielasisanas\_cels)}
\NormalTok{videjais}\OtherTok{=}\FunctionTok{global}\NormalTok{(slanis,}\AttributeTok{fun=}\StringTok{"mean"}\NormalTok{,}\AttributeTok{na.rm=}\ConstantTok{TRUE}\NormalTok{)}
\NormalTok{centrets}\OtherTok{=}\NormalTok{slanis}\SpecialCharTok{{-}}\NormalTok{videjais[,}\DecValTok{1}\NormalTok{]}
\NormalTok{standartnovirze}\OtherTok{=}\NormalTok{terra}\SpecialCharTok{::}\FunctionTok{global}\NormalTok{(centrets,}\AttributeTok{fun=}\StringTok{"rms"}\NormalTok{,}\AttributeTok{na.rm=}\ConstantTok{TRUE}\NormalTok{)}
\NormalTok{merogots}\OtherTok{=}\NormalTok{centrets}\SpecialCharTok{/}\NormalTok{standartnovirze[,}\DecValTok{1}\NormalTok{]}
\FunctionTok{writeRaster}\NormalTok{(merogots,}
      \AttributeTok{filename=}\NormalTok{saglabasanas\_cels,}
      \AttributeTok{overwrite=}\ConstantTok{TRUE}\NormalTok{)}
\end{Highlighting}
\end{Shaded}

\section{ForestsSoil\_MesotrophicMineral\_r3000}\label{ch06.326}

\textbf{filename:} \texttt{ForestsSoil\_MesotrophicMineral\_r3000.tif}

\textbf{layername:} \texttt{egv\_326}

\textbf{English name:} Fractional cover of Mesotrophic Forests on undrained Mineral
Soils within the 3 km landscape

\textbf{Latvian name:} Mezotrofu mežu nesusinātās minerālaugsnēs platības īpatsvars 3
km ainavā

\textbf{Procedure:} The cover fraction within a radius of 3000 m around the analysis grid cell
is calculated as the area-weighted sum of the \hyperref[ch06.323]{analysis cells} inside
the buffer, using the workflow \texttt{egvtools::radius\_function()}. During the calculation of the landscape
metric, inverse distance weighted (power = 2) gap filling on the output is
applied to ensure no missing values at the edges. Then the layer is
rewritten to set its name. Finally, the layer is standardised by
subtracting the arithmetic mean and dividing by the root mean squared error.

\begin{Shaded}
\begin{Highlighting}[]
\CommentTok{\# libs {-}{-}{-}{-}}
\ControlFlowTok{if}\NormalTok{(}\SpecialCharTok{!}\FunctionTok{require}\NormalTok{(terra)) \{}\FunctionTok{install.packages}\NormalTok{(}\StringTok{"terra"}\NormalTok{); }\FunctionTok{require}\NormalTok{(terra)\}}
\ControlFlowTok{if}\NormalTok{(}\SpecialCharTok{!}\FunctionTok{require}\NormalTok{(egvtools)) \{remotes}\SpecialCharTok{::}\FunctionTok{install\_github}\NormalTok{(}\StringTok{"aavotins/egvtools"}\NormalTok{); }\FunctionTok{require}\NormalTok{(egvtools)\}}


\CommentTok{\# Templates {-}{-}{-}{-}{-}}
\NormalTok{template100}\OtherTok{=}\FunctionTok{rast}\NormalTok{(}\StringTok{"./Templates/TemplateRasters/LV100m\_10km.tif"}\NormalTok{)}

\CommentTok{\# radii {-}{-}{-}{-}}
\FunctionTok{radius\_function}\NormalTok{(}
 \AttributeTok{kvadrati\_path =} \StringTok{"./Templates/TemplateGrids/tiles/"}\NormalTok{,}
 \AttributeTok{radii\_path   =} \StringTok{"./Templates/TemplateGridPoints/tiles/"}\NormalTok{,}
 \AttributeTok{tikls100\_path =} \StringTok{"./Templates/TemplateGrids/tikls100\_sauzeme.parquet"}\NormalTok{,}
 \AttributeTok{template\_path =} \StringTok{"./Templates/TemplateRasters/LV100m\_10km.tif"}\NormalTok{,}
 \AttributeTok{input\_layers  =} \FunctionTok{c}\NormalTok{(}\StringTok{"./RasterGrids\_100m/2024/RAW/ForestsSoil\_MesotrophicMineral\_cell.tif"}\NormalTok{),}
 \AttributeTok{layer\_prefixes =} \FunctionTok{c}\NormalTok{(}\StringTok{"ForestsSoil\_MesotrophicMineral"}\NormalTok{),}
 \AttributeTok{output\_dir   =} \StringTok{"./RasterGrids\_100m/2024/RAW/"}\NormalTok{,}
 \AttributeTok{n\_workers   =} \DecValTok{6}\NormalTok{,}
 \AttributeTok{radii     =} \FunctionTok{c}\NormalTok{(}\StringTok{"r3000"}\NormalTok{),}
 \AttributeTok{radius\_mode  =} \StringTok{"sparse"}\NormalTok{,}
 \AttributeTok{extract\_fun  =} \StringTok{"mean"}\NormalTok{,}
 \AttributeTok{fill\_missing  =} \ConstantTok{TRUE}\NormalTok{,}
 \AttributeTok{IDW\_weight   =} \DecValTok{2}\NormalTok{,}
 \AttributeTok{future\_max\_size =} \DecValTok{40} \SpecialCharTok{*} \DecValTok{1024}\SpecialCharTok{\^{}}\DecValTok{3}\NormalTok{)}


\CommentTok{\# ForestsSoil\_MesotrophicMineral\_r3000.tif  egv\_326}
\NormalTok{slanis}\OtherTok{=}\FunctionTok{rast}\NormalTok{(}\StringTok{"./RasterGrids\_100m/2024/RAW/ForestsSoil\_MesotrophicMineral\_r3000.tif"}\NormalTok{)}
\FunctionTok{names}\NormalTok{(slanis)}\OtherTok{=}\StringTok{"egv\_326"}
\NormalTok{slanis2}\OtherTok{=}\FunctionTok{project}\NormalTok{(slanis,template100)}
\FunctionTok{writeRaster}\NormalTok{(slanis2,}
      \StringTok{"./RasterGrids\_100m/2024/RAW/ForestsSoil\_MesotrophicMineral\_r3000.tif"}\NormalTok{,}
      \AttributeTok{overwrite=}\ConstantTok{TRUE}\NormalTok{)}

\CommentTok{\# standardisation {-}{-}{-}{-}}
\ControlFlowTok{if}\NormalTok{(}\SpecialCharTok{!}\FunctionTok{require}\NormalTok{(terra)) \{}\FunctionTok{install.packages}\NormalTok{(}\StringTok{"terra"}\NormalTok{); }\FunctionTok{require}\NormalTok{(terra)\}}
\ControlFlowTok{if}\NormalTok{(}\SpecialCharTok{!}\FunctionTok{require}\NormalTok{(tidyverse)) \{}\FunctionTok{install.packages}\NormalTok{(}\StringTok{"tidyverse"}\NormalTok{); }\FunctionTok{require}\NormalTok{(tidyverse)\}}

\NormalTok{nosaukums}\OtherTok{=}\StringTok{"ForestsSoil\_MesotrophicMineral\_r3000.tif"}
\NormalTok{ielasisanas\_cels}\OtherTok{=}\FunctionTok{paste0}\NormalTok{(}\StringTok{"./RasterGrids\_100m/2024/RAW/"}\NormalTok{,nosaukums)}
\NormalTok{saglabasanas\_cels}\OtherTok{=}\FunctionTok{paste0}\NormalTok{(}\StringTok{"./RasterGrids\_100m/2024/Scaled/"}\NormalTok{,nosaukums)}
\NormalTok{slanis}\OtherTok{=}\FunctionTok{rast}\NormalTok{(ielasisanas\_cels)}
\NormalTok{videjais}\OtherTok{=}\FunctionTok{global}\NormalTok{(slanis,}\AttributeTok{fun=}\StringTok{"mean"}\NormalTok{,}\AttributeTok{na.rm=}\ConstantTok{TRUE}\NormalTok{)}
\NormalTok{centrets}\OtherTok{=}\NormalTok{slanis}\SpecialCharTok{{-}}\NormalTok{videjais[,}\DecValTok{1}\NormalTok{]}
\NormalTok{standartnovirze}\OtherTok{=}\NormalTok{terra}\SpecialCharTok{::}\FunctionTok{global}\NormalTok{(centrets,}\AttributeTok{fun=}\StringTok{"rms"}\NormalTok{,}\AttributeTok{na.rm=}\ConstantTok{TRUE}\NormalTok{)}
\NormalTok{merogots}\OtherTok{=}\NormalTok{centrets}\SpecialCharTok{/}\NormalTok{standartnovirze[,}\DecValTok{1}\NormalTok{]}
\FunctionTok{writeRaster}\NormalTok{(merogots,}
      \AttributeTok{filename=}\NormalTok{saglabasanas\_cels,}
      \AttributeTok{overwrite=}\ConstantTok{TRUE}\NormalTok{)}
\end{Highlighting}
\end{Shaded}

\section{ForestsSoil\_MesotrophicMineral\_r10000}\label{ch06.327}

\textbf{filename:} \texttt{ForestsSoil\_MesotrophicMineral\_r10000.tif}

\textbf{layername:} \texttt{egv\_327}

\textbf{English name:} Fractional cover of Mesotrophic Forests on undrained Mineral
Soils within the 10 km landscape

\textbf{Latvian name:} Mezotrofu mežu nesusinātās minerālaugsnēs platības īpatsvars
10 km ainavā

\textbf{Procedure:} The cover fraction within a radius of 10000 m around the analysis grid cell
is calculated as the area-weighted sum of the \hyperref[ch06.323]{analysis cells} inside
the buffer, using the workflow \texttt{egvtools::radius\_function()}. During the calculation of the landscape
metric, inverse distance weighted (power = 2) gap filling on the output is
applied to ensure no missing values at the edges. Then the layer is
rewritten to set its name. Finally, the layer is standardised by
subtracting the arithmetic mean and dividing by the root mean squared error.

\begin{Shaded}
\begin{Highlighting}[]
\CommentTok{\# libs {-}{-}{-}{-}}
\ControlFlowTok{if}\NormalTok{(}\SpecialCharTok{!}\FunctionTok{require}\NormalTok{(terra)) \{}\FunctionTok{install.packages}\NormalTok{(}\StringTok{"terra"}\NormalTok{); }\FunctionTok{require}\NormalTok{(terra)\}}
\ControlFlowTok{if}\NormalTok{(}\SpecialCharTok{!}\FunctionTok{require}\NormalTok{(egvtools)) \{remotes}\SpecialCharTok{::}\FunctionTok{install\_github}\NormalTok{(}\StringTok{"aavotins/egvtools"}\NormalTok{); }\FunctionTok{require}\NormalTok{(egvtools)\}}


\CommentTok{\# Templates {-}{-}{-}{-}{-}}
\NormalTok{template100}\OtherTok{=}\FunctionTok{rast}\NormalTok{(}\StringTok{"./Templates/TemplateRasters/LV100m\_10km.tif"}\NormalTok{)}

\CommentTok{\# radii {-}{-}{-}{-}}
\FunctionTok{radius\_function}\NormalTok{(}
 \AttributeTok{kvadrati\_path =} \StringTok{"./Templates/TemplateGrids/tiles/"}\NormalTok{,}
 \AttributeTok{radii\_path   =} \StringTok{"./Templates/TemplateGridPoints/tiles/"}\NormalTok{,}
 \AttributeTok{tikls100\_path =} \StringTok{"./Templates/TemplateGrids/tikls100\_sauzeme.parquet"}\NormalTok{,}
 \AttributeTok{template\_path =} \StringTok{"./Templates/TemplateRasters/LV100m\_10km.tif"}\NormalTok{,}
 \AttributeTok{input\_layers  =} \FunctionTok{c}\NormalTok{(}\StringTok{"./RasterGrids\_100m/2024/RAW/ForestsSoil\_MesotrophicMineral\_cell.tif"}\NormalTok{),}
 \AttributeTok{layer\_prefixes =} \FunctionTok{c}\NormalTok{(}\StringTok{"ForestsSoil\_MesotrophicMineral"}\NormalTok{),}
 \AttributeTok{output\_dir   =} \StringTok{"./RasterGrids\_100m/2024/RAW/"}\NormalTok{,}
 \AttributeTok{n\_workers   =} \DecValTok{6}\NormalTok{,}
 \AttributeTok{radii     =} \FunctionTok{c}\NormalTok{(}\StringTok{"r10000"}\NormalTok{),}
 \AttributeTok{radius\_mode  =} \StringTok{"sparse"}\NormalTok{,}
 \AttributeTok{extract\_fun  =} \StringTok{"mean"}\NormalTok{,}
 \AttributeTok{fill\_missing  =} \ConstantTok{TRUE}\NormalTok{,}
 \AttributeTok{IDW\_weight   =} \DecValTok{2}\NormalTok{,}
 \AttributeTok{future\_max\_size =} \DecValTok{40} \SpecialCharTok{*} \DecValTok{1024}\SpecialCharTok{\^{}}\DecValTok{3}\NormalTok{)}


\CommentTok{\# ForestsSoil\_MesotrophicMineral\_r10000.tif egv\_327}
\NormalTok{slanis}\OtherTok{=}\FunctionTok{rast}\NormalTok{(}\StringTok{"./RasterGrids\_100m/2024/RAW/ForestsSoil\_MesotrophicMineral\_r10000.tif"}\NormalTok{)}
\FunctionTok{names}\NormalTok{(slanis)}\OtherTok{=}\StringTok{"egv\_327"}
\NormalTok{slanis2}\OtherTok{=}\FunctionTok{project}\NormalTok{(slanis,template100)}
\FunctionTok{writeRaster}\NormalTok{(slanis2,}
      \StringTok{"./RasterGrids\_100m/2024/RAW/ForestsSoil\_MesotrophicMineral\_r10000.tif"}\NormalTok{,}
      \AttributeTok{overwrite=}\ConstantTok{TRUE}\NormalTok{)}

\CommentTok{\# standardisation {-}{-}{-}{-}}
\ControlFlowTok{if}\NormalTok{(}\SpecialCharTok{!}\FunctionTok{require}\NormalTok{(terra)) \{}\FunctionTok{install.packages}\NormalTok{(}\StringTok{"terra"}\NormalTok{); }\FunctionTok{require}\NormalTok{(terra)\}}
\ControlFlowTok{if}\NormalTok{(}\SpecialCharTok{!}\FunctionTok{require}\NormalTok{(tidyverse)) \{}\FunctionTok{install.packages}\NormalTok{(}\StringTok{"tidyverse"}\NormalTok{); }\FunctionTok{require}\NormalTok{(tidyverse)\}}

\NormalTok{nosaukums}\OtherTok{=}\StringTok{"ForestsSoil\_MesotrophicMineral\_r10000.tif"}
\NormalTok{ielasisanas\_cels}\OtherTok{=}\FunctionTok{paste0}\NormalTok{(}\StringTok{"./RasterGrids\_100m/2024/RAW/"}\NormalTok{,nosaukums)}
\NormalTok{saglabasanas\_cels}\OtherTok{=}\FunctionTok{paste0}\NormalTok{(}\StringTok{"./RasterGrids\_100m/2024/Scaled/"}\NormalTok{,nosaukums)}
\NormalTok{slanis}\OtherTok{=}\FunctionTok{rast}\NormalTok{(ielasisanas\_cels)}
\NormalTok{videjais}\OtherTok{=}\FunctionTok{global}\NormalTok{(slanis,}\AttributeTok{fun=}\StringTok{"mean"}\NormalTok{,}\AttributeTok{na.rm=}\ConstantTok{TRUE}\NormalTok{)}
\NormalTok{centrets}\OtherTok{=}\NormalTok{slanis}\SpecialCharTok{{-}}\NormalTok{videjais[,}\DecValTok{1}\NormalTok{]}
\NormalTok{standartnovirze}\OtherTok{=}\NormalTok{terra}\SpecialCharTok{::}\FunctionTok{global}\NormalTok{(centrets,}\AttributeTok{fun=}\StringTok{"rms"}\NormalTok{,}\AttributeTok{na.rm=}\ConstantTok{TRUE}\NormalTok{)}
\NormalTok{merogots}\OtherTok{=}\NormalTok{centrets}\SpecialCharTok{/}\NormalTok{standartnovirze[,}\DecValTok{1}\NormalTok{]}
\FunctionTok{writeRaster}\NormalTok{(merogots,}
      \AttributeTok{filename=}\NormalTok{saglabasanas\_cels,}
      \AttributeTok{overwrite=}\ConstantTok{TRUE}\NormalTok{)}
\end{Highlighting}
\end{Shaded}

\section{ForestsSoil\_OligotrophicDrained\_cell}\label{ch06.328}

\textbf{filename:} \texttt{ForestsSoil\_OligotrophicDrained\_cell.tif}

\textbf{layername:} \texttt{egv\_328}

\textbf{English name:} Fractional cover of Drained Oligotrophic Forests within the
analysis cell (1 ha)

\textbf{Latvian name:} Susinātu oligotrofu mežu platības īpatsvars analīzes šūnā (1
ha)

\textbf{Procedure:} To prepare this EGV, forest stands with forest type equal to
``17'', ``18'', ``22'' or ``23'' are selected from the \hyperref[Ch04.01]{State Forest Service's State Forest
Registry} and rasterised. Rasterisation is performed using
the workflow \texttt{egvtools::polygon2input()} with background
covering (value 0). The resulting layer
is then aggregated to EGV resolution using the workflow \texttt{egvtools::input2egv()}, which
calculates the arithmetic mean to determine the cover fraction. During
aggregation, inverse distance weighted (power = 2) gap filling on the output is
applied to ensure no missing values at the edges. Finally, the layer is
standardised by subtracting the arithmetic mean and dividing by the root mean squared
error.

\begin{Shaded}
\begin{Highlighting}[]
\CommentTok{\# libs {-}{-}{-}{-}}
\ControlFlowTok{if}\NormalTok{(}\SpecialCharTok{!}\FunctionTok{require}\NormalTok{(egvtools)) \{remotes}\SpecialCharTok{::}\FunctionTok{install\_github}\NormalTok{(}\StringTok{"aavotins/egvtools"}\NormalTok{); }\FunctionTok{require}\NormalTok{(egvtools)\}}
\ControlFlowTok{if}\NormalTok{(}\SpecialCharTok{!}\FunctionTok{require}\NormalTok{(terra)) \{}\FunctionTok{install.packages}\NormalTok{(}\StringTok{"terra"}\NormalTok{); }\FunctionTok{require}\NormalTok{(terra)\}}
\ControlFlowTok{if}\NormalTok{(}\SpecialCharTok{!}\FunctionTok{require}\NormalTok{(sf)) \{}\FunctionTok{install.packages}\NormalTok{(}\StringTok{"sf"}\NormalTok{); }\FunctionTok{require}\NormalTok{(sf)\}}
\ControlFlowTok{if}\NormalTok{(}\SpecialCharTok{!}\FunctionTok{require}\NormalTok{(tidyverse)) \{}\FunctionTok{install.packages}\NormalTok{(}\StringTok{"tidyverse"}\NormalTok{); }\FunctionTok{require}\NormalTok{(tidyverse)\}}
\ControlFlowTok{if}\NormalTok{(}\SpecialCharTok{!}\FunctionTok{require}\NormalTok{(sfarrow)) \{}\FunctionTok{install.packages}\NormalTok{(}\StringTok{"sfarrow"}\NormalTok{); }\FunctionTok{require}\NormalTok{(sfarrow)\}}
\ControlFlowTok{if}\NormalTok{(}\SpecialCharTok{!}\FunctionTok{require}\NormalTok{(readxl)) \{}\FunctionTok{install.packages}\NormalTok{(}\StringTok{"readxl"}\NormalTok{); }\FunctionTok{require}\NormalTok{(readxl)\}}
\ControlFlowTok{if}\NormalTok{(}\SpecialCharTok{!}\FunctionTok{require}\NormalTok{(raster)) \{}\FunctionTok{install.packages}\NormalTok{(}\StringTok{"raster"}\NormalTok{); }\FunctionTok{require}\NormalTok{(raster)\}}
\ControlFlowTok{if}\NormalTok{(}\SpecialCharTok{!}\FunctionTok{require}\NormalTok{(fasterize)) \{}\FunctionTok{install.packages}\NormalTok{(}\StringTok{"fasterize"}\NormalTok{); }\FunctionTok{require}\NormalTok{(fasterize)\}}

\CommentTok{\# templates {-}{-}{-}{-}}
\NormalTok{template100}\OtherTok{=}\FunctionTok{rast}\NormalTok{(}\StringTok{"./Templates/TemplateRasters/LV100m\_10km.tif"}\NormalTok{)}
\NormalTok{template10}\OtherTok{=}\FunctionTok{rast}\NormalTok{(}\StringTok{"./Templates/TemplateRasters/LV10m\_10km.tif"}\NormalTok{)}
\NormalTok{rastrs10}\OtherTok{=}\FunctionTok{raster}\NormalTok{(template10)}

\NormalTok{nulls10}\OtherTok{=}\FunctionTok{rast}\NormalTok{(}\StringTok{"./Templates/TemplateRasters/nulls\_LV10m\_10km.tif"}\NormalTok{)}
\NormalTok{nulls100}\OtherTok{=}\FunctionTok{rast}\NormalTok{(}\StringTok{"./Templates/TemplateRasters/nulls\_LV100m\_10km.tif"}\NormalTok{)}


\CommentTok{\# simple landscape {-}{-}{-}{-}}
\NormalTok{simple\_landscape}\OtherTok{=}\FunctionTok{rast}\NormalTok{(}\StringTok{"RasterGrids\_10m/2024/Ainava\_vienk\_mask.tif"}\NormalTok{)}

\CommentTok{\# mvr {-}{-}{-}{-}}
\NormalTok{mvr}\OtherTok{=}\FunctionTok{st\_read\_parquet}\NormalTok{(}\StringTok{"./Geodata/2024/MVR/nogabali\_2024janv.parquet"}\NormalTok{)}
\NormalTok{mvr}\SpecialCharTok{$}\NormalTok{yes}\OtherTok{=}\DecValTok{1}


\CommentTok{\# ForestsSoil\_OligotrophicDrained\_cell.tif  egv\_328 {-}{-}{-}{-}}
\NormalTok{OligotrophicDrained}\OtherTok{=}\NormalTok{mvr }\SpecialCharTok{\%\textgreater{}\%} 
 \FunctionTok{filter}\NormalTok{(mt }\SpecialCharTok{\%in\%} \FunctionTok{c}\NormalTok{(}\StringTok{"17"}\NormalTok{,}\StringTok{"18"}\NormalTok{,}\StringTok{"22"}\NormalTok{,}\StringTok{"23"}\NormalTok{))}
\NormalTok{p2i\_rez}\OtherTok{=}\NormalTok{egvtools}\SpecialCharTok{::}\FunctionTok{polygon2input}\NormalTok{(}\AttributeTok{vector\_data =}\NormalTok{ OligotrophicDrained,}
                \AttributeTok{template\_path =} \StringTok{"./Templates/TemplateRasters/LV10m\_10km.tif"}\NormalTok{,}
                \AttributeTok{out\_path =} \StringTok{"./RasterGrids\_10m/2024/"}\NormalTok{,}
                \AttributeTok{file\_name =} \StringTok{"ForestsSoil\_OligotrophicDrained\_input.tif"}\NormalTok{,}
                \AttributeTok{value\_field =} \StringTok{"yes"}\NormalTok{,}
                \AttributeTok{prepare=}\ConstantTok{FALSE}\NormalTok{,}
                \AttributeTok{background\_raster =} \StringTok{"./Templates/TemplateRasters/nulls\_LV10m\_10km.tif"}\NormalTok{,}
                \AttributeTok{plot\_result =} \ConstantTok{TRUE}\NormalTok{)}
\NormalTok{p2i\_rez}
\NormalTok{i2e\_rez}\OtherTok{=}\NormalTok{egvtools}\SpecialCharTok{::}\FunctionTok{input2egv}\NormalTok{(}\AttributeTok{input=}\FunctionTok{paste0}\NormalTok{(}\StringTok{"./RasterGrids\_10m/2024/"}\NormalTok{,}
                     \StringTok{"ForestsSoil\_OligotrophicDrained\_input.tif"}\NormalTok{),}
              \AttributeTok{egv\_template=} \StringTok{"./Templates/TemplateRasters/LV100m\_10km.tif"}\NormalTok{,}
              \AttributeTok{summary\_function =} \StringTok{"average"}\NormalTok{,}
              \AttributeTok{missing\_job =} \StringTok{"FillOutput"}\NormalTok{,}
              \AttributeTok{outlocation =} \StringTok{"./RasterGrids\_100m/2024/RAW/"}\NormalTok{,}
              \AttributeTok{outfilename =} \StringTok{"ForestsSoil\_OligotrophicDrained\_cell.tif"}\NormalTok{,}
              \AttributeTok{layername =} \StringTok{"egv\_328"}\NormalTok{,}
              \AttributeTok{idw\_weight =} \DecValTok{2}\NormalTok{,}
              \AttributeTok{plot\_gaps =} \ConstantTok{FALSE}\NormalTok{,}\AttributeTok{plot\_final =} \ConstantTok{TRUE}\NormalTok{)}
\NormalTok{i2e\_rez}
\FunctionTok{rm}\NormalTok{(OligotrophicDrained)}
\FunctionTok{rm}\NormalTok{(p2i\_rez)}
\FunctionTok{rm}\NormalTok{(i2e\_rez)}
\FunctionTok{unlink}\NormalTok{(}\StringTok{"./RasterGrids\_10m/2024/ForestsSoil\_OligotrophicDrained\_input.tif"}\NormalTok{)}

\CommentTok{\# standardisation {-}{-}{-}{-}}
\ControlFlowTok{if}\NormalTok{(}\SpecialCharTok{!}\FunctionTok{require}\NormalTok{(terra)) \{}\FunctionTok{install.packages}\NormalTok{(}\StringTok{"terra"}\NormalTok{); }\FunctionTok{require}\NormalTok{(terra)\}}
\ControlFlowTok{if}\NormalTok{(}\SpecialCharTok{!}\FunctionTok{require}\NormalTok{(tidyverse)) \{}\FunctionTok{install.packages}\NormalTok{(}\StringTok{"tidyverse"}\NormalTok{); }\FunctionTok{require}\NormalTok{(tidyverse)\}}

\NormalTok{nosaukums}\OtherTok{=}\StringTok{"ForestsSoil\_OligotrophicDrained\_cell.tif"}
\NormalTok{ielasisanas\_cels}\OtherTok{=}\FunctionTok{paste0}\NormalTok{(}\StringTok{"./RasterGrids\_100m/2024/RAW/"}\NormalTok{,nosaukums)}
\NormalTok{saglabasanas\_cels}\OtherTok{=}\FunctionTok{paste0}\NormalTok{(}\StringTok{"./RasterGrids\_100m/2024/Scaled/"}\NormalTok{,nosaukums)}
\NormalTok{slanis}\OtherTok{=}\FunctionTok{rast}\NormalTok{(ielasisanas\_cels)}
\NormalTok{videjais}\OtherTok{=}\FunctionTok{global}\NormalTok{(slanis,}\AttributeTok{fun=}\StringTok{"mean"}\NormalTok{,}\AttributeTok{na.rm=}\ConstantTok{TRUE}\NormalTok{)}
\NormalTok{centrets}\OtherTok{=}\NormalTok{slanis}\SpecialCharTok{{-}}\NormalTok{videjais[,}\DecValTok{1}\NormalTok{]}
\NormalTok{standartnovirze}\OtherTok{=}\NormalTok{terra}\SpecialCharTok{::}\FunctionTok{global}\NormalTok{(centrets,}\AttributeTok{fun=}\StringTok{"rms"}\NormalTok{,}\AttributeTok{na.rm=}\ConstantTok{TRUE}\NormalTok{)}
\NormalTok{merogots}\OtherTok{=}\NormalTok{centrets}\SpecialCharTok{/}\NormalTok{standartnovirze[,}\DecValTok{1}\NormalTok{]}
\FunctionTok{writeRaster}\NormalTok{(merogots,}
      \AttributeTok{filename=}\NormalTok{saglabasanas\_cels,}
      \AttributeTok{overwrite=}\ConstantTok{TRUE}\NormalTok{)}
\end{Highlighting}
\end{Shaded}

\section{ForestsSoil\_OligotrophicDrained\_r500}\label{ch06.329}

\textbf{filename:} \texttt{ForestsSoil\_OligotrophicDrained\_r500.tif}

\textbf{layername:} \texttt{egv\_329}

\textbf{English name:} Fractional cover of Drained Oligotrophic Forests within the
0.5 km landscape

\textbf{Latvian name:} Susinātu oligotrofu mežu platības īpatsvars 0,5 km ainavā

\textbf{Procedure:} The cover fraction within a radius of 500 m around the analysis grid cell is
calculated as the area-weighted sum of the \hyperref[ch06.328]{analysis cells} inside the
buffer, using the workflow \texttt{egvtools::radius\_function()}. During the calculation of the landscape metric,
inverse distance weighted (power = 2) gap filling on the output is applied
to ensure no missing values at the edges. Then the layer is rewritten to set
its name. Finally, the layer is standardised by subtracting the arithmetic
mean and dividing by the root mean squared error.

\begin{Shaded}
\begin{Highlighting}[]
\CommentTok{\# libs {-}{-}{-}{-}}
\ControlFlowTok{if}\NormalTok{(}\SpecialCharTok{!}\FunctionTok{require}\NormalTok{(terra)) \{}\FunctionTok{install.packages}\NormalTok{(}\StringTok{"terra"}\NormalTok{); }\FunctionTok{require}\NormalTok{(terra)\}}
\ControlFlowTok{if}\NormalTok{(}\SpecialCharTok{!}\FunctionTok{require}\NormalTok{(egvtools)) \{remotes}\SpecialCharTok{::}\FunctionTok{install\_github}\NormalTok{(}\StringTok{"aavotins/egvtools"}\NormalTok{); }\FunctionTok{require}\NormalTok{(egvtools)\}}


\CommentTok{\# Templates {-}{-}{-}{-}{-}}
\NormalTok{template100}\OtherTok{=}\FunctionTok{rast}\NormalTok{(}\StringTok{"./Templates/TemplateRasters/LV100m\_10km.tif"}\NormalTok{)}

\CommentTok{\# radii {-}{-}{-}{-}}
\FunctionTok{radius\_function}\NormalTok{(}
 \AttributeTok{kvadrati\_path =} \StringTok{"./Templates/TemplateGrids/tiles/"}\NormalTok{,}
 \AttributeTok{radii\_path   =} \StringTok{"./Templates/TemplateGridPoints/tiles/"}\NormalTok{,}
 \AttributeTok{tikls100\_path =} \StringTok{"./Templates/TemplateGrids/tikls100\_sauzeme.parquet"}\NormalTok{,}
 \AttributeTok{template\_path =} \StringTok{"./Templates/TemplateRasters/LV100m\_10km.tif"}\NormalTok{,}
 \AttributeTok{input\_layers  =} \FunctionTok{c}\NormalTok{(}\StringTok{"./RasterGrids\_100m/2024/RAW/ForestsSoil\_OligotrophicDrained\_cell.tif"}\NormalTok{),}
 \AttributeTok{layer\_prefixes =} \FunctionTok{c}\NormalTok{(}\StringTok{"ForestsSoil\_OligotrophicDrained"}\NormalTok{),}
 \AttributeTok{output\_dir   =} \StringTok{"./RasterGrids\_100m/2024/RAW/"}\NormalTok{,}
 \AttributeTok{n\_workers   =} \DecValTok{6}\NormalTok{,}
 \AttributeTok{radii     =} \FunctionTok{c}\NormalTok{(}\StringTok{"r500"}\NormalTok{),}
 \AttributeTok{radius\_mode  =} \StringTok{"sparse"}\NormalTok{,}
 \AttributeTok{extract\_fun  =} \StringTok{"mean"}\NormalTok{,}
 \AttributeTok{fill\_missing  =} \ConstantTok{TRUE}\NormalTok{,}
 \AttributeTok{IDW\_weight   =} \DecValTok{2}\NormalTok{,}
 \AttributeTok{future\_max\_size =} \DecValTok{40} \SpecialCharTok{*} \DecValTok{1024}\SpecialCharTok{\^{}}\DecValTok{3}\NormalTok{)}


\CommentTok{\# ForestsSoil\_OligotrophicDrained\_r500.tif  egv\_329}
\NormalTok{slanis}\OtherTok{=}\FunctionTok{rast}\NormalTok{(}\StringTok{"./RasterGrids\_100m/2024/RAW/ForestsSoil\_OligotrophicDrained\_r500.tif"}\NormalTok{)}
\FunctionTok{names}\NormalTok{(slanis)}\OtherTok{=}\StringTok{"egv\_329"}
\NormalTok{slanis2}\OtherTok{=}\FunctionTok{project}\NormalTok{(slanis,template100)}
\FunctionTok{writeRaster}\NormalTok{(slanis2,}
      \StringTok{"./RasterGrids\_100m/2024/RAW/ForestsSoil\_OligotrophicDrained\_r500.tif"}\NormalTok{,}
      \AttributeTok{overwrite=}\ConstantTok{TRUE}\NormalTok{)}

\CommentTok{\# standardisation {-}{-}{-}{-}}
\ControlFlowTok{if}\NormalTok{(}\SpecialCharTok{!}\FunctionTok{require}\NormalTok{(terra)) \{}\FunctionTok{install.packages}\NormalTok{(}\StringTok{"terra"}\NormalTok{); }\FunctionTok{require}\NormalTok{(terra)\}}
\ControlFlowTok{if}\NormalTok{(}\SpecialCharTok{!}\FunctionTok{require}\NormalTok{(tidyverse)) \{}\FunctionTok{install.packages}\NormalTok{(}\StringTok{"tidyverse"}\NormalTok{); }\FunctionTok{require}\NormalTok{(tidyverse)\}}

\NormalTok{nosaukums}\OtherTok{=}\StringTok{"ForestsSoil\_OligotrophicDrained\_r500.tif"}
\NormalTok{ielasisanas\_cels}\OtherTok{=}\FunctionTok{paste0}\NormalTok{(}\StringTok{"./RasterGrids\_100m/2024/RAW/"}\NormalTok{,nosaukums)}
\NormalTok{saglabasanas\_cels}\OtherTok{=}\FunctionTok{paste0}\NormalTok{(}\StringTok{"./RasterGrids\_100m/2024/Scaled/"}\NormalTok{,nosaukums)}
\NormalTok{slanis}\OtherTok{=}\FunctionTok{rast}\NormalTok{(ielasisanas\_cels)}
\NormalTok{videjais}\OtherTok{=}\FunctionTok{global}\NormalTok{(slanis,}\AttributeTok{fun=}\StringTok{"mean"}\NormalTok{,}\AttributeTok{na.rm=}\ConstantTok{TRUE}\NormalTok{)}
\NormalTok{centrets}\OtherTok{=}\NormalTok{slanis}\SpecialCharTok{{-}}\NormalTok{videjais[,}\DecValTok{1}\NormalTok{]}
\NormalTok{standartnovirze}\OtherTok{=}\NormalTok{terra}\SpecialCharTok{::}\FunctionTok{global}\NormalTok{(centrets,}\AttributeTok{fun=}\StringTok{"rms"}\NormalTok{,}\AttributeTok{na.rm=}\ConstantTok{TRUE}\NormalTok{)}
\NormalTok{merogots}\OtherTok{=}\NormalTok{centrets}\SpecialCharTok{/}\NormalTok{standartnovirze[,}\DecValTok{1}\NormalTok{]}
\FunctionTok{writeRaster}\NormalTok{(merogots,}
      \AttributeTok{filename=}\NormalTok{saglabasanas\_cels,}
      \AttributeTok{overwrite=}\ConstantTok{TRUE}\NormalTok{)}
\end{Highlighting}
\end{Shaded}

\section{ForestsSoil\_OligotrophicDrained\_r1250}\label{ch06.330}

\textbf{filename:} \texttt{ForestsSoil\_OligotrophicDrained\_r1250.tif}

\textbf{layername:} \texttt{egv\_330}

\textbf{English name:} Fractional cover of Drained Oligotrophic Forests within the
1.25 km landscape

\textbf{Latvian name:} Susinātu oligotrofu mežu platības īpatsvars 1,25 km ainavā

\textbf{Procedure:} The cover fraction within a radius of 1250 m around the analysis grid cell
is calculated as the area-weighted sum of the \hyperref[ch06.328]{analysis cells} inside
the buffer, using the workflow \texttt{egvtools::radius\_function()}. During the calculation of the landscape
metric, inverse distance weighted (power = 2) gap filling on the output is
applied to ensure no missing values at the edges. Then the layer is
rewritten to set its name. Finally, the layer is standardised by
subtracting the arithmetic mean and dividing by the root mean squared error.

\begin{Shaded}
\begin{Highlighting}[]
\CommentTok{\# libs {-}{-}{-}{-}}
\ControlFlowTok{if}\NormalTok{(}\SpecialCharTok{!}\FunctionTok{require}\NormalTok{(terra)) \{}\FunctionTok{install.packages}\NormalTok{(}\StringTok{"terra"}\NormalTok{); }\FunctionTok{require}\NormalTok{(terra)\}}
\ControlFlowTok{if}\NormalTok{(}\SpecialCharTok{!}\FunctionTok{require}\NormalTok{(egvtools)) \{remotes}\SpecialCharTok{::}\FunctionTok{install\_github}\NormalTok{(}\StringTok{"aavotins/egvtools"}\NormalTok{); }\FunctionTok{require}\NormalTok{(egvtools)\}}


\CommentTok{\# Templates {-}{-}{-}{-}{-}}
\NormalTok{template100}\OtherTok{=}\FunctionTok{rast}\NormalTok{(}\StringTok{"./Templates/TemplateRasters/LV100m\_10km.tif"}\NormalTok{)}

\CommentTok{\# radii {-}{-}{-}{-}}
\FunctionTok{radius\_function}\NormalTok{(}
 \AttributeTok{kvadrati\_path =} \StringTok{"./Templates/TemplateGrids/tiles/"}\NormalTok{,}
 \AttributeTok{radii\_path   =} \StringTok{"./Templates/TemplateGridPoints/tiles/"}\NormalTok{,}
 \AttributeTok{tikls100\_path =} \StringTok{"./Templates/TemplateGrids/tikls100\_sauzeme.parquet"}\NormalTok{,}
 \AttributeTok{template\_path =} \StringTok{"./Templates/TemplateRasters/LV100m\_10km.tif"}\NormalTok{,}
 \AttributeTok{input\_layers  =} \FunctionTok{c}\NormalTok{(}\StringTok{"./RasterGrids\_100m/2024/RAW/ForestsSoil\_OligotrophicDrained\_cell.tif"}\NormalTok{),}
 \AttributeTok{layer\_prefixes =} \FunctionTok{c}\NormalTok{(}\StringTok{"ForestsSoil\_OligotrophicDrained"}\NormalTok{),}
 \AttributeTok{output\_dir   =} \StringTok{"./RasterGrids\_100m/2024/RAW/"}\NormalTok{,}
 \AttributeTok{n\_workers   =} \DecValTok{6}\NormalTok{,}
 \AttributeTok{radii     =} \FunctionTok{c}\NormalTok{(}\StringTok{"r1250"}\NormalTok{),}
 \AttributeTok{radius\_mode  =} \StringTok{"sparse"}\NormalTok{,}
 \AttributeTok{extract\_fun  =} \StringTok{"mean"}\NormalTok{,}
 \AttributeTok{fill\_missing  =} \ConstantTok{TRUE}\NormalTok{,}
 \AttributeTok{IDW\_weight   =} \DecValTok{2}\NormalTok{,}
 \AttributeTok{future\_max\_size =} \DecValTok{40} \SpecialCharTok{*} \DecValTok{1024}\SpecialCharTok{\^{}}\DecValTok{3}\NormalTok{)}


\CommentTok{\# ForestsSoil\_OligotrophicDrained\_r1250.tif egv\_330}
\NormalTok{slanis}\OtherTok{=}\FunctionTok{rast}\NormalTok{(}\StringTok{"./RasterGrids\_100m/2024/RAW/ForestsSoil\_OligotrophicDrained\_r1250.tif"}\NormalTok{)}
\FunctionTok{names}\NormalTok{(slanis)}\OtherTok{=}\StringTok{"egv\_330"}
\NormalTok{slanis2}\OtherTok{=}\FunctionTok{project}\NormalTok{(slanis,template100)}
\FunctionTok{writeRaster}\NormalTok{(slanis2,}
      \StringTok{"./RasterGrids\_100m/2024/RAW/ForestsSoil\_OligotrophicDrained\_r1250.tif"}\NormalTok{,}
      \AttributeTok{overwrite=}\ConstantTok{TRUE}\NormalTok{)}

\CommentTok{\# standardisation {-}{-}{-}{-}}
\ControlFlowTok{if}\NormalTok{(}\SpecialCharTok{!}\FunctionTok{require}\NormalTok{(terra)) \{}\FunctionTok{install.packages}\NormalTok{(}\StringTok{"terra"}\NormalTok{); }\FunctionTok{require}\NormalTok{(terra)\}}
\ControlFlowTok{if}\NormalTok{(}\SpecialCharTok{!}\FunctionTok{require}\NormalTok{(tidyverse)) \{}\FunctionTok{install.packages}\NormalTok{(}\StringTok{"tidyverse"}\NormalTok{); }\FunctionTok{require}\NormalTok{(tidyverse)\}}

\NormalTok{nosaukums}\OtherTok{=}\StringTok{"ForestsSoil\_OligotrophicDrained\_r1250.tif"}
\NormalTok{ielasisanas\_cels}\OtherTok{=}\FunctionTok{paste0}\NormalTok{(}\StringTok{"./RasterGrids\_100m/2024/RAW/"}\NormalTok{,nosaukums)}
\NormalTok{saglabasanas\_cels}\OtherTok{=}\FunctionTok{paste0}\NormalTok{(}\StringTok{"./RasterGrids\_100m/2024/Scaled/"}\NormalTok{,nosaukums)}
\NormalTok{slanis}\OtherTok{=}\FunctionTok{rast}\NormalTok{(ielasisanas\_cels)}
\NormalTok{videjais}\OtherTok{=}\FunctionTok{global}\NormalTok{(slanis,}\AttributeTok{fun=}\StringTok{"mean"}\NormalTok{,}\AttributeTok{na.rm=}\ConstantTok{TRUE}\NormalTok{)}
\NormalTok{centrets}\OtherTok{=}\NormalTok{slanis}\SpecialCharTok{{-}}\NormalTok{videjais[,}\DecValTok{1}\NormalTok{]}
\NormalTok{standartnovirze}\OtherTok{=}\NormalTok{terra}\SpecialCharTok{::}\FunctionTok{global}\NormalTok{(centrets,}\AttributeTok{fun=}\StringTok{"rms"}\NormalTok{,}\AttributeTok{na.rm=}\ConstantTok{TRUE}\NormalTok{)}
\NormalTok{merogots}\OtherTok{=}\NormalTok{centrets}\SpecialCharTok{/}\NormalTok{standartnovirze[,}\DecValTok{1}\NormalTok{]}
\FunctionTok{writeRaster}\NormalTok{(merogots,}
      \AttributeTok{filename=}\NormalTok{saglabasanas\_cels,}
      \AttributeTok{overwrite=}\ConstantTok{TRUE}\NormalTok{)}
\end{Highlighting}
\end{Shaded}

\section{ForestsSoil\_OligotrophicDrained\_r3000}\label{ch06.331}

\textbf{filename:} \texttt{ForestsSoil\_OligotrophicDrained\_r3000.tif}

\textbf{layername:} \texttt{egv\_331}

\textbf{English name:} Fractional cover of Drained Oligotrophic Forests within the 3
km landscape

\textbf{Latvian name:} Susinātu oligotrofu mežu platības īpatsvars 3 km ainavā

\textbf{Procedure:} The cover fraction within a radius of 3000 m around the analysis grid cell
is calculated as the area-weighted sum of the \hyperref[ch06.328]{analysis cells} inside
the buffer, using the workflow \texttt{egvtools::radius\_function()}. During the calculation of the landscape
metric, inverse distance weighted (power = 2) gap filling on the output is
applied to ensure no missing values at the edges. Then the layer is
rewritten to set its name. Finally, the layer is standardised by
subtracting the arithmetic mean and dividing by the root mean squared error.

\begin{Shaded}
\begin{Highlighting}[]
\CommentTok{\# libs {-}{-}{-}{-}}
\ControlFlowTok{if}\NormalTok{(}\SpecialCharTok{!}\FunctionTok{require}\NormalTok{(terra)) \{}\FunctionTok{install.packages}\NormalTok{(}\StringTok{"terra"}\NormalTok{); }\FunctionTok{require}\NormalTok{(terra)\}}
\ControlFlowTok{if}\NormalTok{(}\SpecialCharTok{!}\FunctionTok{require}\NormalTok{(egvtools)) \{remotes}\SpecialCharTok{::}\FunctionTok{install\_github}\NormalTok{(}\StringTok{"aavotins/egvtools"}\NormalTok{); }\FunctionTok{require}\NormalTok{(egvtools)\}}


\CommentTok{\# Templates {-}{-}{-}{-}{-}}
\NormalTok{template100}\OtherTok{=}\FunctionTok{rast}\NormalTok{(}\StringTok{"./Templates/TemplateRasters/LV100m\_10km.tif"}\NormalTok{)}

\CommentTok{\# radii {-}{-}{-}{-}}
\FunctionTok{radius\_function}\NormalTok{(}
 \AttributeTok{kvadrati\_path =} \StringTok{"./Templates/TemplateGrids/tiles/"}\NormalTok{,}
 \AttributeTok{radii\_path   =} \StringTok{"./Templates/TemplateGridPoints/tiles/"}\NormalTok{,}
 \AttributeTok{tikls100\_path =} \StringTok{"./Templates/TemplateGrids/tikls100\_sauzeme.parquet"}\NormalTok{,}
 \AttributeTok{template\_path =} \StringTok{"./Templates/TemplateRasters/LV100m\_10km.tif"}\NormalTok{,}
 \AttributeTok{input\_layers  =} \FunctionTok{c}\NormalTok{(}\StringTok{"./RasterGrids\_100m/2024/RAW/ForestsSoil\_OligotrophicDrained\_cell.tif"}\NormalTok{),}
 \AttributeTok{layer\_prefixes =} \FunctionTok{c}\NormalTok{(}\StringTok{"ForestsSoil\_OligotrophicDrained"}\NormalTok{),}
 \AttributeTok{output\_dir   =} \StringTok{"./RasterGrids\_100m/2024/RAW/"}\NormalTok{,}
 \AttributeTok{n\_workers   =} \DecValTok{6}\NormalTok{,}
 \AttributeTok{radii     =} \FunctionTok{c}\NormalTok{(}\StringTok{"r3000"}\NormalTok{),}
 \AttributeTok{radius\_mode  =} \StringTok{"sparse"}\NormalTok{,}
 \AttributeTok{extract\_fun  =} \StringTok{"mean"}\NormalTok{,}
 \AttributeTok{fill\_missing  =} \ConstantTok{TRUE}\NormalTok{,}
 \AttributeTok{IDW\_weight   =} \DecValTok{2}\NormalTok{,}
 \AttributeTok{future\_max\_size =} \DecValTok{40} \SpecialCharTok{*} \DecValTok{1024}\SpecialCharTok{\^{}}\DecValTok{3}\NormalTok{)}


\CommentTok{\# ForestsSoil\_OligotrophicDrained\_r3000.tif egv\_331}
\NormalTok{slanis}\OtherTok{=}\FunctionTok{rast}\NormalTok{(}\StringTok{"./RasterGrids\_100m/2024/RAW/ForestsSoil\_OligotrophicDrained\_r3000.tif"}\NormalTok{)}
\FunctionTok{names}\NormalTok{(slanis)}\OtherTok{=}\StringTok{"egv\_331"}
\NormalTok{slanis2}\OtherTok{=}\FunctionTok{project}\NormalTok{(slanis,template100)}
\FunctionTok{writeRaster}\NormalTok{(slanis2,}
      \StringTok{"./RasterGrids\_100m/2024/RAW/ForestsSoil\_OligotrophicDrained\_r3000.tif"}\NormalTok{,}
      \AttributeTok{overwrite=}\ConstantTok{TRUE}\NormalTok{)}

\CommentTok{\# standardisation {-}{-}{-}{-}}
\ControlFlowTok{if}\NormalTok{(}\SpecialCharTok{!}\FunctionTok{require}\NormalTok{(terra)) \{}\FunctionTok{install.packages}\NormalTok{(}\StringTok{"terra"}\NormalTok{); }\FunctionTok{require}\NormalTok{(terra)\}}
\ControlFlowTok{if}\NormalTok{(}\SpecialCharTok{!}\FunctionTok{require}\NormalTok{(tidyverse)) \{}\FunctionTok{install.packages}\NormalTok{(}\StringTok{"tidyverse"}\NormalTok{); }\FunctionTok{require}\NormalTok{(tidyverse)\}}

\NormalTok{nosaukums}\OtherTok{=}\StringTok{"ForestsSoil\_OligotrophicDrained\_r3000.tif"}
\NormalTok{ielasisanas\_cels}\OtherTok{=}\FunctionTok{paste0}\NormalTok{(}\StringTok{"./RasterGrids\_100m/2024/RAW/"}\NormalTok{,nosaukums)}
\NormalTok{saglabasanas\_cels}\OtherTok{=}\FunctionTok{paste0}\NormalTok{(}\StringTok{"./RasterGrids\_100m/2024/Scaled/"}\NormalTok{,nosaukums)}
\NormalTok{slanis}\OtherTok{=}\FunctionTok{rast}\NormalTok{(ielasisanas\_cels)}
\NormalTok{videjais}\OtherTok{=}\FunctionTok{global}\NormalTok{(slanis,}\AttributeTok{fun=}\StringTok{"mean"}\NormalTok{,}\AttributeTok{na.rm=}\ConstantTok{TRUE}\NormalTok{)}
\NormalTok{centrets}\OtherTok{=}\NormalTok{slanis}\SpecialCharTok{{-}}\NormalTok{videjais[,}\DecValTok{1}\NormalTok{]}
\NormalTok{standartnovirze}\OtherTok{=}\NormalTok{terra}\SpecialCharTok{::}\FunctionTok{global}\NormalTok{(centrets,}\AttributeTok{fun=}\StringTok{"rms"}\NormalTok{,}\AttributeTok{na.rm=}\ConstantTok{TRUE}\NormalTok{)}
\NormalTok{merogots}\OtherTok{=}\NormalTok{centrets}\SpecialCharTok{/}\NormalTok{standartnovirze[,}\DecValTok{1}\NormalTok{]}
\FunctionTok{writeRaster}\NormalTok{(merogots,}
      \AttributeTok{filename=}\NormalTok{saglabasanas\_cels,}
      \AttributeTok{overwrite=}\ConstantTok{TRUE}\NormalTok{)}
\end{Highlighting}
\end{Shaded}

\section{ForestsSoil\_OligotrophicDrained\_r10000}\label{ch06.332}

\textbf{filename:} \texttt{ForestsSoil\_OligotrophicDrained\_r10000.tif}

\textbf{layername:} \texttt{egv\_332}

\textbf{English name:} Fractional cover of Drained Oligotrophic Forests within the 10
km landscape

\textbf{Latvian name:} Susinātu oligotrofu mežu platības īpatsvars 10 km ainavā

\textbf{Procedure:} The cover fraction within a radius of 10000 m around the analysis grid cell
is calculated as the area-weighted sum of the \hyperref[ch06.328]{analysis cells} inside
the buffer, using the workflow \texttt{egvtools::radius\_function()}. During the calculation of the landscape
metric, inverse distance weighted (power = 2) gap filling on the output is
applied to ensure no missing values at the edges. Then the layer is
rewritten to set its name. Finally, the layer is standardised by
subtracting the arithmetic mean and dividing by the root mean squared error.

\begin{Shaded}
\begin{Highlighting}[]
\CommentTok{\# libs {-}{-}{-}{-}}
\ControlFlowTok{if}\NormalTok{(}\SpecialCharTok{!}\FunctionTok{require}\NormalTok{(terra)) \{}\FunctionTok{install.packages}\NormalTok{(}\StringTok{"terra"}\NormalTok{); }\FunctionTok{require}\NormalTok{(terra)\}}
\ControlFlowTok{if}\NormalTok{(}\SpecialCharTok{!}\FunctionTok{require}\NormalTok{(egvtools)) \{remotes}\SpecialCharTok{::}\FunctionTok{install\_github}\NormalTok{(}\StringTok{"aavotins/egvtools"}\NormalTok{); }\FunctionTok{require}\NormalTok{(egvtools)\}}


\CommentTok{\# Templates {-}{-}{-}{-}{-}}
\NormalTok{template100}\OtherTok{=}\FunctionTok{rast}\NormalTok{(}\StringTok{"./Templates/TemplateRasters/LV100m\_10km.tif"}\NormalTok{)}

\CommentTok{\# radii {-}{-}{-}{-}}
\FunctionTok{radius\_function}\NormalTok{(}
 \AttributeTok{kvadrati\_path =} \StringTok{"./Templates/TemplateGrids/tiles/"}\NormalTok{,}
 \AttributeTok{radii\_path   =} \StringTok{"./Templates/TemplateGridPoints/tiles/"}\NormalTok{,}
 \AttributeTok{tikls100\_path =} \StringTok{"./Templates/TemplateGrids/tikls100\_sauzeme.parquet"}\NormalTok{,}
 \AttributeTok{template\_path =} \StringTok{"./Templates/TemplateRasters/LV100m\_10km.tif"}\NormalTok{,}
 \AttributeTok{input\_layers  =} \FunctionTok{c}\NormalTok{(}\StringTok{"./RasterGrids\_100m/2024/RAW/ForestsSoil\_OligotrophicDrained\_cell.tif"}\NormalTok{),}
 \AttributeTok{layer\_prefixes =} \FunctionTok{c}\NormalTok{(}\StringTok{"ForestsSoil\_OligotrophicDrained"}\NormalTok{),}
 \AttributeTok{output\_dir   =} \StringTok{"./RasterGrids\_100m/2024/RAW/"}\NormalTok{,}
 \AttributeTok{n\_workers   =} \DecValTok{6}\NormalTok{,}
 \AttributeTok{radii     =} \FunctionTok{c}\NormalTok{(}\StringTok{"r10000"}\NormalTok{),}
 \AttributeTok{radius\_mode  =} \StringTok{"sparse"}\NormalTok{,}
 \AttributeTok{extract\_fun  =} \StringTok{"mean"}\NormalTok{,}
 \AttributeTok{fill\_missing  =} \ConstantTok{TRUE}\NormalTok{,}
 \AttributeTok{IDW\_weight   =} \DecValTok{2}\NormalTok{,}
 \AttributeTok{future\_max\_size =} \DecValTok{40} \SpecialCharTok{*} \DecValTok{1024}\SpecialCharTok{\^{}}\DecValTok{3}\NormalTok{)}


\CommentTok{\# ForestsSoil\_OligotrophicDrained\_r10000.tif    egv\_332}
\NormalTok{slanis}\OtherTok{=}\FunctionTok{rast}\NormalTok{(}\StringTok{"./RasterGrids\_100m/2024/RAW/ForestsSoil\_OligotrophicDrained\_r10000.tif"}\NormalTok{)}
\FunctionTok{names}\NormalTok{(slanis)}\OtherTok{=}\StringTok{"egv\_332"}
\NormalTok{slanis2}\OtherTok{=}\FunctionTok{project}\NormalTok{(slanis,template100)}
\FunctionTok{writeRaster}\NormalTok{(slanis2,}
      \StringTok{"./RasterGrids\_100m/2024/RAW/ForestsSoil\_OligotrophicDrained\_r10000.tif"}\NormalTok{,}
      \AttributeTok{overwrite=}\ConstantTok{TRUE}\NormalTok{)}

\CommentTok{\# standardisation {-}{-}{-}{-}}
\ControlFlowTok{if}\NormalTok{(}\SpecialCharTok{!}\FunctionTok{require}\NormalTok{(terra)) \{}\FunctionTok{install.packages}\NormalTok{(}\StringTok{"terra"}\NormalTok{); }\FunctionTok{require}\NormalTok{(terra)\}}
\ControlFlowTok{if}\NormalTok{(}\SpecialCharTok{!}\FunctionTok{require}\NormalTok{(tidyverse)) \{}\FunctionTok{install.packages}\NormalTok{(}\StringTok{"tidyverse"}\NormalTok{); }\FunctionTok{require}\NormalTok{(tidyverse)\}}

\NormalTok{nosaukums}\OtherTok{=}\StringTok{"ForestsSoil\_OligotrophicDrained\_r10000.tif"}
\NormalTok{ielasisanas\_cels}\OtherTok{=}\FunctionTok{paste0}\NormalTok{(}\StringTok{"./RasterGrids\_100m/2024/RAW/"}\NormalTok{,nosaukums)}
\NormalTok{saglabasanas\_cels}\OtherTok{=}\FunctionTok{paste0}\NormalTok{(}\StringTok{"./RasterGrids\_100m/2024/Scaled/"}\NormalTok{,nosaukums)}
\NormalTok{slanis}\OtherTok{=}\FunctionTok{rast}\NormalTok{(ielasisanas\_cels)}
\NormalTok{videjais}\OtherTok{=}\FunctionTok{global}\NormalTok{(slanis,}\AttributeTok{fun=}\StringTok{"mean"}\NormalTok{,}\AttributeTok{na.rm=}\ConstantTok{TRUE}\NormalTok{)}
\NormalTok{centrets}\OtherTok{=}\NormalTok{slanis}\SpecialCharTok{{-}}\NormalTok{videjais[,}\DecValTok{1}\NormalTok{]}
\NormalTok{standartnovirze}\OtherTok{=}\NormalTok{terra}\SpecialCharTok{::}\FunctionTok{global}\NormalTok{(centrets,}\AttributeTok{fun=}\StringTok{"rms"}\NormalTok{,}\AttributeTok{na.rm=}\ConstantTok{TRUE}\NormalTok{)}
\NormalTok{merogots}\OtherTok{=}\NormalTok{centrets}\SpecialCharTok{/}\NormalTok{standartnovirze[,}\DecValTok{1}\NormalTok{]}
\FunctionTok{writeRaster}\NormalTok{(merogots,}
      \AttributeTok{filename=}\NormalTok{saglabasanas\_cels,}
      \AttributeTok{overwrite=}\ConstantTok{TRUE}\NormalTok{)}
\end{Highlighting}
\end{Shaded}

\section{ForestsSoil\_OligotrophicMineral\_cell}\label{ch06.333}

\textbf{filename:} \texttt{ForestsSoil\_OligotrophicMineral\_cell.tif}

\textbf{layername:} \texttt{egv\_333}

\textbf{English name:} Fractional cover of Oligotrophic Forests on undrained Mineral
Soils within the analysis cell (1 ha)

\textbf{Latvian name:} Oligotrofu mežu nesusinātās minerālaugsnēs platības īpatsvars
analīzes šūnā (1 ha)

\textbf{Procedure:} To prepare this EGV, forest stands with forest type equal to ``1'',
``2'', ``3'', ``7'' or ``8'' are selected from the \hyperref[Ch04.01]{State Forest Service's State Forest
Registry} and rasterised. Rasterisation is performed using the
workflow \texttt{egvtools::polygon2input()} with background
covering (value 0). The resulting layer
is then aggregated to EGV resolution using the workflow \texttt{egvtools::input2egv()}, which
calculates the arithmetic mean to determine the cover fraction. During
aggregation, inverse distance weighted (power = 2) gap filling on the output is
applied to ensure no missing values at the edges. Finally, the layer is
standardised by subtracting the arithmetic mean and dividing by the root mean squared
error.

\begin{Shaded}
\begin{Highlighting}[]
\CommentTok{\# libs {-}{-}{-}{-}}
\ControlFlowTok{if}\NormalTok{(}\SpecialCharTok{!}\FunctionTok{require}\NormalTok{(egvtools)) \{remotes}\SpecialCharTok{::}\FunctionTok{install\_github}\NormalTok{(}\StringTok{"aavotins/egvtools"}\NormalTok{); }\FunctionTok{require}\NormalTok{(egvtools)\}}
\ControlFlowTok{if}\NormalTok{(}\SpecialCharTok{!}\FunctionTok{require}\NormalTok{(terra)) \{}\FunctionTok{install.packages}\NormalTok{(}\StringTok{"terra"}\NormalTok{); }\FunctionTok{require}\NormalTok{(terra)\}}
\ControlFlowTok{if}\NormalTok{(}\SpecialCharTok{!}\FunctionTok{require}\NormalTok{(sf)) \{}\FunctionTok{install.packages}\NormalTok{(}\StringTok{"sf"}\NormalTok{); }\FunctionTok{require}\NormalTok{(sf)\}}
\ControlFlowTok{if}\NormalTok{(}\SpecialCharTok{!}\FunctionTok{require}\NormalTok{(tidyverse)) \{}\FunctionTok{install.packages}\NormalTok{(}\StringTok{"tidyverse"}\NormalTok{); }\FunctionTok{require}\NormalTok{(tidyverse)\}}
\ControlFlowTok{if}\NormalTok{(}\SpecialCharTok{!}\FunctionTok{require}\NormalTok{(sfarrow)) \{}\FunctionTok{install.packages}\NormalTok{(}\StringTok{"sfarrow"}\NormalTok{); }\FunctionTok{require}\NormalTok{(sfarrow)\}}
\ControlFlowTok{if}\NormalTok{(}\SpecialCharTok{!}\FunctionTok{require}\NormalTok{(readxl)) \{}\FunctionTok{install.packages}\NormalTok{(}\StringTok{"readxl"}\NormalTok{); }\FunctionTok{require}\NormalTok{(readxl)\}}
\ControlFlowTok{if}\NormalTok{(}\SpecialCharTok{!}\FunctionTok{require}\NormalTok{(raster)) \{}\FunctionTok{install.packages}\NormalTok{(}\StringTok{"raster"}\NormalTok{); }\FunctionTok{require}\NormalTok{(raster)\}}
\ControlFlowTok{if}\NormalTok{(}\SpecialCharTok{!}\FunctionTok{require}\NormalTok{(fasterize)) \{}\FunctionTok{install.packages}\NormalTok{(}\StringTok{"fasterize"}\NormalTok{); }\FunctionTok{require}\NormalTok{(fasterize)\}}

\CommentTok{\# templates {-}{-}{-}{-}}
\NormalTok{template100}\OtherTok{=}\FunctionTok{rast}\NormalTok{(}\StringTok{"./Templates/TemplateRasters/LV100m\_10km.tif"}\NormalTok{)}
\NormalTok{template10}\OtherTok{=}\FunctionTok{rast}\NormalTok{(}\StringTok{"./Templates/TemplateRasters/LV10m\_10km.tif"}\NormalTok{)}
\NormalTok{rastrs10}\OtherTok{=}\FunctionTok{raster}\NormalTok{(template10)}

\NormalTok{nulls10}\OtherTok{=}\FunctionTok{rast}\NormalTok{(}\StringTok{"./Templates/TemplateRasters/nulls\_LV10m\_10km.tif"}\NormalTok{)}
\NormalTok{nulls100}\OtherTok{=}\FunctionTok{rast}\NormalTok{(}\StringTok{"./Templates/TemplateRasters/nulls\_LV100m\_10km.tif"}\NormalTok{)}


\CommentTok{\# simple landscape {-}{-}{-}{-}}
\NormalTok{simple\_landscape}\OtherTok{=}\FunctionTok{rast}\NormalTok{(}\StringTok{"RasterGrids\_10m/2024/Ainava\_vienk\_mask.tif"}\NormalTok{)}

\CommentTok{\# mvr {-}{-}{-}{-}}
\NormalTok{mvr}\OtherTok{=}\FunctionTok{st\_read\_parquet}\NormalTok{(}\StringTok{"./Geodata/2024/MVR/nogabali\_2024janv.parquet"}\NormalTok{)}
\NormalTok{mvr}\SpecialCharTok{$}\NormalTok{yes}\OtherTok{=}\DecValTok{1}


\CommentTok{\# ForestsSoil\_OligotrophicMineral\_cell.tif  egv\_333 {-}{-}{-}{-}}
\NormalTok{OligotrophicMineral}\OtherTok{=}\NormalTok{mvr }\SpecialCharTok{\%\textgreater{}\%} 
 \FunctionTok{filter}\NormalTok{(mt }\SpecialCharTok{\%in\%} \FunctionTok{c}\NormalTok{(}\StringTok{"1"}\NormalTok{,}\StringTok{"2"}\NormalTok{,}\StringTok{"3"}\NormalTok{,}\StringTok{"7"}\NormalTok{,}\StringTok{"8"}\NormalTok{))}
\NormalTok{p2i\_rez}\OtherTok{=}\NormalTok{egvtools}\SpecialCharTok{::}\FunctionTok{polygon2input}\NormalTok{(}\AttributeTok{vector\_data =}\NormalTok{ OligotrophicMineral,}
                \AttributeTok{template\_path =} \StringTok{"./Templates/TemplateRasters/LV10m\_10km.tif"}\NormalTok{,}
                \AttributeTok{out\_path =} \StringTok{"./RasterGrids\_10m/2024/"}\NormalTok{,}
                \AttributeTok{file\_name =} \StringTok{"ForestsSoil\_OligotrophicMineral\_input.tif"}\NormalTok{,}
                \AttributeTok{value\_field =} \StringTok{"yes"}\NormalTok{,}
                \AttributeTok{prepare=}\ConstantTok{FALSE}\NormalTok{,}
                \AttributeTok{background\_raster =} \StringTok{"./Templates/TemplateRasters/nulls\_LV10m\_10km.tif"}\NormalTok{,}
                \AttributeTok{plot\_result =} \ConstantTok{TRUE}\NormalTok{)}
\NormalTok{p2i\_rez}
\NormalTok{i2e\_rez}\OtherTok{=}\NormalTok{egvtools}\SpecialCharTok{::}\FunctionTok{input2egv}\NormalTok{(}\AttributeTok{input=}\FunctionTok{paste0}\NormalTok{(}\StringTok{"./RasterGrids\_10m/2024/"}\NormalTok{,}
                     \StringTok{"ForestsSoil\_OligotrophicMineral\_input.tif"}\NormalTok{),}
              \AttributeTok{egv\_template=} \StringTok{"./Templates/TemplateRasters/LV100m\_10km.tif"}\NormalTok{,}
              \AttributeTok{summary\_function =} \StringTok{"average"}\NormalTok{,}
              \AttributeTok{missing\_job =} \StringTok{"FillOutput"}\NormalTok{,}
              \AttributeTok{outlocation =} \StringTok{"./RasterGrids\_100m/2024/RAW/"}\NormalTok{,}
              \AttributeTok{outfilename =} \StringTok{"ForestsSoil\_OligotrophicMineral\_cell.tif"}\NormalTok{,}
              \AttributeTok{layername =} \StringTok{"egv\_333"}\NormalTok{,}
              \AttributeTok{idw\_weight =} \DecValTok{2}\NormalTok{,}
              \AttributeTok{plot\_gaps =} \ConstantTok{FALSE}\NormalTok{,}\AttributeTok{plot\_final =} \ConstantTok{TRUE}\NormalTok{)}
\NormalTok{i2e\_rez}
\FunctionTok{rm}\NormalTok{(OligotrophicMineral)}
\FunctionTok{rm}\NormalTok{(p2i\_rez)}
\FunctionTok{rm}\NormalTok{(i2e\_rez)}
\FunctionTok{unlink}\NormalTok{(}\StringTok{"./RasterGrids\_10m/2024/ForestsSoil\_OligotrophicMineral\_input.tif"}\NormalTok{)}

\CommentTok{\# standardisation {-}{-}{-}{-}}
\ControlFlowTok{if}\NormalTok{(}\SpecialCharTok{!}\FunctionTok{require}\NormalTok{(terra)) \{}\FunctionTok{install.packages}\NormalTok{(}\StringTok{"terra"}\NormalTok{); }\FunctionTok{require}\NormalTok{(terra)\}}
\ControlFlowTok{if}\NormalTok{(}\SpecialCharTok{!}\FunctionTok{require}\NormalTok{(tidyverse)) \{}\FunctionTok{install.packages}\NormalTok{(}\StringTok{"tidyverse"}\NormalTok{); }\FunctionTok{require}\NormalTok{(tidyverse)\}}

\NormalTok{nosaukums}\OtherTok{=}\StringTok{"ForestsSoil\_OligotrophicMineral\_cell.tif"}
\NormalTok{ielasisanas\_cels}\OtherTok{=}\FunctionTok{paste0}\NormalTok{(}\StringTok{"./RasterGrids\_100m/2024/RAW/"}\NormalTok{,nosaukums)}
\NormalTok{saglabasanas\_cels}\OtherTok{=}\FunctionTok{paste0}\NormalTok{(}\StringTok{"./RasterGrids\_100m/2024/Scaled/"}\NormalTok{,nosaukums)}
\NormalTok{slanis}\OtherTok{=}\FunctionTok{rast}\NormalTok{(ielasisanas\_cels)}
\NormalTok{videjais}\OtherTok{=}\FunctionTok{global}\NormalTok{(slanis,}\AttributeTok{fun=}\StringTok{"mean"}\NormalTok{,}\AttributeTok{na.rm=}\ConstantTok{TRUE}\NormalTok{)}
\NormalTok{centrets}\OtherTok{=}\NormalTok{slanis}\SpecialCharTok{{-}}\NormalTok{videjais[,}\DecValTok{1}\NormalTok{]}
\NormalTok{standartnovirze}\OtherTok{=}\NormalTok{terra}\SpecialCharTok{::}\FunctionTok{global}\NormalTok{(centrets,}\AttributeTok{fun=}\StringTok{"rms"}\NormalTok{,}\AttributeTok{na.rm=}\ConstantTok{TRUE}\NormalTok{)}
\NormalTok{merogots}\OtherTok{=}\NormalTok{centrets}\SpecialCharTok{/}\NormalTok{standartnovirze[,}\DecValTok{1}\NormalTok{]}
\FunctionTok{writeRaster}\NormalTok{(merogots,}
      \AttributeTok{filename=}\NormalTok{saglabasanas\_cels,}
      \AttributeTok{overwrite=}\ConstantTok{TRUE}\NormalTok{)}
\end{Highlighting}
\end{Shaded}

\section{ForestsSoil\_OligotrophicMineral\_r500}\label{ch06.334}

\textbf{filename:} \texttt{ForestsSoil\_OligotrophicMineral\_r500.tif}

\textbf{layername:} \texttt{egv\_334}

\textbf{English name:} Fractional cover of Oligotrophic Forests on undrained Mineral
Soils within the 0.5 km landscape

\textbf{Latvian name:} Oligotrofu mežu nesusinātās minerālaugsnēs platības īpatsvars
0,5 km ainavā

\textbf{Procedure:} The cover fraction within a radius of 500 m around the analysis grid cell is
calculated as the area-weighted sum of the \hyperref[ch06.333]{analysis cells} inside the
buffer, using the workflow \texttt{egvtools::radius\_function()}. During the calculation of the landscape metric,
inverse distance weighted (power = 2) gap filling on the output is applied
to ensure no missing values at the edges. Then the layer is rewritten to set
its name. Finally, the layer is standardised by subtracting the arithmetic
mean and dividing by the root mean squared error.

\begin{Shaded}
\begin{Highlighting}[]
\CommentTok{\# libs {-}{-}{-}{-}}
\ControlFlowTok{if}\NormalTok{(}\SpecialCharTok{!}\FunctionTok{require}\NormalTok{(terra)) \{}\FunctionTok{install.packages}\NormalTok{(}\StringTok{"terra"}\NormalTok{); }\FunctionTok{require}\NormalTok{(terra)\}}
\ControlFlowTok{if}\NormalTok{(}\SpecialCharTok{!}\FunctionTok{require}\NormalTok{(egvtools)) \{remotes}\SpecialCharTok{::}\FunctionTok{install\_github}\NormalTok{(}\StringTok{"aavotins/egvtools"}\NormalTok{); }\FunctionTok{require}\NormalTok{(egvtools)\}}


\CommentTok{\# Templates {-}{-}{-}{-}{-}}
\NormalTok{template100}\OtherTok{=}\FunctionTok{rast}\NormalTok{(}\StringTok{"./Templates/TemplateRasters/LV100m\_10km.tif"}\NormalTok{)}

\CommentTok{\# radii {-}{-}{-}{-}}
\FunctionTok{radius\_function}\NormalTok{(}
 \AttributeTok{kvadrati\_path =} \StringTok{"./Templates/TemplateGrids/tiles/"}\NormalTok{,}
 \AttributeTok{radii\_path   =} \StringTok{"./Templates/TemplateGridPoints/tiles/"}\NormalTok{,}
 \AttributeTok{tikls100\_path =} \StringTok{"./Templates/TemplateGrids/tikls100\_sauzeme.parquet"}\NormalTok{,}
 \AttributeTok{template\_path =} \StringTok{"./Templates/TemplateRasters/LV100m\_10km.tif"}\NormalTok{,}
 \AttributeTok{input\_layers  =} \FunctionTok{c}\NormalTok{(}\StringTok{"./RasterGrids\_100m/2024/RAW/ForestsSoil\_OligotrophicMineral\_cell.tif"}\NormalTok{),}
 \AttributeTok{layer\_prefixes =} \FunctionTok{c}\NormalTok{(}\StringTok{"ForestsSoil\_OligotrophicMineral"}\NormalTok{),}
 \AttributeTok{output\_dir   =} \StringTok{"./RasterGrids\_100m/2024/RAW/"}\NormalTok{,}
 \AttributeTok{n\_workers   =} \DecValTok{6}\NormalTok{,}
 \AttributeTok{radii     =} \FunctionTok{c}\NormalTok{(}\StringTok{"r500"}\NormalTok{),}
 \AttributeTok{radius\_mode  =} \StringTok{"sparse"}\NormalTok{,}
 \AttributeTok{extract\_fun  =} \StringTok{"mean"}\NormalTok{,}
 \AttributeTok{fill\_missing  =} \ConstantTok{TRUE}\NormalTok{,}
 \AttributeTok{IDW\_weight   =} \DecValTok{2}\NormalTok{,}
 \AttributeTok{future\_max\_size =} \DecValTok{40} \SpecialCharTok{*} \DecValTok{1024}\SpecialCharTok{\^{}}\DecValTok{3}\NormalTok{)}


\CommentTok{\# ForestsSoil\_OligotrophicMineral\_r500.tif  egv\_334}
\NormalTok{slanis}\OtherTok{=}\FunctionTok{rast}\NormalTok{(}\StringTok{"./RasterGrids\_100m/2024/RAW/ForestsSoil\_OligotrophicMineral\_r500.tif"}\NormalTok{)}
\FunctionTok{names}\NormalTok{(slanis)}\OtherTok{=}\StringTok{"egv\_334"}
\NormalTok{slanis2}\OtherTok{=}\FunctionTok{project}\NormalTok{(slanis,template100)}
\FunctionTok{writeRaster}\NormalTok{(slanis2,}
      \StringTok{"./RasterGrids\_100m/2024/RAW/ForestsSoil\_OligotrophicMineral\_r500.tif"}\NormalTok{,}
      \AttributeTok{overwrite=}\ConstantTok{TRUE}\NormalTok{)}

\CommentTok{\# standardisation {-}{-}{-}{-}}
\ControlFlowTok{if}\NormalTok{(}\SpecialCharTok{!}\FunctionTok{require}\NormalTok{(terra)) \{}\FunctionTok{install.packages}\NormalTok{(}\StringTok{"terra"}\NormalTok{); }\FunctionTok{require}\NormalTok{(terra)\}}
\ControlFlowTok{if}\NormalTok{(}\SpecialCharTok{!}\FunctionTok{require}\NormalTok{(tidyverse)) \{}\FunctionTok{install.packages}\NormalTok{(}\StringTok{"tidyverse"}\NormalTok{); }\FunctionTok{require}\NormalTok{(tidyverse)\}}

\NormalTok{nosaukums}\OtherTok{=}\StringTok{"ForestsSoil\_OligotrophicMineral\_r500.tif"}
\NormalTok{ielasisanas\_cels}\OtherTok{=}\FunctionTok{paste0}\NormalTok{(}\StringTok{"./RasterGrids\_100m/2024/RAW/"}\NormalTok{,nosaukums)}
\NormalTok{saglabasanas\_cels}\OtherTok{=}\FunctionTok{paste0}\NormalTok{(}\StringTok{"./RasterGrids\_100m/2024/Scaled/"}\NormalTok{,nosaukums)}
\NormalTok{slanis}\OtherTok{=}\FunctionTok{rast}\NormalTok{(ielasisanas\_cels)}
\NormalTok{videjais}\OtherTok{=}\FunctionTok{global}\NormalTok{(slanis,}\AttributeTok{fun=}\StringTok{"mean"}\NormalTok{,}\AttributeTok{na.rm=}\ConstantTok{TRUE}\NormalTok{)}
\NormalTok{centrets}\OtherTok{=}\NormalTok{slanis}\SpecialCharTok{{-}}\NormalTok{videjais[,}\DecValTok{1}\NormalTok{]}
\NormalTok{standartnovirze}\OtherTok{=}\NormalTok{terra}\SpecialCharTok{::}\FunctionTok{global}\NormalTok{(centrets,}\AttributeTok{fun=}\StringTok{"rms"}\NormalTok{,}\AttributeTok{na.rm=}\ConstantTok{TRUE}\NormalTok{)}
\NormalTok{merogots}\OtherTok{=}\NormalTok{centrets}\SpecialCharTok{/}\NormalTok{standartnovirze[,}\DecValTok{1}\NormalTok{]}
\FunctionTok{writeRaster}\NormalTok{(merogots,}
      \AttributeTok{filename=}\NormalTok{saglabasanas\_cels,}
      \AttributeTok{overwrite=}\ConstantTok{TRUE}\NormalTok{)}
\end{Highlighting}
\end{Shaded}

\section{ForestsSoil\_OligotrophicMineral\_r1250}\label{ch06.335}

\textbf{filename:} \texttt{ForestsSoil\_OligotrophicMineral\_r1250.tif}

\textbf{layername:} \texttt{egv\_335}

\textbf{English name:} Fractional cover of Oligotrophic Forests on undrained Mineral
Soils within the 1.25 km landscape

\textbf{Latvian name:} Oligotrofu mežu nesusinātās minerālaugsnēs platības īpatsvars
1,25 km ainavā

\textbf{Procedure:} The cover fraction within a radius of 1250 m around the analysis grid cell
is calculated as the area-weighted sum of the \hyperref[ch06.333]{analysis cells} inside
the buffer, using the workflow \texttt{egvtools::radius\_function()}. During the calculation of the landscape
metric, inverse distance weighted (power = 2) gap filling on the output is
applied to ensure no missing values at the edges. Then the layer is
rewritten to set its name. Finally, the layer is standardised by
subtracting the arithmetic mean and dividing by the root mean squared error.

\begin{Shaded}
\begin{Highlighting}[]
\CommentTok{\# libs {-}{-}{-}{-}}
\ControlFlowTok{if}\NormalTok{(}\SpecialCharTok{!}\FunctionTok{require}\NormalTok{(terra)) \{}\FunctionTok{install.packages}\NormalTok{(}\StringTok{"terra"}\NormalTok{); }\FunctionTok{require}\NormalTok{(terra)\}}
\ControlFlowTok{if}\NormalTok{(}\SpecialCharTok{!}\FunctionTok{require}\NormalTok{(egvtools)) \{remotes}\SpecialCharTok{::}\FunctionTok{install\_github}\NormalTok{(}\StringTok{"aavotins/egvtools"}\NormalTok{); }\FunctionTok{require}\NormalTok{(egvtools)\}}


\CommentTok{\# Templates {-}{-}{-}{-}{-}}
\NormalTok{template100}\OtherTok{=}\FunctionTok{rast}\NormalTok{(}\StringTok{"./Templates/TemplateRasters/LV100m\_10km.tif"}\NormalTok{)}

\CommentTok{\# radii {-}{-}{-}{-}}
\FunctionTok{radius\_function}\NormalTok{(}
 \AttributeTok{kvadrati\_path =} \StringTok{"./Templates/TemplateGrids/tiles/"}\NormalTok{,}
 \AttributeTok{radii\_path   =} \StringTok{"./Templates/TemplateGridPoints/tiles/"}\NormalTok{,}
 \AttributeTok{tikls100\_path =} \StringTok{"./Templates/TemplateGrids/tikls100\_sauzeme.parquet"}\NormalTok{,}
 \AttributeTok{template\_path =} \StringTok{"./Templates/TemplateRasters/LV100m\_10km.tif"}\NormalTok{,}
 \AttributeTok{input\_layers  =} \FunctionTok{c}\NormalTok{(}\StringTok{"./RasterGrids\_100m/2024/RAW/ForestsSoil\_OligotrophicMineral\_cell.tif"}\NormalTok{),}
 \AttributeTok{layer\_prefixes =} \FunctionTok{c}\NormalTok{(}\StringTok{"ForestsSoil\_OligotrophicMineral"}\NormalTok{),}
 \AttributeTok{output\_dir   =} \StringTok{"./RasterGrids\_100m/2024/RAW/"}\NormalTok{,}
 \AttributeTok{n\_workers   =} \DecValTok{6}\NormalTok{,}
 \AttributeTok{radii     =} \FunctionTok{c}\NormalTok{(}\StringTok{"r1250"}\NormalTok{),}
 \AttributeTok{radius\_mode  =} \StringTok{"sparse"}\NormalTok{,}
 \AttributeTok{extract\_fun  =} \StringTok{"mean"}\NormalTok{,}
 \AttributeTok{fill\_missing  =} \ConstantTok{TRUE}\NormalTok{,}
 \AttributeTok{IDW\_weight   =} \DecValTok{2}\NormalTok{,}
 \AttributeTok{future\_max\_size =} \DecValTok{40} \SpecialCharTok{*} \DecValTok{1024}\SpecialCharTok{\^{}}\DecValTok{3}\NormalTok{)}


\CommentTok{\# ForestsSoil\_OligotrophicMineral\_r1250.tif egv\_335}
\NormalTok{slanis}\OtherTok{=}\FunctionTok{rast}\NormalTok{(}\StringTok{"./RasterGrids\_100m/2024/RAW/ForestsSoil\_OligotrophicMineral\_r1250.tif"}\NormalTok{)}
\FunctionTok{names}\NormalTok{(slanis)}\OtherTok{=}\StringTok{"egv\_335"}
\NormalTok{slanis2}\OtherTok{=}\FunctionTok{project}\NormalTok{(slanis,template100)}
\FunctionTok{writeRaster}\NormalTok{(slanis2,}
      \StringTok{"./RasterGrids\_100m/2024/RAW/ForestsSoil\_OligotrophicMineral\_r1250.tif"}\NormalTok{,}
      \AttributeTok{overwrite=}\ConstantTok{TRUE}\NormalTok{)}

\CommentTok{\# standardisation {-}{-}{-}{-}}
\ControlFlowTok{if}\NormalTok{(}\SpecialCharTok{!}\FunctionTok{require}\NormalTok{(terra)) \{}\FunctionTok{install.packages}\NormalTok{(}\StringTok{"terra"}\NormalTok{); }\FunctionTok{require}\NormalTok{(terra)\}}
\ControlFlowTok{if}\NormalTok{(}\SpecialCharTok{!}\FunctionTok{require}\NormalTok{(tidyverse)) \{}\FunctionTok{install.packages}\NormalTok{(}\StringTok{"tidyverse"}\NormalTok{); }\FunctionTok{require}\NormalTok{(tidyverse)\}}

\NormalTok{nosaukums}\OtherTok{=}\StringTok{"ForestsSoil\_OligotrophicMineral\_r1250.tif"}
\NormalTok{ielasisanas\_cels}\OtherTok{=}\FunctionTok{paste0}\NormalTok{(}\StringTok{"./RasterGrids\_100m/2024/RAW/"}\NormalTok{,nosaukums)}
\NormalTok{saglabasanas\_cels}\OtherTok{=}\FunctionTok{paste0}\NormalTok{(}\StringTok{"./RasterGrids\_100m/2024/Scaled/"}\NormalTok{,nosaukums)}
\NormalTok{slanis}\OtherTok{=}\FunctionTok{rast}\NormalTok{(ielasisanas\_cels)}
\NormalTok{videjais}\OtherTok{=}\FunctionTok{global}\NormalTok{(slanis,}\AttributeTok{fun=}\StringTok{"mean"}\NormalTok{,}\AttributeTok{na.rm=}\ConstantTok{TRUE}\NormalTok{)}
\NormalTok{centrets}\OtherTok{=}\NormalTok{slanis}\SpecialCharTok{{-}}\NormalTok{videjais[,}\DecValTok{1}\NormalTok{]}
\NormalTok{standartnovirze}\OtherTok{=}\NormalTok{terra}\SpecialCharTok{::}\FunctionTok{global}\NormalTok{(centrets,}\AttributeTok{fun=}\StringTok{"rms"}\NormalTok{,}\AttributeTok{na.rm=}\ConstantTok{TRUE}\NormalTok{)}
\NormalTok{merogots}\OtherTok{=}\NormalTok{centrets}\SpecialCharTok{/}\NormalTok{standartnovirze[,}\DecValTok{1}\NormalTok{]}
\FunctionTok{writeRaster}\NormalTok{(merogots,}
      \AttributeTok{filename=}\NormalTok{saglabasanas\_cels,}
      \AttributeTok{overwrite=}\ConstantTok{TRUE}\NormalTok{)}
\end{Highlighting}
\end{Shaded}

\section{ForestsSoil\_OligotrophicMineral\_r3000}\label{ch06.336}

\textbf{filename:} \texttt{ForestsSoil\_OligotrophicMineral\_r3000.tif}

\textbf{layername:} \texttt{egv\_336}

\textbf{English name:} Fractional cover of Oligotrophic Forests on undrained Mineral
Soils within the 3 km landscape

\textbf{Latvian name:} Oligotrofu mežu nesusinātās minerālaugsnēs platības īpatsvars
3 km ainavā

\textbf{Procedure:} The cover fraction within a radius of 3000 m around the analysis grid cell
is calculated as the area-weighted sum of the \hyperref[ch06.333]{analysis cells} inside
the buffer, using the workflow \texttt{egvtools::radius\_function()}. During the calculation of the landscape
metric, inverse distance weighted (power = 2) gap filling on the output is
applied to ensure no missing values at the edges. Then the layer is
rewritten to set its name. Finally, the layer is standardised by
subtracting the arithmetic mean and dividing by the root mean squared error.

\begin{Shaded}
\begin{Highlighting}[]
\CommentTok{\# libs {-}{-}{-}{-}}
\ControlFlowTok{if}\NormalTok{(}\SpecialCharTok{!}\FunctionTok{require}\NormalTok{(terra)) \{}\FunctionTok{install.packages}\NormalTok{(}\StringTok{"terra"}\NormalTok{); }\FunctionTok{require}\NormalTok{(terra)\}}
\ControlFlowTok{if}\NormalTok{(}\SpecialCharTok{!}\FunctionTok{require}\NormalTok{(egvtools)) \{remotes}\SpecialCharTok{::}\FunctionTok{install\_github}\NormalTok{(}\StringTok{"aavotins/egvtools"}\NormalTok{); }\FunctionTok{require}\NormalTok{(egvtools)\}}


\CommentTok{\# Templates {-}{-}{-}{-}{-}}
\NormalTok{template100}\OtherTok{=}\FunctionTok{rast}\NormalTok{(}\StringTok{"./Templates/TemplateRasters/LV100m\_10km.tif"}\NormalTok{)}

\CommentTok{\# radii {-}{-}{-}{-}}
\FunctionTok{radius\_function}\NormalTok{(}
 \AttributeTok{kvadrati\_path =} \StringTok{"./Templates/TemplateGrids/tiles/"}\NormalTok{,}
 \AttributeTok{radii\_path   =} \StringTok{"./Templates/TemplateGridPoints/tiles/"}\NormalTok{,}
 \AttributeTok{tikls100\_path =} \StringTok{"./Templates/TemplateGrids/tikls100\_sauzeme.parquet"}\NormalTok{,}
 \AttributeTok{template\_path =} \StringTok{"./Templates/TemplateRasters/LV100m\_10km.tif"}\NormalTok{,}
 \AttributeTok{input\_layers  =} \FunctionTok{c}\NormalTok{(}\StringTok{"./RasterGrids\_100m/2024/RAW/ForestsSoil\_OligotrophicMineral\_cell.tif"}\NormalTok{),}
 \AttributeTok{layer\_prefixes =} \FunctionTok{c}\NormalTok{(}\StringTok{"ForestsSoil\_OligotrophicMineral"}\NormalTok{),}
 \AttributeTok{output\_dir   =} \StringTok{"./RasterGrids\_100m/2024/RAW/"}\NormalTok{,}
 \AttributeTok{n\_workers   =} \DecValTok{6}\NormalTok{,}
 \AttributeTok{radii     =} \FunctionTok{c}\NormalTok{(}\StringTok{"r3000"}\NormalTok{),}
 \AttributeTok{radius\_mode  =} \StringTok{"sparse"}\NormalTok{,}
 \AttributeTok{extract\_fun  =} \StringTok{"mean"}\NormalTok{,}
 \AttributeTok{fill\_missing  =} \ConstantTok{TRUE}\NormalTok{,}
 \AttributeTok{IDW\_weight   =} \DecValTok{2}\NormalTok{,}
 \AttributeTok{future\_max\_size =} \DecValTok{40} \SpecialCharTok{*} \DecValTok{1024}\SpecialCharTok{\^{}}\DecValTok{3}\NormalTok{)}


\CommentTok{\# ForestsSoil\_OligotrophicMineral\_r3000.tif egv\_336}
\NormalTok{slanis}\OtherTok{=}\FunctionTok{rast}\NormalTok{(}\StringTok{"./RasterGrids\_100m/2024/RAW/ForestsSoil\_OligotrophicMineral\_r3000.tif"}\NormalTok{)}
\FunctionTok{names}\NormalTok{(slanis)}\OtherTok{=}\StringTok{"egv\_336"}
\NormalTok{slanis2}\OtherTok{=}\FunctionTok{project}\NormalTok{(slanis,template100)}
\FunctionTok{writeRaster}\NormalTok{(slanis2,}
      \StringTok{"./RasterGrids\_100m/2024/RAW/ForestsSoil\_OligotrophicMineral\_r3000.tif"}\NormalTok{,}
      \AttributeTok{overwrite=}\ConstantTok{TRUE}\NormalTok{)}

\CommentTok{\# standardisation {-}{-}{-}{-}}
\ControlFlowTok{if}\NormalTok{(}\SpecialCharTok{!}\FunctionTok{require}\NormalTok{(terra)) \{}\FunctionTok{install.packages}\NormalTok{(}\StringTok{"terra"}\NormalTok{); }\FunctionTok{require}\NormalTok{(terra)\}}
\ControlFlowTok{if}\NormalTok{(}\SpecialCharTok{!}\FunctionTok{require}\NormalTok{(tidyverse)) \{}\FunctionTok{install.packages}\NormalTok{(}\StringTok{"tidyverse"}\NormalTok{); }\FunctionTok{require}\NormalTok{(tidyverse)\}}

\NormalTok{nosaukums}\OtherTok{=}\StringTok{"ForestsSoil\_OligotrophicMineral\_r3000.tif"}
\NormalTok{ielasisanas\_cels}\OtherTok{=}\FunctionTok{paste0}\NormalTok{(}\StringTok{"./RasterGrids\_100m/2024/RAW/"}\NormalTok{,nosaukums)}
\NormalTok{saglabasanas\_cels}\OtherTok{=}\FunctionTok{paste0}\NormalTok{(}\StringTok{"./RasterGrids\_100m/2024/Scaled/"}\NormalTok{,nosaukums)}
\NormalTok{slanis}\OtherTok{=}\FunctionTok{rast}\NormalTok{(ielasisanas\_cels)}
\NormalTok{videjais}\OtherTok{=}\FunctionTok{global}\NormalTok{(slanis,}\AttributeTok{fun=}\StringTok{"mean"}\NormalTok{,}\AttributeTok{na.rm=}\ConstantTok{TRUE}\NormalTok{)}
\NormalTok{centrets}\OtherTok{=}\NormalTok{slanis}\SpecialCharTok{{-}}\NormalTok{videjais[,}\DecValTok{1}\NormalTok{]}
\NormalTok{standartnovirze}\OtherTok{=}\NormalTok{terra}\SpecialCharTok{::}\FunctionTok{global}\NormalTok{(centrets,}\AttributeTok{fun=}\StringTok{"rms"}\NormalTok{,}\AttributeTok{na.rm=}\ConstantTok{TRUE}\NormalTok{)}
\NormalTok{merogots}\OtherTok{=}\NormalTok{centrets}\SpecialCharTok{/}\NormalTok{standartnovirze[,}\DecValTok{1}\NormalTok{]}
\FunctionTok{writeRaster}\NormalTok{(merogots,}
      \AttributeTok{filename=}\NormalTok{saglabasanas\_cels,}
      \AttributeTok{overwrite=}\ConstantTok{TRUE}\NormalTok{)}
\end{Highlighting}
\end{Shaded}

\section{ForestsSoil\_OligotrophicMineral\_r10000}\label{ch06.337}

\textbf{filename:} \texttt{ForestsSoil\_OligotrophicMineral\_r10000.tif}

\textbf{layername:} \texttt{egv\_337}

\textbf{English name:} Fractional cover of Oligotrophic Forests on undrained Mineral
Soils within the 10 km landscape

\textbf{Latvian name:} Oligotrofu mežu nesusinātās minerālaugsnēs platības īpatsvars
10 km ainavā

\textbf{Procedure:} The cover fraction within a radius of 10000 m around the analysis grid cell
is calculated as the area-weighted sum of the \hyperref[ch06.333]{analysis cells} inside
the buffer, using the workflow \texttt{egvtools::radius\_function()}. During the calculation of the landscape
metric, inverse distance weighted (power = 2) gap filling on the output is
applied to ensure no missing values at the edges. Then the layer is
rewritten to set its name. Finally, the layer is standardised by
subtracting the arithmetic mean and dividing by the root mean squared error.

\begin{Shaded}
\begin{Highlighting}[]
\CommentTok{\# libs {-}{-}{-}{-}}
\ControlFlowTok{if}\NormalTok{(}\SpecialCharTok{!}\FunctionTok{require}\NormalTok{(terra)) \{}\FunctionTok{install.packages}\NormalTok{(}\StringTok{"terra"}\NormalTok{); }\FunctionTok{require}\NormalTok{(terra)\}}
\ControlFlowTok{if}\NormalTok{(}\SpecialCharTok{!}\FunctionTok{require}\NormalTok{(egvtools)) \{remotes}\SpecialCharTok{::}\FunctionTok{install\_github}\NormalTok{(}\StringTok{"aavotins/egvtools"}\NormalTok{); }\FunctionTok{require}\NormalTok{(egvtools)\}}


\CommentTok{\# Templates {-}{-}{-}{-}{-}}
\NormalTok{template100}\OtherTok{=}\FunctionTok{rast}\NormalTok{(}\StringTok{"./Templates/TemplateRasters/LV100m\_10km.tif"}\NormalTok{)}

\CommentTok{\# radii {-}{-}{-}{-}}
\FunctionTok{radius\_function}\NormalTok{(}
 \AttributeTok{kvadrati\_path =} \StringTok{"./Templates/TemplateGrids/tiles/"}\NormalTok{,}
 \AttributeTok{radii\_path   =} \StringTok{"./Templates/TemplateGridPoints/tiles/"}\NormalTok{,}
 \AttributeTok{tikls100\_path =} \StringTok{"./Templates/TemplateGrids/tikls100\_sauzeme.parquet"}\NormalTok{,}
 \AttributeTok{template\_path =} \StringTok{"./Templates/TemplateRasters/LV100m\_10km.tif"}\NormalTok{,}
 \AttributeTok{input\_layers  =} \FunctionTok{c}\NormalTok{(}\StringTok{"./RasterGrids\_100m/2024/RAW/ForestsSoil\_OligotrophicMineral\_cell.tif"}\NormalTok{),}
 \AttributeTok{layer\_prefixes =} \FunctionTok{c}\NormalTok{(}\StringTok{"ForestsSoil\_OligotrophicMineral"}\NormalTok{),}
 \AttributeTok{output\_dir   =} \StringTok{"./RasterGrids\_100m/2024/RAW/"}\NormalTok{,}
 \AttributeTok{n\_workers   =} \DecValTok{6}\NormalTok{,}
 \AttributeTok{radii     =} \FunctionTok{c}\NormalTok{(}\StringTok{"r10000"}\NormalTok{),}
 \AttributeTok{radius\_mode  =} \StringTok{"sparse"}\NormalTok{,}
 \AttributeTok{extract\_fun  =} \StringTok{"mean"}\NormalTok{,}
 \AttributeTok{fill\_missing  =} \ConstantTok{TRUE}\NormalTok{,}
 \AttributeTok{IDW\_weight   =} \DecValTok{2}\NormalTok{,}
 \AttributeTok{future\_max\_size =} \DecValTok{40} \SpecialCharTok{*} \DecValTok{1024}\SpecialCharTok{\^{}}\DecValTok{3}\NormalTok{)}


\CommentTok{\# ForestsSoil\_OligotrophicMineral\_r10000.tif    egv\_337}
\NormalTok{slanis}\OtherTok{=}\FunctionTok{rast}\NormalTok{(}\StringTok{"./RasterGrids\_100m/2024/RAW/ForestsSoil\_OligotrophicMineral\_r10000.tif"}\NormalTok{)}
\FunctionTok{names}\NormalTok{(slanis)}\OtherTok{=}\StringTok{"egv\_337"}
\NormalTok{slanis2}\OtherTok{=}\FunctionTok{project}\NormalTok{(slanis,template100)}
\FunctionTok{writeRaster}\NormalTok{(slanis2,}
      \StringTok{"./RasterGrids\_100m/2024/RAW/ForestsSoil\_OligotrophicMineral\_r10000.tif"}\NormalTok{,}
      \AttributeTok{overwrite=}\ConstantTok{TRUE}\NormalTok{)}

\CommentTok{\# standardisation {-}{-}{-}{-}}
\ControlFlowTok{if}\NormalTok{(}\SpecialCharTok{!}\FunctionTok{require}\NormalTok{(terra)) \{}\FunctionTok{install.packages}\NormalTok{(}\StringTok{"terra"}\NormalTok{); }\FunctionTok{require}\NormalTok{(terra)\}}
\ControlFlowTok{if}\NormalTok{(}\SpecialCharTok{!}\FunctionTok{require}\NormalTok{(tidyverse)) \{}\FunctionTok{install.packages}\NormalTok{(}\StringTok{"tidyverse"}\NormalTok{); }\FunctionTok{require}\NormalTok{(tidyverse)\}}

\NormalTok{nosaukums}\OtherTok{=}\StringTok{"ForestsSoil\_OligotrophicMineral\_r10000.tif"}
\NormalTok{ielasisanas\_cels}\OtherTok{=}\FunctionTok{paste0}\NormalTok{(}\StringTok{"./RasterGrids\_100m/2024/RAW/"}\NormalTok{,nosaukums)}
\NormalTok{saglabasanas\_cels}\OtherTok{=}\FunctionTok{paste0}\NormalTok{(}\StringTok{"./RasterGrids\_100m/2024/Scaled/"}\NormalTok{,nosaukums)}
\NormalTok{slanis}\OtherTok{=}\FunctionTok{rast}\NormalTok{(ielasisanas\_cels)}
\NormalTok{videjais}\OtherTok{=}\FunctionTok{global}\NormalTok{(slanis,}\AttributeTok{fun=}\StringTok{"mean"}\NormalTok{,}\AttributeTok{na.rm=}\ConstantTok{TRUE}\NormalTok{)}
\NormalTok{centrets}\OtherTok{=}\NormalTok{slanis}\SpecialCharTok{{-}}\NormalTok{videjais[,}\DecValTok{1}\NormalTok{]}
\NormalTok{standartnovirze}\OtherTok{=}\NormalTok{terra}\SpecialCharTok{::}\FunctionTok{global}\NormalTok{(centrets,}\AttributeTok{fun=}\StringTok{"rms"}\NormalTok{,}\AttributeTok{na.rm=}\ConstantTok{TRUE}\NormalTok{)}
\NormalTok{merogots}\OtherTok{=}\NormalTok{centrets}\SpecialCharTok{/}\NormalTok{standartnovirze[,}\DecValTok{1}\NormalTok{]}
\FunctionTok{writeRaster}\NormalTok{(merogots,}
      \AttributeTok{filename=}\NormalTok{saglabasanas\_cels,}
      \AttributeTok{overwrite=}\ConstantTok{TRUE}\NormalTok{)}
\end{Highlighting}
\end{Shaded}

\section{ForestsSoil\_OligotrophicOrganic\_cell}\label{ch06.338}

\textbf{filename:} \texttt{ForestsSoil\_OligotrophicOrganic\_cell.tif}

\textbf{layername:} \texttt{egv\_338}

\textbf{English name:} Fractional cover of Oligotrophic Forests on undrained Organic
Soils within the analysis cell (1 ha)

\textbf{Latvian name:} Oligotrofu mežu nesusinātās organiskajās augsnēs platības
īpatsvars analīzes šūnā (1 ha)

\textbf{Procedure:} To prepare this EGV, forest stands with forest type equal to ``12''
or ``14'' are selected from the \hyperref[Ch04.01]{State Forest Service's State Forest
Registry} and rasterised. Rasterisation is performed using
the workflow \texttt{egvtools::polygon2input()} with background
covering (value 0). The resulting layer
is then aggregated to EGV resolution using the workflow \texttt{egvtools::input2egv()}, which
calculates the arithmetic mean to determine the cover fraction. During
aggregation, inverse distance weighted (power = 2) gap filling on the output is
applied to ensure no missing values at the edges. Finally, the layer is
standardised by subtracting the arithmetic mean and dividing by the root mean squared
error.

\begin{Shaded}
\begin{Highlighting}[]
\CommentTok{\# libs {-}{-}{-}{-}}
\ControlFlowTok{if}\NormalTok{(}\SpecialCharTok{!}\FunctionTok{require}\NormalTok{(egvtools)) \{remotes}\SpecialCharTok{::}\FunctionTok{install\_github}\NormalTok{(}\StringTok{"aavotins/egvtools"}\NormalTok{); }\FunctionTok{require}\NormalTok{(egvtools)\}}
\ControlFlowTok{if}\NormalTok{(}\SpecialCharTok{!}\FunctionTok{require}\NormalTok{(terra)) \{}\FunctionTok{install.packages}\NormalTok{(}\StringTok{"terra"}\NormalTok{); }\FunctionTok{require}\NormalTok{(terra)\}}
\ControlFlowTok{if}\NormalTok{(}\SpecialCharTok{!}\FunctionTok{require}\NormalTok{(sf)) \{}\FunctionTok{install.packages}\NormalTok{(}\StringTok{"sf"}\NormalTok{); }\FunctionTok{require}\NormalTok{(sf)\}}
\ControlFlowTok{if}\NormalTok{(}\SpecialCharTok{!}\FunctionTok{require}\NormalTok{(tidyverse)) \{}\FunctionTok{install.packages}\NormalTok{(}\StringTok{"tidyverse"}\NormalTok{); }\FunctionTok{require}\NormalTok{(tidyverse)\}}
\ControlFlowTok{if}\NormalTok{(}\SpecialCharTok{!}\FunctionTok{require}\NormalTok{(sfarrow)) \{}\FunctionTok{install.packages}\NormalTok{(}\StringTok{"sfarrow"}\NormalTok{); }\FunctionTok{require}\NormalTok{(sfarrow)\}}
\ControlFlowTok{if}\NormalTok{(}\SpecialCharTok{!}\FunctionTok{require}\NormalTok{(readxl)) \{}\FunctionTok{install.packages}\NormalTok{(}\StringTok{"readxl"}\NormalTok{); }\FunctionTok{require}\NormalTok{(readxl)\}}
\ControlFlowTok{if}\NormalTok{(}\SpecialCharTok{!}\FunctionTok{require}\NormalTok{(raster)) \{}\FunctionTok{install.packages}\NormalTok{(}\StringTok{"raster"}\NormalTok{); }\FunctionTok{require}\NormalTok{(raster)\}}
\ControlFlowTok{if}\NormalTok{(}\SpecialCharTok{!}\FunctionTok{require}\NormalTok{(fasterize)) \{}\FunctionTok{install.packages}\NormalTok{(}\StringTok{"fasterize"}\NormalTok{); }\FunctionTok{require}\NormalTok{(fasterize)\}}

\CommentTok{\# templates {-}{-}{-}{-}}
\NormalTok{template100}\OtherTok{=}\FunctionTok{rast}\NormalTok{(}\StringTok{"./Templates/TemplateRasters/LV100m\_10km.tif"}\NormalTok{)}
\NormalTok{template10}\OtherTok{=}\FunctionTok{rast}\NormalTok{(}\StringTok{"./Templates/TemplateRasters/LV10m\_10km.tif"}\NormalTok{)}
\NormalTok{rastrs10}\OtherTok{=}\FunctionTok{raster}\NormalTok{(template10)}

\NormalTok{nulls10}\OtherTok{=}\FunctionTok{rast}\NormalTok{(}\StringTok{"./Templates/TemplateRasters/nulls\_LV10m\_10km.tif"}\NormalTok{)}
\NormalTok{nulls100}\OtherTok{=}\FunctionTok{rast}\NormalTok{(}\StringTok{"./Templates/TemplateRasters/nulls\_LV100m\_10km.tif"}\NormalTok{)}


\CommentTok{\# simple landscape {-}{-}{-}{-}}
\NormalTok{simple\_landscape}\OtherTok{=}\FunctionTok{rast}\NormalTok{(}\StringTok{"RasterGrids\_10m/2024/Ainava\_vienk\_mask.tif"}\NormalTok{)}

\CommentTok{\# mvr {-}{-}{-}{-}}
\NormalTok{mvr}\OtherTok{=}\FunctionTok{st\_read\_parquet}\NormalTok{(}\StringTok{"./Geodata/2024/MVR/nogabali\_2024janv.parquet"}\NormalTok{)}
\NormalTok{mvr}\SpecialCharTok{$}\NormalTok{yes}\OtherTok{=}\DecValTok{1}


\CommentTok{\# ForestsSoil\_OligotrophicOrganic\_cell.tif  egv\_338 {-}{-}{-}{-}}
\NormalTok{OligotrophicOrganic}\OtherTok{=}\NormalTok{mvr }\SpecialCharTok{\%\textgreater{}\%} 
 \FunctionTok{filter}\NormalTok{(mt }\SpecialCharTok{\%in\%} \FunctionTok{c}\NormalTok{(}\StringTok{"12"}\NormalTok{,}\StringTok{"14"}\NormalTok{))}
\NormalTok{p2i\_rez}\OtherTok{=}\NormalTok{egvtools}\SpecialCharTok{::}\FunctionTok{polygon2input}\NormalTok{(}\AttributeTok{vector\_data =}\NormalTok{ OligotrophicOrganic,}
                \AttributeTok{template\_path =} \StringTok{"./Templates/TemplateRasters/LV10m\_10km.tif"}\NormalTok{,}
                \AttributeTok{out\_path =} \StringTok{"./RasterGrids\_10m/2024/"}\NormalTok{,}
                \AttributeTok{file\_name =} \StringTok{"ForestsSoil\_OligotrophicOrganic\_input.tif"}\NormalTok{,}
                \AttributeTok{value\_field =} \StringTok{"yes"}\NormalTok{,}
                \AttributeTok{prepare=}\ConstantTok{FALSE}\NormalTok{,}
                \AttributeTok{background\_raster =} \StringTok{"./Templates/TemplateRasters/nulls\_LV10m\_10km.tif"}\NormalTok{,}
                \AttributeTok{plot\_result =} \ConstantTok{TRUE}\NormalTok{)}
\NormalTok{p2i\_rez}
\NormalTok{i2e\_rez}\OtherTok{=}\NormalTok{egvtools}\SpecialCharTok{::}\FunctionTok{input2egv}\NormalTok{(}\AttributeTok{input=}\FunctionTok{paste0}\NormalTok{(}\StringTok{"./RasterGrids\_10m/2024/"}\NormalTok{,}
                     \StringTok{"ForestsSoil\_OligotrophicOrganic\_input.tif"}\NormalTok{),}
              \AttributeTok{egv\_template=} \StringTok{"./Templates/TemplateRasters/LV100m\_10km.tif"}\NormalTok{,}
              \AttributeTok{summary\_function =} \StringTok{"average"}\NormalTok{,}
              \AttributeTok{missing\_job =} \StringTok{"FillOutput"}\NormalTok{,}
              \AttributeTok{outlocation =} \StringTok{"./RasterGrids\_100m/2024/RAW/"}\NormalTok{,}
              \AttributeTok{outfilename =} \StringTok{"ForestsSoil\_OligotrophicOrganic\_cell.tif"}\NormalTok{,}
              \AttributeTok{layername =} \StringTok{"egv\_338"}\NormalTok{,}
              \AttributeTok{idw\_weight =} \DecValTok{2}\NormalTok{,}
              \AttributeTok{plot\_gaps =} \ConstantTok{FALSE}\NormalTok{,}\AttributeTok{plot\_final =} \ConstantTok{TRUE}\NormalTok{)}
\NormalTok{i2e\_rez}
\FunctionTok{rm}\NormalTok{(OligotrophicOrganic)}
\FunctionTok{rm}\NormalTok{(p2i\_rez)}
\FunctionTok{rm}\NormalTok{(i2e\_rez)}
\FunctionTok{unlink}\NormalTok{(}\StringTok{"./RasterGrids\_10m/2024/ForestsSoil\_OligotrophicOrganic\_input.tif"}\NormalTok{)}

\CommentTok{\# standardisation {-}{-}{-}{-}}
\ControlFlowTok{if}\NormalTok{(}\SpecialCharTok{!}\FunctionTok{require}\NormalTok{(terra)) \{}\FunctionTok{install.packages}\NormalTok{(}\StringTok{"terra"}\NormalTok{); }\FunctionTok{require}\NormalTok{(terra)\}}
\ControlFlowTok{if}\NormalTok{(}\SpecialCharTok{!}\FunctionTok{require}\NormalTok{(tidyverse)) \{}\FunctionTok{install.packages}\NormalTok{(}\StringTok{"tidyverse"}\NormalTok{); }\FunctionTok{require}\NormalTok{(tidyverse)\}}

\NormalTok{nosaukums}\OtherTok{=}\StringTok{"ForestsSoil\_OligotrophicMineral\_cell.tif"}
\NormalTok{ielasisanas\_cels}\OtherTok{=}\FunctionTok{paste0}\NormalTok{(}\StringTok{"./RasterGrids\_100m/2024/RAW/"}\NormalTok{,nosaukums)}
\NormalTok{saglabasanas\_cels}\OtherTok{=}\FunctionTok{paste0}\NormalTok{(}\StringTok{"./RasterGrids\_100m/2024/Scaled/"}\NormalTok{,nosaukums)}
\NormalTok{slanis}\OtherTok{=}\FunctionTok{rast}\NormalTok{(ielasisanas\_cels)}
\NormalTok{videjais}\OtherTok{=}\FunctionTok{global}\NormalTok{(slanis,}\AttributeTok{fun=}\StringTok{"mean"}\NormalTok{,}\AttributeTok{na.rm=}\ConstantTok{TRUE}\NormalTok{)}
\NormalTok{centrets}\OtherTok{=}\NormalTok{slanis}\SpecialCharTok{{-}}\NormalTok{videjais[,}\DecValTok{1}\NormalTok{]}
\NormalTok{standartnovirze}\OtherTok{=}\NormalTok{terra}\SpecialCharTok{::}\FunctionTok{global}\NormalTok{(centrets,}\AttributeTok{fun=}\StringTok{"rms"}\NormalTok{,}\AttributeTok{na.rm=}\ConstantTok{TRUE}\NormalTok{)}
\NormalTok{merogots}\OtherTok{=}\NormalTok{centrets}\SpecialCharTok{/}\NormalTok{standartnovirze[,}\DecValTok{1}\NormalTok{]}
\FunctionTok{writeRaster}\NormalTok{(merogots,}
      \AttributeTok{filename=}\NormalTok{saglabasanas\_cels,}
      \AttributeTok{overwrite=}\ConstantTok{TRUE}\NormalTok{)}
\end{Highlighting}
\end{Shaded}

\section{ForestsSoil\_OligotrophicOrganic\_r500}\label{ch06.339}

\textbf{filename:} \texttt{ForestsSoil\_OligotrophicOrganic\_r500.tif}

\textbf{layername:} \texttt{egv\_339}

\textbf{English name:} Fractional cover of Oligotrophic Forests on undrained Organic
Soils within the 0.5 km landscape

\textbf{Latvian name:} Oligotrofu mežu nesusinātās organiskajās augsnēs platības
īpatsvars 0,5 km ainavā

\textbf{Procedure:} The cover fraction within a radius of 500 m around the analysis grid cell is
calculated as the area-weighted sum of the \hyperref[ch06.338]{analysis cells} inside the
buffer, using the workflow \texttt{egvtools::radius\_function()}. During the calculation of the landscape metric,
inverse distance weighted (power = 2) gap filling on the output is applied
to ensure no missing values at the edges. Then the layer is rewritten to set
its name. Finally, the layer is standardised by subtracting the arithmetic
mean and dividing by the root mean squared error.

\begin{Shaded}
\begin{Highlighting}[]
\CommentTok{\# libs {-}{-}{-}{-}}
\ControlFlowTok{if}\NormalTok{(}\SpecialCharTok{!}\FunctionTok{require}\NormalTok{(terra)) \{}\FunctionTok{install.packages}\NormalTok{(}\StringTok{"terra"}\NormalTok{); }\FunctionTok{require}\NormalTok{(terra)\}}
\ControlFlowTok{if}\NormalTok{(}\SpecialCharTok{!}\FunctionTok{require}\NormalTok{(egvtools)) \{remotes}\SpecialCharTok{::}\FunctionTok{install\_github}\NormalTok{(}\StringTok{"aavotins/egvtools"}\NormalTok{); }\FunctionTok{require}\NormalTok{(egvtools)\}}


\CommentTok{\# Templates {-}{-}{-}{-}{-}}
\NormalTok{template100}\OtherTok{=}\FunctionTok{rast}\NormalTok{(}\StringTok{"./Templates/TemplateRasters/LV100m\_10km.tif"}\NormalTok{)}

\CommentTok{\# radii {-}{-}{-}{-}}
\FunctionTok{radius\_function}\NormalTok{(}
 \AttributeTok{kvadrati\_path =} \StringTok{"./Templates/TemplateGrids/tiles/"}\NormalTok{,}
 \AttributeTok{radii\_path   =} \StringTok{"./Templates/TemplateGridPoints/tiles/"}\NormalTok{,}
 \AttributeTok{tikls100\_path =} \StringTok{"./Templates/TemplateGrids/tikls100\_sauzeme.parquet"}\NormalTok{,}
 \AttributeTok{template\_path =} \StringTok{"./Templates/TemplateRasters/LV100m\_10km.tif"}\NormalTok{,}
 \AttributeTok{input\_layers  =} \FunctionTok{c}\NormalTok{(}\StringTok{"./RasterGrids\_100m/2024/RAW/ForestsSoil\_OligotrophicOrganic\_cell.tif"}\NormalTok{),}
 \AttributeTok{layer\_prefixes =} \FunctionTok{c}\NormalTok{(}\StringTok{"ForestsSoil\_OligotrophicOrganic"}\NormalTok{),}
 \AttributeTok{output\_dir   =} \StringTok{"./RasterGrids\_100m/2024/RAW/"}\NormalTok{,}
 \AttributeTok{n\_workers   =} \DecValTok{6}\NormalTok{,}
 \AttributeTok{radii     =} \FunctionTok{c}\NormalTok{(}\StringTok{"r500"}\NormalTok{),}
 \AttributeTok{radius\_mode  =} \StringTok{"sparse"}\NormalTok{,}
 \AttributeTok{extract\_fun  =} \StringTok{"mean"}\NormalTok{,}
 \AttributeTok{fill\_missing  =} \ConstantTok{TRUE}\NormalTok{,}
 \AttributeTok{IDW\_weight   =} \DecValTok{2}\NormalTok{,}
 \AttributeTok{future\_max\_size =} \DecValTok{40} \SpecialCharTok{*} \DecValTok{1024}\SpecialCharTok{\^{}}\DecValTok{3}\NormalTok{)}


\CommentTok{\# ForestsSoil\_OligotrophicOrganic\_r500.tif  egv\_339}
\NormalTok{slanis}\OtherTok{=}\FunctionTok{rast}\NormalTok{(}\StringTok{"./RasterGrids\_100m/2024/RAW/ForestsSoil\_OligotrophicOrganic\_r500.tif"}\NormalTok{)}
\FunctionTok{names}\NormalTok{(slanis)}\OtherTok{=}\StringTok{"egv\_339"}
\NormalTok{slanis2}\OtherTok{=}\FunctionTok{project}\NormalTok{(slanis,template100)}
\FunctionTok{writeRaster}\NormalTok{(slanis2,}
      \StringTok{"./RasterGrids\_100m/2024/RAW/ForestsSoil\_OligotrophicOrganic\_r500.tif"}\NormalTok{,}
      \AttributeTok{overwrite=}\ConstantTok{TRUE}\NormalTok{)}

\CommentTok{\# standardisation {-}{-}{-}{-}}
\ControlFlowTok{if}\NormalTok{(}\SpecialCharTok{!}\FunctionTok{require}\NormalTok{(terra)) \{}\FunctionTok{install.packages}\NormalTok{(}\StringTok{"terra"}\NormalTok{); }\FunctionTok{require}\NormalTok{(terra)\}}
\ControlFlowTok{if}\NormalTok{(}\SpecialCharTok{!}\FunctionTok{require}\NormalTok{(tidyverse)) \{}\FunctionTok{install.packages}\NormalTok{(}\StringTok{"tidyverse"}\NormalTok{); }\FunctionTok{require}\NormalTok{(tidyverse)\}}

\NormalTok{nosaukums}\OtherTok{=}\StringTok{"ForestsSoil\_OligotrophicOrganic\_r500.tif"}
\NormalTok{ielasisanas\_cels}\OtherTok{=}\FunctionTok{paste0}\NormalTok{(}\StringTok{"./RasterGrids\_100m/2024/RAW/"}\NormalTok{,nosaukums)}
\NormalTok{saglabasanas\_cels}\OtherTok{=}\FunctionTok{paste0}\NormalTok{(}\StringTok{"./RasterGrids\_100m/2024/Scaled/"}\NormalTok{,nosaukums)}
\NormalTok{slanis}\OtherTok{=}\FunctionTok{rast}\NormalTok{(ielasisanas\_cels)}
\NormalTok{videjais}\OtherTok{=}\FunctionTok{global}\NormalTok{(slanis,}\AttributeTok{fun=}\StringTok{"mean"}\NormalTok{,}\AttributeTok{na.rm=}\ConstantTok{TRUE}\NormalTok{)}
\NormalTok{centrets}\OtherTok{=}\NormalTok{slanis}\SpecialCharTok{{-}}\NormalTok{videjais[,}\DecValTok{1}\NormalTok{]}
\NormalTok{standartnovirze}\OtherTok{=}\NormalTok{terra}\SpecialCharTok{::}\FunctionTok{global}\NormalTok{(centrets,}\AttributeTok{fun=}\StringTok{"rms"}\NormalTok{,}\AttributeTok{na.rm=}\ConstantTok{TRUE}\NormalTok{)}
\NormalTok{merogots}\OtherTok{=}\NormalTok{centrets}\SpecialCharTok{/}\NormalTok{standartnovirze[,}\DecValTok{1}\NormalTok{]}
\FunctionTok{writeRaster}\NormalTok{(merogots,}
      \AttributeTok{filename=}\NormalTok{saglabasanas\_cels,}
      \AttributeTok{overwrite=}\ConstantTok{TRUE}\NormalTok{)}
\end{Highlighting}
\end{Shaded}

\section{ForestsSoil\_OligotrophicOrganic\_r1250}\label{ch06.340}

\textbf{filename:} \texttt{ForestsSoil\_OligotrophicOrganic\_r1250.tif}

\textbf{layername:} \texttt{egv\_340}

\textbf{English name:} Fractional cover of Oligotrophic Forests on undrained Organic
Soils within the 1.25 km landscape

\textbf{Latvian name:} Oligotrofu mežu nesusinātās organiskajās augsnēs platības
īpatsvars 1,25 km ainavā

\textbf{Procedure:} The cover fraction within a radius of 1250 m around the analysis grid cell
is calculated as the area-weighted sum of the \hyperref[ch06.338]{analysis cells} inside
the buffer, using the workflow \texttt{egvtools::radius\_function()}. During the calculation of the landscape
metric, inverse distance weighted (power = 2) gap filling on the output is
applied to ensure no missing values at the edges. Then the layer is
rewritten to set its name. Finally, the layer is standardised by
subtracting the arithmetic mean and dividing by the root mean squared error.

\begin{Shaded}
\begin{Highlighting}[]
\CommentTok{\# libs {-}{-}{-}{-}}
\ControlFlowTok{if}\NormalTok{(}\SpecialCharTok{!}\FunctionTok{require}\NormalTok{(terra)) \{}\FunctionTok{install.packages}\NormalTok{(}\StringTok{"terra"}\NormalTok{); }\FunctionTok{require}\NormalTok{(terra)\}}
\ControlFlowTok{if}\NormalTok{(}\SpecialCharTok{!}\FunctionTok{require}\NormalTok{(egvtools)) \{remotes}\SpecialCharTok{::}\FunctionTok{install\_github}\NormalTok{(}\StringTok{"aavotins/egvtools"}\NormalTok{); }\FunctionTok{require}\NormalTok{(egvtools)\}}


\CommentTok{\# Templates {-}{-}{-}{-}{-}}
\NormalTok{template100}\OtherTok{=}\FunctionTok{rast}\NormalTok{(}\StringTok{"./Templates/TemplateRasters/LV100m\_10km.tif"}\NormalTok{)}

\CommentTok{\# radii {-}{-}{-}{-}}
\FunctionTok{radius\_function}\NormalTok{(}
 \AttributeTok{kvadrati\_path =} \StringTok{"./Templates/TemplateGrids/tiles/"}\NormalTok{,}
 \AttributeTok{radii\_path   =} \StringTok{"./Templates/TemplateGridPoints/tiles/"}\NormalTok{,}
 \AttributeTok{tikls100\_path =} \StringTok{"./Templates/TemplateGrids/tikls100\_sauzeme.parquet"}\NormalTok{,}
 \AttributeTok{template\_path =} \StringTok{"./Templates/TemplateRasters/LV100m\_10km.tif"}\NormalTok{,}
 \AttributeTok{input\_layers  =} \FunctionTok{c}\NormalTok{(}\StringTok{"./RasterGrids\_100m/2024/RAW/ForestsSoil\_OligotrophicOrganic\_cell.tif"}\NormalTok{),}
 \AttributeTok{layer\_prefixes =} \FunctionTok{c}\NormalTok{(}\StringTok{"ForestsSoil\_OligotrophicOrganic"}\NormalTok{),}
 \AttributeTok{output\_dir   =} \StringTok{"./RasterGrids\_100m/2024/RAW/"}\NormalTok{,}
 \AttributeTok{n\_workers   =} \DecValTok{6}\NormalTok{,}
 \AttributeTok{radii     =} \FunctionTok{c}\NormalTok{(}\StringTok{"r1250"}\NormalTok{),}
 \AttributeTok{radius\_mode  =} \StringTok{"sparse"}\NormalTok{,}
 \AttributeTok{extract\_fun  =} \StringTok{"mean"}\NormalTok{,}
 \AttributeTok{fill\_missing  =} \ConstantTok{TRUE}\NormalTok{,}
 \AttributeTok{IDW\_weight   =} \DecValTok{2}\NormalTok{,}
 \AttributeTok{future\_max\_size =} \DecValTok{40} \SpecialCharTok{*} \DecValTok{1024}\SpecialCharTok{\^{}}\DecValTok{3}\NormalTok{)}


\CommentTok{\# ForestsSoil\_OligotrophicOrganic\_r1250.tif egv\_340}
\NormalTok{slanis}\OtherTok{=}\FunctionTok{rast}\NormalTok{(}\StringTok{"./RasterGrids\_100m/2024/RAW/ForestsSoil\_OligotrophicOrganic\_r1250.tif"}\NormalTok{)}
\FunctionTok{names}\NormalTok{(slanis)}\OtherTok{=}\StringTok{"egv\_340"}
\NormalTok{slanis2}\OtherTok{=}\FunctionTok{project}\NormalTok{(slanis,template100)}
\FunctionTok{writeRaster}\NormalTok{(slanis2,}
      \StringTok{"./RasterGrids\_100m/2024/RAW/ForestsSoil\_OligotrophicOrganic\_r1250.tif"}\NormalTok{,}
      \AttributeTok{overwrite=}\ConstantTok{TRUE}\NormalTok{)}

\CommentTok{\# standardisation {-}{-}{-}{-}}
\ControlFlowTok{if}\NormalTok{(}\SpecialCharTok{!}\FunctionTok{require}\NormalTok{(terra)) \{}\FunctionTok{install.packages}\NormalTok{(}\StringTok{"terra"}\NormalTok{); }\FunctionTok{require}\NormalTok{(terra)\}}
\ControlFlowTok{if}\NormalTok{(}\SpecialCharTok{!}\FunctionTok{require}\NormalTok{(tidyverse)) \{}\FunctionTok{install.packages}\NormalTok{(}\StringTok{"tidyverse"}\NormalTok{); }\FunctionTok{require}\NormalTok{(tidyverse)\}}

\NormalTok{nosaukums}\OtherTok{=}\StringTok{"ForestsSoil\_OligotrophicOrganic\_r1250.tif"}
\NormalTok{ielasisanas\_cels}\OtherTok{=}\FunctionTok{paste0}\NormalTok{(}\StringTok{"./RasterGrids\_100m/2024/RAW/"}\NormalTok{,nosaukums)}
\NormalTok{saglabasanas\_cels}\OtherTok{=}\FunctionTok{paste0}\NormalTok{(}\StringTok{"./RasterGrids\_100m/2024/Scaled/"}\NormalTok{,nosaukums)}
\NormalTok{slanis}\OtherTok{=}\FunctionTok{rast}\NormalTok{(ielasisanas\_cels)}
\NormalTok{videjais}\OtherTok{=}\FunctionTok{global}\NormalTok{(slanis,}\AttributeTok{fun=}\StringTok{"mean"}\NormalTok{,}\AttributeTok{na.rm=}\ConstantTok{TRUE}\NormalTok{)}
\NormalTok{centrets}\OtherTok{=}\NormalTok{slanis}\SpecialCharTok{{-}}\NormalTok{videjais[,}\DecValTok{1}\NormalTok{]}
\NormalTok{standartnovirze}\OtherTok{=}\NormalTok{terra}\SpecialCharTok{::}\FunctionTok{global}\NormalTok{(centrets,}\AttributeTok{fun=}\StringTok{"rms"}\NormalTok{,}\AttributeTok{na.rm=}\ConstantTok{TRUE}\NormalTok{)}
\NormalTok{merogots}\OtherTok{=}\NormalTok{centrets}\SpecialCharTok{/}\NormalTok{standartnovirze[,}\DecValTok{1}\NormalTok{]}
\FunctionTok{writeRaster}\NormalTok{(merogots,}
      \AttributeTok{filename=}\NormalTok{saglabasanas\_cels,}
      \AttributeTok{overwrite=}\ConstantTok{TRUE}\NormalTok{)}
\end{Highlighting}
\end{Shaded}

\section{ForestsSoil\_OligotrophicOrganic\_r3000}\label{ch06.341}

\textbf{filename:} \texttt{ForestsSoil\_OligotrophicOrganic\_r3000.tif}

\textbf{layername:} \texttt{egv\_341}

\textbf{English name:} Fractional cover of Oligotrophic Forests on undrained Organic
Soils within the 3 km landscape

\textbf{Latvian name:} Oligotrofu mežu nesusinātās organiskajās augsnēs platības
īpatsvars 3 km ainavā

\textbf{Procedure:} The cover fraction within a radius of 3000 m around the analysis grid cell
is calculated as the area-weighted sum of the \hyperref[ch06.338]{analysis cells} inside
the buffer, using the workflow \texttt{egvtools::radius\_function()}. During the calculation of the landscape
metric, inverse distance weighted (power = 2) gap filling on the output is
applied to ensure no missing values at the edges. Then the layer is
rewritten to set its name. Finally, the layer is standardised by
subtracting the arithmetic mean and dividing by the root mean squared error.

\begin{Shaded}
\begin{Highlighting}[]
\CommentTok{\# libs {-}{-}{-}{-}}
\ControlFlowTok{if}\NormalTok{(}\SpecialCharTok{!}\FunctionTok{require}\NormalTok{(terra)) \{}\FunctionTok{install.packages}\NormalTok{(}\StringTok{"terra"}\NormalTok{); }\FunctionTok{require}\NormalTok{(terra)\}}
\ControlFlowTok{if}\NormalTok{(}\SpecialCharTok{!}\FunctionTok{require}\NormalTok{(egvtools)) \{remotes}\SpecialCharTok{::}\FunctionTok{install\_github}\NormalTok{(}\StringTok{"aavotins/egvtools"}\NormalTok{); }\FunctionTok{require}\NormalTok{(egvtools)\}}


\CommentTok{\# Templates {-}{-}{-}{-}{-}}
\NormalTok{template100}\OtherTok{=}\FunctionTok{rast}\NormalTok{(}\StringTok{"./Templates/TemplateRasters/LV100m\_10km.tif"}\NormalTok{)}

\CommentTok{\# radii {-}{-}{-}{-}}
\FunctionTok{radius\_function}\NormalTok{(}
 \AttributeTok{kvadrati\_path =} \StringTok{"./Templates/TemplateGrids/tiles/"}\NormalTok{,}
 \AttributeTok{radii\_path   =} \StringTok{"./Templates/TemplateGridPoints/tiles/"}\NormalTok{,}
 \AttributeTok{tikls100\_path =} \StringTok{"./Templates/TemplateGrids/tikls100\_sauzeme.parquet"}\NormalTok{,}
 \AttributeTok{template\_path =} \StringTok{"./Templates/TemplateRasters/LV100m\_10km.tif"}\NormalTok{,}
 \AttributeTok{input\_layers  =} \FunctionTok{c}\NormalTok{(}\StringTok{"./RasterGrids\_100m/2024/RAW/ForestsSoil\_OligotrophicOrganic\_cell.tif"}\NormalTok{),}
 \AttributeTok{layer\_prefixes =} \FunctionTok{c}\NormalTok{(}\StringTok{"ForestsSoil\_OligotrophicOrganic"}\NormalTok{),}
 \AttributeTok{output\_dir   =} \StringTok{"./RasterGrids\_100m/2024/RAW/"}\NormalTok{,}
 \AttributeTok{n\_workers   =} \DecValTok{6}\NormalTok{,}
 \AttributeTok{radii     =} \FunctionTok{c}\NormalTok{(}\StringTok{"r3000"}\NormalTok{),}
 \AttributeTok{radius\_mode  =} \StringTok{"sparse"}\NormalTok{,}
 \AttributeTok{extract\_fun  =} \StringTok{"mean"}\NormalTok{,}
 \AttributeTok{fill\_missing  =} \ConstantTok{TRUE}\NormalTok{,}
 \AttributeTok{IDW\_weight   =} \DecValTok{2}\NormalTok{,}
 \AttributeTok{future\_max\_size =} \DecValTok{40} \SpecialCharTok{*} \DecValTok{1024}\SpecialCharTok{\^{}}\DecValTok{3}\NormalTok{)}


\CommentTok{\# ForestsSoil\_OligotrophicOrganic\_r3000.tif egv\_341}
\NormalTok{slanis}\OtherTok{=}\FunctionTok{rast}\NormalTok{(}\StringTok{"./RasterGrids\_100m/2024/RAW/ForestsSoil\_OligotrophicOrganic\_r3000.tif"}\NormalTok{)}
\FunctionTok{names}\NormalTok{(slanis)}\OtherTok{=}\StringTok{"egv\_341"}
\NormalTok{slanis2}\OtherTok{=}\FunctionTok{project}\NormalTok{(slanis,template100)}
\FunctionTok{writeRaster}\NormalTok{(slanis2,}
      \StringTok{"./RasterGrids\_100m/2024/RAW/ForestsSoil\_OligotrophicOrganic\_r3000.tif"}\NormalTok{,}
      \AttributeTok{overwrite=}\ConstantTok{TRUE}\NormalTok{)}

\CommentTok{\# standardisation {-}{-}{-}{-}}
\ControlFlowTok{if}\NormalTok{(}\SpecialCharTok{!}\FunctionTok{require}\NormalTok{(terra)) \{}\FunctionTok{install.packages}\NormalTok{(}\StringTok{"terra"}\NormalTok{); }\FunctionTok{require}\NormalTok{(terra)\}}
\ControlFlowTok{if}\NormalTok{(}\SpecialCharTok{!}\FunctionTok{require}\NormalTok{(tidyverse)) \{}\FunctionTok{install.packages}\NormalTok{(}\StringTok{"tidyverse"}\NormalTok{); }\FunctionTok{require}\NormalTok{(tidyverse)\}}

\NormalTok{nosaukums}\OtherTok{=}\StringTok{"ForestsSoil\_OligotrophicOrganic\_r3000.tif"}
\NormalTok{ielasisanas\_cels}\OtherTok{=}\FunctionTok{paste0}\NormalTok{(}\StringTok{"./RasterGrids\_100m/2024/RAW/"}\NormalTok{,nosaukums)}
\NormalTok{saglabasanas\_cels}\OtherTok{=}\FunctionTok{paste0}\NormalTok{(}\StringTok{"./RasterGrids\_100m/2024/Scaled/"}\NormalTok{,nosaukums)}
\NormalTok{slanis}\OtherTok{=}\FunctionTok{rast}\NormalTok{(ielasisanas\_cels)}
\NormalTok{videjais}\OtherTok{=}\FunctionTok{global}\NormalTok{(slanis,}\AttributeTok{fun=}\StringTok{"mean"}\NormalTok{,}\AttributeTok{na.rm=}\ConstantTok{TRUE}\NormalTok{)}
\NormalTok{centrets}\OtherTok{=}\NormalTok{slanis}\SpecialCharTok{{-}}\NormalTok{videjais[,}\DecValTok{1}\NormalTok{]}
\NormalTok{standartnovirze}\OtherTok{=}\NormalTok{terra}\SpecialCharTok{::}\FunctionTok{global}\NormalTok{(centrets,}\AttributeTok{fun=}\StringTok{"rms"}\NormalTok{,}\AttributeTok{na.rm=}\ConstantTok{TRUE}\NormalTok{)}
\NormalTok{merogots}\OtherTok{=}\NormalTok{centrets}\SpecialCharTok{/}\NormalTok{standartnovirze[,}\DecValTok{1}\NormalTok{]}
\FunctionTok{writeRaster}\NormalTok{(merogots,}
      \AttributeTok{filename=}\NormalTok{saglabasanas\_cels,}
      \AttributeTok{overwrite=}\ConstantTok{TRUE}\NormalTok{)}
\end{Highlighting}
\end{Shaded}

\section{ForestsSoil\_OligotrophicOrganic\_r10000}\label{ch06.342}

\textbf{filename:} \texttt{ForestsSoil\_OligotrophicOrganic\_r10000.tif}

\textbf{layername:} \texttt{egv\_342}

\textbf{English name:} Fractional cover of Oligotrophic Forests on undrained Organic
Soils within the 10 km landscape

\textbf{Latvian name:} Oligotrofu mežu nesusinātās organiskajās augsnēs platības
īpatsvars 10 km ainavā

\textbf{Procedure:} The cover fraction within a radius of 10000 m around the analysis grid cell
is calculated as the area-weighted sum of the \hyperref[ch06.338]{analysis cells} inside
the buffer, using the workflow \texttt{egvtools::radius\_function()}. During the calculation of the landscape
metric, inverse distance weighted (power = 2) gap filling on the output is
applied to ensure no missing values at the edges. Then the layer is
rewritten to set its name. Finally, the layer is standardised by
subtracting the arithmetic mean and dividing by the root mean squared error.

\begin{Shaded}
\begin{Highlighting}[]
\CommentTok{\# libs {-}{-}{-}{-}}
\ControlFlowTok{if}\NormalTok{(}\SpecialCharTok{!}\FunctionTok{require}\NormalTok{(terra)) \{}\FunctionTok{install.packages}\NormalTok{(}\StringTok{"terra"}\NormalTok{); }\FunctionTok{require}\NormalTok{(terra)\}}
\ControlFlowTok{if}\NormalTok{(}\SpecialCharTok{!}\FunctionTok{require}\NormalTok{(egvtools)) \{remotes}\SpecialCharTok{::}\FunctionTok{install\_github}\NormalTok{(}\StringTok{"aavotins/egvtools"}\NormalTok{); }\FunctionTok{require}\NormalTok{(egvtools)\}}


\CommentTok{\# Templates {-}{-}{-}{-}{-}}
\NormalTok{template100}\OtherTok{=}\FunctionTok{rast}\NormalTok{(}\StringTok{"./Templates/TemplateRasters/LV100m\_10km.tif"}\NormalTok{)}

\CommentTok{\# radii {-}{-}{-}{-}}
\FunctionTok{radius\_function}\NormalTok{(}
 \AttributeTok{kvadrati\_path =} \StringTok{"./Templates/TemplateGrids/tiles/"}\NormalTok{,}
 \AttributeTok{radii\_path   =} \StringTok{"./Templates/TemplateGridPoints/tiles/"}\NormalTok{,}
 \AttributeTok{tikls100\_path =} \StringTok{"./Templates/TemplateGrids/tikls100\_sauzeme.parquet"}\NormalTok{,}
 \AttributeTok{template\_path =} \StringTok{"./Templates/TemplateRasters/LV100m\_10km.tif"}\NormalTok{,}
 \AttributeTok{input\_layers  =} \FunctionTok{c}\NormalTok{(}\StringTok{"./RasterGrids\_100m/2024/RAW/ForestsSoil\_OligotrophicOrganic\_cell.tif"}\NormalTok{),}
 \AttributeTok{layer\_prefixes =} \FunctionTok{c}\NormalTok{(}\StringTok{"ForestsSoil\_OligotrophicOrganic"}\NormalTok{),}
 \AttributeTok{output\_dir   =} \StringTok{"./RasterGrids\_100m/2024/RAW/"}\NormalTok{,}
 \AttributeTok{n\_workers   =} \DecValTok{6}\NormalTok{,}
 \AttributeTok{radii     =} \FunctionTok{c}\NormalTok{(}\StringTok{"r10000"}\NormalTok{),}
 \AttributeTok{radius\_mode  =} \StringTok{"sparse"}\NormalTok{,}
 \AttributeTok{extract\_fun  =} \StringTok{"mean"}\NormalTok{,}
 \AttributeTok{fill\_missing  =} \ConstantTok{TRUE}\NormalTok{,}
 \AttributeTok{IDW\_weight   =} \DecValTok{2}\NormalTok{,}
 \AttributeTok{future\_max\_size =} \DecValTok{40} \SpecialCharTok{*} \DecValTok{1024}\SpecialCharTok{\^{}}\DecValTok{3}\NormalTok{)}


\CommentTok{\# ForestsSoil\_OligotrophicOrganic\_r10000.tif    egv\_342}
\NormalTok{slanis}\OtherTok{=}\FunctionTok{rast}\NormalTok{(}\StringTok{"./RasterGrids\_100m/2024/RAW/ForestsSoil\_OligotrophicOrganic\_r10000.tif"}\NormalTok{)}
\FunctionTok{names}\NormalTok{(slanis)}\OtherTok{=}\StringTok{"egv\_342"}
\NormalTok{slanis2}\OtherTok{=}\FunctionTok{project}\NormalTok{(slanis,template100)}
\FunctionTok{writeRaster}\NormalTok{(slanis2,}
      \StringTok{"./RasterGrids\_100m/2024/RAW/ForestsSoil\_OligotrophicOrganic\_r10000.tif"}\NormalTok{,}
      \AttributeTok{overwrite=}\ConstantTok{TRUE}\NormalTok{)}

\CommentTok{\# standardisation {-}{-}{-}{-}}
\ControlFlowTok{if}\NormalTok{(}\SpecialCharTok{!}\FunctionTok{require}\NormalTok{(terra)) \{}\FunctionTok{install.packages}\NormalTok{(}\StringTok{"terra"}\NormalTok{); }\FunctionTok{require}\NormalTok{(terra)\}}
\ControlFlowTok{if}\NormalTok{(}\SpecialCharTok{!}\FunctionTok{require}\NormalTok{(tidyverse)) \{}\FunctionTok{install.packages}\NormalTok{(}\StringTok{"tidyverse"}\NormalTok{); }\FunctionTok{require}\NormalTok{(tidyverse)\}}

\NormalTok{nosaukums}\OtherTok{=}\StringTok{"ForestsSoil\_OligotrophicOrganic\_r10000.tif"}
\NormalTok{ielasisanas\_cels}\OtherTok{=}\FunctionTok{paste0}\NormalTok{(}\StringTok{"./RasterGrids\_100m/2024/RAW/"}\NormalTok{,nosaukums)}
\NormalTok{saglabasanas\_cels}\OtherTok{=}\FunctionTok{paste0}\NormalTok{(}\StringTok{"./RasterGrids\_100m/2024/Scaled/"}\NormalTok{,nosaukums)}
\NormalTok{slanis}\OtherTok{=}\FunctionTok{rast}\NormalTok{(ielasisanas\_cels)}
\NormalTok{videjais}\OtherTok{=}\FunctionTok{global}\NormalTok{(slanis,}\AttributeTok{fun=}\StringTok{"mean"}\NormalTok{,}\AttributeTok{na.rm=}\ConstantTok{TRUE}\NormalTok{)}
\NormalTok{centrets}\OtherTok{=}\NormalTok{slanis}\SpecialCharTok{{-}}\NormalTok{videjais[,}\DecValTok{1}\NormalTok{]}
\NormalTok{standartnovirze}\OtherTok{=}\NormalTok{terra}\SpecialCharTok{::}\FunctionTok{global}\NormalTok{(centrets,}\AttributeTok{fun=}\StringTok{"rms"}\NormalTok{,}\AttributeTok{na.rm=}\ConstantTok{TRUE}\NormalTok{)}
\NormalTok{merogots}\OtherTok{=}\NormalTok{centrets}\SpecialCharTok{/}\NormalTok{standartnovirze[,}\DecValTok{1}\NormalTok{]}
\FunctionTok{writeRaster}\NormalTok{(merogots,}
      \AttributeTok{filename=}\NormalTok{saglabasanas\_cels,}
      \AttributeTok{overwrite=}\ConstantTok{TRUE}\NormalTok{)}
\end{Highlighting}
\end{Shaded}

\section{ForestsTreesAge\_BorealDeciduousOld\_cell}\label{ch06.343}

\textbf{filename:} \texttt{ForestsTreesAge\_BorealDeciduousOld\_cell.tif}

\textbf{layername:} \texttt{egv\_343}

\textbf{English name:} Fractional cover of Old (over rotation age) Boreal Deciduous
Forests within the analysis cell (1 ha)

\textbf{Latvian name:} Vecu (kopš cirtmeta) šaurlapju mežu platības īpatsvars
analīzes šūnā (1 ha)

\textbf{Procedure:} Most EGVs describing forests are spatially restricted to areas outside
of clearcuts and dead stands. This mask is created using a combination of
the \hyperref[Ch04.01]{State Forest Service's
State Forest Registry} land category 12 and 14, and \hyperref[Ch04.09]{The
Global Forest Watch} pixels classified as lost tree canopy cover since
2020 (raster layer matching input, presence = 1, absence = 0).

To prepare this EGV, stands from the \hyperref[Ch04.01]{State Forest Service's State Forest
Registry} are classified into (in order):

\begin{itemize}
\item
  coniferous (see \hyperref[Ch01]{Terminology and acronyms} for species codes) if
  timber volume of those species exceeded 75\%;
\item
  Boreal deciduous if timber volume of those species exceeded 75\%;
\item
  temperate deciduous if timber volume of those species exceeded 50\%;
\item
  mixed otherwise;
\end{itemize}

then Boreal deciduous stands exceeding the legal rotation age are selected and
geometries are rasterised (presence = 1, NA otherwise). Rasterisation is
performed using the workflow \texttt{egvtools::polygon2input()}, restricting to pixels outside clearcut
mask and covering background with value 0. The resulting layer
is then aggregated to EGV resolution using the workflow \texttt{egvtools::input2egv()}, which
calculates the arithmetic mean to determine the cover fraction. During
aggregation, inverse distance weighted (power = 2) gap filling on the output is
applied to ensure no missing values at the edges. Finally, the layer is
standardised by subtracting the arithmetic mean and dividing by the root mean squared
error.

\begin{Shaded}
\begin{Highlighting}[]
\CommentTok{\# libs {-}{-}{-}{-}}
\ControlFlowTok{if}\NormalTok{(}\SpecialCharTok{!}\FunctionTok{require}\NormalTok{(egvtools)) \{remotes}\SpecialCharTok{::}\FunctionTok{install\_github}\NormalTok{(}\StringTok{"aavotins/egvtools"}\NormalTok{); }\FunctionTok{require}\NormalTok{(egvtools)\}}
\ControlFlowTok{if}\NormalTok{(}\SpecialCharTok{!}\FunctionTok{require}\NormalTok{(terra)) \{}\FunctionTok{install.packages}\NormalTok{(}\StringTok{"terra"}\NormalTok{); }\FunctionTok{require}\NormalTok{(terra)\}}
\ControlFlowTok{if}\NormalTok{(}\SpecialCharTok{!}\FunctionTok{require}\NormalTok{(sf)) \{}\FunctionTok{install.packages}\NormalTok{(}\StringTok{"sf"}\NormalTok{); }\FunctionTok{require}\NormalTok{(sf)\}}
\ControlFlowTok{if}\NormalTok{(}\SpecialCharTok{!}\FunctionTok{require}\NormalTok{(tidyverse)) \{}\FunctionTok{install.packages}\NormalTok{(}\StringTok{"tidyverse"}\NormalTok{); }\FunctionTok{require}\NormalTok{(tidyverse)\}}
\ControlFlowTok{if}\NormalTok{(}\SpecialCharTok{!}\FunctionTok{require}\NormalTok{(sfarrow)) \{}\FunctionTok{install.packages}\NormalTok{(}\StringTok{"sfarrow"}\NormalTok{); }\FunctionTok{require}\NormalTok{(sfarrow)\}}
\ControlFlowTok{if}\NormalTok{(}\SpecialCharTok{!}\FunctionTok{require}\NormalTok{(readxl)) \{}\FunctionTok{install.packages}\NormalTok{(}\StringTok{"readxl"}\NormalTok{); }\FunctionTok{require}\NormalTok{(readxl)\}}
\ControlFlowTok{if}\NormalTok{(}\SpecialCharTok{!}\FunctionTok{require}\NormalTok{(raster)) \{}\FunctionTok{install.packages}\NormalTok{(}\StringTok{"raster"}\NormalTok{); }\FunctionTok{require}\NormalTok{(raster)\}}
\ControlFlowTok{if}\NormalTok{(}\SpecialCharTok{!}\FunctionTok{require}\NormalTok{(fasterize)) \{}\FunctionTok{install.packages}\NormalTok{(}\StringTok{"fasterize"}\NormalTok{); }\FunctionTok{require}\NormalTok{(fasterize)\}}

\CommentTok{\# templates {-}{-}{-}{-}}
\NormalTok{template100}\OtherTok{=}\FunctionTok{rast}\NormalTok{(}\StringTok{"./Templates/TemplateRasters/LV100m\_10km.tif"}\NormalTok{)}
\NormalTok{template10}\OtherTok{=}\FunctionTok{rast}\NormalTok{(}\StringTok{"./Templates/TemplateRasters/LV10m\_10km.tif"}\NormalTok{)}
\NormalTok{rastrs10}\OtherTok{=}\FunctionTok{raster}\NormalTok{(template10)}

\NormalTok{nulls10}\OtherTok{=}\FunctionTok{rast}\NormalTok{(}\StringTok{"./Templates/TemplateRasters/nulls\_LV10m\_10km.tif"}\NormalTok{)}
\NormalTok{nulls100}\OtherTok{=}\FunctionTok{rast}\NormalTok{(}\StringTok{"./Templates/TemplateRasters/nulls\_LV100m\_10km.tif"}\NormalTok{)}


\CommentTok{\# simple landscape {-}{-}{-}{-}}
\NormalTok{simple\_landscape}\OtherTok{=}\FunctionTok{rast}\NormalTok{(}\StringTok{"RasterGrids\_10m/2024/Ainava\_vienk\_mask.tif"}\NormalTok{)}

\CommentTok{\# mvr {-}{-}{-}{-}}
\NormalTok{mvr}\OtherTok{=}\FunctionTok{st\_read\_parquet}\NormalTok{(}\StringTok{"./Geodata/2024/MVR/nogabali\_2024janv.parquet"}\NormalTok{)}
\NormalTok{mvr}\SpecialCharTok{$}\NormalTok{yes}\OtherTok{=}\DecValTok{1}

\CommentTok{\# clear cut mask {-}{-}{-}{-}}
\NormalTok{izcirtumi}\OtherTok{=}\NormalTok{mvr }\SpecialCharTok{\%\textgreater{}\%} 
 \FunctionTok{filter}\NormalTok{(zkat }\SpecialCharTok{\%in\%} \FunctionTok{c}\NormalTok{(}\StringTok{"12"}\NormalTok{,}\StringTok{"14"}\NormalTok{)) }\SpecialCharTok{\%\textgreater{}\%} 
\NormalTok{ dplyr}\SpecialCharTok{::}\FunctionTok{select}\NormalTok{(yes)}
\NormalTok{r\_izcirtumi\_mvr}\OtherTok{=}\FunctionTok{fasterize}\NormalTok{(izcirtumi,rastrs10,}\AttributeTok{field=}\StringTok{"yes"}\NormalTok{)}
\NormalTok{t\_izcirtumi\_mvr}\OtherTok{=}\FunctionTok{rast}\NormalTok{(r\_izcirtumi\_mvr)}
\FunctionTok{plot}\NormalTok{(t\_izcirtumi\_mvr)}

\NormalTok{tcl}\OtherTok{=}\FunctionTok{rast}\NormalTok{(}\StringTok{"./Geodata/2024/Trees/GFW/TreeCoverLoss\_v1\_12.tif"}\NormalTok{)}
\NormalTok{tcl2}\OtherTok{=}\FunctionTok{ifel}\NormalTok{(tcl}\SpecialCharTok{\textless{}}\DecValTok{20}\NormalTok{,}\DecValTok{0}\NormalTok{,}\DecValTok{1}\NormalTok{)}
\NormalTok{tclX}\OtherTok{=}\FunctionTok{cover}\NormalTok{(tcl2,nulls10)}
\FunctionTok{plot}\NormalTok{(tclX)}

\NormalTok{clearcut\_mask}\OtherTok{=}\FunctionTok{cover}\NormalTok{(t\_izcirtumi\_mvr,tclX,}
          \AttributeTok{filename=}\StringTok{"./RasterGrids\_10m/2024/Mask\_clearcuts.tif"}\NormalTok{,}
          \AttributeTok{overwrite=}\ConstantTok{TRUE}\NormalTok{)}
\FunctionTok{plot}\NormalTok{(clearcut\_mask)}

\FunctionTok{rm}\NormalTok{(izcirtumi)}
\FunctionTok{rm}\NormalTok{(r\_izcirtumi\_mvr)}
\FunctionTok{rm}\NormalTok{(t\_izcirtumi\_mvr)}
\FunctionTok{rm}\NormalTok{(tcl)}
\FunctionTok{rm}\NormalTok{(tcl2)}
\FunctionTok{rm}\NormalTok{(tclX)}

\CommentTok{\# ForestsTreesAge\_BorealDeciduousOld\_cell.tif   egv\_343 {-}{-}{-}{-}}
\NormalTok{skujkoki}\OtherTok{=}\FunctionTok{c}\NormalTok{(}\StringTok{"1"}\NormalTok{,}\StringTok{"3"}\NormalTok{,}\StringTok{"13"}\NormalTok{,}\StringTok{"14"}\NormalTok{,}\StringTok{"15"}\NormalTok{,}\StringTok{"22"}\NormalTok{,}\StringTok{"23"}\NormalTok{,}\StringTok{"28"}\NormalTok{) }\CommentTok{\# 8}
\NormalTok{saurlapji}\OtherTok{=}\FunctionTok{c}\NormalTok{(}\StringTok{"4"}\NormalTok{,}\StringTok{"6"}\NormalTok{,}\StringTok{"8"}\NormalTok{,}\StringTok{"9"}\NormalTok{,}\StringTok{"19"}\NormalTok{,}\StringTok{"20"}\NormalTok{,}\StringTok{"21"}\NormalTok{,}\StringTok{"32"}\NormalTok{,}\StringTok{"35"}\NormalTok{,}\StringTok{"68"}\NormalTok{) }\CommentTok{\# 10}
\NormalTok{platlapji}\OtherTok{=}\FunctionTok{c}\NormalTok{(}\StringTok{"10"}\NormalTok{,}\StringTok{"11"}\NormalTok{,}\StringTok{"12"}\NormalTok{,}\StringTok{"16"}\NormalTok{,}\StringTok{"17"}\NormalTok{,}\StringTok{"18"}\NormalTok{,}\StringTok{"24"}\NormalTok{,}\StringTok{"25"}\NormalTok{,}\StringTok{"26"}\NormalTok{,}\StringTok{"27"}\NormalTok{,}\StringTok{"28"}\NormalTok{,}\StringTok{"29"}\NormalTok{,}\StringTok{"50"}\NormalTok{,}
      \StringTok{"61"}\NormalTok{,}\StringTok{"62"}\NormalTok{,}\StringTok{"63"}\NormalTok{,}\StringTok{"64"}\NormalTok{,}\StringTok{"65"}\NormalTok{,}\StringTok{"66"}\NormalTok{,}\StringTok{"67"}\NormalTok{,}\StringTok{"69"}\NormalTok{) }\CommentTok{\# 21}
\NormalTok{mvr}\OtherTok{=}\NormalTok{mvr }\SpecialCharTok{\%\textgreater{}\%} 
 \FunctionTok{mutate}\NormalTok{(}\AttributeTok{kraja\_skujkoku=}\FunctionTok{ifelse}\NormalTok{(s10 }\SpecialCharTok{\%in\%}\NormalTok{ skujkoki,v10,}\DecValTok{0}\NormalTok{)}\SpecialCharTok{+}
      \FunctionTok{ifelse}\NormalTok{(s11 }\SpecialCharTok{\%in\%}\NormalTok{ skujkoki,v11,}\DecValTok{0}\NormalTok{)}\SpecialCharTok{+}\FunctionTok{ifelse}\NormalTok{(s12 }\SpecialCharTok{\%in\%}\NormalTok{ skujkoki,v12,}\DecValTok{0}\NormalTok{)}\SpecialCharTok{+}
      \FunctionTok{ifelse}\NormalTok{(s13 }\SpecialCharTok{\%in\%}\NormalTok{ skujkoki,v13,}\DecValTok{0}\NormalTok{)}\SpecialCharTok{+}\FunctionTok{ifelse}\NormalTok{(s14 }\SpecialCharTok{\%in\%}\NormalTok{ skujkoki,v14,}\DecValTok{0}\NormalTok{),}
     \AttributeTok{kraja\_saurlapju=}\FunctionTok{ifelse}\NormalTok{(s10 }\SpecialCharTok{\%in\%}\NormalTok{ saurlapji,v10,}\DecValTok{0}\NormalTok{)}\SpecialCharTok{+}
      \FunctionTok{ifelse}\NormalTok{(s11 }\SpecialCharTok{\%in\%}\NormalTok{ saurlapji,v11,}\DecValTok{0}\NormalTok{)}\SpecialCharTok{+}\FunctionTok{ifelse}\NormalTok{(s12 }\SpecialCharTok{\%in\%}\NormalTok{ saurlapji,v12,}\DecValTok{0}\NormalTok{)}\SpecialCharTok{+}
      \FunctionTok{ifelse}\NormalTok{(s13 }\SpecialCharTok{\%in\%}\NormalTok{ saurlapji,v13,}\DecValTok{0}\NormalTok{)}\SpecialCharTok{+}\FunctionTok{ifelse}\NormalTok{(s14 }\SpecialCharTok{\%in\%}\NormalTok{ saurlapji,v14,}\DecValTok{0}\NormalTok{),}
     \AttributeTok{kraja\_platlapju=}\FunctionTok{ifelse}\NormalTok{(s10 }\SpecialCharTok{\%in\%}\NormalTok{ platlapji,v10,}\DecValTok{0}\NormalTok{)}\SpecialCharTok{+}
      \FunctionTok{ifelse}\NormalTok{(s11 }\SpecialCharTok{\%in\%}\NormalTok{ platlapji,v11,}\DecValTok{0}\NormalTok{)}\SpecialCharTok{+}\FunctionTok{ifelse}\NormalTok{(s12 }\SpecialCharTok{\%in\%}\NormalTok{ platlapji,v12,}\DecValTok{0}\NormalTok{)}\SpecialCharTok{+}
      \FunctionTok{ifelse}\NormalTok{(s13 }\SpecialCharTok{\%in\%}\NormalTok{ platlapji,v13,}\DecValTok{0}\NormalTok{)}\SpecialCharTok{+}\FunctionTok{ifelse}\NormalTok{(s14 }\SpecialCharTok{\%in\%}\NormalTok{ platlapji,v14,}\DecValTok{0}\NormalTok{)) }\SpecialCharTok{\%\textgreater{}\%} 
 \FunctionTok{mutate}\NormalTok{(}\AttributeTok{kopeja\_kraja=}\NormalTok{kraja\_skujkoku}\SpecialCharTok{+}\NormalTok{kraja\_platlapju}\SpecialCharTok{+}\NormalTok{kraja\_saurlapju) }\SpecialCharTok{\%\textgreater{}\%} 
 \FunctionTok{mutate}\NormalTok{(}\AttributeTok{tips=}\FunctionTok{ifelse}\NormalTok{(kraja\_skujkoku}\SpecialCharTok{/}\NormalTok{kopeja\_kraja}\SpecialCharTok{\textgreater{}=}\FloatTok{0.75}\NormalTok{,}\StringTok{"skujkoku"}\NormalTok{,}
           \FunctionTok{ifelse}\NormalTok{(kraja\_saurlapju}\SpecialCharTok{/}\NormalTok{kopeja\_kraja}\SpecialCharTok{\textgreater{}=}\FloatTok{0.75}\NormalTok{,}\StringTok{"saurlapju"}\NormalTok{,}
              \FunctionTok{ifelse}\NormalTok{(kraja\_platlapju}\SpecialCharTok{/}\NormalTok{kopeja\_kraja}\SpecialCharTok{\textgreater{}}\FloatTok{0.5}\NormalTok{,}\StringTok{"platlapju"}\NormalTok{,}
                  \StringTok{"jauktu koku"}\NormalTok{))))}
\NormalTok{nogabali}\OtherTok{=}\NormalTok{mvr }\SpecialCharTok{\%\textgreater{}\%} 
 \FunctionTok{filter}\NormalTok{(zkat}\SpecialCharTok{==}\StringTok{"10"}\SpecialCharTok{\&}\NormalTok{tips}\SpecialCharTok{==}\StringTok{"saurlapju"}\SpecialCharTok{\&}\NormalTok{(vgr}\SpecialCharTok{==}\StringTok{"4"}\SpecialCharTok{|}\NormalTok{vgr}\SpecialCharTok{==}\StringTok{"5"}\NormalTok{))}

\NormalTok{p2i\_rez}\OtherTok{=}\NormalTok{egvtools}\SpecialCharTok{::}\FunctionTok{polygon2input}\NormalTok{(}\AttributeTok{vector\_data =}\NormalTok{ nogabali,}
                \AttributeTok{template\_path =} \StringTok{"./Templates/TemplateRasters/LV10m\_10km.tif"}\NormalTok{,}
                \AttributeTok{out\_path =} \StringTok{"./RasterGrids\_10m/2024/"}\NormalTok{,}
                \AttributeTok{file\_name =} \StringTok{"ForestsTreesAge\_BorealDeciduousOld\_input.tif"}\NormalTok{,}
                \AttributeTok{value\_field =} \StringTok{"yes"}\NormalTok{,}
                \AttributeTok{restrict\_to =}\NormalTok{ clearcut\_mask,}
                \AttributeTok{restrict\_values =} \DecValTok{0}\NormalTok{,}
                \AttributeTok{prepare=}\ConstantTok{FALSE}\NormalTok{,}
                \AttributeTok{background\_raster =} \StringTok{"./Templates/TemplateRasters/nulls\_LV10m\_10km.tif"}\NormalTok{,}
                \AttributeTok{plot\_result =} \ConstantTok{TRUE}\NormalTok{)}
\NormalTok{p2i\_rez}
\NormalTok{i2e\_rez}\OtherTok{=}\NormalTok{egvtools}\SpecialCharTok{::}\FunctionTok{input2egv}\NormalTok{(}\AttributeTok{input=}\FunctionTok{paste0}\NormalTok{(}\StringTok{"./RasterGrids\_10m/2024/"}\NormalTok{,}
                     \StringTok{"ForestsTreesAge\_BorealDeciduousOld\_input.tif"}\NormalTok{),}
              \AttributeTok{egv\_template=} \StringTok{"./Templates/TemplateRasters/LV100m\_10km.tif"}\NormalTok{,}
              \AttributeTok{summary\_function =} \StringTok{"average"}\NormalTok{,}
              \AttributeTok{missing\_job =} \StringTok{"FillOutput"}\NormalTok{,}
              \AttributeTok{outlocation =} \StringTok{"./RasterGrids\_100m/2024/RAW/"}\NormalTok{,}
              \AttributeTok{outfilename =} \StringTok{"ForestsTreesAge\_BorealDeciduousOld\_cell.tif"}\NormalTok{,}
              \AttributeTok{layername =} \StringTok{"egv\_343"}\NormalTok{,}
              \AttributeTok{idw\_weight =} \DecValTok{2}\NormalTok{,}
              \AttributeTok{plot\_gaps =} \ConstantTok{FALSE}\NormalTok{,}\AttributeTok{plot\_final =} \ConstantTok{TRUE}\NormalTok{)}
\NormalTok{i2e\_rez}
\FunctionTok{rm}\NormalTok{(nogabali)}
\FunctionTok{rm}\NormalTok{(p2i\_rez)}
\FunctionTok{rm}\NormalTok{(i2e\_rez)}
\FunctionTok{unlink}\NormalTok{(}\StringTok{"./RasterGrids\_10m/2024/ForestsTreesAge\_BorealDeciduousOld\_input.tif"}\NormalTok{)}

\CommentTok{\# standardisation {-}{-}{-}{-}}
\ControlFlowTok{if}\NormalTok{(}\SpecialCharTok{!}\FunctionTok{require}\NormalTok{(terra)) \{}\FunctionTok{install.packages}\NormalTok{(}\StringTok{"terra"}\NormalTok{); }\FunctionTok{require}\NormalTok{(terra)\}}
\ControlFlowTok{if}\NormalTok{(}\SpecialCharTok{!}\FunctionTok{require}\NormalTok{(tidyverse)) \{}\FunctionTok{install.packages}\NormalTok{(}\StringTok{"tidyverse"}\NormalTok{); }\FunctionTok{require}\NormalTok{(tidyverse)\}}

\NormalTok{nosaukums}\OtherTok{=}\StringTok{"ForestsTreesAge\_BorealDeciduousOld\_cell.tif"}
\NormalTok{ielasisanas\_cels}\OtherTok{=}\FunctionTok{paste0}\NormalTok{(}\StringTok{"./RasterGrids\_100m/2024/RAW/"}\NormalTok{,nosaukums)}
\NormalTok{saglabasanas\_cels}\OtherTok{=}\FunctionTok{paste0}\NormalTok{(}\StringTok{"./RasterGrids\_100m/2024/Scaled/"}\NormalTok{,nosaukums)}
\NormalTok{slanis}\OtherTok{=}\FunctionTok{rast}\NormalTok{(ielasisanas\_cels)}
\NormalTok{videjais}\OtherTok{=}\FunctionTok{global}\NormalTok{(slanis,}\AttributeTok{fun=}\StringTok{"mean"}\NormalTok{,}\AttributeTok{na.rm=}\ConstantTok{TRUE}\NormalTok{)}
\NormalTok{centrets}\OtherTok{=}\NormalTok{slanis}\SpecialCharTok{{-}}\NormalTok{videjais[,}\DecValTok{1}\NormalTok{]}
\NormalTok{standartnovirze}\OtherTok{=}\NormalTok{terra}\SpecialCharTok{::}\FunctionTok{global}\NormalTok{(centrets,}\AttributeTok{fun=}\StringTok{"rms"}\NormalTok{,}\AttributeTok{na.rm=}\ConstantTok{TRUE}\NormalTok{)}
\NormalTok{merogots}\OtherTok{=}\NormalTok{centrets}\SpecialCharTok{/}\NormalTok{standartnovirze[,}\DecValTok{1}\NormalTok{]}
\FunctionTok{writeRaster}\NormalTok{(merogots,}
      \AttributeTok{filename=}\NormalTok{saglabasanas\_cels,}
      \AttributeTok{overwrite=}\ConstantTok{TRUE}\NormalTok{)}
\end{Highlighting}
\end{Shaded}

\section{ForestsTreesAge\_BorealDeciduousOld\_r500}\label{ch06.344}

\textbf{filename:} \texttt{ForestsTreesAge\_BorealDeciduousOld\_r500.tif}

\textbf{layername:} \texttt{egv\_344}

\textbf{English name:} Fractional cover of Old (over rotation age) Boreal Deciduous
Forests within the 0.5 km landscape

\textbf{Latvian name:} Vecu (kopš cirtmeta) šaurlapju mežu platības īpatsvars 0,5 km
ainavā

\textbf{Procedure:} The cover fraction within a radius of 500 m around the analysis grid cell is
calculated as the area-weighted sum of the \hyperref[ch06.343]{analysis cells} inside the
buffer, using the workflow \texttt{egvtools::radius\_function()}. During the calculation of the landscape metric,
inverse distance weighted (power = 2) gap filling on the output is applied
to ensure no missing values at the edges. Then the layer is rewritten to set
its name. Finally, the layer is standardised by subtracting the arithmetic
mean and dividing by the root mean squared error.

\begin{Shaded}
\begin{Highlighting}[]
\CommentTok{\# libs {-}{-}{-}{-}}
\ControlFlowTok{if}\NormalTok{(}\SpecialCharTok{!}\FunctionTok{require}\NormalTok{(terra)) \{}\FunctionTok{install.packages}\NormalTok{(}\StringTok{"terra"}\NormalTok{); }\FunctionTok{require}\NormalTok{(terra)\}}
\ControlFlowTok{if}\NormalTok{(}\SpecialCharTok{!}\FunctionTok{require}\NormalTok{(egvtools)) \{remotes}\SpecialCharTok{::}\FunctionTok{install\_github}\NormalTok{(}\StringTok{"aavotins/egvtools"}\NormalTok{); }\FunctionTok{require}\NormalTok{(egvtools)\}}


\CommentTok{\# Templates {-}{-}{-}{-}{-}}
\NormalTok{template100}\OtherTok{=}\FunctionTok{rast}\NormalTok{(}\StringTok{"./Templates/TemplateRasters/LV100m\_10km.tif"}\NormalTok{)}

\CommentTok{\# radii {-}{-}{-}{-}}
\FunctionTok{radius\_function}\NormalTok{(}
 \AttributeTok{kvadrati\_path =} \StringTok{"./Templates/TemplateGrids/tiles/"}\NormalTok{,}
 \AttributeTok{radii\_path   =} \StringTok{"./Templates/TemplateGridPoints/tiles/"}\NormalTok{,}
 \AttributeTok{tikls100\_path =} \StringTok{"./Templates/TemplateGrids/tikls100\_sauzeme.parquet"}\NormalTok{,}
 \AttributeTok{template\_path =} \StringTok{"./Templates/TemplateRasters/LV100m\_10km.tif"}\NormalTok{,}
 \AttributeTok{input\_layers  =} \FunctionTok{c}\NormalTok{(}\StringTok{"./RasterGrids\_100m/2024/RAW/ForestsTreesAge\_BorealDeciduousOld\_cell.tif"}\NormalTok{),}
 \AttributeTok{layer\_prefixes =} \FunctionTok{c}\NormalTok{(}\StringTok{"ForestsTreesAge\_BorealDeciduousOld"}\NormalTok{),}
 \AttributeTok{output\_dir   =} \StringTok{"./RasterGrids\_100m/2024/RAW/"}\NormalTok{,}
 \AttributeTok{n\_workers   =} \DecValTok{6}\NormalTok{,}
 \AttributeTok{radii     =} \FunctionTok{c}\NormalTok{(}\StringTok{"r500"}\NormalTok{),}
 \AttributeTok{radius\_mode  =} \StringTok{"sparse"}\NormalTok{,}
 \AttributeTok{extract\_fun  =} \StringTok{"mean"}\NormalTok{,}
 \AttributeTok{fill\_missing  =} \ConstantTok{TRUE}\NormalTok{,}
 \AttributeTok{IDW\_weight   =} \DecValTok{2}\NormalTok{,}
 \AttributeTok{future\_max\_size =} \DecValTok{40} \SpecialCharTok{*} \DecValTok{1024}\SpecialCharTok{\^{}}\DecValTok{3}\NormalTok{)}


\CommentTok{\# ForestsTreesAge\_BorealDeciduousOld\_r500.tif   egv\_344}
\NormalTok{slanis}\OtherTok{=}\FunctionTok{rast}\NormalTok{(}\StringTok{"./RasterGrids\_100m/2024/RAW/ForestsTreesAge\_BorealDeciduousOld\_r500.tif"}\NormalTok{)}
\FunctionTok{names}\NormalTok{(slanis)}\OtherTok{=}\StringTok{"egv\_344"}
\NormalTok{slanis2}\OtherTok{=}\FunctionTok{project}\NormalTok{(slanis,template100)}
\FunctionTok{writeRaster}\NormalTok{(slanis2,}
      \StringTok{"./RasterGrids\_100m/2024/RAW/ForestsTreesAge\_BorealDeciduousOld\_r500.tif"}\NormalTok{,}
      \AttributeTok{overwrite=}\ConstantTok{TRUE}\NormalTok{)}

\CommentTok{\# standardisation {-}{-}{-}{-}}
\ControlFlowTok{if}\NormalTok{(}\SpecialCharTok{!}\FunctionTok{require}\NormalTok{(terra)) \{}\FunctionTok{install.packages}\NormalTok{(}\StringTok{"terra"}\NormalTok{); }\FunctionTok{require}\NormalTok{(terra)\}}
\ControlFlowTok{if}\NormalTok{(}\SpecialCharTok{!}\FunctionTok{require}\NormalTok{(tidyverse)) \{}\FunctionTok{install.packages}\NormalTok{(}\StringTok{"tidyverse"}\NormalTok{); }\FunctionTok{require}\NormalTok{(tidyverse)\}}

\NormalTok{nosaukums}\OtherTok{=}\StringTok{"ForestsTreesAge\_BorealDeciduousOld\_r500.tif"}
\NormalTok{ielasisanas\_cels}\OtherTok{=}\FunctionTok{paste0}\NormalTok{(}\StringTok{"./RasterGrids\_100m/2024/RAW/"}\NormalTok{,nosaukums)}
\NormalTok{saglabasanas\_cels}\OtherTok{=}\FunctionTok{paste0}\NormalTok{(}\StringTok{"./RasterGrids\_100m/2024/Scaled/"}\NormalTok{,nosaukums)}
\NormalTok{slanis}\OtherTok{=}\FunctionTok{rast}\NormalTok{(ielasisanas\_cels)}
\NormalTok{videjais}\OtherTok{=}\FunctionTok{global}\NormalTok{(slanis,}\AttributeTok{fun=}\StringTok{"mean"}\NormalTok{,}\AttributeTok{na.rm=}\ConstantTok{TRUE}\NormalTok{)}
\NormalTok{centrets}\OtherTok{=}\NormalTok{slanis}\SpecialCharTok{{-}}\NormalTok{videjais[,}\DecValTok{1}\NormalTok{]}
\NormalTok{standartnovirze}\OtherTok{=}\NormalTok{terra}\SpecialCharTok{::}\FunctionTok{global}\NormalTok{(centrets,}\AttributeTok{fun=}\StringTok{"rms"}\NormalTok{,}\AttributeTok{na.rm=}\ConstantTok{TRUE}\NormalTok{)}
\NormalTok{merogots}\OtherTok{=}\NormalTok{centrets}\SpecialCharTok{/}\NormalTok{standartnovirze[,}\DecValTok{1}\NormalTok{]}
\FunctionTok{writeRaster}\NormalTok{(merogots,}
      \AttributeTok{filename=}\NormalTok{saglabasanas\_cels,}
      \AttributeTok{overwrite=}\ConstantTok{TRUE}\NormalTok{)}
\end{Highlighting}
\end{Shaded}

\section{ForestsTreesAge\_BorealDeciduousOld\_r1250}\label{ch06.345}

\textbf{filename:} \texttt{ForestsTreesAge\_BorealDeciduousOld\_r1250.tif}

\textbf{layername:} \texttt{egv\_345}

\textbf{English name:} Fractional cover of Old (over rotation age) Boreal Deciduous
Forests within the 1.25 km landscape

\textbf{Latvian name:} Vecu (kopš cirtmeta) šaurlapju mežu platības īpatsvars 1,25 km
ainavā

\textbf{Procedure:} The cover fraction within a radius of 1250 m around the analysis grid cell
is calculated as the area-weighted sum of the \hyperref[ch06.343]{analysis cells} inside
the buffer, using the workflow \texttt{egvtools::radius\_function()}. During the calculation of the landscape
metric, inverse distance weighted (power = 2) gap filling on the output is
applied to ensure no missing values at the edges. Then the layer is
rewritten to set its name. Finally, the layer is standardised by
subtracting the arithmetic mean and dividing by the root mean squared error.

\begin{Shaded}
\begin{Highlighting}[]
\CommentTok{\# libs {-}{-}{-}{-}}
\ControlFlowTok{if}\NormalTok{(}\SpecialCharTok{!}\FunctionTok{require}\NormalTok{(terra)) \{}\FunctionTok{install.packages}\NormalTok{(}\StringTok{"terra"}\NormalTok{); }\FunctionTok{require}\NormalTok{(terra)\}}
\ControlFlowTok{if}\NormalTok{(}\SpecialCharTok{!}\FunctionTok{require}\NormalTok{(egvtools)) \{remotes}\SpecialCharTok{::}\FunctionTok{install\_github}\NormalTok{(}\StringTok{"aavotins/egvtools"}\NormalTok{); }\FunctionTok{require}\NormalTok{(egvtools)\}}


\CommentTok{\# Templates {-}{-}{-}{-}{-}}
\NormalTok{template100}\OtherTok{=}\FunctionTok{rast}\NormalTok{(}\StringTok{"./Templates/TemplateRasters/LV100m\_10km.tif"}\NormalTok{)}

\CommentTok{\# radii {-}{-}{-}{-}}
\FunctionTok{radius\_function}\NormalTok{(}
 \AttributeTok{kvadrati\_path =} \StringTok{"./Templates/TemplateGrids/tiles/"}\NormalTok{,}
 \AttributeTok{radii\_path   =} \StringTok{"./Templates/TemplateGridPoints/tiles/"}\NormalTok{,}
 \AttributeTok{tikls100\_path =} \StringTok{"./Templates/TemplateGrids/tikls100\_sauzeme.parquet"}\NormalTok{,}
 \AttributeTok{template\_path =} \StringTok{"./Templates/TemplateRasters/LV100m\_10km.tif"}\NormalTok{,}
 \AttributeTok{input\_layers  =} \FunctionTok{c}\NormalTok{(}\StringTok{"./RasterGrids\_100m/2024/RAW/ForestsTreesAge\_BorealDeciduousOld\_cell.tif"}\NormalTok{),}
 \AttributeTok{layer\_prefixes =} \FunctionTok{c}\NormalTok{(}\StringTok{"ForestsTreesAge\_BorealDeciduousOld"}\NormalTok{),}
 \AttributeTok{output\_dir   =} \StringTok{"./RasterGrids\_100m/2024/RAW/"}\NormalTok{,}
 \AttributeTok{n\_workers   =} \DecValTok{6}\NormalTok{,}
 \AttributeTok{radii     =} \FunctionTok{c}\NormalTok{(}\StringTok{"r1250"}\NormalTok{),}
 \AttributeTok{radius\_mode  =} \StringTok{"sparse"}\NormalTok{,}
 \AttributeTok{extract\_fun  =} \StringTok{"mean"}\NormalTok{,}
 \AttributeTok{fill\_missing  =} \ConstantTok{TRUE}\NormalTok{,}
 \AttributeTok{IDW\_weight   =} \DecValTok{2}\NormalTok{,}
 \AttributeTok{future\_max\_size =} \DecValTok{40} \SpecialCharTok{*} \DecValTok{1024}\SpecialCharTok{\^{}}\DecValTok{3}\NormalTok{)}


\CommentTok{\# ForestsTreesAge\_BorealDeciduousOld\_r1250.tif  egv\_345}
\NormalTok{slanis}\OtherTok{=}\FunctionTok{rast}\NormalTok{(}\StringTok{"./RasterGrids\_100m/2024/RAW/ForestsTreesAge\_BorealDeciduousOld\_r1250.tif"}\NormalTok{)}
\FunctionTok{names}\NormalTok{(slanis)}\OtherTok{=}\StringTok{"egv\_345"}
\NormalTok{slanis2}\OtherTok{=}\FunctionTok{project}\NormalTok{(slanis,template100)}
\FunctionTok{writeRaster}\NormalTok{(slanis2,}
      \StringTok{"./RasterGrids\_100m/2024/RAW/ForestsTreesAge\_BorealDeciduousOld\_r1250.tif"}\NormalTok{,}
      \AttributeTok{overwrite=}\ConstantTok{TRUE}\NormalTok{)}

\CommentTok{\# standardisation {-}{-}{-}{-}}
\ControlFlowTok{if}\NormalTok{(}\SpecialCharTok{!}\FunctionTok{require}\NormalTok{(terra)) \{}\FunctionTok{install.packages}\NormalTok{(}\StringTok{"terra"}\NormalTok{); }\FunctionTok{require}\NormalTok{(terra)\}}
\ControlFlowTok{if}\NormalTok{(}\SpecialCharTok{!}\FunctionTok{require}\NormalTok{(tidyverse)) \{}\FunctionTok{install.packages}\NormalTok{(}\StringTok{"tidyverse"}\NormalTok{); }\FunctionTok{require}\NormalTok{(tidyverse)\}}

\NormalTok{nosaukums}\OtherTok{=}\StringTok{"ForestsTreesAge\_BorealDeciduousOld\_r1250.tif"}
\NormalTok{ielasisanas\_cels}\OtherTok{=}\FunctionTok{paste0}\NormalTok{(}\StringTok{"./RasterGrids\_100m/2024/RAW/"}\NormalTok{,nosaukums)}
\NormalTok{saglabasanas\_cels}\OtherTok{=}\FunctionTok{paste0}\NormalTok{(}\StringTok{"./RasterGrids\_100m/2024/Scaled/"}\NormalTok{,nosaukums)}
\NormalTok{slanis}\OtherTok{=}\FunctionTok{rast}\NormalTok{(ielasisanas\_cels)}
\NormalTok{videjais}\OtherTok{=}\FunctionTok{global}\NormalTok{(slanis,}\AttributeTok{fun=}\StringTok{"mean"}\NormalTok{,}\AttributeTok{na.rm=}\ConstantTok{TRUE}\NormalTok{)}
\NormalTok{centrets}\OtherTok{=}\NormalTok{slanis}\SpecialCharTok{{-}}\NormalTok{videjais[,}\DecValTok{1}\NormalTok{]}
\NormalTok{standartnovirze}\OtherTok{=}\NormalTok{terra}\SpecialCharTok{::}\FunctionTok{global}\NormalTok{(centrets,}\AttributeTok{fun=}\StringTok{"rms"}\NormalTok{,}\AttributeTok{na.rm=}\ConstantTok{TRUE}\NormalTok{)}
\NormalTok{merogots}\OtherTok{=}\NormalTok{centrets}\SpecialCharTok{/}\NormalTok{standartnovirze[,}\DecValTok{1}\NormalTok{]}
\FunctionTok{writeRaster}\NormalTok{(merogots,}
      \AttributeTok{filename=}\NormalTok{saglabasanas\_cels,}
      \AttributeTok{overwrite=}\ConstantTok{TRUE}\NormalTok{)}
\end{Highlighting}
\end{Shaded}

\section{ForestsTreesAge\_BorealDeciduousOld\_r3000}\label{ch06.346}

\textbf{filename:} \texttt{ForestsTreesAge\_BorealDeciduousOld\_r3000.tif}

\textbf{layername:} \texttt{egv\_346}

\textbf{English name:} Fractional cover of Old (over rotation age) Boreal Deciduous
Forests within the 3 km landscape

\textbf{Latvian name:} Vecu (kopš cirtmeta) šaurlapju mežu platības īpatsvars 3 km
ainavā

\textbf{Procedure:} The cover fraction within a radius of 3000 m around the analysis grid cell
is calculated as the area-weighted sum of the \hyperref[ch06.343]{analysis cells} inside
the buffer, using the workflow \texttt{egvtools::radius\_function()}. During the calculation of the landscape
metric, inverse distance weighted (power = 2) gap filling on the output is
applied to ensure no missing values at the edges. Then the layer is
rewritten to set its name. Finally, the layer is standardised by
subtracting the arithmetic mean and dividing by the root mean squared error.

\begin{Shaded}
\begin{Highlighting}[]
\CommentTok{\# libs {-}{-}{-}{-}}
\ControlFlowTok{if}\NormalTok{(}\SpecialCharTok{!}\FunctionTok{require}\NormalTok{(terra)) \{}\FunctionTok{install.packages}\NormalTok{(}\StringTok{"terra"}\NormalTok{); }\FunctionTok{require}\NormalTok{(terra)\}}
\ControlFlowTok{if}\NormalTok{(}\SpecialCharTok{!}\FunctionTok{require}\NormalTok{(egvtools)) \{remotes}\SpecialCharTok{::}\FunctionTok{install\_github}\NormalTok{(}\StringTok{"aavotins/egvtools"}\NormalTok{); }\FunctionTok{require}\NormalTok{(egvtools)\}}


\CommentTok{\# Templates {-}{-}{-}{-}{-}}
\NormalTok{template100}\OtherTok{=}\FunctionTok{rast}\NormalTok{(}\StringTok{"./Templates/TemplateRasters/LV100m\_10km.tif"}\NormalTok{)}

\CommentTok{\# radii {-}{-}{-}{-}}
\FunctionTok{radius\_function}\NormalTok{(}
 \AttributeTok{kvadrati\_path =} \StringTok{"./Templates/TemplateGrids/tiles/"}\NormalTok{,}
 \AttributeTok{radii\_path   =} \StringTok{"./Templates/TemplateGridPoints/tiles/"}\NormalTok{,}
 \AttributeTok{tikls100\_path =} \StringTok{"./Templates/TemplateGrids/tikls100\_sauzeme.parquet"}\NormalTok{,}
 \AttributeTok{template\_path =} \StringTok{"./Templates/TemplateRasters/LV100m\_10km.tif"}\NormalTok{,}
 \AttributeTok{input\_layers  =} \FunctionTok{c}\NormalTok{(}\StringTok{"./RasterGrids\_100m/2024/RAW/ForestsTreesAge\_BorealDeciduousOld\_cell.tif"}\NormalTok{),}
 \AttributeTok{layer\_prefixes =} \FunctionTok{c}\NormalTok{(}\StringTok{"ForestsTreesAge\_BorealDeciduousOld"}\NormalTok{),}
 \AttributeTok{output\_dir   =} \StringTok{"./RasterGrids\_100m/2024/RAW/"}\NormalTok{,}
 \AttributeTok{n\_workers   =} \DecValTok{6}\NormalTok{,}
 \AttributeTok{radii     =} \FunctionTok{c}\NormalTok{(}\StringTok{"r3000"}\NormalTok{),}
 \AttributeTok{radius\_mode  =} \StringTok{"sparse"}\NormalTok{,}
 \AttributeTok{extract\_fun  =} \StringTok{"mean"}\NormalTok{,}
 \AttributeTok{fill\_missing  =} \ConstantTok{TRUE}\NormalTok{,}
 \AttributeTok{IDW\_weight   =} \DecValTok{2}\NormalTok{,}
 \AttributeTok{future\_max\_size =} \DecValTok{40} \SpecialCharTok{*} \DecValTok{1024}\SpecialCharTok{\^{}}\DecValTok{3}\NormalTok{)}


\CommentTok{\# ForestsTreesAge\_BorealDeciduousOld\_r3000.tif  egv\_346}
\NormalTok{slanis}\OtherTok{=}\FunctionTok{rast}\NormalTok{(}\StringTok{"./RasterGrids\_100m/2024/RAW/ForestsTreesAge\_BorealDeciduousOld\_r3000.tif"}\NormalTok{)}
\FunctionTok{names}\NormalTok{(slanis)}\OtherTok{=}\StringTok{"egv\_346"}
\NormalTok{slanis2}\OtherTok{=}\FunctionTok{project}\NormalTok{(slanis,template100)}
\FunctionTok{writeRaster}\NormalTok{(slanis2,}
      \StringTok{"./RasterGrids\_100m/2024/RAW/ForestsTreesAge\_BorealDeciduousOld\_r3000.tif"}\NormalTok{,}
      \AttributeTok{overwrite=}\ConstantTok{TRUE}\NormalTok{)}

\CommentTok{\# standardisation {-}{-}{-}{-}}
\ControlFlowTok{if}\NormalTok{(}\SpecialCharTok{!}\FunctionTok{require}\NormalTok{(terra)) \{}\FunctionTok{install.packages}\NormalTok{(}\StringTok{"terra"}\NormalTok{); }\FunctionTok{require}\NormalTok{(terra)\}}
\ControlFlowTok{if}\NormalTok{(}\SpecialCharTok{!}\FunctionTok{require}\NormalTok{(tidyverse)) \{}\FunctionTok{install.packages}\NormalTok{(}\StringTok{"tidyverse"}\NormalTok{); }\FunctionTok{require}\NormalTok{(tidyverse)\}}

\NormalTok{nosaukums}\OtherTok{=}\StringTok{"ForestsTreesAge\_BorealDeciduousOld\_r3000.tif"}
\NormalTok{ielasisanas\_cels}\OtherTok{=}\FunctionTok{paste0}\NormalTok{(}\StringTok{"./RasterGrids\_100m/2024/RAW/"}\NormalTok{,nosaukums)}
\NormalTok{saglabasanas\_cels}\OtherTok{=}\FunctionTok{paste0}\NormalTok{(}\StringTok{"./RasterGrids\_100m/2024/Scaled/"}\NormalTok{,nosaukums)}
\NormalTok{slanis}\OtherTok{=}\FunctionTok{rast}\NormalTok{(ielasisanas\_cels)}
\NormalTok{videjais}\OtherTok{=}\FunctionTok{global}\NormalTok{(slanis,}\AttributeTok{fun=}\StringTok{"mean"}\NormalTok{,}\AttributeTok{na.rm=}\ConstantTok{TRUE}\NormalTok{)}
\NormalTok{centrets}\OtherTok{=}\NormalTok{slanis}\SpecialCharTok{{-}}\NormalTok{videjais[,}\DecValTok{1}\NormalTok{]}
\NormalTok{standartnovirze}\OtherTok{=}\NormalTok{terra}\SpecialCharTok{::}\FunctionTok{global}\NormalTok{(centrets,}\AttributeTok{fun=}\StringTok{"rms"}\NormalTok{,}\AttributeTok{na.rm=}\ConstantTok{TRUE}\NormalTok{)}
\NormalTok{merogots}\OtherTok{=}\NormalTok{centrets}\SpecialCharTok{/}\NormalTok{standartnovirze[,}\DecValTok{1}\NormalTok{]}
\FunctionTok{writeRaster}\NormalTok{(merogots,}
      \AttributeTok{filename=}\NormalTok{saglabasanas\_cels,}
      \AttributeTok{overwrite=}\ConstantTok{TRUE}\NormalTok{)}
\end{Highlighting}
\end{Shaded}

\section{ForestsTreesAge\_BorealDeciduousOld\_r10000}\label{ch06.347}

\textbf{filename:} \texttt{ForestsTreesAge\_BorealDeciduousOld\_r10000.tif}

\textbf{layername:} \texttt{egv\_347}

\textbf{English name:} Fractional cover of Old (over rotation age) Boreal Deciduous
Forests within the 10 km landscape

\textbf{Latvian name:} Vecu (kopš cirtmeta) šaurlapju mežu platības īpatsvars 10 km
ainavā

\textbf{Procedure:} The cover fraction within a radius of 10000 m around the analysis grid cell
is calculated as the area-weighted sum of the \hyperref[ch06.343]{analysis cells} inside
the buffer, using the workflow \texttt{egvtools::radius\_function()}. During the calculation of the landscape
metric, inverse distance weighted (power = 2) gap filling on the output is
applied to ensure no missing values at the edges. Then the layer is
rewritten to set its name. Finally, the layer is standardised by
subtracting the arithmetic mean and dividing by the root mean squared error.

\begin{Shaded}
\begin{Highlighting}[]
\CommentTok{\# libs {-}{-}{-}{-}}
\ControlFlowTok{if}\NormalTok{(}\SpecialCharTok{!}\FunctionTok{require}\NormalTok{(terra)) \{}\FunctionTok{install.packages}\NormalTok{(}\StringTok{"terra"}\NormalTok{); }\FunctionTok{require}\NormalTok{(terra)\}}
\ControlFlowTok{if}\NormalTok{(}\SpecialCharTok{!}\FunctionTok{require}\NormalTok{(egvtools)) \{remotes}\SpecialCharTok{::}\FunctionTok{install\_github}\NormalTok{(}\StringTok{"aavotins/egvtools"}\NormalTok{); }\FunctionTok{require}\NormalTok{(egvtools)\}}


\CommentTok{\# Templates {-}{-}{-}{-}{-}}
\NormalTok{template100}\OtherTok{=}\FunctionTok{rast}\NormalTok{(}\StringTok{"./Templates/TemplateRasters/LV100m\_10km.tif"}\NormalTok{)}

\CommentTok{\# radii {-}{-}{-}{-}}
\FunctionTok{radius\_function}\NormalTok{(}
 \AttributeTok{kvadrati\_path =} \StringTok{"./Templates/TemplateGrids/tiles/"}\NormalTok{,}
 \AttributeTok{radii\_path   =} \StringTok{"./Templates/TemplateGridPoints/tiles/"}\NormalTok{,}
 \AttributeTok{tikls100\_path =} \StringTok{"./Templates/TemplateGrids/tikls100\_sauzeme.parquet"}\NormalTok{,}
 \AttributeTok{template\_path =} \StringTok{"./Templates/TemplateRasters/LV100m\_10km.tif"}\NormalTok{,}
 \AttributeTok{input\_layers  =} \FunctionTok{c}\NormalTok{(}\StringTok{"./RasterGrids\_100m/2024/RAW/ForestsTreesAge\_BorealDeciduousOld\_cell.tif"}\NormalTok{),}
 \AttributeTok{layer\_prefixes =} \FunctionTok{c}\NormalTok{(}\StringTok{"ForestsTreesAge\_BorealDeciduousOld"}\NormalTok{),}
 \AttributeTok{output\_dir   =} \StringTok{"./RasterGrids\_100m/2024/RAW/"}\NormalTok{,}
 \AttributeTok{n\_workers   =} \DecValTok{6}\NormalTok{,}
 \AttributeTok{radii     =} \FunctionTok{c}\NormalTok{(}\StringTok{"r10000"}\NormalTok{),}
 \AttributeTok{radius\_mode  =} \StringTok{"sparse"}\NormalTok{,}
 \AttributeTok{extract\_fun  =} \StringTok{"mean"}\NormalTok{,}
 \AttributeTok{fill\_missing  =} \ConstantTok{TRUE}\NormalTok{,}
 \AttributeTok{IDW\_weight   =} \DecValTok{2}\NormalTok{,}
 \AttributeTok{future\_max\_size =} \DecValTok{40} \SpecialCharTok{*} \DecValTok{1024}\SpecialCharTok{\^{}}\DecValTok{3}\NormalTok{)}


\CommentTok{\# ForestsTreesAge\_BorealDeciduousOld\_r10000.tif egv\_347}
\NormalTok{slanis}\OtherTok{=}\FunctionTok{rast}\NormalTok{(}\StringTok{"./RasterGrids\_100m/2024/RAW/ForestsTreesAge\_BorealDeciduousOld\_r10000.tif"}\NormalTok{)}
\FunctionTok{names}\NormalTok{(slanis)}\OtherTok{=}\StringTok{"egv\_347"}
\NormalTok{slanis2}\OtherTok{=}\FunctionTok{project}\NormalTok{(slanis,template100)}
\FunctionTok{writeRaster}\NormalTok{(slanis2,}
      \StringTok{"./RasterGrids\_100m/2024/RAW/ForestsTreesAge\_BorealDeciduousOld\_r10000.tif"}\NormalTok{,}
      \AttributeTok{overwrite=}\ConstantTok{TRUE}\NormalTok{)}

\CommentTok{\# standardisation {-}{-}{-}{-}}
\ControlFlowTok{if}\NormalTok{(}\SpecialCharTok{!}\FunctionTok{require}\NormalTok{(terra)) \{}\FunctionTok{install.packages}\NormalTok{(}\StringTok{"terra"}\NormalTok{); }\FunctionTok{require}\NormalTok{(terra)\}}
\ControlFlowTok{if}\NormalTok{(}\SpecialCharTok{!}\FunctionTok{require}\NormalTok{(tidyverse)) \{}\FunctionTok{install.packages}\NormalTok{(}\StringTok{"tidyverse"}\NormalTok{); }\FunctionTok{require}\NormalTok{(tidyverse)\}}

\NormalTok{nosaukums}\OtherTok{=}\StringTok{"ForestsTreesAge\_BorealDeciduousOld\_r10000.tif"}
\NormalTok{ielasisanas\_cels}\OtherTok{=}\FunctionTok{paste0}\NormalTok{(}\StringTok{"./RasterGrids\_100m/2024/RAW/"}\NormalTok{,nosaukums)}
\NormalTok{saglabasanas\_cels}\OtherTok{=}\FunctionTok{paste0}\NormalTok{(}\StringTok{"./RasterGrids\_100m/2024/Scaled/"}\NormalTok{,nosaukums)}
\NormalTok{slanis}\OtherTok{=}\FunctionTok{rast}\NormalTok{(ielasisanas\_cels)}
\NormalTok{videjais}\OtherTok{=}\FunctionTok{global}\NormalTok{(slanis,}\AttributeTok{fun=}\StringTok{"mean"}\NormalTok{,}\AttributeTok{na.rm=}\ConstantTok{TRUE}\NormalTok{)}
\NormalTok{centrets}\OtherTok{=}\NormalTok{slanis}\SpecialCharTok{{-}}\NormalTok{videjais[,}\DecValTok{1}\NormalTok{]}
\NormalTok{standartnovirze}\OtherTok{=}\NormalTok{terra}\SpecialCharTok{::}\FunctionTok{global}\NormalTok{(centrets,}\AttributeTok{fun=}\StringTok{"rms"}\NormalTok{,}\AttributeTok{na.rm=}\ConstantTok{TRUE}\NormalTok{)}
\NormalTok{merogots}\OtherTok{=}\NormalTok{centrets}\SpecialCharTok{/}\NormalTok{standartnovirze[,}\DecValTok{1}\NormalTok{]}
\FunctionTok{writeRaster}\NormalTok{(merogots,}
      \AttributeTok{filename=}\NormalTok{saglabasanas\_cels,}
      \AttributeTok{overwrite=}\ConstantTok{TRUE}\NormalTok{)}
\end{Highlighting}
\end{Shaded}

\section{ForestsTreesAge\_BorealDeciduousYoung\_cell}\label{ch06.348}

\textbf{filename:} \texttt{ForestsTreesAge\_BorealDeciduousYoung\_cell.tif}

\textbf{layername:} \texttt{egv\_348}

\textbf{English name:} Fractional cover of Young (pre-rotation age) Boreal Deciduous
Forests within the analysis cell (1 ha)

\textbf{Latvian name:} Jaunu (pirms cirtmeta) šaurlapju mežu platības īpatsvars
analīzes šūnā (1 ha)

\textbf{Procedure:} Most EGVs describing forests are spatially restricted to areas outside
of clearcuts and dead stands. This mask is created using a combination of
the \hyperref[Ch04.01]{State Forest Service's
State Forest Registry} land category 12 and 14, and \hyperref[Ch04.09]{The
Global Forest Watch} pixels classified as lost tree canopy cover since
2020 (raster layer matching input, presence = 1, absence = 0).

To prepare this EGV, stands from the \hyperref[Ch04.01]{State Forest Service's State Forest
Registry} are classified into (in order):

\begin{itemize}
\item
  coniferous (see \hyperref[Ch01]{Terminology and acronyms} for species codes) if
  timber volume of those species exceeded 75\%;
\item
  Boreal deciduous if timber volume of those species exceeded 75\%;
\item
  temperate deciduous if timber volume of those species exceeded 50\%;
\item
  mixed otherwise;
\end{itemize}

then Boreal deciduous stands younger than the legal rotation age are selected
and geometries are rasterised (presence = 1, NA otherwise). Rasterisation is
performed using the workflow \texttt{egvtools::polygon2input()}, restricting to pixels outside clearcut
mask and covering background with value 0. The resulting layer
is then aggregated to EGV resolution using the workflow \texttt{egvtools::input2egv()}, which
calculates the arithmetic mean to determine the cover fraction. During
aggregation, inverse distance weighted (power = 2) gap filling on the output is
applied to ensure no missing values at the edges. Finally, the layer is
standardised by subtracting the arithmetic mean and dividing by the root mean squared
error.

\begin{Shaded}
\begin{Highlighting}[]
\CommentTok{\# libs {-}{-}{-}{-}}
\ControlFlowTok{if}\NormalTok{(}\SpecialCharTok{!}\FunctionTok{require}\NormalTok{(egvtools)) \{remotes}\SpecialCharTok{::}\FunctionTok{install\_github}\NormalTok{(}\StringTok{"aavotins/egvtools"}\NormalTok{); }\FunctionTok{require}\NormalTok{(egvtools)\}}
\ControlFlowTok{if}\NormalTok{(}\SpecialCharTok{!}\FunctionTok{require}\NormalTok{(terra)) \{}\FunctionTok{install.packages}\NormalTok{(}\StringTok{"terra"}\NormalTok{); }\FunctionTok{require}\NormalTok{(terra)\}}
\ControlFlowTok{if}\NormalTok{(}\SpecialCharTok{!}\FunctionTok{require}\NormalTok{(sf)) \{}\FunctionTok{install.packages}\NormalTok{(}\StringTok{"sf"}\NormalTok{); }\FunctionTok{require}\NormalTok{(sf)\}}
\ControlFlowTok{if}\NormalTok{(}\SpecialCharTok{!}\FunctionTok{require}\NormalTok{(tidyverse)) \{}\FunctionTok{install.packages}\NormalTok{(}\StringTok{"tidyverse"}\NormalTok{); }\FunctionTok{require}\NormalTok{(tidyverse)\}}
\ControlFlowTok{if}\NormalTok{(}\SpecialCharTok{!}\FunctionTok{require}\NormalTok{(sfarrow)) \{}\FunctionTok{install.packages}\NormalTok{(}\StringTok{"sfarrow"}\NormalTok{); }\FunctionTok{require}\NormalTok{(sfarrow)\}}
\ControlFlowTok{if}\NormalTok{(}\SpecialCharTok{!}\FunctionTok{require}\NormalTok{(readxl)) \{}\FunctionTok{install.packages}\NormalTok{(}\StringTok{"readxl"}\NormalTok{); }\FunctionTok{require}\NormalTok{(readxl)\}}
\ControlFlowTok{if}\NormalTok{(}\SpecialCharTok{!}\FunctionTok{require}\NormalTok{(raster)) \{}\FunctionTok{install.packages}\NormalTok{(}\StringTok{"raster"}\NormalTok{); }\FunctionTok{require}\NormalTok{(raster)\}}
\ControlFlowTok{if}\NormalTok{(}\SpecialCharTok{!}\FunctionTok{require}\NormalTok{(fasterize)) \{}\FunctionTok{install.packages}\NormalTok{(}\StringTok{"fasterize"}\NormalTok{); }\FunctionTok{require}\NormalTok{(fasterize)\}}

\CommentTok{\# templates {-}{-}{-}{-}}
\NormalTok{template100}\OtherTok{=}\FunctionTok{rast}\NormalTok{(}\StringTok{"./Templates/TemplateRasters/LV100m\_10km.tif"}\NormalTok{)}
\NormalTok{template10}\OtherTok{=}\FunctionTok{rast}\NormalTok{(}\StringTok{"./Templates/TemplateRasters/LV10m\_10km.tif"}\NormalTok{)}
\NormalTok{rastrs10}\OtherTok{=}\FunctionTok{raster}\NormalTok{(template10)}

\NormalTok{nulls10}\OtherTok{=}\FunctionTok{rast}\NormalTok{(}\StringTok{"./Templates/TemplateRasters/nulls\_LV10m\_10km.tif"}\NormalTok{)}
\NormalTok{nulls100}\OtherTok{=}\FunctionTok{rast}\NormalTok{(}\StringTok{"./Templates/TemplateRasters/nulls\_LV100m\_10km.tif"}\NormalTok{)}


\CommentTok{\# simple landscape {-}{-}{-}{-}}
\NormalTok{simple\_landscape}\OtherTok{=}\FunctionTok{rast}\NormalTok{(}\StringTok{"RasterGrids\_10m/2024/Ainava\_vienk\_mask.tif"}\NormalTok{)}

\CommentTok{\# mvr {-}{-}{-}{-}}
\NormalTok{mvr}\OtherTok{=}\FunctionTok{st\_read\_parquet}\NormalTok{(}\StringTok{"./Geodata/2024/MVR/nogabali\_2024janv.parquet"}\NormalTok{)}
\NormalTok{mvr}\SpecialCharTok{$}\NormalTok{yes}\OtherTok{=}\DecValTok{1}

\CommentTok{\# clear cut mask {-}{-}{-}{-}}
\NormalTok{izcirtumi}\OtherTok{=}\NormalTok{mvr }\SpecialCharTok{\%\textgreater{}\%} 
 \FunctionTok{filter}\NormalTok{(zkat }\SpecialCharTok{\%in\%} \FunctionTok{c}\NormalTok{(}\StringTok{"12"}\NormalTok{,}\StringTok{"14"}\NormalTok{)) }\SpecialCharTok{\%\textgreater{}\%} 
\NormalTok{ dplyr}\SpecialCharTok{::}\FunctionTok{select}\NormalTok{(yes)}
\NormalTok{r\_izcirtumi\_mvr}\OtherTok{=}\FunctionTok{fasterize}\NormalTok{(izcirtumi,rastrs10,}\AttributeTok{field=}\StringTok{"yes"}\NormalTok{)}
\NormalTok{t\_izcirtumi\_mvr}\OtherTok{=}\FunctionTok{rast}\NormalTok{(r\_izcirtumi\_mvr)}
\FunctionTok{plot}\NormalTok{(t\_izcirtumi\_mvr)}

\NormalTok{tcl}\OtherTok{=}\FunctionTok{rast}\NormalTok{(}\StringTok{"./Geodata/2024/Trees/GFW/TreeCoverLoss\_v1\_12.tif"}\NormalTok{)}
\NormalTok{tcl2}\OtherTok{=}\FunctionTok{ifel}\NormalTok{(tcl}\SpecialCharTok{\textless{}}\DecValTok{20}\NormalTok{,}\DecValTok{0}\NormalTok{,}\DecValTok{1}\NormalTok{)}
\NormalTok{tclX}\OtherTok{=}\FunctionTok{cover}\NormalTok{(tcl2,nulls10)}
\FunctionTok{plot}\NormalTok{(tclX)}

\NormalTok{clearcut\_mask}\OtherTok{=}\FunctionTok{cover}\NormalTok{(t\_izcirtumi\_mvr,tclX,}
          \AttributeTok{filename=}\StringTok{"./RasterGrids\_10m/2024/Mask\_clearcuts.tif"}\NormalTok{,}
          \AttributeTok{overwrite=}\ConstantTok{TRUE}\NormalTok{)}
\FunctionTok{plot}\NormalTok{(clearcut\_mask)}

\FunctionTok{rm}\NormalTok{(izcirtumi)}
\FunctionTok{rm}\NormalTok{(r\_izcirtumi\_mvr)}
\FunctionTok{rm}\NormalTok{(t\_izcirtumi\_mvr)}
\FunctionTok{rm}\NormalTok{(tcl)}
\FunctionTok{rm}\NormalTok{(tcl2)}
\FunctionTok{rm}\NormalTok{(tclX)}

\CommentTok{\# ForestsTreesAge\_BorealDeciduousYoung\_cell.tif egv\_348 {-}{-}{-}{-}}
\NormalTok{skujkoki}\OtherTok{=}\FunctionTok{c}\NormalTok{(}\StringTok{"1"}\NormalTok{,}\StringTok{"3"}\NormalTok{,}\StringTok{"13"}\NormalTok{,}\StringTok{"14"}\NormalTok{,}\StringTok{"15"}\NormalTok{,}\StringTok{"22"}\NormalTok{,}\StringTok{"23"}\NormalTok{,}\StringTok{"28"}\NormalTok{) }\CommentTok{\# 8}
\NormalTok{saurlapji}\OtherTok{=}\FunctionTok{c}\NormalTok{(}\StringTok{"4"}\NormalTok{,}\StringTok{"6"}\NormalTok{,}\StringTok{"8"}\NormalTok{,}\StringTok{"9"}\NormalTok{,}\StringTok{"19"}\NormalTok{,}\StringTok{"20"}\NormalTok{,}\StringTok{"21"}\NormalTok{,}\StringTok{"32"}\NormalTok{,}\StringTok{"35"}\NormalTok{,}\StringTok{"68"}\NormalTok{) }\CommentTok{\# 10}
\NormalTok{platlapji}\OtherTok{=}\FunctionTok{c}\NormalTok{(}\StringTok{"10"}\NormalTok{,}\StringTok{"11"}\NormalTok{,}\StringTok{"12"}\NormalTok{,}\StringTok{"16"}\NormalTok{,}\StringTok{"17"}\NormalTok{,}\StringTok{"18"}\NormalTok{,}\StringTok{"24"}\NormalTok{,}\StringTok{"25"}\NormalTok{,}\StringTok{"26"}\NormalTok{,}\StringTok{"27"}\NormalTok{,}\StringTok{"28"}\NormalTok{,}\StringTok{"29"}\NormalTok{,}\StringTok{"50"}\NormalTok{,}
      \StringTok{"61"}\NormalTok{,}\StringTok{"62"}\NormalTok{,}\StringTok{"63"}\NormalTok{,}\StringTok{"64"}\NormalTok{,}\StringTok{"65"}\NormalTok{,}\StringTok{"66"}\NormalTok{,}\StringTok{"67"}\NormalTok{,}\StringTok{"69"}\NormalTok{) }\CommentTok{\# 21}
\NormalTok{mvr}\OtherTok{=}\NormalTok{mvr }\SpecialCharTok{\%\textgreater{}\%} 
 \FunctionTok{mutate}\NormalTok{(}\AttributeTok{kraja\_skujkoku=}\FunctionTok{ifelse}\NormalTok{(s10 }\SpecialCharTok{\%in\%}\NormalTok{ skujkoki,v10,}\DecValTok{0}\NormalTok{)}\SpecialCharTok{+}
      \FunctionTok{ifelse}\NormalTok{(s11 }\SpecialCharTok{\%in\%}\NormalTok{ skujkoki,v11,}\DecValTok{0}\NormalTok{)}\SpecialCharTok{+}\FunctionTok{ifelse}\NormalTok{(s12 }\SpecialCharTok{\%in\%}\NormalTok{ skujkoki,v12,}\DecValTok{0}\NormalTok{)}\SpecialCharTok{+}
      \FunctionTok{ifelse}\NormalTok{(s13 }\SpecialCharTok{\%in\%}\NormalTok{ skujkoki,v13,}\DecValTok{0}\NormalTok{)}\SpecialCharTok{+}\FunctionTok{ifelse}\NormalTok{(s14 }\SpecialCharTok{\%in\%}\NormalTok{ skujkoki,v14,}\DecValTok{0}\NormalTok{),}
     \AttributeTok{kraja\_saurlapju=}\FunctionTok{ifelse}\NormalTok{(s10 }\SpecialCharTok{\%in\%}\NormalTok{ saurlapji,v10,}\DecValTok{0}\NormalTok{)}\SpecialCharTok{+}
      \FunctionTok{ifelse}\NormalTok{(s11 }\SpecialCharTok{\%in\%}\NormalTok{ saurlapji,v11,}\DecValTok{0}\NormalTok{)}\SpecialCharTok{+}\FunctionTok{ifelse}\NormalTok{(s12 }\SpecialCharTok{\%in\%}\NormalTok{ saurlapji,v12,}\DecValTok{0}\NormalTok{)}\SpecialCharTok{+}
      \FunctionTok{ifelse}\NormalTok{(s13 }\SpecialCharTok{\%in\%}\NormalTok{ saurlapji,v13,}\DecValTok{0}\NormalTok{)}\SpecialCharTok{+}\FunctionTok{ifelse}\NormalTok{(s14 }\SpecialCharTok{\%in\%}\NormalTok{ saurlapji,v14,}\DecValTok{0}\NormalTok{),}
     \AttributeTok{kraja\_platlapju=}\FunctionTok{ifelse}\NormalTok{(s10 }\SpecialCharTok{\%in\%}\NormalTok{ platlapji,v10,}\DecValTok{0}\NormalTok{)}\SpecialCharTok{+}
      \FunctionTok{ifelse}\NormalTok{(s11 }\SpecialCharTok{\%in\%}\NormalTok{ platlapji,v11,}\DecValTok{0}\NormalTok{)}\SpecialCharTok{+}\FunctionTok{ifelse}\NormalTok{(s12 }\SpecialCharTok{\%in\%}\NormalTok{ platlapji,v12,}\DecValTok{0}\NormalTok{)}\SpecialCharTok{+}
      \FunctionTok{ifelse}\NormalTok{(s13 }\SpecialCharTok{\%in\%}\NormalTok{ platlapji,v13,}\DecValTok{0}\NormalTok{)}\SpecialCharTok{+}\FunctionTok{ifelse}\NormalTok{(s14 }\SpecialCharTok{\%in\%}\NormalTok{ platlapji,v14,}\DecValTok{0}\NormalTok{)) }\SpecialCharTok{\%\textgreater{}\%} 
 \FunctionTok{mutate}\NormalTok{(}\AttributeTok{kopeja\_kraja=}\NormalTok{kraja\_skujkoku}\SpecialCharTok{+}\NormalTok{kraja\_platlapju}\SpecialCharTok{+}\NormalTok{kraja\_saurlapju) }\SpecialCharTok{\%\textgreater{}\%} 
 \FunctionTok{mutate}\NormalTok{(}\AttributeTok{tips=}\FunctionTok{ifelse}\NormalTok{(kraja\_skujkoku}\SpecialCharTok{/}\NormalTok{kopeja\_kraja}\SpecialCharTok{\textgreater{}=}\FloatTok{0.75}\NormalTok{,}\StringTok{"skujkoku"}\NormalTok{,}
           \FunctionTok{ifelse}\NormalTok{(kraja\_saurlapju}\SpecialCharTok{/}\NormalTok{kopeja\_kraja}\SpecialCharTok{\textgreater{}=}\FloatTok{0.75}\NormalTok{,}\StringTok{"saurlapju"}\NormalTok{,}
              \FunctionTok{ifelse}\NormalTok{(kraja\_platlapju}\SpecialCharTok{/}\NormalTok{kopeja\_kraja}\SpecialCharTok{\textgreater{}}\FloatTok{0.5}\NormalTok{,}\StringTok{"platlapju"}\NormalTok{,}
                  \StringTok{"jauktu koku"}\NormalTok{))))}
\NormalTok{nogabali}\OtherTok{=}\NormalTok{mvr }\SpecialCharTok{\%\textgreater{}\%} 
 \FunctionTok{filter}\NormalTok{(zkat}\SpecialCharTok{==}\StringTok{"10"}\SpecialCharTok{\&}\NormalTok{tips}\SpecialCharTok{==}\StringTok{"saurlapju"}\SpecialCharTok{\&}\NormalTok{(vgr}\SpecialCharTok{==}\StringTok{"1"}\SpecialCharTok{|}\NormalTok{vgr}\SpecialCharTok{==}\StringTok{"2"}\SpecialCharTok{|}\NormalTok{vgr}\SpecialCharTok{==}\StringTok{"3"}\NormalTok{))}

\NormalTok{p2i\_rez}\OtherTok{=}\NormalTok{egvtools}\SpecialCharTok{::}\FunctionTok{polygon2input}\NormalTok{(}\AttributeTok{vector\_data =}\NormalTok{ nogabali,}
                \AttributeTok{template\_path =} \StringTok{"./Templates/TemplateRasters/LV10m\_10km.tif"}\NormalTok{,}
                \AttributeTok{out\_path =} \StringTok{"./RasterGrids\_10m/2024/"}\NormalTok{,}
                \AttributeTok{file\_name =} \StringTok{"ForestsTreesAge\_BorealDeciduousYoung\_input.tif"}\NormalTok{,}
                \AttributeTok{value\_field =} \StringTok{"yes"}\NormalTok{,}
                \AttributeTok{restrict\_to =}\NormalTok{ clearcut\_mask,}
                \AttributeTok{restrict\_values =} \DecValTok{0}\NormalTok{,}
                \AttributeTok{prepare=}\ConstantTok{FALSE}\NormalTok{,}
                \AttributeTok{background\_raster =} \StringTok{"./Templates/TemplateRasters/nulls\_LV10m\_10km.tif"}\NormalTok{,}
                \AttributeTok{plot\_result =} \ConstantTok{TRUE}\NormalTok{)}
\NormalTok{p2i\_rez}
\NormalTok{i2e\_rez}\OtherTok{=}\NormalTok{egvtools}\SpecialCharTok{::}\FunctionTok{input2egv}\NormalTok{(}\AttributeTok{input=}\FunctionTok{paste0}\NormalTok{(}\StringTok{"./RasterGrids\_10m/2024/"}\NormalTok{,}
                     \StringTok{"ForestsTreesAge\_BorealDeciduousYoung\_input.tif"}\NormalTok{),}
              \AttributeTok{egv\_template=} \StringTok{"./Templates/TemplateRasters/LV100m\_10km.tif"}\NormalTok{,}
              \AttributeTok{summary\_function =} \StringTok{"average"}\NormalTok{,}
              \AttributeTok{missing\_job =} \StringTok{"FillOutput"}\NormalTok{,}
              \AttributeTok{outlocation =} \StringTok{"./RasterGrids\_100m/2024/RAW/"}\NormalTok{,}
              \AttributeTok{outfilename =} \StringTok{"ForestsTreesAge\_BorealDeciduousYoung\_cell.tif"}\NormalTok{,}
              \AttributeTok{layername =} \StringTok{"egv\_348"}\NormalTok{,}
              \AttributeTok{idw\_weight =} \DecValTok{2}\NormalTok{,}
              \AttributeTok{plot\_gaps =} \ConstantTok{FALSE}\NormalTok{,}\AttributeTok{plot\_final =} \ConstantTok{TRUE}\NormalTok{)}
\NormalTok{i2e\_rez}
\FunctionTok{rm}\NormalTok{(nogabali)}
\FunctionTok{rm}\NormalTok{(p2i\_rez)}
\FunctionTok{rm}\NormalTok{(i2e\_rez)}
\FunctionTok{unlink}\NormalTok{(}\StringTok{"./RasterGrids\_10m/2024/ForestsTreesAge\_BorealDeciduousYoung\_input.tif"}\NormalTok{)}

\CommentTok{\# standardisation {-}{-}{-}{-}}
\ControlFlowTok{if}\NormalTok{(}\SpecialCharTok{!}\FunctionTok{require}\NormalTok{(terra)) \{}\FunctionTok{install.packages}\NormalTok{(}\StringTok{"terra"}\NormalTok{); }\FunctionTok{require}\NormalTok{(terra)\}}
\ControlFlowTok{if}\NormalTok{(}\SpecialCharTok{!}\FunctionTok{require}\NormalTok{(tidyverse)) \{}\FunctionTok{install.packages}\NormalTok{(}\StringTok{"tidyverse"}\NormalTok{); }\FunctionTok{require}\NormalTok{(tidyverse)\}}

\NormalTok{nosaukums}\OtherTok{=}\StringTok{"ForestsTreesAge\_BorealDeciduousYoung\_cell.tif"}
\NormalTok{ielasisanas\_cels}\OtherTok{=}\FunctionTok{paste0}\NormalTok{(}\StringTok{"./RasterGrids\_100m/2024/RAW/"}\NormalTok{,nosaukums)}
\NormalTok{saglabasanas\_cels}\OtherTok{=}\FunctionTok{paste0}\NormalTok{(}\StringTok{"./RasterGrids\_100m/2024/Scaled/"}\NormalTok{,nosaukums)}
\NormalTok{slanis}\OtherTok{=}\FunctionTok{rast}\NormalTok{(ielasisanas\_cels)}
\NormalTok{videjais}\OtherTok{=}\FunctionTok{global}\NormalTok{(slanis,}\AttributeTok{fun=}\StringTok{"mean"}\NormalTok{,}\AttributeTok{na.rm=}\ConstantTok{TRUE}\NormalTok{)}
\NormalTok{centrets}\OtherTok{=}\NormalTok{slanis}\SpecialCharTok{{-}}\NormalTok{videjais[,}\DecValTok{1}\NormalTok{]}
\NormalTok{standartnovirze}\OtherTok{=}\NormalTok{terra}\SpecialCharTok{::}\FunctionTok{global}\NormalTok{(centrets,}\AttributeTok{fun=}\StringTok{"rms"}\NormalTok{,}\AttributeTok{na.rm=}\ConstantTok{TRUE}\NormalTok{)}
\NormalTok{merogots}\OtherTok{=}\NormalTok{centrets}\SpecialCharTok{/}\NormalTok{standartnovirze[,}\DecValTok{1}\NormalTok{]}
\FunctionTok{writeRaster}\NormalTok{(merogots,}
      \AttributeTok{filename=}\NormalTok{saglabasanas\_cels,}
      \AttributeTok{overwrite=}\ConstantTok{TRUE}\NormalTok{)}
\end{Highlighting}
\end{Shaded}

\section{ForestsTreesAge\_BorealDeciduousYoung\_r500}\label{ch06.349}

\textbf{filename:} \texttt{ForestsTreesAge\_BorealDeciduousYoung\_r500.tif}

\textbf{layername:} \texttt{egv\_349}

\textbf{English name:} Fractional cover of Young (pre-rotation age) Boreal Deciduous
Forests within the 0.5 km landscape

\textbf{Latvian name:} Jaunu (pirms cirtmeta) šaurlapju mežu platības īpatsvars 0,5
km ainavā

\textbf{Procedure:} The cover fraction within a radius of 500 m around the analysis grid cell is
calculated as the area-weighted sum of the \hyperref[ch06.348]{analysis cells} inside the
buffer, using the workflow \texttt{egvtools::radius\_function()}. During the calculation of the landscape metric,
inverse distance weighted (power = 2) gap filling on the output is applied
to ensure no missing values at the edges. Then the layer is rewritten to set
its name. Finally, the layer is standardised by subtracting the arithmetic
mean and dividing by the root mean squared error.

\begin{Shaded}
\begin{Highlighting}[]
\CommentTok{\# libs {-}{-}{-}{-}}
\ControlFlowTok{if}\NormalTok{(}\SpecialCharTok{!}\FunctionTok{require}\NormalTok{(terra)) \{}\FunctionTok{install.packages}\NormalTok{(}\StringTok{"terra"}\NormalTok{); }\FunctionTok{require}\NormalTok{(terra)\}}
\ControlFlowTok{if}\NormalTok{(}\SpecialCharTok{!}\FunctionTok{require}\NormalTok{(egvtools)) \{remotes}\SpecialCharTok{::}\FunctionTok{install\_github}\NormalTok{(}\StringTok{"aavotins/egvtools"}\NormalTok{); }\FunctionTok{require}\NormalTok{(egvtools)\}}


\CommentTok{\# Templates {-}{-}{-}{-}{-}}
\NormalTok{template100}\OtherTok{=}\FunctionTok{rast}\NormalTok{(}\StringTok{"./Templates/TemplateRasters/LV100m\_10km.tif"}\NormalTok{)}

\CommentTok{\# radii {-}{-}{-}{-}}
\FunctionTok{radius\_function}\NormalTok{(}
 \AttributeTok{kvadrati\_path =} \StringTok{"./Templates/TemplateGrids/tiles/"}\NormalTok{,}
 \AttributeTok{radii\_path   =} \StringTok{"./Templates/TemplateGridPoints/tiles/"}\NormalTok{,}
 \AttributeTok{tikls100\_path =} \StringTok{"./Templates/TemplateGrids/tikls100\_sauzeme.parquet"}\NormalTok{,}
 \AttributeTok{template\_path =} \StringTok{"./Templates/TemplateRasters/LV100m\_10km.tif"}\NormalTok{,}
 \AttributeTok{input\_layers  =} \FunctionTok{c}\NormalTok{(}\StringTok{"./RasterGrids\_100m/2024/RAW/ForestsTreesAge\_BorealDeciduousYoung\_cell.tif"}\NormalTok{),}
 \AttributeTok{layer\_prefixes =} \FunctionTok{c}\NormalTok{(}\StringTok{"ForestsTreesAge\_BorealDeciduousYoung"}\NormalTok{),}
 \AttributeTok{output\_dir   =} \StringTok{"./RasterGrids\_100m/2024/RAW/"}\NormalTok{,}
 \AttributeTok{n\_workers   =} \DecValTok{6}\NormalTok{,}
 \AttributeTok{radii     =} \FunctionTok{c}\NormalTok{(}\StringTok{"r500"}\NormalTok{),}
 \AttributeTok{radius\_mode  =} \StringTok{"sparse"}\NormalTok{,}
 \AttributeTok{extract\_fun  =} \StringTok{"mean"}\NormalTok{,}
 \AttributeTok{fill\_missing  =} \ConstantTok{TRUE}\NormalTok{,}
 \AttributeTok{IDW\_weight   =} \DecValTok{2}\NormalTok{,}
 \AttributeTok{future\_max\_size =} \DecValTok{40} \SpecialCharTok{*} \DecValTok{1024}\SpecialCharTok{\^{}}\DecValTok{3}\NormalTok{)}


\CommentTok{\# ForestsTreesAge\_BorealDeciduousYoung\_r500.tif egv\_349}
\NormalTok{slanis}\OtherTok{=}\FunctionTok{rast}\NormalTok{(}\StringTok{"./RasterGrids\_100m/2024/RAW/ForestsTreesAge\_BorealDeciduousYoung\_r500.tif"}\NormalTok{)}
\FunctionTok{names}\NormalTok{(slanis)}\OtherTok{=}\StringTok{"egv\_349"}
\NormalTok{slanis2}\OtherTok{=}\FunctionTok{project}\NormalTok{(slanis,template100)}
\FunctionTok{writeRaster}\NormalTok{(slanis2,}
      \StringTok{"./RasterGrids\_100m/2024/RAW/ForestsTreesAge\_BorealDeciduousYoung\_r500.tif"}\NormalTok{,}
      \AttributeTok{overwrite=}\ConstantTok{TRUE}\NormalTok{)}

\CommentTok{\# standardisation {-}{-}{-}{-}}
\ControlFlowTok{if}\NormalTok{(}\SpecialCharTok{!}\FunctionTok{require}\NormalTok{(terra)) \{}\FunctionTok{install.packages}\NormalTok{(}\StringTok{"terra"}\NormalTok{); }\FunctionTok{require}\NormalTok{(terra)\}}
\ControlFlowTok{if}\NormalTok{(}\SpecialCharTok{!}\FunctionTok{require}\NormalTok{(tidyverse)) \{}\FunctionTok{install.packages}\NormalTok{(}\StringTok{"tidyverse"}\NormalTok{); }\FunctionTok{require}\NormalTok{(tidyverse)\}}

\NormalTok{nosaukums}\OtherTok{=}\StringTok{"ForestsTreesAge\_BorealDeciduousYoung\_r500.tif"}
\NormalTok{ielasisanas\_cels}\OtherTok{=}\FunctionTok{paste0}\NormalTok{(}\StringTok{"./RasterGrids\_100m/2024/RAW/"}\NormalTok{,nosaukums)}
\NormalTok{saglabasanas\_cels}\OtherTok{=}\FunctionTok{paste0}\NormalTok{(}\StringTok{"./RasterGrids\_100m/2024/Scaled/"}\NormalTok{,nosaukums)}
\NormalTok{slanis}\OtherTok{=}\FunctionTok{rast}\NormalTok{(ielasisanas\_cels)}
\NormalTok{videjais}\OtherTok{=}\FunctionTok{global}\NormalTok{(slanis,}\AttributeTok{fun=}\StringTok{"mean"}\NormalTok{,}\AttributeTok{na.rm=}\ConstantTok{TRUE}\NormalTok{)}
\NormalTok{centrets}\OtherTok{=}\NormalTok{slanis}\SpecialCharTok{{-}}\NormalTok{videjais[,}\DecValTok{1}\NormalTok{]}
\NormalTok{standartnovirze}\OtherTok{=}\NormalTok{terra}\SpecialCharTok{::}\FunctionTok{global}\NormalTok{(centrets,}\AttributeTok{fun=}\StringTok{"rms"}\NormalTok{,}\AttributeTok{na.rm=}\ConstantTok{TRUE}\NormalTok{)}
\NormalTok{merogots}\OtherTok{=}\NormalTok{centrets}\SpecialCharTok{/}\NormalTok{standartnovirze[,}\DecValTok{1}\NormalTok{]}
\FunctionTok{writeRaster}\NormalTok{(merogots,}
      \AttributeTok{filename=}\NormalTok{saglabasanas\_cels,}
      \AttributeTok{overwrite=}\ConstantTok{TRUE}\NormalTok{)}
\end{Highlighting}
\end{Shaded}

\section{ForestsTreesAge\_BorealDeciduousYoung\_r1250}\label{ch06.350}

\textbf{filename:} \texttt{ForestsTreesAge\_BorealDeciduousYoung\_r1250.tif}

\textbf{layername:} \texttt{egv\_350}

\textbf{English name:} Fractional cover of Young (pre-rotation age) Boreal Deciduous
Forests within the 1.25 km landscape

\textbf{Latvian name:} Jaunu (pirms cirtmeta) šaurlapju mežu platības īpatsvars 1,25
km ainavā

\textbf{Procedure:} The cover fraction within a radius of 1250 m around the analysis grid cell
is calculated as the area-weighted sum of the \hyperref[ch06.348]{analysis cells} inside
the buffer, using the workflow \texttt{egvtools::radius\_function()}. During the calculation of the landscape
metric, inverse distance weighted (power = 2) gap filling on the output is
applied to ensure no missing values at the edges. Then the layer is
rewritten to set its name. Finally, the layer is standardised by
subtracting the arithmetic mean and dividing by the root mean squared error.

\begin{Shaded}
\begin{Highlighting}[]
\CommentTok{\# libs {-}{-}{-}{-}}
\ControlFlowTok{if}\NormalTok{(}\SpecialCharTok{!}\FunctionTok{require}\NormalTok{(terra)) \{}\FunctionTok{install.packages}\NormalTok{(}\StringTok{"terra"}\NormalTok{); }\FunctionTok{require}\NormalTok{(terra)\}}
\ControlFlowTok{if}\NormalTok{(}\SpecialCharTok{!}\FunctionTok{require}\NormalTok{(egvtools)) \{remotes}\SpecialCharTok{::}\FunctionTok{install\_github}\NormalTok{(}\StringTok{"aavotins/egvtools"}\NormalTok{); }\FunctionTok{require}\NormalTok{(egvtools)\}}


\CommentTok{\# Templates {-}{-}{-}{-}{-}}
\NormalTok{template100}\OtherTok{=}\FunctionTok{rast}\NormalTok{(}\StringTok{"./Templates/TemplateRasters/LV100m\_10km.tif"}\NormalTok{)}

\CommentTok{\# radii {-}{-}{-}{-}}
\FunctionTok{radius\_function}\NormalTok{(}
 \AttributeTok{kvadrati\_path =} \StringTok{"./Templates/TemplateGrids/tiles/"}\NormalTok{,}
 \AttributeTok{radii\_path   =} \StringTok{"./Templates/TemplateGridPoints/tiles/"}\NormalTok{,}
 \AttributeTok{tikls100\_path =} \StringTok{"./Templates/TemplateGrids/tikls100\_sauzeme.parquet"}\NormalTok{,}
 \AttributeTok{template\_path =} \StringTok{"./Templates/TemplateRasters/LV100m\_10km.tif"}\NormalTok{,}
 \AttributeTok{input\_layers  =} \FunctionTok{c}\NormalTok{(}\StringTok{"./RasterGrids\_100m/2024/RAW/ForestsTreesAge\_BorealDeciduousYoung\_cell.tif"}\NormalTok{),}
 \AttributeTok{layer\_prefixes =} \FunctionTok{c}\NormalTok{(}\StringTok{"ForestsTreesAge\_BorealDeciduousYoung"}\NormalTok{),}
 \AttributeTok{output\_dir   =} \StringTok{"./RasterGrids\_100m/2024/RAW/"}\NormalTok{,}
 \AttributeTok{n\_workers   =} \DecValTok{6}\NormalTok{,}
 \AttributeTok{radii     =} \FunctionTok{c}\NormalTok{(}\StringTok{"r1250"}\NormalTok{),}
 \AttributeTok{radius\_mode  =} \StringTok{"sparse"}\NormalTok{,}
 \AttributeTok{extract\_fun  =} \StringTok{"mean"}\NormalTok{,}
 \AttributeTok{fill\_missing  =} \ConstantTok{TRUE}\NormalTok{,}
 \AttributeTok{IDW\_weight   =} \DecValTok{2}\NormalTok{,}
 \AttributeTok{future\_max\_size =} \DecValTok{40} \SpecialCharTok{*} \DecValTok{1024}\SpecialCharTok{\^{}}\DecValTok{3}\NormalTok{)}


\CommentTok{\# ForestsTreesAge\_BorealDeciduousYoung\_r1250.tif    egv\_350}
\NormalTok{slanis}\OtherTok{=}\FunctionTok{rast}\NormalTok{(}\StringTok{"./RasterGrids\_100m/2024/RAW/ForestsTreesAge\_BorealDeciduousYoung\_r1250.tif"}\NormalTok{)}
\FunctionTok{names}\NormalTok{(slanis)}\OtherTok{=}\StringTok{"egv\_350"}
\NormalTok{slanis2}\OtherTok{=}\FunctionTok{project}\NormalTok{(slanis,template100)}
\FunctionTok{writeRaster}\NormalTok{(slanis2,}
      \StringTok{"./RasterGrids\_100m/2024/RAW/ForestsTreesAge\_BorealDeciduousYoung\_r1250.tif"}\NormalTok{,}
      \AttributeTok{overwrite=}\ConstantTok{TRUE}\NormalTok{)}

\CommentTok{\# standardisation {-}{-}{-}{-}}
\ControlFlowTok{if}\NormalTok{(}\SpecialCharTok{!}\FunctionTok{require}\NormalTok{(terra)) \{}\FunctionTok{install.packages}\NormalTok{(}\StringTok{"terra"}\NormalTok{); }\FunctionTok{require}\NormalTok{(terra)\}}
\ControlFlowTok{if}\NormalTok{(}\SpecialCharTok{!}\FunctionTok{require}\NormalTok{(tidyverse)) \{}\FunctionTok{install.packages}\NormalTok{(}\StringTok{"tidyverse"}\NormalTok{); }\FunctionTok{require}\NormalTok{(tidyverse)\}}

\NormalTok{nosaukums}\OtherTok{=}\StringTok{"ForestsTreesAge\_BorealDeciduousYoung\_r1250.tif"}
\NormalTok{ielasisanas\_cels}\OtherTok{=}\FunctionTok{paste0}\NormalTok{(}\StringTok{"./RasterGrids\_100m/2024/RAW/"}\NormalTok{,nosaukums)}
\NormalTok{saglabasanas\_cels}\OtherTok{=}\FunctionTok{paste0}\NormalTok{(}\StringTok{"./RasterGrids\_100m/2024/Scaled/"}\NormalTok{,nosaukums)}
\NormalTok{slanis}\OtherTok{=}\FunctionTok{rast}\NormalTok{(ielasisanas\_cels)}
\NormalTok{videjais}\OtherTok{=}\FunctionTok{global}\NormalTok{(slanis,}\AttributeTok{fun=}\StringTok{"mean"}\NormalTok{,}\AttributeTok{na.rm=}\ConstantTok{TRUE}\NormalTok{)}
\NormalTok{centrets}\OtherTok{=}\NormalTok{slanis}\SpecialCharTok{{-}}\NormalTok{videjais[,}\DecValTok{1}\NormalTok{]}
\NormalTok{standartnovirze}\OtherTok{=}\NormalTok{terra}\SpecialCharTok{::}\FunctionTok{global}\NormalTok{(centrets,}\AttributeTok{fun=}\StringTok{"rms"}\NormalTok{,}\AttributeTok{na.rm=}\ConstantTok{TRUE}\NormalTok{)}
\NormalTok{merogots}\OtherTok{=}\NormalTok{centrets}\SpecialCharTok{/}\NormalTok{standartnovirze[,}\DecValTok{1}\NormalTok{]}
\FunctionTok{writeRaster}\NormalTok{(merogots,}
      \AttributeTok{filename=}\NormalTok{saglabasanas\_cels,}
      \AttributeTok{overwrite=}\ConstantTok{TRUE}\NormalTok{)}
\end{Highlighting}
\end{Shaded}

\section{ForestsTreesAge\_BorealDeciduousYoung\_r3000}\label{ch06.351}

\textbf{filename:} \texttt{ForestsTreesAge\_BorealDeciduousYoung\_r3000.tif}

\textbf{layername:} \texttt{egv\_351}

\textbf{English name:} Fractional cover of Young (pre-rotation age) Boreal Deciduous
Forests within the 3 km landscape

\textbf{Latvian name:} Jaunu (pirms cirtmeta) šaurlapju mežu platības īpatsvars 3 km
ainavā

\textbf{Procedure:} The cover fraction within a radius of 3000 m around the analysis grid cell
is calculated as the area-weighted sum of the \hyperref[ch06.348]{analysis cells} inside
the buffer, using the workflow \texttt{egvtools::radius\_function()}. During the calculation of the landscape
metric, inverse distance weighted (power = 2) gap filling on the output is
applied to ensure no missing values at the edges. Then the layer is
rewritten to set its name. Finally, the layer is standardised by
subtracting the arithmetic mean and dividing by the root mean squared error.

\begin{Shaded}
\begin{Highlighting}[]
\CommentTok{\# libs {-}{-}{-}{-}}
\ControlFlowTok{if}\NormalTok{(}\SpecialCharTok{!}\FunctionTok{require}\NormalTok{(terra)) \{}\FunctionTok{install.packages}\NormalTok{(}\StringTok{"terra"}\NormalTok{); }\FunctionTok{require}\NormalTok{(terra)\}}
\ControlFlowTok{if}\NormalTok{(}\SpecialCharTok{!}\FunctionTok{require}\NormalTok{(egvtools)) \{remotes}\SpecialCharTok{::}\FunctionTok{install\_github}\NormalTok{(}\StringTok{"aavotins/egvtools"}\NormalTok{); }\FunctionTok{require}\NormalTok{(egvtools)\}}


\CommentTok{\# Templates {-}{-}{-}{-}{-}}
\NormalTok{template100}\OtherTok{=}\FunctionTok{rast}\NormalTok{(}\StringTok{"./Templates/TemplateRasters/LV100m\_10km.tif"}\NormalTok{)}

\CommentTok{\# radii {-}{-}{-}{-}}
\FunctionTok{radius\_function}\NormalTok{(}
 \AttributeTok{kvadrati\_path =} \StringTok{"./Templates/TemplateGrids/tiles/"}\NormalTok{,}
 \AttributeTok{radii\_path   =} \StringTok{"./Templates/TemplateGridPoints/tiles/"}\NormalTok{,}
 \AttributeTok{tikls100\_path =} \StringTok{"./Templates/TemplateGrids/tikls100\_sauzeme.parquet"}\NormalTok{,}
 \AttributeTok{template\_path =} \StringTok{"./Templates/TemplateRasters/LV100m\_10km.tif"}\NormalTok{,}
 \AttributeTok{input\_layers  =} \FunctionTok{c}\NormalTok{(}\StringTok{"./RasterGrids\_100m/2024/RAW/ForestsTreesAge\_BorealDeciduousYoung\_cell.tif"}\NormalTok{),}
 \AttributeTok{layer\_prefixes =} \FunctionTok{c}\NormalTok{(}\StringTok{"ForestsTreesAge\_BorealDeciduousYoung"}\NormalTok{),}
 \AttributeTok{output\_dir   =} \StringTok{"./RasterGrids\_100m/2024/RAW/"}\NormalTok{,}
 \AttributeTok{n\_workers   =} \DecValTok{6}\NormalTok{,}
 \AttributeTok{radii     =} \FunctionTok{c}\NormalTok{(}\StringTok{"r3000"}\NormalTok{),}
 \AttributeTok{radius\_mode  =} \StringTok{"sparse"}\NormalTok{,}
 \AttributeTok{extract\_fun  =} \StringTok{"mean"}\NormalTok{,}
 \AttributeTok{fill\_missing  =} \ConstantTok{TRUE}\NormalTok{,}
 \AttributeTok{IDW\_weight   =} \DecValTok{2}\NormalTok{,}
 \AttributeTok{future\_max\_size =} \DecValTok{40} \SpecialCharTok{*} \DecValTok{1024}\SpecialCharTok{\^{}}\DecValTok{3}\NormalTok{)}


\CommentTok{\# ForestsTreesAge\_BorealDeciduousYoung\_r3000.tif    egv\_351}
\NormalTok{slanis}\OtherTok{=}\FunctionTok{rast}\NormalTok{(}\StringTok{"./RasterGrids\_100m/2024/RAW/ForestsTreesAge\_BorealDeciduousYoung\_r3000.tif"}\NormalTok{)}
\FunctionTok{names}\NormalTok{(slanis)}\OtherTok{=}\StringTok{"egv\_351"}
\NormalTok{slanis2}\OtherTok{=}\FunctionTok{project}\NormalTok{(slanis,template100)}
\FunctionTok{writeRaster}\NormalTok{(slanis2,}
      \StringTok{"./RasterGrids\_100m/2024/RAW/ForestsTreesAge\_BorealDeciduousYoung\_r3000.tif"}\NormalTok{,}
      \AttributeTok{overwrite=}\ConstantTok{TRUE}\NormalTok{)}

\CommentTok{\# standardisation {-}{-}{-}{-}}
\ControlFlowTok{if}\NormalTok{(}\SpecialCharTok{!}\FunctionTok{require}\NormalTok{(terra)) \{}\FunctionTok{install.packages}\NormalTok{(}\StringTok{"terra"}\NormalTok{); }\FunctionTok{require}\NormalTok{(terra)\}}
\ControlFlowTok{if}\NormalTok{(}\SpecialCharTok{!}\FunctionTok{require}\NormalTok{(tidyverse)) \{}\FunctionTok{install.packages}\NormalTok{(}\StringTok{"tidyverse"}\NormalTok{); }\FunctionTok{require}\NormalTok{(tidyverse)\}}

\NormalTok{nosaukums}\OtherTok{=}\StringTok{"ForestsTreesAge\_BorealDeciduousYoung\_r3000.tif"}
\NormalTok{ielasisanas\_cels}\OtherTok{=}\FunctionTok{paste0}\NormalTok{(}\StringTok{"./RasterGrids\_100m/2024/RAW/"}\NormalTok{,nosaukums)}
\NormalTok{saglabasanas\_cels}\OtherTok{=}\FunctionTok{paste0}\NormalTok{(}\StringTok{"./RasterGrids\_100m/2024/Scaled/"}\NormalTok{,nosaukums)}
\NormalTok{slanis}\OtherTok{=}\FunctionTok{rast}\NormalTok{(ielasisanas\_cels)}
\NormalTok{videjais}\OtherTok{=}\FunctionTok{global}\NormalTok{(slanis,}\AttributeTok{fun=}\StringTok{"mean"}\NormalTok{,}\AttributeTok{na.rm=}\ConstantTok{TRUE}\NormalTok{)}
\NormalTok{centrets}\OtherTok{=}\NormalTok{slanis}\SpecialCharTok{{-}}\NormalTok{videjais[,}\DecValTok{1}\NormalTok{]}
\NormalTok{standartnovirze}\OtherTok{=}\NormalTok{terra}\SpecialCharTok{::}\FunctionTok{global}\NormalTok{(centrets,}\AttributeTok{fun=}\StringTok{"rms"}\NormalTok{,}\AttributeTok{na.rm=}\ConstantTok{TRUE}\NormalTok{)}
\NormalTok{merogots}\OtherTok{=}\NormalTok{centrets}\SpecialCharTok{/}\NormalTok{standartnovirze[,}\DecValTok{1}\NormalTok{]}
\FunctionTok{writeRaster}\NormalTok{(merogots,}
      \AttributeTok{filename=}\NormalTok{saglabasanas\_cels,}
      \AttributeTok{overwrite=}\ConstantTok{TRUE}\NormalTok{)}
\end{Highlighting}
\end{Shaded}

\section{ForestsTreesAge\_BorealDeciduousYoung\_r10000}\label{ch06.352}

\textbf{filename:} \texttt{ForestsTreesAge\_BorealDeciduousYoung\_r10000.tif}

\textbf{layername:} \texttt{egv\_352}

\textbf{English name:} Fractional cover of Young (pre-rotation age) Boreal Deciduous
Forests within the 10 km landscape

\textbf{Latvian name:} Jaunu (pirms cirtmeta) šaurlapju mežu platības īpatsvars 10 km
ainavā

\textbf{Procedure:} The cover fraction within a radius of 10000 m around the analysis grid cell
is calculated as the area-weighted sum of the \hyperref[ch06.348]{analysis cells} inside
the buffer, using the workflow \texttt{egvtools::radius\_function()}. During the calculation of the landscape
metric, inverse distance weighted (power = 2) gap filling on the output is
applied to ensure no missing values at the edges. Then the layer is
rewritten to set its name. Finally, the layer is standardised by
subtracting the arithmetic mean and dividing by the root mean squared error.

\begin{Shaded}
\begin{Highlighting}[]
\CommentTok{\# libs {-}{-}{-}{-}}
\ControlFlowTok{if}\NormalTok{(}\SpecialCharTok{!}\FunctionTok{require}\NormalTok{(terra)) \{}\FunctionTok{install.packages}\NormalTok{(}\StringTok{"terra"}\NormalTok{); }\FunctionTok{require}\NormalTok{(terra)\}}
\ControlFlowTok{if}\NormalTok{(}\SpecialCharTok{!}\FunctionTok{require}\NormalTok{(egvtools)) \{remotes}\SpecialCharTok{::}\FunctionTok{install\_github}\NormalTok{(}\StringTok{"aavotins/egvtools"}\NormalTok{); }\FunctionTok{require}\NormalTok{(egvtools)\}}


\CommentTok{\# Templates {-}{-}{-}{-}{-}}
\NormalTok{template100}\OtherTok{=}\FunctionTok{rast}\NormalTok{(}\StringTok{"./Templates/TemplateRasters/LV100m\_10km.tif"}\NormalTok{)}

\CommentTok{\# radii {-}{-}{-}{-}}
\FunctionTok{radius\_function}\NormalTok{(}
 \AttributeTok{kvadrati\_path =} \StringTok{"./Templates/TemplateGrids/tiles/"}\NormalTok{,}
 \AttributeTok{radii\_path   =} \StringTok{"./Templates/TemplateGridPoints/tiles/"}\NormalTok{,}
 \AttributeTok{tikls100\_path =} \StringTok{"./Templates/TemplateGrids/tikls100\_sauzeme.parquet"}\NormalTok{,}
 \AttributeTok{template\_path =} \StringTok{"./Templates/TemplateRasters/LV100m\_10km.tif"}\NormalTok{,}
 \AttributeTok{input\_layers  =} \FunctionTok{c}\NormalTok{(}\StringTok{"./RasterGrids\_100m/2024/RAW/ForestsTreesAge\_BorealDeciduousYoung\_cell.tif"}\NormalTok{),}
 \AttributeTok{layer\_prefixes =} \FunctionTok{c}\NormalTok{(}\StringTok{"ForestsTreesAge\_BorealDeciduousYoung"}\NormalTok{),}
 \AttributeTok{output\_dir   =} \StringTok{"./RasterGrids\_100m/2024/RAW/"}\NormalTok{,}
 \AttributeTok{n\_workers   =} \DecValTok{6}\NormalTok{,}
 \AttributeTok{radii     =} \FunctionTok{c}\NormalTok{(}\StringTok{"r10000"}\NormalTok{),}
 \AttributeTok{radius\_mode  =} \StringTok{"sparse"}\NormalTok{,}
 \AttributeTok{extract\_fun  =} \StringTok{"mean"}\NormalTok{,}
 \AttributeTok{fill\_missing  =} \ConstantTok{TRUE}\NormalTok{,}
 \AttributeTok{IDW\_weight   =} \DecValTok{2}\NormalTok{,}
 \AttributeTok{future\_max\_size =} \DecValTok{40} \SpecialCharTok{*} \DecValTok{1024}\SpecialCharTok{\^{}}\DecValTok{3}\NormalTok{)}


\CommentTok{\# ForestsTreesAge\_BorealDeciduousYoung\_r10000.tif   egv\_352}
\NormalTok{slanis}\OtherTok{=}\FunctionTok{rast}\NormalTok{(}\StringTok{"./RasterGrids\_100m/2024/RAW/ForestsTreesAge\_BorealDeciduousYoung\_r10000.tif"}\NormalTok{)}
\FunctionTok{names}\NormalTok{(slanis)}\OtherTok{=}\StringTok{"egv\_352"}
\NormalTok{slanis2}\OtherTok{=}\FunctionTok{project}\NormalTok{(slanis,template100)}
\FunctionTok{writeRaster}\NormalTok{(slanis2,}
      \StringTok{"./RasterGrids\_100m/2024/RAW/ForestsTreesAge\_BorealDeciduousYoung\_r10000.tif"}\NormalTok{,}
      \AttributeTok{overwrite=}\ConstantTok{TRUE}\NormalTok{)}

\CommentTok{\# standardisation {-}{-}{-}{-}}
\ControlFlowTok{if}\NormalTok{(}\SpecialCharTok{!}\FunctionTok{require}\NormalTok{(terra)) \{}\FunctionTok{install.packages}\NormalTok{(}\StringTok{"terra"}\NormalTok{); }\FunctionTok{require}\NormalTok{(terra)\}}
\ControlFlowTok{if}\NormalTok{(}\SpecialCharTok{!}\FunctionTok{require}\NormalTok{(tidyverse)) \{}\FunctionTok{install.packages}\NormalTok{(}\StringTok{"tidyverse"}\NormalTok{); }\FunctionTok{require}\NormalTok{(tidyverse)\}}

\NormalTok{nosaukums}\OtherTok{=}\StringTok{"ForestsTreesAge\_BorealDeciduousYoung\_r10000.tif"}
\NormalTok{ielasisanas\_cels}\OtherTok{=}\FunctionTok{paste0}\NormalTok{(}\StringTok{"./RasterGrids\_100m/2024/RAW/"}\NormalTok{,nosaukums)}
\NormalTok{saglabasanas\_cels}\OtherTok{=}\FunctionTok{paste0}\NormalTok{(}\StringTok{"./RasterGrids\_100m/2024/Scaled/"}\NormalTok{,nosaukums)}
\NormalTok{slanis}\OtherTok{=}\FunctionTok{rast}\NormalTok{(ielasisanas\_cels)}
\NormalTok{videjais}\OtherTok{=}\FunctionTok{global}\NormalTok{(slanis,}\AttributeTok{fun=}\StringTok{"mean"}\NormalTok{,}\AttributeTok{na.rm=}\ConstantTok{TRUE}\NormalTok{)}
\NormalTok{centrets}\OtherTok{=}\NormalTok{slanis}\SpecialCharTok{{-}}\NormalTok{videjais[,}\DecValTok{1}\NormalTok{]}
\NormalTok{standartnovirze}\OtherTok{=}\NormalTok{terra}\SpecialCharTok{::}\FunctionTok{global}\NormalTok{(centrets,}\AttributeTok{fun=}\StringTok{"rms"}\NormalTok{,}\AttributeTok{na.rm=}\ConstantTok{TRUE}\NormalTok{)}
\NormalTok{merogots}\OtherTok{=}\NormalTok{centrets}\SpecialCharTok{/}\NormalTok{standartnovirze[,}\DecValTok{1}\NormalTok{]}
\FunctionTok{writeRaster}\NormalTok{(merogots,}
      \AttributeTok{filename=}\NormalTok{saglabasanas\_cels,}
      \AttributeTok{overwrite=}\ConstantTok{TRUE}\NormalTok{)}
\end{Highlighting}
\end{Shaded}

\section{ForestsTreesAge\_ConiferousOld\_cell}\label{ch06.353}

\textbf{filename:} \texttt{ForestsTreesAge\_ConiferousOld\_cell.tif}

\textbf{layername:} \texttt{egv\_353}

\textbf{English name:} Fractional cover of Old (over rotation age) Coniferous Forests
within the analysis cell (1 ha)

\textbf{Latvian name:} Vecu (kopš cirtmeta) skujkoku mežu platības īpatsvars analīzes
šūnā (1 ha)

\textbf{Procedure:} Most EGVs describing forests are spatially restricted to areas outside
of clearcuts and dead stands. This mask is created using a combination of
the \hyperref[Ch04.01]{State Forest Service's
State Forest Registry} land category 12 and 14, and \hyperref[Ch04.09]{The
Global Forest Watch} pixels classified as lost tree canopy cover since
2020 (raster layer matching input, presence = 1, absence = 0).

To prepare this EGV, stands from the \hyperref[Ch04.01]{State Forest Service's State Forest
Registry} are classified into (in order):

\begin{itemize}
\item
  coniferous (see \hyperref[Ch01]{Terminology and acronyms} for species codes) if
  timber volume of those species exceeded 75\%;
\item
  Boreal deciduous if timber volume of those species exceeded 75\%;
\item
  temperate deciduous if timber volume of those species exceeded 50\%;
\item
  mixed otherwise;
\end{itemize}

then coniferous stands exceeding the legal rotation age are selected and
geometries are rasterised (presence = 1, NA otherwise). Rasterisation is
performed using the workflow \texttt{egvtools::polygon2input()}, restricting to pixels outside clearcut
mask and covering background with value 0. The resulting layer
is then aggregated to EGV resolution using the workflow \texttt{egvtools::input2egv()}, which
calculates the arithmetic mean to determine the cover fraction. During
aggregation, inverse distance weighted (power = 2) gap filling on the output is
applied to ensure no missing values at the edges. Finally, the layer is
standardised by subtracting the arithmetic mean and dividing by the root mean squared
error.

\begin{Shaded}
\begin{Highlighting}[]
\CommentTok{\# libs {-}{-}{-}{-}}
\ControlFlowTok{if}\NormalTok{(}\SpecialCharTok{!}\FunctionTok{require}\NormalTok{(egvtools)) \{remotes}\SpecialCharTok{::}\FunctionTok{install\_github}\NormalTok{(}\StringTok{"aavotins/egvtools"}\NormalTok{); }\FunctionTok{require}\NormalTok{(egvtools)\}}
\ControlFlowTok{if}\NormalTok{(}\SpecialCharTok{!}\FunctionTok{require}\NormalTok{(terra)) \{}\FunctionTok{install.packages}\NormalTok{(}\StringTok{"terra"}\NormalTok{); }\FunctionTok{require}\NormalTok{(terra)\}}
\ControlFlowTok{if}\NormalTok{(}\SpecialCharTok{!}\FunctionTok{require}\NormalTok{(sf)) \{}\FunctionTok{install.packages}\NormalTok{(}\StringTok{"sf"}\NormalTok{); }\FunctionTok{require}\NormalTok{(sf)\}}
\ControlFlowTok{if}\NormalTok{(}\SpecialCharTok{!}\FunctionTok{require}\NormalTok{(tidyverse)) \{}\FunctionTok{install.packages}\NormalTok{(}\StringTok{"tidyverse"}\NormalTok{); }\FunctionTok{require}\NormalTok{(tidyverse)\}}
\ControlFlowTok{if}\NormalTok{(}\SpecialCharTok{!}\FunctionTok{require}\NormalTok{(sfarrow)) \{}\FunctionTok{install.packages}\NormalTok{(}\StringTok{"sfarrow"}\NormalTok{); }\FunctionTok{require}\NormalTok{(sfarrow)\}}
\ControlFlowTok{if}\NormalTok{(}\SpecialCharTok{!}\FunctionTok{require}\NormalTok{(readxl)) \{}\FunctionTok{install.packages}\NormalTok{(}\StringTok{"readxl"}\NormalTok{); }\FunctionTok{require}\NormalTok{(readxl)\}}
\ControlFlowTok{if}\NormalTok{(}\SpecialCharTok{!}\FunctionTok{require}\NormalTok{(raster)) \{}\FunctionTok{install.packages}\NormalTok{(}\StringTok{"raster"}\NormalTok{); }\FunctionTok{require}\NormalTok{(raster)\}}
\ControlFlowTok{if}\NormalTok{(}\SpecialCharTok{!}\FunctionTok{require}\NormalTok{(fasterize)) \{}\FunctionTok{install.packages}\NormalTok{(}\StringTok{"fasterize"}\NormalTok{); }\FunctionTok{require}\NormalTok{(fasterize)\}}

\CommentTok{\# templates {-}{-}{-}{-}}
\NormalTok{template100}\OtherTok{=}\FunctionTok{rast}\NormalTok{(}\StringTok{"./Templates/TemplateRasters/LV100m\_10km.tif"}\NormalTok{)}
\NormalTok{template10}\OtherTok{=}\FunctionTok{rast}\NormalTok{(}\StringTok{"./Templates/TemplateRasters/LV10m\_10km.tif"}\NormalTok{)}
\NormalTok{rastrs10}\OtherTok{=}\FunctionTok{raster}\NormalTok{(template10)}

\NormalTok{nulls10}\OtherTok{=}\FunctionTok{rast}\NormalTok{(}\StringTok{"./Templates/TemplateRasters/nulls\_LV10m\_10km.tif"}\NormalTok{)}
\NormalTok{nulls100}\OtherTok{=}\FunctionTok{rast}\NormalTok{(}\StringTok{"./Templates/TemplateRasters/nulls\_LV100m\_10km.tif"}\NormalTok{)}


\CommentTok{\# simple landscape {-}{-}{-}{-}}
\NormalTok{simple\_landscape}\OtherTok{=}\FunctionTok{rast}\NormalTok{(}\StringTok{"RasterGrids\_10m/2024/Ainava\_vienk\_mask.tif"}\NormalTok{)}

\CommentTok{\# mvr {-}{-}{-}{-}}
\NormalTok{mvr}\OtherTok{=}\FunctionTok{st\_read\_parquet}\NormalTok{(}\StringTok{"./Geodata/2024/MVR/nogabali\_2024janv.parquet"}\NormalTok{)}
\NormalTok{mvr}\SpecialCharTok{$}\NormalTok{yes}\OtherTok{=}\DecValTok{1}

\CommentTok{\# clear cut mask {-}{-}{-}{-}}
\NormalTok{izcirtumi}\OtherTok{=}\NormalTok{mvr }\SpecialCharTok{\%\textgreater{}\%} 
 \FunctionTok{filter}\NormalTok{(zkat }\SpecialCharTok{\%in\%} \FunctionTok{c}\NormalTok{(}\StringTok{"12"}\NormalTok{,}\StringTok{"14"}\NormalTok{)) }\SpecialCharTok{\%\textgreater{}\%} 
\NormalTok{ dplyr}\SpecialCharTok{::}\FunctionTok{select}\NormalTok{(yes)}
\NormalTok{r\_izcirtumi\_mvr}\OtherTok{=}\FunctionTok{fasterize}\NormalTok{(izcirtumi,rastrs10,}\AttributeTok{field=}\StringTok{"yes"}\NormalTok{)}
\NormalTok{t\_izcirtumi\_mvr}\OtherTok{=}\FunctionTok{rast}\NormalTok{(r\_izcirtumi\_mvr)}
\FunctionTok{plot}\NormalTok{(t\_izcirtumi\_mvr)}

\NormalTok{tcl}\OtherTok{=}\FunctionTok{rast}\NormalTok{(}\StringTok{"./Geodata/2024/Trees/GFW/TreeCoverLoss\_v1\_12.tif"}\NormalTok{)}
\NormalTok{tcl2}\OtherTok{=}\FunctionTok{ifel}\NormalTok{(tcl}\SpecialCharTok{\textless{}}\DecValTok{20}\NormalTok{,}\DecValTok{0}\NormalTok{,}\DecValTok{1}\NormalTok{)}
\NormalTok{tclX}\OtherTok{=}\FunctionTok{cover}\NormalTok{(tcl2,nulls10)}
\FunctionTok{plot}\NormalTok{(tclX)}

\NormalTok{clearcut\_mask}\OtherTok{=}\FunctionTok{cover}\NormalTok{(t\_izcirtumi\_mvr,tclX,}
          \AttributeTok{filename=}\StringTok{"./RasterGrids\_10m/2024/Mask\_clearcuts.tif"}\NormalTok{,}
          \AttributeTok{overwrite=}\ConstantTok{TRUE}\NormalTok{)}
\FunctionTok{plot}\NormalTok{(clearcut\_mask)}

\FunctionTok{rm}\NormalTok{(izcirtumi)}
\FunctionTok{rm}\NormalTok{(r\_izcirtumi\_mvr)}
\FunctionTok{rm}\NormalTok{(t\_izcirtumi\_mvr)}
\FunctionTok{rm}\NormalTok{(tcl)}
\FunctionTok{rm}\NormalTok{(tcl2)}
\FunctionTok{rm}\NormalTok{(tclX)}

\CommentTok{\# ForestsTreesAge\_ConiferousOld\_cell.tif    egv\_353 {-}{-}{-}{-}}
\NormalTok{skujkoki}\OtherTok{=}\FunctionTok{c}\NormalTok{(}\StringTok{"1"}\NormalTok{,}\StringTok{"3"}\NormalTok{,}\StringTok{"13"}\NormalTok{,}\StringTok{"14"}\NormalTok{,}\StringTok{"15"}\NormalTok{,}\StringTok{"22"}\NormalTok{,}\StringTok{"23"}\NormalTok{,}\StringTok{"28"}\NormalTok{) }\CommentTok{\# 8}
\NormalTok{saurlapji}\OtherTok{=}\FunctionTok{c}\NormalTok{(}\StringTok{"4"}\NormalTok{,}\StringTok{"6"}\NormalTok{,}\StringTok{"8"}\NormalTok{,}\StringTok{"9"}\NormalTok{,}\StringTok{"19"}\NormalTok{,}\StringTok{"20"}\NormalTok{,}\StringTok{"21"}\NormalTok{,}\StringTok{"32"}\NormalTok{,}\StringTok{"35"}\NormalTok{,}\StringTok{"68"}\NormalTok{) }\CommentTok{\# 10}
\NormalTok{platlapji}\OtherTok{=}\FunctionTok{c}\NormalTok{(}\StringTok{"10"}\NormalTok{,}\StringTok{"11"}\NormalTok{,}\StringTok{"12"}\NormalTok{,}\StringTok{"16"}\NormalTok{,}\StringTok{"17"}\NormalTok{,}\StringTok{"18"}\NormalTok{,}\StringTok{"24"}\NormalTok{,}\StringTok{"25"}\NormalTok{,}\StringTok{"26"}\NormalTok{,}\StringTok{"27"}\NormalTok{,}\StringTok{"28"}\NormalTok{,}\StringTok{"29"}\NormalTok{,}\StringTok{"50"}\NormalTok{,}
      \StringTok{"61"}\NormalTok{,}\StringTok{"62"}\NormalTok{,}\StringTok{"63"}\NormalTok{,}\StringTok{"64"}\NormalTok{,}\StringTok{"65"}\NormalTok{,}\StringTok{"66"}\NormalTok{,}\StringTok{"67"}\NormalTok{,}\StringTok{"69"}\NormalTok{) }\CommentTok{\# 21}
\NormalTok{mvr}\OtherTok{=}\NormalTok{mvr }\SpecialCharTok{\%\textgreater{}\%} 
 \FunctionTok{mutate}\NormalTok{(}\AttributeTok{kraja\_skujkoku=}\FunctionTok{ifelse}\NormalTok{(s10 }\SpecialCharTok{\%in\%}\NormalTok{ skujkoki,v10,}\DecValTok{0}\NormalTok{)}\SpecialCharTok{+}
      \FunctionTok{ifelse}\NormalTok{(s11 }\SpecialCharTok{\%in\%}\NormalTok{ skujkoki,v11,}\DecValTok{0}\NormalTok{)}\SpecialCharTok{+}\FunctionTok{ifelse}\NormalTok{(s12 }\SpecialCharTok{\%in\%}\NormalTok{ skujkoki,v12,}\DecValTok{0}\NormalTok{)}\SpecialCharTok{+}
      \FunctionTok{ifelse}\NormalTok{(s13 }\SpecialCharTok{\%in\%}\NormalTok{ skujkoki,v13,}\DecValTok{0}\NormalTok{)}\SpecialCharTok{+}\FunctionTok{ifelse}\NormalTok{(s14 }\SpecialCharTok{\%in\%}\NormalTok{ skujkoki,v14,}\DecValTok{0}\NormalTok{),}
     \AttributeTok{kraja\_saurlapju=}\FunctionTok{ifelse}\NormalTok{(s10 }\SpecialCharTok{\%in\%}\NormalTok{ saurlapji,v10,}\DecValTok{0}\NormalTok{)}\SpecialCharTok{+}
      \FunctionTok{ifelse}\NormalTok{(s11 }\SpecialCharTok{\%in\%}\NormalTok{ saurlapji,v11,}\DecValTok{0}\NormalTok{)}\SpecialCharTok{+}\FunctionTok{ifelse}\NormalTok{(s12 }\SpecialCharTok{\%in\%}\NormalTok{ saurlapji,v12,}\DecValTok{0}\NormalTok{)}\SpecialCharTok{+}
      \FunctionTok{ifelse}\NormalTok{(s13 }\SpecialCharTok{\%in\%}\NormalTok{ saurlapji,v13,}\DecValTok{0}\NormalTok{)}\SpecialCharTok{+}\FunctionTok{ifelse}\NormalTok{(s14 }\SpecialCharTok{\%in\%}\NormalTok{ saurlapji,v14,}\DecValTok{0}\NormalTok{),}
     \AttributeTok{kraja\_platlapju=}\FunctionTok{ifelse}\NormalTok{(s10 }\SpecialCharTok{\%in\%}\NormalTok{ platlapji,v10,}\DecValTok{0}\NormalTok{)}\SpecialCharTok{+}
      \FunctionTok{ifelse}\NormalTok{(s11 }\SpecialCharTok{\%in\%}\NormalTok{ platlapji,v11,}\DecValTok{0}\NormalTok{)}\SpecialCharTok{+}\FunctionTok{ifelse}\NormalTok{(s12 }\SpecialCharTok{\%in\%}\NormalTok{ platlapji,v12,}\DecValTok{0}\NormalTok{)}\SpecialCharTok{+}
      \FunctionTok{ifelse}\NormalTok{(s13 }\SpecialCharTok{\%in\%}\NormalTok{ platlapji,v13,}\DecValTok{0}\NormalTok{)}\SpecialCharTok{+}\FunctionTok{ifelse}\NormalTok{(s14 }\SpecialCharTok{\%in\%}\NormalTok{ platlapji,v14,}\DecValTok{0}\NormalTok{)) }\SpecialCharTok{\%\textgreater{}\%} 
 \FunctionTok{mutate}\NormalTok{(}\AttributeTok{kopeja\_kraja=}\NormalTok{kraja\_skujkoku}\SpecialCharTok{+}\NormalTok{kraja\_platlapju}\SpecialCharTok{+}\NormalTok{kraja\_saurlapju) }\SpecialCharTok{\%\textgreater{}\%} 
 \FunctionTok{mutate}\NormalTok{(}\AttributeTok{tips=}\FunctionTok{ifelse}\NormalTok{(kraja\_skujkoku}\SpecialCharTok{/}\NormalTok{kopeja\_kraja}\SpecialCharTok{\textgreater{}=}\FloatTok{0.75}\NormalTok{,}\StringTok{"skujkoku"}\NormalTok{,}
           \FunctionTok{ifelse}\NormalTok{(kraja\_saurlapju}\SpecialCharTok{/}\NormalTok{kopeja\_kraja}\SpecialCharTok{\textgreater{}=}\FloatTok{0.75}\NormalTok{,}\StringTok{"saurlapju"}\NormalTok{,}
              \FunctionTok{ifelse}\NormalTok{(kraja\_platlapju}\SpecialCharTok{/}\NormalTok{kopeja\_kraja}\SpecialCharTok{\textgreater{}}\FloatTok{0.5}\NormalTok{,}\StringTok{"platlapju"}\NormalTok{,}
                  \StringTok{"jauktu koku"}\NormalTok{))))}
\NormalTok{nogabali}\OtherTok{=}\NormalTok{mvr }\SpecialCharTok{\%\textgreater{}\%} 
 \FunctionTok{filter}\NormalTok{(zkat}\SpecialCharTok{==}\StringTok{"10"}\SpecialCharTok{\&}\NormalTok{tips}\SpecialCharTok{==}\StringTok{"skujkoku"}\SpecialCharTok{\&}\NormalTok{(vgr}\SpecialCharTok{==}\StringTok{"4"}\SpecialCharTok{|}\NormalTok{vgr}\SpecialCharTok{==}\StringTok{"5"}\NormalTok{))}

\NormalTok{p2i\_rez}\OtherTok{=}\NormalTok{egvtools}\SpecialCharTok{::}\FunctionTok{polygon2input}\NormalTok{(}\AttributeTok{vector\_data =}\NormalTok{ nogabali,}
                \AttributeTok{template\_path =} \StringTok{"./Templates/TemplateRasters/LV10m\_10km.tif"}\NormalTok{,}
                \AttributeTok{out\_path =} \StringTok{"./RasterGrids\_10m/2024/"}\NormalTok{,}
                \AttributeTok{file\_name =} \StringTok{"ForestsTreesAge\_ConiferousOld\_input.tif"}\NormalTok{,}
                \AttributeTok{value\_field =} \StringTok{"yes"}\NormalTok{,}
                \AttributeTok{restrict\_to =}\NormalTok{ clearcut\_mask,}
                \AttributeTok{restrict\_values =} \DecValTok{0}\NormalTok{,}
                \AttributeTok{prepare=}\ConstantTok{FALSE}\NormalTok{,}
                \AttributeTok{background\_raster =} \StringTok{"./Templates/TemplateRasters/nulls\_LV10m\_10km.tif"}\NormalTok{,}
                \AttributeTok{plot\_result =} \ConstantTok{TRUE}\NormalTok{)}
\NormalTok{p2i\_rez}
\NormalTok{i2e\_rez}\OtherTok{=}\NormalTok{egvtools}\SpecialCharTok{::}\FunctionTok{input2egv}\NormalTok{(}\AttributeTok{input=}\FunctionTok{paste0}\NormalTok{(}\StringTok{"./RasterGrids\_10m/2024/"}\NormalTok{,}
                     \StringTok{"ForestsTreesAge\_ConiferousOld\_input.tif"}\NormalTok{),}
              \AttributeTok{egv\_template=} \StringTok{"./Templates/TemplateRasters/LV100m\_10km.tif"}\NormalTok{,}
              \AttributeTok{summary\_function =} \StringTok{"average"}\NormalTok{,}
              \AttributeTok{missing\_job =} \StringTok{"FillOutput"}\NormalTok{,}
              \AttributeTok{outlocation =} \StringTok{"./RasterGrids\_100m/2024/RAW/"}\NormalTok{,}
              \AttributeTok{outfilename =} \StringTok{"ForestsTreesAge\_ConiferousOld\_cell.tif"}\NormalTok{,}
              \AttributeTok{layername =} \StringTok{"egv\_353"}\NormalTok{,}
              \AttributeTok{idw\_weight =} \DecValTok{2}\NormalTok{,}
              \AttributeTok{plot\_gaps =} \ConstantTok{FALSE}\NormalTok{,}\AttributeTok{plot\_final =} \ConstantTok{TRUE}\NormalTok{)}
\NormalTok{i2e\_rez}
\FunctionTok{rm}\NormalTok{(nogabali)}
\FunctionTok{rm}\NormalTok{(p2i\_rez)}
\FunctionTok{rm}\NormalTok{(i2e\_rez)}
\FunctionTok{unlink}\NormalTok{(}\StringTok{"./RasterGrids\_10m/2024/ForestsTreesAge\_ConiferousOld\_input.tif"}\NormalTok{)}

\CommentTok{\# standardisation {-}{-}{-}{-}}
\ControlFlowTok{if}\NormalTok{(}\SpecialCharTok{!}\FunctionTok{require}\NormalTok{(terra)) \{}\FunctionTok{install.packages}\NormalTok{(}\StringTok{"terra"}\NormalTok{); }\FunctionTok{require}\NormalTok{(terra)\}}
\ControlFlowTok{if}\NormalTok{(}\SpecialCharTok{!}\FunctionTok{require}\NormalTok{(tidyverse)) \{}\FunctionTok{install.packages}\NormalTok{(}\StringTok{"tidyverse"}\NormalTok{); }\FunctionTok{require}\NormalTok{(tidyverse)\}}

\NormalTok{nosaukums}\OtherTok{=}\StringTok{"ForestsTreesAge\_ConiferousOld\_cell.tif"}
\NormalTok{ielasisanas\_cels}\OtherTok{=}\FunctionTok{paste0}\NormalTok{(}\StringTok{"./RasterGrids\_100m/2024/RAW/"}\NormalTok{,nosaukums)}
\NormalTok{saglabasanas\_cels}\OtherTok{=}\FunctionTok{paste0}\NormalTok{(}\StringTok{"./RasterGrids\_100m/2024/Scaled/"}\NormalTok{,nosaukums)}
\NormalTok{slanis}\OtherTok{=}\FunctionTok{rast}\NormalTok{(ielasisanas\_cels)}
\NormalTok{videjais}\OtherTok{=}\FunctionTok{global}\NormalTok{(slanis,}\AttributeTok{fun=}\StringTok{"mean"}\NormalTok{,}\AttributeTok{na.rm=}\ConstantTok{TRUE}\NormalTok{)}
\NormalTok{centrets}\OtherTok{=}\NormalTok{slanis}\SpecialCharTok{{-}}\NormalTok{videjais[,}\DecValTok{1}\NormalTok{]}
\NormalTok{standartnovirze}\OtherTok{=}\NormalTok{terra}\SpecialCharTok{::}\FunctionTok{global}\NormalTok{(centrets,}\AttributeTok{fun=}\StringTok{"rms"}\NormalTok{,}\AttributeTok{na.rm=}\ConstantTok{TRUE}\NormalTok{)}
\NormalTok{merogots}\OtherTok{=}\NormalTok{centrets}\SpecialCharTok{/}\NormalTok{standartnovirze[,}\DecValTok{1}\NormalTok{]}
\FunctionTok{writeRaster}\NormalTok{(merogots,}
      \AttributeTok{filename=}\NormalTok{saglabasanas\_cels,}
      \AttributeTok{overwrite=}\ConstantTok{TRUE}\NormalTok{)}
\end{Highlighting}
\end{Shaded}

\section{ForestsTreesAge\_ConiferousOld\_r500}\label{ch06.354}

\textbf{filename:} \texttt{ForestsTreesAge\_ConiferousOld\_r500.tif}

\textbf{layername:} \texttt{egv\_354}

\textbf{English name:} Fractional cover of Old (over rotation age) Coniferous Forests
within the 0.5 km landscape

\textbf{Latvian name:} Vecu (kopš cirtmeta) skujkoku mežu platības īpatsvars 0,5 km
ainavā

\textbf{Procedure:} The cover fraction within a radius of 500 m around the analysis grid cell is
calculated as the area-weighted sum of the \hyperref[ch06.353]{analysis cells} inside the
buffer, using the workflow \texttt{egvtools::radius\_function()}. During the calculation of the landscape metric,
inverse distance weighted (power = 2) gap filling on the output is applied
to ensure no missing values at the edges. Then the layer is rewritten to set
its name. Finally, the layer is standardised by subtracting the arithmetic
mean and dividing by the root mean squared error.

\begin{Shaded}
\begin{Highlighting}[]
\CommentTok{\# libs {-}{-}{-}{-}}
\ControlFlowTok{if}\NormalTok{(}\SpecialCharTok{!}\FunctionTok{require}\NormalTok{(terra)) \{}\FunctionTok{install.packages}\NormalTok{(}\StringTok{"terra"}\NormalTok{); }\FunctionTok{require}\NormalTok{(terra)\}}
\ControlFlowTok{if}\NormalTok{(}\SpecialCharTok{!}\FunctionTok{require}\NormalTok{(egvtools)) \{remotes}\SpecialCharTok{::}\FunctionTok{install\_github}\NormalTok{(}\StringTok{"aavotins/egvtools"}\NormalTok{); }\FunctionTok{require}\NormalTok{(egvtools)\}}


\CommentTok{\# Templates {-}{-}{-}{-}{-}}
\NormalTok{template100}\OtherTok{=}\FunctionTok{rast}\NormalTok{(}\StringTok{"./Templates/TemplateRasters/LV100m\_10km.tif"}\NormalTok{)}

\CommentTok{\# radii {-}{-}{-}{-}}
\FunctionTok{radius\_function}\NormalTok{(}
 \AttributeTok{kvadrati\_path =} \StringTok{"./Templates/TemplateGrids/tiles/"}\NormalTok{,}
 \AttributeTok{radii\_path   =} \StringTok{"./Templates/TemplateGridPoints/tiles/"}\NormalTok{,}
 \AttributeTok{tikls100\_path =} \StringTok{"./Templates/TemplateGrids/tikls100\_sauzeme.parquet"}\NormalTok{,}
 \AttributeTok{template\_path =} \StringTok{"./Templates/TemplateRasters/LV100m\_10km.tif"}\NormalTok{,}
 \AttributeTok{input\_layers  =} \FunctionTok{c}\NormalTok{(}\StringTok{"./RasterGrids\_100m/2024/RAW/ForestsTreesAge\_ConiferousOld\_cell.tif"}\NormalTok{),}
 \AttributeTok{layer\_prefixes =} \FunctionTok{c}\NormalTok{(}\StringTok{"ForestsTreesAge\_ConiferousOld"}\NormalTok{),}
 \AttributeTok{output\_dir   =} \StringTok{"./RasterGrids\_100m/2024/RAW/"}\NormalTok{,}
 \AttributeTok{n\_workers   =} \DecValTok{6}\NormalTok{,}
 \AttributeTok{radii     =} \FunctionTok{c}\NormalTok{(}\StringTok{"r500"}\NormalTok{),}
 \AttributeTok{radius\_mode  =} \StringTok{"sparse"}\NormalTok{,}
 \AttributeTok{extract\_fun  =} \StringTok{"mean"}\NormalTok{,}
 \AttributeTok{fill\_missing  =} \ConstantTok{TRUE}\NormalTok{,}
 \AttributeTok{IDW\_weight   =} \DecValTok{2}\NormalTok{,}
 \AttributeTok{future\_max\_size =} \DecValTok{40} \SpecialCharTok{*} \DecValTok{1024}\SpecialCharTok{\^{}}\DecValTok{3}\NormalTok{)}


\CommentTok{\# ForestsTreesAge\_ConiferousOld\_r500.tif    egv\_354}
\NormalTok{slanis}\OtherTok{=}\FunctionTok{rast}\NormalTok{(}\StringTok{"./RasterGrids\_100m/2024/RAW/ForestsTreesAge\_ConiferousOld\_r500.tif"}\NormalTok{)}
\FunctionTok{names}\NormalTok{(slanis)}\OtherTok{=}\StringTok{"egv\_354"}
\NormalTok{slanis2}\OtherTok{=}\FunctionTok{project}\NormalTok{(slanis,template100)}
\FunctionTok{writeRaster}\NormalTok{(slanis2,}
      \StringTok{"./RasterGrids\_100m/2024/RAW/ForestsTreesAge\_ConiferousOld\_r500.tif"}\NormalTok{,}
      \AttributeTok{overwrite=}\ConstantTok{TRUE}\NormalTok{)}

\CommentTok{\# standardisation {-}{-}{-}{-}}
\ControlFlowTok{if}\NormalTok{(}\SpecialCharTok{!}\FunctionTok{require}\NormalTok{(terra)) \{}\FunctionTok{install.packages}\NormalTok{(}\StringTok{"terra"}\NormalTok{); }\FunctionTok{require}\NormalTok{(terra)\}}
\ControlFlowTok{if}\NormalTok{(}\SpecialCharTok{!}\FunctionTok{require}\NormalTok{(tidyverse)) \{}\FunctionTok{install.packages}\NormalTok{(}\StringTok{"tidyverse"}\NormalTok{); }\FunctionTok{require}\NormalTok{(tidyverse)\}}

\NormalTok{nosaukums}\OtherTok{=}\StringTok{"ForestsTreesAge\_ConiferousOld\_r500.tif"}
\NormalTok{ielasisanas\_cels}\OtherTok{=}\FunctionTok{paste0}\NormalTok{(}\StringTok{"./RasterGrids\_100m/2024/RAW/"}\NormalTok{,nosaukums)}
\NormalTok{saglabasanas\_cels}\OtherTok{=}\FunctionTok{paste0}\NormalTok{(}\StringTok{"./RasterGrids\_100m/2024/Scaled/"}\NormalTok{,nosaukums)}
\NormalTok{slanis}\OtherTok{=}\FunctionTok{rast}\NormalTok{(ielasisanas\_cels)}
\NormalTok{videjais}\OtherTok{=}\FunctionTok{global}\NormalTok{(slanis,}\AttributeTok{fun=}\StringTok{"mean"}\NormalTok{,}\AttributeTok{na.rm=}\ConstantTok{TRUE}\NormalTok{)}
\NormalTok{centrets}\OtherTok{=}\NormalTok{slanis}\SpecialCharTok{{-}}\NormalTok{videjais[,}\DecValTok{1}\NormalTok{]}
\NormalTok{standartnovirze}\OtherTok{=}\NormalTok{terra}\SpecialCharTok{::}\FunctionTok{global}\NormalTok{(centrets,}\AttributeTok{fun=}\StringTok{"rms"}\NormalTok{,}\AttributeTok{na.rm=}\ConstantTok{TRUE}\NormalTok{)}
\NormalTok{merogots}\OtherTok{=}\NormalTok{centrets}\SpecialCharTok{/}\NormalTok{standartnovirze[,}\DecValTok{1}\NormalTok{]}
\FunctionTok{writeRaster}\NormalTok{(merogots,}
      \AttributeTok{filename=}\NormalTok{saglabasanas\_cels,}
      \AttributeTok{overwrite=}\ConstantTok{TRUE}\NormalTok{)}
\end{Highlighting}
\end{Shaded}

\section{ForestsTreesAge\_ConiferousOld\_r1250}\label{ch06.355}

\textbf{filename:} \texttt{ForestsTreesAge\_ConiferousOld\_r1250.tif}

\textbf{layername:} \texttt{egv\_355}

\textbf{English name:} Fractional cover of Old (over rotation age) Coniferous Forests
within the 1.25 km landscape

\textbf{Latvian name:} Vecu (kopš cirtmeta) skujkoku mežu platības īpatsvars 1,25 km
ainavā

\textbf{Procedure:} The cover fraction within a radius of 1250 m around the analysis grid cell
is calculated as the area-weighted sum of the \hyperref[ch06.353]{analysis cells} inside
the buffer, using the workflow \texttt{egvtools::radius\_function()}. During the calculation of the landscape
metric, inverse distance weighted (power = 2) gap filling on the output is
applied to ensure no missing values at the edges. Then the layer is
rewritten to set its name. Finally, the layer is standardised by
subtracting the arithmetic mean and dividing by the root mean squared error.

\begin{Shaded}
\begin{Highlighting}[]
\CommentTok{\# libs {-}{-}{-}{-}}
\ControlFlowTok{if}\NormalTok{(}\SpecialCharTok{!}\FunctionTok{require}\NormalTok{(terra)) \{}\FunctionTok{install.packages}\NormalTok{(}\StringTok{"terra"}\NormalTok{); }\FunctionTok{require}\NormalTok{(terra)\}}
\ControlFlowTok{if}\NormalTok{(}\SpecialCharTok{!}\FunctionTok{require}\NormalTok{(egvtools)) \{remotes}\SpecialCharTok{::}\FunctionTok{install\_github}\NormalTok{(}\StringTok{"aavotins/egvtools"}\NormalTok{); }\FunctionTok{require}\NormalTok{(egvtools)\}}


\CommentTok{\# Templates {-}{-}{-}{-}{-}}
\NormalTok{template100}\OtherTok{=}\FunctionTok{rast}\NormalTok{(}\StringTok{"./Templates/TemplateRasters/LV100m\_10km.tif"}\NormalTok{)}

\CommentTok{\# radii {-}{-}{-}{-}}
\FunctionTok{radius\_function}\NormalTok{(}
 \AttributeTok{kvadrati\_path =} \StringTok{"./Templates/TemplateGrids/tiles/"}\NormalTok{,}
 \AttributeTok{radii\_path   =} \StringTok{"./Templates/TemplateGridPoints/tiles/"}\NormalTok{,}
 \AttributeTok{tikls100\_path =} \StringTok{"./Templates/TemplateGrids/tikls100\_sauzeme.parquet"}\NormalTok{,}
 \AttributeTok{template\_path =} \StringTok{"./Templates/TemplateRasters/LV100m\_10km.tif"}\NormalTok{,}
 \AttributeTok{input\_layers  =} \FunctionTok{c}\NormalTok{(}\StringTok{"./RasterGrids\_100m/2024/RAW/ForestsTreesAge\_ConiferousOld\_cell.tif"}\NormalTok{),}
 \AttributeTok{layer\_prefixes =} \FunctionTok{c}\NormalTok{(}\StringTok{"ForestsTreesAge\_ConiferousOld"}\NormalTok{),}
 \AttributeTok{output\_dir   =} \StringTok{"./RasterGrids\_100m/2024/RAW/"}\NormalTok{,}
 \AttributeTok{n\_workers   =} \DecValTok{6}\NormalTok{,}
 \AttributeTok{radii     =} \FunctionTok{c}\NormalTok{(}\StringTok{"r1250"}\NormalTok{),}
 \AttributeTok{radius\_mode  =} \StringTok{"sparse"}\NormalTok{,}
 \AttributeTok{extract\_fun  =} \StringTok{"mean"}\NormalTok{,}
 \AttributeTok{fill\_missing  =} \ConstantTok{TRUE}\NormalTok{,}
 \AttributeTok{IDW\_weight   =} \DecValTok{2}\NormalTok{,}
 \AttributeTok{future\_max\_size =} \DecValTok{40} \SpecialCharTok{*} \DecValTok{1024}\SpecialCharTok{\^{}}\DecValTok{3}\NormalTok{)}


\CommentTok{\# ForestsTreesAge\_ConiferousOld\_r1250.tif   egv\_355}
\NormalTok{slanis}\OtherTok{=}\FunctionTok{rast}\NormalTok{(}\StringTok{"./RasterGrids\_100m/2024/RAW/ForestsTreesAge\_ConiferousOld\_r1250.tif"}\NormalTok{)}
\FunctionTok{names}\NormalTok{(slanis)}\OtherTok{=}\StringTok{"egv\_355"}
\NormalTok{slanis2}\OtherTok{=}\FunctionTok{project}\NormalTok{(slanis,template100)}
\FunctionTok{writeRaster}\NormalTok{(slanis2,}
      \StringTok{"./RasterGrids\_100m/2024/RAW/ForestsTreesAge\_ConiferousOld\_r1250.tif"}\NormalTok{,}
      \AttributeTok{overwrite=}\ConstantTok{TRUE}\NormalTok{)}

\CommentTok{\# standardisation {-}{-}{-}{-}}
\ControlFlowTok{if}\NormalTok{(}\SpecialCharTok{!}\FunctionTok{require}\NormalTok{(terra)) \{}\FunctionTok{install.packages}\NormalTok{(}\StringTok{"terra"}\NormalTok{); }\FunctionTok{require}\NormalTok{(terra)\}}
\ControlFlowTok{if}\NormalTok{(}\SpecialCharTok{!}\FunctionTok{require}\NormalTok{(tidyverse)) \{}\FunctionTok{install.packages}\NormalTok{(}\StringTok{"tidyverse"}\NormalTok{); }\FunctionTok{require}\NormalTok{(tidyverse)\}}

\NormalTok{nosaukums}\OtherTok{=}\StringTok{"ForestsTreesAge\_ConiferousOld\_r1250.tif"}
\NormalTok{ielasisanas\_cels}\OtherTok{=}\FunctionTok{paste0}\NormalTok{(}\StringTok{"./RasterGrids\_100m/2024/RAW/"}\NormalTok{,nosaukums)}
\NormalTok{saglabasanas\_cels}\OtherTok{=}\FunctionTok{paste0}\NormalTok{(}\StringTok{"./RasterGrids\_100m/2024/Scaled/"}\NormalTok{,nosaukums)}
\NormalTok{slanis}\OtherTok{=}\FunctionTok{rast}\NormalTok{(ielasisanas\_cels)}
\NormalTok{videjais}\OtherTok{=}\FunctionTok{global}\NormalTok{(slanis,}\AttributeTok{fun=}\StringTok{"mean"}\NormalTok{,}\AttributeTok{na.rm=}\ConstantTok{TRUE}\NormalTok{)}
\NormalTok{centrets}\OtherTok{=}\NormalTok{slanis}\SpecialCharTok{{-}}\NormalTok{videjais[,}\DecValTok{1}\NormalTok{]}
\NormalTok{standartnovirze}\OtherTok{=}\NormalTok{terra}\SpecialCharTok{::}\FunctionTok{global}\NormalTok{(centrets,}\AttributeTok{fun=}\StringTok{"rms"}\NormalTok{,}\AttributeTok{na.rm=}\ConstantTok{TRUE}\NormalTok{)}
\NormalTok{merogots}\OtherTok{=}\NormalTok{centrets}\SpecialCharTok{/}\NormalTok{standartnovirze[,}\DecValTok{1}\NormalTok{]}
\FunctionTok{writeRaster}\NormalTok{(merogots,}
      \AttributeTok{filename=}\NormalTok{saglabasanas\_cels,}
      \AttributeTok{overwrite=}\ConstantTok{TRUE}\NormalTok{)}
\end{Highlighting}
\end{Shaded}

\section{ForestsTreesAge\_ConiferousOld\_r3000}\label{ch06.356}

\textbf{filename:} \texttt{ForestsTreesAge\_ConiferousOld\_r3000.tif}

\textbf{layername:} \texttt{egv\_356}

\textbf{English name:} Fractional cover of Old (over rotation age) Coniferous Forests
within the 3 km landscape

\textbf{Latvian name:} Vecu (kopš cirtmeta) skujkoku mežu platības īpatsvars 3 km
ainavā

\textbf{Procedure:} The cover fraction within a radius of 3000 m around the analysis grid cell
is calculated as the area-weighted sum of the \hyperref[ch06.353]{analysis cells} inside
the buffer, using the workflow \texttt{egvtools::radius\_function()}. During the calculation of the landscape
metric, inverse distance weighted (power = 2) gap filling on the output is
applied to ensure no missing values at the edges. Then the layer is
rewritten to set its name. Finally, the layer is standardised by
subtracting the arithmetic mean and dividing by the root mean squared error.

\begin{Shaded}
\begin{Highlighting}[]
\CommentTok{\# libs {-}{-}{-}{-}}
\ControlFlowTok{if}\NormalTok{(}\SpecialCharTok{!}\FunctionTok{require}\NormalTok{(terra)) \{}\FunctionTok{install.packages}\NormalTok{(}\StringTok{"terra"}\NormalTok{); }\FunctionTok{require}\NormalTok{(terra)\}}
\ControlFlowTok{if}\NormalTok{(}\SpecialCharTok{!}\FunctionTok{require}\NormalTok{(egvtools)) \{remotes}\SpecialCharTok{::}\FunctionTok{install\_github}\NormalTok{(}\StringTok{"aavotins/egvtools"}\NormalTok{); }\FunctionTok{require}\NormalTok{(egvtools)\}}


\CommentTok{\# Templates {-}{-}{-}{-}{-}}
\NormalTok{template100}\OtherTok{=}\FunctionTok{rast}\NormalTok{(}\StringTok{"./Templates/TemplateRasters/LV100m\_10km.tif"}\NormalTok{)}

\CommentTok{\# radii {-}{-}{-}{-}}
\FunctionTok{radius\_function}\NormalTok{(}
 \AttributeTok{kvadrati\_path =} \StringTok{"./Templates/TemplateGrids/tiles/"}\NormalTok{,}
 \AttributeTok{radii\_path   =} \StringTok{"./Templates/TemplateGridPoints/tiles/"}\NormalTok{,}
 \AttributeTok{tikls100\_path =} \StringTok{"./Templates/TemplateGrids/tikls100\_sauzeme.parquet"}\NormalTok{,}
 \AttributeTok{template\_path =} \StringTok{"./Templates/TemplateRasters/LV100m\_10km.tif"}\NormalTok{,}
 \AttributeTok{input\_layers  =} \FunctionTok{c}\NormalTok{(}\StringTok{"./RasterGrids\_100m/2024/RAW/ForestsTreesAge\_ConiferousOld\_cell.tif"}\NormalTok{),}
 \AttributeTok{layer\_prefixes =} \FunctionTok{c}\NormalTok{(}\StringTok{"ForestsTreesAge\_ConiferousOld"}\NormalTok{),}
 \AttributeTok{output\_dir   =} \StringTok{"./RasterGrids\_100m/2024/RAW/"}\NormalTok{,}
 \AttributeTok{n\_workers   =} \DecValTok{6}\NormalTok{,}
 \AttributeTok{radii     =} \FunctionTok{c}\NormalTok{(}\StringTok{"r3000"}\NormalTok{),}
 \AttributeTok{radius\_mode  =} \StringTok{"sparse"}\NormalTok{,}
 \AttributeTok{extract\_fun  =} \StringTok{"mean"}\NormalTok{,}
 \AttributeTok{fill\_missing  =} \ConstantTok{TRUE}\NormalTok{,}
 \AttributeTok{IDW\_weight   =} \DecValTok{2}\NormalTok{,}
 \AttributeTok{future\_max\_size =} \DecValTok{40} \SpecialCharTok{*} \DecValTok{1024}\SpecialCharTok{\^{}}\DecValTok{3}\NormalTok{)}


\CommentTok{\# ForestsTreesAge\_ConiferousOld\_r3000.tif   egv\_356}
\NormalTok{slanis}\OtherTok{=}\FunctionTok{rast}\NormalTok{(}\StringTok{"./RasterGrids\_100m/2024/RAW/ForestsTreesAge\_ConiferousOld\_r3000.tif"}\NormalTok{)}
\FunctionTok{names}\NormalTok{(slanis)}\OtherTok{=}\StringTok{"egv\_356"}
\NormalTok{slanis2}\OtherTok{=}\FunctionTok{project}\NormalTok{(slanis,template100)}
\FunctionTok{writeRaster}\NormalTok{(slanis2,}
      \StringTok{"./RasterGrids\_100m/2024/RAW/ForestsTreesAge\_ConiferousOld\_r3000.tif"}\NormalTok{,}
      \AttributeTok{overwrite=}\ConstantTok{TRUE}\NormalTok{)}

\CommentTok{\# standardisation {-}{-}{-}{-}}
\ControlFlowTok{if}\NormalTok{(}\SpecialCharTok{!}\FunctionTok{require}\NormalTok{(terra)) \{}\FunctionTok{install.packages}\NormalTok{(}\StringTok{"terra"}\NormalTok{); }\FunctionTok{require}\NormalTok{(terra)\}}
\ControlFlowTok{if}\NormalTok{(}\SpecialCharTok{!}\FunctionTok{require}\NormalTok{(tidyverse)) \{}\FunctionTok{install.packages}\NormalTok{(}\StringTok{"tidyverse"}\NormalTok{); }\FunctionTok{require}\NormalTok{(tidyverse)\}}

\NormalTok{nosaukums}\OtherTok{=}\StringTok{"ForestsTreesAge\_ConiferousOld\_r3000.tif"}
\NormalTok{ielasisanas\_cels}\OtherTok{=}\FunctionTok{paste0}\NormalTok{(}\StringTok{"./RasterGrids\_100m/2024/RAW/"}\NormalTok{,nosaukums)}
\NormalTok{saglabasanas\_cels}\OtherTok{=}\FunctionTok{paste0}\NormalTok{(}\StringTok{"./RasterGrids\_100m/2024/Scaled/"}\NormalTok{,nosaukums)}
\NormalTok{slanis}\OtherTok{=}\FunctionTok{rast}\NormalTok{(ielasisanas\_cels)}
\NormalTok{videjais}\OtherTok{=}\FunctionTok{global}\NormalTok{(slanis,}\AttributeTok{fun=}\StringTok{"mean"}\NormalTok{,}\AttributeTok{na.rm=}\ConstantTok{TRUE}\NormalTok{)}
\NormalTok{centrets}\OtherTok{=}\NormalTok{slanis}\SpecialCharTok{{-}}\NormalTok{videjais[,}\DecValTok{1}\NormalTok{]}
\NormalTok{standartnovirze}\OtherTok{=}\NormalTok{terra}\SpecialCharTok{::}\FunctionTok{global}\NormalTok{(centrets,}\AttributeTok{fun=}\StringTok{"rms"}\NormalTok{,}\AttributeTok{na.rm=}\ConstantTok{TRUE}\NormalTok{)}
\NormalTok{merogots}\OtherTok{=}\NormalTok{centrets}\SpecialCharTok{/}\NormalTok{standartnovirze[,}\DecValTok{1}\NormalTok{]}
\FunctionTok{writeRaster}\NormalTok{(merogots,}
      \AttributeTok{filename=}\NormalTok{saglabasanas\_cels,}
      \AttributeTok{overwrite=}\ConstantTok{TRUE}\NormalTok{)}
\end{Highlighting}
\end{Shaded}

\section{ForestsTreesAge\_ConiferousOld\_r10000}\label{ch06.357}

\textbf{filename:} \texttt{ForestsTreesAge\_ConiferousOld\_r10000.tif}

\textbf{layername:} \texttt{egv\_357}

\textbf{English name:} Fractional cover of Old (over rotation age) Coniferous Forests
within the 10 km landscape

\textbf{Latvian name:} Vecu (kopš cirtmeta) skujkoku mežu platības īpatsvars 10 km
ainavā

\textbf{Procedure:} The cover fraction within a radius of 10000 m around the analysis grid cell
is calculated as the area-weighted sum of the \hyperref[ch06.353]{analysis cells} inside
the buffer, using the workflow \texttt{egvtools::radius\_function()}. During the calculation of the landscape
metric, inverse distance weighted (power = 2) gap filling on the output is
applied to ensure no missing values at the edges. Then the layer is
rewritten to set its name. Finally, the layer is standardised by
subtracting the arithmetic mean and dividing by the root mean squared error.

\begin{Shaded}
\begin{Highlighting}[]
\CommentTok{\# libs {-}{-}{-}{-}}
\ControlFlowTok{if}\NormalTok{(}\SpecialCharTok{!}\FunctionTok{require}\NormalTok{(terra)) \{}\FunctionTok{install.packages}\NormalTok{(}\StringTok{"terra"}\NormalTok{); }\FunctionTok{require}\NormalTok{(terra)\}}
\ControlFlowTok{if}\NormalTok{(}\SpecialCharTok{!}\FunctionTok{require}\NormalTok{(egvtools)) \{remotes}\SpecialCharTok{::}\FunctionTok{install\_github}\NormalTok{(}\StringTok{"aavotins/egvtools"}\NormalTok{); }\FunctionTok{require}\NormalTok{(egvtools)\}}


\CommentTok{\# Templates {-}{-}{-}{-}{-}}
\NormalTok{template100}\OtherTok{=}\FunctionTok{rast}\NormalTok{(}\StringTok{"./Templates/TemplateRasters/LV100m\_10km.tif"}\NormalTok{)}

\CommentTok{\# radii {-}{-}{-}{-}}
\FunctionTok{radius\_function}\NormalTok{(}
 \AttributeTok{kvadrati\_path =} \StringTok{"./Templates/TemplateGrids/tiles/"}\NormalTok{,}
 \AttributeTok{radii\_path   =} \StringTok{"./Templates/TemplateGridPoints/tiles/"}\NormalTok{,}
 \AttributeTok{tikls100\_path =} \StringTok{"./Templates/TemplateGrids/tikls100\_sauzeme.parquet"}\NormalTok{,}
 \AttributeTok{template\_path =} \StringTok{"./Templates/TemplateRasters/LV100m\_10km.tif"}\NormalTok{,}
 \AttributeTok{input\_layers  =} \FunctionTok{c}\NormalTok{(}\StringTok{"./RasterGrids\_100m/2024/RAW/ForestsTreesAge\_ConiferousOld\_cell.tif"}\NormalTok{),}
 \AttributeTok{layer\_prefixes =} \FunctionTok{c}\NormalTok{(}\StringTok{"ForestsTreesAge\_ConiferousOld"}\NormalTok{),}
 \AttributeTok{output\_dir   =} \StringTok{"./RasterGrids\_100m/2024/RAW/"}\NormalTok{,}
 \AttributeTok{n\_workers   =} \DecValTok{6}\NormalTok{,}
 \AttributeTok{radii     =} \FunctionTok{c}\NormalTok{(}\StringTok{"r10000"}\NormalTok{),}
 \AttributeTok{radius\_mode  =} \StringTok{"sparse"}\NormalTok{,}
 \AttributeTok{extract\_fun  =} \StringTok{"mean"}\NormalTok{,}
 \AttributeTok{fill\_missing  =} \ConstantTok{TRUE}\NormalTok{,}
 \AttributeTok{IDW\_weight   =} \DecValTok{2}\NormalTok{,}
 \AttributeTok{future\_max\_size =} \DecValTok{40} \SpecialCharTok{*} \DecValTok{1024}\SpecialCharTok{\^{}}\DecValTok{3}\NormalTok{)}


\CommentTok{\# ForestsTreesAge\_ConiferousOld\_r10000.tif  egv\_357}
\NormalTok{slanis}\OtherTok{=}\FunctionTok{rast}\NormalTok{(}\StringTok{"./RasterGrids\_100m/2024/RAW/ForestsTreesAge\_ConiferousOld\_r10000.tif"}\NormalTok{)}
\FunctionTok{names}\NormalTok{(slanis)}\OtherTok{=}\StringTok{"egv\_357"}
\NormalTok{slanis2}\OtherTok{=}\FunctionTok{project}\NormalTok{(slanis,template100)}
\FunctionTok{writeRaster}\NormalTok{(slanis2,}
      \StringTok{"./RasterGrids\_100m/2024/RAW/ForestsTreesAge\_ConiferousOld\_r10000.tif"}\NormalTok{,}
      \AttributeTok{overwrite=}\ConstantTok{TRUE}\NormalTok{)}

\CommentTok{\# standardisation {-}{-}{-}{-}}
\ControlFlowTok{if}\NormalTok{(}\SpecialCharTok{!}\FunctionTok{require}\NormalTok{(terra)) \{}\FunctionTok{install.packages}\NormalTok{(}\StringTok{"terra"}\NormalTok{); }\FunctionTok{require}\NormalTok{(terra)\}}
\ControlFlowTok{if}\NormalTok{(}\SpecialCharTok{!}\FunctionTok{require}\NormalTok{(tidyverse)) \{}\FunctionTok{install.packages}\NormalTok{(}\StringTok{"tidyverse"}\NormalTok{); }\FunctionTok{require}\NormalTok{(tidyverse)\}}

\NormalTok{nosaukums}\OtherTok{=}\StringTok{"ForestsTreesAge\_ConiferousOld\_r10000.tif"}
\NormalTok{ielasisanas\_cels}\OtherTok{=}\FunctionTok{paste0}\NormalTok{(}\StringTok{"./RasterGrids\_100m/2024/RAW/"}\NormalTok{,nosaukums)}
\NormalTok{saglabasanas\_cels}\OtherTok{=}\FunctionTok{paste0}\NormalTok{(}\StringTok{"./RasterGrids\_100m/2024/Scaled/"}\NormalTok{,nosaukums)}
\NormalTok{slanis}\OtherTok{=}\FunctionTok{rast}\NormalTok{(ielasisanas\_cels)}
\NormalTok{videjais}\OtherTok{=}\FunctionTok{global}\NormalTok{(slanis,}\AttributeTok{fun=}\StringTok{"mean"}\NormalTok{,}\AttributeTok{na.rm=}\ConstantTok{TRUE}\NormalTok{)}
\NormalTok{centrets}\OtherTok{=}\NormalTok{slanis}\SpecialCharTok{{-}}\NormalTok{videjais[,}\DecValTok{1}\NormalTok{]}
\NormalTok{standartnovirze}\OtherTok{=}\NormalTok{terra}\SpecialCharTok{::}\FunctionTok{global}\NormalTok{(centrets,}\AttributeTok{fun=}\StringTok{"rms"}\NormalTok{,}\AttributeTok{na.rm=}\ConstantTok{TRUE}\NormalTok{)}
\NormalTok{merogots}\OtherTok{=}\NormalTok{centrets}\SpecialCharTok{/}\NormalTok{standartnovirze[,}\DecValTok{1}\NormalTok{]}
\FunctionTok{writeRaster}\NormalTok{(merogots,}
      \AttributeTok{filename=}\NormalTok{saglabasanas\_cels,}
      \AttributeTok{overwrite=}\ConstantTok{TRUE}\NormalTok{)}
\end{Highlighting}
\end{Shaded}

\section{ForestsTreesAge\_ConiferousYoung\_cell}\label{ch06.358}

\textbf{filename:} \texttt{ForestsTreesAge\_ConiferousYoung\_cell.tif}

\textbf{layername:} \texttt{egv\_358}

\textbf{English name:} Fractional cover of Young (pre-rotation age) Coniferous
Forests within the analysis cell (1 ha)

\textbf{Latvian name:} Jaunu (pirms cirtmeta) skujkoku mežu platības īpatsvars
analīzes šūnā (1 ha)

\textbf{Procedure:} Most EGVs describing forests are spatially restricted to areas outside
of clearcuts and dead stands. This mask is created using a combination of
the \hyperref[Ch04.01]{State Forest Service's
State Forest Registry} land category 12 and 14, and \hyperref[Ch04.09]{The
Global Forest Watch} pixels classified as lost tree canopy cover since
2020 (raster layer matching input, presence = 1, absence = 0).

To prepare this EGV, stands from the \hyperref[Ch04.01]{State Forest Service's State Forest
Registry} are classified into (in order):

\begin{itemize}
\item
  coniferous (see \hyperref[Ch01]{Terminology and acronyms} for species codes) if
  timber volume of those species exceeded 75\%;
\item
  Boreal deciduous if timber volume of those species exceeded 75\%;
\item
  temperate deciduous if timber volume of those species exceeded 50\%;
\item
  mixed otherwise;
\end{itemize}

then coniferous stands younger than the legal rotation age are selected and
geometries are rasterised (presence = 1, NA otherwise). Rasterisation is
performed using the workflow \texttt{egvtools::polygon2input()}, restricting to pixels outside clearcut
mask and covering background with value 0. The resulting layer
is then aggregated to EGV resolution using the workflow \texttt{egvtools::input2egv()}, which
calculates the arithmetic mean to determine the cover fraction. During
aggregation, inverse distance weighted (power = 2) gap filling on the output is
applied to ensure no missing values at the edges. Finally, the layer is
standardised by subtracting the arithmetic mean and dividing by the root mean squared
error.

\begin{Shaded}
\begin{Highlighting}[]
\CommentTok{\# libs {-}{-}{-}{-}}
\ControlFlowTok{if}\NormalTok{(}\SpecialCharTok{!}\FunctionTok{require}\NormalTok{(egvtools)) \{remotes}\SpecialCharTok{::}\FunctionTok{install\_github}\NormalTok{(}\StringTok{"aavotins/egvtools"}\NormalTok{); }\FunctionTok{require}\NormalTok{(egvtools)\}}
\ControlFlowTok{if}\NormalTok{(}\SpecialCharTok{!}\FunctionTok{require}\NormalTok{(terra)) \{}\FunctionTok{install.packages}\NormalTok{(}\StringTok{"terra"}\NormalTok{); }\FunctionTok{require}\NormalTok{(terra)\}}
\ControlFlowTok{if}\NormalTok{(}\SpecialCharTok{!}\FunctionTok{require}\NormalTok{(sf)) \{}\FunctionTok{install.packages}\NormalTok{(}\StringTok{"sf"}\NormalTok{); }\FunctionTok{require}\NormalTok{(sf)\}}
\ControlFlowTok{if}\NormalTok{(}\SpecialCharTok{!}\FunctionTok{require}\NormalTok{(tidyverse)) \{}\FunctionTok{install.packages}\NormalTok{(}\StringTok{"tidyverse"}\NormalTok{); }\FunctionTok{require}\NormalTok{(tidyverse)\}}
\ControlFlowTok{if}\NormalTok{(}\SpecialCharTok{!}\FunctionTok{require}\NormalTok{(sfarrow)) \{}\FunctionTok{install.packages}\NormalTok{(}\StringTok{"sfarrow"}\NormalTok{); }\FunctionTok{require}\NormalTok{(sfarrow)\}}
\ControlFlowTok{if}\NormalTok{(}\SpecialCharTok{!}\FunctionTok{require}\NormalTok{(readxl)) \{}\FunctionTok{install.packages}\NormalTok{(}\StringTok{"readxl"}\NormalTok{); }\FunctionTok{require}\NormalTok{(readxl)\}}
\ControlFlowTok{if}\NormalTok{(}\SpecialCharTok{!}\FunctionTok{require}\NormalTok{(raster)) \{}\FunctionTok{install.packages}\NormalTok{(}\StringTok{"raster"}\NormalTok{); }\FunctionTok{require}\NormalTok{(raster)\}}
\ControlFlowTok{if}\NormalTok{(}\SpecialCharTok{!}\FunctionTok{require}\NormalTok{(fasterize)) \{}\FunctionTok{install.packages}\NormalTok{(}\StringTok{"fasterize"}\NormalTok{); }\FunctionTok{require}\NormalTok{(fasterize)\}}

\CommentTok{\# templates {-}{-}{-}{-}}
\NormalTok{template100}\OtherTok{=}\FunctionTok{rast}\NormalTok{(}\StringTok{"./Templates/TemplateRasters/LV100m\_10km.tif"}\NormalTok{)}
\NormalTok{template10}\OtherTok{=}\FunctionTok{rast}\NormalTok{(}\StringTok{"./Templates/TemplateRasters/LV10m\_10km.tif"}\NormalTok{)}
\NormalTok{rastrs10}\OtherTok{=}\FunctionTok{raster}\NormalTok{(template10)}

\NormalTok{nulls10}\OtherTok{=}\FunctionTok{rast}\NormalTok{(}\StringTok{"./Templates/TemplateRasters/nulls\_LV10m\_10km.tif"}\NormalTok{)}
\NormalTok{nulls100}\OtherTok{=}\FunctionTok{rast}\NormalTok{(}\StringTok{"./Templates/TemplateRasters/nulls\_LV100m\_10km.tif"}\NormalTok{)}


\CommentTok{\# simple landscape {-}{-}{-}{-}}
\NormalTok{simple\_landscape}\OtherTok{=}\FunctionTok{rast}\NormalTok{(}\StringTok{"RasterGrids\_10m/2024/Ainava\_vienk\_mask.tif"}\NormalTok{)}

\CommentTok{\# mvr {-}{-}{-}{-}}
\NormalTok{mvr}\OtherTok{=}\FunctionTok{st\_read\_parquet}\NormalTok{(}\StringTok{"./Geodata/2024/MVR/nogabali\_2024janv.parquet"}\NormalTok{)}
\NormalTok{mvr}\SpecialCharTok{$}\NormalTok{yes}\OtherTok{=}\DecValTok{1}

\CommentTok{\# clear cut mask {-}{-}{-}{-}}
\NormalTok{izcirtumi}\OtherTok{=}\NormalTok{mvr }\SpecialCharTok{\%\textgreater{}\%} 
 \FunctionTok{filter}\NormalTok{(zkat }\SpecialCharTok{\%in\%} \FunctionTok{c}\NormalTok{(}\StringTok{"12"}\NormalTok{,}\StringTok{"14"}\NormalTok{)) }\SpecialCharTok{\%\textgreater{}\%} 
\NormalTok{ dplyr}\SpecialCharTok{::}\FunctionTok{select}\NormalTok{(yes)}
\NormalTok{r\_izcirtumi\_mvr}\OtherTok{=}\FunctionTok{fasterize}\NormalTok{(izcirtumi,rastrs10,}\AttributeTok{field=}\StringTok{"yes"}\NormalTok{)}
\NormalTok{t\_izcirtumi\_mvr}\OtherTok{=}\FunctionTok{rast}\NormalTok{(r\_izcirtumi\_mvr)}
\FunctionTok{plot}\NormalTok{(t\_izcirtumi\_mvr)}

\NormalTok{tcl}\OtherTok{=}\FunctionTok{rast}\NormalTok{(}\StringTok{"./Geodata/2024/Trees/GFW/TreeCoverLoss\_v1\_12.tif"}\NormalTok{)}
\NormalTok{tcl2}\OtherTok{=}\FunctionTok{ifel}\NormalTok{(tcl}\SpecialCharTok{\textless{}}\DecValTok{20}\NormalTok{,}\DecValTok{0}\NormalTok{,}\DecValTok{1}\NormalTok{)}
\NormalTok{tclX}\OtherTok{=}\FunctionTok{cover}\NormalTok{(tcl2,nulls10)}
\FunctionTok{plot}\NormalTok{(tclX)}

\NormalTok{clearcut\_mask}\OtherTok{=}\FunctionTok{cover}\NormalTok{(t\_izcirtumi\_mvr,tclX,}
          \AttributeTok{filename=}\StringTok{"./RasterGrids\_10m/2024/Mask\_clearcuts.tif"}\NormalTok{,}
          \AttributeTok{overwrite=}\ConstantTok{TRUE}\NormalTok{)}
\FunctionTok{plot}\NormalTok{(clearcut\_mask)}

\FunctionTok{rm}\NormalTok{(izcirtumi)}
\FunctionTok{rm}\NormalTok{(r\_izcirtumi\_mvr)}
\FunctionTok{rm}\NormalTok{(t\_izcirtumi\_mvr)}
\FunctionTok{rm}\NormalTok{(tcl)}
\FunctionTok{rm}\NormalTok{(tcl2)}
\FunctionTok{rm}\NormalTok{(tclX)}

\CommentTok{\# ForestsTreesAge\_ConiferousYoung\_cell.tif  egv\_358 {-}{-}{-}{-}}
\NormalTok{skujkoki}\OtherTok{=}\FunctionTok{c}\NormalTok{(}\StringTok{"1"}\NormalTok{,}\StringTok{"3"}\NormalTok{,}\StringTok{"13"}\NormalTok{,}\StringTok{"14"}\NormalTok{,}\StringTok{"15"}\NormalTok{,}\StringTok{"22"}\NormalTok{,}\StringTok{"23"}\NormalTok{,}\StringTok{"28"}\NormalTok{) }\CommentTok{\# 8}
\NormalTok{saurlapji}\OtherTok{=}\FunctionTok{c}\NormalTok{(}\StringTok{"4"}\NormalTok{,}\StringTok{"6"}\NormalTok{,}\StringTok{"8"}\NormalTok{,}\StringTok{"9"}\NormalTok{,}\StringTok{"19"}\NormalTok{,}\StringTok{"20"}\NormalTok{,}\StringTok{"21"}\NormalTok{,}\StringTok{"32"}\NormalTok{,}\StringTok{"35"}\NormalTok{,}\StringTok{"68"}\NormalTok{) }\CommentTok{\# 10}
\NormalTok{platlapji}\OtherTok{=}\FunctionTok{c}\NormalTok{(}\StringTok{"10"}\NormalTok{,}\StringTok{"11"}\NormalTok{,}\StringTok{"12"}\NormalTok{,}\StringTok{"16"}\NormalTok{,}\StringTok{"17"}\NormalTok{,}\StringTok{"18"}\NormalTok{,}\StringTok{"24"}\NormalTok{,}\StringTok{"25"}\NormalTok{,}\StringTok{"26"}\NormalTok{,}\StringTok{"27"}\NormalTok{,}\StringTok{"28"}\NormalTok{,}\StringTok{"29"}\NormalTok{,}\StringTok{"50"}\NormalTok{,}
      \StringTok{"61"}\NormalTok{,}\StringTok{"62"}\NormalTok{,}\StringTok{"63"}\NormalTok{,}\StringTok{"64"}\NormalTok{,}\StringTok{"65"}\NormalTok{,}\StringTok{"66"}\NormalTok{,}\StringTok{"67"}\NormalTok{,}\StringTok{"69"}\NormalTok{) }\CommentTok{\# 21}
\NormalTok{mvr}\OtherTok{=}\NormalTok{mvr }\SpecialCharTok{\%\textgreater{}\%} 
 \FunctionTok{mutate}\NormalTok{(}\AttributeTok{kraja\_skujkoku=}\FunctionTok{ifelse}\NormalTok{(s10 }\SpecialCharTok{\%in\%}\NormalTok{ skujkoki,v10,}\DecValTok{0}\NormalTok{)}\SpecialCharTok{+}
      \FunctionTok{ifelse}\NormalTok{(s11 }\SpecialCharTok{\%in\%}\NormalTok{ skujkoki,v11,}\DecValTok{0}\NormalTok{)}\SpecialCharTok{+}\FunctionTok{ifelse}\NormalTok{(s12 }\SpecialCharTok{\%in\%}\NormalTok{ skujkoki,v12,}\DecValTok{0}\NormalTok{)}\SpecialCharTok{+}
      \FunctionTok{ifelse}\NormalTok{(s13 }\SpecialCharTok{\%in\%}\NormalTok{ skujkoki,v13,}\DecValTok{0}\NormalTok{)}\SpecialCharTok{+}\FunctionTok{ifelse}\NormalTok{(s14 }\SpecialCharTok{\%in\%}\NormalTok{ skujkoki,v14,}\DecValTok{0}\NormalTok{),}
     \AttributeTok{kraja\_saurlapju=}\FunctionTok{ifelse}\NormalTok{(s10 }\SpecialCharTok{\%in\%}\NormalTok{ saurlapji,v10,}\DecValTok{0}\NormalTok{)}\SpecialCharTok{+}
      \FunctionTok{ifelse}\NormalTok{(s11 }\SpecialCharTok{\%in\%}\NormalTok{ saurlapji,v11,}\DecValTok{0}\NormalTok{)}\SpecialCharTok{+}\FunctionTok{ifelse}\NormalTok{(s12 }\SpecialCharTok{\%in\%}\NormalTok{ saurlapji,v12,}\DecValTok{0}\NormalTok{)}\SpecialCharTok{+}
      \FunctionTok{ifelse}\NormalTok{(s13 }\SpecialCharTok{\%in\%}\NormalTok{ saurlapji,v13,}\DecValTok{0}\NormalTok{)}\SpecialCharTok{+}\FunctionTok{ifelse}\NormalTok{(s14 }\SpecialCharTok{\%in\%}\NormalTok{ saurlapji,v14,}\DecValTok{0}\NormalTok{),}
     \AttributeTok{kraja\_platlapju=}\FunctionTok{ifelse}\NormalTok{(s10 }\SpecialCharTok{\%in\%}\NormalTok{ platlapji,v10,}\DecValTok{0}\NormalTok{)}\SpecialCharTok{+}
      \FunctionTok{ifelse}\NormalTok{(s11 }\SpecialCharTok{\%in\%}\NormalTok{ platlapji,v11,}\DecValTok{0}\NormalTok{)}\SpecialCharTok{+}\FunctionTok{ifelse}\NormalTok{(s12 }\SpecialCharTok{\%in\%}\NormalTok{ platlapji,v12,}\DecValTok{0}\NormalTok{)}\SpecialCharTok{+}
      \FunctionTok{ifelse}\NormalTok{(s13 }\SpecialCharTok{\%in\%}\NormalTok{ platlapji,v13,}\DecValTok{0}\NormalTok{)}\SpecialCharTok{+}\FunctionTok{ifelse}\NormalTok{(s14 }\SpecialCharTok{\%in\%}\NormalTok{ platlapji,v14,}\DecValTok{0}\NormalTok{)) }\SpecialCharTok{\%\textgreater{}\%} 
 \FunctionTok{mutate}\NormalTok{(}\AttributeTok{kopeja\_kraja=}\NormalTok{kraja\_skujkoku}\SpecialCharTok{+}\NormalTok{kraja\_platlapju}\SpecialCharTok{+}\NormalTok{kraja\_saurlapju) }\SpecialCharTok{\%\textgreater{}\%} 
 \FunctionTok{mutate}\NormalTok{(}\AttributeTok{tips=}\FunctionTok{ifelse}\NormalTok{(kraja\_skujkoku}\SpecialCharTok{/}\NormalTok{kopeja\_kraja}\SpecialCharTok{\textgreater{}=}\FloatTok{0.75}\NormalTok{,}\StringTok{"skujkoku"}\NormalTok{,}
           \FunctionTok{ifelse}\NormalTok{(kraja\_saurlapju}\SpecialCharTok{/}\NormalTok{kopeja\_kraja}\SpecialCharTok{\textgreater{}=}\FloatTok{0.75}\NormalTok{,}\StringTok{"saurlapju"}\NormalTok{,}
              \FunctionTok{ifelse}\NormalTok{(kraja\_platlapju}\SpecialCharTok{/}\NormalTok{kopeja\_kraja}\SpecialCharTok{\textgreater{}}\FloatTok{0.5}\NormalTok{,}\StringTok{"platlapju"}\NormalTok{,}
                  \StringTok{"jauktu koku"}\NormalTok{))))}
\NormalTok{nogabali}\OtherTok{=}\NormalTok{mvr }\SpecialCharTok{\%\textgreater{}\%} 
 \FunctionTok{filter}\NormalTok{(zkat}\SpecialCharTok{==}\StringTok{"10"}\SpecialCharTok{\&}\NormalTok{tips}\SpecialCharTok{==}\StringTok{"skujkoku"}\SpecialCharTok{\&}\NormalTok{(vgr}\SpecialCharTok{==}\StringTok{"1"}\SpecialCharTok{|}\NormalTok{vgr}\SpecialCharTok{==}\StringTok{"2"}\SpecialCharTok{|}\NormalTok{vgr}\SpecialCharTok{==}\StringTok{"3"}\NormalTok{))}

\NormalTok{p2i\_rez}\OtherTok{=}\NormalTok{egvtools}\SpecialCharTok{::}\FunctionTok{polygon2input}\NormalTok{(}\AttributeTok{vector\_data =}\NormalTok{ nogabali,}
                \AttributeTok{template\_path =} \StringTok{"./Templates/TemplateRasters/LV10m\_10km.tif"}\NormalTok{,}
                \AttributeTok{out\_path =} \StringTok{"./RasterGrids\_10m/2024/"}\NormalTok{,}
                \AttributeTok{file\_name =} \StringTok{"ForestsTreesAge\_ConiferousYoung\_input.tif"}\NormalTok{,}
                \AttributeTok{value\_field =} \StringTok{"yes"}\NormalTok{,}
                \AttributeTok{restrict\_to =}\NormalTok{ clearcut\_mask,}
                \AttributeTok{restrict\_values =} \DecValTok{0}\NormalTok{,}
                \AttributeTok{prepare=}\ConstantTok{FALSE}\NormalTok{,}
                \AttributeTok{background\_raster =} \StringTok{"./Templates/TemplateRasters/nulls\_LV10m\_10km.tif"}\NormalTok{,}
                \AttributeTok{plot\_result =} \ConstantTok{TRUE}\NormalTok{)}
\NormalTok{p2i\_rez}
\NormalTok{i2e\_rez}\OtherTok{=}\NormalTok{egvtools}\SpecialCharTok{::}\FunctionTok{input2egv}\NormalTok{(}\AttributeTok{input=}\FunctionTok{paste0}\NormalTok{(}\StringTok{"./RasterGrids\_10m/2024/"}\NormalTok{,}
                     \StringTok{"ForestsTreesAge\_ConiferousYoung\_input.tif"}\NormalTok{),}
              \AttributeTok{egv\_template=} \StringTok{"./Templates/TemplateRasters/LV100m\_10km.tif"}\NormalTok{,}
              \AttributeTok{summary\_function =} \StringTok{"average"}\NormalTok{,}
              \AttributeTok{missing\_job =} \StringTok{"FillOutput"}\NormalTok{,}
              \AttributeTok{outlocation =} \StringTok{"./RasterGrids\_100m/2024/RAW/"}\NormalTok{,}
              \AttributeTok{outfilename =} \StringTok{"ForestsTreesAge\_ConiferousYoung\_cell.tif"}\NormalTok{,}
              \AttributeTok{layername =} \StringTok{"egv\_358"}\NormalTok{,}
              \AttributeTok{idw\_weight =} \DecValTok{2}\NormalTok{,}
              \AttributeTok{plot\_gaps =} \ConstantTok{FALSE}\NormalTok{,}\AttributeTok{plot\_final =} \ConstantTok{TRUE}\NormalTok{)}
\NormalTok{i2e\_rez}
\FunctionTok{rm}\NormalTok{(nogabali)}
\FunctionTok{rm}\NormalTok{(p2i\_rez)}
\FunctionTok{rm}\NormalTok{(i2e\_rez)}
\FunctionTok{unlink}\NormalTok{(}\StringTok{"./RasterGrids\_10m/2024/ForestsTreesAge\_ConiferousYoung\_input.tif"}\NormalTok{)}

\CommentTok{\# standardisation {-}{-}{-}{-}}
\ControlFlowTok{if}\NormalTok{(}\SpecialCharTok{!}\FunctionTok{require}\NormalTok{(terra)) \{}\FunctionTok{install.packages}\NormalTok{(}\StringTok{"terra"}\NormalTok{); }\FunctionTok{require}\NormalTok{(terra)\}}
\ControlFlowTok{if}\NormalTok{(}\SpecialCharTok{!}\FunctionTok{require}\NormalTok{(tidyverse)) \{}\FunctionTok{install.packages}\NormalTok{(}\StringTok{"tidyverse"}\NormalTok{); }\FunctionTok{require}\NormalTok{(tidyverse)\}}

\NormalTok{nosaukums}\OtherTok{=}\StringTok{"ForestsTreesAge\_ConiferousYoung\_cell.tif"}
\NormalTok{ielasisanas\_cels}\OtherTok{=}\FunctionTok{paste0}\NormalTok{(}\StringTok{"./RasterGrids\_100m/2024/RAW/"}\NormalTok{,nosaukums)}
\NormalTok{saglabasanas\_cels}\OtherTok{=}\FunctionTok{paste0}\NormalTok{(}\StringTok{"./RasterGrids\_100m/2024/Scaled/"}\NormalTok{,nosaukums)}
\NormalTok{slanis}\OtherTok{=}\FunctionTok{rast}\NormalTok{(ielasisanas\_cels)}
\NormalTok{videjais}\OtherTok{=}\FunctionTok{global}\NormalTok{(slanis,}\AttributeTok{fun=}\StringTok{"mean"}\NormalTok{,}\AttributeTok{na.rm=}\ConstantTok{TRUE}\NormalTok{)}
\NormalTok{centrets}\OtherTok{=}\NormalTok{slanis}\SpecialCharTok{{-}}\NormalTok{videjais[,}\DecValTok{1}\NormalTok{]}
\NormalTok{standartnovirze}\OtherTok{=}\NormalTok{terra}\SpecialCharTok{::}\FunctionTok{global}\NormalTok{(centrets,}\AttributeTok{fun=}\StringTok{"rms"}\NormalTok{,}\AttributeTok{na.rm=}\ConstantTok{TRUE}\NormalTok{)}
\NormalTok{merogots}\OtherTok{=}\NormalTok{centrets}\SpecialCharTok{/}\NormalTok{standartnovirze[,}\DecValTok{1}\NormalTok{]}
\FunctionTok{writeRaster}\NormalTok{(merogots,}
      \AttributeTok{filename=}\NormalTok{saglabasanas\_cels,}
      \AttributeTok{overwrite=}\ConstantTok{TRUE}\NormalTok{)}
\end{Highlighting}
\end{Shaded}

\section{ForestsTreesAge\_ConiferousYoung\_r500}\label{ch06.359}

\textbf{filename:} \texttt{ForestsTreesAge\_ConiferousYoung\_r500.tif}

\textbf{layername:} \texttt{egv\_359}

\textbf{English name:} Fractional cover of Young (pre-rotation age) Coniferous
Forests within the 0.5 km landscape

\textbf{Latvian name:} Jaunu (pirms cirtmeta) skujkoku mežu platības īpatsvars 0,5 km
ainavā

\textbf{Procedure:} The cover fraction within a radius of 500 m around the analysis grid cell is
calculated as the area-weighted sum of the \hyperref[ch06.358]{analysis cells} inside the
buffer, using the workflow \texttt{egvtools::radius\_function()}. During the calculation of the landscape metric,
inverse distance weighted (power = 2) gap filling on the output is applied
to ensure no missing values at the edges. Then the layer is rewritten to set
its name. Finally, the layer is standardised by subtracting the arithmetic
mean and dividing by the root mean squared error.

\begin{Shaded}
\begin{Highlighting}[]
\CommentTok{\# libs {-}{-}{-}{-}}
\ControlFlowTok{if}\NormalTok{(}\SpecialCharTok{!}\FunctionTok{require}\NormalTok{(terra)) \{}\FunctionTok{install.packages}\NormalTok{(}\StringTok{"terra"}\NormalTok{); }\FunctionTok{require}\NormalTok{(terra)\}}
\ControlFlowTok{if}\NormalTok{(}\SpecialCharTok{!}\FunctionTok{require}\NormalTok{(egvtools)) \{remotes}\SpecialCharTok{::}\FunctionTok{install\_github}\NormalTok{(}\StringTok{"aavotins/egvtools"}\NormalTok{); }\FunctionTok{require}\NormalTok{(egvtools)\}}


\CommentTok{\# Templates {-}{-}{-}{-}{-}}
\NormalTok{template100}\OtherTok{=}\FunctionTok{rast}\NormalTok{(}\StringTok{"./Templates/TemplateRasters/LV100m\_10km.tif"}\NormalTok{)}

\CommentTok{\# radii {-}{-}{-}{-}}
\FunctionTok{radius\_function}\NormalTok{(}
 \AttributeTok{kvadrati\_path =} \StringTok{"./Templates/TemplateGrids/tiles/"}\NormalTok{,}
 \AttributeTok{radii\_path   =} \StringTok{"./Templates/TemplateGridPoints/tiles/"}\NormalTok{,}
 \AttributeTok{tikls100\_path =} \StringTok{"./Templates/TemplateGrids/tikls100\_sauzeme.parquet"}\NormalTok{,}
 \AttributeTok{template\_path =} \StringTok{"./Templates/TemplateRasters/LV100m\_10km.tif"}\NormalTok{,}
 \AttributeTok{input\_layers  =} \FunctionTok{c}\NormalTok{(}\StringTok{"./RasterGrids\_100m/2024/RAW/ForestsTreesAge\_ConiferousYoung\_cell.tif"}\NormalTok{),}
 \AttributeTok{layer\_prefixes =} \FunctionTok{c}\NormalTok{(}\StringTok{"ForestsTreesAge\_ConiferousYoung"}\NormalTok{),}
 \AttributeTok{output\_dir   =} \StringTok{"./RasterGrids\_100m/2024/RAW/"}\NormalTok{,}
 \AttributeTok{n\_workers   =} \DecValTok{6}\NormalTok{,}
 \AttributeTok{radii     =} \FunctionTok{c}\NormalTok{(}\StringTok{"r500"}\NormalTok{),}
 \AttributeTok{radius\_mode  =} \StringTok{"sparse"}\NormalTok{,}
 \AttributeTok{extract\_fun  =} \StringTok{"mean"}\NormalTok{,}
 \AttributeTok{fill\_missing  =} \ConstantTok{TRUE}\NormalTok{,}
 \AttributeTok{IDW\_weight   =} \DecValTok{2}\NormalTok{,}
 \AttributeTok{future\_max\_size =} \DecValTok{40} \SpecialCharTok{*} \DecValTok{1024}\SpecialCharTok{\^{}}\DecValTok{3}\NormalTok{)}


\CommentTok{\# ForestsTreesAge\_ConiferousYoung\_r500.tif  egv\_359}
\NormalTok{slanis}\OtherTok{=}\FunctionTok{rast}\NormalTok{(}\StringTok{"./RasterGrids\_100m/2024/RAW/ForestsTreesAge\_ConiferousYoung\_r500.tif"}\NormalTok{)}
\FunctionTok{names}\NormalTok{(slanis)}\OtherTok{=}\StringTok{"egv\_359"}
\NormalTok{slanis2}\OtherTok{=}\FunctionTok{project}\NormalTok{(slanis,template100)}
\FunctionTok{writeRaster}\NormalTok{(slanis2,}
      \StringTok{"./RasterGrids\_100m/2024/RAW/ForestsTreesAge\_ConiferousYoung\_r500.tif"}\NormalTok{,}
      \AttributeTok{overwrite=}\ConstantTok{TRUE}\NormalTok{)}

\CommentTok{\# standardisation {-}{-}{-}{-}}
\ControlFlowTok{if}\NormalTok{(}\SpecialCharTok{!}\FunctionTok{require}\NormalTok{(terra)) \{}\FunctionTok{install.packages}\NormalTok{(}\StringTok{"terra"}\NormalTok{); }\FunctionTok{require}\NormalTok{(terra)\}}
\ControlFlowTok{if}\NormalTok{(}\SpecialCharTok{!}\FunctionTok{require}\NormalTok{(tidyverse)) \{}\FunctionTok{install.packages}\NormalTok{(}\StringTok{"tidyverse"}\NormalTok{); }\FunctionTok{require}\NormalTok{(tidyverse)\}}

\NormalTok{nosaukums}\OtherTok{=}\StringTok{"ForestsTreesAge\_ConiferousYoung\_r500.tif"}
\NormalTok{ielasisanas\_cels}\OtherTok{=}\FunctionTok{paste0}\NormalTok{(}\StringTok{"./RasterGrids\_100m/2024/RAW/"}\NormalTok{,nosaukums)}
\NormalTok{saglabasanas\_cels}\OtherTok{=}\FunctionTok{paste0}\NormalTok{(}\StringTok{"./RasterGrids\_100m/2024/Scaled/"}\NormalTok{,nosaukums)}
\NormalTok{slanis}\OtherTok{=}\FunctionTok{rast}\NormalTok{(ielasisanas\_cels)}
\NormalTok{videjais}\OtherTok{=}\FunctionTok{global}\NormalTok{(slanis,}\AttributeTok{fun=}\StringTok{"mean"}\NormalTok{,}\AttributeTok{na.rm=}\ConstantTok{TRUE}\NormalTok{)}
\NormalTok{centrets}\OtherTok{=}\NormalTok{slanis}\SpecialCharTok{{-}}\NormalTok{videjais[,}\DecValTok{1}\NormalTok{]}
\NormalTok{standartnovirze}\OtherTok{=}\NormalTok{terra}\SpecialCharTok{::}\FunctionTok{global}\NormalTok{(centrets,}\AttributeTok{fun=}\StringTok{"rms"}\NormalTok{,}\AttributeTok{na.rm=}\ConstantTok{TRUE}\NormalTok{)}
\NormalTok{merogots}\OtherTok{=}\NormalTok{centrets}\SpecialCharTok{/}\NormalTok{standartnovirze[,}\DecValTok{1}\NormalTok{]}
\FunctionTok{writeRaster}\NormalTok{(merogots,}
      \AttributeTok{filename=}\NormalTok{saglabasanas\_cels,}
      \AttributeTok{overwrite=}\ConstantTok{TRUE}\NormalTok{)}
\end{Highlighting}
\end{Shaded}

\section{ForestsTreesAge\_ConiferousYoung\_r1250}\label{ch06.360}

\textbf{filename:} \texttt{ForestsTreesAge\_ConiferousYoung\_r1250.tif}

\textbf{layername:} \texttt{egv\_360}

\textbf{English name:} Fractional cover of Young (pre-rotation age) Coniferous
Forests within the 1.25 km landscape

\textbf{Latvian name:} Jaunu (pirms cirtmeta) skujkoku mežu platības īpatsvars 1,25
km ainavā

\textbf{Procedure:} The cover fraction within a radius of 1250 m around the analysis grid cell
is calculated as the area-weighted sum of the \hyperref[ch06.358]{analysis cells} inside
the buffer, using the workflow \texttt{egvtools::radius\_function()}. During the calculation of the landscape
metric, inverse distance weighted (power = 2) gap filling on the output is
applied to ensure no missing values at the edges. Then the layer is
rewritten to set its name. Finally, the layer is standardised by
subtracting the arithmetic mean and dividing by the root mean squared error.

\begin{Shaded}
\begin{Highlighting}[]
\CommentTok{\# libs {-}{-}{-}{-}}
\ControlFlowTok{if}\NormalTok{(}\SpecialCharTok{!}\FunctionTok{require}\NormalTok{(terra)) \{}\FunctionTok{install.packages}\NormalTok{(}\StringTok{"terra"}\NormalTok{); }\FunctionTok{require}\NormalTok{(terra)\}}
\ControlFlowTok{if}\NormalTok{(}\SpecialCharTok{!}\FunctionTok{require}\NormalTok{(egvtools)) \{remotes}\SpecialCharTok{::}\FunctionTok{install\_github}\NormalTok{(}\StringTok{"aavotins/egvtools"}\NormalTok{); }\FunctionTok{require}\NormalTok{(egvtools)\}}


\CommentTok{\# Templates {-}{-}{-}{-}{-}}
\NormalTok{template100}\OtherTok{=}\FunctionTok{rast}\NormalTok{(}\StringTok{"./Templates/TemplateRasters/LV100m\_10km.tif"}\NormalTok{)}

\CommentTok{\# radii {-}{-}{-}{-}}
\FunctionTok{radius\_function}\NormalTok{(}
 \AttributeTok{kvadrati\_path =} \StringTok{"./Templates/TemplateGrids/tiles/"}\NormalTok{,}
 \AttributeTok{radii\_path   =} \StringTok{"./Templates/TemplateGridPoints/tiles/"}\NormalTok{,}
 \AttributeTok{tikls100\_path =} \StringTok{"./Templates/TemplateGrids/tikls100\_sauzeme.parquet"}\NormalTok{,}
 \AttributeTok{template\_path =} \StringTok{"./Templates/TemplateRasters/LV100m\_10km.tif"}\NormalTok{,}
 \AttributeTok{input\_layers  =} \FunctionTok{c}\NormalTok{(}\StringTok{"./RasterGrids\_100m/2024/RAW/ForestsTreesAge\_ConiferousYoung\_cell.tif"}\NormalTok{),}
 \AttributeTok{layer\_prefixes =} \FunctionTok{c}\NormalTok{(}\StringTok{"ForestsTreesAge\_ConiferousYoung"}\NormalTok{),}
 \AttributeTok{output\_dir   =} \StringTok{"./RasterGrids\_100m/2024/RAW/"}\NormalTok{,}
 \AttributeTok{n\_workers   =} \DecValTok{6}\NormalTok{,}
 \AttributeTok{radii     =} \FunctionTok{c}\NormalTok{(}\StringTok{"r1250"}\NormalTok{),}
 \AttributeTok{radius\_mode  =} \StringTok{"sparse"}\NormalTok{,}
 \AttributeTok{extract\_fun  =} \StringTok{"mean"}\NormalTok{,}
 \AttributeTok{fill\_missing  =} \ConstantTok{TRUE}\NormalTok{,}
 \AttributeTok{IDW\_weight   =} \DecValTok{2}\NormalTok{,}
 \AttributeTok{future\_max\_size =} \DecValTok{40} \SpecialCharTok{*} \DecValTok{1024}\SpecialCharTok{\^{}}\DecValTok{3}\NormalTok{)}


\CommentTok{\# ForestsTreesAge\_ConiferousYoung\_r1250.tif egv\_360}
\NormalTok{slanis}\OtherTok{=}\FunctionTok{rast}\NormalTok{(}\StringTok{"./RasterGrids\_100m/2024/RAW/ForestsTreesAge\_ConiferousYoung\_r1250.tif"}\NormalTok{)}
\FunctionTok{names}\NormalTok{(slanis)}\OtherTok{=}\StringTok{"egv\_360"}
\NormalTok{slanis2}\OtherTok{=}\FunctionTok{project}\NormalTok{(slanis,template100)}
\FunctionTok{writeRaster}\NormalTok{(slanis2,}
      \StringTok{"./RasterGrids\_100m/2024/RAW/ForestsTreesAge\_ConiferousYoung\_r1250.tif"}\NormalTok{,}
      \AttributeTok{overwrite=}\ConstantTok{TRUE}\NormalTok{)}

\CommentTok{\# standardisation {-}{-}{-}{-}}
\ControlFlowTok{if}\NormalTok{(}\SpecialCharTok{!}\FunctionTok{require}\NormalTok{(terra)) \{}\FunctionTok{install.packages}\NormalTok{(}\StringTok{"terra"}\NormalTok{); }\FunctionTok{require}\NormalTok{(terra)\}}
\ControlFlowTok{if}\NormalTok{(}\SpecialCharTok{!}\FunctionTok{require}\NormalTok{(tidyverse)) \{}\FunctionTok{install.packages}\NormalTok{(}\StringTok{"tidyverse"}\NormalTok{); }\FunctionTok{require}\NormalTok{(tidyverse)\}}

\NormalTok{nosaukums}\OtherTok{=}\StringTok{"ForestsTreesAge\_ConiferousYoung\_r1250.tif"}
\NormalTok{ielasisanas\_cels}\OtherTok{=}\FunctionTok{paste0}\NormalTok{(}\StringTok{"./RasterGrids\_100m/2024/RAW/"}\NormalTok{,nosaukums)}
\NormalTok{saglabasanas\_cels}\OtherTok{=}\FunctionTok{paste0}\NormalTok{(}\StringTok{"./RasterGrids\_100m/2024/Scaled/"}\NormalTok{,nosaukums)}
\NormalTok{slanis}\OtherTok{=}\FunctionTok{rast}\NormalTok{(ielasisanas\_cels)}
\NormalTok{videjais}\OtherTok{=}\FunctionTok{global}\NormalTok{(slanis,}\AttributeTok{fun=}\StringTok{"mean"}\NormalTok{,}\AttributeTok{na.rm=}\ConstantTok{TRUE}\NormalTok{)}
\NormalTok{centrets}\OtherTok{=}\NormalTok{slanis}\SpecialCharTok{{-}}\NormalTok{videjais[,}\DecValTok{1}\NormalTok{]}
\NormalTok{standartnovirze}\OtherTok{=}\NormalTok{terra}\SpecialCharTok{::}\FunctionTok{global}\NormalTok{(centrets,}\AttributeTok{fun=}\StringTok{"rms"}\NormalTok{,}\AttributeTok{na.rm=}\ConstantTok{TRUE}\NormalTok{)}
\NormalTok{merogots}\OtherTok{=}\NormalTok{centrets}\SpecialCharTok{/}\NormalTok{standartnovirze[,}\DecValTok{1}\NormalTok{]}
\FunctionTok{writeRaster}\NormalTok{(merogots,}
      \AttributeTok{filename=}\NormalTok{saglabasanas\_cels,}
      \AttributeTok{overwrite=}\ConstantTok{TRUE}\NormalTok{)}
\end{Highlighting}
\end{Shaded}

\section{ForestsTreesAge\_ConiferousYoung\_r3000}\label{ch06.361}

\textbf{filename:} \texttt{ForestsTreesAge\_ConiferousYoung\_r3000.tif}

\textbf{layername:} \texttt{egv\_361}

\textbf{English name:} Fractional cover of Young (pre-rotation age) Coniferous
Forests within the 3 km landscape

\textbf{Latvian name:} Jaunu (pirms cirtmeta) skujkoku mežu platības īpatsvars 3 km
ainavā

\textbf{Procedure:} The cover fraction within a radius of 3000 m around the analysis grid cell
is calculated as the area-weighted sum of the \hyperref[ch06.358]{analysis cells} inside
the buffer, using the workflow \texttt{egvtools::radius\_function()}. During the calculation of the landscape
metric, inverse distance weighted (power = 2) gap filling on the output is
applied to ensure no missing values at the edges. Then the layer is
rewritten to set its name. Finally, the layer is standardised by
subtracting the arithmetic mean and dividing by the root mean squared error.

\begin{Shaded}
\begin{Highlighting}[]
\CommentTok{\# libs {-}{-}{-}{-}}
\ControlFlowTok{if}\NormalTok{(}\SpecialCharTok{!}\FunctionTok{require}\NormalTok{(terra)) \{}\FunctionTok{install.packages}\NormalTok{(}\StringTok{"terra"}\NormalTok{); }\FunctionTok{require}\NormalTok{(terra)\}}
\ControlFlowTok{if}\NormalTok{(}\SpecialCharTok{!}\FunctionTok{require}\NormalTok{(egvtools)) \{remotes}\SpecialCharTok{::}\FunctionTok{install\_github}\NormalTok{(}\StringTok{"aavotins/egvtools"}\NormalTok{); }\FunctionTok{require}\NormalTok{(egvtools)\}}


\CommentTok{\# Templates {-}{-}{-}{-}{-}}
\NormalTok{template100}\OtherTok{=}\FunctionTok{rast}\NormalTok{(}\StringTok{"./Templates/TemplateRasters/LV100m\_10km.tif"}\NormalTok{)}

\CommentTok{\# radii {-}{-}{-}{-}}
\FunctionTok{radius\_function}\NormalTok{(}
 \AttributeTok{kvadrati\_path =} \StringTok{"./Templates/TemplateGrids/tiles/"}\NormalTok{,}
 \AttributeTok{radii\_path   =} \StringTok{"./Templates/TemplateGridPoints/tiles/"}\NormalTok{,}
 \AttributeTok{tikls100\_path =} \StringTok{"./Templates/TemplateGrids/tikls100\_sauzeme.parquet"}\NormalTok{,}
 \AttributeTok{template\_path =} \StringTok{"./Templates/TemplateRasters/LV100m\_10km.tif"}\NormalTok{,}
 \AttributeTok{input\_layers  =} \FunctionTok{c}\NormalTok{(}\StringTok{"./RasterGrids\_100m/2024/RAW/ForestsTreesAge\_ConiferousYoung\_cell.tif"}\NormalTok{),}
 \AttributeTok{layer\_prefixes =} \FunctionTok{c}\NormalTok{(}\StringTok{"ForestsTreesAge\_ConiferousYoung"}\NormalTok{),}
 \AttributeTok{output\_dir   =} \StringTok{"./RasterGrids\_100m/2024/RAW/"}\NormalTok{,}
 \AttributeTok{n\_workers   =} \DecValTok{6}\NormalTok{,}
 \AttributeTok{radii     =} \FunctionTok{c}\NormalTok{(}\StringTok{"r3000"}\NormalTok{),}
 \AttributeTok{radius\_mode  =} \StringTok{"sparse"}\NormalTok{,}
 \AttributeTok{extract\_fun  =} \StringTok{"mean"}\NormalTok{,}
 \AttributeTok{fill\_missing  =} \ConstantTok{TRUE}\NormalTok{,}
 \AttributeTok{IDW\_weight   =} \DecValTok{2}\NormalTok{,}
 \AttributeTok{future\_max\_size =} \DecValTok{40} \SpecialCharTok{*} \DecValTok{1024}\SpecialCharTok{\^{}}\DecValTok{3}\NormalTok{)}


\CommentTok{\# ForestsTreesAge\_ConiferousYoung\_r3000.tif egv\_361}
\NormalTok{slanis}\OtherTok{=}\FunctionTok{rast}\NormalTok{(}\StringTok{"./RasterGrids\_100m/2024/RAW/ForestsTreesAge\_ConiferousYoung\_r3000.tif"}\NormalTok{)}
\FunctionTok{names}\NormalTok{(slanis)}\OtherTok{=}\StringTok{"egv\_361"}
\NormalTok{slanis2}\OtherTok{=}\FunctionTok{project}\NormalTok{(slanis,template100)}
\FunctionTok{writeRaster}\NormalTok{(slanis2,}
      \StringTok{"./RasterGrids\_100m/2024/RAW/ForestsTreesAge\_ConiferousYoung\_r3000.tif"}\NormalTok{,}
      \AttributeTok{overwrite=}\ConstantTok{TRUE}\NormalTok{)}

\CommentTok{\# standardisation {-}{-}{-}{-}}
\ControlFlowTok{if}\NormalTok{(}\SpecialCharTok{!}\FunctionTok{require}\NormalTok{(terra)) \{}\FunctionTok{install.packages}\NormalTok{(}\StringTok{"terra"}\NormalTok{); }\FunctionTok{require}\NormalTok{(terra)\}}
\ControlFlowTok{if}\NormalTok{(}\SpecialCharTok{!}\FunctionTok{require}\NormalTok{(tidyverse)) \{}\FunctionTok{install.packages}\NormalTok{(}\StringTok{"tidyverse"}\NormalTok{); }\FunctionTok{require}\NormalTok{(tidyverse)\}}

\NormalTok{nosaukums}\OtherTok{=}\StringTok{"ForestsTreesAge\_ConiferousYoung\_r3000.tif"}
\NormalTok{ielasisanas\_cels}\OtherTok{=}\FunctionTok{paste0}\NormalTok{(}\StringTok{"./RasterGrids\_100m/2024/RAW/"}\NormalTok{,nosaukums)}
\NormalTok{saglabasanas\_cels}\OtherTok{=}\FunctionTok{paste0}\NormalTok{(}\StringTok{"./RasterGrids\_100m/2024/Scaled/"}\NormalTok{,nosaukums)}
\NormalTok{slanis}\OtherTok{=}\FunctionTok{rast}\NormalTok{(ielasisanas\_cels)}
\NormalTok{videjais}\OtherTok{=}\FunctionTok{global}\NormalTok{(slanis,}\AttributeTok{fun=}\StringTok{"mean"}\NormalTok{,}\AttributeTok{na.rm=}\ConstantTok{TRUE}\NormalTok{)}
\NormalTok{centrets}\OtherTok{=}\NormalTok{slanis}\SpecialCharTok{{-}}\NormalTok{videjais[,}\DecValTok{1}\NormalTok{]}
\NormalTok{standartnovirze}\OtherTok{=}\NormalTok{terra}\SpecialCharTok{::}\FunctionTok{global}\NormalTok{(centrets,}\AttributeTok{fun=}\StringTok{"rms"}\NormalTok{,}\AttributeTok{na.rm=}\ConstantTok{TRUE}\NormalTok{)}
\NormalTok{merogots}\OtherTok{=}\NormalTok{centrets}\SpecialCharTok{/}\NormalTok{standartnovirze[,}\DecValTok{1}\NormalTok{]}
\FunctionTok{writeRaster}\NormalTok{(merogots,}
      \AttributeTok{filename=}\NormalTok{saglabasanas\_cels,}
      \AttributeTok{overwrite=}\ConstantTok{TRUE}\NormalTok{)}
\end{Highlighting}
\end{Shaded}

\section{ForestsTreesAge\_ConiferousYoung\_r10000}\label{ch06.362}

\textbf{filename:} \texttt{ForestsTreesAge\_ConiferousYoung\_r10000.tif}

\textbf{layername:} \texttt{egv\_362}

\textbf{English name:} Fractional cover of Young (pre-rotation age) Coniferous
Forests within the 10 km landscape

\textbf{Latvian name:} Jaunu (pirms cirtmeta) skujkoku mežu platības īpatsvars 10 km
ainavā

\textbf{Procedure:} The cover fraction within a radius of 10000 m around the analysis grid cell
is calculated as the area-weighted sum of the \hyperref[ch06.358]{analysis cells} inside
the buffer, using the workflow \texttt{egvtools::radius\_function()}. During the calculation of the landscape
metric, inverse distance weighted (power = 2) gap filling on the output is
applied to ensure no missing values at the edges. Then the layer is
rewritten to set its name. Finally, the layer is standardised by
subtracting the arithmetic mean and dividing by the root mean squared error.

\begin{Shaded}
\begin{Highlighting}[]
\CommentTok{\# libs {-}{-}{-}{-}}
\ControlFlowTok{if}\NormalTok{(}\SpecialCharTok{!}\FunctionTok{require}\NormalTok{(terra)) \{}\FunctionTok{install.packages}\NormalTok{(}\StringTok{"terra"}\NormalTok{); }\FunctionTok{require}\NormalTok{(terra)\}}
\ControlFlowTok{if}\NormalTok{(}\SpecialCharTok{!}\FunctionTok{require}\NormalTok{(egvtools)) \{remotes}\SpecialCharTok{::}\FunctionTok{install\_github}\NormalTok{(}\StringTok{"aavotins/egvtools"}\NormalTok{); }\FunctionTok{require}\NormalTok{(egvtools)\}}


\CommentTok{\# Templates {-}{-}{-}{-}{-}}
\NormalTok{template100}\OtherTok{=}\FunctionTok{rast}\NormalTok{(}\StringTok{"./Templates/TemplateRasters/LV100m\_10km.tif"}\NormalTok{)}

\CommentTok{\# radii {-}{-}{-}{-}}
\FunctionTok{radius\_function}\NormalTok{(}
 \AttributeTok{kvadrati\_path =} \StringTok{"./Templates/TemplateGrids/tiles/"}\NormalTok{,}
 \AttributeTok{radii\_path   =} \StringTok{"./Templates/TemplateGridPoints/tiles/"}\NormalTok{,}
 \AttributeTok{tikls100\_path =} \StringTok{"./Templates/TemplateGrids/tikls100\_sauzeme.parquet"}\NormalTok{,}
 \AttributeTok{template\_path =} \StringTok{"./Templates/TemplateRasters/LV100m\_10km.tif"}\NormalTok{,}
 \AttributeTok{input\_layers  =} \FunctionTok{c}\NormalTok{(}\StringTok{"./RasterGrids\_100m/2024/RAW/ForestsTreesAge\_ConiferousYoung\_cell.tif"}\NormalTok{),}
 \AttributeTok{layer\_prefixes =} \FunctionTok{c}\NormalTok{(}\StringTok{"ForestsTreesAge\_ConiferousYoung"}\NormalTok{),}
 \AttributeTok{output\_dir   =} \StringTok{"./RasterGrids\_100m/2024/RAW/"}\NormalTok{,}
 \AttributeTok{n\_workers   =} \DecValTok{6}\NormalTok{,}
 \AttributeTok{radii     =} \FunctionTok{c}\NormalTok{(}\StringTok{"r10000"}\NormalTok{),}
 \AttributeTok{radius\_mode  =} \StringTok{"sparse"}\NormalTok{,}
 \AttributeTok{extract\_fun  =} \StringTok{"mean"}\NormalTok{,}
 \AttributeTok{fill\_missing  =} \ConstantTok{TRUE}\NormalTok{,}
 \AttributeTok{IDW\_weight   =} \DecValTok{2}\NormalTok{,}
 \AttributeTok{future\_max\_size =} \DecValTok{40} \SpecialCharTok{*} \DecValTok{1024}\SpecialCharTok{\^{}}\DecValTok{3}\NormalTok{)}


\CommentTok{\# ForestsTreesAge\_ConiferousYoung\_r10000.tif    egv\_362}
\NormalTok{slanis}\OtherTok{=}\FunctionTok{rast}\NormalTok{(}\StringTok{"./RasterGrids\_100m/2024/RAW/ForestsTreesAge\_ConiferousYoung\_r10000.tif"}\NormalTok{)}
\FunctionTok{names}\NormalTok{(slanis)}\OtherTok{=}\StringTok{"egv\_362"}
\NormalTok{slanis2}\OtherTok{=}\FunctionTok{project}\NormalTok{(slanis,template100)}
\FunctionTok{writeRaster}\NormalTok{(slanis2,}
      \StringTok{"./RasterGrids\_100m/2024/RAW/ForestsTreesAge\_ConiferousYoung\_r10000.tif"}\NormalTok{,}
      \AttributeTok{overwrite=}\ConstantTok{TRUE}\NormalTok{)}

\CommentTok{\# standardisation {-}{-}{-}{-}}
\ControlFlowTok{if}\NormalTok{(}\SpecialCharTok{!}\FunctionTok{require}\NormalTok{(terra)) \{}\FunctionTok{install.packages}\NormalTok{(}\StringTok{"terra"}\NormalTok{); }\FunctionTok{require}\NormalTok{(terra)\}}
\ControlFlowTok{if}\NormalTok{(}\SpecialCharTok{!}\FunctionTok{require}\NormalTok{(tidyverse)) \{}\FunctionTok{install.packages}\NormalTok{(}\StringTok{"tidyverse"}\NormalTok{); }\FunctionTok{require}\NormalTok{(tidyverse)\}}

\NormalTok{nosaukums}\OtherTok{=}\StringTok{"ForestsTreesAge\_ConiferousYoung\_r10000.tif"}
\NormalTok{ielasisanas\_cels}\OtherTok{=}\FunctionTok{paste0}\NormalTok{(}\StringTok{"./RasterGrids\_100m/2024/RAW/"}\NormalTok{,nosaukums)}
\NormalTok{saglabasanas\_cels}\OtherTok{=}\FunctionTok{paste0}\NormalTok{(}\StringTok{"./RasterGrids\_100m/2024/Scaled/"}\NormalTok{,nosaukums)}
\NormalTok{slanis}\OtherTok{=}\FunctionTok{rast}\NormalTok{(ielasisanas\_cels)}
\NormalTok{videjais}\OtherTok{=}\FunctionTok{global}\NormalTok{(slanis,}\AttributeTok{fun=}\StringTok{"mean"}\NormalTok{,}\AttributeTok{na.rm=}\ConstantTok{TRUE}\NormalTok{)}
\NormalTok{centrets}\OtherTok{=}\NormalTok{slanis}\SpecialCharTok{{-}}\NormalTok{videjais[,}\DecValTok{1}\NormalTok{]}
\NormalTok{standartnovirze}\OtherTok{=}\NormalTok{terra}\SpecialCharTok{::}\FunctionTok{global}\NormalTok{(centrets,}\AttributeTok{fun=}\StringTok{"rms"}\NormalTok{,}\AttributeTok{na.rm=}\ConstantTok{TRUE}\NormalTok{)}
\NormalTok{merogots}\OtherTok{=}\NormalTok{centrets}\SpecialCharTok{/}\NormalTok{standartnovirze[,}\DecValTok{1}\NormalTok{]}
\FunctionTok{writeRaster}\NormalTok{(merogots,}
      \AttributeTok{filename=}\NormalTok{saglabasanas\_cels,}
      \AttributeTok{overwrite=}\ConstantTok{TRUE}\NormalTok{)}
\end{Highlighting}
\end{Shaded}

\section{ForestsTreesAge\_MixedOld\_cell}\label{ch06.363}

\textbf{filename:} \texttt{ForestsTreesAge\_MixedOld\_cell.tif}

\textbf{layername:} \texttt{egv\_363}

\textbf{English name:} Fractional cover of Old (over rotation age) Mixed Forests
within the analysis cell (1 ha)

\textbf{Latvian name:} Vecu (kopš cirtmeta) jauktu koku mežu platības īpatsvars
analīzes šūnā (1 ha)

\textbf{Procedure:} Most EGVs describing forests are spatially restricted to areas outside
of clearcuts and dead stands. This mask is created using a combination of
the \hyperref[Ch04.01]{State Forest Service's
State Forest Registry} land category 12 and 14, and \hyperref[Ch04.09]{The
Global Forest Watch} pixels classified as lost tree canopy cover since
2020 (raster layer matching input, presence = 1, absence = 0).

To prepare this EGV, stands from the \hyperref[Ch04.01]{State Forest Service's State Forest
Registry} are classified into (in order):

\begin{itemize}
\item
  coniferous (see \hyperref[Ch01]{Terminology and acronyms} for species codes) if
  timber volume of those species exceeded 75\%;
\item
  Boreal deciduous if timber volume of those species exceeded 75\%;
\item
  temperate deciduous if timber volume of those species exceeded 50\%;
\item
  mixed otherwise;
\end{itemize}

then mixed stands exceeding the legal rotation age are selected and
geometries are rasterised (presence = 1, NA otherwise). Rasterisation is
performed using the workflow \texttt{egvtools::polygon2input()}, restricting to pixels outside clearcut
mask and covering background with value 0. The resulting layer
is then aggregated to EGV resolution using the workflow \texttt{egvtools::input2egv()}, which
calculates the arithmetic mean to determine the cover fraction. During
aggregation, inverse distance weighted (power = 2) gap filling on the output is
applied to ensure no missing values at the edges. Finally, the layer is
standardised by subtracting the arithmetic mean and dividing by the root mean squared
error.

\begin{Shaded}
\begin{Highlighting}[]
\CommentTok{\# libs {-}{-}{-}{-}}
\ControlFlowTok{if}\NormalTok{(}\SpecialCharTok{!}\FunctionTok{require}\NormalTok{(egvtools)) \{remotes}\SpecialCharTok{::}\FunctionTok{install\_github}\NormalTok{(}\StringTok{"aavotins/egvtools"}\NormalTok{); }\FunctionTok{require}\NormalTok{(egvtools)\}}
\ControlFlowTok{if}\NormalTok{(}\SpecialCharTok{!}\FunctionTok{require}\NormalTok{(terra)) \{}\FunctionTok{install.packages}\NormalTok{(}\StringTok{"terra"}\NormalTok{); }\FunctionTok{require}\NormalTok{(terra)\}}
\ControlFlowTok{if}\NormalTok{(}\SpecialCharTok{!}\FunctionTok{require}\NormalTok{(sf)) \{}\FunctionTok{install.packages}\NormalTok{(}\StringTok{"sf"}\NormalTok{); }\FunctionTok{require}\NormalTok{(sf)\}}
\ControlFlowTok{if}\NormalTok{(}\SpecialCharTok{!}\FunctionTok{require}\NormalTok{(tidyverse)) \{}\FunctionTok{install.packages}\NormalTok{(}\StringTok{"tidyverse"}\NormalTok{); }\FunctionTok{require}\NormalTok{(tidyverse)\}}
\ControlFlowTok{if}\NormalTok{(}\SpecialCharTok{!}\FunctionTok{require}\NormalTok{(sfarrow)) \{}\FunctionTok{install.packages}\NormalTok{(}\StringTok{"sfarrow"}\NormalTok{); }\FunctionTok{require}\NormalTok{(sfarrow)\}}
\ControlFlowTok{if}\NormalTok{(}\SpecialCharTok{!}\FunctionTok{require}\NormalTok{(readxl)) \{}\FunctionTok{install.packages}\NormalTok{(}\StringTok{"readxl"}\NormalTok{); }\FunctionTok{require}\NormalTok{(readxl)\}}
\ControlFlowTok{if}\NormalTok{(}\SpecialCharTok{!}\FunctionTok{require}\NormalTok{(raster)) \{}\FunctionTok{install.packages}\NormalTok{(}\StringTok{"raster"}\NormalTok{); }\FunctionTok{require}\NormalTok{(raster)\}}
\ControlFlowTok{if}\NormalTok{(}\SpecialCharTok{!}\FunctionTok{require}\NormalTok{(fasterize)) \{}\FunctionTok{install.packages}\NormalTok{(}\StringTok{"fasterize"}\NormalTok{); }\FunctionTok{require}\NormalTok{(fasterize)\}}

\CommentTok{\# templates {-}{-}{-}{-}}
\NormalTok{template100}\OtherTok{=}\FunctionTok{rast}\NormalTok{(}\StringTok{"./Templates/TemplateRasters/LV100m\_10km.tif"}\NormalTok{)}
\NormalTok{template10}\OtherTok{=}\FunctionTok{rast}\NormalTok{(}\StringTok{"./Templates/TemplateRasters/LV10m\_10km.tif"}\NormalTok{)}
\NormalTok{rastrs10}\OtherTok{=}\FunctionTok{raster}\NormalTok{(template10)}

\NormalTok{nulls10}\OtherTok{=}\FunctionTok{rast}\NormalTok{(}\StringTok{"./Templates/TemplateRasters/nulls\_LV10m\_10km.tif"}\NormalTok{)}
\NormalTok{nulls100}\OtherTok{=}\FunctionTok{rast}\NormalTok{(}\StringTok{"./Templates/TemplateRasters/nulls\_LV100m\_10km.tif"}\NormalTok{)}


\CommentTok{\# simple landscape {-}{-}{-}{-}}
\NormalTok{simple\_landscape}\OtherTok{=}\FunctionTok{rast}\NormalTok{(}\StringTok{"RasterGrids\_10m/2024/Ainava\_vienk\_mask.tif"}\NormalTok{)}

\CommentTok{\# mvr {-}{-}{-}{-}}
\NormalTok{mvr}\OtherTok{=}\FunctionTok{st\_read\_parquet}\NormalTok{(}\StringTok{"./Geodata/2024/MVR/nogabali\_2024janv.parquet"}\NormalTok{)}
\NormalTok{mvr}\SpecialCharTok{$}\NormalTok{yes}\OtherTok{=}\DecValTok{1}

\CommentTok{\# clear cut mask {-}{-}{-}{-}}
\NormalTok{izcirtumi}\OtherTok{=}\NormalTok{mvr }\SpecialCharTok{\%\textgreater{}\%} 
 \FunctionTok{filter}\NormalTok{(zkat }\SpecialCharTok{\%in\%} \FunctionTok{c}\NormalTok{(}\StringTok{"12"}\NormalTok{,}\StringTok{"14"}\NormalTok{)) }\SpecialCharTok{\%\textgreater{}\%} 
\NormalTok{ dplyr}\SpecialCharTok{::}\FunctionTok{select}\NormalTok{(yes)}
\NormalTok{r\_izcirtumi\_mvr}\OtherTok{=}\FunctionTok{fasterize}\NormalTok{(izcirtumi,rastrs10,}\AttributeTok{field=}\StringTok{"yes"}\NormalTok{)}
\NormalTok{t\_izcirtumi\_mvr}\OtherTok{=}\FunctionTok{rast}\NormalTok{(r\_izcirtumi\_mvr)}
\FunctionTok{plot}\NormalTok{(t\_izcirtumi\_mvr)}

\NormalTok{tcl}\OtherTok{=}\FunctionTok{rast}\NormalTok{(}\StringTok{"./Geodata/2024/Trees/GFW/TreeCoverLoss\_v1\_12.tif"}\NormalTok{)}
\NormalTok{tcl2}\OtherTok{=}\FunctionTok{ifel}\NormalTok{(tcl}\SpecialCharTok{\textless{}}\DecValTok{20}\NormalTok{,}\DecValTok{0}\NormalTok{,}\DecValTok{1}\NormalTok{)}
\NormalTok{tclX}\OtherTok{=}\FunctionTok{cover}\NormalTok{(tcl2,nulls10)}
\FunctionTok{plot}\NormalTok{(tclX)}

\NormalTok{clearcut\_mask}\OtherTok{=}\FunctionTok{cover}\NormalTok{(t\_izcirtumi\_mvr,tclX,}
          \AttributeTok{filename=}\StringTok{"./RasterGrids\_10m/2024/Mask\_clearcuts.tif"}\NormalTok{,}
          \AttributeTok{overwrite=}\ConstantTok{TRUE}\NormalTok{)}
\FunctionTok{plot}\NormalTok{(clearcut\_mask)}

\FunctionTok{rm}\NormalTok{(izcirtumi)}
\FunctionTok{rm}\NormalTok{(r\_izcirtumi\_mvr)}
\FunctionTok{rm}\NormalTok{(t\_izcirtumi\_mvr)}
\FunctionTok{rm}\NormalTok{(tcl)}
\FunctionTok{rm}\NormalTok{(tcl2)}
\FunctionTok{rm}\NormalTok{(tclX)}

\CommentTok{\# ForestsTreesAge\_MixedOld\_cell.tif egv\_363 {-}{-}{-}{-}}
\NormalTok{skujkoki}\OtherTok{=}\FunctionTok{c}\NormalTok{(}\StringTok{"1"}\NormalTok{,}\StringTok{"3"}\NormalTok{,}\StringTok{"13"}\NormalTok{,}\StringTok{"14"}\NormalTok{,}\StringTok{"15"}\NormalTok{,}\StringTok{"22"}\NormalTok{,}\StringTok{"23"}\NormalTok{,}\StringTok{"28"}\NormalTok{) }\CommentTok{\# 8}
\NormalTok{saurlapji}\OtherTok{=}\FunctionTok{c}\NormalTok{(}\StringTok{"4"}\NormalTok{,}\StringTok{"6"}\NormalTok{,}\StringTok{"8"}\NormalTok{,}\StringTok{"9"}\NormalTok{,}\StringTok{"19"}\NormalTok{,}\StringTok{"20"}\NormalTok{,}\StringTok{"21"}\NormalTok{,}\StringTok{"32"}\NormalTok{,}\StringTok{"35"}\NormalTok{,}\StringTok{"68"}\NormalTok{) }\CommentTok{\# 10}
\NormalTok{platlapji}\OtherTok{=}\FunctionTok{c}\NormalTok{(}\StringTok{"10"}\NormalTok{,}\StringTok{"11"}\NormalTok{,}\StringTok{"12"}\NormalTok{,}\StringTok{"16"}\NormalTok{,}\StringTok{"17"}\NormalTok{,}\StringTok{"18"}\NormalTok{,}\StringTok{"24"}\NormalTok{,}\StringTok{"25"}\NormalTok{,}\StringTok{"26"}\NormalTok{,}\StringTok{"27"}\NormalTok{,}\StringTok{"28"}\NormalTok{,}\StringTok{"29"}\NormalTok{,}\StringTok{"50"}\NormalTok{,}
      \StringTok{"61"}\NormalTok{,}\StringTok{"62"}\NormalTok{,}\StringTok{"63"}\NormalTok{,}\StringTok{"64"}\NormalTok{,}\StringTok{"65"}\NormalTok{,}\StringTok{"66"}\NormalTok{,}\StringTok{"67"}\NormalTok{,}\StringTok{"69"}\NormalTok{) }\CommentTok{\# 21}
\NormalTok{mvr}\OtherTok{=}\NormalTok{mvr }\SpecialCharTok{\%\textgreater{}\%} 
 \FunctionTok{mutate}\NormalTok{(}\AttributeTok{kraja\_skujkoku=}\FunctionTok{ifelse}\NormalTok{(s10 }\SpecialCharTok{\%in\%}\NormalTok{ skujkoki,v10,}\DecValTok{0}\NormalTok{)}\SpecialCharTok{+}
      \FunctionTok{ifelse}\NormalTok{(s11 }\SpecialCharTok{\%in\%}\NormalTok{ skujkoki,v11,}\DecValTok{0}\NormalTok{)}\SpecialCharTok{+}\FunctionTok{ifelse}\NormalTok{(s12 }\SpecialCharTok{\%in\%}\NormalTok{ skujkoki,v12,}\DecValTok{0}\NormalTok{)}\SpecialCharTok{+}
      \FunctionTok{ifelse}\NormalTok{(s13 }\SpecialCharTok{\%in\%}\NormalTok{ skujkoki,v13,}\DecValTok{0}\NormalTok{)}\SpecialCharTok{+}\FunctionTok{ifelse}\NormalTok{(s14 }\SpecialCharTok{\%in\%}\NormalTok{ skujkoki,v14,}\DecValTok{0}\NormalTok{),}
     \AttributeTok{kraja\_saurlapju=}\FunctionTok{ifelse}\NormalTok{(s10 }\SpecialCharTok{\%in\%}\NormalTok{ saurlapji,v10,}\DecValTok{0}\NormalTok{)}\SpecialCharTok{+}
      \FunctionTok{ifelse}\NormalTok{(s11 }\SpecialCharTok{\%in\%}\NormalTok{ saurlapji,v11,}\DecValTok{0}\NormalTok{)}\SpecialCharTok{+}\FunctionTok{ifelse}\NormalTok{(s12 }\SpecialCharTok{\%in\%}\NormalTok{ saurlapji,v12,}\DecValTok{0}\NormalTok{)}\SpecialCharTok{+}
      \FunctionTok{ifelse}\NormalTok{(s13 }\SpecialCharTok{\%in\%}\NormalTok{ saurlapji,v13,}\DecValTok{0}\NormalTok{)}\SpecialCharTok{+}\FunctionTok{ifelse}\NormalTok{(s14 }\SpecialCharTok{\%in\%}\NormalTok{ saurlapji,v14,}\DecValTok{0}\NormalTok{),}
     \AttributeTok{kraja\_platlapju=}\FunctionTok{ifelse}\NormalTok{(s10 }\SpecialCharTok{\%in\%}\NormalTok{ platlapji,v10,}\DecValTok{0}\NormalTok{)}\SpecialCharTok{+}
      \FunctionTok{ifelse}\NormalTok{(s11 }\SpecialCharTok{\%in\%}\NormalTok{ platlapji,v11,}\DecValTok{0}\NormalTok{)}\SpecialCharTok{+}\FunctionTok{ifelse}\NormalTok{(s12 }\SpecialCharTok{\%in\%}\NormalTok{ platlapji,v12,}\DecValTok{0}\NormalTok{)}\SpecialCharTok{+}
      \FunctionTok{ifelse}\NormalTok{(s13 }\SpecialCharTok{\%in\%}\NormalTok{ platlapji,v13,}\DecValTok{0}\NormalTok{)}\SpecialCharTok{+}\FunctionTok{ifelse}\NormalTok{(s14 }\SpecialCharTok{\%in\%}\NormalTok{ platlapji,v14,}\DecValTok{0}\NormalTok{)) }\SpecialCharTok{\%\textgreater{}\%} 
 \FunctionTok{mutate}\NormalTok{(}\AttributeTok{kopeja\_kraja=}\NormalTok{kraja\_skujkoku}\SpecialCharTok{+}\NormalTok{kraja\_platlapju}\SpecialCharTok{+}\NormalTok{kraja\_saurlapju) }\SpecialCharTok{\%\textgreater{}\%} 
 \FunctionTok{mutate}\NormalTok{(}\AttributeTok{tips=}\FunctionTok{ifelse}\NormalTok{(kraja\_skujkoku}\SpecialCharTok{/}\NormalTok{kopeja\_kraja}\SpecialCharTok{\textgreater{}=}\FloatTok{0.75}\NormalTok{,}\StringTok{"skujkoku"}\NormalTok{,}
           \FunctionTok{ifelse}\NormalTok{(kraja\_saurlapju}\SpecialCharTok{/}\NormalTok{kopeja\_kraja}\SpecialCharTok{\textgreater{}=}\FloatTok{0.75}\NormalTok{,}\StringTok{"saurlapju"}\NormalTok{,}
              \FunctionTok{ifelse}\NormalTok{(kraja\_platlapju}\SpecialCharTok{/}\NormalTok{kopeja\_kraja}\SpecialCharTok{\textgreater{}}\FloatTok{0.5}\NormalTok{,}\StringTok{"platlapju"}\NormalTok{,}
                  \StringTok{"jauktu koku"}\NormalTok{))))}
\NormalTok{nogabali}\OtherTok{=}\NormalTok{mvr }\SpecialCharTok{\%\textgreater{}\%} 
 \FunctionTok{filter}\NormalTok{(zkat}\SpecialCharTok{==}\StringTok{"10"}\SpecialCharTok{\&}\NormalTok{tips}\SpecialCharTok{==}\StringTok{"jauktu koku"}\SpecialCharTok{\&}\NormalTok{(vgr}\SpecialCharTok{==}\StringTok{"4"}\SpecialCharTok{|}\NormalTok{vgr}\SpecialCharTok{==}\StringTok{"5"}\NormalTok{))}

\NormalTok{p2i\_rez}\OtherTok{=}\NormalTok{egvtools}\SpecialCharTok{::}\FunctionTok{polygon2input}\NormalTok{(}\AttributeTok{vector\_data =}\NormalTok{ nogabali,}
                \AttributeTok{template\_path =} \StringTok{"./Templates/TemplateRasters/LV10m\_10km.tif"}\NormalTok{,}
                \AttributeTok{out\_path =} \StringTok{"./RasterGrids\_10m/2024/"}\NormalTok{,}
                \AttributeTok{file\_name =} \StringTok{"ForestsTreesAge\_MixedOld\_input.tif"}\NormalTok{,}
                \AttributeTok{value\_field =} \StringTok{"yes"}\NormalTok{,}
                \AttributeTok{restrict\_to =}\NormalTok{ clearcut\_mask,}
                \AttributeTok{restrict\_values =} \DecValTok{0}\NormalTok{,}
                \AttributeTok{prepare=}\ConstantTok{FALSE}\NormalTok{,}
                \AttributeTok{background\_raster =} \StringTok{"./Templates/TemplateRasters/nulls\_LV10m\_10km.tif"}\NormalTok{,}
                \AttributeTok{plot\_result =} \ConstantTok{TRUE}\NormalTok{)}
\NormalTok{p2i\_rez}
\NormalTok{i2e\_rez}\OtherTok{=}\NormalTok{egvtools}\SpecialCharTok{::}\FunctionTok{input2egv}\NormalTok{(}\AttributeTok{input=}\FunctionTok{paste0}\NormalTok{(}\StringTok{"./RasterGrids\_10m/2024/"}\NormalTok{,}
                     \StringTok{"ForestsTreesAge\_MixedOld\_input.tif"}\NormalTok{),}
              \AttributeTok{egv\_template=} \StringTok{"./Templates/TemplateRasters/LV100m\_10km.tif"}\NormalTok{,}
              \AttributeTok{summary\_function =} \StringTok{"average"}\NormalTok{,}
              \AttributeTok{missing\_job =} \StringTok{"FillOutput"}\NormalTok{,}
              \AttributeTok{outlocation =} \StringTok{"./RasterGrids\_100m/2024/RAW/"}\NormalTok{,}
              \AttributeTok{outfilename =} \StringTok{"ForestsTreesAge\_MixedOld\_cell.tif"}\NormalTok{,}
              \AttributeTok{layername =} \StringTok{"egv\_363"}\NormalTok{,}
              \AttributeTok{idw\_weight =} \DecValTok{2}\NormalTok{,}
              \AttributeTok{plot\_gaps =} \ConstantTok{FALSE}\NormalTok{,}\AttributeTok{plot\_final =} \ConstantTok{TRUE}\NormalTok{)}
\NormalTok{i2e\_rez}
\FunctionTok{rm}\NormalTok{(nogabali)}
\FunctionTok{rm}\NormalTok{(p2i\_rez)}
\FunctionTok{rm}\NormalTok{(i2e\_rez)}
\FunctionTok{unlink}\NormalTok{(}\StringTok{"./RasterGrids\_10m/2024/ForestsTreesAge\_MixedOld\_input.tif"}\NormalTok{)}

\CommentTok{\# standardisation {-}{-}{-}{-}}
\ControlFlowTok{if}\NormalTok{(}\SpecialCharTok{!}\FunctionTok{require}\NormalTok{(terra)) \{}\FunctionTok{install.packages}\NormalTok{(}\StringTok{"terra"}\NormalTok{); }\FunctionTok{require}\NormalTok{(terra)\}}
\ControlFlowTok{if}\NormalTok{(}\SpecialCharTok{!}\FunctionTok{require}\NormalTok{(tidyverse)) \{}\FunctionTok{install.packages}\NormalTok{(}\StringTok{"tidyverse"}\NormalTok{); }\FunctionTok{require}\NormalTok{(tidyverse)\}}

\NormalTok{nosaukums}\OtherTok{=}\StringTok{"ForestsTreesAge\_MixedOld\_cell.tif"}
\NormalTok{ielasisanas\_cels}\OtherTok{=}\FunctionTok{paste0}\NormalTok{(}\StringTok{"./RasterGrids\_100m/2024/RAW/"}\NormalTok{,nosaukums)}
\NormalTok{saglabasanas\_cels}\OtherTok{=}\FunctionTok{paste0}\NormalTok{(}\StringTok{"./RasterGrids\_100m/2024/Scaled/"}\NormalTok{,nosaukums)}
\NormalTok{slanis}\OtherTok{=}\FunctionTok{rast}\NormalTok{(ielasisanas\_cels)}
\NormalTok{videjais}\OtherTok{=}\FunctionTok{global}\NormalTok{(slanis,}\AttributeTok{fun=}\StringTok{"mean"}\NormalTok{,}\AttributeTok{na.rm=}\ConstantTok{TRUE}\NormalTok{)}
\NormalTok{centrets}\OtherTok{=}\NormalTok{slanis}\SpecialCharTok{{-}}\NormalTok{videjais[,}\DecValTok{1}\NormalTok{]}
\NormalTok{standartnovirze}\OtherTok{=}\NormalTok{terra}\SpecialCharTok{::}\FunctionTok{global}\NormalTok{(centrets,}\AttributeTok{fun=}\StringTok{"rms"}\NormalTok{,}\AttributeTok{na.rm=}\ConstantTok{TRUE}\NormalTok{)}
\NormalTok{merogots}\OtherTok{=}\NormalTok{centrets}\SpecialCharTok{/}\NormalTok{standartnovirze[,}\DecValTok{1}\NormalTok{]}
\FunctionTok{writeRaster}\NormalTok{(merogots,}
      \AttributeTok{filename=}\NormalTok{saglabasanas\_cels,}
      \AttributeTok{overwrite=}\ConstantTok{TRUE}\NormalTok{)}
\end{Highlighting}
\end{Shaded}

\section{ForestsTreesAge\_MixedOld\_r500}\label{ch06.364}

\textbf{filename:} \texttt{ForestsTreesAge\_MixedOld\_r500.tif}

\textbf{layername:} \texttt{egv\_364}

\textbf{English name:} Fractional cover of Old (over rotation age) Mixed Forests
within the 0.5 km landscape

\textbf{Latvian name:} Vecu (kopš cirtmeta) jauktu koku mežu platības īpatsvars 0,5
km ainavā

\textbf{Procedure:} The cover fraction within a radius of 500 m around the analysis grid cell is
calculated as the area-weighted sum of the \hyperref[ch06.363]{analysis cells} inside the
buffer, using the workflow \texttt{egvtools::radius\_function()}. During the calculation of the landscape metric,
inverse distance weighted (power = 2) gap filling on the output is applied
to ensure no missing values at the edges. Then the layer is rewritten to set
its name. Finally, the layer is standardised by subtracting the arithmetic
mean and dividing by the root mean squared error.

\begin{Shaded}
\begin{Highlighting}[]
\CommentTok{\# libs {-}{-}{-}{-}}
\ControlFlowTok{if}\NormalTok{(}\SpecialCharTok{!}\FunctionTok{require}\NormalTok{(terra)) \{}\FunctionTok{install.packages}\NormalTok{(}\StringTok{"terra"}\NormalTok{); }\FunctionTok{require}\NormalTok{(terra)\}}
\ControlFlowTok{if}\NormalTok{(}\SpecialCharTok{!}\FunctionTok{require}\NormalTok{(egvtools)) \{remotes}\SpecialCharTok{::}\FunctionTok{install\_github}\NormalTok{(}\StringTok{"aavotins/egvtools"}\NormalTok{); }\FunctionTok{require}\NormalTok{(egvtools)\}}


\CommentTok{\# Templates {-}{-}{-}{-}{-}}
\NormalTok{template100}\OtherTok{=}\FunctionTok{rast}\NormalTok{(}\StringTok{"./Templates/TemplateRasters/LV100m\_10km.tif"}\NormalTok{)}

\CommentTok{\# radii {-}{-}{-}{-}}
\FunctionTok{radius\_function}\NormalTok{(}
 \AttributeTok{kvadrati\_path =} \StringTok{"./Templates/TemplateGrids/tiles/"}\NormalTok{,}
 \AttributeTok{radii\_path   =} \StringTok{"./Templates/TemplateGridPoints/tiles/"}\NormalTok{,}
 \AttributeTok{tikls100\_path =} \StringTok{"./Templates/TemplateGrids/tikls100\_sauzeme.parquet"}\NormalTok{,}
 \AttributeTok{template\_path =} \StringTok{"./Templates/TemplateRasters/LV100m\_10km.tif"}\NormalTok{,}
 \AttributeTok{input\_layers  =} \FunctionTok{c}\NormalTok{(}\StringTok{"./RasterGrids\_100m/2024/RAW/ForestsTreesAge\_MixedOld\_cell.tif"}\NormalTok{),}
 \AttributeTok{layer\_prefixes =} \FunctionTok{c}\NormalTok{(}\StringTok{"ForestsTreesAge\_MixedOld"}\NormalTok{),}
 \AttributeTok{output\_dir   =} \StringTok{"./RasterGrids\_100m/2024/RAW/"}\NormalTok{,}
 \AttributeTok{n\_workers   =} \DecValTok{6}\NormalTok{,}
 \AttributeTok{radii     =} \FunctionTok{c}\NormalTok{(}\StringTok{"r500"}\NormalTok{),}
 \AttributeTok{radius\_mode  =} \StringTok{"sparse"}\NormalTok{,}
 \AttributeTok{extract\_fun  =} \StringTok{"mean"}\NormalTok{,}
 \AttributeTok{fill\_missing  =} \ConstantTok{TRUE}\NormalTok{,}
 \AttributeTok{IDW\_weight   =} \DecValTok{2}\NormalTok{,}
 \AttributeTok{future\_max\_size =} \DecValTok{40} \SpecialCharTok{*} \DecValTok{1024}\SpecialCharTok{\^{}}\DecValTok{3}\NormalTok{)}


\CommentTok{\# ForestsTreesAge\_MixedOld\_r500.tif egv\_364}
\NormalTok{slanis}\OtherTok{=}\FunctionTok{rast}\NormalTok{(}\StringTok{"./RasterGrids\_100m/2024/RAW/ForestsTreesAge\_MixedOld\_r500.tif"}\NormalTok{)}
\FunctionTok{names}\NormalTok{(slanis)}\OtherTok{=}\StringTok{"egv\_364"}
\NormalTok{slanis2}\OtherTok{=}\FunctionTok{project}\NormalTok{(slanis,template100)}
\FunctionTok{writeRaster}\NormalTok{(slanis2,}
      \StringTok{"./RasterGrids\_100m/2024/RAW/ForestsTreesAge\_MixedOld\_r500.tif"}\NormalTok{,}
      \AttributeTok{overwrite=}\ConstantTok{TRUE}\NormalTok{)}

\CommentTok{\# standardisation {-}{-}{-}{-}}
\ControlFlowTok{if}\NormalTok{(}\SpecialCharTok{!}\FunctionTok{require}\NormalTok{(terra)) \{}\FunctionTok{install.packages}\NormalTok{(}\StringTok{"terra"}\NormalTok{); }\FunctionTok{require}\NormalTok{(terra)\}}
\ControlFlowTok{if}\NormalTok{(}\SpecialCharTok{!}\FunctionTok{require}\NormalTok{(tidyverse)) \{}\FunctionTok{install.packages}\NormalTok{(}\StringTok{"tidyverse"}\NormalTok{); }\FunctionTok{require}\NormalTok{(tidyverse)\}}

\NormalTok{nosaukums}\OtherTok{=}\StringTok{"ForestsTreesAge\_MixedOld\_r500.tif"}
\NormalTok{ielasisanas\_cels}\OtherTok{=}\FunctionTok{paste0}\NormalTok{(}\StringTok{"./RasterGrids\_100m/2024/RAW/"}\NormalTok{,nosaukums)}
\NormalTok{saglabasanas\_cels}\OtherTok{=}\FunctionTok{paste0}\NormalTok{(}\StringTok{"./RasterGrids\_100m/2024/Scaled/"}\NormalTok{,nosaukums)}
\NormalTok{slanis}\OtherTok{=}\FunctionTok{rast}\NormalTok{(ielasisanas\_cels)}
\NormalTok{videjais}\OtherTok{=}\FunctionTok{global}\NormalTok{(slanis,}\AttributeTok{fun=}\StringTok{"mean"}\NormalTok{,}\AttributeTok{na.rm=}\ConstantTok{TRUE}\NormalTok{)}
\NormalTok{centrets}\OtherTok{=}\NormalTok{slanis}\SpecialCharTok{{-}}\NormalTok{videjais[,}\DecValTok{1}\NormalTok{]}
\NormalTok{standartnovirze}\OtherTok{=}\NormalTok{terra}\SpecialCharTok{::}\FunctionTok{global}\NormalTok{(centrets,}\AttributeTok{fun=}\StringTok{"rms"}\NormalTok{,}\AttributeTok{na.rm=}\ConstantTok{TRUE}\NormalTok{)}
\NormalTok{merogots}\OtherTok{=}\NormalTok{centrets}\SpecialCharTok{/}\NormalTok{standartnovirze[,}\DecValTok{1}\NormalTok{]}
\FunctionTok{writeRaster}\NormalTok{(merogots,}
      \AttributeTok{filename=}\NormalTok{saglabasanas\_cels,}
      \AttributeTok{overwrite=}\ConstantTok{TRUE}\NormalTok{)}
\end{Highlighting}
\end{Shaded}

\section{ForestsTreesAge\_MixedOld\_r1250}\label{ch06.365}

\textbf{filename:} \texttt{ForestsTreesAge\_MixedOld\_r1250.tif}

\textbf{layername:} \texttt{egv\_365}

\textbf{English name:} Fractional cover of Old (over rotation age) Mixed Forests
within the 1.25 km landscape

\textbf{Latvian name:} Vecu (kopš cirtmeta) jauktu koku mežu platības īpatsvars 1,25
km ainavā

\textbf{Procedure:} The cover fraction within a radius of 1250 m around the analysis grid cell
is calculated as the area-weighted sum of the \hyperref[ch06.363]{analysis cells} inside
the buffer, using the workflow \texttt{egvtools::radius\_function()}. During the calculation of the landscape
metric, inverse distance weighted (power = 2) gap filling on the output is
applied to ensure no missing values at the edges. Then the layer is
rewritten to set its name. Finally, the layer is standardised by
subtracting the arithmetic mean and dividing by the root mean squared error.

\begin{Shaded}
\begin{Highlighting}[]
\CommentTok{\# libs {-}{-}{-}{-}}
\ControlFlowTok{if}\NormalTok{(}\SpecialCharTok{!}\FunctionTok{require}\NormalTok{(terra)) \{}\FunctionTok{install.packages}\NormalTok{(}\StringTok{"terra"}\NormalTok{); }\FunctionTok{require}\NormalTok{(terra)\}}
\ControlFlowTok{if}\NormalTok{(}\SpecialCharTok{!}\FunctionTok{require}\NormalTok{(egvtools)) \{remotes}\SpecialCharTok{::}\FunctionTok{install\_github}\NormalTok{(}\StringTok{"aavotins/egvtools"}\NormalTok{); }\FunctionTok{require}\NormalTok{(egvtools)\}}


\CommentTok{\# Templates {-}{-}{-}{-}{-}}
\NormalTok{template100}\OtherTok{=}\FunctionTok{rast}\NormalTok{(}\StringTok{"./Templates/TemplateRasters/LV100m\_10km.tif"}\NormalTok{)}

\CommentTok{\# radii {-}{-}{-}{-}}
\FunctionTok{radius\_function}\NormalTok{(}
 \AttributeTok{kvadrati\_path =} \StringTok{"./Templates/TemplateGrids/tiles/"}\NormalTok{,}
 \AttributeTok{radii\_path   =} \StringTok{"./Templates/TemplateGridPoints/tiles/"}\NormalTok{,}
 \AttributeTok{tikls100\_path =} \StringTok{"./Templates/TemplateGrids/tikls100\_sauzeme.parquet"}\NormalTok{,}
 \AttributeTok{template\_path =} \StringTok{"./Templates/TemplateRasters/LV100m\_10km.tif"}\NormalTok{,}
 \AttributeTok{input\_layers  =} \FunctionTok{c}\NormalTok{(}\StringTok{"./RasterGrids\_100m/2024/RAW/ForestsTreesAge\_MixedOld\_cell.tif"}\NormalTok{),}
 \AttributeTok{layer\_prefixes =} \FunctionTok{c}\NormalTok{(}\StringTok{"ForestsTreesAge\_MixedOld"}\NormalTok{),}
 \AttributeTok{output\_dir   =} \StringTok{"./RasterGrids\_100m/2024/RAW/"}\NormalTok{,}
 \AttributeTok{n\_workers   =} \DecValTok{6}\NormalTok{,}
 \AttributeTok{radii     =} \FunctionTok{c}\NormalTok{(}\StringTok{"r1250"}\NormalTok{),}
 \AttributeTok{radius\_mode  =} \StringTok{"sparse"}\NormalTok{,}
 \AttributeTok{extract\_fun  =} \StringTok{"mean"}\NormalTok{,}
 \AttributeTok{fill\_missing  =} \ConstantTok{TRUE}\NormalTok{,}
 \AttributeTok{IDW\_weight   =} \DecValTok{2}\NormalTok{,}
 \AttributeTok{future\_max\_size =} \DecValTok{40} \SpecialCharTok{*} \DecValTok{1024}\SpecialCharTok{\^{}}\DecValTok{3}\NormalTok{)}


\CommentTok{\# ForestsTreesAge\_MixedOld\_r1250.tif    egv\_365}
\NormalTok{slanis}\OtherTok{=}\FunctionTok{rast}\NormalTok{(}\StringTok{"./RasterGrids\_100m/2024/RAW/ForestsTreesAge\_MixedOld\_r1250.tif"}\NormalTok{)}
\FunctionTok{names}\NormalTok{(slanis)}\OtherTok{=}\StringTok{"egv\_365"}
\NormalTok{slanis2}\OtherTok{=}\FunctionTok{project}\NormalTok{(slanis,template100)}
\FunctionTok{writeRaster}\NormalTok{(slanis2,}
      \StringTok{"./RasterGrids\_100m/2024/RAW/ForestsTreesAge\_MixedOld\_r1250.tif"}\NormalTok{,}
      \AttributeTok{overwrite=}\ConstantTok{TRUE}\NormalTok{)}

\CommentTok{\# standardisation {-}{-}{-}{-}}
\ControlFlowTok{if}\NormalTok{(}\SpecialCharTok{!}\FunctionTok{require}\NormalTok{(terra)) \{}\FunctionTok{install.packages}\NormalTok{(}\StringTok{"terra"}\NormalTok{); }\FunctionTok{require}\NormalTok{(terra)\}}
\ControlFlowTok{if}\NormalTok{(}\SpecialCharTok{!}\FunctionTok{require}\NormalTok{(tidyverse)) \{}\FunctionTok{install.packages}\NormalTok{(}\StringTok{"tidyverse"}\NormalTok{); }\FunctionTok{require}\NormalTok{(tidyverse)\}}

\NormalTok{nosaukums}\OtherTok{=}\StringTok{"ForestsTreesAge\_MixedOld\_r1250.tif"}
\NormalTok{ielasisanas\_cels}\OtherTok{=}\FunctionTok{paste0}\NormalTok{(}\StringTok{"./RasterGrids\_100m/2024/RAW/"}\NormalTok{,nosaukums)}
\NormalTok{saglabasanas\_cels}\OtherTok{=}\FunctionTok{paste0}\NormalTok{(}\StringTok{"./RasterGrids\_100m/2024/Scaled/"}\NormalTok{,nosaukums)}
\NormalTok{slanis}\OtherTok{=}\FunctionTok{rast}\NormalTok{(ielasisanas\_cels)}
\NormalTok{videjais}\OtherTok{=}\FunctionTok{global}\NormalTok{(slanis,}\AttributeTok{fun=}\StringTok{"mean"}\NormalTok{,}\AttributeTok{na.rm=}\ConstantTok{TRUE}\NormalTok{)}
\NormalTok{centrets}\OtherTok{=}\NormalTok{slanis}\SpecialCharTok{{-}}\NormalTok{videjais[,}\DecValTok{1}\NormalTok{]}
\NormalTok{standartnovirze}\OtherTok{=}\NormalTok{terra}\SpecialCharTok{::}\FunctionTok{global}\NormalTok{(centrets,}\AttributeTok{fun=}\StringTok{"rms"}\NormalTok{,}\AttributeTok{na.rm=}\ConstantTok{TRUE}\NormalTok{)}
\NormalTok{merogots}\OtherTok{=}\NormalTok{centrets}\SpecialCharTok{/}\NormalTok{standartnovirze[,}\DecValTok{1}\NormalTok{]}
\FunctionTok{writeRaster}\NormalTok{(merogots,}
      \AttributeTok{filename=}\NormalTok{saglabasanas\_cels,}
      \AttributeTok{overwrite=}\ConstantTok{TRUE}\NormalTok{)}
\end{Highlighting}
\end{Shaded}

\section{ForestsTreesAge\_MixedOld\_r3000}\label{ch06.366}

\textbf{filename:} \texttt{ForestsTreesAge\_MixedOld\_r3000.tif}

\textbf{layername:} \texttt{egv\_366}

\textbf{English name:} Fractional cover of Old (over rotation age) Mixed Forests
within the 3 km landscape

\textbf{Latvian name:} Vecu (kopš cirtmeta) jauktu koku mežu platības īpatsvars 3 km
ainavā

\textbf{Procedure:} The cover fraction within a radius of 3000 m around the analysis grid cell
is calculated as the area-weighted sum of the \hyperref[ch06.363]{analysis cells} inside
the buffer, using the workflow \texttt{egvtools::radius\_function()}. During the calculation of the landscape
metric, inverse distance weighted (power = 2) gap filling on the output is
applied to ensure no missing values at the edges. Then the layer is
rewritten to set its name. Finally, the layer is standardised by
subtracting the arithmetic mean and dividing by the root mean squared error.

\begin{Shaded}
\begin{Highlighting}[]
\CommentTok{\# libs {-}{-}{-}{-}}
\ControlFlowTok{if}\NormalTok{(}\SpecialCharTok{!}\FunctionTok{require}\NormalTok{(terra)) \{}\FunctionTok{install.packages}\NormalTok{(}\StringTok{"terra"}\NormalTok{); }\FunctionTok{require}\NormalTok{(terra)\}}
\ControlFlowTok{if}\NormalTok{(}\SpecialCharTok{!}\FunctionTok{require}\NormalTok{(egvtools)) \{remotes}\SpecialCharTok{::}\FunctionTok{install\_github}\NormalTok{(}\StringTok{"aavotins/egvtools"}\NormalTok{); }\FunctionTok{require}\NormalTok{(egvtools)\}}


\CommentTok{\# Templates {-}{-}{-}{-}{-}}
\NormalTok{template100}\OtherTok{=}\FunctionTok{rast}\NormalTok{(}\StringTok{"./Templates/TemplateRasters/LV100m\_10km.tif"}\NormalTok{)}

\CommentTok{\# radii {-}{-}{-}{-}}
\FunctionTok{radius\_function}\NormalTok{(}
 \AttributeTok{kvadrati\_path =} \StringTok{"./Templates/TemplateGrids/tiles/"}\NormalTok{,}
 \AttributeTok{radii\_path   =} \StringTok{"./Templates/TemplateGridPoints/tiles/"}\NormalTok{,}
 \AttributeTok{tikls100\_path =} \StringTok{"./Templates/TemplateGrids/tikls100\_sauzeme.parquet"}\NormalTok{,}
 \AttributeTok{template\_path =} \StringTok{"./Templates/TemplateRasters/LV100m\_10km.tif"}\NormalTok{,}
 \AttributeTok{input\_layers  =} \FunctionTok{c}\NormalTok{(}\StringTok{"./RasterGrids\_100m/2024/RAW/ForestsTreesAge\_MixedOld\_cell.tif"}\NormalTok{),}
 \AttributeTok{layer\_prefixes =} \FunctionTok{c}\NormalTok{(}\StringTok{"ForestsTreesAge\_MixedOld"}\NormalTok{),}
 \AttributeTok{output\_dir   =} \StringTok{"./RasterGrids\_100m/2024/RAW/"}\NormalTok{,}
 \AttributeTok{n\_workers   =} \DecValTok{6}\NormalTok{,}
 \AttributeTok{radii     =} \FunctionTok{c}\NormalTok{(}\StringTok{"r3000"}\NormalTok{),}
 \AttributeTok{radius\_mode  =} \StringTok{"sparse"}\NormalTok{,}
 \AttributeTok{extract\_fun  =} \StringTok{"mean"}\NormalTok{,}
 \AttributeTok{fill\_missing  =} \ConstantTok{TRUE}\NormalTok{,}
 \AttributeTok{IDW\_weight   =} \DecValTok{2}\NormalTok{,}
 \AttributeTok{future\_max\_size =} \DecValTok{40} \SpecialCharTok{*} \DecValTok{1024}\SpecialCharTok{\^{}}\DecValTok{3}\NormalTok{)}


\CommentTok{\# ForestsTreesAge\_MixedOld\_r3000.tif    egv\_366}
\NormalTok{slanis}\OtherTok{=}\FunctionTok{rast}\NormalTok{(}\StringTok{"./RasterGrids\_100m/2024/RAW/ForestsTreesAge\_MixedOld\_r3000.tif"}\NormalTok{)}
\FunctionTok{names}\NormalTok{(slanis)}\OtherTok{=}\StringTok{"egv\_366"}
\NormalTok{slanis2}\OtherTok{=}\FunctionTok{project}\NormalTok{(slanis,template100)}
\FunctionTok{writeRaster}\NormalTok{(slanis2,}
      \StringTok{"./RasterGrids\_100m/2024/RAW/ForestsTreesAge\_MixedOld\_r3000.tif"}\NormalTok{,}
      \AttributeTok{overwrite=}\ConstantTok{TRUE}\NormalTok{)}

\CommentTok{\# standardisation {-}{-}{-}{-}}
\ControlFlowTok{if}\NormalTok{(}\SpecialCharTok{!}\FunctionTok{require}\NormalTok{(terra)) \{}\FunctionTok{install.packages}\NormalTok{(}\StringTok{"terra"}\NormalTok{); }\FunctionTok{require}\NormalTok{(terra)\}}
\ControlFlowTok{if}\NormalTok{(}\SpecialCharTok{!}\FunctionTok{require}\NormalTok{(tidyverse)) \{}\FunctionTok{install.packages}\NormalTok{(}\StringTok{"tidyverse"}\NormalTok{); }\FunctionTok{require}\NormalTok{(tidyverse)\}}

\NormalTok{nosaukums}\OtherTok{=}\StringTok{"ForestsTreesAge\_MixedOld\_r3000.tif"}
\NormalTok{ielasisanas\_cels}\OtherTok{=}\FunctionTok{paste0}\NormalTok{(}\StringTok{"./RasterGrids\_100m/2024/RAW/"}\NormalTok{,nosaukums)}
\NormalTok{saglabasanas\_cels}\OtherTok{=}\FunctionTok{paste0}\NormalTok{(}\StringTok{"./RasterGrids\_100m/2024/Scaled/"}\NormalTok{,nosaukums)}
\NormalTok{slanis}\OtherTok{=}\FunctionTok{rast}\NormalTok{(ielasisanas\_cels)}
\NormalTok{videjais}\OtherTok{=}\FunctionTok{global}\NormalTok{(slanis,}\AttributeTok{fun=}\StringTok{"mean"}\NormalTok{,}\AttributeTok{na.rm=}\ConstantTok{TRUE}\NormalTok{)}
\NormalTok{centrets}\OtherTok{=}\NormalTok{slanis}\SpecialCharTok{{-}}\NormalTok{videjais[,}\DecValTok{1}\NormalTok{]}
\NormalTok{standartnovirze}\OtherTok{=}\NormalTok{terra}\SpecialCharTok{::}\FunctionTok{global}\NormalTok{(centrets,}\AttributeTok{fun=}\StringTok{"rms"}\NormalTok{,}\AttributeTok{na.rm=}\ConstantTok{TRUE}\NormalTok{)}
\NormalTok{merogots}\OtherTok{=}\NormalTok{centrets}\SpecialCharTok{/}\NormalTok{standartnovirze[,}\DecValTok{1}\NormalTok{]}
\FunctionTok{writeRaster}\NormalTok{(merogots,}
      \AttributeTok{filename=}\NormalTok{saglabasanas\_cels,}
      \AttributeTok{overwrite=}\ConstantTok{TRUE}\NormalTok{)}
\end{Highlighting}
\end{Shaded}

\section{ForestsTreesAge\_MixedOld\_r10000}\label{ch06.367}

\textbf{filename:} \texttt{ForestsTreesAge\_MixedOld\_r10000.tif}

\textbf{layername:} \texttt{egv\_367}

\textbf{English name:} Fractional cover of Old (over rotation age) Mixed Forests
within the 10 km landscape

\textbf{Latvian name:} Vecu (kopš cirtmeta) jauktu koku mežu platības īpatsvars 10 km
ainavā

\textbf{Procedure:} The cover fraction within a radius of 10000 m around the analysis grid cell
is calculated as the area-weighted sum of the \hyperref[ch06.363]{analysis cells} inside
the buffer, using the workflow \texttt{egvtools::radius\_function()}. During the calculation of the landscape
metric, inverse distance weighted (power = 2) gap filling on the output is
applied to ensure no missing values at the edges. Then the layer is
rewritten to set its name. Finally, the layer is standardised by
subtracting the arithmetic mean and dividing by the root mean squared error.

\begin{Shaded}
\begin{Highlighting}[]
\CommentTok{\# libs {-}{-}{-}{-}}
\ControlFlowTok{if}\NormalTok{(}\SpecialCharTok{!}\FunctionTok{require}\NormalTok{(terra)) \{}\FunctionTok{install.packages}\NormalTok{(}\StringTok{"terra"}\NormalTok{); }\FunctionTok{require}\NormalTok{(terra)\}}
\ControlFlowTok{if}\NormalTok{(}\SpecialCharTok{!}\FunctionTok{require}\NormalTok{(egvtools)) \{remotes}\SpecialCharTok{::}\FunctionTok{install\_github}\NormalTok{(}\StringTok{"aavotins/egvtools"}\NormalTok{); }\FunctionTok{require}\NormalTok{(egvtools)\}}


\CommentTok{\# Templates {-}{-}{-}{-}{-}}
\NormalTok{template100}\OtherTok{=}\FunctionTok{rast}\NormalTok{(}\StringTok{"./Templates/TemplateRasters/LV100m\_10km.tif"}\NormalTok{)}

\CommentTok{\# radii {-}{-}{-}{-}}
\FunctionTok{radius\_function}\NormalTok{(}
 \AttributeTok{kvadrati\_path =} \StringTok{"./Templates/TemplateGrids/tiles/"}\NormalTok{,}
 \AttributeTok{radii\_path   =} \StringTok{"./Templates/TemplateGridPoints/tiles/"}\NormalTok{,}
 \AttributeTok{tikls100\_path =} \StringTok{"./Templates/TemplateGrids/tikls100\_sauzeme.parquet"}\NormalTok{,}
 \AttributeTok{template\_path =} \StringTok{"./Templates/TemplateRasters/LV100m\_10km.tif"}\NormalTok{,}
 \AttributeTok{input\_layers  =} \FunctionTok{c}\NormalTok{(}\StringTok{"./RasterGrids\_100m/2024/RAW/ForestsTreesAge\_MixedOld\_cell.tif"}\NormalTok{),}
 \AttributeTok{layer\_prefixes =} \FunctionTok{c}\NormalTok{(}\StringTok{"ForestsTreesAge\_MixedOld"}\NormalTok{),}
 \AttributeTok{output\_dir   =} \StringTok{"./RasterGrids\_100m/2024/RAW/"}\NormalTok{,}
 \AttributeTok{n\_workers   =} \DecValTok{6}\NormalTok{,}
 \AttributeTok{radii     =} \FunctionTok{c}\NormalTok{(}\StringTok{"r10000"}\NormalTok{),}
 \AttributeTok{radius\_mode  =} \StringTok{"sparse"}\NormalTok{,}
 \AttributeTok{extract\_fun  =} \StringTok{"mean"}\NormalTok{,}
 \AttributeTok{fill\_missing  =} \ConstantTok{TRUE}\NormalTok{,}
 \AttributeTok{IDW\_weight   =} \DecValTok{2}\NormalTok{,}
 \AttributeTok{future\_max\_size =} \DecValTok{40} \SpecialCharTok{*} \DecValTok{1024}\SpecialCharTok{\^{}}\DecValTok{3}\NormalTok{)}


\CommentTok{\# ForestsTreesAge\_MixedOld\_r10000.tif   egv\_367}
\NormalTok{slanis}\OtherTok{=}\FunctionTok{rast}\NormalTok{(}\StringTok{"./RasterGrids\_100m/2024/RAW/ForestsTreesAge\_MixedOld\_r10000.tif"}\NormalTok{)}
\FunctionTok{names}\NormalTok{(slanis)}\OtherTok{=}\StringTok{"egv\_367"}
\NormalTok{slanis2}\OtherTok{=}\FunctionTok{project}\NormalTok{(slanis,template100)}
\FunctionTok{writeRaster}\NormalTok{(slanis2,}
      \StringTok{"./RasterGrids\_100m/2024/RAW/ForestsTreesAge\_MixedOld\_r10000.tif"}\NormalTok{,}
      \AttributeTok{overwrite=}\ConstantTok{TRUE}\NormalTok{)}

\CommentTok{\# standardisation {-}{-}{-}{-}}
\ControlFlowTok{if}\NormalTok{(}\SpecialCharTok{!}\FunctionTok{require}\NormalTok{(terra)) \{}\FunctionTok{install.packages}\NormalTok{(}\StringTok{"terra"}\NormalTok{); }\FunctionTok{require}\NormalTok{(terra)\}}
\ControlFlowTok{if}\NormalTok{(}\SpecialCharTok{!}\FunctionTok{require}\NormalTok{(tidyverse)) \{}\FunctionTok{install.packages}\NormalTok{(}\StringTok{"tidyverse"}\NormalTok{); }\FunctionTok{require}\NormalTok{(tidyverse)\}}

\NormalTok{nosaukums}\OtherTok{=}\StringTok{"ForestsTreesAge\_MixedOld\_r10000.tif"}
\NormalTok{ielasisanas\_cels}\OtherTok{=}\FunctionTok{paste0}\NormalTok{(}\StringTok{"./RasterGrids\_100m/2024/RAW/"}\NormalTok{,nosaukums)}
\NormalTok{saglabasanas\_cels}\OtherTok{=}\FunctionTok{paste0}\NormalTok{(}\StringTok{"./RasterGrids\_100m/2024/Scaled/"}\NormalTok{,nosaukums)}
\NormalTok{slanis}\OtherTok{=}\FunctionTok{rast}\NormalTok{(ielasisanas\_cels)}
\NormalTok{videjais}\OtherTok{=}\FunctionTok{global}\NormalTok{(slanis,}\AttributeTok{fun=}\StringTok{"mean"}\NormalTok{,}\AttributeTok{na.rm=}\ConstantTok{TRUE}\NormalTok{)}
\NormalTok{centrets}\OtherTok{=}\NormalTok{slanis}\SpecialCharTok{{-}}\NormalTok{videjais[,}\DecValTok{1}\NormalTok{]}
\NormalTok{standartnovirze}\OtherTok{=}\NormalTok{terra}\SpecialCharTok{::}\FunctionTok{global}\NormalTok{(centrets,}\AttributeTok{fun=}\StringTok{"rms"}\NormalTok{,}\AttributeTok{na.rm=}\ConstantTok{TRUE}\NormalTok{)}
\NormalTok{merogots}\OtherTok{=}\NormalTok{centrets}\SpecialCharTok{/}\NormalTok{standartnovirze[,}\DecValTok{1}\NormalTok{]}
\FunctionTok{writeRaster}\NormalTok{(merogots,}
      \AttributeTok{filename=}\NormalTok{saglabasanas\_cels,}
      \AttributeTok{overwrite=}\ConstantTok{TRUE}\NormalTok{)}
\end{Highlighting}
\end{Shaded}

\section{ForestsTreesAge\_MixedYoung\_cell}\label{ch06.368}

\textbf{filename:} \texttt{ForestsTreesAge\_MixedYoung\_cell.tif}

\textbf{layername:} \texttt{egv\_368}

\textbf{English name:} Fractional cover of Young (pre-rotation age) Mixed Forests
within the analysis cell (1 ha)

\textbf{Latvian name:} Jaunu (pirms cirtmeta) jauktu koku mežu platības īpatsvars
analīzes šūnā (1 ha)

\textbf{Procedure:} Most EGVs describing forests are spatially restricted to areas outside
of clearcuts and dead stands. This mask is created using a combination of
the \hyperref[Ch04.01]{State Forest Service's
State Forest Registry} land category 12 and 14, and \hyperref[Ch04.09]{The
Global Forest Watch} pixels classified as lost tree canopy cover since
2020 (raster layer matching input, presence = 1, absence = 0).

To prepare this EGV, stands from the \hyperref[Ch04.01]{State Forest Service's State Forest
Registry} are classified into (in order):

\begin{itemize}
\item
  coniferous (see \hyperref[Ch01]{Terminology and acronyms} for species codes) if
  timber volume of those species exceeded 75\%;
\item
  Boreal deciduous if timber volume of those species exceeded 75\%;
\item
  temperate deciduous if timber volume of those species exceeded 50\%;
\item
  mixed otherwise;
\end{itemize}

then mixed stands younger than the legal rotation age are selected and
geometries are rasterised (presence = 1, NA otherwise). Rasterisation is
performed using the workflow \texttt{egvtools::polygon2input()}, restricting to pixels outside clearcut
mask and covering background with value 0. The resulting layer
is then aggregated to EGV resolution using the workflow \texttt{egvtools::input2egv()}, which
calculates the arithmetic mean to determine the cover fraction. During
aggregation, inverse distance weighted (power = 2) gap filling on the output is
applied to ensure no missing values at the edges. Finally, the layer is
standardised by subtracting the arithmetic mean and dividing by the root mean squared
error.

\begin{Shaded}
\begin{Highlighting}[]
\CommentTok{\# libs {-}{-}{-}{-}}
\ControlFlowTok{if}\NormalTok{(}\SpecialCharTok{!}\FunctionTok{require}\NormalTok{(egvtools)) \{remotes}\SpecialCharTok{::}\FunctionTok{install\_github}\NormalTok{(}\StringTok{"aavotins/egvtools"}\NormalTok{); }\FunctionTok{require}\NormalTok{(egvtools)\}}
\ControlFlowTok{if}\NormalTok{(}\SpecialCharTok{!}\FunctionTok{require}\NormalTok{(terra)) \{}\FunctionTok{install.packages}\NormalTok{(}\StringTok{"terra"}\NormalTok{); }\FunctionTok{require}\NormalTok{(terra)\}}
\ControlFlowTok{if}\NormalTok{(}\SpecialCharTok{!}\FunctionTok{require}\NormalTok{(sf)) \{}\FunctionTok{install.packages}\NormalTok{(}\StringTok{"sf"}\NormalTok{); }\FunctionTok{require}\NormalTok{(sf)\}}
\ControlFlowTok{if}\NormalTok{(}\SpecialCharTok{!}\FunctionTok{require}\NormalTok{(tidyverse)) \{}\FunctionTok{install.packages}\NormalTok{(}\StringTok{"tidyverse"}\NormalTok{); }\FunctionTok{require}\NormalTok{(tidyverse)\}}
\ControlFlowTok{if}\NormalTok{(}\SpecialCharTok{!}\FunctionTok{require}\NormalTok{(sfarrow)) \{}\FunctionTok{install.packages}\NormalTok{(}\StringTok{"sfarrow"}\NormalTok{); }\FunctionTok{require}\NormalTok{(sfarrow)\}}
\ControlFlowTok{if}\NormalTok{(}\SpecialCharTok{!}\FunctionTok{require}\NormalTok{(readxl)) \{}\FunctionTok{install.packages}\NormalTok{(}\StringTok{"readxl"}\NormalTok{); }\FunctionTok{require}\NormalTok{(readxl)\}}
\ControlFlowTok{if}\NormalTok{(}\SpecialCharTok{!}\FunctionTok{require}\NormalTok{(raster)) \{}\FunctionTok{install.packages}\NormalTok{(}\StringTok{"raster"}\NormalTok{); }\FunctionTok{require}\NormalTok{(raster)\}}
\ControlFlowTok{if}\NormalTok{(}\SpecialCharTok{!}\FunctionTok{require}\NormalTok{(fasterize)) \{}\FunctionTok{install.packages}\NormalTok{(}\StringTok{"fasterize"}\NormalTok{); }\FunctionTok{require}\NormalTok{(fasterize)\}}

\CommentTok{\# templates {-}{-}{-}{-}}
\NormalTok{template100}\OtherTok{=}\FunctionTok{rast}\NormalTok{(}\StringTok{"./Templates/TemplateRasters/LV100m\_10km.tif"}\NormalTok{)}
\NormalTok{template10}\OtherTok{=}\FunctionTok{rast}\NormalTok{(}\StringTok{"./Templates/TemplateRasters/LV10m\_10km.tif"}\NormalTok{)}
\NormalTok{rastrs10}\OtherTok{=}\FunctionTok{raster}\NormalTok{(template10)}

\NormalTok{nulls10}\OtherTok{=}\FunctionTok{rast}\NormalTok{(}\StringTok{"./Templates/TemplateRasters/nulls\_LV10m\_10km.tif"}\NormalTok{)}
\NormalTok{nulls100}\OtherTok{=}\FunctionTok{rast}\NormalTok{(}\StringTok{"./Templates/TemplateRasters/nulls\_LV100m\_10km.tif"}\NormalTok{)}


\CommentTok{\# simple landscape {-}{-}{-}{-}}
\NormalTok{simple\_landscape}\OtherTok{=}\FunctionTok{rast}\NormalTok{(}\StringTok{"RasterGrids\_10m/2024/Ainava\_vienk\_mask.tif"}\NormalTok{)}

\CommentTok{\# mvr {-}{-}{-}{-}}
\NormalTok{mvr}\OtherTok{=}\FunctionTok{st\_read\_parquet}\NormalTok{(}\StringTok{"./Geodata/2024/MVR/nogabali\_2024janv.parquet"}\NormalTok{)}
\NormalTok{mvr}\SpecialCharTok{$}\NormalTok{yes}\OtherTok{=}\DecValTok{1}

\CommentTok{\# clear cut mask {-}{-}{-}{-}}
\NormalTok{izcirtumi}\OtherTok{=}\NormalTok{mvr }\SpecialCharTok{\%\textgreater{}\%} 
 \FunctionTok{filter}\NormalTok{(zkat }\SpecialCharTok{\%in\%} \FunctionTok{c}\NormalTok{(}\StringTok{"12"}\NormalTok{,}\StringTok{"14"}\NormalTok{)) }\SpecialCharTok{\%\textgreater{}\%} 
\NormalTok{ dplyr}\SpecialCharTok{::}\FunctionTok{select}\NormalTok{(yes)}
\NormalTok{r\_izcirtumi\_mvr}\OtherTok{=}\FunctionTok{fasterize}\NormalTok{(izcirtumi,rastrs10,}\AttributeTok{field=}\StringTok{"yes"}\NormalTok{)}
\NormalTok{t\_izcirtumi\_mvr}\OtherTok{=}\FunctionTok{rast}\NormalTok{(r\_izcirtumi\_mvr)}
\FunctionTok{plot}\NormalTok{(t\_izcirtumi\_mvr)}

\NormalTok{tcl}\OtherTok{=}\FunctionTok{rast}\NormalTok{(}\StringTok{"./Geodata/2024/Trees/GFW/TreeCoverLoss\_v1\_12.tif"}\NormalTok{)}
\NormalTok{tcl2}\OtherTok{=}\FunctionTok{ifel}\NormalTok{(tcl}\SpecialCharTok{\textless{}}\DecValTok{20}\NormalTok{,}\DecValTok{0}\NormalTok{,}\DecValTok{1}\NormalTok{)}
\NormalTok{tclX}\OtherTok{=}\FunctionTok{cover}\NormalTok{(tcl2,nulls10)}
\FunctionTok{plot}\NormalTok{(tclX)}

\NormalTok{clearcut\_mask}\OtherTok{=}\FunctionTok{cover}\NormalTok{(t\_izcirtumi\_mvr,tclX,}
          \AttributeTok{filename=}\StringTok{"./RasterGrids\_10m/2024/Mask\_clearcuts.tif"}\NormalTok{,}
          \AttributeTok{overwrite=}\ConstantTok{TRUE}\NormalTok{)}
\FunctionTok{plot}\NormalTok{(clearcut\_mask)}

\FunctionTok{rm}\NormalTok{(izcirtumi)}
\FunctionTok{rm}\NormalTok{(r\_izcirtumi\_mvr)}
\FunctionTok{rm}\NormalTok{(t\_izcirtumi\_mvr)}
\FunctionTok{rm}\NormalTok{(tcl)}
\FunctionTok{rm}\NormalTok{(tcl2)}
\FunctionTok{rm}\NormalTok{(tclX)}

\CommentTok{\# ForestsTreesAge\_MixedYoung\_cell.tif   egv\_368 {-}{-}{-}{-}}
\NormalTok{skujkoki}\OtherTok{=}\FunctionTok{c}\NormalTok{(}\StringTok{"1"}\NormalTok{,}\StringTok{"3"}\NormalTok{,}\StringTok{"13"}\NormalTok{,}\StringTok{"14"}\NormalTok{,}\StringTok{"15"}\NormalTok{,}\StringTok{"22"}\NormalTok{,}\StringTok{"23"}\NormalTok{,}\StringTok{"28"}\NormalTok{) }\CommentTok{\# 8}
\NormalTok{saurlapji}\OtherTok{=}\FunctionTok{c}\NormalTok{(}\StringTok{"4"}\NormalTok{,}\StringTok{"6"}\NormalTok{,}\StringTok{"8"}\NormalTok{,}\StringTok{"9"}\NormalTok{,}\StringTok{"19"}\NormalTok{,}\StringTok{"20"}\NormalTok{,}\StringTok{"21"}\NormalTok{,}\StringTok{"32"}\NormalTok{,}\StringTok{"35"}\NormalTok{,}\StringTok{"68"}\NormalTok{) }\CommentTok{\# 10}
\NormalTok{platlapji}\OtherTok{=}\FunctionTok{c}\NormalTok{(}\StringTok{"10"}\NormalTok{,}\StringTok{"11"}\NormalTok{,}\StringTok{"12"}\NormalTok{,}\StringTok{"16"}\NormalTok{,}\StringTok{"17"}\NormalTok{,}\StringTok{"18"}\NormalTok{,}\StringTok{"24"}\NormalTok{,}\StringTok{"25"}\NormalTok{,}\StringTok{"26"}\NormalTok{,}\StringTok{"27"}\NormalTok{,}\StringTok{"28"}\NormalTok{,}\StringTok{"29"}\NormalTok{,}\StringTok{"50"}\NormalTok{,}
      \StringTok{"61"}\NormalTok{,}\StringTok{"62"}\NormalTok{,}\StringTok{"63"}\NormalTok{,}\StringTok{"64"}\NormalTok{,}\StringTok{"65"}\NormalTok{,}\StringTok{"66"}\NormalTok{,}\StringTok{"67"}\NormalTok{,}\StringTok{"69"}\NormalTok{) }\CommentTok{\# 21}
\NormalTok{mvr}\OtherTok{=}\NormalTok{mvr }\SpecialCharTok{\%\textgreater{}\%} 
 \FunctionTok{mutate}\NormalTok{(}\AttributeTok{kraja\_skujkoku=}\FunctionTok{ifelse}\NormalTok{(s10 }\SpecialCharTok{\%in\%}\NormalTok{ skujkoki,v10,}\DecValTok{0}\NormalTok{)}\SpecialCharTok{+}
      \FunctionTok{ifelse}\NormalTok{(s11 }\SpecialCharTok{\%in\%}\NormalTok{ skujkoki,v11,}\DecValTok{0}\NormalTok{)}\SpecialCharTok{+}\FunctionTok{ifelse}\NormalTok{(s12 }\SpecialCharTok{\%in\%}\NormalTok{ skujkoki,v12,}\DecValTok{0}\NormalTok{)}\SpecialCharTok{+}
      \FunctionTok{ifelse}\NormalTok{(s13 }\SpecialCharTok{\%in\%}\NormalTok{ skujkoki,v13,}\DecValTok{0}\NormalTok{)}\SpecialCharTok{+}\FunctionTok{ifelse}\NormalTok{(s14 }\SpecialCharTok{\%in\%}\NormalTok{ skujkoki,v14,}\DecValTok{0}\NormalTok{),}
     \AttributeTok{kraja\_saurlapju=}\FunctionTok{ifelse}\NormalTok{(s10 }\SpecialCharTok{\%in\%}\NormalTok{ saurlapji,v10,}\DecValTok{0}\NormalTok{)}\SpecialCharTok{+}
      \FunctionTok{ifelse}\NormalTok{(s11 }\SpecialCharTok{\%in\%}\NormalTok{ saurlapji,v11,}\DecValTok{0}\NormalTok{)}\SpecialCharTok{+}\FunctionTok{ifelse}\NormalTok{(s12 }\SpecialCharTok{\%in\%}\NormalTok{ saurlapji,v12,}\DecValTok{0}\NormalTok{)}\SpecialCharTok{+}
      \FunctionTok{ifelse}\NormalTok{(s13 }\SpecialCharTok{\%in\%}\NormalTok{ saurlapji,v13,}\DecValTok{0}\NormalTok{)}\SpecialCharTok{+}\FunctionTok{ifelse}\NormalTok{(s14 }\SpecialCharTok{\%in\%}\NormalTok{ saurlapji,v14,}\DecValTok{0}\NormalTok{),}
     \AttributeTok{kraja\_platlapju=}\FunctionTok{ifelse}\NormalTok{(s10 }\SpecialCharTok{\%in\%}\NormalTok{ platlapji,v10,}\DecValTok{0}\NormalTok{)}\SpecialCharTok{+}
      \FunctionTok{ifelse}\NormalTok{(s11 }\SpecialCharTok{\%in\%}\NormalTok{ platlapji,v11,}\DecValTok{0}\NormalTok{)}\SpecialCharTok{+}\FunctionTok{ifelse}\NormalTok{(s12 }\SpecialCharTok{\%in\%}\NormalTok{ platlapji,v12,}\DecValTok{0}\NormalTok{)}\SpecialCharTok{+}
      \FunctionTok{ifelse}\NormalTok{(s13 }\SpecialCharTok{\%in\%}\NormalTok{ platlapji,v13,}\DecValTok{0}\NormalTok{)}\SpecialCharTok{+}\FunctionTok{ifelse}\NormalTok{(s14 }\SpecialCharTok{\%in\%}\NormalTok{ platlapji,v14,}\DecValTok{0}\NormalTok{)) }\SpecialCharTok{\%\textgreater{}\%} 
 \FunctionTok{mutate}\NormalTok{(}\AttributeTok{kopeja\_kraja=}\NormalTok{kraja\_skujkoku}\SpecialCharTok{+}\NormalTok{kraja\_platlapju}\SpecialCharTok{+}\NormalTok{kraja\_saurlapju) }\SpecialCharTok{\%\textgreater{}\%} 
 \FunctionTok{mutate}\NormalTok{(}\AttributeTok{tips=}\FunctionTok{ifelse}\NormalTok{(kraja\_skujkoku}\SpecialCharTok{/}\NormalTok{kopeja\_kraja}\SpecialCharTok{\textgreater{}=}\FloatTok{0.75}\NormalTok{,}\StringTok{"skujkoku"}\NormalTok{,}
           \FunctionTok{ifelse}\NormalTok{(kraja\_saurlapju}\SpecialCharTok{/}\NormalTok{kopeja\_kraja}\SpecialCharTok{\textgreater{}=}\FloatTok{0.75}\NormalTok{,}\StringTok{"saurlapju"}\NormalTok{,}
              \FunctionTok{ifelse}\NormalTok{(kraja\_platlapju}\SpecialCharTok{/}\NormalTok{kopeja\_kraja}\SpecialCharTok{\textgreater{}}\FloatTok{0.5}\NormalTok{,}\StringTok{"platlapju"}\NormalTok{,}
                  \StringTok{"jauktu koku"}\NormalTok{))))}
\NormalTok{nogabali}\OtherTok{=}\NormalTok{mvr }\SpecialCharTok{\%\textgreater{}\%} 
 \FunctionTok{filter}\NormalTok{(zkat}\SpecialCharTok{==}\StringTok{"10"}\SpecialCharTok{\&}\NormalTok{tips}\SpecialCharTok{==}\StringTok{"jauktu koku"}\SpecialCharTok{\&}\NormalTok{(vgr}\SpecialCharTok{==}\StringTok{"1"}\SpecialCharTok{|}\NormalTok{vgr}\SpecialCharTok{==}\StringTok{"2"}\SpecialCharTok{|}\NormalTok{vgr}\SpecialCharTok{==}\StringTok{"3"}\NormalTok{))}

\NormalTok{p2i\_rez}\OtherTok{=}\NormalTok{egvtools}\SpecialCharTok{::}\FunctionTok{polygon2input}\NormalTok{(}\AttributeTok{vector\_data =}\NormalTok{ nogabali,}
                \AttributeTok{template\_path =} \StringTok{"./Templates/TemplateRasters/LV10m\_10km.tif"}\NormalTok{,}
                \AttributeTok{out\_path =} \StringTok{"./RasterGrids\_10m/2024/"}\NormalTok{,}
                \AttributeTok{file\_name =} \StringTok{"ForestsTreesAge\_MixedYoung\_input.tif"}\NormalTok{,}
                \AttributeTok{value\_field =} \StringTok{"yes"}\NormalTok{,}
                \AttributeTok{restrict\_to =}\NormalTok{ clearcut\_mask,}
                \AttributeTok{restrict\_values =} \DecValTok{0}\NormalTok{,}
                \AttributeTok{prepare=}\ConstantTok{FALSE}\NormalTok{,}
                \AttributeTok{background\_raster =} \StringTok{"./Templates/TemplateRasters/nulls\_LV10m\_10km.tif"}\NormalTok{,}
                \AttributeTok{plot\_result =} \ConstantTok{TRUE}\NormalTok{)}
\NormalTok{p2i\_rez}
\NormalTok{i2e\_rez}\OtherTok{=}\NormalTok{egvtools}\SpecialCharTok{::}\FunctionTok{input2egv}\NormalTok{(}\AttributeTok{input=}\FunctionTok{paste0}\NormalTok{(}\StringTok{"./RasterGrids\_10m/2024/"}\NormalTok{,}
                     \StringTok{"ForestsTreesAge\_MixedYoung\_input.tif"}\NormalTok{),}
              \AttributeTok{egv\_template=} \StringTok{"./Templates/TemplateRasters/LV100m\_10km.tif"}\NormalTok{,}
              \AttributeTok{summary\_function =} \StringTok{"average"}\NormalTok{,}
              \AttributeTok{missing\_job =} \StringTok{"FillOutput"}\NormalTok{,}
              \AttributeTok{outlocation =} \StringTok{"./RasterGrids\_100m/2024/RAW/"}\NormalTok{,}
              \AttributeTok{outfilename =} \StringTok{"ForestsTreesAge\_MixedYoung\_cell.tif"}\NormalTok{,}
              \AttributeTok{layername =} \StringTok{"egv\_368"}\NormalTok{,}
              \AttributeTok{idw\_weight =} \DecValTok{2}\NormalTok{,}
              \AttributeTok{plot\_gaps =} \ConstantTok{FALSE}\NormalTok{,}\AttributeTok{plot\_final =} \ConstantTok{TRUE}\NormalTok{)}
\NormalTok{i2e\_rez}
\FunctionTok{rm}\NormalTok{(nogabali)}
\FunctionTok{rm}\NormalTok{(p2i\_rez)}
\FunctionTok{rm}\NormalTok{(i2e\_rez)}
\FunctionTok{unlink}\NormalTok{(}\StringTok{"./RasterGrids\_10m/2024/ForestsTreesAge\_MixedYoung\_input.tif"}\NormalTok{)}

\CommentTok{\# standardisation {-}{-}{-}{-}}
\ControlFlowTok{if}\NormalTok{(}\SpecialCharTok{!}\FunctionTok{require}\NormalTok{(terra)) \{}\FunctionTok{install.packages}\NormalTok{(}\StringTok{"terra"}\NormalTok{); }\FunctionTok{require}\NormalTok{(terra)\}}
\ControlFlowTok{if}\NormalTok{(}\SpecialCharTok{!}\FunctionTok{require}\NormalTok{(tidyverse)) \{}\FunctionTok{install.packages}\NormalTok{(}\StringTok{"tidyverse"}\NormalTok{); }\FunctionTok{require}\NormalTok{(tidyverse)\}}

\NormalTok{nosaukums}\OtherTok{=}\StringTok{"ForestsTreesAge\_MixedYoung\_cell.tif"}
\NormalTok{ielasisanas\_cels}\OtherTok{=}\FunctionTok{paste0}\NormalTok{(}\StringTok{"./RasterGrids\_100m/2024/RAW/"}\NormalTok{,nosaukums)}
\NormalTok{saglabasanas\_cels}\OtherTok{=}\FunctionTok{paste0}\NormalTok{(}\StringTok{"./RasterGrids\_100m/2024/Scaled/"}\NormalTok{,nosaukums)}
\NormalTok{slanis}\OtherTok{=}\FunctionTok{rast}\NormalTok{(ielasisanas\_cels)}
\NormalTok{videjais}\OtherTok{=}\FunctionTok{global}\NormalTok{(slanis,}\AttributeTok{fun=}\StringTok{"mean"}\NormalTok{,}\AttributeTok{na.rm=}\ConstantTok{TRUE}\NormalTok{)}
\NormalTok{centrets}\OtherTok{=}\NormalTok{slanis}\SpecialCharTok{{-}}\NormalTok{videjais[,}\DecValTok{1}\NormalTok{]}
\NormalTok{standartnovirze}\OtherTok{=}\NormalTok{terra}\SpecialCharTok{::}\FunctionTok{global}\NormalTok{(centrets,}\AttributeTok{fun=}\StringTok{"rms"}\NormalTok{,}\AttributeTok{na.rm=}\ConstantTok{TRUE}\NormalTok{)}
\NormalTok{merogots}\OtherTok{=}\NormalTok{centrets}\SpecialCharTok{/}\NormalTok{standartnovirze[,}\DecValTok{1}\NormalTok{]}
\FunctionTok{writeRaster}\NormalTok{(merogots,}
      \AttributeTok{filename=}\NormalTok{saglabasanas\_cels,}
      \AttributeTok{overwrite=}\ConstantTok{TRUE}\NormalTok{)}
\end{Highlighting}
\end{Shaded}

\section{ForestsTreesAge\_MixedYoung\_r500}\label{ch06.369}

\textbf{filename:} \texttt{ForestsTreesAge\_MixedYoung\_r500.tif}

\textbf{layername:} \texttt{egv\_369}

\textbf{English name:} Fractional cover of Young (pre-rotation age) Mixed Forests
within the 0.5 km landscape

\textbf{Latvian name:} Jaunu (pirms cirtmeta) jauktu koku mežu platības īpatsvars 0,5
km ainavā

\textbf{Procedure:} The cover fraction within a radius of 500 m around the analysis grid cell is
calculated as the area-weighted sum of the \hyperref[ch06.368]{analysis cells} inside the
buffer, using the workflow \texttt{egvtools::radius\_function()}. During the calculation of the landscape metric,
inverse distance weighted (power = 2) gap filling on the output is applied
to ensure no missing values at the edges. Then the layer is rewritten to set
its name. Finally, the layer is standardised by subtracting the arithmetic
mean and dividing by the root mean squared error.

\begin{Shaded}
\begin{Highlighting}[]
\CommentTok{\# libs {-}{-}{-}{-}}
\ControlFlowTok{if}\NormalTok{(}\SpecialCharTok{!}\FunctionTok{require}\NormalTok{(terra)) \{}\FunctionTok{install.packages}\NormalTok{(}\StringTok{"terra"}\NormalTok{); }\FunctionTok{require}\NormalTok{(terra)\}}
\ControlFlowTok{if}\NormalTok{(}\SpecialCharTok{!}\FunctionTok{require}\NormalTok{(egvtools)) \{remotes}\SpecialCharTok{::}\FunctionTok{install\_github}\NormalTok{(}\StringTok{"aavotins/egvtools"}\NormalTok{); }\FunctionTok{require}\NormalTok{(egvtools)\}}


\CommentTok{\# Templates {-}{-}{-}{-}{-}}
\NormalTok{template100}\OtherTok{=}\FunctionTok{rast}\NormalTok{(}\StringTok{"./Templates/TemplateRasters/LV100m\_10km.tif"}\NormalTok{)}

\CommentTok{\# radii {-}{-}{-}{-}}
\FunctionTok{radius\_function}\NormalTok{(}
 \AttributeTok{kvadrati\_path =} \StringTok{"./Templates/TemplateGrids/tiles/"}\NormalTok{,}
 \AttributeTok{radii\_path   =} \StringTok{"./Templates/TemplateGridPoints/tiles/"}\NormalTok{,}
 \AttributeTok{tikls100\_path =} \StringTok{"./Templates/TemplateGrids/tikls100\_sauzeme.parquet"}\NormalTok{,}
 \AttributeTok{template\_path =} \StringTok{"./Templates/TemplateRasters/LV100m\_10km.tif"}\NormalTok{,}
 \AttributeTok{input\_layers  =} \FunctionTok{c}\NormalTok{(}\StringTok{"./RasterGrids\_100m/2024/RAW/ForestsTreesAge\_MixedYoung\_cell.tif"}\NormalTok{),}
 \AttributeTok{layer\_prefixes =} \FunctionTok{c}\NormalTok{(}\StringTok{"ForestsTreesAge\_MixedYoung"}\NormalTok{),}
 \AttributeTok{output\_dir   =} \StringTok{"./RasterGrids\_100m/2024/RAW/"}\NormalTok{,}
 \AttributeTok{n\_workers   =} \DecValTok{6}\NormalTok{,}
 \AttributeTok{radii     =} \FunctionTok{c}\NormalTok{(}\StringTok{"r500"}\NormalTok{),}
 \AttributeTok{radius\_mode  =} \StringTok{"sparse"}\NormalTok{,}
 \AttributeTok{extract\_fun  =} \StringTok{"mean"}\NormalTok{,}
 \AttributeTok{fill\_missing  =} \ConstantTok{TRUE}\NormalTok{,}
 \AttributeTok{IDW\_weight   =} \DecValTok{2}\NormalTok{,}
 \AttributeTok{future\_max\_size =} \DecValTok{40} \SpecialCharTok{*} \DecValTok{1024}\SpecialCharTok{\^{}}\DecValTok{3}\NormalTok{)}


\CommentTok{\# ForestsTreesAge\_MixedYoung\_r500.tif   egv\_369}
\NormalTok{slanis}\OtherTok{=}\FunctionTok{rast}\NormalTok{(}\StringTok{"./RasterGrids\_100m/2024/RAW/ForestsTreesAge\_MixedYoung\_r500.tif"}\NormalTok{)}
\FunctionTok{names}\NormalTok{(slanis)}\OtherTok{=}\StringTok{"egv\_369"}
\NormalTok{slanis2}\OtherTok{=}\FunctionTok{project}\NormalTok{(slanis,template100)}
\FunctionTok{writeRaster}\NormalTok{(slanis2,}
      \StringTok{"./RasterGrids\_100m/2024/RAW/ForestsTreesAge\_MixedYoung\_r500.tif"}\NormalTok{,}
      \AttributeTok{overwrite=}\ConstantTok{TRUE}\NormalTok{)}

\CommentTok{\# standardisation {-}{-}{-}{-}}
\ControlFlowTok{if}\NormalTok{(}\SpecialCharTok{!}\FunctionTok{require}\NormalTok{(terra)) \{}\FunctionTok{install.packages}\NormalTok{(}\StringTok{"terra"}\NormalTok{); }\FunctionTok{require}\NormalTok{(terra)\}}
\ControlFlowTok{if}\NormalTok{(}\SpecialCharTok{!}\FunctionTok{require}\NormalTok{(tidyverse)) \{}\FunctionTok{install.packages}\NormalTok{(}\StringTok{"tidyverse"}\NormalTok{); }\FunctionTok{require}\NormalTok{(tidyverse)\}}

\NormalTok{nosaukums}\OtherTok{=}\StringTok{"ForestsTreesAge\_MixedYoung\_r500.tif"}
\NormalTok{ielasisanas\_cels}\OtherTok{=}\FunctionTok{paste0}\NormalTok{(}\StringTok{"./RasterGrids\_100m/2024/RAW/"}\NormalTok{,nosaukums)}
\NormalTok{saglabasanas\_cels}\OtherTok{=}\FunctionTok{paste0}\NormalTok{(}\StringTok{"./RasterGrids\_100m/2024/Scaled/"}\NormalTok{,nosaukums)}
\NormalTok{slanis}\OtherTok{=}\FunctionTok{rast}\NormalTok{(ielasisanas\_cels)}
\NormalTok{videjais}\OtherTok{=}\FunctionTok{global}\NormalTok{(slanis,}\AttributeTok{fun=}\StringTok{"mean"}\NormalTok{,}\AttributeTok{na.rm=}\ConstantTok{TRUE}\NormalTok{)}
\NormalTok{centrets}\OtherTok{=}\NormalTok{slanis}\SpecialCharTok{{-}}\NormalTok{videjais[,}\DecValTok{1}\NormalTok{]}
\NormalTok{standartnovirze}\OtherTok{=}\NormalTok{terra}\SpecialCharTok{::}\FunctionTok{global}\NormalTok{(centrets,}\AttributeTok{fun=}\StringTok{"rms"}\NormalTok{,}\AttributeTok{na.rm=}\ConstantTok{TRUE}\NormalTok{)}
\NormalTok{merogots}\OtherTok{=}\NormalTok{centrets}\SpecialCharTok{/}\NormalTok{standartnovirze[,}\DecValTok{1}\NormalTok{]}
\FunctionTok{writeRaster}\NormalTok{(merogots,}
      \AttributeTok{filename=}\NormalTok{saglabasanas\_cels,}
      \AttributeTok{overwrite=}\ConstantTok{TRUE}\NormalTok{)}
\end{Highlighting}
\end{Shaded}

\section{ForestsTreesAge\_MixedYoung\_r1250}\label{ch06.370}

\textbf{filename:} \texttt{ForestsTreesAge\_MixedYoung\_r1250.tif}

\textbf{layername:} \texttt{egv\_370}

\textbf{English name:} Fractional cover of Young (pre-rotation age) Mixed Forests
within the 1.25 km landscape

\textbf{Latvian name:} Jaunu (pirms cirtmeta) jauktu koku mežu platības īpatsvars
1,25 km ainavā

\textbf{Procedure:} The cover fraction within a radius of 1250 m around the analysis grid cell
is calculated as the area-weighted sum of the \hyperref[ch06.368]{analysis cells} inside
the buffer, using the workflow \texttt{egvtools::radius\_function()}. During the calculation of the landscape
metric, inverse distance weighted (power = 2) gap filling on the output is
applied to ensure no missing values at the edges. Then the layer is
rewritten to set its name. Finally, the layer is standardised by
subtracting the arithmetic mean and dividing by the root mean squared error.

\begin{Shaded}
\begin{Highlighting}[]
\CommentTok{\# libs {-}{-}{-}{-}}
\ControlFlowTok{if}\NormalTok{(}\SpecialCharTok{!}\FunctionTok{require}\NormalTok{(terra)) \{}\FunctionTok{install.packages}\NormalTok{(}\StringTok{"terra"}\NormalTok{); }\FunctionTok{require}\NormalTok{(terra)\}}
\ControlFlowTok{if}\NormalTok{(}\SpecialCharTok{!}\FunctionTok{require}\NormalTok{(egvtools)) \{remotes}\SpecialCharTok{::}\FunctionTok{install\_github}\NormalTok{(}\StringTok{"aavotins/egvtools"}\NormalTok{); }\FunctionTok{require}\NormalTok{(egvtools)\}}


\CommentTok{\# Templates {-}{-}{-}{-}{-}}
\NormalTok{template100}\OtherTok{=}\FunctionTok{rast}\NormalTok{(}\StringTok{"./Templates/TemplateRasters/LV100m\_10km.tif"}\NormalTok{)}

\CommentTok{\# radii {-}{-}{-}{-}}
\FunctionTok{radius\_function}\NormalTok{(}
 \AttributeTok{kvadrati\_path =} \StringTok{"./Templates/TemplateGrids/tiles/"}\NormalTok{,}
 \AttributeTok{radii\_path   =} \StringTok{"./Templates/TemplateGridPoints/tiles/"}\NormalTok{,}
 \AttributeTok{tikls100\_path =} \StringTok{"./Templates/TemplateGrids/tikls100\_sauzeme.parquet"}\NormalTok{,}
 \AttributeTok{template\_path =} \StringTok{"./Templates/TemplateRasters/LV100m\_10km.tif"}\NormalTok{,}
 \AttributeTok{input\_layers  =} \FunctionTok{c}\NormalTok{(}\StringTok{"./RasterGrids\_100m/2024/RAW/ForestsTreesAge\_MixedYoung\_cell.tif"}\NormalTok{),}
 \AttributeTok{layer\_prefixes =} \FunctionTok{c}\NormalTok{(}\StringTok{"ForestsTreesAge\_MixedYoung"}\NormalTok{),}
 \AttributeTok{output\_dir   =} \StringTok{"./RasterGrids\_100m/2024/RAW/"}\NormalTok{,}
 \AttributeTok{n\_workers   =} \DecValTok{6}\NormalTok{,}
 \AttributeTok{radii     =} \FunctionTok{c}\NormalTok{(}\StringTok{"r1250"}\NormalTok{),}
 \AttributeTok{radius\_mode  =} \StringTok{"sparse"}\NormalTok{,}
 \AttributeTok{extract\_fun  =} \StringTok{"mean"}\NormalTok{,}
 \AttributeTok{fill\_missing  =} \ConstantTok{TRUE}\NormalTok{,}
 \AttributeTok{IDW\_weight   =} \DecValTok{2}\NormalTok{,}
 \AttributeTok{future\_max\_size =} \DecValTok{40} \SpecialCharTok{*} \DecValTok{1024}\SpecialCharTok{\^{}}\DecValTok{3}\NormalTok{)}


\CommentTok{\# ForestsTreesAge\_MixedYoung\_r1250.tif  egv\_370}
\NormalTok{slanis}\OtherTok{=}\FunctionTok{rast}\NormalTok{(}\StringTok{"./RasterGrids\_100m/2024/RAW/ForestsTreesAge\_MixedYoung\_r1250.tif"}\NormalTok{)}
\FunctionTok{names}\NormalTok{(slanis)}\OtherTok{=}\StringTok{"egv\_370"}
\NormalTok{slanis2}\OtherTok{=}\FunctionTok{project}\NormalTok{(slanis,template100)}
\FunctionTok{writeRaster}\NormalTok{(slanis2,}
      \StringTok{"./RasterGrids\_100m/2024/RAW/ForestsTreesAge\_MixedYoung\_r1250.tif"}\NormalTok{,}
      \AttributeTok{overwrite=}\ConstantTok{TRUE}\NormalTok{)}

\CommentTok{\# standardisation {-}{-}{-}{-}}
\ControlFlowTok{if}\NormalTok{(}\SpecialCharTok{!}\FunctionTok{require}\NormalTok{(terra)) \{}\FunctionTok{install.packages}\NormalTok{(}\StringTok{"terra"}\NormalTok{); }\FunctionTok{require}\NormalTok{(terra)\}}
\ControlFlowTok{if}\NormalTok{(}\SpecialCharTok{!}\FunctionTok{require}\NormalTok{(tidyverse)) \{}\FunctionTok{install.packages}\NormalTok{(}\StringTok{"tidyverse"}\NormalTok{); }\FunctionTok{require}\NormalTok{(tidyverse)\}}

\NormalTok{nosaukums}\OtherTok{=}\StringTok{"ForestsTreesAge\_MixedYoung\_r1250.tif"}
\NormalTok{ielasisanas\_cels}\OtherTok{=}\FunctionTok{paste0}\NormalTok{(}\StringTok{"./RasterGrids\_100m/2024/RAW/"}\NormalTok{,nosaukums)}
\NormalTok{saglabasanas\_cels}\OtherTok{=}\FunctionTok{paste0}\NormalTok{(}\StringTok{"./RasterGrids\_100m/2024/Scaled/"}\NormalTok{,nosaukums)}
\NormalTok{slanis}\OtherTok{=}\FunctionTok{rast}\NormalTok{(ielasisanas\_cels)}
\NormalTok{videjais}\OtherTok{=}\FunctionTok{global}\NormalTok{(slanis,}\AttributeTok{fun=}\StringTok{"mean"}\NormalTok{,}\AttributeTok{na.rm=}\ConstantTok{TRUE}\NormalTok{)}
\NormalTok{centrets}\OtherTok{=}\NormalTok{slanis}\SpecialCharTok{{-}}\NormalTok{videjais[,}\DecValTok{1}\NormalTok{]}
\NormalTok{standartnovirze}\OtherTok{=}\NormalTok{terra}\SpecialCharTok{::}\FunctionTok{global}\NormalTok{(centrets,}\AttributeTok{fun=}\StringTok{"rms"}\NormalTok{,}\AttributeTok{na.rm=}\ConstantTok{TRUE}\NormalTok{)}
\NormalTok{merogots}\OtherTok{=}\NormalTok{centrets}\SpecialCharTok{/}\NormalTok{standartnovirze[,}\DecValTok{1}\NormalTok{]}
\FunctionTok{writeRaster}\NormalTok{(merogots,}
      \AttributeTok{filename=}\NormalTok{saglabasanas\_cels,}
      \AttributeTok{overwrite=}\ConstantTok{TRUE}\NormalTok{)}
\end{Highlighting}
\end{Shaded}

\section{ForestsTreesAge\_MixedYoung\_r3000}\label{ch06.371}

\textbf{filename:} \texttt{ForestsTreesAge\_MixedYoung\_r3000.tif}

\textbf{layername:} \texttt{egv\_371}

\textbf{English name:} Fractional cover of Young (pre-rotation age) Mixed Forests
within the 3 km landscape

\textbf{Latvian name:} Jaunu (pirms cirtmeta) jauktu koku mežu platības īpatsvars 3
km ainavā

\textbf{Procedure:} The cover fraction within a radius of 3000 m around the analysis grid cell
is calculated as the area-weighted sum of the \hyperref[ch06.368]{analysis cells} inside
the buffer, using the workflow \texttt{egvtools::radius\_function()}. During the calculation of the landscape
metric, inverse distance weighted (power = 2) gap filling on the output is
applied to ensure no missing values at the edges. Then the layer is
rewritten to set its name. Finally, the layer is standardised by
subtracting the arithmetic mean and dividing by the root mean squared error.

\begin{Shaded}
\begin{Highlighting}[]
\CommentTok{\# libs {-}{-}{-}{-}}
\ControlFlowTok{if}\NormalTok{(}\SpecialCharTok{!}\FunctionTok{require}\NormalTok{(terra)) \{}\FunctionTok{install.packages}\NormalTok{(}\StringTok{"terra"}\NormalTok{); }\FunctionTok{require}\NormalTok{(terra)\}}
\ControlFlowTok{if}\NormalTok{(}\SpecialCharTok{!}\FunctionTok{require}\NormalTok{(egvtools)) \{remotes}\SpecialCharTok{::}\FunctionTok{install\_github}\NormalTok{(}\StringTok{"aavotins/egvtools"}\NormalTok{); }\FunctionTok{require}\NormalTok{(egvtools)\}}


\CommentTok{\# Templates {-}{-}{-}{-}{-}}
\NormalTok{template100}\OtherTok{=}\FunctionTok{rast}\NormalTok{(}\StringTok{"./Templates/TemplateRasters/LV100m\_10km.tif"}\NormalTok{)}

\CommentTok{\# radii {-}{-}{-}{-}}
\FunctionTok{radius\_function}\NormalTok{(}
 \AttributeTok{kvadrati\_path =} \StringTok{"./Templates/TemplateGrids/tiles/"}\NormalTok{,}
 \AttributeTok{radii\_path   =} \StringTok{"./Templates/TemplateGridPoints/tiles/"}\NormalTok{,}
 \AttributeTok{tikls100\_path =} \StringTok{"./Templates/TemplateGrids/tikls100\_sauzeme.parquet"}\NormalTok{,}
 \AttributeTok{template\_path =} \StringTok{"./Templates/TemplateRasters/LV100m\_10km.tif"}\NormalTok{,}
 \AttributeTok{input\_layers  =} \FunctionTok{c}\NormalTok{(}\StringTok{"./RasterGrids\_100m/2024/RAW/ForestsTreesAge\_MixedYoung\_cell.tif"}\NormalTok{),}
 \AttributeTok{layer\_prefixes =} \FunctionTok{c}\NormalTok{(}\StringTok{"ForestsTreesAge\_MixedYoung"}\NormalTok{),}
 \AttributeTok{output\_dir   =} \StringTok{"./RasterGrids\_100m/2024/RAW/"}\NormalTok{,}
 \AttributeTok{n\_workers   =} \DecValTok{6}\NormalTok{,}
 \AttributeTok{radii     =} \FunctionTok{c}\NormalTok{(}\StringTok{"r3000"}\NormalTok{),}
 \AttributeTok{radius\_mode  =} \StringTok{"sparse"}\NormalTok{,}
 \AttributeTok{extract\_fun  =} \StringTok{"mean"}\NormalTok{,}
 \AttributeTok{fill\_missing  =} \ConstantTok{TRUE}\NormalTok{,}
 \AttributeTok{IDW\_weight   =} \DecValTok{2}\NormalTok{,}
 \AttributeTok{future\_max\_size =} \DecValTok{40} \SpecialCharTok{*} \DecValTok{1024}\SpecialCharTok{\^{}}\DecValTok{3}\NormalTok{)}


\CommentTok{\# ForestsTreesAge\_MixedYoung\_r3000.tif  egv\_371}
\NormalTok{slanis}\OtherTok{=}\FunctionTok{rast}\NormalTok{(}\StringTok{"./RasterGrids\_100m/2024/RAW/ForestsTreesAge\_MixedYoung\_r3000.tif"}\NormalTok{)}
\FunctionTok{names}\NormalTok{(slanis)}\OtherTok{=}\StringTok{"egv\_371"}
\NormalTok{slanis2}\OtherTok{=}\FunctionTok{project}\NormalTok{(slanis,template100)}
\FunctionTok{writeRaster}\NormalTok{(slanis2,}
      \StringTok{"./RasterGrids\_100m/2024/RAW/ForestsTreesAge\_MixedYoung\_r3000.tif"}\NormalTok{,}
      \AttributeTok{overwrite=}\ConstantTok{TRUE}\NormalTok{)}

\CommentTok{\# standardisation {-}{-}{-}{-}}
\ControlFlowTok{if}\NormalTok{(}\SpecialCharTok{!}\FunctionTok{require}\NormalTok{(terra)) \{}\FunctionTok{install.packages}\NormalTok{(}\StringTok{"terra"}\NormalTok{); }\FunctionTok{require}\NormalTok{(terra)\}}
\ControlFlowTok{if}\NormalTok{(}\SpecialCharTok{!}\FunctionTok{require}\NormalTok{(tidyverse)) \{}\FunctionTok{install.packages}\NormalTok{(}\StringTok{"tidyverse"}\NormalTok{); }\FunctionTok{require}\NormalTok{(tidyverse)\}}

\NormalTok{nosaukums}\OtherTok{=}\StringTok{"ForestsTreesAge\_MixedYoung\_r3000.tif"}
\NormalTok{ielasisanas\_cels}\OtherTok{=}\FunctionTok{paste0}\NormalTok{(}\StringTok{"./RasterGrids\_100m/2024/RAW/"}\NormalTok{,nosaukums)}
\NormalTok{saglabasanas\_cels}\OtherTok{=}\FunctionTok{paste0}\NormalTok{(}\StringTok{"./RasterGrids\_100m/2024/Scaled/"}\NormalTok{,nosaukums)}
\NormalTok{slanis}\OtherTok{=}\FunctionTok{rast}\NormalTok{(ielasisanas\_cels)}
\NormalTok{videjais}\OtherTok{=}\FunctionTok{global}\NormalTok{(slanis,}\AttributeTok{fun=}\StringTok{"mean"}\NormalTok{,}\AttributeTok{na.rm=}\ConstantTok{TRUE}\NormalTok{)}
\NormalTok{centrets}\OtherTok{=}\NormalTok{slanis}\SpecialCharTok{{-}}\NormalTok{videjais[,}\DecValTok{1}\NormalTok{]}
\NormalTok{standartnovirze}\OtherTok{=}\NormalTok{terra}\SpecialCharTok{::}\FunctionTok{global}\NormalTok{(centrets,}\AttributeTok{fun=}\StringTok{"rms"}\NormalTok{,}\AttributeTok{na.rm=}\ConstantTok{TRUE}\NormalTok{)}
\NormalTok{merogots}\OtherTok{=}\NormalTok{centrets}\SpecialCharTok{/}\NormalTok{standartnovirze[,}\DecValTok{1}\NormalTok{]}
\FunctionTok{writeRaster}\NormalTok{(merogots,}
      \AttributeTok{filename=}\NormalTok{saglabasanas\_cels,}
      \AttributeTok{overwrite=}\ConstantTok{TRUE}\NormalTok{)}
\end{Highlighting}
\end{Shaded}

\section{ForestsTreesAge\_MixedYoung\_r10000}\label{ch06.372}

\textbf{filename:} \texttt{ForestsTreesAge\_MixedYoung\_r10000.tif}

\textbf{layername:} \texttt{egv\_372}

\textbf{English name:} Fractional cover of Young (pre-rotation age) Mixed Forests
within the 10 km landscape

\textbf{Latvian name:} Jaunu (pirms cirtmeta) jauktu koku mežu platības īpatsvars 10
km ainavā

\textbf{Procedure:} The cover fraction within a radius of 10000 m around the analysis grid cell
is calculated as the area-weighted sum of the \hyperref[ch06.368]{analysis cells} inside
the buffer, using the workflow \texttt{egvtools::radius\_function()}. During the calculation of the landscape
metric, inverse distance weighted (power = 2) gap filling on the output is
applied to ensure no missing values at the edges. Then the layer is
rewritten to set its name. Finally, the layer is standardised by
subtracting the arithmetic mean and dividing by the root mean squared error.

\begin{Shaded}
\begin{Highlighting}[]
\CommentTok{\# libs {-}{-}{-}{-}}
\ControlFlowTok{if}\NormalTok{(}\SpecialCharTok{!}\FunctionTok{require}\NormalTok{(terra)) \{}\FunctionTok{install.packages}\NormalTok{(}\StringTok{"terra"}\NormalTok{); }\FunctionTok{require}\NormalTok{(terra)\}}
\ControlFlowTok{if}\NormalTok{(}\SpecialCharTok{!}\FunctionTok{require}\NormalTok{(egvtools)) \{remotes}\SpecialCharTok{::}\FunctionTok{install\_github}\NormalTok{(}\StringTok{"aavotins/egvtools"}\NormalTok{); }\FunctionTok{require}\NormalTok{(egvtools)\}}


\CommentTok{\# Templates {-}{-}{-}{-}{-}}
\NormalTok{template100}\OtherTok{=}\FunctionTok{rast}\NormalTok{(}\StringTok{"./Templates/TemplateRasters/LV100m\_10km.tif"}\NormalTok{)}

\CommentTok{\# radii {-}{-}{-}{-}}
\FunctionTok{radius\_function}\NormalTok{(}
 \AttributeTok{kvadrati\_path =} \StringTok{"./Templates/TemplateGrids/tiles/"}\NormalTok{,}
 \AttributeTok{radii\_path   =} \StringTok{"./Templates/TemplateGridPoints/tiles/"}\NormalTok{,}
 \AttributeTok{tikls100\_path =} \StringTok{"./Templates/TemplateGrids/tikls100\_sauzeme.parquet"}\NormalTok{,}
 \AttributeTok{template\_path =} \StringTok{"./Templates/TemplateRasters/LV100m\_10km.tif"}\NormalTok{,}
 \AttributeTok{input\_layers  =} \FunctionTok{c}\NormalTok{(}\StringTok{"./RasterGrids\_100m/2024/RAW/ForestsTreesAge\_MixedYoung\_cell.tif"}\NormalTok{),}
 \AttributeTok{layer\_prefixes =} \FunctionTok{c}\NormalTok{(}\StringTok{"ForestsTreesAge\_MixedYoung"}\NormalTok{),}
 \AttributeTok{output\_dir   =} \StringTok{"./RasterGrids\_100m/2024/RAW/"}\NormalTok{,}
 \AttributeTok{n\_workers   =} \DecValTok{6}\NormalTok{,}
 \AttributeTok{radii     =} \FunctionTok{c}\NormalTok{(}\StringTok{"r10000"}\NormalTok{),}
 \AttributeTok{radius\_mode  =} \StringTok{"sparse"}\NormalTok{,}
 \AttributeTok{extract\_fun  =} \StringTok{"mean"}\NormalTok{,}
 \AttributeTok{fill\_missing  =} \ConstantTok{TRUE}\NormalTok{,}
 \AttributeTok{IDW\_weight   =} \DecValTok{2}\NormalTok{,}
 \AttributeTok{future\_max\_size =} \DecValTok{40} \SpecialCharTok{*} \DecValTok{1024}\SpecialCharTok{\^{}}\DecValTok{3}\NormalTok{)}


\CommentTok{\# ForestsTreesAge\_MixedYoung\_r10000.tif egv\_372}
\NormalTok{slanis}\OtherTok{=}\FunctionTok{rast}\NormalTok{(}\StringTok{"./RasterGrids\_100m/2024/RAW/ForestsTreesAge\_MixedYoung\_r10000.tif"}\NormalTok{)}
\FunctionTok{names}\NormalTok{(slanis)}\OtherTok{=}\StringTok{"egv\_372"}
\NormalTok{slanis2}\OtherTok{=}\FunctionTok{project}\NormalTok{(slanis,template100)}
\FunctionTok{writeRaster}\NormalTok{(slanis2,}
      \StringTok{"./RasterGrids\_100m/2024/RAW/ForestsTreesAge\_MixedYoung\_r10000.tif"}\NormalTok{,}
      \AttributeTok{overwrite=}\ConstantTok{TRUE}\NormalTok{)}

\CommentTok{\# standardisation {-}{-}{-}{-}}
\ControlFlowTok{if}\NormalTok{(}\SpecialCharTok{!}\FunctionTok{require}\NormalTok{(terra)) \{}\FunctionTok{install.packages}\NormalTok{(}\StringTok{"terra"}\NormalTok{); }\FunctionTok{require}\NormalTok{(terra)\}}
\ControlFlowTok{if}\NormalTok{(}\SpecialCharTok{!}\FunctionTok{require}\NormalTok{(tidyverse)) \{}\FunctionTok{install.packages}\NormalTok{(}\StringTok{"tidyverse"}\NormalTok{); }\FunctionTok{require}\NormalTok{(tidyverse)\}}

\NormalTok{nosaukums}\OtherTok{=}\StringTok{"ForestsTreesAge\_MixedYoung\_r10000.tif"}
\NormalTok{ielasisanas\_cels}\OtherTok{=}\FunctionTok{paste0}\NormalTok{(}\StringTok{"./RasterGrids\_100m/2024/RAW/"}\NormalTok{,nosaukums)}
\NormalTok{saglabasanas\_cels}\OtherTok{=}\FunctionTok{paste0}\NormalTok{(}\StringTok{"./RasterGrids\_100m/2024/Scaled/"}\NormalTok{,nosaukums)}
\NormalTok{slanis}\OtherTok{=}\FunctionTok{rast}\NormalTok{(ielasisanas\_cels)}
\NormalTok{videjais}\OtherTok{=}\FunctionTok{global}\NormalTok{(slanis,}\AttributeTok{fun=}\StringTok{"mean"}\NormalTok{,}\AttributeTok{na.rm=}\ConstantTok{TRUE}\NormalTok{)}
\NormalTok{centrets}\OtherTok{=}\NormalTok{slanis}\SpecialCharTok{{-}}\NormalTok{videjais[,}\DecValTok{1}\NormalTok{]}
\NormalTok{standartnovirze}\OtherTok{=}\NormalTok{terra}\SpecialCharTok{::}\FunctionTok{global}\NormalTok{(centrets,}\AttributeTok{fun=}\StringTok{"rms"}\NormalTok{,}\AttributeTok{na.rm=}\ConstantTok{TRUE}\NormalTok{)}
\NormalTok{merogots}\OtherTok{=}\NormalTok{centrets}\SpecialCharTok{/}\NormalTok{standartnovirze[,}\DecValTok{1}\NormalTok{]}
\FunctionTok{writeRaster}\NormalTok{(merogots,}
      \AttributeTok{filename=}\NormalTok{saglabasanas\_cels,}
      \AttributeTok{overwrite=}\ConstantTok{TRUE}\NormalTok{)}
\end{Highlighting}
\end{Shaded}

\section{ForestsTreesAge\_TemperateDeciduousOld\_cell}\label{ch06.373}

\textbf{filename:} \texttt{ForestsTreesAge\_TemperateDeciduousOld\_cell.tif}

\textbf{layername:} \texttt{egv\_373}

\textbf{English name:} Fractional cover of Old (over rotation age) Temperate
Deciduous Forests within the analysis cell (1 ha)

\textbf{Latvian name:} Vecu (kopš cirtmeta) platlapju mežu platības īpatsvars
analīzes šūnā (1 ha)

\textbf{Procedure:} Most EGVs describing forests are spatially restricted to areas outside
of clearcuts and dead stands. This mask is created using a combination of
the \hyperref[Ch04.01]{State Forest Service's
State Forest Registry} land category 12 and 14, and \hyperref[Ch04.09]{The
Global Forest Watch} pixels classified as lost tree canopy cover since
2020 (raster layer matching input, presence = 1, absence = 0).

To prepare this EGV, stands from the \hyperref[Ch04.01]{State Forest Service's State Forest
Registry} are classified into (in order):

\begin{itemize}
\item
  coniferous (see \hyperref[Ch01]{Terminology and acronyms} for species codes) if
  timber volume of those species exceeded 75\%;
\item
  Boreal deciduous if timber volume of those species exceeded 75\%;
\item
  temperate deciduous if timber volume of those species exceeded 50\%;
\item
  mixed otherwise;
\end{itemize}

then temperate deciduous stands exceeding the legal rotation age are selected
and geometries are rasterised (presence = 1, NA otherwise). Rasterisation is
performed using the workflow \texttt{egvtools::polygon2input()}, restricting to pixels outside clearcut
mask and covering background with value 0. The resulting layer
is then aggregated to EGV resolution using the workflow \texttt{egvtools::input2egv()}, which
calculates the arithmetic mean to determine the cover fraction. During
aggregation, inverse distance weighted (power = 2) gap filling on the output is
applied to ensure no missing values at the edges. Finally, the layer is
standardised by subtracting the arithmetic mean and dividing by the root mean squared
error.

\begin{Shaded}
\begin{Highlighting}[]
\CommentTok{\# libs {-}{-}{-}{-}}
\ControlFlowTok{if}\NormalTok{(}\SpecialCharTok{!}\FunctionTok{require}\NormalTok{(egvtools)) \{remotes}\SpecialCharTok{::}\FunctionTok{install\_github}\NormalTok{(}\StringTok{"aavotins/egvtools"}\NormalTok{); }\FunctionTok{require}\NormalTok{(egvtools)\}}
\ControlFlowTok{if}\NormalTok{(}\SpecialCharTok{!}\FunctionTok{require}\NormalTok{(terra)) \{}\FunctionTok{install.packages}\NormalTok{(}\StringTok{"terra"}\NormalTok{); }\FunctionTok{require}\NormalTok{(terra)\}}
\ControlFlowTok{if}\NormalTok{(}\SpecialCharTok{!}\FunctionTok{require}\NormalTok{(sf)) \{}\FunctionTok{install.packages}\NormalTok{(}\StringTok{"sf"}\NormalTok{); }\FunctionTok{require}\NormalTok{(sf)\}}
\ControlFlowTok{if}\NormalTok{(}\SpecialCharTok{!}\FunctionTok{require}\NormalTok{(tidyverse)) \{}\FunctionTok{install.packages}\NormalTok{(}\StringTok{"tidyverse"}\NormalTok{); }\FunctionTok{require}\NormalTok{(tidyverse)\}}
\ControlFlowTok{if}\NormalTok{(}\SpecialCharTok{!}\FunctionTok{require}\NormalTok{(sfarrow)) \{}\FunctionTok{install.packages}\NormalTok{(}\StringTok{"sfarrow"}\NormalTok{); }\FunctionTok{require}\NormalTok{(sfarrow)\}}
\ControlFlowTok{if}\NormalTok{(}\SpecialCharTok{!}\FunctionTok{require}\NormalTok{(readxl)) \{}\FunctionTok{install.packages}\NormalTok{(}\StringTok{"readxl"}\NormalTok{); }\FunctionTok{require}\NormalTok{(readxl)\}}
\ControlFlowTok{if}\NormalTok{(}\SpecialCharTok{!}\FunctionTok{require}\NormalTok{(raster)) \{}\FunctionTok{install.packages}\NormalTok{(}\StringTok{"raster"}\NormalTok{); }\FunctionTok{require}\NormalTok{(raster)\}}
\ControlFlowTok{if}\NormalTok{(}\SpecialCharTok{!}\FunctionTok{require}\NormalTok{(fasterize)) \{}\FunctionTok{install.packages}\NormalTok{(}\StringTok{"fasterize"}\NormalTok{); }\FunctionTok{require}\NormalTok{(fasterize)\}}

\CommentTok{\# templates {-}{-}{-}{-}}
\NormalTok{template100}\OtherTok{=}\FunctionTok{rast}\NormalTok{(}\StringTok{"./Templates/TemplateRasters/LV100m\_10km.tif"}\NormalTok{)}
\NormalTok{template10}\OtherTok{=}\FunctionTok{rast}\NormalTok{(}\StringTok{"./Templates/TemplateRasters/LV10m\_10km.tif"}\NormalTok{)}
\NormalTok{rastrs10}\OtherTok{=}\FunctionTok{raster}\NormalTok{(template10)}

\NormalTok{nulls10}\OtherTok{=}\FunctionTok{rast}\NormalTok{(}\StringTok{"./Templates/TemplateRasters/nulls\_LV10m\_10km.tif"}\NormalTok{)}
\NormalTok{nulls100}\OtherTok{=}\FunctionTok{rast}\NormalTok{(}\StringTok{"./Templates/TemplateRasters/nulls\_LV100m\_10km.tif"}\NormalTok{)}


\CommentTok{\# simple landscape {-}{-}{-}{-}}
\NormalTok{simple\_landscape}\OtherTok{=}\FunctionTok{rast}\NormalTok{(}\StringTok{"RasterGrids\_10m/2024/Ainava\_vienk\_mask.tif"}\NormalTok{)}

\CommentTok{\# mvr {-}{-}{-}{-}}
\NormalTok{mvr}\OtherTok{=}\FunctionTok{st\_read\_parquet}\NormalTok{(}\StringTok{"./Geodata/2024/MVR/nogabali\_2024janv.parquet"}\NormalTok{)}
\NormalTok{mvr}\SpecialCharTok{$}\NormalTok{yes}\OtherTok{=}\DecValTok{1}

\CommentTok{\# clear cut mask {-}{-}{-}{-}}
\NormalTok{izcirtumi}\OtherTok{=}\NormalTok{mvr }\SpecialCharTok{\%\textgreater{}\%} 
 \FunctionTok{filter}\NormalTok{(zkat }\SpecialCharTok{\%in\%} \FunctionTok{c}\NormalTok{(}\StringTok{"12"}\NormalTok{,}\StringTok{"14"}\NormalTok{)) }\SpecialCharTok{\%\textgreater{}\%} 
\NormalTok{ dplyr}\SpecialCharTok{::}\FunctionTok{select}\NormalTok{(yes)}
\NormalTok{r\_izcirtumi\_mvr}\OtherTok{=}\FunctionTok{fasterize}\NormalTok{(izcirtumi,rastrs10,}\AttributeTok{field=}\StringTok{"yes"}\NormalTok{)}
\NormalTok{t\_izcirtumi\_mvr}\OtherTok{=}\FunctionTok{rast}\NormalTok{(r\_izcirtumi\_mvr)}
\FunctionTok{plot}\NormalTok{(t\_izcirtumi\_mvr)}

\NormalTok{tcl}\OtherTok{=}\FunctionTok{rast}\NormalTok{(}\StringTok{"./Geodata/2024/Trees/GFW/TreeCoverLoss\_v1\_12.tif"}\NormalTok{)}
\NormalTok{tcl2}\OtherTok{=}\FunctionTok{ifel}\NormalTok{(tcl}\SpecialCharTok{\textless{}}\DecValTok{20}\NormalTok{,}\DecValTok{0}\NormalTok{,}\DecValTok{1}\NormalTok{)}
\NormalTok{tclX}\OtherTok{=}\FunctionTok{cover}\NormalTok{(tcl2,nulls10)}
\FunctionTok{plot}\NormalTok{(tclX)}

\NormalTok{clearcut\_mask}\OtherTok{=}\FunctionTok{cover}\NormalTok{(t\_izcirtumi\_mvr,tclX,}
          \AttributeTok{filename=}\StringTok{"./RasterGrids\_10m/2024/Mask\_clearcuts.tif"}\NormalTok{,}
          \AttributeTok{overwrite=}\ConstantTok{TRUE}\NormalTok{)}
\FunctionTok{plot}\NormalTok{(clearcut\_mask)}

\FunctionTok{rm}\NormalTok{(izcirtumi)}
\FunctionTok{rm}\NormalTok{(r\_izcirtumi\_mvr)}
\FunctionTok{rm}\NormalTok{(t\_izcirtumi\_mvr)}
\FunctionTok{rm}\NormalTok{(tcl)}
\FunctionTok{rm}\NormalTok{(tcl2)}
\FunctionTok{rm}\NormalTok{(tclX)}

\CommentTok{\# ForestsTreesAge\_TemperateDeciduousOld\_cell.tif    egv\_373 {-}{-}{-}{-}}
\NormalTok{skujkoki}\OtherTok{=}\FunctionTok{c}\NormalTok{(}\StringTok{"1"}\NormalTok{,}\StringTok{"3"}\NormalTok{,}\StringTok{"13"}\NormalTok{,}\StringTok{"14"}\NormalTok{,}\StringTok{"15"}\NormalTok{,}\StringTok{"22"}\NormalTok{,}\StringTok{"23"}\NormalTok{,}\StringTok{"28"}\NormalTok{) }\CommentTok{\# 8}
\NormalTok{saurlapji}\OtherTok{=}\FunctionTok{c}\NormalTok{(}\StringTok{"4"}\NormalTok{,}\StringTok{"6"}\NormalTok{,}\StringTok{"8"}\NormalTok{,}\StringTok{"9"}\NormalTok{,}\StringTok{"19"}\NormalTok{,}\StringTok{"20"}\NormalTok{,}\StringTok{"21"}\NormalTok{,}\StringTok{"32"}\NormalTok{,}\StringTok{"35"}\NormalTok{,}\StringTok{"68"}\NormalTok{) }\CommentTok{\# 10}
\NormalTok{platlapji}\OtherTok{=}\FunctionTok{c}\NormalTok{(}\StringTok{"10"}\NormalTok{,}\StringTok{"11"}\NormalTok{,}\StringTok{"12"}\NormalTok{,}\StringTok{"16"}\NormalTok{,}\StringTok{"17"}\NormalTok{,}\StringTok{"18"}\NormalTok{,}\StringTok{"24"}\NormalTok{,}\StringTok{"25"}\NormalTok{,}\StringTok{"26"}\NormalTok{,}\StringTok{"27"}\NormalTok{,}\StringTok{"28"}\NormalTok{,}\StringTok{"29"}\NormalTok{,}\StringTok{"50"}\NormalTok{,}
      \StringTok{"61"}\NormalTok{,}\StringTok{"62"}\NormalTok{,}\StringTok{"63"}\NormalTok{,}\StringTok{"64"}\NormalTok{,}\StringTok{"65"}\NormalTok{,}\StringTok{"66"}\NormalTok{,}\StringTok{"67"}\NormalTok{,}\StringTok{"69"}\NormalTok{) }\CommentTok{\# 21}
\NormalTok{mvr}\OtherTok{=}\NormalTok{mvr }\SpecialCharTok{\%\textgreater{}\%} 
 \FunctionTok{mutate}\NormalTok{(}\AttributeTok{kraja\_skujkoku=}\FunctionTok{ifelse}\NormalTok{(s10 }\SpecialCharTok{\%in\%}\NormalTok{ skujkoki,v10,}\DecValTok{0}\NormalTok{)}\SpecialCharTok{+}
      \FunctionTok{ifelse}\NormalTok{(s11 }\SpecialCharTok{\%in\%}\NormalTok{ skujkoki,v11,}\DecValTok{0}\NormalTok{)}\SpecialCharTok{+}\FunctionTok{ifelse}\NormalTok{(s12 }\SpecialCharTok{\%in\%}\NormalTok{ skujkoki,v12,}\DecValTok{0}\NormalTok{)}\SpecialCharTok{+}
      \FunctionTok{ifelse}\NormalTok{(s13 }\SpecialCharTok{\%in\%}\NormalTok{ skujkoki,v13,}\DecValTok{0}\NormalTok{)}\SpecialCharTok{+}\FunctionTok{ifelse}\NormalTok{(s14 }\SpecialCharTok{\%in\%}\NormalTok{ skujkoki,v14,}\DecValTok{0}\NormalTok{),}
     \AttributeTok{kraja\_saurlapju=}\FunctionTok{ifelse}\NormalTok{(s10 }\SpecialCharTok{\%in\%}\NormalTok{ saurlapji,v10,}\DecValTok{0}\NormalTok{)}\SpecialCharTok{+}
      \FunctionTok{ifelse}\NormalTok{(s11 }\SpecialCharTok{\%in\%}\NormalTok{ saurlapji,v11,}\DecValTok{0}\NormalTok{)}\SpecialCharTok{+}\FunctionTok{ifelse}\NormalTok{(s12 }\SpecialCharTok{\%in\%}\NormalTok{ saurlapji,v12,}\DecValTok{0}\NormalTok{)}\SpecialCharTok{+}
      \FunctionTok{ifelse}\NormalTok{(s13 }\SpecialCharTok{\%in\%}\NormalTok{ saurlapji,v13,}\DecValTok{0}\NormalTok{)}\SpecialCharTok{+}\FunctionTok{ifelse}\NormalTok{(s14 }\SpecialCharTok{\%in\%}\NormalTok{ saurlapji,v14,}\DecValTok{0}\NormalTok{),}
     \AttributeTok{kraja\_platlapju=}\FunctionTok{ifelse}\NormalTok{(s10 }\SpecialCharTok{\%in\%}\NormalTok{ platlapji,v10,}\DecValTok{0}\NormalTok{)}\SpecialCharTok{+}
      \FunctionTok{ifelse}\NormalTok{(s11 }\SpecialCharTok{\%in\%}\NormalTok{ platlapji,v11,}\DecValTok{0}\NormalTok{)}\SpecialCharTok{+}\FunctionTok{ifelse}\NormalTok{(s12 }\SpecialCharTok{\%in\%}\NormalTok{ platlapji,v12,}\DecValTok{0}\NormalTok{)}\SpecialCharTok{+}
      \FunctionTok{ifelse}\NormalTok{(s13 }\SpecialCharTok{\%in\%}\NormalTok{ platlapji,v13,}\DecValTok{0}\NormalTok{)}\SpecialCharTok{+}\FunctionTok{ifelse}\NormalTok{(s14 }\SpecialCharTok{\%in\%}\NormalTok{ platlapji,v14,}\DecValTok{0}\NormalTok{)) }\SpecialCharTok{\%\textgreater{}\%} 
 \FunctionTok{mutate}\NormalTok{(}\AttributeTok{kopeja\_kraja=}\NormalTok{kraja\_skujkoku}\SpecialCharTok{+}\NormalTok{kraja\_platlapju}\SpecialCharTok{+}\NormalTok{kraja\_saurlapju) }\SpecialCharTok{\%\textgreater{}\%} 
 \FunctionTok{mutate}\NormalTok{(}\AttributeTok{tips=}\FunctionTok{ifelse}\NormalTok{(kraja\_skujkoku}\SpecialCharTok{/}\NormalTok{kopeja\_kraja}\SpecialCharTok{\textgreater{}=}\FloatTok{0.75}\NormalTok{,}\StringTok{"skujkoku"}\NormalTok{,}
           \FunctionTok{ifelse}\NormalTok{(kraja\_saurlapju}\SpecialCharTok{/}\NormalTok{kopeja\_kraja}\SpecialCharTok{\textgreater{}=}\FloatTok{0.75}\NormalTok{,}\StringTok{"saurlapju"}\NormalTok{,}
              \FunctionTok{ifelse}\NormalTok{(kraja\_platlapju}\SpecialCharTok{/}\NormalTok{kopeja\_kraja}\SpecialCharTok{\textgreater{}}\FloatTok{0.5}\NormalTok{,}\StringTok{"platlapju"}\NormalTok{,}
                  \StringTok{"jauktu koku"}\NormalTok{))))}
\NormalTok{nogabali}\OtherTok{=}\NormalTok{mvr }\SpecialCharTok{\%\textgreater{}\%} 
 \FunctionTok{filter}\NormalTok{(zkat}\SpecialCharTok{==}\StringTok{"10"}\SpecialCharTok{\&}\NormalTok{tips}\SpecialCharTok{==}\StringTok{"platlapju"}\SpecialCharTok{\&}\NormalTok{(vgr}\SpecialCharTok{==}\StringTok{"4"}\SpecialCharTok{|}\NormalTok{vgr}\SpecialCharTok{==}\StringTok{"5"}\NormalTok{))}

\NormalTok{p2i\_rez}\OtherTok{=}\NormalTok{egvtools}\SpecialCharTok{::}\FunctionTok{polygon2input}\NormalTok{(}\AttributeTok{vector\_data =}\NormalTok{ nogabali,}
                \AttributeTok{template\_path =} \StringTok{"./Templates/TemplateRasters/LV10m\_10km.tif"}\NormalTok{,}
                \AttributeTok{out\_path =} \StringTok{"./RasterGrids\_10m/2024/"}\NormalTok{,}
                \AttributeTok{file\_name =} \StringTok{"ForestsTreesAge\_TemperateDeciduousOld\_input.tif"}\NormalTok{,}
                \AttributeTok{value\_field =} \StringTok{"yes"}\NormalTok{,}
                \AttributeTok{restrict\_to =}\NormalTok{ clearcut\_mask,}
                \AttributeTok{restrict\_values =} \DecValTok{0}\NormalTok{,}
                \AttributeTok{prepare=}\ConstantTok{FALSE}\NormalTok{,}
                \AttributeTok{background\_raster =} \StringTok{"./Templates/TemplateRasters/nulls\_LV10m\_10km.tif"}\NormalTok{,}
                \AttributeTok{plot\_result =} \ConstantTok{TRUE}\NormalTok{)}
\NormalTok{p2i\_rez}
\NormalTok{i2e\_rez}\OtherTok{=}\NormalTok{egvtools}\SpecialCharTok{::}\FunctionTok{input2egv}\NormalTok{(}\AttributeTok{input=}\FunctionTok{paste0}\NormalTok{(}\StringTok{"./RasterGrids\_10m/2024/"}\NormalTok{,}
                     \StringTok{"ForestsTreesAge\_TemperateDeciduousOld\_input.tif"}\NormalTok{),}
              \AttributeTok{egv\_template=} \StringTok{"./Templates/TemplateRasters/LV100m\_10km.tif"}\NormalTok{,}
              \AttributeTok{summary\_function =} \StringTok{"average"}\NormalTok{,}
              \AttributeTok{missing\_job =} \StringTok{"FillOutput"}\NormalTok{,}
              \AttributeTok{outlocation =} \StringTok{"./RasterGrids\_100m/2024/RAW/"}\NormalTok{,}
              \AttributeTok{outfilename =} \StringTok{"ForestsTreesAge\_TemperateDeciduousOld\_cell.tif"}\NormalTok{,}
              \AttributeTok{layername =} \StringTok{"egv\_373"}\NormalTok{,}
              \AttributeTok{idw\_weight =} \DecValTok{2}\NormalTok{,}
              \AttributeTok{plot\_gaps =} \ConstantTok{FALSE}\NormalTok{,}\AttributeTok{plot\_final =} \ConstantTok{TRUE}\NormalTok{)}
\NormalTok{i2e\_rez}
\FunctionTok{rm}\NormalTok{(nogabali)}
\FunctionTok{rm}\NormalTok{(p2i\_rez)}
\FunctionTok{rm}\NormalTok{(i2e\_rez)}
\FunctionTok{unlink}\NormalTok{(}\StringTok{"./RasterGrids\_10m/2024/ForestsTreesAge\_TemperateDeciduousOld\_input.tif"}\NormalTok{)}

\CommentTok{\# standardisation {-}{-}{-}{-}}
\ControlFlowTok{if}\NormalTok{(}\SpecialCharTok{!}\FunctionTok{require}\NormalTok{(terra)) \{}\FunctionTok{install.packages}\NormalTok{(}\StringTok{"terra"}\NormalTok{); }\FunctionTok{require}\NormalTok{(terra)\}}
\ControlFlowTok{if}\NormalTok{(}\SpecialCharTok{!}\FunctionTok{require}\NormalTok{(tidyverse)) \{}\FunctionTok{install.packages}\NormalTok{(}\StringTok{"tidyverse"}\NormalTok{); }\FunctionTok{require}\NormalTok{(tidyverse)\}}

\NormalTok{nosaukums}\OtherTok{=}\StringTok{"ForestsTreesAge\_TemperateDeciduousOld\_cell.tif"}
\NormalTok{ielasisanas\_cels}\OtherTok{=}\FunctionTok{paste0}\NormalTok{(}\StringTok{"./RasterGrids\_100m/2024/RAW/"}\NormalTok{,nosaukums)}
\NormalTok{saglabasanas\_cels}\OtherTok{=}\FunctionTok{paste0}\NormalTok{(}\StringTok{"./RasterGrids\_100m/2024/Scaled/"}\NormalTok{,nosaukums)}
\NormalTok{slanis}\OtherTok{=}\FunctionTok{rast}\NormalTok{(ielasisanas\_cels)}
\NormalTok{videjais}\OtherTok{=}\FunctionTok{global}\NormalTok{(slanis,}\AttributeTok{fun=}\StringTok{"mean"}\NormalTok{,}\AttributeTok{na.rm=}\ConstantTok{TRUE}\NormalTok{)}
\NormalTok{centrets}\OtherTok{=}\NormalTok{slanis}\SpecialCharTok{{-}}\NormalTok{videjais[,}\DecValTok{1}\NormalTok{]}
\NormalTok{standartnovirze}\OtherTok{=}\NormalTok{terra}\SpecialCharTok{::}\FunctionTok{global}\NormalTok{(centrets,}\AttributeTok{fun=}\StringTok{"rms"}\NormalTok{,}\AttributeTok{na.rm=}\ConstantTok{TRUE}\NormalTok{)}
\NormalTok{merogots}\OtherTok{=}\NormalTok{centrets}\SpecialCharTok{/}\NormalTok{standartnovirze[,}\DecValTok{1}\NormalTok{]}
\FunctionTok{writeRaster}\NormalTok{(merogots,}
      \AttributeTok{filename=}\NormalTok{saglabasanas\_cels,}
      \AttributeTok{overwrite=}\ConstantTok{TRUE}\NormalTok{)}
\end{Highlighting}
\end{Shaded}

\section{ForestsTreesAge\_TemperateDeciduousOld\_r500}\label{ch06.374}

\textbf{filename:} \texttt{ForestsTreesAge\_TemperateDeciduousOld\_r500.tif}

\textbf{layername:} \texttt{egv\_374}

\textbf{English name:} Fractional cover of Old (over rotation age) Temperate
Deciduous Forests within the 0.5 km landscape

\textbf{Latvian name:} Vecu (kopš cirtmeta) platlapju mežu platības īpatsvars 0,5 km
ainavā

\textbf{Procedure:} The cover fraction within a radius of 500 m around the analysis grid cell is
calculated as the area-weighted sum of the \hyperref[ch06.373]{analysis cells} inside the
buffer, using the workflow \texttt{egvtools::radius\_function()}. During the calculation of the landscape metric,
inverse distance weighted (power = 2) gap filling on the output is applied
to ensure no missing values at the edges. Then the layer is rewritten to set
its name. Finally, the layer is standardised by subtracting the arithmetic
mean and dividing by the root mean squared error.

\begin{Shaded}
\begin{Highlighting}[]
\CommentTok{\# libs {-}{-}{-}{-}}
\ControlFlowTok{if}\NormalTok{(}\SpecialCharTok{!}\FunctionTok{require}\NormalTok{(terra)) \{}\FunctionTok{install.packages}\NormalTok{(}\StringTok{"terra"}\NormalTok{); }\FunctionTok{require}\NormalTok{(terra)\}}
\ControlFlowTok{if}\NormalTok{(}\SpecialCharTok{!}\FunctionTok{require}\NormalTok{(egvtools)) \{remotes}\SpecialCharTok{::}\FunctionTok{install\_github}\NormalTok{(}\StringTok{"aavotins/egvtools"}\NormalTok{); }\FunctionTok{require}\NormalTok{(egvtools)\}}


\CommentTok{\# Templates {-}{-}{-}{-}{-}}
\NormalTok{template100}\OtherTok{=}\FunctionTok{rast}\NormalTok{(}\StringTok{"./Templates/TemplateRasters/LV100m\_10km.tif"}\NormalTok{)}

\CommentTok{\# radii {-}{-}{-}{-}}
\FunctionTok{radius\_function}\NormalTok{(}
 \AttributeTok{kvadrati\_path =} \StringTok{"./Templates/TemplateGrids/tiles/"}\NormalTok{,}
 \AttributeTok{radii\_path   =} \StringTok{"./Templates/TemplateGridPoints/tiles/"}\NormalTok{,}
 \AttributeTok{tikls100\_path =} \StringTok{"./Templates/TemplateGrids/tikls100\_sauzeme.parquet"}\NormalTok{,}
 \AttributeTok{template\_path =} \StringTok{"./Templates/TemplateRasters/LV100m\_10km.tif"}\NormalTok{,}
 \AttributeTok{input\_layers  =} \FunctionTok{c}\NormalTok{(}\StringTok{"./RasterGrids\_100m/2024/RAW/ForestsTreesAge\_TemperateDeciduousOld\_cell.tif"}\NormalTok{),}
 \AttributeTok{layer\_prefixes =} \FunctionTok{c}\NormalTok{(}\StringTok{"ForestsTreesAge\_TemperateDeciduousOld"}\NormalTok{),}
 \AttributeTok{output\_dir   =} \StringTok{"./RasterGrids\_100m/2024/RAW/"}\NormalTok{,}
 \AttributeTok{n\_workers   =} \DecValTok{6}\NormalTok{,}
 \AttributeTok{radii     =} \FunctionTok{c}\NormalTok{(}\StringTok{"r500"}\NormalTok{),}
 \AttributeTok{radius\_mode  =} \StringTok{"sparse"}\NormalTok{,}
 \AttributeTok{extract\_fun  =} \StringTok{"mean"}\NormalTok{,}
 \AttributeTok{fill\_missing  =} \ConstantTok{TRUE}\NormalTok{,}
 \AttributeTok{IDW\_weight   =} \DecValTok{2}\NormalTok{,}
 \AttributeTok{future\_max\_size =} \DecValTok{40} \SpecialCharTok{*} \DecValTok{1024}\SpecialCharTok{\^{}}\DecValTok{3}\NormalTok{)}


\CommentTok{\# ForestsTreesAge\_TemperateDeciduousOld\_r500.tif    egv\_374}
\NormalTok{slanis}\OtherTok{=}\FunctionTok{rast}\NormalTok{(}\StringTok{"./RasterGrids\_100m/2024/RAW/ForestsTreesAge\_TemperateDeciduousOld\_r500.tif"}\NormalTok{)}
\FunctionTok{names}\NormalTok{(slanis)}\OtherTok{=}\StringTok{"egv\_374"}
\NormalTok{slanis2}\OtherTok{=}\FunctionTok{project}\NormalTok{(slanis,template100)}
\FunctionTok{writeRaster}\NormalTok{(slanis2,}
      \StringTok{"./RasterGrids\_100m/2024/RAW/ForestsTreesAge\_TemperateDeciduousOld\_r500.tif"}\NormalTok{,}
      \AttributeTok{overwrite=}\ConstantTok{TRUE}\NormalTok{)}

\CommentTok{\# standardisation {-}{-}{-}{-}}
\ControlFlowTok{if}\NormalTok{(}\SpecialCharTok{!}\FunctionTok{require}\NormalTok{(terra)) \{}\FunctionTok{install.packages}\NormalTok{(}\StringTok{"terra"}\NormalTok{); }\FunctionTok{require}\NormalTok{(terra)\}}
\ControlFlowTok{if}\NormalTok{(}\SpecialCharTok{!}\FunctionTok{require}\NormalTok{(tidyverse)) \{}\FunctionTok{install.packages}\NormalTok{(}\StringTok{"tidyverse"}\NormalTok{); }\FunctionTok{require}\NormalTok{(tidyverse)\}}

\NormalTok{nosaukums}\OtherTok{=}\StringTok{"ForestsTreesAge\_TemperateDeciduousOld\_r500.tif"}
\NormalTok{ielasisanas\_cels}\OtherTok{=}\FunctionTok{paste0}\NormalTok{(}\StringTok{"./RasterGrids\_100m/2024/RAW/"}\NormalTok{,nosaukums)}
\NormalTok{saglabasanas\_cels}\OtherTok{=}\FunctionTok{paste0}\NormalTok{(}\StringTok{"./RasterGrids\_100m/2024/Scaled/"}\NormalTok{,nosaukums)}
\NormalTok{slanis}\OtherTok{=}\FunctionTok{rast}\NormalTok{(ielasisanas\_cels)}
\NormalTok{videjais}\OtherTok{=}\FunctionTok{global}\NormalTok{(slanis,}\AttributeTok{fun=}\StringTok{"mean"}\NormalTok{,}\AttributeTok{na.rm=}\ConstantTok{TRUE}\NormalTok{)}
\NormalTok{centrets}\OtherTok{=}\NormalTok{slanis}\SpecialCharTok{{-}}\NormalTok{videjais[,}\DecValTok{1}\NormalTok{]}
\NormalTok{standartnovirze}\OtherTok{=}\NormalTok{terra}\SpecialCharTok{::}\FunctionTok{global}\NormalTok{(centrets,}\AttributeTok{fun=}\StringTok{"rms"}\NormalTok{,}\AttributeTok{na.rm=}\ConstantTok{TRUE}\NormalTok{)}
\NormalTok{merogots}\OtherTok{=}\NormalTok{centrets}\SpecialCharTok{/}\NormalTok{standartnovirze[,}\DecValTok{1}\NormalTok{]}
\FunctionTok{writeRaster}\NormalTok{(merogots,}
      \AttributeTok{filename=}\NormalTok{saglabasanas\_cels,}
      \AttributeTok{overwrite=}\ConstantTok{TRUE}\NormalTok{)}
\end{Highlighting}
\end{Shaded}

\section{ForestsTreesAge\_TemperateDeciduousOld\_r1250}\label{ch06.375}

\textbf{filename:} \texttt{ForestsTreesAge\_TemperateDeciduousOld\_r1250.tif}

\textbf{layername:} \texttt{egv\_375}

\textbf{English name:} Fractional cover of Old (over rotation age) Temperate
Deciduous Forests within the 1.25 km landscape

\textbf{Latvian name:} Vecu (kopš cirtmeta) platlapju mežu platības īpatsvars 1,25 km
ainavā

\textbf{Procedure:} The cover fraction within a radius of 1250 m around the analysis grid cell
is calculated as the area-weighted sum of the \hyperref[ch06.373]{analysis cells} inside
the buffer, using the workflow \texttt{egvtools::radius\_function()}. During the calculation of the landscape
metric, inverse distance weighted (power = 2) gap filling on the output is
applied to ensure no missing values at the edges. Then the layer is
rewritten to set its name. Finally, the layer is standardised by
subtracting the arithmetic mean and dividing by the root mean squared error.

\begin{Shaded}
\begin{Highlighting}[]
\CommentTok{\# libs {-}{-}{-}{-}}
\ControlFlowTok{if}\NormalTok{(}\SpecialCharTok{!}\FunctionTok{require}\NormalTok{(terra)) \{}\FunctionTok{install.packages}\NormalTok{(}\StringTok{"terra"}\NormalTok{); }\FunctionTok{require}\NormalTok{(terra)\}}
\ControlFlowTok{if}\NormalTok{(}\SpecialCharTok{!}\FunctionTok{require}\NormalTok{(egvtools)) \{remotes}\SpecialCharTok{::}\FunctionTok{install\_github}\NormalTok{(}\StringTok{"aavotins/egvtools"}\NormalTok{); }\FunctionTok{require}\NormalTok{(egvtools)\}}


\CommentTok{\# Templates {-}{-}{-}{-}{-}}
\NormalTok{template100}\OtherTok{=}\FunctionTok{rast}\NormalTok{(}\StringTok{"./Templates/TemplateRasters/LV100m\_10km.tif"}\NormalTok{)}

\CommentTok{\# radii {-}{-}{-}{-}}
\FunctionTok{radius\_function}\NormalTok{(}
 \AttributeTok{kvadrati\_path =} \StringTok{"./Templates/TemplateGrids/tiles/"}\NormalTok{,}
 \AttributeTok{radii\_path   =} \StringTok{"./Templates/TemplateGridPoints/tiles/"}\NormalTok{,}
 \AttributeTok{tikls100\_path =} \StringTok{"./Templates/TemplateGrids/tikls100\_sauzeme.parquet"}\NormalTok{,}
 \AttributeTok{template\_path =} \StringTok{"./Templates/TemplateRasters/LV100m\_10km.tif"}\NormalTok{,}
 \AttributeTok{input\_layers  =} \FunctionTok{c}\NormalTok{(}\StringTok{"./RasterGrids\_100m/2024/RAW/ForestsTreesAge\_TemperateDeciduousOld\_cell.tif"}\NormalTok{),}
 \AttributeTok{layer\_prefixes =} \FunctionTok{c}\NormalTok{(}\StringTok{"ForestsTreesAge\_TemperateDeciduousOld"}\NormalTok{),}
 \AttributeTok{output\_dir   =} \StringTok{"./RasterGrids\_100m/2024/RAW/"}\NormalTok{,}
 \AttributeTok{n\_workers   =} \DecValTok{6}\NormalTok{,}
 \AttributeTok{radii     =} \FunctionTok{c}\NormalTok{(}\StringTok{"r1250"}\NormalTok{),}
 \AttributeTok{radius\_mode  =} \StringTok{"sparse"}\NormalTok{,}
 \AttributeTok{extract\_fun  =} \StringTok{"mean"}\NormalTok{,}
 \AttributeTok{fill\_missing  =} \ConstantTok{TRUE}\NormalTok{,}
 \AttributeTok{IDW\_weight   =} \DecValTok{2}\NormalTok{,}
 \AttributeTok{future\_max\_size =} \DecValTok{40} \SpecialCharTok{*} \DecValTok{1024}\SpecialCharTok{\^{}}\DecValTok{3}\NormalTok{)}


\CommentTok{\# ForestsTreesAge\_TemperateDeciduousOld\_r1250.tif   egv\_375}
\NormalTok{slanis}\OtherTok{=}\FunctionTok{rast}\NormalTok{(}\StringTok{"./RasterGrids\_100m/2024/RAW/ForestsTreesAge\_TemperateDeciduousOld\_r1250.tif"}\NormalTok{)}
\FunctionTok{names}\NormalTok{(slanis)}\OtherTok{=}\StringTok{"egv\_375"}
\NormalTok{slanis2}\OtherTok{=}\FunctionTok{project}\NormalTok{(slanis,template100)}
\FunctionTok{writeRaster}\NormalTok{(slanis2,}
      \StringTok{"./RasterGrids\_100m/2024/RAW/ForestsTreesAge\_TemperateDeciduousOld\_r1250.tif"}\NormalTok{,}
      \AttributeTok{overwrite=}\ConstantTok{TRUE}\NormalTok{)}

\CommentTok{\# standardisation {-}{-}{-}{-}}
\ControlFlowTok{if}\NormalTok{(}\SpecialCharTok{!}\FunctionTok{require}\NormalTok{(terra)) \{}\FunctionTok{install.packages}\NormalTok{(}\StringTok{"terra"}\NormalTok{); }\FunctionTok{require}\NormalTok{(terra)\}}
\ControlFlowTok{if}\NormalTok{(}\SpecialCharTok{!}\FunctionTok{require}\NormalTok{(tidyverse)) \{}\FunctionTok{install.packages}\NormalTok{(}\StringTok{"tidyverse"}\NormalTok{); }\FunctionTok{require}\NormalTok{(tidyverse)\}}

\NormalTok{nosaukums}\OtherTok{=}\StringTok{"ForestsTreesAge\_TemperateDeciduousOld\_r1250.tif"}
\NormalTok{ielasisanas\_cels}\OtherTok{=}\FunctionTok{paste0}\NormalTok{(}\StringTok{"./RasterGrids\_100m/2024/RAW/"}\NormalTok{,nosaukums)}
\NormalTok{saglabasanas\_cels}\OtherTok{=}\FunctionTok{paste0}\NormalTok{(}\StringTok{"./RasterGrids\_100m/2024/Scaled/"}\NormalTok{,nosaukums)}
\NormalTok{slanis}\OtherTok{=}\FunctionTok{rast}\NormalTok{(ielasisanas\_cels)}
\NormalTok{videjais}\OtherTok{=}\FunctionTok{global}\NormalTok{(slanis,}\AttributeTok{fun=}\StringTok{"mean"}\NormalTok{,}\AttributeTok{na.rm=}\ConstantTok{TRUE}\NormalTok{)}
\NormalTok{centrets}\OtherTok{=}\NormalTok{slanis}\SpecialCharTok{{-}}\NormalTok{videjais[,}\DecValTok{1}\NormalTok{]}
\NormalTok{standartnovirze}\OtherTok{=}\NormalTok{terra}\SpecialCharTok{::}\FunctionTok{global}\NormalTok{(centrets,}\AttributeTok{fun=}\StringTok{"rms"}\NormalTok{,}\AttributeTok{na.rm=}\ConstantTok{TRUE}\NormalTok{)}
\NormalTok{merogots}\OtherTok{=}\NormalTok{centrets}\SpecialCharTok{/}\NormalTok{standartnovirze[,}\DecValTok{1}\NormalTok{]}
\FunctionTok{writeRaster}\NormalTok{(merogots,}
      \AttributeTok{filename=}\NormalTok{saglabasanas\_cels,}
      \AttributeTok{overwrite=}\ConstantTok{TRUE}\NormalTok{)}
\end{Highlighting}
\end{Shaded}

\section{ForestsTreesAge\_TemperateDeciduousOld\_r3000}\label{ch06.376}

\textbf{filename:} \texttt{ForestsTreesAge\_TemperateDeciduousOld\_r3000.tif}

\textbf{layername:} \texttt{egv\_376}

\textbf{English name:} Fractional cover of Old (over rotation age) Temperate
Deciduous Forests within the 3 km landscape

\textbf{Latvian name:} Vecu (kopš cirtmeta) platlapju mežu platības īpatsvars 3 km
ainavā

\textbf{Procedure:} The cover fraction within a radius of 3000 m around the analysis grid cell
is calculated as the area-weighted sum of the \hyperref[ch06.373]{analysis cells} inside
the buffer, using the workflow \texttt{egvtools::radius\_function()}. During the calculation of the landscape
metric, inverse distance weighted (power = 2) gap filling on the output is
applied to ensure no missing values at the edges. Then the layer is
rewritten to set its name. Finally, the layer is standardised by
subtracting the arithmetic mean and dividing by the root mean squared error.

\begin{Shaded}
\begin{Highlighting}[]
\CommentTok{\# libs {-}{-}{-}{-}}
\ControlFlowTok{if}\NormalTok{(}\SpecialCharTok{!}\FunctionTok{require}\NormalTok{(terra)) \{}\FunctionTok{install.packages}\NormalTok{(}\StringTok{"terra"}\NormalTok{); }\FunctionTok{require}\NormalTok{(terra)\}}
\ControlFlowTok{if}\NormalTok{(}\SpecialCharTok{!}\FunctionTok{require}\NormalTok{(egvtools)) \{remotes}\SpecialCharTok{::}\FunctionTok{install\_github}\NormalTok{(}\StringTok{"aavotins/egvtools"}\NormalTok{); }\FunctionTok{require}\NormalTok{(egvtools)\}}


\CommentTok{\# Templates {-}{-}{-}{-}{-}}
\NormalTok{template100}\OtherTok{=}\FunctionTok{rast}\NormalTok{(}\StringTok{"./Templates/TemplateRasters/LV100m\_10km.tif"}\NormalTok{)}

\CommentTok{\# radii {-}{-}{-}{-}}
\FunctionTok{radius\_function}\NormalTok{(}
 \AttributeTok{kvadrati\_path =} \StringTok{"./Templates/TemplateGrids/tiles/"}\NormalTok{,}
 \AttributeTok{radii\_path   =} \StringTok{"./Templates/TemplateGridPoints/tiles/"}\NormalTok{,}
 \AttributeTok{tikls100\_path =} \StringTok{"./Templates/TemplateGrids/tikls100\_sauzeme.parquet"}\NormalTok{,}
 \AttributeTok{template\_path =} \StringTok{"./Templates/TemplateRasters/LV100m\_10km.tif"}\NormalTok{,}
 \AttributeTok{input\_layers  =} \FunctionTok{c}\NormalTok{(}\StringTok{"./RasterGrids\_100m/2024/RAW/ForestsTreesAge\_TemperateDeciduousOld\_cell.tif"}\NormalTok{),}
 \AttributeTok{layer\_prefixes =} \FunctionTok{c}\NormalTok{(}\StringTok{"ForestsTreesAge\_TemperateDeciduousOld"}\NormalTok{),}
 \AttributeTok{output\_dir   =} \StringTok{"./RasterGrids\_100m/2024/RAW/"}\NormalTok{,}
 \AttributeTok{n\_workers   =} \DecValTok{6}\NormalTok{,}
 \AttributeTok{radii     =} \FunctionTok{c}\NormalTok{(}\StringTok{"r3000"}\NormalTok{),}
 \AttributeTok{radius\_mode  =} \StringTok{"sparse"}\NormalTok{,}
 \AttributeTok{extract\_fun  =} \StringTok{"mean"}\NormalTok{,}
 \AttributeTok{fill\_missing  =} \ConstantTok{TRUE}\NormalTok{,}
 \AttributeTok{IDW\_weight   =} \DecValTok{2}\NormalTok{,}
 \AttributeTok{future\_max\_size =} \DecValTok{40} \SpecialCharTok{*} \DecValTok{1024}\SpecialCharTok{\^{}}\DecValTok{3}\NormalTok{)}


\CommentTok{\# ForestsTreesAge\_TemperateDeciduousOld\_r3000.tif   egv\_376}
\NormalTok{slanis}\OtherTok{=}\FunctionTok{rast}\NormalTok{(}\StringTok{"./RasterGrids\_100m/2024/RAW/ForestsTreesAge\_TemperateDeciduousOld\_r3000.tif"}\NormalTok{)}
\FunctionTok{names}\NormalTok{(slanis)}\OtherTok{=}\StringTok{"egv\_376"}
\NormalTok{slanis2}\OtherTok{=}\FunctionTok{project}\NormalTok{(slanis,template100)}
\FunctionTok{writeRaster}\NormalTok{(slanis2,}
      \StringTok{"./RasterGrids\_100m/2024/RAW/ForestsTreesAge\_TemperateDeciduousOld\_r3000.tif"}\NormalTok{,}
      \AttributeTok{overwrite=}\ConstantTok{TRUE}\NormalTok{)}

\CommentTok{\# standardisation {-}{-}{-}{-}}
\ControlFlowTok{if}\NormalTok{(}\SpecialCharTok{!}\FunctionTok{require}\NormalTok{(terra)) \{}\FunctionTok{install.packages}\NormalTok{(}\StringTok{"terra"}\NormalTok{); }\FunctionTok{require}\NormalTok{(terra)\}}
\ControlFlowTok{if}\NormalTok{(}\SpecialCharTok{!}\FunctionTok{require}\NormalTok{(tidyverse)) \{}\FunctionTok{install.packages}\NormalTok{(}\StringTok{"tidyverse"}\NormalTok{); }\FunctionTok{require}\NormalTok{(tidyverse)\}}

\NormalTok{nosaukums}\OtherTok{=}\StringTok{"ForestsTreesAge\_TemperateDeciduousOld\_r3000.tif"}
\NormalTok{ielasisanas\_cels}\OtherTok{=}\FunctionTok{paste0}\NormalTok{(}\StringTok{"./RasterGrids\_100m/2024/RAW/"}\NormalTok{,nosaukums)}
\NormalTok{saglabasanas\_cels}\OtherTok{=}\FunctionTok{paste0}\NormalTok{(}\StringTok{"./RasterGrids\_100m/2024/Scaled/"}\NormalTok{,nosaukums)}
\NormalTok{slanis}\OtherTok{=}\FunctionTok{rast}\NormalTok{(ielasisanas\_cels)}
\NormalTok{videjais}\OtherTok{=}\FunctionTok{global}\NormalTok{(slanis,}\AttributeTok{fun=}\StringTok{"mean"}\NormalTok{,}\AttributeTok{na.rm=}\ConstantTok{TRUE}\NormalTok{)}
\NormalTok{centrets}\OtherTok{=}\NormalTok{slanis}\SpecialCharTok{{-}}\NormalTok{videjais[,}\DecValTok{1}\NormalTok{]}
\NormalTok{standartnovirze}\OtherTok{=}\NormalTok{terra}\SpecialCharTok{::}\FunctionTok{global}\NormalTok{(centrets,}\AttributeTok{fun=}\StringTok{"rms"}\NormalTok{,}\AttributeTok{na.rm=}\ConstantTok{TRUE}\NormalTok{)}
\NormalTok{merogots}\OtherTok{=}\NormalTok{centrets}\SpecialCharTok{/}\NormalTok{standartnovirze[,}\DecValTok{1}\NormalTok{]}
\FunctionTok{writeRaster}\NormalTok{(merogots,}
      \AttributeTok{filename=}\NormalTok{saglabasanas\_cels,}
      \AttributeTok{overwrite=}\ConstantTok{TRUE}\NormalTok{)}
\end{Highlighting}
\end{Shaded}

\section{ForestsTreesAge\_TemperateDeciduousOld\_r10000}\label{ch06.377}

\textbf{filename:} \texttt{ForestsTreesAge\_TemperateDeciduousOld\_r10000.tif}

\textbf{layername:} \texttt{egv\_377}

\textbf{English name:} Fractional cover of Old (over rotation age) Temperate
Deciduous Forests within the 10 km landscape

\textbf{Latvian name:} Vecu (kopš cirtmeta) platlapju mežu platības īpatsvars 10 km
ainavā

\textbf{Procedure:} The cover fraction within a radius of 10000 m around the analysis grid cell
is calculated as the area-weighted sum of the \hyperref[ch06.373]{analysis cells} inside
the buffer, using the workflow \texttt{egvtools::radius\_function()}. During the calculation of the landscape
metric, inverse distance weighted (power = 2) gap filling on the output is
applied to ensure no missing values at the edges. Then the layer is
rewritten to set its name. Finally, the layer is standardised by
subtracting the arithmetic mean and dividing by the root mean squared error.

\begin{Shaded}
\begin{Highlighting}[]
\CommentTok{\# libs {-}{-}{-}{-}}
\ControlFlowTok{if}\NormalTok{(}\SpecialCharTok{!}\FunctionTok{require}\NormalTok{(terra)) \{}\FunctionTok{install.packages}\NormalTok{(}\StringTok{"terra"}\NormalTok{); }\FunctionTok{require}\NormalTok{(terra)\}}
\ControlFlowTok{if}\NormalTok{(}\SpecialCharTok{!}\FunctionTok{require}\NormalTok{(egvtools)) \{remotes}\SpecialCharTok{::}\FunctionTok{install\_github}\NormalTok{(}\StringTok{"aavotins/egvtools"}\NormalTok{); }\FunctionTok{require}\NormalTok{(egvtools)\}}


\CommentTok{\# Templates {-}{-}{-}{-}{-}}
\NormalTok{template100}\OtherTok{=}\FunctionTok{rast}\NormalTok{(}\StringTok{"./Templates/TemplateRasters/LV100m\_10km.tif"}\NormalTok{)}

\CommentTok{\# radii {-}{-}{-}{-}}
\FunctionTok{radius\_function}\NormalTok{(}
 \AttributeTok{kvadrati\_path =} \StringTok{"./Templates/TemplateGrids/tiles/"}\NormalTok{,}
 \AttributeTok{radii\_path   =} \StringTok{"./Templates/TemplateGridPoints/tiles/"}\NormalTok{,}
 \AttributeTok{tikls100\_path =} \StringTok{"./Templates/TemplateGrids/tikls100\_sauzeme.parquet"}\NormalTok{,}
 \AttributeTok{template\_path =} \StringTok{"./Templates/TemplateRasters/LV100m\_10km.tif"}\NormalTok{,}
 \AttributeTok{input\_layers  =} \FunctionTok{c}\NormalTok{(}\StringTok{"./RasterGrids\_100m/2024/RAW/ForestsTreesAge\_TemperateDeciduousOld\_cell.tif"}\NormalTok{),}
 \AttributeTok{layer\_prefixes =} \FunctionTok{c}\NormalTok{(}\StringTok{"ForestsTreesAge\_TemperateDeciduousOld"}\NormalTok{),}
 \AttributeTok{output\_dir   =} \StringTok{"./RasterGrids\_100m/2024/RAW/"}\NormalTok{,}
 \AttributeTok{n\_workers   =} \DecValTok{6}\NormalTok{,}
 \AttributeTok{radii     =} \FunctionTok{c}\NormalTok{(}\StringTok{"r10000"}\NormalTok{),}
 \AttributeTok{radius\_mode  =} \StringTok{"sparse"}\NormalTok{,}
 \AttributeTok{extract\_fun  =} \StringTok{"mean"}\NormalTok{,}
 \AttributeTok{fill\_missing  =} \ConstantTok{TRUE}\NormalTok{,}
 \AttributeTok{IDW\_weight   =} \DecValTok{2}\NormalTok{,}
 \AttributeTok{future\_max\_size =} \DecValTok{40} \SpecialCharTok{*} \DecValTok{1024}\SpecialCharTok{\^{}}\DecValTok{3}\NormalTok{)}


\CommentTok{\# ForestsTreesAge\_TemperateDeciduousOld\_r10000.tif  egv\_377}
\NormalTok{slanis}\OtherTok{=}\FunctionTok{rast}\NormalTok{(}\StringTok{"./RasterGrids\_100m/2024/RAW/ForestsTreesAge\_TemperateDeciduousOld\_r10000.tif"}\NormalTok{)}
\FunctionTok{names}\NormalTok{(slanis)}\OtherTok{=}\StringTok{"egv\_377"}
\NormalTok{slanis2}\OtherTok{=}\FunctionTok{project}\NormalTok{(slanis,template100)}
\FunctionTok{writeRaster}\NormalTok{(slanis2,}
      \StringTok{"./RasterGrids\_100m/2024/RAW/ForestsTreesAge\_TemperateDeciduousOld\_r10000.tif"}\NormalTok{,}
      \AttributeTok{overwrite=}\ConstantTok{TRUE}\NormalTok{)}

\CommentTok{\# standardisation {-}{-}{-}{-}}
\ControlFlowTok{if}\NormalTok{(}\SpecialCharTok{!}\FunctionTok{require}\NormalTok{(terra)) \{}\FunctionTok{install.packages}\NormalTok{(}\StringTok{"terra"}\NormalTok{); }\FunctionTok{require}\NormalTok{(terra)\}}
\ControlFlowTok{if}\NormalTok{(}\SpecialCharTok{!}\FunctionTok{require}\NormalTok{(tidyverse)) \{}\FunctionTok{install.packages}\NormalTok{(}\StringTok{"tidyverse"}\NormalTok{); }\FunctionTok{require}\NormalTok{(tidyverse)\}}

\NormalTok{nosaukums}\OtherTok{=}\StringTok{"ForestsTreesAge\_TemperateDeciduousOld\_r10000.tif"}
\NormalTok{ielasisanas\_cels}\OtherTok{=}\FunctionTok{paste0}\NormalTok{(}\StringTok{"./RasterGrids\_100m/2024/RAW/"}\NormalTok{,nosaukums)}
\NormalTok{saglabasanas\_cels}\OtherTok{=}\FunctionTok{paste0}\NormalTok{(}\StringTok{"./RasterGrids\_100m/2024/Scaled/"}\NormalTok{,nosaukums)}
\NormalTok{slanis}\OtherTok{=}\FunctionTok{rast}\NormalTok{(ielasisanas\_cels)}
\NormalTok{videjais}\OtherTok{=}\FunctionTok{global}\NormalTok{(slanis,}\AttributeTok{fun=}\StringTok{"mean"}\NormalTok{,}\AttributeTok{na.rm=}\ConstantTok{TRUE}\NormalTok{)}
\NormalTok{centrets}\OtherTok{=}\NormalTok{slanis}\SpecialCharTok{{-}}\NormalTok{videjais[,}\DecValTok{1}\NormalTok{]}
\NormalTok{standartnovirze}\OtherTok{=}\NormalTok{terra}\SpecialCharTok{::}\FunctionTok{global}\NormalTok{(centrets,}\AttributeTok{fun=}\StringTok{"rms"}\NormalTok{,}\AttributeTok{na.rm=}\ConstantTok{TRUE}\NormalTok{)}
\NormalTok{merogots}\OtherTok{=}\NormalTok{centrets}\SpecialCharTok{/}\NormalTok{standartnovirze[,}\DecValTok{1}\NormalTok{]}
\FunctionTok{writeRaster}\NormalTok{(merogots,}
      \AttributeTok{filename=}\NormalTok{saglabasanas\_cels,}
      \AttributeTok{overwrite=}\ConstantTok{TRUE}\NormalTok{)}
\end{Highlighting}
\end{Shaded}

\section{ForestsTreesAge\_TemperateDeciduousYoung\_cell}\label{ch06.378}

\textbf{filename:} \texttt{ForestsTreesAge\_TemperateDeciduousYoung\_cell.tif}

\textbf{layername:} \texttt{egv\_378}

\textbf{English name:} Fractional cover of Young (pre-rotation age) Temperate
Deciduous Forests within the analysis cell (1 ha)

\textbf{Latvian name:} Jaunu (pirms cirtmeta) platlapju mežu platības īpatsvars
analīzes šūnā (1 ha)

\textbf{Procedure:} Most EGVs describing forests are spatially restricted to areas outside
of clearcuts and dead stands. This mask is created using a combination of
the \hyperref[Ch04.01]{State Forest Service's
State Forest Registry} land category 12 and 14, and \hyperref[Ch04.09]{The
Global Forest Watch} pixels classified as lost tree canopy cover since
2020 (raster layer matching input, presence = 1, absence = 0).

To prepare this EGV, stands from the \hyperref[Ch04.01]{State Forest Service's State Forest
Registry} are classified into (in order):

\begin{itemize}
\item
  coniferous (see \hyperref[Ch01]{Terminology and acronyms} for species codes) if
  timber volume of those species exceeded 75\%;
\item
  Boreal deciduous if timber volume of those species exceeded 75\%;
\item
  temperate deciduous if timber volume of those species exceeded 50\%;
\item
  mixed otherwise;
\end{itemize}

then temperate deciduous stands younger than the legal rotation age are
selected and geometries are rasterised (presence = 1, NA otherwise). Rasterisation is
performed using the workflow \texttt{egvtools::polygon2input()}, restricting to pixels outside clearcut
mask and covering background with value 0. The resulting layer
is then aggregated to EGV resolution using the workflow \texttt{egvtools::input2egv()}, which
calculates the arithmetic mean to determine the cover fraction. During
aggregation, inverse distance weighted (power = 2) gap filling on the output is
applied to ensure no missing values at the edges. Finally, the layer is
standardised by subtracting the arithmetic mean and dividing by the root mean squared
error.

\begin{Shaded}
\begin{Highlighting}[]
\CommentTok{\# libs {-}{-}{-}{-}}
\ControlFlowTok{if}\NormalTok{(}\SpecialCharTok{!}\FunctionTok{require}\NormalTok{(egvtools)) \{remotes}\SpecialCharTok{::}\FunctionTok{install\_github}\NormalTok{(}\StringTok{"aavotins/egvtools"}\NormalTok{); }\FunctionTok{require}\NormalTok{(egvtools)\}}
\ControlFlowTok{if}\NormalTok{(}\SpecialCharTok{!}\FunctionTok{require}\NormalTok{(terra)) \{}\FunctionTok{install.packages}\NormalTok{(}\StringTok{"terra"}\NormalTok{); }\FunctionTok{require}\NormalTok{(terra)\}}
\ControlFlowTok{if}\NormalTok{(}\SpecialCharTok{!}\FunctionTok{require}\NormalTok{(sf)) \{}\FunctionTok{install.packages}\NormalTok{(}\StringTok{"sf"}\NormalTok{); }\FunctionTok{require}\NormalTok{(sf)\}}
\ControlFlowTok{if}\NormalTok{(}\SpecialCharTok{!}\FunctionTok{require}\NormalTok{(tidyverse)) \{}\FunctionTok{install.packages}\NormalTok{(}\StringTok{"tidyverse"}\NormalTok{); }\FunctionTok{require}\NormalTok{(tidyverse)\}}
\ControlFlowTok{if}\NormalTok{(}\SpecialCharTok{!}\FunctionTok{require}\NormalTok{(sfarrow)) \{}\FunctionTok{install.packages}\NormalTok{(}\StringTok{"sfarrow"}\NormalTok{); }\FunctionTok{require}\NormalTok{(sfarrow)\}}
\ControlFlowTok{if}\NormalTok{(}\SpecialCharTok{!}\FunctionTok{require}\NormalTok{(readxl)) \{}\FunctionTok{install.packages}\NormalTok{(}\StringTok{"readxl"}\NormalTok{); }\FunctionTok{require}\NormalTok{(readxl)\}}
\ControlFlowTok{if}\NormalTok{(}\SpecialCharTok{!}\FunctionTok{require}\NormalTok{(raster)) \{}\FunctionTok{install.packages}\NormalTok{(}\StringTok{"raster"}\NormalTok{); }\FunctionTok{require}\NormalTok{(raster)\}}
\ControlFlowTok{if}\NormalTok{(}\SpecialCharTok{!}\FunctionTok{require}\NormalTok{(fasterize)) \{}\FunctionTok{install.packages}\NormalTok{(}\StringTok{"fasterize"}\NormalTok{); }\FunctionTok{require}\NormalTok{(fasterize)\}}

\CommentTok{\# templates {-}{-}{-}{-}}
\NormalTok{template100}\OtherTok{=}\FunctionTok{rast}\NormalTok{(}\StringTok{"./Templates/TemplateRasters/LV100m\_10km.tif"}\NormalTok{)}
\NormalTok{template10}\OtherTok{=}\FunctionTok{rast}\NormalTok{(}\StringTok{"./Templates/TemplateRasters/LV10m\_10km.tif"}\NormalTok{)}
\NormalTok{rastrs10}\OtherTok{=}\FunctionTok{raster}\NormalTok{(template10)}

\NormalTok{nulls10}\OtherTok{=}\FunctionTok{rast}\NormalTok{(}\StringTok{"./Templates/TemplateRasters/nulls\_LV10m\_10km.tif"}\NormalTok{)}
\NormalTok{nulls100}\OtherTok{=}\FunctionTok{rast}\NormalTok{(}\StringTok{"./Templates/TemplateRasters/nulls\_LV100m\_10km.tif"}\NormalTok{)}


\CommentTok{\# simple landscape {-}{-}{-}{-}}
\NormalTok{simple\_landscape}\OtherTok{=}\FunctionTok{rast}\NormalTok{(}\StringTok{"RasterGrids\_10m/2024/Ainava\_vienk\_mask.tif"}\NormalTok{)}

\CommentTok{\# mvr {-}{-}{-}{-}}
\NormalTok{mvr}\OtherTok{=}\FunctionTok{st\_read\_parquet}\NormalTok{(}\StringTok{"./Geodata/2024/MVR/nogabali\_2024janv.parquet"}\NormalTok{)}
\NormalTok{mvr}\SpecialCharTok{$}\NormalTok{yes}\OtherTok{=}\DecValTok{1}

\CommentTok{\# clear cut mask {-}{-}{-}{-}}
\NormalTok{izcirtumi}\OtherTok{=}\NormalTok{mvr }\SpecialCharTok{\%\textgreater{}\%} 
 \FunctionTok{filter}\NormalTok{(zkat }\SpecialCharTok{\%in\%} \FunctionTok{c}\NormalTok{(}\StringTok{"12"}\NormalTok{,}\StringTok{"14"}\NormalTok{)) }\SpecialCharTok{\%\textgreater{}\%} 
\NormalTok{ dplyr}\SpecialCharTok{::}\FunctionTok{select}\NormalTok{(yes)}
\NormalTok{r\_izcirtumi\_mvr}\OtherTok{=}\FunctionTok{fasterize}\NormalTok{(izcirtumi,rastrs10,}\AttributeTok{field=}\StringTok{"yes"}\NormalTok{)}
\NormalTok{t\_izcirtumi\_mvr}\OtherTok{=}\FunctionTok{rast}\NormalTok{(r\_izcirtumi\_mvr)}
\FunctionTok{plot}\NormalTok{(t\_izcirtumi\_mvr)}

\NormalTok{tcl}\OtherTok{=}\FunctionTok{rast}\NormalTok{(}\StringTok{"./Geodata/2024/Trees/GFW/TreeCoverLoss\_v1\_12.tif"}\NormalTok{)}
\NormalTok{tcl2}\OtherTok{=}\FunctionTok{ifel}\NormalTok{(tcl}\SpecialCharTok{\textless{}}\DecValTok{20}\NormalTok{,}\DecValTok{0}\NormalTok{,}\DecValTok{1}\NormalTok{)}
\NormalTok{tclX}\OtherTok{=}\FunctionTok{cover}\NormalTok{(tcl2,nulls10)}
\FunctionTok{plot}\NormalTok{(tclX)}

\NormalTok{clearcut\_mask}\OtherTok{=}\FunctionTok{cover}\NormalTok{(t\_izcirtumi\_mvr,tclX,}
          \AttributeTok{filename=}\StringTok{"./RasterGrids\_10m/2024/Mask\_clearcuts.tif"}\NormalTok{,}
          \AttributeTok{overwrite=}\ConstantTok{TRUE}\NormalTok{)}
\FunctionTok{plot}\NormalTok{(clearcut\_mask)}

\FunctionTok{rm}\NormalTok{(izcirtumi)}
\FunctionTok{rm}\NormalTok{(r\_izcirtumi\_mvr)}
\FunctionTok{rm}\NormalTok{(t\_izcirtumi\_mvr)}
\FunctionTok{rm}\NormalTok{(tcl)}
\FunctionTok{rm}\NormalTok{(tcl2)}
\FunctionTok{rm}\NormalTok{(tclX)}

\CommentTok{\# ForestsTreesAge\_TemperateDeciduousYoung\_cell.tif  egv\_378 {-}{-}{-}{-}}
\NormalTok{skujkoki}\OtherTok{=}\FunctionTok{c}\NormalTok{(}\StringTok{"1"}\NormalTok{,}\StringTok{"3"}\NormalTok{,}\StringTok{"13"}\NormalTok{,}\StringTok{"14"}\NormalTok{,}\StringTok{"15"}\NormalTok{,}\StringTok{"22"}\NormalTok{,}\StringTok{"23"}\NormalTok{,}\StringTok{"28"}\NormalTok{) }\CommentTok{\# 8}
\NormalTok{saurlapji}\OtherTok{=}\FunctionTok{c}\NormalTok{(}\StringTok{"4"}\NormalTok{,}\StringTok{"6"}\NormalTok{,}\StringTok{"8"}\NormalTok{,}\StringTok{"9"}\NormalTok{,}\StringTok{"19"}\NormalTok{,}\StringTok{"20"}\NormalTok{,}\StringTok{"21"}\NormalTok{,}\StringTok{"32"}\NormalTok{,}\StringTok{"35"}\NormalTok{,}\StringTok{"68"}\NormalTok{) }\CommentTok{\# 10}
\NormalTok{platlapji}\OtherTok{=}\FunctionTok{c}\NormalTok{(}\StringTok{"10"}\NormalTok{,}\StringTok{"11"}\NormalTok{,}\StringTok{"12"}\NormalTok{,}\StringTok{"16"}\NormalTok{,}\StringTok{"17"}\NormalTok{,}\StringTok{"18"}\NormalTok{,}\StringTok{"24"}\NormalTok{,}\StringTok{"25"}\NormalTok{,}\StringTok{"26"}\NormalTok{,}\StringTok{"27"}\NormalTok{,}\StringTok{"28"}\NormalTok{,}\StringTok{"29"}\NormalTok{,}\StringTok{"50"}\NormalTok{,}
      \StringTok{"61"}\NormalTok{,}\StringTok{"62"}\NormalTok{,}\StringTok{"63"}\NormalTok{,}\StringTok{"64"}\NormalTok{,}\StringTok{"65"}\NormalTok{,}\StringTok{"66"}\NormalTok{,}\StringTok{"67"}\NormalTok{,}\StringTok{"69"}\NormalTok{) }\CommentTok{\# 21}
\NormalTok{mvr}\OtherTok{=}\NormalTok{mvr }\SpecialCharTok{\%\textgreater{}\%} 
 \FunctionTok{mutate}\NormalTok{(}\AttributeTok{kraja\_skujkoku=}\FunctionTok{ifelse}\NormalTok{(s10 }\SpecialCharTok{\%in\%}\NormalTok{ skujkoki,v10,}\DecValTok{0}\NormalTok{)}\SpecialCharTok{+}
      \FunctionTok{ifelse}\NormalTok{(s11 }\SpecialCharTok{\%in\%}\NormalTok{ skujkoki,v11,}\DecValTok{0}\NormalTok{)}\SpecialCharTok{+}\FunctionTok{ifelse}\NormalTok{(s12 }\SpecialCharTok{\%in\%}\NormalTok{ skujkoki,v12,}\DecValTok{0}\NormalTok{)}\SpecialCharTok{+}
      \FunctionTok{ifelse}\NormalTok{(s13 }\SpecialCharTok{\%in\%}\NormalTok{ skujkoki,v13,}\DecValTok{0}\NormalTok{)}\SpecialCharTok{+}\FunctionTok{ifelse}\NormalTok{(s14 }\SpecialCharTok{\%in\%}\NormalTok{ skujkoki,v14,}\DecValTok{0}\NormalTok{),}
     \AttributeTok{kraja\_saurlapju=}\FunctionTok{ifelse}\NormalTok{(s10 }\SpecialCharTok{\%in\%}\NormalTok{ saurlapji,v10,}\DecValTok{0}\NormalTok{)}\SpecialCharTok{+}
      \FunctionTok{ifelse}\NormalTok{(s11 }\SpecialCharTok{\%in\%}\NormalTok{ saurlapji,v11,}\DecValTok{0}\NormalTok{)}\SpecialCharTok{+}\FunctionTok{ifelse}\NormalTok{(s12 }\SpecialCharTok{\%in\%}\NormalTok{ saurlapji,v12,}\DecValTok{0}\NormalTok{)}\SpecialCharTok{+}
      \FunctionTok{ifelse}\NormalTok{(s13 }\SpecialCharTok{\%in\%}\NormalTok{ saurlapji,v13,}\DecValTok{0}\NormalTok{)}\SpecialCharTok{+}\FunctionTok{ifelse}\NormalTok{(s14 }\SpecialCharTok{\%in\%}\NormalTok{ saurlapji,v14,}\DecValTok{0}\NormalTok{),}
     \AttributeTok{kraja\_platlapju=}\FunctionTok{ifelse}\NormalTok{(s10 }\SpecialCharTok{\%in\%}\NormalTok{ platlapji,v10,}\DecValTok{0}\NormalTok{)}\SpecialCharTok{+}
      \FunctionTok{ifelse}\NormalTok{(s11 }\SpecialCharTok{\%in\%}\NormalTok{ platlapji,v11,}\DecValTok{0}\NormalTok{)}\SpecialCharTok{+}\FunctionTok{ifelse}\NormalTok{(s12 }\SpecialCharTok{\%in\%}\NormalTok{ platlapji,v12,}\DecValTok{0}\NormalTok{)}\SpecialCharTok{+}
      \FunctionTok{ifelse}\NormalTok{(s13 }\SpecialCharTok{\%in\%}\NormalTok{ platlapji,v13,}\DecValTok{0}\NormalTok{)}\SpecialCharTok{+}\FunctionTok{ifelse}\NormalTok{(s14 }\SpecialCharTok{\%in\%}\NormalTok{ platlapji,v14,}\DecValTok{0}\NormalTok{)) }\SpecialCharTok{\%\textgreater{}\%} 
 \FunctionTok{mutate}\NormalTok{(}\AttributeTok{kopeja\_kraja=}\NormalTok{kraja\_skujkoku}\SpecialCharTok{+}\NormalTok{kraja\_platlapju}\SpecialCharTok{+}\NormalTok{kraja\_saurlapju) }\SpecialCharTok{\%\textgreater{}\%} 
 \FunctionTok{mutate}\NormalTok{(}\AttributeTok{tips=}\FunctionTok{ifelse}\NormalTok{(kraja\_skujkoku}\SpecialCharTok{/}\NormalTok{kopeja\_kraja}\SpecialCharTok{\textgreater{}=}\FloatTok{0.75}\NormalTok{,}\StringTok{"skujkoku"}\NormalTok{,}
           \FunctionTok{ifelse}\NormalTok{(kraja\_saurlapju}\SpecialCharTok{/}\NormalTok{kopeja\_kraja}\SpecialCharTok{\textgreater{}=}\FloatTok{0.75}\NormalTok{,}\StringTok{"saurlapju"}\NormalTok{,}
              \FunctionTok{ifelse}\NormalTok{(kraja\_platlapju}\SpecialCharTok{/}\NormalTok{kopeja\_kraja}\SpecialCharTok{\textgreater{}}\FloatTok{0.5}\NormalTok{,}\StringTok{"platlapju"}\NormalTok{,}
                  \StringTok{"jauktu koku"}\NormalTok{))))}
\NormalTok{nogabali}\OtherTok{=}\NormalTok{mvr }\SpecialCharTok{\%\textgreater{}\%} 
 \FunctionTok{filter}\NormalTok{(zkat}\SpecialCharTok{==}\StringTok{"10"}\SpecialCharTok{\&}\NormalTok{tips}\SpecialCharTok{==}\StringTok{"platlapju"}\SpecialCharTok{\&}\NormalTok{(vgr}\SpecialCharTok{==}\StringTok{"1"}\SpecialCharTok{|}\NormalTok{vgr}\SpecialCharTok{==}\StringTok{"2"}\SpecialCharTok{|}\NormalTok{vgr}\SpecialCharTok{==}\StringTok{"3"}\NormalTok{))}

\NormalTok{p2i\_rez}\OtherTok{=}\NormalTok{egvtools}\SpecialCharTok{::}\FunctionTok{polygon2input}\NormalTok{(}\AttributeTok{vector\_data =}\NormalTok{ nogabali,}
                \AttributeTok{template\_path =} \StringTok{"./Templates/TemplateRasters/LV10m\_10km.tif"}\NormalTok{,}
                \AttributeTok{out\_path =} \StringTok{"./RasterGrids\_10m/2024/"}\NormalTok{,}
                \AttributeTok{file\_name =} \StringTok{"ForestsTreesAge\_TemperateDeciduousYoung\_input.tif"}\NormalTok{,}
                \AttributeTok{value\_field =} \StringTok{"yes"}\NormalTok{,}
                \AttributeTok{restrict\_to =}\NormalTok{ clearcut\_mask,}
                \AttributeTok{restrict\_values =} \DecValTok{0}\NormalTok{,}
                \AttributeTok{prepare=}\ConstantTok{FALSE}\NormalTok{,}
                \AttributeTok{background\_raster =} \StringTok{"./Templates/TemplateRasters/nulls\_LV10m\_10km.tif"}\NormalTok{,}
                \AttributeTok{plot\_result =} \ConstantTok{TRUE}\NormalTok{)}
\NormalTok{p2i\_rez}
\NormalTok{i2e\_rez}\OtherTok{=}\NormalTok{egvtools}\SpecialCharTok{::}\FunctionTok{input2egv}\NormalTok{(}\AttributeTok{input=}\FunctionTok{paste0}\NormalTok{(}\StringTok{"./RasterGrids\_10m/2024/"}\NormalTok{,}
                     \StringTok{"ForestsTreesAge\_TemperateDeciduousYoung\_input.tif"}\NormalTok{),}
              \AttributeTok{egv\_template=} \StringTok{"./Templates/TemplateRasters/LV100m\_10km.tif"}\NormalTok{,}
              \AttributeTok{summary\_function =} \StringTok{"average"}\NormalTok{,}
              \AttributeTok{missing\_job =} \StringTok{"FillOutput"}\NormalTok{,}
              \AttributeTok{outlocation =} \StringTok{"./RasterGrids\_100m/2024/RAW/"}\NormalTok{,}
              \AttributeTok{outfilename =} \StringTok{"ForestsTreesAge\_TemperateDeciduousYoung\_cell.tif"}\NormalTok{,}
              \AttributeTok{layername =} \StringTok{"egv\_378"}\NormalTok{,}
              \AttributeTok{idw\_weight =} \DecValTok{2}\NormalTok{,}
              \AttributeTok{plot\_gaps =} \ConstantTok{FALSE}\NormalTok{,}\AttributeTok{plot\_final =} \ConstantTok{TRUE}\NormalTok{)}
\NormalTok{i2e\_rez}
\FunctionTok{rm}\NormalTok{(nogabali)}
\FunctionTok{rm}\NormalTok{(p2i\_rez)}
\FunctionTok{rm}\NormalTok{(i2e\_rez)}
\FunctionTok{unlink}\NormalTok{(}\StringTok{"./RasterGrids\_10m/2024/ForestsTreesAge\_TemperateDeciduousYoung\_input.tif"}\NormalTok{)}

\CommentTok{\# standardisation {-}{-}{-}{-}}
\ControlFlowTok{if}\NormalTok{(}\SpecialCharTok{!}\FunctionTok{require}\NormalTok{(terra)) \{}\FunctionTok{install.packages}\NormalTok{(}\StringTok{"terra"}\NormalTok{); }\FunctionTok{require}\NormalTok{(terra)\}}
\ControlFlowTok{if}\NormalTok{(}\SpecialCharTok{!}\FunctionTok{require}\NormalTok{(tidyverse)) \{}\FunctionTok{install.packages}\NormalTok{(}\StringTok{"tidyverse"}\NormalTok{); }\FunctionTok{require}\NormalTok{(tidyverse)\}}

\NormalTok{nosaukums}\OtherTok{=}\StringTok{"ForestsTreesAge\_TemperateDeciduousYoung\_cell.tif"}
\NormalTok{ielasisanas\_cels}\OtherTok{=}\FunctionTok{paste0}\NormalTok{(}\StringTok{"./RasterGrids\_100m/2024/RAW/"}\NormalTok{,nosaukums)}
\NormalTok{saglabasanas\_cels}\OtherTok{=}\FunctionTok{paste0}\NormalTok{(}\StringTok{"./RasterGrids\_100m/2024/Scaled/"}\NormalTok{,nosaukums)}
\NormalTok{slanis}\OtherTok{=}\FunctionTok{rast}\NormalTok{(ielasisanas\_cels)}
\NormalTok{videjais}\OtherTok{=}\FunctionTok{global}\NormalTok{(slanis,}\AttributeTok{fun=}\StringTok{"mean"}\NormalTok{,}\AttributeTok{na.rm=}\ConstantTok{TRUE}\NormalTok{)}
\NormalTok{centrets}\OtherTok{=}\NormalTok{slanis}\SpecialCharTok{{-}}\NormalTok{videjais[,}\DecValTok{1}\NormalTok{]}
\NormalTok{standartnovirze}\OtherTok{=}\NormalTok{terra}\SpecialCharTok{::}\FunctionTok{global}\NormalTok{(centrets,}\AttributeTok{fun=}\StringTok{"rms"}\NormalTok{,}\AttributeTok{na.rm=}\ConstantTok{TRUE}\NormalTok{)}
\NormalTok{merogots}\OtherTok{=}\NormalTok{centrets}\SpecialCharTok{/}\NormalTok{standartnovirze[,}\DecValTok{1}\NormalTok{]}
\FunctionTok{writeRaster}\NormalTok{(merogots,}
      \AttributeTok{filename=}\NormalTok{saglabasanas\_cels,}
      \AttributeTok{overwrite=}\ConstantTok{TRUE}\NormalTok{)}
\end{Highlighting}
\end{Shaded}

\section{ForestsTreesAge\_TemperateDeciduousYoung\_r500}\label{ch06.379}

\textbf{filename:} \texttt{ForestsTreesAge\_TemperateDeciduousYoung\_r500.tif}

\textbf{layername:} \texttt{egv\_379}

\textbf{English name:} Fractional cover of Young (pre-rotation age) Temperate
Deciduous Forests within the 0.5 km landscape

\textbf{Latvian name:} Jaunu (pirms cirtmeta) platlapju mežu platības īpatsvars 0,5
km ainavā

\textbf{Procedure:} The cover fraction within a radius of 500 m around the analysis grid cell is
calculated as the area-weighted sum of the \hyperref[ch06.378]{analysis cells} inside the
buffer, using the workflow \texttt{egvtools::radius\_function()}. During the calculation of the landscape metric,
inverse distance weighted (power = 2) gap filling on the output is applied
to ensure no missing values at the edges. Then the layer is rewritten to set
its name. Finally, the layer is standardised by subtracting the arithmetic
mean and dividing by the root mean squared error.

\begin{Shaded}
\begin{Highlighting}[]
\CommentTok{\# libs {-}{-}{-}{-}}
\ControlFlowTok{if}\NormalTok{(}\SpecialCharTok{!}\FunctionTok{require}\NormalTok{(terra)) \{}\FunctionTok{install.packages}\NormalTok{(}\StringTok{"terra"}\NormalTok{); }\FunctionTok{require}\NormalTok{(terra)\}}
\ControlFlowTok{if}\NormalTok{(}\SpecialCharTok{!}\FunctionTok{require}\NormalTok{(egvtools)) \{remotes}\SpecialCharTok{::}\FunctionTok{install\_github}\NormalTok{(}\StringTok{"aavotins/egvtools"}\NormalTok{); }\FunctionTok{require}\NormalTok{(egvtools)\}}


\CommentTok{\# Templates {-}{-}{-}{-}{-}}
\NormalTok{template100}\OtherTok{=}\FunctionTok{rast}\NormalTok{(}\StringTok{"./Templates/TemplateRasters/LV100m\_10km.tif"}\NormalTok{)}

\CommentTok{\# radii {-}{-}{-}{-}}
\FunctionTok{radius\_function}\NormalTok{(}
 \AttributeTok{kvadrati\_path =} \StringTok{"./Templates/TemplateGrids/tiles/"}\NormalTok{,}
 \AttributeTok{radii\_path   =} \StringTok{"./Templates/TemplateGridPoints/tiles/"}\NormalTok{,}
 \AttributeTok{tikls100\_path =} \StringTok{"./Templates/TemplateGrids/tikls100\_sauzeme.parquet"}\NormalTok{,}
 \AttributeTok{template\_path =} \StringTok{"./Templates/TemplateRasters/LV100m\_10km.tif"}\NormalTok{,}
 \AttributeTok{input\_layers  =} \FunctionTok{c}\NormalTok{(}\StringTok{"./RasterGrids\_100m/2024/RAW/ForestsTreesAge\_TemperateDeciduousYoung\_cell.tif"}\NormalTok{),}
 \AttributeTok{layer\_prefixes =} \FunctionTok{c}\NormalTok{(}\StringTok{"ForestsTreesAge\_TemperateDeciduousYoung"}\NormalTok{),}
 \AttributeTok{output\_dir   =} \StringTok{"./RasterGrids\_100m/2024/RAW/"}\NormalTok{,}
 \AttributeTok{n\_workers   =} \DecValTok{6}\NormalTok{,}
 \AttributeTok{radii     =} \FunctionTok{c}\NormalTok{(}\StringTok{"r500"}\NormalTok{),}
 \AttributeTok{radius\_mode  =} \StringTok{"sparse"}\NormalTok{,}
 \AttributeTok{extract\_fun  =} \StringTok{"mean"}\NormalTok{,}
 \AttributeTok{fill\_missing  =} \ConstantTok{TRUE}\NormalTok{,}
 \AttributeTok{IDW\_weight   =} \DecValTok{2}\NormalTok{,}
 \AttributeTok{future\_max\_size =} \DecValTok{40} \SpecialCharTok{*} \DecValTok{1024}\SpecialCharTok{\^{}}\DecValTok{3}\NormalTok{)}


\CommentTok{\# ForestsTreesAge\_TemperateDeciduousYoung\_r500.tif  egv\_379}
\NormalTok{slanis}\OtherTok{=}\FunctionTok{rast}\NormalTok{(}\StringTok{"./RasterGrids\_100m/2024/RAW/ForestsTreesAge\_TemperateDeciduousYoung\_r500.tif"}\NormalTok{)}
\FunctionTok{names}\NormalTok{(slanis)}\OtherTok{=}\StringTok{"egv\_379"}
\NormalTok{slanis2}\OtherTok{=}\FunctionTok{project}\NormalTok{(slanis,template100)}
\FunctionTok{writeRaster}\NormalTok{(slanis2,}
      \StringTok{"./RasterGrids\_100m/2024/RAW/ForestsTreesAge\_TemperateDeciduousYoung\_r500.tif"}\NormalTok{,}
      \AttributeTok{overwrite=}\ConstantTok{TRUE}\NormalTok{)}

\CommentTok{\# standardisation {-}{-}{-}{-}}
\ControlFlowTok{if}\NormalTok{(}\SpecialCharTok{!}\FunctionTok{require}\NormalTok{(terra)) \{}\FunctionTok{install.packages}\NormalTok{(}\StringTok{"terra"}\NormalTok{); }\FunctionTok{require}\NormalTok{(terra)\}}
\ControlFlowTok{if}\NormalTok{(}\SpecialCharTok{!}\FunctionTok{require}\NormalTok{(tidyverse)) \{}\FunctionTok{install.packages}\NormalTok{(}\StringTok{"tidyverse"}\NormalTok{); }\FunctionTok{require}\NormalTok{(tidyverse)\}}

\NormalTok{nosaukums}\OtherTok{=}\StringTok{"ForestsTreesAge\_TemperateDeciduousYoung\_r500.tif"}
\NormalTok{ielasisanas\_cels}\OtherTok{=}\FunctionTok{paste0}\NormalTok{(}\StringTok{"./RasterGrids\_100m/2024/RAW/"}\NormalTok{,nosaukums)}
\NormalTok{saglabasanas\_cels}\OtherTok{=}\FunctionTok{paste0}\NormalTok{(}\StringTok{"./RasterGrids\_100m/2024/Scaled/"}\NormalTok{,nosaukums)}
\NormalTok{slanis}\OtherTok{=}\FunctionTok{rast}\NormalTok{(ielasisanas\_cels)}
\NormalTok{videjais}\OtherTok{=}\FunctionTok{global}\NormalTok{(slanis,}\AttributeTok{fun=}\StringTok{"mean"}\NormalTok{,}\AttributeTok{na.rm=}\ConstantTok{TRUE}\NormalTok{)}
\NormalTok{centrets}\OtherTok{=}\NormalTok{slanis}\SpecialCharTok{{-}}\NormalTok{videjais[,}\DecValTok{1}\NormalTok{]}
\NormalTok{standartnovirze}\OtherTok{=}\NormalTok{terra}\SpecialCharTok{::}\FunctionTok{global}\NormalTok{(centrets,}\AttributeTok{fun=}\StringTok{"rms"}\NormalTok{,}\AttributeTok{na.rm=}\ConstantTok{TRUE}\NormalTok{)}
\NormalTok{merogots}\OtherTok{=}\NormalTok{centrets}\SpecialCharTok{/}\NormalTok{standartnovirze[,}\DecValTok{1}\NormalTok{]}
\FunctionTok{writeRaster}\NormalTok{(merogots,}
      \AttributeTok{filename=}\NormalTok{saglabasanas\_cels,}
      \AttributeTok{overwrite=}\ConstantTok{TRUE}\NormalTok{)}
\end{Highlighting}
\end{Shaded}

\section{ForestsTreesAge\_TemperateDeciduousYoung\_r1250}\label{ch06.380}

\textbf{filename:} \texttt{ForestsTreesAge\_TemperateDeciduousYoung\_r1250.tif}

\textbf{layername:} \texttt{egv\_380}

\textbf{English name:} Fractional cover of Young (pre-rotation age) Temperate
Deciduous Forests within the 1.25 km landscape

\textbf{Latvian name:} Jaunu (pirms cirtmeta) platlapju mežu platības īpatsvars 1,25
km ainavā

\textbf{Procedure:} The cover fraction within a radius of 1250 m around the analysis grid cell
is calculated as the area-weighted sum of the \hyperref[ch06.378]{analysis cells} inside
the buffer, using the workflow \texttt{egvtools::radius\_function()}. During the calculation of the landscape
metric, inverse distance weighted (power = 2) gap filling on the output is
applied to ensure no missing values at the edges. Then the layer is
rewritten to set its name. Finally, the layer is standardised by
subtracting the arithmetic mean and dividing by the root mean squared error.

\begin{Shaded}
\begin{Highlighting}[]
\CommentTok{\# libs {-}{-}{-}{-}}
\ControlFlowTok{if}\NormalTok{(}\SpecialCharTok{!}\FunctionTok{require}\NormalTok{(terra)) \{}\FunctionTok{install.packages}\NormalTok{(}\StringTok{"terra"}\NormalTok{); }\FunctionTok{require}\NormalTok{(terra)\}}
\ControlFlowTok{if}\NormalTok{(}\SpecialCharTok{!}\FunctionTok{require}\NormalTok{(egvtools)) \{remotes}\SpecialCharTok{::}\FunctionTok{install\_github}\NormalTok{(}\StringTok{"aavotins/egvtools"}\NormalTok{); }\FunctionTok{require}\NormalTok{(egvtools)\}}


\CommentTok{\# Templates {-}{-}{-}{-}{-}}
\NormalTok{template100}\OtherTok{=}\FunctionTok{rast}\NormalTok{(}\StringTok{"./Templates/TemplateRasters/LV100m\_10km.tif"}\NormalTok{)}

\CommentTok{\# radii {-}{-}{-}{-}}
\FunctionTok{radius\_function}\NormalTok{(}
 \AttributeTok{kvadrati\_path =} \StringTok{"./Templates/TemplateGrids/tiles/"}\NormalTok{,}
 \AttributeTok{radii\_path   =} \StringTok{"./Templates/TemplateGridPoints/tiles/"}\NormalTok{,}
 \AttributeTok{tikls100\_path =} \StringTok{"./Templates/TemplateGrids/tikls100\_sauzeme.parquet"}\NormalTok{,}
 \AttributeTok{template\_path =} \StringTok{"./Templates/TemplateRasters/LV100m\_10km.tif"}\NormalTok{,}
 \AttributeTok{input\_layers  =} \FunctionTok{c}\NormalTok{(}\StringTok{"./RasterGrids\_100m/2024/RAW/ForestsTreesAge\_TemperateDeciduousYoung\_cell.tif"}\NormalTok{),}
 \AttributeTok{layer\_prefixes =} \FunctionTok{c}\NormalTok{(}\StringTok{"ForestsTreesAge\_TemperateDeciduousYoung"}\NormalTok{),}
 \AttributeTok{output\_dir   =} \StringTok{"./RasterGrids\_100m/2024/RAW/"}\NormalTok{,}
 \AttributeTok{n\_workers   =} \DecValTok{6}\NormalTok{,}
 \AttributeTok{radii     =} \FunctionTok{c}\NormalTok{(}\StringTok{"r1250"}\NormalTok{),}
 \AttributeTok{radius\_mode  =} \StringTok{"sparse"}\NormalTok{,}
 \AttributeTok{extract\_fun  =} \StringTok{"mean"}\NormalTok{,}
 \AttributeTok{fill\_missing  =} \ConstantTok{TRUE}\NormalTok{,}
 \AttributeTok{IDW\_weight   =} \DecValTok{2}\NormalTok{,}
 \AttributeTok{future\_max\_size =} \DecValTok{40} \SpecialCharTok{*} \DecValTok{1024}\SpecialCharTok{\^{}}\DecValTok{3}\NormalTok{)}


\CommentTok{\# ForestsTreesAge\_TemperateDeciduousYoung\_r1250.tif egv\_380}
\NormalTok{slanis}\OtherTok{=}\FunctionTok{rast}\NormalTok{(}\StringTok{"./RasterGrids\_100m/2024/RAW/ForestsTreesAge\_TemperateDeciduousYoung\_r1250.tif"}\NormalTok{)}
\FunctionTok{names}\NormalTok{(slanis)}\OtherTok{=}\StringTok{"egv\_380"}
\NormalTok{slanis2}\OtherTok{=}\FunctionTok{project}\NormalTok{(slanis,template100)}
\FunctionTok{writeRaster}\NormalTok{(slanis2,}
      \StringTok{"./RasterGrids\_100m/2024/RAW/ForestsTreesAge\_TemperateDeciduousYoung\_r1250.tif"}\NormalTok{,}
      \AttributeTok{overwrite=}\ConstantTok{TRUE}\NormalTok{)}

\CommentTok{\# standardisation {-}{-}{-}{-}}
\ControlFlowTok{if}\NormalTok{(}\SpecialCharTok{!}\FunctionTok{require}\NormalTok{(terra)) \{}\FunctionTok{install.packages}\NormalTok{(}\StringTok{"terra"}\NormalTok{); }\FunctionTok{require}\NormalTok{(terra)\}}
\ControlFlowTok{if}\NormalTok{(}\SpecialCharTok{!}\FunctionTok{require}\NormalTok{(tidyverse)) \{}\FunctionTok{install.packages}\NormalTok{(}\StringTok{"tidyverse"}\NormalTok{); }\FunctionTok{require}\NormalTok{(tidyverse)\}}

\NormalTok{nosaukums}\OtherTok{=}\StringTok{"ForestsTreesAge\_TemperateDeciduousYoung\_r1250.tif"}
\NormalTok{ielasisanas\_cels}\OtherTok{=}\FunctionTok{paste0}\NormalTok{(}\StringTok{"./RasterGrids\_100m/2024/RAW/"}\NormalTok{,nosaukums)}
\NormalTok{saglabasanas\_cels}\OtherTok{=}\FunctionTok{paste0}\NormalTok{(}\StringTok{"./RasterGrids\_100m/2024/Scaled/"}\NormalTok{,nosaukums)}
\NormalTok{slanis}\OtherTok{=}\FunctionTok{rast}\NormalTok{(ielasisanas\_cels)}
\NormalTok{videjais}\OtherTok{=}\FunctionTok{global}\NormalTok{(slanis,}\AttributeTok{fun=}\StringTok{"mean"}\NormalTok{,}\AttributeTok{na.rm=}\ConstantTok{TRUE}\NormalTok{)}
\NormalTok{centrets}\OtherTok{=}\NormalTok{slanis}\SpecialCharTok{{-}}\NormalTok{videjais[,}\DecValTok{1}\NormalTok{]}
\NormalTok{standartnovirze}\OtherTok{=}\NormalTok{terra}\SpecialCharTok{::}\FunctionTok{global}\NormalTok{(centrets,}\AttributeTok{fun=}\StringTok{"rms"}\NormalTok{,}\AttributeTok{na.rm=}\ConstantTok{TRUE}\NormalTok{)}
\NormalTok{merogots}\OtherTok{=}\NormalTok{centrets}\SpecialCharTok{/}\NormalTok{standartnovirze[,}\DecValTok{1}\NormalTok{]}
\FunctionTok{writeRaster}\NormalTok{(merogots,}
      \AttributeTok{filename=}\NormalTok{saglabasanas\_cels,}
      \AttributeTok{overwrite=}\ConstantTok{TRUE}\NormalTok{)}
\end{Highlighting}
\end{Shaded}

\section{ForestsTreesAge\_TemperateDeciduousYoung\_r3000}\label{ch06.381}

\textbf{filename:} \texttt{ForestsTreesAge\_TemperateDeciduousYoung\_r3000.tif}

\textbf{layername:} \texttt{egv\_381}

\textbf{English name:} Fractional cover of Young (pre-rotation age) Temperate
Deciduous Forests within the 3 km landscape

\textbf{Latvian name:} Jaunu (pirms cirtmeta) platlapju mežu platības īpatsvars 3 km
ainavā

\textbf{Procedure:} The cover fraction within a radius of 3000 m around the analysis grid cell
is calculated as the area-weighted sum of the \hyperref[ch06.378]{analysis cells} inside
the buffer, using the workflow \texttt{egvtools::radius\_function()}. During the calculation of the landscape
metric, inverse distance weighted (power = 2) gap filling on the output is
applied to ensure no missing values at the edges. Then the layer is
rewritten to set its name. Finally, the layer is standardised by
subtracting the arithmetic mean and dividing by the root mean squared error.

\begin{Shaded}
\begin{Highlighting}[]
\CommentTok{\# libs {-}{-}{-}{-}}
\ControlFlowTok{if}\NormalTok{(}\SpecialCharTok{!}\FunctionTok{require}\NormalTok{(terra)) \{}\FunctionTok{install.packages}\NormalTok{(}\StringTok{"terra"}\NormalTok{); }\FunctionTok{require}\NormalTok{(terra)\}}
\ControlFlowTok{if}\NormalTok{(}\SpecialCharTok{!}\FunctionTok{require}\NormalTok{(egvtools)) \{remotes}\SpecialCharTok{::}\FunctionTok{install\_github}\NormalTok{(}\StringTok{"aavotins/egvtools"}\NormalTok{); }\FunctionTok{require}\NormalTok{(egvtools)\}}


\CommentTok{\# Templates {-}{-}{-}{-}{-}}
\NormalTok{template100}\OtherTok{=}\FunctionTok{rast}\NormalTok{(}\StringTok{"./Templates/TemplateRasters/LV100m\_10km.tif"}\NormalTok{)}

\CommentTok{\# radii {-}{-}{-}{-}}
\FunctionTok{radius\_function}\NormalTok{(}
 \AttributeTok{kvadrati\_path =} \StringTok{"./Templates/TemplateGrids/tiles/"}\NormalTok{,}
 \AttributeTok{radii\_path   =} \StringTok{"./Templates/TemplateGridPoints/tiles/"}\NormalTok{,}
 \AttributeTok{tikls100\_path =} \StringTok{"./Templates/TemplateGrids/tikls100\_sauzeme.parquet"}\NormalTok{,}
 \AttributeTok{template\_path =} \StringTok{"./Templates/TemplateRasters/LV100m\_10km.tif"}\NormalTok{,}
 \AttributeTok{input\_layers  =} \FunctionTok{c}\NormalTok{(}\StringTok{"./RasterGrids\_100m/2024/RAW/ForestsTreesAge\_TemperateDeciduousYoung\_cell.tif"}\NormalTok{),}
 \AttributeTok{layer\_prefixes =} \FunctionTok{c}\NormalTok{(}\StringTok{"ForestsTreesAge\_TemperateDeciduousYoung"}\NormalTok{),}
 \AttributeTok{output\_dir   =} \StringTok{"./RasterGrids\_100m/2024/RAW/"}\NormalTok{,}
 \AttributeTok{n\_workers   =} \DecValTok{6}\NormalTok{,}
 \AttributeTok{radii     =} \FunctionTok{c}\NormalTok{(}\StringTok{"r3000"}\NormalTok{),}
 \AttributeTok{radius\_mode  =} \StringTok{"sparse"}\NormalTok{,}
 \AttributeTok{extract\_fun  =} \StringTok{"mean"}\NormalTok{,}
 \AttributeTok{fill\_missing  =} \ConstantTok{TRUE}\NormalTok{,}
 \AttributeTok{IDW\_weight   =} \DecValTok{2}\NormalTok{,}
 \AttributeTok{future\_max\_size =} \DecValTok{40} \SpecialCharTok{*} \DecValTok{1024}\SpecialCharTok{\^{}}\DecValTok{3}\NormalTok{)}


\CommentTok{\# ForestsTreesAge\_TemperateDeciduousYoung\_r3000.tif egv\_381}
\NormalTok{slanis}\OtherTok{=}\FunctionTok{rast}\NormalTok{(}\StringTok{"./RasterGrids\_100m/2024/RAW/ForestsTreesAge\_TemperateDeciduousYoung\_r3000.tif"}\NormalTok{)}
\FunctionTok{names}\NormalTok{(slanis)}\OtherTok{=}\StringTok{"egv\_381"}
\NormalTok{slanis2}\OtherTok{=}\FunctionTok{project}\NormalTok{(slanis,template100)}
\FunctionTok{writeRaster}\NormalTok{(slanis2,}
      \StringTok{"./RasterGrids\_100m/2024/RAW/ForestsTreesAge\_TemperateDeciduousYoung\_r3000.tif"}\NormalTok{,}
      \AttributeTok{overwrite=}\ConstantTok{TRUE}\NormalTok{)}

\CommentTok{\# standardisation {-}{-}{-}{-}}
\ControlFlowTok{if}\NormalTok{(}\SpecialCharTok{!}\FunctionTok{require}\NormalTok{(terra)) \{}\FunctionTok{install.packages}\NormalTok{(}\StringTok{"terra"}\NormalTok{); }\FunctionTok{require}\NormalTok{(terra)\}}
\ControlFlowTok{if}\NormalTok{(}\SpecialCharTok{!}\FunctionTok{require}\NormalTok{(tidyverse)) \{}\FunctionTok{install.packages}\NormalTok{(}\StringTok{"tidyverse"}\NormalTok{); }\FunctionTok{require}\NormalTok{(tidyverse)\}}

\NormalTok{nosaukums}\OtherTok{=}\StringTok{"ForestsTreesAge\_TemperateDeciduousYoung\_r3000.tif"}
\NormalTok{ielasisanas\_cels}\OtherTok{=}\FunctionTok{paste0}\NormalTok{(}\StringTok{"./RasterGrids\_100m/2024/RAW/"}\NormalTok{,nosaukums)}
\NormalTok{saglabasanas\_cels}\OtherTok{=}\FunctionTok{paste0}\NormalTok{(}\StringTok{"./RasterGrids\_100m/2024/Scaled/"}\NormalTok{,nosaukums)}
\NormalTok{slanis}\OtherTok{=}\FunctionTok{rast}\NormalTok{(ielasisanas\_cels)}
\NormalTok{videjais}\OtherTok{=}\FunctionTok{global}\NormalTok{(slanis,}\AttributeTok{fun=}\StringTok{"mean"}\NormalTok{,}\AttributeTok{na.rm=}\ConstantTok{TRUE}\NormalTok{)}
\NormalTok{centrets}\OtherTok{=}\NormalTok{slanis}\SpecialCharTok{{-}}\NormalTok{videjais[,}\DecValTok{1}\NormalTok{]}
\NormalTok{standartnovirze}\OtherTok{=}\NormalTok{terra}\SpecialCharTok{::}\FunctionTok{global}\NormalTok{(centrets,}\AttributeTok{fun=}\StringTok{"rms"}\NormalTok{,}\AttributeTok{na.rm=}\ConstantTok{TRUE}\NormalTok{)}
\NormalTok{merogots}\OtherTok{=}\NormalTok{centrets}\SpecialCharTok{/}\NormalTok{standartnovirze[,}\DecValTok{1}\NormalTok{]}
\FunctionTok{writeRaster}\NormalTok{(merogots,}
      \AttributeTok{filename=}\NormalTok{saglabasanas\_cels,}
      \AttributeTok{overwrite=}\ConstantTok{TRUE}\NormalTok{)}
\end{Highlighting}
\end{Shaded}

\section{ForestsTreesAge\_TemperateDeciduousYoung\_r10000}\label{ch06.382}

\textbf{filename:} \texttt{ForestsTreesAge\_TemperateDeciduousYoung\_r10000.tif}

\textbf{layername:} \texttt{egv\_382}

\textbf{English name:} Fractional cover of Young (pre-rotation age) Temperate
Deciduous Forests within the 10 km landscape

\textbf{Latvian name:} Jaunu (pirms cirtmeta) platlapju mežu platības īpatsvars 10 km
ainavā

\textbf{Procedure:} The cover fraction within a radius of 10000 m around the analysis grid cell
is calculated as the area-weighted sum of the \hyperref[ch06.378]{analysis cells} inside
the buffer, using the workflow \texttt{egvtools::radius\_function()}. During the calculation of the landscape
metric, inverse distance weighted (power = 2) gap filling on the output is
applied to ensure no missing values at the edges. Then the layer is
rewritten to set its name. Finally, the layer is standardised by
subtracting the arithmetic mean and dividing by the root mean squared error.

\begin{Shaded}
\begin{Highlighting}[]
\CommentTok{\# libs {-}{-}{-}{-}}
\ControlFlowTok{if}\NormalTok{(}\SpecialCharTok{!}\FunctionTok{require}\NormalTok{(terra)) \{}\FunctionTok{install.packages}\NormalTok{(}\StringTok{"terra"}\NormalTok{); }\FunctionTok{require}\NormalTok{(terra)\}}
\ControlFlowTok{if}\NormalTok{(}\SpecialCharTok{!}\FunctionTok{require}\NormalTok{(egvtools)) \{remotes}\SpecialCharTok{::}\FunctionTok{install\_github}\NormalTok{(}\StringTok{"aavotins/egvtools"}\NormalTok{); }\FunctionTok{require}\NormalTok{(egvtools)\}}


\CommentTok{\# Templates {-}{-}{-}{-}{-}}
\NormalTok{template100}\OtherTok{=}\FunctionTok{rast}\NormalTok{(}\StringTok{"./Templates/TemplateRasters/LV100m\_10km.tif"}\NormalTok{)}

\CommentTok{\# radii {-}{-}{-}{-}}
\FunctionTok{radius\_function}\NormalTok{(}
 \AttributeTok{kvadrati\_path =} \StringTok{"./Templates/TemplateGrids/tiles/"}\NormalTok{,}
 \AttributeTok{radii\_path   =} \StringTok{"./Templates/TemplateGridPoints/tiles/"}\NormalTok{,}
 \AttributeTok{tikls100\_path =} \StringTok{"./Templates/TemplateGrids/tikls100\_sauzeme.parquet"}\NormalTok{,}
 \AttributeTok{template\_path =} \StringTok{"./Templates/TemplateRasters/LV100m\_10km.tif"}\NormalTok{,}
 \AttributeTok{input\_layers  =} \FunctionTok{c}\NormalTok{(}\StringTok{"./RasterGrids\_100m/2024/RAW/ForestsTreesAge\_TemperateDeciduousYoung\_cell.tif"}\NormalTok{),}
 \AttributeTok{layer\_prefixes =} \FunctionTok{c}\NormalTok{(}\StringTok{"ForestsTreesAge\_TemperateDeciduousYoung"}\NormalTok{),}
 \AttributeTok{output\_dir   =} \StringTok{"./RasterGrids\_100m/2024/RAW/"}\NormalTok{,}
 \AttributeTok{n\_workers   =} \DecValTok{6}\NormalTok{,}
 \AttributeTok{radii     =} \FunctionTok{c}\NormalTok{(}\StringTok{"r10000"}\NormalTok{),}
 \AttributeTok{radius\_mode  =} \StringTok{"sparse"}\NormalTok{,}
 \AttributeTok{extract\_fun  =} \StringTok{"mean"}\NormalTok{,}
 \AttributeTok{fill\_missing  =} \ConstantTok{TRUE}\NormalTok{,}
 \AttributeTok{IDW\_weight   =} \DecValTok{2}\NormalTok{,}
 \AttributeTok{future\_max\_size =} \DecValTok{40} \SpecialCharTok{*} \DecValTok{1024}\SpecialCharTok{\^{}}\DecValTok{3}\NormalTok{)}


\CommentTok{\# ForestsTreesAge\_TemperateDeciduousYoung\_r10000.tif    egv\_382}
\NormalTok{slanis}\OtherTok{=}\FunctionTok{rast}\NormalTok{(}\StringTok{"./RasterGrids\_100m/2024/RAW/ForestsTreesAge\_TemperateDeciduousYoung\_r10000.tif"}\NormalTok{)}
\FunctionTok{names}\NormalTok{(slanis)}\OtherTok{=}\StringTok{"egv\_382"}
\NormalTok{slanis2}\OtherTok{=}\FunctionTok{project}\NormalTok{(slanis,template100)}
\FunctionTok{writeRaster}\NormalTok{(slanis2,}
      \StringTok{"./RasterGrids\_100m/2024/RAW/ForestsTreesAge\_TemperateDeciduousYoung\_r10000.tif"}\NormalTok{,}
      \AttributeTok{overwrite=}\ConstantTok{TRUE}\NormalTok{)}

\CommentTok{\# standardisation {-}{-}{-}{-}}
\ControlFlowTok{if}\NormalTok{(}\SpecialCharTok{!}\FunctionTok{require}\NormalTok{(terra)) \{}\FunctionTok{install.packages}\NormalTok{(}\StringTok{"terra"}\NormalTok{); }\FunctionTok{require}\NormalTok{(terra)\}}
\ControlFlowTok{if}\NormalTok{(}\SpecialCharTok{!}\FunctionTok{require}\NormalTok{(tidyverse)) \{}\FunctionTok{install.packages}\NormalTok{(}\StringTok{"tidyverse"}\NormalTok{); }\FunctionTok{require}\NormalTok{(tidyverse)\}}

\NormalTok{nosaukums}\OtherTok{=}\StringTok{"ForestsTreesAge\_TemperateDeciduousYoung\_r10000.tif"}
\NormalTok{ielasisanas\_cels}\OtherTok{=}\FunctionTok{paste0}\NormalTok{(}\StringTok{"./RasterGrids\_100m/2024/RAW/"}\NormalTok{,nosaukums)}
\NormalTok{saglabasanas\_cels}\OtherTok{=}\FunctionTok{paste0}\NormalTok{(}\StringTok{"./RasterGrids\_100m/2024/Scaled/"}\NormalTok{,nosaukums)}
\NormalTok{slanis}\OtherTok{=}\FunctionTok{rast}\NormalTok{(ielasisanas\_cels)}
\NormalTok{videjais}\OtherTok{=}\FunctionTok{global}\NormalTok{(slanis,}\AttributeTok{fun=}\StringTok{"mean"}\NormalTok{,}\AttributeTok{na.rm=}\ConstantTok{TRUE}\NormalTok{)}
\NormalTok{centrets}\OtherTok{=}\NormalTok{slanis}\SpecialCharTok{{-}}\NormalTok{videjais[,}\DecValTok{1}\NormalTok{]}
\NormalTok{standartnovirze}\OtherTok{=}\NormalTok{terra}\SpecialCharTok{::}\FunctionTok{global}\NormalTok{(centrets,}\AttributeTok{fun=}\StringTok{"rms"}\NormalTok{,}\AttributeTok{na.rm=}\ConstantTok{TRUE}\NormalTok{)}
\NormalTok{merogots}\OtherTok{=}\NormalTok{centrets}\SpecialCharTok{/}\NormalTok{standartnovirze[,}\DecValTok{1}\NormalTok{]}
\FunctionTok{writeRaster}\NormalTok{(merogots,}
      \AttributeTok{filename=}\NormalTok{saglabasanas\_cels,}
      \AttributeTok{overwrite=}\ConstantTok{TRUE}\NormalTok{)}
\end{Highlighting}
\end{Shaded}

\section{ForestsTrees\_BorealDeciduous\_cell}\label{ch06.383}

\textbf{filename:} \texttt{ForestsTrees\_BorealDeciduous\_cell.tif}

\textbf{layername:} \texttt{egv\_383}

\textbf{English name:} Fractional cover of Boreal Deciduous Forests within the
analysis cell (1 ha)

\textbf{Latvian name:} Šaurlapju mežu platības īpatsvars analīzes šūnā (1 ha)

\textbf{Procedure:} Most EGVs describing forests are spatially restricted to areas outside
of clearcuts and dead stands. This mask is created using a combination of
the \hyperref[Ch04.01]{State Forest Service's
State Forest Registry} land category 12 and 14, and \hyperref[Ch04.09]{The
Global Forest Watch} pixels classified as lost tree canopy cover since
2020 (raster layer matching input, presence = 1, absence = 0).

To prepare this EGV, stands from the \hyperref[Ch04.01]{State Forest Service's State Forest
Registry} are classified into (in order):

\begin{itemize}
\item
  coniferous (see \hyperref[Ch01]{Terminology and acronyms} for species codes) if
  timber volume of those species exceeded 75\%;
\item
  Boreal deciduous if timber volume of those species exceeded 75\%;
\item
  temperate deciduous if timber volume of those species exceeded 50\%;
\item
  mixed otherwise;
\end{itemize}

then Boreal deciduous stands are selected and geometries are
rasterised (presence = 1, NA otherwise). Rasterisation is
performed using the workflow \texttt{egvtools::polygon2input()}, restricting to pixels outside clearcut
mask and covering background with value 0. The resulting layer
is then aggregated to EGV resolution using the workflow \texttt{egvtools::input2egv()}, which
calculates the arithmetic mean to determine the cover fraction. During
aggregation, inverse distance weighted (power = 2) gap filling on the output is
applied to ensure no missing values at the edges. Finally, the layer is
standardised by subtracting the arithmetic mean and dividing by the root mean squared
error.

\begin{Shaded}
\begin{Highlighting}[]
\CommentTok{\# libs {-}{-}{-}{-}}
\ControlFlowTok{if}\NormalTok{(}\SpecialCharTok{!}\FunctionTok{require}\NormalTok{(egvtools)) \{remotes}\SpecialCharTok{::}\FunctionTok{install\_github}\NormalTok{(}\StringTok{"aavotins/egvtools"}\NormalTok{); }\FunctionTok{require}\NormalTok{(egvtools)\}}
\ControlFlowTok{if}\NormalTok{(}\SpecialCharTok{!}\FunctionTok{require}\NormalTok{(terra)) \{}\FunctionTok{install.packages}\NormalTok{(}\StringTok{"terra"}\NormalTok{); }\FunctionTok{require}\NormalTok{(terra)\}}
\ControlFlowTok{if}\NormalTok{(}\SpecialCharTok{!}\FunctionTok{require}\NormalTok{(sf)) \{}\FunctionTok{install.packages}\NormalTok{(}\StringTok{"sf"}\NormalTok{); }\FunctionTok{require}\NormalTok{(sf)\}}
\ControlFlowTok{if}\NormalTok{(}\SpecialCharTok{!}\FunctionTok{require}\NormalTok{(tidyverse)) \{}\FunctionTok{install.packages}\NormalTok{(}\StringTok{"tidyverse"}\NormalTok{); }\FunctionTok{require}\NormalTok{(tidyverse)\}}
\ControlFlowTok{if}\NormalTok{(}\SpecialCharTok{!}\FunctionTok{require}\NormalTok{(sfarrow)) \{}\FunctionTok{install.packages}\NormalTok{(}\StringTok{"sfarrow"}\NormalTok{); }\FunctionTok{require}\NormalTok{(sfarrow)\}}
\ControlFlowTok{if}\NormalTok{(}\SpecialCharTok{!}\FunctionTok{require}\NormalTok{(readxl)) \{}\FunctionTok{install.packages}\NormalTok{(}\StringTok{"readxl"}\NormalTok{); }\FunctionTok{require}\NormalTok{(readxl)\}}
\ControlFlowTok{if}\NormalTok{(}\SpecialCharTok{!}\FunctionTok{require}\NormalTok{(raster)) \{}\FunctionTok{install.packages}\NormalTok{(}\StringTok{"raster"}\NormalTok{); }\FunctionTok{require}\NormalTok{(raster)\}}
\ControlFlowTok{if}\NormalTok{(}\SpecialCharTok{!}\FunctionTok{require}\NormalTok{(fasterize)) \{}\FunctionTok{install.packages}\NormalTok{(}\StringTok{"fasterize"}\NormalTok{); }\FunctionTok{require}\NormalTok{(fasterize)\}}

\CommentTok{\# templates {-}{-}{-}{-}}
\NormalTok{template100}\OtherTok{=}\FunctionTok{rast}\NormalTok{(}\StringTok{"./Templates/TemplateRasters/LV100m\_10km.tif"}\NormalTok{)}
\NormalTok{template10}\OtherTok{=}\FunctionTok{rast}\NormalTok{(}\StringTok{"./Templates/TemplateRasters/LV10m\_10km.tif"}\NormalTok{)}
\NormalTok{rastrs10}\OtherTok{=}\FunctionTok{raster}\NormalTok{(template10)}

\NormalTok{nulls10}\OtherTok{=}\FunctionTok{rast}\NormalTok{(}\StringTok{"./Templates/TemplateRasters/nulls\_LV10m\_10km.tif"}\NormalTok{)}
\NormalTok{nulls100}\OtherTok{=}\FunctionTok{rast}\NormalTok{(}\StringTok{"./Templates/TemplateRasters/nulls\_LV100m\_10km.tif"}\NormalTok{)}


\CommentTok{\# simple landscape {-}{-}{-}{-}}
\NormalTok{simple\_landscape}\OtherTok{=}\FunctionTok{rast}\NormalTok{(}\StringTok{"RasterGrids\_10m/2024/Ainava\_vienk\_mask.tif"}\NormalTok{)}

\CommentTok{\# mvr {-}{-}{-}{-}}
\NormalTok{mvr}\OtherTok{=}\FunctionTok{st\_read\_parquet}\NormalTok{(}\StringTok{"./Geodata/2024/MVR/nogabali\_2024janv.parquet"}\NormalTok{)}
\NormalTok{mvr}\SpecialCharTok{$}\NormalTok{yes}\OtherTok{=}\DecValTok{1}

\CommentTok{\# clear cut mask {-}{-}{-}{-}}
\NormalTok{izcirtumi}\OtherTok{=}\NormalTok{mvr }\SpecialCharTok{\%\textgreater{}\%} 
 \FunctionTok{filter}\NormalTok{(zkat }\SpecialCharTok{\%in\%} \FunctionTok{c}\NormalTok{(}\StringTok{"12"}\NormalTok{,}\StringTok{"14"}\NormalTok{)) }\SpecialCharTok{\%\textgreater{}\%} 
\NormalTok{ dplyr}\SpecialCharTok{::}\FunctionTok{select}\NormalTok{(yes)}
\NormalTok{r\_izcirtumi\_mvr}\OtherTok{=}\FunctionTok{fasterize}\NormalTok{(izcirtumi,rastrs10,}\AttributeTok{field=}\StringTok{"yes"}\NormalTok{)}
\NormalTok{t\_izcirtumi\_mvr}\OtherTok{=}\FunctionTok{rast}\NormalTok{(r\_izcirtumi\_mvr)}
\FunctionTok{plot}\NormalTok{(t\_izcirtumi\_mvr)}

\NormalTok{tcl}\OtherTok{=}\FunctionTok{rast}\NormalTok{(}\StringTok{"./Geodata/2024/Trees/GFW/TreeCoverLoss\_v1\_12.tif"}\NormalTok{)}
\NormalTok{tcl2}\OtherTok{=}\FunctionTok{ifel}\NormalTok{(tcl}\SpecialCharTok{\textless{}}\DecValTok{20}\NormalTok{,}\DecValTok{0}\NormalTok{,}\DecValTok{1}\NormalTok{)}
\NormalTok{tclX}\OtherTok{=}\FunctionTok{cover}\NormalTok{(tcl2,nulls10)}
\FunctionTok{plot}\NormalTok{(tclX)}

\NormalTok{clearcut\_mask}\OtherTok{=}\FunctionTok{cover}\NormalTok{(t\_izcirtumi\_mvr,tclX,}
          \AttributeTok{filename=}\StringTok{"./RasterGrids\_10m/2024/Mask\_clearcuts.tif"}\NormalTok{,}
          \AttributeTok{overwrite=}\ConstantTok{TRUE}\NormalTok{)}
\FunctionTok{plot}\NormalTok{(clearcut\_mask)}

\FunctionTok{rm}\NormalTok{(izcirtumi)}
\FunctionTok{rm}\NormalTok{(r\_izcirtumi\_mvr)}
\FunctionTok{rm}\NormalTok{(t\_izcirtumi\_mvr)}
\FunctionTok{rm}\NormalTok{(tcl)}
\FunctionTok{rm}\NormalTok{(tcl2)}
\FunctionTok{rm}\NormalTok{(tclX)}

\CommentTok{\# ForestsTrees\_BorealDeciduous\_cell.tif egv\_383 {-}{-}{-}{-}}
\NormalTok{skujkoki}\OtherTok{=}\FunctionTok{c}\NormalTok{(}\StringTok{"1"}\NormalTok{,}\StringTok{"3"}\NormalTok{,}\StringTok{"13"}\NormalTok{,}\StringTok{"14"}\NormalTok{,}\StringTok{"15"}\NormalTok{,}\StringTok{"22"}\NormalTok{,}\StringTok{"23"}\NormalTok{,}\StringTok{"28"}\NormalTok{) }\CommentTok{\# 8}
\NormalTok{saurlapji}\OtherTok{=}\FunctionTok{c}\NormalTok{(}\StringTok{"4"}\NormalTok{,}\StringTok{"6"}\NormalTok{,}\StringTok{"8"}\NormalTok{,}\StringTok{"9"}\NormalTok{,}\StringTok{"19"}\NormalTok{,}\StringTok{"20"}\NormalTok{,}\StringTok{"21"}\NormalTok{,}\StringTok{"32"}\NormalTok{,}\StringTok{"35"}\NormalTok{,}\StringTok{"68"}\NormalTok{) }\CommentTok{\# 10}
\NormalTok{platlapji}\OtherTok{=}\FunctionTok{c}\NormalTok{(}\StringTok{"10"}\NormalTok{,}\StringTok{"11"}\NormalTok{,}\StringTok{"12"}\NormalTok{,}\StringTok{"16"}\NormalTok{,}\StringTok{"17"}\NormalTok{,}\StringTok{"18"}\NormalTok{,}\StringTok{"24"}\NormalTok{,}\StringTok{"25"}\NormalTok{,}\StringTok{"26"}\NormalTok{,}\StringTok{"27"}\NormalTok{,}\StringTok{"28"}\NormalTok{,}\StringTok{"29"}\NormalTok{,}\StringTok{"50"}\NormalTok{,}
      \StringTok{"61"}\NormalTok{,}\StringTok{"62"}\NormalTok{,}\StringTok{"63"}\NormalTok{,}\StringTok{"64"}\NormalTok{,}\StringTok{"65"}\NormalTok{,}\StringTok{"66"}\NormalTok{,}\StringTok{"67"}\NormalTok{,}\StringTok{"69"}\NormalTok{) }\CommentTok{\# 21}
\NormalTok{mvr}\OtherTok{=}\NormalTok{mvr }\SpecialCharTok{\%\textgreater{}\%} 
 \FunctionTok{mutate}\NormalTok{(}\AttributeTok{kraja\_skujkoku=}\FunctionTok{ifelse}\NormalTok{(s10 }\SpecialCharTok{\%in\%}\NormalTok{ skujkoki,v10,}\DecValTok{0}\NormalTok{)}\SpecialCharTok{+}
      \FunctionTok{ifelse}\NormalTok{(s11 }\SpecialCharTok{\%in\%}\NormalTok{ skujkoki,v11,}\DecValTok{0}\NormalTok{)}\SpecialCharTok{+}\FunctionTok{ifelse}\NormalTok{(s12 }\SpecialCharTok{\%in\%}\NormalTok{ skujkoki,v12,}\DecValTok{0}\NormalTok{)}\SpecialCharTok{+}
      \FunctionTok{ifelse}\NormalTok{(s13 }\SpecialCharTok{\%in\%}\NormalTok{ skujkoki,v13,}\DecValTok{0}\NormalTok{)}\SpecialCharTok{+}\FunctionTok{ifelse}\NormalTok{(s14 }\SpecialCharTok{\%in\%}\NormalTok{ skujkoki,v14,}\DecValTok{0}\NormalTok{),}
     \AttributeTok{kraja\_saurlapju=}\FunctionTok{ifelse}\NormalTok{(s10 }\SpecialCharTok{\%in\%}\NormalTok{ saurlapji,v10,}\DecValTok{0}\NormalTok{)}\SpecialCharTok{+}
      \FunctionTok{ifelse}\NormalTok{(s11 }\SpecialCharTok{\%in\%}\NormalTok{ saurlapji,v11,}\DecValTok{0}\NormalTok{)}\SpecialCharTok{+}\FunctionTok{ifelse}\NormalTok{(s12 }\SpecialCharTok{\%in\%}\NormalTok{ saurlapji,v12,}\DecValTok{0}\NormalTok{)}\SpecialCharTok{+}
      \FunctionTok{ifelse}\NormalTok{(s13 }\SpecialCharTok{\%in\%}\NormalTok{ saurlapji,v13,}\DecValTok{0}\NormalTok{)}\SpecialCharTok{+}\FunctionTok{ifelse}\NormalTok{(s14 }\SpecialCharTok{\%in\%}\NormalTok{ saurlapji,v14,}\DecValTok{0}\NormalTok{),}
     \AttributeTok{kraja\_platlapju=}\FunctionTok{ifelse}\NormalTok{(s10 }\SpecialCharTok{\%in\%}\NormalTok{ platlapji,v10,}\DecValTok{0}\NormalTok{)}\SpecialCharTok{+}
      \FunctionTok{ifelse}\NormalTok{(s11 }\SpecialCharTok{\%in\%}\NormalTok{ platlapji,v11,}\DecValTok{0}\NormalTok{)}\SpecialCharTok{+}\FunctionTok{ifelse}\NormalTok{(s12 }\SpecialCharTok{\%in\%}\NormalTok{ platlapji,v12,}\DecValTok{0}\NormalTok{)}\SpecialCharTok{+}
      \FunctionTok{ifelse}\NormalTok{(s13 }\SpecialCharTok{\%in\%}\NormalTok{ platlapji,v13,}\DecValTok{0}\NormalTok{)}\SpecialCharTok{+}\FunctionTok{ifelse}\NormalTok{(s14 }\SpecialCharTok{\%in\%}\NormalTok{ platlapji,v14,}\DecValTok{0}\NormalTok{)) }\SpecialCharTok{\%\textgreater{}\%} 
 \FunctionTok{mutate}\NormalTok{(}\AttributeTok{kopeja\_kraja=}\NormalTok{kraja\_skujkoku}\SpecialCharTok{+}\NormalTok{kraja\_platlapju}\SpecialCharTok{+}\NormalTok{kraja\_saurlapju) }\SpecialCharTok{\%\textgreater{}\%} 
 \FunctionTok{mutate}\NormalTok{(}\AttributeTok{tips=}\FunctionTok{ifelse}\NormalTok{(kraja\_skujkoku}\SpecialCharTok{/}\NormalTok{kopeja\_kraja}\SpecialCharTok{\textgreater{}=}\FloatTok{0.75}\NormalTok{,}\StringTok{"skujkoku"}\NormalTok{,}
           \FunctionTok{ifelse}\NormalTok{(kraja\_saurlapju}\SpecialCharTok{/}\NormalTok{kopeja\_kraja}\SpecialCharTok{\textgreater{}=}\FloatTok{0.75}\NormalTok{,}\StringTok{"saurlapju"}\NormalTok{,}
              \FunctionTok{ifelse}\NormalTok{(kraja\_platlapju}\SpecialCharTok{/}\NormalTok{kopeja\_kraja}\SpecialCharTok{\textgreater{}}\FloatTok{0.5}\NormalTok{,}\StringTok{"platlapju"}\NormalTok{,}
                  \StringTok{"jauktu koku"}\NormalTok{))))}
\NormalTok{nogabali}\OtherTok{=}\NormalTok{mvr }\SpecialCharTok{\%\textgreater{}\%} 
 \FunctionTok{filter}\NormalTok{(zkat}\SpecialCharTok{==}\StringTok{"10"}\SpecialCharTok{\&}\NormalTok{tips}\SpecialCharTok{==}\StringTok{"saurlapju"}\NormalTok{)}

\NormalTok{p2i\_rez}\OtherTok{=}\NormalTok{egvtools}\SpecialCharTok{::}\FunctionTok{polygon2input}\NormalTok{(}\AttributeTok{vector\_data =}\NormalTok{ nogabali,}
                \AttributeTok{template\_path =} \StringTok{"./Templates/TemplateRasters/LV10m\_10km.tif"}\NormalTok{,}
                \AttributeTok{out\_path =} \StringTok{"./RasterGrids\_10m/2024/"}\NormalTok{,}
                \AttributeTok{file\_name =} \StringTok{"ForestsTrees\_BorealDeciduous\_input.tif"}\NormalTok{,}
                \AttributeTok{value\_field =} \StringTok{"yes"}\NormalTok{,}
                \AttributeTok{restrict\_to =}\NormalTok{ clearcut\_mask,}
                \AttributeTok{restrict\_values =} \DecValTok{0}\NormalTok{,}
                \AttributeTok{prepare=}\ConstantTok{FALSE}\NormalTok{,}
                \AttributeTok{background\_raster =} \StringTok{"./Templates/TemplateRasters/nulls\_LV10m\_10km.tif"}\NormalTok{,}
                \AttributeTok{plot\_result =} \ConstantTok{TRUE}\NormalTok{)}
\NormalTok{p2i\_rez}
\NormalTok{i2e\_rez}\OtherTok{=}\NormalTok{egvtools}\SpecialCharTok{::}\FunctionTok{input2egv}\NormalTok{(}\AttributeTok{input=}\FunctionTok{paste0}\NormalTok{(}\StringTok{"./RasterGrids\_10m/2024/"}\NormalTok{,}
                     \StringTok{"ForestsTrees\_BorealDeciduous\_input.tif"}\NormalTok{),}
              \AttributeTok{egv\_template=} \StringTok{"./Templates/TemplateRasters/LV100m\_10km.tif"}\NormalTok{,}
              \AttributeTok{summary\_function =} \StringTok{"average"}\NormalTok{,}
              \AttributeTok{missing\_job =} \StringTok{"FillOutput"}\NormalTok{,}
              \AttributeTok{outlocation =} \StringTok{"./RasterGrids\_100m/2024/RAW/"}\NormalTok{,}
              \AttributeTok{outfilename =} \StringTok{"ForestsTrees\_BorealDeciduous\_cell.tif"}\NormalTok{,}
              \AttributeTok{layername =} \StringTok{"egv\_383"}\NormalTok{,}
              \AttributeTok{idw\_weight =} \DecValTok{2}\NormalTok{,}
              \AttributeTok{plot\_gaps =} \ConstantTok{FALSE}\NormalTok{,}\AttributeTok{plot\_final =} \ConstantTok{TRUE}\NormalTok{)}
\NormalTok{i2e\_rez}
\FunctionTok{rm}\NormalTok{(nogabali)}
\FunctionTok{rm}\NormalTok{(p2i\_rez)}
\FunctionTok{rm}\NormalTok{(i2e\_rez)}
\FunctionTok{unlink}\NormalTok{(}\StringTok{"./RasterGrids\_10m/2024/ForestsTrees\_BorealDeciduous\_input.tif"}\NormalTok{)}

\CommentTok{\# standardisation {-}{-}{-}{-}}
\ControlFlowTok{if}\NormalTok{(}\SpecialCharTok{!}\FunctionTok{require}\NormalTok{(terra)) \{}\FunctionTok{install.packages}\NormalTok{(}\StringTok{"terra"}\NormalTok{); }\FunctionTok{require}\NormalTok{(terra)\}}
\ControlFlowTok{if}\NormalTok{(}\SpecialCharTok{!}\FunctionTok{require}\NormalTok{(tidyverse)) \{}\FunctionTok{install.packages}\NormalTok{(}\StringTok{"tidyverse"}\NormalTok{); }\FunctionTok{require}\NormalTok{(tidyverse)\}}

\NormalTok{nosaukums}\OtherTok{=}\StringTok{"ForestsTrees\_BorealDeciduous\_cell.tif"}
\NormalTok{ielasisanas\_cels}\OtherTok{=}\FunctionTok{paste0}\NormalTok{(}\StringTok{"./RasterGrids\_100m/2024/RAW/"}\NormalTok{,nosaukums)}
\NormalTok{saglabasanas\_cels}\OtherTok{=}\FunctionTok{paste0}\NormalTok{(}\StringTok{"./RasterGrids\_100m/2024/Scaled/"}\NormalTok{,nosaukums)}
\NormalTok{slanis}\OtherTok{=}\FunctionTok{rast}\NormalTok{(ielasisanas\_cels)}
\NormalTok{videjais}\OtherTok{=}\FunctionTok{global}\NormalTok{(slanis,}\AttributeTok{fun=}\StringTok{"mean"}\NormalTok{,}\AttributeTok{na.rm=}\ConstantTok{TRUE}\NormalTok{)}
\NormalTok{centrets}\OtherTok{=}\NormalTok{slanis}\SpecialCharTok{{-}}\NormalTok{videjais[,}\DecValTok{1}\NormalTok{]}
\NormalTok{standartnovirze}\OtherTok{=}\NormalTok{terra}\SpecialCharTok{::}\FunctionTok{global}\NormalTok{(centrets,}\AttributeTok{fun=}\StringTok{"rms"}\NormalTok{,}\AttributeTok{na.rm=}\ConstantTok{TRUE}\NormalTok{)}
\NormalTok{merogots}\OtherTok{=}\NormalTok{centrets}\SpecialCharTok{/}\NormalTok{standartnovirze[,}\DecValTok{1}\NormalTok{]}
\FunctionTok{writeRaster}\NormalTok{(merogots,}
      \AttributeTok{filename=}\NormalTok{saglabasanas\_cels,}
      \AttributeTok{overwrite=}\ConstantTok{TRUE}\NormalTok{)}
\end{Highlighting}
\end{Shaded}

\section{ForestsTrees\_BorealDeciduous\_r500}\label{ch06.384}

\textbf{filename:} \texttt{ForestsTrees\_BorealDeciduous\_r500.tif}

\textbf{layername:} \texttt{egv\_384}

\textbf{English name:} Fractional cover of Boreal Deciduous Forests within the 0.5 km
landscape

\textbf{Latvian name:} Šaurlapju mežu platības īpatsvars 0,5 km ainavā

\textbf{Procedure:} The cover fraction within a radius of 500 m around the analysis grid cell is
calculated as the area-weighted sum of the \hyperref[ch06.383]{analysis cells} inside the
buffer, using the workflow \texttt{egvtools::radius\_function()}. During the calculation of the landscape metric,
inverse distance weighted (power = 2) gap filling on the output is applied
to ensure no missing values at the edges. Then the layer is rewritten to set
its name. Finally, the layer is standardised by subtracting the arithmetic
mean and dividing by the root mean squared error.

\begin{Shaded}
\begin{Highlighting}[]
\CommentTok{\# libs {-}{-}{-}{-}}
\ControlFlowTok{if}\NormalTok{(}\SpecialCharTok{!}\FunctionTok{require}\NormalTok{(terra)) \{}\FunctionTok{install.packages}\NormalTok{(}\StringTok{"terra"}\NormalTok{); }\FunctionTok{require}\NormalTok{(terra)\}}
\ControlFlowTok{if}\NormalTok{(}\SpecialCharTok{!}\FunctionTok{require}\NormalTok{(egvtools)) \{remotes}\SpecialCharTok{::}\FunctionTok{install\_github}\NormalTok{(}\StringTok{"aavotins/egvtools"}\NormalTok{); }\FunctionTok{require}\NormalTok{(egvtools)\}}


\CommentTok{\# Templates {-}{-}{-}{-}{-}}
\NormalTok{template100}\OtherTok{=}\FunctionTok{rast}\NormalTok{(}\StringTok{"./Templates/TemplateRasters/LV100m\_10km.tif"}\NormalTok{)}

\CommentTok{\# radii {-}{-}{-}{-}}
\FunctionTok{radius\_function}\NormalTok{(}
 \AttributeTok{kvadrati\_path =} \StringTok{"./Templates/TemplateGrids/tiles/"}\NormalTok{,}
 \AttributeTok{radii\_path   =} \StringTok{"./Templates/TemplateGridPoints/tiles/"}\NormalTok{,}
 \AttributeTok{tikls100\_path =} \StringTok{"./Templates/TemplateGrids/tikls100\_sauzeme.parquet"}\NormalTok{,}
 \AttributeTok{template\_path =} \StringTok{"./Templates/TemplateRasters/LV100m\_10km.tif"}\NormalTok{,}
 \AttributeTok{input\_layers  =} \FunctionTok{c}\NormalTok{(}\StringTok{"./RasterGrids\_100m/2024/RAW/ForestsTrees\_BorealDeciduous\_cell.tif"}\NormalTok{),}
 \AttributeTok{layer\_prefixes =} \FunctionTok{c}\NormalTok{(}\StringTok{"ForestsTrees\_BorealDeciduous"}\NormalTok{),}
 \AttributeTok{output\_dir   =} \StringTok{"./RasterGrids\_100m/2024/RAW/"}\NormalTok{,}
 \AttributeTok{n\_workers   =} \DecValTok{6}\NormalTok{,}
 \AttributeTok{radii     =} \FunctionTok{c}\NormalTok{(}\StringTok{"r500"}\NormalTok{),}
 \AttributeTok{radius\_mode  =} \StringTok{"sparse"}\NormalTok{,}
 \AttributeTok{extract\_fun  =} \StringTok{"mean"}\NormalTok{,}
 \AttributeTok{fill\_missing  =} \ConstantTok{TRUE}\NormalTok{,}
 \AttributeTok{IDW\_weight   =} \DecValTok{2}\NormalTok{,}
 \AttributeTok{future\_max\_size =} \DecValTok{40} \SpecialCharTok{*} \DecValTok{1024}\SpecialCharTok{\^{}}\DecValTok{3}\NormalTok{)}


\CommentTok{\# ForestsTrees\_BorealDeciduous\_r500.tif egv\_384}
\NormalTok{slanis}\OtherTok{=}\FunctionTok{rast}\NormalTok{(}\StringTok{"./RasterGrids\_100m/2024/RAW/ForestsTrees\_BorealDeciduous\_r500.tif"}\NormalTok{)}
\FunctionTok{names}\NormalTok{(slanis)}\OtherTok{=}\StringTok{"egv\_384"}
\NormalTok{slanis2}\OtherTok{=}\FunctionTok{project}\NormalTok{(slanis,template100)}
\FunctionTok{writeRaster}\NormalTok{(slanis2,}
      \StringTok{"./RasterGrids\_100m/2024/RAW/ForestsTrees\_BorealDeciduous\_r500.tif"}\NormalTok{,}
      \AttributeTok{overwrite=}\ConstantTok{TRUE}\NormalTok{)}

\CommentTok{\# standardisation {-}{-}{-}{-}}
\ControlFlowTok{if}\NormalTok{(}\SpecialCharTok{!}\FunctionTok{require}\NormalTok{(terra)) \{}\FunctionTok{install.packages}\NormalTok{(}\StringTok{"terra"}\NormalTok{); }\FunctionTok{require}\NormalTok{(terra)\}}
\ControlFlowTok{if}\NormalTok{(}\SpecialCharTok{!}\FunctionTok{require}\NormalTok{(tidyverse)) \{}\FunctionTok{install.packages}\NormalTok{(}\StringTok{"tidyverse"}\NormalTok{); }\FunctionTok{require}\NormalTok{(tidyverse)\}}

\NormalTok{nosaukums}\OtherTok{=}\StringTok{"ForestsTrees\_BorealDeciduous\_r500.tif"}
\NormalTok{ielasisanas\_cels}\OtherTok{=}\FunctionTok{paste0}\NormalTok{(}\StringTok{"./RasterGrids\_100m/2024/RAW/"}\NormalTok{,nosaukums)}
\NormalTok{saglabasanas\_cels}\OtherTok{=}\FunctionTok{paste0}\NormalTok{(}\StringTok{"./RasterGrids\_100m/2024/Scaled/"}\NormalTok{,nosaukums)}
\NormalTok{slanis}\OtherTok{=}\FunctionTok{rast}\NormalTok{(ielasisanas\_cels)}
\NormalTok{videjais}\OtherTok{=}\FunctionTok{global}\NormalTok{(slanis,}\AttributeTok{fun=}\StringTok{"mean"}\NormalTok{,}\AttributeTok{na.rm=}\ConstantTok{TRUE}\NormalTok{)}
\NormalTok{centrets}\OtherTok{=}\NormalTok{slanis}\SpecialCharTok{{-}}\NormalTok{videjais[,}\DecValTok{1}\NormalTok{]}
\NormalTok{standartnovirze}\OtherTok{=}\NormalTok{terra}\SpecialCharTok{::}\FunctionTok{global}\NormalTok{(centrets,}\AttributeTok{fun=}\StringTok{"rms"}\NormalTok{,}\AttributeTok{na.rm=}\ConstantTok{TRUE}\NormalTok{)}
\NormalTok{merogots}\OtherTok{=}\NormalTok{centrets}\SpecialCharTok{/}\NormalTok{standartnovirze[,}\DecValTok{1}\NormalTok{]}
\FunctionTok{writeRaster}\NormalTok{(merogots,}
      \AttributeTok{filename=}\NormalTok{saglabasanas\_cels,}
      \AttributeTok{overwrite=}\ConstantTok{TRUE}\NormalTok{)}
\end{Highlighting}
\end{Shaded}

\section{ForestsTrees\_BorealDeciduous\_r1250}\label{ch06.385}

\textbf{filename:} \texttt{ForestsTrees\_BorealDeciduous\_r1250.tif}

\textbf{layername:} \texttt{egv\_385}

\textbf{English name:} Fractional cover of Boreal Deciduous Forests within the 1.25
km landscape

\textbf{Latvian name:} Šaurlapju mežu platības īpatsvars 1,25 km ainavā

\textbf{Procedure:} The cover fraction within a radius of 1250 m around the analysis grid cell
is calculated as the area-weighted sum of the \hyperref[ch06.383]{analysis cells} inside
the buffer, using the workflow \texttt{egvtools::radius\_function()}. During the calculation of the landscape
metric, inverse distance weighted (power = 2) gap filling on the output is
applied to ensure no missing values at the edges. Then the layer is
rewritten to set its name. Finally, the layer is standardised by
subtracting the arithmetic mean and dividing by the root mean squared error.

\begin{Shaded}
\begin{Highlighting}[]
\CommentTok{\# libs {-}{-}{-}{-}}
\ControlFlowTok{if}\NormalTok{(}\SpecialCharTok{!}\FunctionTok{require}\NormalTok{(terra)) \{}\FunctionTok{install.packages}\NormalTok{(}\StringTok{"terra"}\NormalTok{); }\FunctionTok{require}\NormalTok{(terra)\}}
\ControlFlowTok{if}\NormalTok{(}\SpecialCharTok{!}\FunctionTok{require}\NormalTok{(egvtools)) \{remotes}\SpecialCharTok{::}\FunctionTok{install\_github}\NormalTok{(}\StringTok{"aavotins/egvtools"}\NormalTok{); }\FunctionTok{require}\NormalTok{(egvtools)\}}


\CommentTok{\# Templates {-}{-}{-}{-}{-}}
\NormalTok{template100}\OtherTok{=}\FunctionTok{rast}\NormalTok{(}\StringTok{"./Templates/TemplateRasters/LV100m\_10km.tif"}\NormalTok{)}

\CommentTok{\# radii {-}{-}{-}{-}}
\FunctionTok{radius\_function}\NormalTok{(}
 \AttributeTok{kvadrati\_path =} \StringTok{"./Templates/TemplateGrids/tiles/"}\NormalTok{,}
 \AttributeTok{radii\_path   =} \StringTok{"./Templates/TemplateGridPoints/tiles/"}\NormalTok{,}
 \AttributeTok{tikls100\_path =} \StringTok{"./Templates/TemplateGrids/tikls100\_sauzeme.parquet"}\NormalTok{,}
 \AttributeTok{template\_path =} \StringTok{"./Templates/TemplateRasters/LV100m\_10km.tif"}\NormalTok{,}
 \AttributeTok{input\_layers  =} \FunctionTok{c}\NormalTok{(}\StringTok{"./RasterGrids\_100m/2024/RAW/ForestsTrees\_BorealDeciduous\_cell.tif"}\NormalTok{),}
 \AttributeTok{layer\_prefixes =} \FunctionTok{c}\NormalTok{(}\StringTok{"ForestsTrees\_BorealDeciduous"}\NormalTok{),}
 \AttributeTok{output\_dir   =} \StringTok{"./RasterGrids\_100m/2024/RAW/"}\NormalTok{,}
 \AttributeTok{n\_workers   =} \DecValTok{6}\NormalTok{,}
 \AttributeTok{radii     =} \FunctionTok{c}\NormalTok{(}\StringTok{"r1250"}\NormalTok{),}
 \AttributeTok{radius\_mode  =} \StringTok{"sparse"}\NormalTok{,}
 \AttributeTok{extract\_fun  =} \StringTok{"mean"}\NormalTok{,}
 \AttributeTok{fill\_missing  =} \ConstantTok{TRUE}\NormalTok{,}
 \AttributeTok{IDW\_weight   =} \DecValTok{2}\NormalTok{,}
 \AttributeTok{future\_max\_size =} \DecValTok{40} \SpecialCharTok{*} \DecValTok{1024}\SpecialCharTok{\^{}}\DecValTok{3}\NormalTok{)}


\CommentTok{\# ForestsTrees\_BorealDeciduous\_r1250.tif    egv\_385}
\NormalTok{slanis}\OtherTok{=}\FunctionTok{rast}\NormalTok{(}\StringTok{"./RasterGrids\_100m/2024/RAW/ForestsTrees\_BorealDeciduous\_r1250.tif"}\NormalTok{)}
\FunctionTok{names}\NormalTok{(slanis)}\OtherTok{=}\StringTok{"egv\_385"}
\NormalTok{slanis2}\OtherTok{=}\FunctionTok{project}\NormalTok{(slanis,template100)}
\FunctionTok{writeRaster}\NormalTok{(slanis2,}
      \StringTok{"./RasterGrids\_100m/2024/RAW/ForestsTrees\_BorealDeciduous\_r1250.tif"}\NormalTok{,}
      \AttributeTok{overwrite=}\ConstantTok{TRUE}\NormalTok{)}

\CommentTok{\# standardisation {-}{-}{-}{-}}
\ControlFlowTok{if}\NormalTok{(}\SpecialCharTok{!}\FunctionTok{require}\NormalTok{(terra)) \{}\FunctionTok{install.packages}\NormalTok{(}\StringTok{"terra"}\NormalTok{); }\FunctionTok{require}\NormalTok{(terra)\}}
\ControlFlowTok{if}\NormalTok{(}\SpecialCharTok{!}\FunctionTok{require}\NormalTok{(tidyverse)) \{}\FunctionTok{install.packages}\NormalTok{(}\StringTok{"tidyverse"}\NormalTok{); }\FunctionTok{require}\NormalTok{(tidyverse)\}}

\NormalTok{nosaukums}\OtherTok{=}\StringTok{"ForestsTrees\_BorealDeciduous\_r1250.tif"}
\NormalTok{ielasisanas\_cels}\OtherTok{=}\FunctionTok{paste0}\NormalTok{(}\StringTok{"./RasterGrids\_100m/2024/RAW/"}\NormalTok{,nosaukums)}
\NormalTok{saglabasanas\_cels}\OtherTok{=}\FunctionTok{paste0}\NormalTok{(}\StringTok{"./RasterGrids\_100m/2024/Scaled/"}\NormalTok{,nosaukums)}
\NormalTok{slanis}\OtherTok{=}\FunctionTok{rast}\NormalTok{(ielasisanas\_cels)}
\NormalTok{videjais}\OtherTok{=}\FunctionTok{global}\NormalTok{(slanis,}\AttributeTok{fun=}\StringTok{"mean"}\NormalTok{,}\AttributeTok{na.rm=}\ConstantTok{TRUE}\NormalTok{)}
\NormalTok{centrets}\OtherTok{=}\NormalTok{slanis}\SpecialCharTok{{-}}\NormalTok{videjais[,}\DecValTok{1}\NormalTok{]}
\NormalTok{standartnovirze}\OtherTok{=}\NormalTok{terra}\SpecialCharTok{::}\FunctionTok{global}\NormalTok{(centrets,}\AttributeTok{fun=}\StringTok{"rms"}\NormalTok{,}\AttributeTok{na.rm=}\ConstantTok{TRUE}\NormalTok{)}
\NormalTok{merogots}\OtherTok{=}\NormalTok{centrets}\SpecialCharTok{/}\NormalTok{standartnovirze[,}\DecValTok{1}\NormalTok{]}
\FunctionTok{writeRaster}\NormalTok{(merogots,}
      \AttributeTok{filename=}\NormalTok{saglabasanas\_cels,}
      \AttributeTok{overwrite=}\ConstantTok{TRUE}\NormalTok{)}
\end{Highlighting}
\end{Shaded}

\section{ForestsTrees\_BorealDeciduous\_r3000}\label{ch06.386}

\textbf{filename:} \texttt{ForestsTrees\_BorealDeciduous\_r3000.tif}

\textbf{layername:} \texttt{egv\_386}

\textbf{English name:} Fractional cover of Boreal Deciduous Forests within the 3 km
landscape

\textbf{Latvian name:} Šaurlapju mežu platības īpatsvars 3 km ainavā

\textbf{Procedure:} The cover fraction within a radius of 3000 m around the analysis grid cell
is calculated as the area-weighted sum of the \hyperref[ch06.383]{analysis cells} inside
the buffer, using the workflow \texttt{egvtools::radius\_function()}. During the calculation of the landscape
metric, inverse distance weighted (power = 2) gap filling on the output is
applied to ensure no missing values at the edges. Then the layer is
rewritten to set its name. Finally, the layer is standardised by
subtracting the arithmetic mean and dividing by the root mean squared error.

\begin{Shaded}
\begin{Highlighting}[]
\CommentTok{\# libs {-}{-}{-}{-}}
\ControlFlowTok{if}\NormalTok{(}\SpecialCharTok{!}\FunctionTok{require}\NormalTok{(terra)) \{}\FunctionTok{install.packages}\NormalTok{(}\StringTok{"terra"}\NormalTok{); }\FunctionTok{require}\NormalTok{(terra)\}}
\ControlFlowTok{if}\NormalTok{(}\SpecialCharTok{!}\FunctionTok{require}\NormalTok{(egvtools)) \{remotes}\SpecialCharTok{::}\FunctionTok{install\_github}\NormalTok{(}\StringTok{"aavotins/egvtools"}\NormalTok{); }\FunctionTok{require}\NormalTok{(egvtools)\}}


\CommentTok{\# Templates {-}{-}{-}{-}{-}}
\NormalTok{template100}\OtherTok{=}\FunctionTok{rast}\NormalTok{(}\StringTok{"./Templates/TemplateRasters/LV100m\_10km.tif"}\NormalTok{)}

\CommentTok{\# radii {-}{-}{-}{-}}
\FunctionTok{radius\_function}\NormalTok{(}
 \AttributeTok{kvadrati\_path =} \StringTok{"./Templates/TemplateGrids/tiles/"}\NormalTok{,}
 \AttributeTok{radii\_path   =} \StringTok{"./Templates/TemplateGridPoints/tiles/"}\NormalTok{,}
 \AttributeTok{tikls100\_path =} \StringTok{"./Templates/TemplateGrids/tikls100\_sauzeme.parquet"}\NormalTok{,}
 \AttributeTok{template\_path =} \StringTok{"./Templates/TemplateRasters/LV100m\_10km.tif"}\NormalTok{,}
 \AttributeTok{input\_layers  =} \FunctionTok{c}\NormalTok{(}\StringTok{"./RasterGrids\_100m/2024/RAW/ForestsTrees\_BorealDeciduous\_cell.tif"}\NormalTok{),}
 \AttributeTok{layer\_prefixes =} \FunctionTok{c}\NormalTok{(}\StringTok{"ForestsTrees\_BorealDeciduous"}\NormalTok{),}
 \AttributeTok{output\_dir   =} \StringTok{"./RasterGrids\_100m/2024/RAW/"}\NormalTok{,}
 \AttributeTok{n\_workers   =} \DecValTok{6}\NormalTok{,}
 \AttributeTok{radii     =} \FunctionTok{c}\NormalTok{(}\StringTok{"r3000"}\NormalTok{),}
 \AttributeTok{radius\_mode  =} \StringTok{"sparse"}\NormalTok{,}
 \AttributeTok{extract\_fun  =} \StringTok{"mean"}\NormalTok{,}
 \AttributeTok{fill\_missing  =} \ConstantTok{TRUE}\NormalTok{,}
 \AttributeTok{IDW\_weight   =} \DecValTok{2}\NormalTok{,}
 \AttributeTok{future\_max\_size =} \DecValTok{40} \SpecialCharTok{*} \DecValTok{1024}\SpecialCharTok{\^{}}\DecValTok{3}\NormalTok{)}


\CommentTok{\# ForestsTrees\_BorealDeciduous\_r3000.tif    egv\_386}
\NormalTok{slanis}\OtherTok{=}\FunctionTok{rast}\NormalTok{(}\StringTok{"./RasterGrids\_100m/2024/RAW/ForestsTrees\_BorealDeciduous\_r3000.tif"}\NormalTok{)}
\FunctionTok{names}\NormalTok{(slanis)}\OtherTok{=}\StringTok{"egv\_386"}
\NormalTok{slanis2}\OtherTok{=}\FunctionTok{project}\NormalTok{(slanis,template100)}
\FunctionTok{writeRaster}\NormalTok{(slanis2,}
      \StringTok{"./RasterGrids\_100m/2024/RAW/ForestsTrees\_BorealDeciduous\_r3000.tif"}\NormalTok{,}
      \AttributeTok{overwrite=}\ConstantTok{TRUE}\NormalTok{)}

\CommentTok{\# standardisation {-}{-}{-}{-}}
\ControlFlowTok{if}\NormalTok{(}\SpecialCharTok{!}\FunctionTok{require}\NormalTok{(terra)) \{}\FunctionTok{install.packages}\NormalTok{(}\StringTok{"terra"}\NormalTok{); }\FunctionTok{require}\NormalTok{(terra)\}}
\ControlFlowTok{if}\NormalTok{(}\SpecialCharTok{!}\FunctionTok{require}\NormalTok{(tidyverse)) \{}\FunctionTok{install.packages}\NormalTok{(}\StringTok{"tidyverse"}\NormalTok{); }\FunctionTok{require}\NormalTok{(tidyverse)\}}

\NormalTok{nosaukums}\OtherTok{=}\StringTok{"ForestsTrees\_BorealDeciduous\_r3000.tif"}
\NormalTok{ielasisanas\_cels}\OtherTok{=}\FunctionTok{paste0}\NormalTok{(}\StringTok{"./RasterGrids\_100m/2024/RAW/"}\NormalTok{,nosaukums)}
\NormalTok{saglabasanas\_cels}\OtherTok{=}\FunctionTok{paste0}\NormalTok{(}\StringTok{"./RasterGrids\_100m/2024/Scaled/"}\NormalTok{,nosaukums)}
\NormalTok{slanis}\OtherTok{=}\FunctionTok{rast}\NormalTok{(ielasisanas\_cels)}
\NormalTok{videjais}\OtherTok{=}\FunctionTok{global}\NormalTok{(slanis,}\AttributeTok{fun=}\StringTok{"mean"}\NormalTok{,}\AttributeTok{na.rm=}\ConstantTok{TRUE}\NormalTok{)}
\NormalTok{centrets}\OtherTok{=}\NormalTok{slanis}\SpecialCharTok{{-}}\NormalTok{videjais[,}\DecValTok{1}\NormalTok{]}
\NormalTok{standartnovirze}\OtherTok{=}\NormalTok{terra}\SpecialCharTok{::}\FunctionTok{global}\NormalTok{(centrets,}\AttributeTok{fun=}\StringTok{"rms"}\NormalTok{,}\AttributeTok{na.rm=}\ConstantTok{TRUE}\NormalTok{)}
\NormalTok{merogots}\OtherTok{=}\NormalTok{centrets}\SpecialCharTok{/}\NormalTok{standartnovirze[,}\DecValTok{1}\NormalTok{]}
\FunctionTok{writeRaster}\NormalTok{(merogots,}
      \AttributeTok{filename=}\NormalTok{saglabasanas\_cels,}
      \AttributeTok{overwrite=}\ConstantTok{TRUE}\NormalTok{)}
\end{Highlighting}
\end{Shaded}

\section{ForestsTrees\_BorealDeciduous\_r10000}\label{ch06.387}

\textbf{filename:} \texttt{ForestsTrees\_BorealDeciduous\_r10000.tif}

\textbf{layername:} \texttt{egv\_387}

\textbf{English name:} Fractional cover of Boreal Deciduous Forests within the 10 km
landscape

\textbf{Latvian name:} Šaurlapju mežu platības īpatsvars 10 km ainavā

\textbf{Procedure:} The cover fraction within a radius of 10000 m around the analysis grid cell
is calculated as the area-weighted sum of the \hyperref[ch06.383]{analysis cells} inside
the buffer, using the workflow \texttt{egvtools::radius\_function()}. During the calculation of the landscape
metric, inverse distance weighted (power = 2) gap filling on the output is
applied to ensure no missing values at the edges. Then the layer is
rewritten to set its name. Finally, the layer is standardised by
subtracting the arithmetic mean and dividing by the root mean squared error.

\begin{Shaded}
\begin{Highlighting}[]
\CommentTok{\# libs {-}{-}{-}{-}}
\ControlFlowTok{if}\NormalTok{(}\SpecialCharTok{!}\FunctionTok{require}\NormalTok{(terra)) \{}\FunctionTok{install.packages}\NormalTok{(}\StringTok{"terra"}\NormalTok{); }\FunctionTok{require}\NormalTok{(terra)\}}
\ControlFlowTok{if}\NormalTok{(}\SpecialCharTok{!}\FunctionTok{require}\NormalTok{(egvtools)) \{remotes}\SpecialCharTok{::}\FunctionTok{install\_github}\NormalTok{(}\StringTok{"aavotins/egvtools"}\NormalTok{); }\FunctionTok{require}\NormalTok{(egvtools)\}}


\CommentTok{\# Templates {-}{-}{-}{-}{-}}
\NormalTok{template100}\OtherTok{=}\FunctionTok{rast}\NormalTok{(}\StringTok{"./Templates/TemplateRasters/LV100m\_10km.tif"}\NormalTok{)}

\CommentTok{\# radii {-}{-}{-}{-}}
\FunctionTok{radius\_function}\NormalTok{(}
 \AttributeTok{kvadrati\_path =} \StringTok{"./Templates/TemplateGrids/tiles/"}\NormalTok{,}
 \AttributeTok{radii\_path   =} \StringTok{"./Templates/TemplateGridPoints/tiles/"}\NormalTok{,}
 \AttributeTok{tikls100\_path =} \StringTok{"./Templates/TemplateGrids/tikls100\_sauzeme.parquet"}\NormalTok{,}
 \AttributeTok{template\_path =} \StringTok{"./Templates/TemplateRasters/LV100m\_10km.tif"}\NormalTok{,}
 \AttributeTok{input\_layers  =} \FunctionTok{c}\NormalTok{(}\StringTok{"./RasterGrids\_100m/2024/RAW/ForestsTrees\_BorealDeciduous\_cell.tif"}\NormalTok{),}
 \AttributeTok{layer\_prefixes =} \FunctionTok{c}\NormalTok{(}\StringTok{"ForestsTrees\_BorealDeciduous"}\NormalTok{),}
 \AttributeTok{output\_dir   =} \StringTok{"./RasterGrids\_100m/2024/RAW/"}\NormalTok{,}
 \AttributeTok{n\_workers   =} \DecValTok{6}\NormalTok{,}
 \AttributeTok{radii     =} \FunctionTok{c}\NormalTok{(}\StringTok{"r10000"}\NormalTok{),}
 \AttributeTok{radius\_mode  =} \StringTok{"sparse"}\NormalTok{,}
 \AttributeTok{extract\_fun  =} \StringTok{"mean"}\NormalTok{,}
 \AttributeTok{fill\_missing  =} \ConstantTok{TRUE}\NormalTok{,}
 \AttributeTok{IDW\_weight   =} \DecValTok{2}\NormalTok{,}
 \AttributeTok{future\_max\_size =} \DecValTok{40} \SpecialCharTok{*} \DecValTok{1024}\SpecialCharTok{\^{}}\DecValTok{3}\NormalTok{)}


\CommentTok{\# ForestsTrees\_BorealDeciduous\_r10000.tif   egv\_387}
\NormalTok{slanis}\OtherTok{=}\FunctionTok{rast}\NormalTok{(}\StringTok{"./RasterGrids\_100m/2024/RAW/ForestsTrees\_BorealDeciduous\_r10000.tif"}\NormalTok{)}
\FunctionTok{names}\NormalTok{(slanis)}\OtherTok{=}\StringTok{"egv\_387"}
\NormalTok{slanis2}\OtherTok{=}\FunctionTok{project}\NormalTok{(slanis,template100)}
\FunctionTok{writeRaster}\NormalTok{(slanis2,}
      \StringTok{"./RasterGrids\_100m/2024/RAW/ForestsTrees\_BorealDeciduous\_r10000.tif"}\NormalTok{,}
      \AttributeTok{overwrite=}\ConstantTok{TRUE}\NormalTok{)}

\CommentTok{\# standardisation {-}{-}{-}{-}}
\ControlFlowTok{if}\NormalTok{(}\SpecialCharTok{!}\FunctionTok{require}\NormalTok{(terra)) \{}\FunctionTok{install.packages}\NormalTok{(}\StringTok{"terra"}\NormalTok{); }\FunctionTok{require}\NormalTok{(terra)\}}
\ControlFlowTok{if}\NormalTok{(}\SpecialCharTok{!}\FunctionTok{require}\NormalTok{(tidyverse)) \{}\FunctionTok{install.packages}\NormalTok{(}\StringTok{"tidyverse"}\NormalTok{); }\FunctionTok{require}\NormalTok{(tidyverse)\}}

\NormalTok{nosaukums}\OtherTok{=}\StringTok{"ForestsTrees\_BorealDeciduous\_r10000.tif"}
\NormalTok{ielasisanas\_cels}\OtherTok{=}\FunctionTok{paste0}\NormalTok{(}\StringTok{"./RasterGrids\_100m/2024/RAW/"}\NormalTok{,nosaukums)}
\NormalTok{saglabasanas\_cels}\OtherTok{=}\FunctionTok{paste0}\NormalTok{(}\StringTok{"./RasterGrids\_100m/2024/Scaled/"}\NormalTok{,nosaukums)}
\NormalTok{slanis}\OtherTok{=}\FunctionTok{rast}\NormalTok{(ielasisanas\_cels)}
\NormalTok{videjais}\OtherTok{=}\FunctionTok{global}\NormalTok{(slanis,}\AttributeTok{fun=}\StringTok{"mean"}\NormalTok{,}\AttributeTok{na.rm=}\ConstantTok{TRUE}\NormalTok{)}
\NormalTok{centrets}\OtherTok{=}\NormalTok{slanis}\SpecialCharTok{{-}}\NormalTok{videjais[,}\DecValTok{1}\NormalTok{]}
\NormalTok{standartnovirze}\OtherTok{=}\NormalTok{terra}\SpecialCharTok{::}\FunctionTok{global}\NormalTok{(centrets,}\AttributeTok{fun=}\StringTok{"rms"}\NormalTok{,}\AttributeTok{na.rm=}\ConstantTok{TRUE}\NormalTok{)}
\NormalTok{merogots}\OtherTok{=}\NormalTok{centrets}\SpecialCharTok{/}\NormalTok{standartnovirze[,}\DecValTok{1}\NormalTok{]}
\FunctionTok{writeRaster}\NormalTok{(merogots,}
      \AttributeTok{filename=}\NormalTok{saglabasanas\_cels,}
      \AttributeTok{overwrite=}\ConstantTok{TRUE}\NormalTok{)}
\end{Highlighting}
\end{Shaded}

\section{ForestsTrees\_Coniferous\_cell}\label{ch06.388}

\textbf{filename:} \texttt{ForestsTrees\_Coniferous\_cell.tif}

\textbf{layername:} \texttt{egv\_388}

\textbf{English name:} Fractional cover of Coniferous Forests within the analysis
cell (1 ha)

\textbf{Latvian name:} Skujkoku mežu platības īpatsvars analīzes šūnā (1 ha)

\textbf{Procedure:} Most EGVs describing forests are spatially restricted to areas outside
of clearcuts and dead stands. This mask is created using a combination of
the \hyperref[Ch04.01]{State Forest Service's
State Forest Registry} land category 12 and 14, and \hyperref[Ch04.09]{The
Global Forest Watch} pixels classified as lost tree canopy cover since
2020 (raster layer matching input, presence = 1, absence = 0).

To prepare this EGV, stands from the \hyperref[Ch04.01]{State Forest Service's State Forest
Registry} are classified into (in order):

\begin{itemize}
\item
  coniferous (see \hyperref[Ch01]{Terminology and acronyms} for species codes) if
  timber volume of those species exceeded 75\%;
\item
  Boreal deciduous if timber volume of those species exceeded 75\%;
\item
  temperate deciduous if timber volume of those species exceeded 50\%;
\item
  mixed otherwise;
\end{itemize}

then coniferous stands are selected and geometries are
rasterised (presence = 1, NA otherwise). Rasterisation is
performed using the workflow \texttt{egvtools::polygon2input()}, restricting to pixels outside clearcut
mask and covering background with value 0. The resulting layer
is then aggregated to EGV resolution using the workflow \texttt{egvtools::input2egv()}, which
calculates the arithmetic mean to determine the cover fraction. During
aggregation, inverse distance weighted (power = 2) gap filling on the output is
applied to ensure no missing values at the edges. Finally, the layer is
standardised by subtracting the arithmetic mean and dividing by the root mean squared
error.

\begin{Shaded}
\begin{Highlighting}[]
\CommentTok{\# libs {-}{-}{-}{-}}
\ControlFlowTok{if}\NormalTok{(}\SpecialCharTok{!}\FunctionTok{require}\NormalTok{(egvtools)) \{remotes}\SpecialCharTok{::}\FunctionTok{install\_github}\NormalTok{(}\StringTok{"aavotins/egvtools"}\NormalTok{); }\FunctionTok{require}\NormalTok{(egvtools)\}}
\ControlFlowTok{if}\NormalTok{(}\SpecialCharTok{!}\FunctionTok{require}\NormalTok{(terra)) \{}\FunctionTok{install.packages}\NormalTok{(}\StringTok{"terra"}\NormalTok{); }\FunctionTok{require}\NormalTok{(terra)\}}
\ControlFlowTok{if}\NormalTok{(}\SpecialCharTok{!}\FunctionTok{require}\NormalTok{(sf)) \{}\FunctionTok{install.packages}\NormalTok{(}\StringTok{"sf"}\NormalTok{); }\FunctionTok{require}\NormalTok{(sf)\}}
\ControlFlowTok{if}\NormalTok{(}\SpecialCharTok{!}\FunctionTok{require}\NormalTok{(tidyverse)) \{}\FunctionTok{install.packages}\NormalTok{(}\StringTok{"tidyverse"}\NormalTok{); }\FunctionTok{require}\NormalTok{(tidyverse)\}}
\ControlFlowTok{if}\NormalTok{(}\SpecialCharTok{!}\FunctionTok{require}\NormalTok{(sfarrow)) \{}\FunctionTok{install.packages}\NormalTok{(}\StringTok{"sfarrow"}\NormalTok{); }\FunctionTok{require}\NormalTok{(sfarrow)\}}
\ControlFlowTok{if}\NormalTok{(}\SpecialCharTok{!}\FunctionTok{require}\NormalTok{(readxl)) \{}\FunctionTok{install.packages}\NormalTok{(}\StringTok{"readxl"}\NormalTok{); }\FunctionTok{require}\NormalTok{(readxl)\}}
\ControlFlowTok{if}\NormalTok{(}\SpecialCharTok{!}\FunctionTok{require}\NormalTok{(raster)) \{}\FunctionTok{install.packages}\NormalTok{(}\StringTok{"raster"}\NormalTok{); }\FunctionTok{require}\NormalTok{(raster)\}}
\ControlFlowTok{if}\NormalTok{(}\SpecialCharTok{!}\FunctionTok{require}\NormalTok{(fasterize)) \{}\FunctionTok{install.packages}\NormalTok{(}\StringTok{"fasterize"}\NormalTok{); }\FunctionTok{require}\NormalTok{(fasterize)\}}

\CommentTok{\# templates {-}{-}{-}{-}}
\NormalTok{template100}\OtherTok{=}\FunctionTok{rast}\NormalTok{(}\StringTok{"./Templates/TemplateRasters/LV100m\_10km.tif"}\NormalTok{)}
\NormalTok{template10}\OtherTok{=}\FunctionTok{rast}\NormalTok{(}\StringTok{"./Templates/TemplateRasters/LV10m\_10km.tif"}\NormalTok{)}
\NormalTok{rastrs10}\OtherTok{=}\FunctionTok{raster}\NormalTok{(template10)}

\NormalTok{nulls10}\OtherTok{=}\FunctionTok{rast}\NormalTok{(}\StringTok{"./Templates/TemplateRasters/nulls\_LV10m\_10km.tif"}\NormalTok{)}
\NormalTok{nulls100}\OtherTok{=}\FunctionTok{rast}\NormalTok{(}\StringTok{"./Templates/TemplateRasters/nulls\_LV100m\_10km.tif"}\NormalTok{)}


\CommentTok{\# simple landscape {-}{-}{-}{-}}
\NormalTok{simple\_landscape}\OtherTok{=}\FunctionTok{rast}\NormalTok{(}\StringTok{"RasterGrids\_10m/2024/Ainava\_vienk\_mask.tif"}\NormalTok{)}

\CommentTok{\# mvr {-}{-}{-}{-}}
\NormalTok{mvr}\OtherTok{=}\FunctionTok{st\_read\_parquet}\NormalTok{(}\StringTok{"./Geodata/2024/MVR/nogabali\_2024janv.parquet"}\NormalTok{)}
\NormalTok{mvr}\SpecialCharTok{$}\NormalTok{yes}\OtherTok{=}\DecValTok{1}

\CommentTok{\# clear cut mask {-}{-}{-}{-}}
\NormalTok{izcirtumi}\OtherTok{=}\NormalTok{mvr }\SpecialCharTok{\%\textgreater{}\%} 
 \FunctionTok{filter}\NormalTok{(zkat }\SpecialCharTok{\%in\%} \FunctionTok{c}\NormalTok{(}\StringTok{"12"}\NormalTok{,}\StringTok{"14"}\NormalTok{)) }\SpecialCharTok{\%\textgreater{}\%} 
\NormalTok{ dplyr}\SpecialCharTok{::}\FunctionTok{select}\NormalTok{(yes)}
\NormalTok{r\_izcirtumi\_mvr}\OtherTok{=}\FunctionTok{fasterize}\NormalTok{(izcirtumi,rastrs10,}\AttributeTok{field=}\StringTok{"yes"}\NormalTok{)}
\NormalTok{t\_izcirtumi\_mvr}\OtherTok{=}\FunctionTok{rast}\NormalTok{(r\_izcirtumi\_mvr)}
\FunctionTok{plot}\NormalTok{(t\_izcirtumi\_mvr)}

\NormalTok{tcl}\OtherTok{=}\FunctionTok{rast}\NormalTok{(}\StringTok{"./Geodata/2024/Trees/GFW/TreeCoverLoss\_v1\_12.tif"}\NormalTok{)}
\NormalTok{tcl2}\OtherTok{=}\FunctionTok{ifel}\NormalTok{(tcl}\SpecialCharTok{\textless{}}\DecValTok{20}\NormalTok{,}\DecValTok{0}\NormalTok{,}\DecValTok{1}\NormalTok{)}
\NormalTok{tclX}\OtherTok{=}\FunctionTok{cover}\NormalTok{(tcl2,nulls10)}
\FunctionTok{plot}\NormalTok{(tclX)}

\NormalTok{clearcut\_mask}\OtherTok{=}\FunctionTok{cover}\NormalTok{(t\_izcirtumi\_mvr,tclX,}
          \AttributeTok{filename=}\StringTok{"./RasterGrids\_10m/2024/Mask\_clearcuts.tif"}\NormalTok{,}
          \AttributeTok{overwrite=}\ConstantTok{TRUE}\NormalTok{)}
\FunctionTok{plot}\NormalTok{(clearcut\_mask)}

\FunctionTok{rm}\NormalTok{(izcirtumi)}
\FunctionTok{rm}\NormalTok{(r\_izcirtumi\_mvr)}
\FunctionTok{rm}\NormalTok{(t\_izcirtumi\_mvr)}
\FunctionTok{rm}\NormalTok{(tcl)}
\FunctionTok{rm}\NormalTok{(tcl2)}
\FunctionTok{rm}\NormalTok{(tclX)}

\CommentTok{\# ForestsTrees\_Coniferous\_cell.tif  egv\_388 {-}{-}{-}{-}}
\NormalTok{skujkoki}\OtherTok{=}\FunctionTok{c}\NormalTok{(}\StringTok{"1"}\NormalTok{,}\StringTok{"3"}\NormalTok{,}\StringTok{"13"}\NormalTok{,}\StringTok{"14"}\NormalTok{,}\StringTok{"15"}\NormalTok{,}\StringTok{"22"}\NormalTok{,}\StringTok{"23"}\NormalTok{,}\StringTok{"28"}\NormalTok{) }\CommentTok{\# 8}
\NormalTok{saurlapji}\OtherTok{=}\FunctionTok{c}\NormalTok{(}\StringTok{"4"}\NormalTok{,}\StringTok{"6"}\NormalTok{,}\StringTok{"8"}\NormalTok{,}\StringTok{"9"}\NormalTok{,}\StringTok{"19"}\NormalTok{,}\StringTok{"20"}\NormalTok{,}\StringTok{"21"}\NormalTok{,}\StringTok{"32"}\NormalTok{,}\StringTok{"35"}\NormalTok{,}\StringTok{"68"}\NormalTok{) }\CommentTok{\# 10}
\NormalTok{platlapji}\OtherTok{=}\FunctionTok{c}\NormalTok{(}\StringTok{"10"}\NormalTok{,}\StringTok{"11"}\NormalTok{,}\StringTok{"12"}\NormalTok{,}\StringTok{"16"}\NormalTok{,}\StringTok{"17"}\NormalTok{,}\StringTok{"18"}\NormalTok{,}\StringTok{"24"}\NormalTok{,}\StringTok{"25"}\NormalTok{,}\StringTok{"26"}\NormalTok{,}\StringTok{"27"}\NormalTok{,}\StringTok{"28"}\NormalTok{,}\StringTok{"29"}\NormalTok{,}\StringTok{"50"}\NormalTok{,}
      \StringTok{"61"}\NormalTok{,}\StringTok{"62"}\NormalTok{,}\StringTok{"63"}\NormalTok{,}\StringTok{"64"}\NormalTok{,}\StringTok{"65"}\NormalTok{,}\StringTok{"66"}\NormalTok{,}\StringTok{"67"}\NormalTok{,}\StringTok{"69"}\NormalTok{) }\CommentTok{\# 21}
\NormalTok{mvr}\OtherTok{=}\NormalTok{mvr }\SpecialCharTok{\%\textgreater{}\%} 
 \FunctionTok{mutate}\NormalTok{(}\AttributeTok{kraja\_skujkoku=}\FunctionTok{ifelse}\NormalTok{(s10 }\SpecialCharTok{\%in\%}\NormalTok{ skujkoki,v10,}\DecValTok{0}\NormalTok{)}\SpecialCharTok{+}
      \FunctionTok{ifelse}\NormalTok{(s11 }\SpecialCharTok{\%in\%}\NormalTok{ skujkoki,v11,}\DecValTok{0}\NormalTok{)}\SpecialCharTok{+}\FunctionTok{ifelse}\NormalTok{(s12 }\SpecialCharTok{\%in\%}\NormalTok{ skujkoki,v12,}\DecValTok{0}\NormalTok{)}\SpecialCharTok{+}
      \FunctionTok{ifelse}\NormalTok{(s13 }\SpecialCharTok{\%in\%}\NormalTok{ skujkoki,v13,}\DecValTok{0}\NormalTok{)}\SpecialCharTok{+}\FunctionTok{ifelse}\NormalTok{(s14 }\SpecialCharTok{\%in\%}\NormalTok{ skujkoki,v14,}\DecValTok{0}\NormalTok{),}
     \AttributeTok{kraja\_saurlapju=}\FunctionTok{ifelse}\NormalTok{(s10 }\SpecialCharTok{\%in\%}\NormalTok{ saurlapji,v10,}\DecValTok{0}\NormalTok{)}\SpecialCharTok{+}
      \FunctionTok{ifelse}\NormalTok{(s11 }\SpecialCharTok{\%in\%}\NormalTok{ saurlapji,v11,}\DecValTok{0}\NormalTok{)}\SpecialCharTok{+}\FunctionTok{ifelse}\NormalTok{(s12 }\SpecialCharTok{\%in\%}\NormalTok{ saurlapji,v12,}\DecValTok{0}\NormalTok{)}\SpecialCharTok{+}
      \FunctionTok{ifelse}\NormalTok{(s13 }\SpecialCharTok{\%in\%}\NormalTok{ saurlapji,v13,}\DecValTok{0}\NormalTok{)}\SpecialCharTok{+}\FunctionTok{ifelse}\NormalTok{(s14 }\SpecialCharTok{\%in\%}\NormalTok{ saurlapji,v14,}\DecValTok{0}\NormalTok{),}
     \AttributeTok{kraja\_platlapju=}\FunctionTok{ifelse}\NormalTok{(s10 }\SpecialCharTok{\%in\%}\NormalTok{ platlapji,v10,}\DecValTok{0}\NormalTok{)}\SpecialCharTok{+}
      \FunctionTok{ifelse}\NormalTok{(s11 }\SpecialCharTok{\%in\%}\NormalTok{ platlapji,v11,}\DecValTok{0}\NormalTok{)}\SpecialCharTok{+}\FunctionTok{ifelse}\NormalTok{(s12 }\SpecialCharTok{\%in\%}\NormalTok{ platlapji,v12,}\DecValTok{0}\NormalTok{)}\SpecialCharTok{+}
      \FunctionTok{ifelse}\NormalTok{(s13 }\SpecialCharTok{\%in\%}\NormalTok{ platlapji,v13,}\DecValTok{0}\NormalTok{)}\SpecialCharTok{+}\FunctionTok{ifelse}\NormalTok{(s14 }\SpecialCharTok{\%in\%}\NormalTok{ platlapji,v14,}\DecValTok{0}\NormalTok{)) }\SpecialCharTok{\%\textgreater{}\%} 
 \FunctionTok{mutate}\NormalTok{(}\AttributeTok{kopeja\_kraja=}\NormalTok{kraja\_skujkoku}\SpecialCharTok{+}\NormalTok{kraja\_platlapju}\SpecialCharTok{+}\NormalTok{kraja\_saurlapju) }\SpecialCharTok{\%\textgreater{}\%} 
 \FunctionTok{mutate}\NormalTok{(}\AttributeTok{tips=}\FunctionTok{ifelse}\NormalTok{(kraja\_skujkoku}\SpecialCharTok{/}\NormalTok{kopeja\_kraja}\SpecialCharTok{\textgreater{}=}\FloatTok{0.75}\NormalTok{,}\StringTok{"skujkoku"}\NormalTok{,}
           \FunctionTok{ifelse}\NormalTok{(kraja\_saurlapju}\SpecialCharTok{/}\NormalTok{kopeja\_kraja}\SpecialCharTok{\textgreater{}=}\FloatTok{0.75}\NormalTok{,}\StringTok{"saurlapju"}\NormalTok{,}
              \FunctionTok{ifelse}\NormalTok{(kraja\_platlapju}\SpecialCharTok{/}\NormalTok{kopeja\_kraja}\SpecialCharTok{\textgreater{}}\FloatTok{0.5}\NormalTok{,}\StringTok{"platlapju"}\NormalTok{,}
                  \StringTok{"jauktu koku"}\NormalTok{))))}
\NormalTok{nogabali}\OtherTok{=}\NormalTok{mvr }\SpecialCharTok{\%\textgreater{}\%} 
 \FunctionTok{filter}\NormalTok{(zkat}\SpecialCharTok{==}\StringTok{"10"}\SpecialCharTok{\&}\NormalTok{tips}\SpecialCharTok{==}\StringTok{"skujkoku"}\NormalTok{)}

\NormalTok{p2i\_rez}\OtherTok{=}\NormalTok{egvtools}\SpecialCharTok{::}\FunctionTok{polygon2input}\NormalTok{(}\AttributeTok{vector\_data =}\NormalTok{ nogabali,}
                \AttributeTok{template\_path =} \StringTok{"./Templates/TemplateRasters/LV10m\_10km.tif"}\NormalTok{,}
                \AttributeTok{out\_path =} \StringTok{"./RasterGrids\_10m/2024/"}\NormalTok{,}
                \AttributeTok{file\_name =} \StringTok{"ForestsTrees\_Coniferous\_input.tif"}\NormalTok{,}
                \AttributeTok{value\_field =} \StringTok{"yes"}\NormalTok{,}
                \AttributeTok{restrict\_to =}\NormalTok{ clearcut\_mask,}
                \AttributeTok{restrict\_values =} \DecValTok{0}\NormalTok{,}
                \AttributeTok{prepare=}\ConstantTok{FALSE}\NormalTok{,}
                \AttributeTok{background\_raster =} \StringTok{"./Templates/TemplateRasters/nulls\_LV10m\_10km.tif"}\NormalTok{,}
                \AttributeTok{plot\_result =} \ConstantTok{TRUE}\NormalTok{)}
\NormalTok{p2i\_rez}
\NormalTok{i2e\_rez}\OtherTok{=}\NormalTok{egvtools}\SpecialCharTok{::}\FunctionTok{input2egv}\NormalTok{(}\AttributeTok{input=}\FunctionTok{paste0}\NormalTok{(}\StringTok{"./RasterGrids\_10m/2024/"}\NormalTok{,}
                     \StringTok{"ForestsTrees\_Coniferous\_input.tif"}\NormalTok{),}
              \AttributeTok{egv\_template=} \StringTok{"./Templates/TemplateRasters/LV100m\_10km.tif"}\NormalTok{,}
              \AttributeTok{summary\_function =} \StringTok{"average"}\NormalTok{,}
              \AttributeTok{missing\_job =} \StringTok{"FillOutput"}\NormalTok{,}
              \AttributeTok{outlocation =} \StringTok{"./RasterGrids\_100m/2024/RAW/"}\NormalTok{,}
              \AttributeTok{outfilename =} \StringTok{"ForestsTrees\_Coniferous\_cell.tif"}\NormalTok{,}
              \AttributeTok{layername =} \StringTok{"egv\_388"}\NormalTok{,}
              \AttributeTok{idw\_weight =} \DecValTok{2}\NormalTok{,}
              \AttributeTok{plot\_gaps =} \ConstantTok{FALSE}\NormalTok{,}\AttributeTok{plot\_final =} \ConstantTok{TRUE}\NormalTok{)}
\NormalTok{i2e\_rez}
\FunctionTok{rm}\NormalTok{(nogabali)}
\FunctionTok{rm}\NormalTok{(p2i\_rez)}
\FunctionTok{rm}\NormalTok{(i2e\_rez)}
\FunctionTok{unlink}\NormalTok{(}\StringTok{"./RasterGrids\_10m/2024/ForestsTrees\_Coniferous\_input.tif"}\NormalTok{)}

\CommentTok{\# standardisation {-}{-}{-}{-}}
\ControlFlowTok{if}\NormalTok{(}\SpecialCharTok{!}\FunctionTok{require}\NormalTok{(terra)) \{}\FunctionTok{install.packages}\NormalTok{(}\StringTok{"terra"}\NormalTok{); }\FunctionTok{require}\NormalTok{(terra)\}}
\ControlFlowTok{if}\NormalTok{(}\SpecialCharTok{!}\FunctionTok{require}\NormalTok{(tidyverse)) \{}\FunctionTok{install.packages}\NormalTok{(}\StringTok{"tidyverse"}\NormalTok{); }\FunctionTok{require}\NormalTok{(tidyverse)\}}

\NormalTok{nosaukums}\OtherTok{=}\StringTok{"ForestsTrees\_Coniferous\_cell.tif"}
\NormalTok{ielasisanas\_cels}\OtherTok{=}\FunctionTok{paste0}\NormalTok{(}\StringTok{"./RasterGrids\_100m/2024/RAW/"}\NormalTok{,nosaukums)}
\NormalTok{saglabasanas\_cels}\OtherTok{=}\FunctionTok{paste0}\NormalTok{(}\StringTok{"./RasterGrids\_100m/2024/Scaled/"}\NormalTok{,nosaukums)}
\NormalTok{slanis}\OtherTok{=}\FunctionTok{rast}\NormalTok{(ielasisanas\_cels)}
\NormalTok{videjais}\OtherTok{=}\FunctionTok{global}\NormalTok{(slanis,}\AttributeTok{fun=}\StringTok{"mean"}\NormalTok{,}\AttributeTok{na.rm=}\ConstantTok{TRUE}\NormalTok{)}
\NormalTok{centrets}\OtherTok{=}\NormalTok{slanis}\SpecialCharTok{{-}}\NormalTok{videjais[,}\DecValTok{1}\NormalTok{]}
\NormalTok{standartnovirze}\OtherTok{=}\NormalTok{terra}\SpecialCharTok{::}\FunctionTok{global}\NormalTok{(centrets,}\AttributeTok{fun=}\StringTok{"rms"}\NormalTok{,}\AttributeTok{na.rm=}\ConstantTok{TRUE}\NormalTok{)}
\NormalTok{merogots}\OtherTok{=}\NormalTok{centrets}\SpecialCharTok{/}\NormalTok{standartnovirze[,}\DecValTok{1}\NormalTok{]}
\FunctionTok{writeRaster}\NormalTok{(merogots,}
      \AttributeTok{filename=}\NormalTok{saglabasanas\_cels,}
      \AttributeTok{overwrite=}\ConstantTok{TRUE}\NormalTok{)}
\end{Highlighting}
\end{Shaded}

\section{ForestsTrees\_Coniferous\_r500}\label{ch06.389}

\textbf{filename:} \texttt{ForestsTrees\_Coniferous\_r500.tif}

\textbf{layername:} \texttt{egv\_389}

\textbf{English name:} Fractional cover of Coniferous Forests within the 0.5 km
landscape

\textbf{Latvian name:} Skujkoku mežu platības īpatsvars 0,5 km ainavā

\textbf{Procedure:} The cover fraction within a radius of 500 m around the analysis grid cell is
calculated as the area-weighted sum of the \hyperref[ch06.388]{analysis cells} inside the
buffer, using the workflow \texttt{egvtools::radius\_function()}. During the calculation of the landscape metric,
inverse distance weighted (power = 2) gap filling on the output is applied
to ensure no missing values at the edges. Then the layer is rewritten to set
its name. Finally, the layer is standardised by subtracting the arithmetic
mean and dividing by the root mean squared error.

\begin{Shaded}
\begin{Highlighting}[]
\CommentTok{\# libs {-}{-}{-}{-}}
\ControlFlowTok{if}\NormalTok{(}\SpecialCharTok{!}\FunctionTok{require}\NormalTok{(terra)) \{}\FunctionTok{install.packages}\NormalTok{(}\StringTok{"terra"}\NormalTok{); }\FunctionTok{require}\NormalTok{(terra)\}}
\ControlFlowTok{if}\NormalTok{(}\SpecialCharTok{!}\FunctionTok{require}\NormalTok{(egvtools)) \{remotes}\SpecialCharTok{::}\FunctionTok{install\_github}\NormalTok{(}\StringTok{"aavotins/egvtools"}\NormalTok{); }\FunctionTok{require}\NormalTok{(egvtools)\}}


\CommentTok{\# Templates {-}{-}{-}{-}{-}}
\NormalTok{template100}\OtherTok{=}\FunctionTok{rast}\NormalTok{(}\StringTok{"./Templates/TemplateRasters/LV100m\_10km.tif"}\NormalTok{)}

\CommentTok{\# radii {-}{-}{-}{-}}
\FunctionTok{radius\_function}\NormalTok{(}
 \AttributeTok{kvadrati\_path =} \StringTok{"./Templates/TemplateGrids/tiles/"}\NormalTok{,}
 \AttributeTok{radii\_path   =} \StringTok{"./Templates/TemplateGridPoints/tiles/"}\NormalTok{,}
 \AttributeTok{tikls100\_path =} \StringTok{"./Templates/TemplateGrids/tikls100\_sauzeme.parquet"}\NormalTok{,}
 \AttributeTok{template\_path =} \StringTok{"./Templates/TemplateRasters/LV100m\_10km.tif"}\NormalTok{,}
 \AttributeTok{input\_layers  =} \FunctionTok{c}\NormalTok{(}\StringTok{"./RasterGrids\_100m/2024/RAW/ForestsTrees\_Coniferous\_cell.tif"}\NormalTok{),}
 \AttributeTok{layer\_prefixes =} \FunctionTok{c}\NormalTok{(}\StringTok{"ForestsTrees\_Coniferous"}\NormalTok{),}
 \AttributeTok{output\_dir   =} \StringTok{"./RasterGrids\_100m/2024/RAW/"}\NormalTok{,}
 \AttributeTok{n\_workers   =} \DecValTok{6}\NormalTok{,}
 \AttributeTok{radii     =} \FunctionTok{c}\NormalTok{(}\StringTok{"r500"}\NormalTok{),}
 \AttributeTok{radius\_mode  =} \StringTok{"sparse"}\NormalTok{,}
 \AttributeTok{extract\_fun  =} \StringTok{"mean"}\NormalTok{,}
 \AttributeTok{fill\_missing  =} \ConstantTok{TRUE}\NormalTok{,}
 \AttributeTok{IDW\_weight   =} \DecValTok{2}\NormalTok{,}
 \AttributeTok{future\_max\_size =} \DecValTok{40} \SpecialCharTok{*} \DecValTok{1024}\SpecialCharTok{\^{}}\DecValTok{3}\NormalTok{)}


\CommentTok{\# ForestsTrees\_Coniferous\_r500.tif  egv\_389}
\NormalTok{slanis}\OtherTok{=}\FunctionTok{rast}\NormalTok{(}\StringTok{"./RasterGrids\_100m/2024/RAW/ForestsTrees\_Coniferous\_r500.tif"}\NormalTok{)}
\FunctionTok{names}\NormalTok{(slanis)}\OtherTok{=}\StringTok{"egv\_389"}
\NormalTok{slanis2}\OtherTok{=}\FunctionTok{project}\NormalTok{(slanis,template100)}
\FunctionTok{writeRaster}\NormalTok{(slanis2,}
      \StringTok{"./RasterGrids\_100m/2024/RAW/ForestsTrees\_Coniferous\_r500.tif"}\NormalTok{,}
      \AttributeTok{overwrite=}\ConstantTok{TRUE}\NormalTok{)}

\CommentTok{\# standardisation {-}{-}{-}{-}}
\ControlFlowTok{if}\NormalTok{(}\SpecialCharTok{!}\FunctionTok{require}\NormalTok{(terra)) \{}\FunctionTok{install.packages}\NormalTok{(}\StringTok{"terra"}\NormalTok{); }\FunctionTok{require}\NormalTok{(terra)\}}
\ControlFlowTok{if}\NormalTok{(}\SpecialCharTok{!}\FunctionTok{require}\NormalTok{(tidyverse)) \{}\FunctionTok{install.packages}\NormalTok{(}\StringTok{"tidyverse"}\NormalTok{); }\FunctionTok{require}\NormalTok{(tidyverse)\}}

\NormalTok{nosaukums}\OtherTok{=}\StringTok{"ForestsTrees\_Coniferous\_r500.tif"}
\NormalTok{ielasisanas\_cels}\OtherTok{=}\FunctionTok{paste0}\NormalTok{(}\StringTok{"./RasterGrids\_100m/2024/RAW/"}\NormalTok{,nosaukums)}
\NormalTok{saglabasanas\_cels}\OtherTok{=}\FunctionTok{paste0}\NormalTok{(}\StringTok{"./RasterGrids\_100m/2024/Scaled/"}\NormalTok{,nosaukums)}
\NormalTok{slanis}\OtherTok{=}\FunctionTok{rast}\NormalTok{(ielasisanas\_cels)}
\NormalTok{videjais}\OtherTok{=}\FunctionTok{global}\NormalTok{(slanis,}\AttributeTok{fun=}\StringTok{"mean"}\NormalTok{,}\AttributeTok{na.rm=}\ConstantTok{TRUE}\NormalTok{)}
\NormalTok{centrets}\OtherTok{=}\NormalTok{slanis}\SpecialCharTok{{-}}\NormalTok{videjais[,}\DecValTok{1}\NormalTok{]}
\NormalTok{standartnovirze}\OtherTok{=}\NormalTok{terra}\SpecialCharTok{::}\FunctionTok{global}\NormalTok{(centrets,}\AttributeTok{fun=}\StringTok{"rms"}\NormalTok{,}\AttributeTok{na.rm=}\ConstantTok{TRUE}\NormalTok{)}
\NormalTok{merogots}\OtherTok{=}\NormalTok{centrets}\SpecialCharTok{/}\NormalTok{standartnovirze[,}\DecValTok{1}\NormalTok{]}
\FunctionTok{writeRaster}\NormalTok{(merogots,}
      \AttributeTok{filename=}\NormalTok{saglabasanas\_cels,}
      \AttributeTok{overwrite=}\ConstantTok{TRUE}\NormalTok{)}
\end{Highlighting}
\end{Shaded}

\section{ForestsTrees\_Coniferous\_r1250}\label{ch06.390}

\textbf{filename:} \texttt{ForestsTrees\_Coniferous\_r1250.tif}

\textbf{layername:} \texttt{egv\_390}

\textbf{English name:} Fractional cover of Coniferous Forests within the 1.25 km
landscape

\textbf{Latvian name:} Skujkoku mežu platības īpatsvars 1,25 km ainavā

\textbf{Procedure:} The cover fraction within a radius of 1250 m around the analysis grid cell
is calculated as the area-weighted sum of the \hyperref[ch06.388]{analysis cells} inside
the buffer, using the workflow \texttt{egvtools::radius\_function()}. During the calculation of the landscape
metric, inverse distance weighted (power = 2) gap filling on the output is
applied to ensure no missing values at the edges. Then the layer is
rewritten to set its name. Finally, the layer is standardised by
subtracting the arithmetic mean and dividing by the root mean squared error.

\begin{Shaded}
\begin{Highlighting}[]
\CommentTok{\# libs {-}{-}{-}{-}}
\ControlFlowTok{if}\NormalTok{(}\SpecialCharTok{!}\FunctionTok{require}\NormalTok{(terra)) \{}\FunctionTok{install.packages}\NormalTok{(}\StringTok{"terra"}\NormalTok{); }\FunctionTok{require}\NormalTok{(terra)\}}
\ControlFlowTok{if}\NormalTok{(}\SpecialCharTok{!}\FunctionTok{require}\NormalTok{(egvtools)) \{remotes}\SpecialCharTok{::}\FunctionTok{install\_github}\NormalTok{(}\StringTok{"aavotins/egvtools"}\NormalTok{); }\FunctionTok{require}\NormalTok{(egvtools)\}}


\CommentTok{\# Templates {-}{-}{-}{-}{-}}
\NormalTok{template100}\OtherTok{=}\FunctionTok{rast}\NormalTok{(}\StringTok{"./Templates/TemplateRasters/LV100m\_10km.tif"}\NormalTok{)}

\CommentTok{\# radii {-}{-}{-}{-}}
\FunctionTok{radius\_function}\NormalTok{(}
 \AttributeTok{kvadrati\_path =} \StringTok{"./Templates/TemplateGrids/tiles/"}\NormalTok{,}
 \AttributeTok{radii\_path   =} \StringTok{"./Templates/TemplateGridPoints/tiles/"}\NormalTok{,}
 \AttributeTok{tikls100\_path =} \StringTok{"./Templates/TemplateGrids/tikls100\_sauzeme.parquet"}\NormalTok{,}
 \AttributeTok{template\_path =} \StringTok{"./Templates/TemplateRasters/LV100m\_10km.tif"}\NormalTok{,}
 \AttributeTok{input\_layers  =} \FunctionTok{c}\NormalTok{(}\StringTok{"./RasterGrids\_100m/2024/RAW/ForestsTrees\_Coniferous\_cell.tif"}\NormalTok{),}
 \AttributeTok{layer\_prefixes =} \FunctionTok{c}\NormalTok{(}\StringTok{"ForestsTrees\_Coniferous"}\NormalTok{),}
 \AttributeTok{output\_dir   =} \StringTok{"./RasterGrids\_100m/2024/RAW/"}\NormalTok{,}
 \AttributeTok{n\_workers   =} \DecValTok{6}\NormalTok{,}
 \AttributeTok{radii     =} \FunctionTok{c}\NormalTok{(}\StringTok{"r1250"}\NormalTok{),}
 \AttributeTok{radius\_mode  =} \StringTok{"sparse"}\NormalTok{,}
 \AttributeTok{extract\_fun  =} \StringTok{"mean"}\NormalTok{,}
 \AttributeTok{fill\_missing  =} \ConstantTok{TRUE}\NormalTok{,}
 \AttributeTok{IDW\_weight   =} \DecValTok{2}\NormalTok{,}
 \AttributeTok{future\_max\_size =} \DecValTok{40} \SpecialCharTok{*} \DecValTok{1024}\SpecialCharTok{\^{}}\DecValTok{3}\NormalTok{)}


\CommentTok{\# ForestsTrees\_Coniferous\_r1250.tif egv\_390}
\NormalTok{slanis}\OtherTok{=}\FunctionTok{rast}\NormalTok{(}\StringTok{"./RasterGrids\_100m/2024/RAW/ForestsTrees\_Coniferous\_r1250.tif"}\NormalTok{)}
\FunctionTok{names}\NormalTok{(slanis)}\OtherTok{=}\StringTok{"egv\_390"}
\NormalTok{slanis2}\OtherTok{=}\FunctionTok{project}\NormalTok{(slanis,template100)}
\FunctionTok{writeRaster}\NormalTok{(slanis2,}
      \StringTok{"./RasterGrids\_100m/2024/RAW/ForestsTrees\_Coniferous\_r1250.tif"}\NormalTok{,}
      \AttributeTok{overwrite=}\ConstantTok{TRUE}\NormalTok{)}

\CommentTok{\# standardisation {-}{-}{-}{-}}
\ControlFlowTok{if}\NormalTok{(}\SpecialCharTok{!}\FunctionTok{require}\NormalTok{(terra)) \{}\FunctionTok{install.packages}\NormalTok{(}\StringTok{"terra"}\NormalTok{); }\FunctionTok{require}\NormalTok{(terra)\}}
\ControlFlowTok{if}\NormalTok{(}\SpecialCharTok{!}\FunctionTok{require}\NormalTok{(tidyverse)) \{}\FunctionTok{install.packages}\NormalTok{(}\StringTok{"tidyverse"}\NormalTok{); }\FunctionTok{require}\NormalTok{(tidyverse)\}}

\NormalTok{nosaukums}\OtherTok{=}\StringTok{"ForestsTrees\_Coniferous\_r1250.tif"}
\NormalTok{ielasisanas\_cels}\OtherTok{=}\FunctionTok{paste0}\NormalTok{(}\StringTok{"./RasterGrids\_100m/2024/RAW/"}\NormalTok{,nosaukums)}
\NormalTok{saglabasanas\_cels}\OtherTok{=}\FunctionTok{paste0}\NormalTok{(}\StringTok{"./RasterGrids\_100m/2024/Scaled/"}\NormalTok{,nosaukums)}
\NormalTok{slanis}\OtherTok{=}\FunctionTok{rast}\NormalTok{(ielasisanas\_cels)}
\NormalTok{videjais}\OtherTok{=}\FunctionTok{global}\NormalTok{(slanis,}\AttributeTok{fun=}\StringTok{"mean"}\NormalTok{,}\AttributeTok{na.rm=}\ConstantTok{TRUE}\NormalTok{)}
\NormalTok{centrets}\OtherTok{=}\NormalTok{slanis}\SpecialCharTok{{-}}\NormalTok{videjais[,}\DecValTok{1}\NormalTok{]}
\NormalTok{standartnovirze}\OtherTok{=}\NormalTok{terra}\SpecialCharTok{::}\FunctionTok{global}\NormalTok{(centrets,}\AttributeTok{fun=}\StringTok{"rms"}\NormalTok{,}\AttributeTok{na.rm=}\ConstantTok{TRUE}\NormalTok{)}
\NormalTok{merogots}\OtherTok{=}\NormalTok{centrets}\SpecialCharTok{/}\NormalTok{standartnovirze[,}\DecValTok{1}\NormalTok{]}
\FunctionTok{writeRaster}\NormalTok{(merogots,}
      \AttributeTok{filename=}\NormalTok{saglabasanas\_cels,}
      \AttributeTok{overwrite=}\ConstantTok{TRUE}\NormalTok{)}
\end{Highlighting}
\end{Shaded}

\section{ForestsTrees\_Coniferous\_r3000}\label{ch06.391}

\textbf{filename:} \texttt{ForestsTrees\_Coniferous\_r3000.tif}

\textbf{layername:} \texttt{egv\_391}

\textbf{English name:} Fractional cover of Coniferous Forests within the 3 km
landscape

\textbf{Latvian name:} Skujkoku mežu platības īpatsvars 3 km ainavā

\textbf{Procedure:} The cover fraction within a radius of 3000 m around the analysis grid cell
is calculated as the area-weighted sum of the \hyperref[ch06.388]{analysis cells} inside
the buffer, using the workflow \texttt{egvtools::radius\_function()}. During the calculation of the landscape
metric, inverse distance weighted (power = 2) gap filling on the output is
applied to ensure no missing values at the edges. Then the layer is
rewritten to set its name. Finally, the layer is standardised by
subtracting the arithmetic mean and dividing by the root mean squared error.

\begin{Shaded}
\begin{Highlighting}[]
\CommentTok{\# libs {-}{-}{-}{-}}
\ControlFlowTok{if}\NormalTok{(}\SpecialCharTok{!}\FunctionTok{require}\NormalTok{(terra)) \{}\FunctionTok{install.packages}\NormalTok{(}\StringTok{"terra"}\NormalTok{); }\FunctionTok{require}\NormalTok{(terra)\}}
\ControlFlowTok{if}\NormalTok{(}\SpecialCharTok{!}\FunctionTok{require}\NormalTok{(egvtools)) \{remotes}\SpecialCharTok{::}\FunctionTok{install\_github}\NormalTok{(}\StringTok{"aavotins/egvtools"}\NormalTok{); }\FunctionTok{require}\NormalTok{(egvtools)\}}


\CommentTok{\# Templates {-}{-}{-}{-}{-}}
\NormalTok{template100}\OtherTok{=}\FunctionTok{rast}\NormalTok{(}\StringTok{"./Templates/TemplateRasters/LV100m\_10km.tif"}\NormalTok{)}

\CommentTok{\# radii {-}{-}{-}{-}}
\FunctionTok{radius\_function}\NormalTok{(}
 \AttributeTok{kvadrati\_path =} \StringTok{"./Templates/TemplateGrids/tiles/"}\NormalTok{,}
 \AttributeTok{radii\_path   =} \StringTok{"./Templates/TemplateGridPoints/tiles/"}\NormalTok{,}
 \AttributeTok{tikls100\_path =} \StringTok{"./Templates/TemplateGrids/tikls100\_sauzeme.parquet"}\NormalTok{,}
 \AttributeTok{template\_path =} \StringTok{"./Templates/TemplateRasters/LV100m\_10km.tif"}\NormalTok{,}
 \AttributeTok{input\_layers  =} \FunctionTok{c}\NormalTok{(}\StringTok{"./RasterGrids\_100m/2024/RAW/ForestsTrees\_Coniferous\_cell.tif"}\NormalTok{),}
 \AttributeTok{layer\_prefixes =} \FunctionTok{c}\NormalTok{(}\StringTok{"ForestsTrees\_Coniferous"}\NormalTok{),}
 \AttributeTok{output\_dir   =} \StringTok{"./RasterGrids\_100m/2024/RAW/"}\NormalTok{,}
 \AttributeTok{n\_workers   =} \DecValTok{6}\NormalTok{,}
 \AttributeTok{radii     =} \FunctionTok{c}\NormalTok{(}\StringTok{"r3000"}\NormalTok{),}
 \AttributeTok{radius\_mode  =} \StringTok{"sparse"}\NormalTok{,}
 \AttributeTok{extract\_fun  =} \StringTok{"mean"}\NormalTok{,}
 \AttributeTok{fill\_missing  =} \ConstantTok{TRUE}\NormalTok{,}
 \AttributeTok{IDW\_weight   =} \DecValTok{2}\NormalTok{,}
 \AttributeTok{future\_max\_size =} \DecValTok{40} \SpecialCharTok{*} \DecValTok{1024}\SpecialCharTok{\^{}}\DecValTok{3}\NormalTok{)}


\CommentTok{\# ForestsTrees\_Coniferous\_r3000.tif egv\_391}
\NormalTok{slanis}\OtherTok{=}\FunctionTok{rast}\NormalTok{(}\StringTok{"./RasterGrids\_100m/2024/RAW/ForestsTrees\_Coniferous\_r3000.tif"}\NormalTok{)}
\FunctionTok{names}\NormalTok{(slanis)}\OtherTok{=}\StringTok{"egv\_391"}
\NormalTok{slanis2}\OtherTok{=}\FunctionTok{project}\NormalTok{(slanis,template100)}
\FunctionTok{writeRaster}\NormalTok{(slanis2,}
      \StringTok{"./RasterGrids\_100m/2024/RAW/ForestsTrees\_Coniferous\_r3000.tif"}\NormalTok{,}
      \AttributeTok{overwrite=}\ConstantTok{TRUE}\NormalTok{)}

\CommentTok{\# standardisation {-}{-}{-}{-}}
\ControlFlowTok{if}\NormalTok{(}\SpecialCharTok{!}\FunctionTok{require}\NormalTok{(terra)) \{}\FunctionTok{install.packages}\NormalTok{(}\StringTok{"terra"}\NormalTok{); }\FunctionTok{require}\NormalTok{(terra)\}}
\ControlFlowTok{if}\NormalTok{(}\SpecialCharTok{!}\FunctionTok{require}\NormalTok{(tidyverse)) \{}\FunctionTok{install.packages}\NormalTok{(}\StringTok{"tidyverse"}\NormalTok{); }\FunctionTok{require}\NormalTok{(tidyverse)\}}

\NormalTok{nosaukums}\OtherTok{=}\StringTok{"ForestsTrees\_Coniferous\_r3000.tif"}
\NormalTok{ielasisanas\_cels}\OtherTok{=}\FunctionTok{paste0}\NormalTok{(}\StringTok{"./RasterGrids\_100m/2024/RAW/"}\NormalTok{,nosaukums)}
\NormalTok{saglabasanas\_cels}\OtherTok{=}\FunctionTok{paste0}\NormalTok{(}\StringTok{"./RasterGrids\_100m/2024/Scaled/"}\NormalTok{,nosaukums)}
\NormalTok{slanis}\OtherTok{=}\FunctionTok{rast}\NormalTok{(ielasisanas\_cels)}
\NormalTok{videjais}\OtherTok{=}\FunctionTok{global}\NormalTok{(slanis,}\AttributeTok{fun=}\StringTok{"mean"}\NormalTok{,}\AttributeTok{na.rm=}\ConstantTok{TRUE}\NormalTok{)}
\NormalTok{centrets}\OtherTok{=}\NormalTok{slanis}\SpecialCharTok{{-}}\NormalTok{videjais[,}\DecValTok{1}\NormalTok{]}
\NormalTok{standartnovirze}\OtherTok{=}\NormalTok{terra}\SpecialCharTok{::}\FunctionTok{global}\NormalTok{(centrets,}\AttributeTok{fun=}\StringTok{"rms"}\NormalTok{,}\AttributeTok{na.rm=}\ConstantTok{TRUE}\NormalTok{)}
\NormalTok{merogots}\OtherTok{=}\NormalTok{centrets}\SpecialCharTok{/}\NormalTok{standartnovirze[,}\DecValTok{1}\NormalTok{]}
\FunctionTok{writeRaster}\NormalTok{(merogots,}
      \AttributeTok{filename=}\NormalTok{saglabasanas\_cels,}
      \AttributeTok{overwrite=}\ConstantTok{TRUE}\NormalTok{)}
\end{Highlighting}
\end{Shaded}

\section{ForestsTrees\_Coniferous\_r10000}\label{ch06.392}

\textbf{filename:} \texttt{ForestsTrees\_Coniferous\_r10000.tif}

\textbf{layername:} \texttt{egv\_392}

\textbf{English name:} Fractional cover of Coniferous Forests within the 10 km
landscape

\textbf{Latvian name:} Skujkoku mežu platības īpatsvars 10 km ainavā

\textbf{Procedure:} The cover fraction within a radius of 10000 m around the analysis grid cell
is calculated as the area-weighted sum of the \hyperref[ch06.388]{analysis cells} inside
the buffer, using the workflow \texttt{egvtools::radius\_function()}. During the calculation of the landscape
metric, inverse distance weighted (power = 2) gap filling on the output is
applied to ensure no missing values at the edges. Then the layer is
rewritten to set its name. Finally, the layer is standardised by
subtracting the arithmetic mean and dividing by the root mean squared error.

\begin{Shaded}
\begin{Highlighting}[]
\CommentTok{\# libs {-}{-}{-}{-}}
\ControlFlowTok{if}\NormalTok{(}\SpecialCharTok{!}\FunctionTok{require}\NormalTok{(terra)) \{}\FunctionTok{install.packages}\NormalTok{(}\StringTok{"terra"}\NormalTok{); }\FunctionTok{require}\NormalTok{(terra)\}}
\ControlFlowTok{if}\NormalTok{(}\SpecialCharTok{!}\FunctionTok{require}\NormalTok{(egvtools)) \{remotes}\SpecialCharTok{::}\FunctionTok{install\_github}\NormalTok{(}\StringTok{"aavotins/egvtools"}\NormalTok{); }\FunctionTok{require}\NormalTok{(egvtools)\}}


\CommentTok{\# Templates {-}{-}{-}{-}{-}}
\NormalTok{template100}\OtherTok{=}\FunctionTok{rast}\NormalTok{(}\StringTok{"./Templates/TemplateRasters/LV100m\_10km.tif"}\NormalTok{)}

\CommentTok{\# radii {-}{-}{-}{-}}
\FunctionTok{radius\_function}\NormalTok{(}
 \AttributeTok{kvadrati\_path =} \StringTok{"./Templates/TemplateGrids/tiles/"}\NormalTok{,}
 \AttributeTok{radii\_path   =} \StringTok{"./Templates/TemplateGridPoints/tiles/"}\NormalTok{,}
 \AttributeTok{tikls100\_path =} \StringTok{"./Templates/TemplateGrids/tikls100\_sauzeme.parquet"}\NormalTok{,}
 \AttributeTok{template\_path =} \StringTok{"./Templates/TemplateRasters/LV100m\_10km.tif"}\NormalTok{,}
 \AttributeTok{input\_layers  =} \FunctionTok{c}\NormalTok{(}\StringTok{"./RasterGrids\_100m/2024/RAW/ForestsTrees\_Coniferous\_cell.tif"}\NormalTok{),}
 \AttributeTok{layer\_prefixes =} \FunctionTok{c}\NormalTok{(}\StringTok{"ForestsTrees\_Coniferous"}\NormalTok{),}
 \AttributeTok{output\_dir   =} \StringTok{"./RasterGrids\_100m/2024/RAW/"}\NormalTok{,}
 \AttributeTok{n\_workers   =} \DecValTok{6}\NormalTok{,}
 \AttributeTok{radii     =} \FunctionTok{c}\NormalTok{(}\StringTok{"r10000"}\NormalTok{),}
 \AttributeTok{radius\_mode  =} \StringTok{"sparse"}\NormalTok{,}
 \AttributeTok{extract\_fun  =} \StringTok{"mean"}\NormalTok{,}
 \AttributeTok{fill\_missing  =} \ConstantTok{TRUE}\NormalTok{,}
 \AttributeTok{IDW\_weight   =} \DecValTok{2}\NormalTok{,}
 \AttributeTok{future\_max\_size =} \DecValTok{40} \SpecialCharTok{*} \DecValTok{1024}\SpecialCharTok{\^{}}\DecValTok{3}\NormalTok{)}


\CommentTok{\# ForestsTrees\_Coniferous\_r10000.tif    egv\_392}
\NormalTok{slanis}\OtherTok{=}\FunctionTok{rast}\NormalTok{(}\StringTok{"./RasterGrids\_100m/2024/RAW/ForestsTrees\_Coniferous\_r10000.tif"}\NormalTok{)}
\FunctionTok{names}\NormalTok{(slanis)}\OtherTok{=}\StringTok{"egv\_392"}
\NormalTok{slanis2}\OtherTok{=}\FunctionTok{project}\NormalTok{(slanis,template100)}
\FunctionTok{writeRaster}\NormalTok{(slanis2,}
      \StringTok{"./RasterGrids\_100m/2024/RAW/ForestsTrees\_Coniferous\_r10000.tif"}\NormalTok{,}
      \AttributeTok{overwrite=}\ConstantTok{TRUE}\NormalTok{)}

\CommentTok{\# standardisation {-}{-}{-}{-}}
\ControlFlowTok{if}\NormalTok{(}\SpecialCharTok{!}\FunctionTok{require}\NormalTok{(terra)) \{}\FunctionTok{install.packages}\NormalTok{(}\StringTok{"terra"}\NormalTok{); }\FunctionTok{require}\NormalTok{(terra)\}}
\ControlFlowTok{if}\NormalTok{(}\SpecialCharTok{!}\FunctionTok{require}\NormalTok{(tidyverse)) \{}\FunctionTok{install.packages}\NormalTok{(}\StringTok{"tidyverse"}\NormalTok{); }\FunctionTok{require}\NormalTok{(tidyverse)\}}

\NormalTok{nosaukums}\OtherTok{=}\StringTok{"ForestsTrees\_Coniferous\_r10000.tif"}
\NormalTok{ielasisanas\_cels}\OtherTok{=}\FunctionTok{paste0}\NormalTok{(}\StringTok{"./RasterGrids\_100m/2024/RAW/"}\NormalTok{,nosaukums)}
\NormalTok{saglabasanas\_cels}\OtherTok{=}\FunctionTok{paste0}\NormalTok{(}\StringTok{"./RasterGrids\_100m/2024/Scaled/"}\NormalTok{,nosaukums)}
\NormalTok{slanis}\OtherTok{=}\FunctionTok{rast}\NormalTok{(ielasisanas\_cels)}
\NormalTok{videjais}\OtherTok{=}\FunctionTok{global}\NormalTok{(slanis,}\AttributeTok{fun=}\StringTok{"mean"}\NormalTok{,}\AttributeTok{na.rm=}\ConstantTok{TRUE}\NormalTok{)}
\NormalTok{centrets}\OtherTok{=}\NormalTok{slanis}\SpecialCharTok{{-}}\NormalTok{videjais[,}\DecValTok{1}\NormalTok{]}
\NormalTok{standartnovirze}\OtherTok{=}\NormalTok{terra}\SpecialCharTok{::}\FunctionTok{global}\NormalTok{(centrets,}\AttributeTok{fun=}\StringTok{"rms"}\NormalTok{,}\AttributeTok{na.rm=}\ConstantTok{TRUE}\NormalTok{)}
\NormalTok{merogots}\OtherTok{=}\NormalTok{centrets}\SpecialCharTok{/}\NormalTok{standartnovirze[,}\DecValTok{1}\NormalTok{]}
\FunctionTok{writeRaster}\NormalTok{(merogots,}
      \AttributeTok{filename=}\NormalTok{saglabasanas\_cels,}
      \AttributeTok{overwrite=}\ConstantTok{TRUE}\NormalTok{)}
\end{Highlighting}
\end{Shaded}

\section{ForestsTrees\_Mixed\_cell}\label{ch06.393}

\textbf{filename:} \texttt{ForestsTrees\_Mixed\_cell.tif}

\textbf{layername:} \texttt{egv\_393}

\textbf{English name:} Fractional cover of Mixed Forests within the analysis cell (1
ha)

\textbf{Latvian name:} Jauktu koku mežu platības īpatsvars analīzes šūnā (1 ha)

\textbf{Procedure:} Most EGVs describing forests are spatially restricted to areas outside
of clearcuts and dead stands. This mask is created using a combination of
the \hyperref[Ch04.01]{State Forest Service's
State Forest Registry} land category 12 and 14, and \hyperref[Ch04.09]{The
Global Forest Watch} pixels classified as lost tree canopy cover since
2020 (raster layer matching input, presence = 1, absence = 0).

To prepare this EGV, stands from the \hyperref[Ch04.01]{State Forest Service's State Forest
Registry} are classified into (in order):

\begin{itemize}
\item
  coniferous (see \hyperref[Ch01]{Terminology and acronyms} for species codes) if
  timber volume of those species exceeded 75\%;
\item
  Boreal deciduous if timber volume of those species exceeded 75\%;
\item
  temperate deciduous if timber volume of those species exceeded 50\%;
\item
  mixed otherwise;
\end{itemize}

then mixed stands are selected and geometries are
rasterised (presence = 1, NA otherwise). Rasterisation is
performed using the workflow \texttt{egvtools::polygon2input()}, restricting to pixels outside clearcut
mask and covering background with value 0. The resulting layer
is then aggregated to EGV resolution using the workflow \texttt{egvtools::input2egv()}, which
calculates the arithmetic mean to determine the cover fraction. During
aggregation, inverse distance weighted (power = 2) gap filling on the output is
applied to ensure no missing values at the edges. Finally, the layer is
standardised by subtracting the arithmetic mean and dividing by the root mean squared
error.

\begin{Shaded}
\begin{Highlighting}[]
\CommentTok{\# libs {-}{-}{-}{-}}
\ControlFlowTok{if}\NormalTok{(}\SpecialCharTok{!}\FunctionTok{require}\NormalTok{(egvtools)) \{remotes}\SpecialCharTok{::}\FunctionTok{install\_github}\NormalTok{(}\StringTok{"aavotins/egvtools"}\NormalTok{); }\FunctionTok{require}\NormalTok{(egvtools)\}}
\ControlFlowTok{if}\NormalTok{(}\SpecialCharTok{!}\FunctionTok{require}\NormalTok{(terra)) \{}\FunctionTok{install.packages}\NormalTok{(}\StringTok{"terra"}\NormalTok{); }\FunctionTok{require}\NormalTok{(terra)\}}
\ControlFlowTok{if}\NormalTok{(}\SpecialCharTok{!}\FunctionTok{require}\NormalTok{(sf)) \{}\FunctionTok{install.packages}\NormalTok{(}\StringTok{"sf"}\NormalTok{); }\FunctionTok{require}\NormalTok{(sf)\}}
\ControlFlowTok{if}\NormalTok{(}\SpecialCharTok{!}\FunctionTok{require}\NormalTok{(tidyverse)) \{}\FunctionTok{install.packages}\NormalTok{(}\StringTok{"tidyverse"}\NormalTok{); }\FunctionTok{require}\NormalTok{(tidyverse)\}}
\ControlFlowTok{if}\NormalTok{(}\SpecialCharTok{!}\FunctionTok{require}\NormalTok{(sfarrow)) \{}\FunctionTok{install.packages}\NormalTok{(}\StringTok{"sfarrow"}\NormalTok{); }\FunctionTok{require}\NormalTok{(sfarrow)\}}
\ControlFlowTok{if}\NormalTok{(}\SpecialCharTok{!}\FunctionTok{require}\NormalTok{(readxl)) \{}\FunctionTok{install.packages}\NormalTok{(}\StringTok{"readxl"}\NormalTok{); }\FunctionTok{require}\NormalTok{(readxl)\}}
\ControlFlowTok{if}\NormalTok{(}\SpecialCharTok{!}\FunctionTok{require}\NormalTok{(raster)) \{}\FunctionTok{install.packages}\NormalTok{(}\StringTok{"raster"}\NormalTok{); }\FunctionTok{require}\NormalTok{(raster)\}}
\ControlFlowTok{if}\NormalTok{(}\SpecialCharTok{!}\FunctionTok{require}\NormalTok{(fasterize)) \{}\FunctionTok{install.packages}\NormalTok{(}\StringTok{"fasterize"}\NormalTok{); }\FunctionTok{require}\NormalTok{(fasterize)\}}

\CommentTok{\# templates {-}{-}{-}{-}}
\NormalTok{template100}\OtherTok{=}\FunctionTok{rast}\NormalTok{(}\StringTok{"./Templates/TemplateRasters/LV100m\_10km.tif"}\NormalTok{)}
\NormalTok{template10}\OtherTok{=}\FunctionTok{rast}\NormalTok{(}\StringTok{"./Templates/TemplateRasters/LV10m\_10km.tif"}\NormalTok{)}
\NormalTok{rastrs10}\OtherTok{=}\FunctionTok{raster}\NormalTok{(template10)}

\NormalTok{nulls10}\OtherTok{=}\FunctionTok{rast}\NormalTok{(}\StringTok{"./Templates/TemplateRasters/nulls\_LV10m\_10km.tif"}\NormalTok{)}
\NormalTok{nulls100}\OtherTok{=}\FunctionTok{rast}\NormalTok{(}\StringTok{"./Templates/TemplateRasters/nulls\_LV100m\_10km.tif"}\NormalTok{)}


\CommentTok{\# simple landscape {-}{-}{-}{-}}
\NormalTok{simple\_landscape}\OtherTok{=}\FunctionTok{rast}\NormalTok{(}\StringTok{"RasterGrids\_10m/2024/Ainava\_vienk\_mask.tif"}\NormalTok{)}

\CommentTok{\# mvr {-}{-}{-}{-}}
\NormalTok{mvr}\OtherTok{=}\FunctionTok{st\_read\_parquet}\NormalTok{(}\StringTok{"./Geodata/2024/MVR/nogabali\_2024janv.parquet"}\NormalTok{)}
\NormalTok{mvr}\SpecialCharTok{$}\NormalTok{yes}\OtherTok{=}\DecValTok{1}

\CommentTok{\# clear cut mask {-}{-}{-}{-}}
\NormalTok{izcirtumi}\OtherTok{=}\NormalTok{mvr }\SpecialCharTok{\%\textgreater{}\%} 
 \FunctionTok{filter}\NormalTok{(zkat }\SpecialCharTok{\%in\%} \FunctionTok{c}\NormalTok{(}\StringTok{"12"}\NormalTok{,}\StringTok{"14"}\NormalTok{)) }\SpecialCharTok{\%\textgreater{}\%} 
\NormalTok{ dplyr}\SpecialCharTok{::}\FunctionTok{select}\NormalTok{(yes)}
\NormalTok{r\_izcirtumi\_mvr}\OtherTok{=}\FunctionTok{fasterize}\NormalTok{(izcirtumi,rastrs10,}\AttributeTok{field=}\StringTok{"yes"}\NormalTok{)}
\NormalTok{t\_izcirtumi\_mvr}\OtherTok{=}\FunctionTok{rast}\NormalTok{(r\_izcirtumi\_mvr)}
\FunctionTok{plot}\NormalTok{(t\_izcirtumi\_mvr)}

\NormalTok{tcl}\OtherTok{=}\FunctionTok{rast}\NormalTok{(}\StringTok{"./Geodata/2024/Trees/GFW/TreeCoverLoss\_v1\_12.tif"}\NormalTok{)}
\NormalTok{tcl2}\OtherTok{=}\FunctionTok{ifel}\NormalTok{(tcl}\SpecialCharTok{\textless{}}\DecValTok{20}\NormalTok{,}\DecValTok{0}\NormalTok{,}\DecValTok{1}\NormalTok{)}
\NormalTok{tclX}\OtherTok{=}\FunctionTok{cover}\NormalTok{(tcl2,nulls10)}
\FunctionTok{plot}\NormalTok{(tclX)}

\NormalTok{clearcut\_mask}\OtherTok{=}\FunctionTok{cover}\NormalTok{(t\_izcirtumi\_mvr,tclX,}
          \AttributeTok{filename=}\StringTok{"./RasterGrids\_10m/2024/Mask\_clearcuts.tif"}\NormalTok{,}
          \AttributeTok{overwrite=}\ConstantTok{TRUE}\NormalTok{)}
\FunctionTok{plot}\NormalTok{(clearcut\_mask)}

\FunctionTok{rm}\NormalTok{(izcirtumi)}
\FunctionTok{rm}\NormalTok{(r\_izcirtumi\_mvr)}
\FunctionTok{rm}\NormalTok{(t\_izcirtumi\_mvr)}
\FunctionTok{rm}\NormalTok{(tcl)}
\FunctionTok{rm}\NormalTok{(tcl2)}
\FunctionTok{rm}\NormalTok{(tclX)}

\CommentTok{\# ForestsTrees\_Mixed\_cell.tif   egv\_393 {-}{-}{-}{-}}
\NormalTok{skujkoki}\OtherTok{=}\FunctionTok{c}\NormalTok{(}\StringTok{"1"}\NormalTok{,}\StringTok{"3"}\NormalTok{,}\StringTok{"13"}\NormalTok{,}\StringTok{"14"}\NormalTok{,}\StringTok{"15"}\NormalTok{,}\StringTok{"22"}\NormalTok{,}\StringTok{"23"}\NormalTok{,}\StringTok{"28"}\NormalTok{) }\CommentTok{\# 8}
\NormalTok{saurlapji}\OtherTok{=}\FunctionTok{c}\NormalTok{(}\StringTok{"4"}\NormalTok{,}\StringTok{"6"}\NormalTok{,}\StringTok{"8"}\NormalTok{,}\StringTok{"9"}\NormalTok{,}\StringTok{"19"}\NormalTok{,}\StringTok{"20"}\NormalTok{,}\StringTok{"21"}\NormalTok{,}\StringTok{"32"}\NormalTok{,}\StringTok{"35"}\NormalTok{,}\StringTok{"68"}\NormalTok{) }\CommentTok{\# 10}
\NormalTok{platlapji}\OtherTok{=}\FunctionTok{c}\NormalTok{(}\StringTok{"10"}\NormalTok{,}\StringTok{"11"}\NormalTok{,}\StringTok{"12"}\NormalTok{,}\StringTok{"16"}\NormalTok{,}\StringTok{"17"}\NormalTok{,}\StringTok{"18"}\NormalTok{,}\StringTok{"24"}\NormalTok{,}\StringTok{"25"}\NormalTok{,}\StringTok{"26"}\NormalTok{,}\StringTok{"27"}\NormalTok{,}\StringTok{"28"}\NormalTok{,}\StringTok{"29"}\NormalTok{,}\StringTok{"50"}\NormalTok{,}
      \StringTok{"61"}\NormalTok{,}\StringTok{"62"}\NormalTok{,}\StringTok{"63"}\NormalTok{,}\StringTok{"64"}\NormalTok{,}\StringTok{"65"}\NormalTok{,}\StringTok{"66"}\NormalTok{,}\StringTok{"67"}\NormalTok{,}\StringTok{"69"}\NormalTok{) }\CommentTok{\# 21}
\NormalTok{mvr}\OtherTok{=}\NormalTok{mvr }\SpecialCharTok{\%\textgreater{}\%} 
 \FunctionTok{mutate}\NormalTok{(}\AttributeTok{kraja\_skujkoku=}\FunctionTok{ifelse}\NormalTok{(s10 }\SpecialCharTok{\%in\%}\NormalTok{ skujkoki,v10,}\DecValTok{0}\NormalTok{)}\SpecialCharTok{+}
      \FunctionTok{ifelse}\NormalTok{(s11 }\SpecialCharTok{\%in\%}\NormalTok{ skujkoki,v11,}\DecValTok{0}\NormalTok{)}\SpecialCharTok{+}\FunctionTok{ifelse}\NormalTok{(s12 }\SpecialCharTok{\%in\%}\NormalTok{ skujkoki,v12,}\DecValTok{0}\NormalTok{)}\SpecialCharTok{+}
      \FunctionTok{ifelse}\NormalTok{(s13 }\SpecialCharTok{\%in\%}\NormalTok{ skujkoki,v13,}\DecValTok{0}\NormalTok{)}\SpecialCharTok{+}\FunctionTok{ifelse}\NormalTok{(s14 }\SpecialCharTok{\%in\%}\NormalTok{ skujkoki,v14,}\DecValTok{0}\NormalTok{),}
     \AttributeTok{kraja\_saurlapju=}\FunctionTok{ifelse}\NormalTok{(s10 }\SpecialCharTok{\%in\%}\NormalTok{ saurlapji,v10,}\DecValTok{0}\NormalTok{)}\SpecialCharTok{+}
      \FunctionTok{ifelse}\NormalTok{(s11 }\SpecialCharTok{\%in\%}\NormalTok{ saurlapji,v11,}\DecValTok{0}\NormalTok{)}\SpecialCharTok{+}\FunctionTok{ifelse}\NormalTok{(s12 }\SpecialCharTok{\%in\%}\NormalTok{ saurlapji,v12,}\DecValTok{0}\NormalTok{)}\SpecialCharTok{+}
      \FunctionTok{ifelse}\NormalTok{(s13 }\SpecialCharTok{\%in\%}\NormalTok{ saurlapji,v13,}\DecValTok{0}\NormalTok{)}\SpecialCharTok{+}\FunctionTok{ifelse}\NormalTok{(s14 }\SpecialCharTok{\%in\%}\NormalTok{ saurlapji,v14,}\DecValTok{0}\NormalTok{),}
     \AttributeTok{kraja\_platlapju=}\FunctionTok{ifelse}\NormalTok{(s10 }\SpecialCharTok{\%in\%}\NormalTok{ platlapji,v10,}\DecValTok{0}\NormalTok{)}\SpecialCharTok{+}
      \FunctionTok{ifelse}\NormalTok{(s11 }\SpecialCharTok{\%in\%}\NormalTok{ platlapji,v11,}\DecValTok{0}\NormalTok{)}\SpecialCharTok{+}\FunctionTok{ifelse}\NormalTok{(s12 }\SpecialCharTok{\%in\%}\NormalTok{ platlapji,v12,}\DecValTok{0}\NormalTok{)}\SpecialCharTok{+}
      \FunctionTok{ifelse}\NormalTok{(s13 }\SpecialCharTok{\%in\%}\NormalTok{ platlapji,v13,}\DecValTok{0}\NormalTok{)}\SpecialCharTok{+}\FunctionTok{ifelse}\NormalTok{(s14 }\SpecialCharTok{\%in\%}\NormalTok{ platlapji,v14,}\DecValTok{0}\NormalTok{)) }\SpecialCharTok{\%\textgreater{}\%} 
 \FunctionTok{mutate}\NormalTok{(}\AttributeTok{kopeja\_kraja=}\NormalTok{kraja\_skujkoku}\SpecialCharTok{+}\NormalTok{kraja\_platlapju}\SpecialCharTok{+}\NormalTok{kraja\_saurlapju) }\SpecialCharTok{\%\textgreater{}\%} 
 \FunctionTok{mutate}\NormalTok{(}\AttributeTok{tips=}\FunctionTok{ifelse}\NormalTok{(kraja\_skujkoku}\SpecialCharTok{/}\NormalTok{kopeja\_kraja}\SpecialCharTok{\textgreater{}=}\FloatTok{0.75}\NormalTok{,}\StringTok{"skujkoku"}\NormalTok{,}
           \FunctionTok{ifelse}\NormalTok{(kraja\_saurlapju}\SpecialCharTok{/}\NormalTok{kopeja\_kraja}\SpecialCharTok{\textgreater{}=}\FloatTok{0.75}\NormalTok{,}\StringTok{"saurlapju"}\NormalTok{,}
              \FunctionTok{ifelse}\NormalTok{(kraja\_platlapju}\SpecialCharTok{/}\NormalTok{kopeja\_kraja}\SpecialCharTok{\textgreater{}}\FloatTok{0.5}\NormalTok{,}\StringTok{"platlapju"}\NormalTok{,}
                  \StringTok{"jauktu koku"}\NormalTok{))))}
\NormalTok{nogabali}\OtherTok{=}\NormalTok{mvr }\SpecialCharTok{\%\textgreater{}\%} 
 \FunctionTok{filter}\NormalTok{(zkat}\SpecialCharTok{==}\StringTok{"10"}\SpecialCharTok{\&}\NormalTok{tips}\SpecialCharTok{==}\StringTok{"jauktu koku"}\NormalTok{)}

\NormalTok{p2i\_rez}\OtherTok{=}\NormalTok{egvtools}\SpecialCharTok{::}\FunctionTok{polygon2input}\NormalTok{(}\AttributeTok{vector\_data =}\NormalTok{ nogabali,}
                \AttributeTok{template\_path =} \StringTok{"./Templates/TemplateRasters/LV10m\_10km.tif"}\NormalTok{,}
                \AttributeTok{out\_path =} \StringTok{"./RasterGrids\_10m/2024/"}\NormalTok{,}
                \AttributeTok{file\_name =} \StringTok{"ForestsTrees\_Mixed\_input.tif"}\NormalTok{,}
                \AttributeTok{value\_field =} \StringTok{"yes"}\NormalTok{,}
                \AttributeTok{restrict\_to =}\NormalTok{ clearcut\_mask,}
                \AttributeTok{restrict\_values =} \DecValTok{0}\NormalTok{,}
                \AttributeTok{prepare=}\ConstantTok{FALSE}\NormalTok{,}
                \AttributeTok{background\_raster =} \StringTok{"./Templates/TemplateRasters/nulls\_LV10m\_10km.tif"}\NormalTok{,}
                \AttributeTok{plot\_result =} \ConstantTok{TRUE}\NormalTok{)}
\NormalTok{p2i\_rez}
\NormalTok{i2e\_rez}\OtherTok{=}\NormalTok{egvtools}\SpecialCharTok{::}\FunctionTok{input2egv}\NormalTok{(}\AttributeTok{input=}\FunctionTok{paste0}\NormalTok{(}\StringTok{"./RasterGrids\_10m/2024/"}\NormalTok{,}
                     \StringTok{"ForestsTrees\_Mixed\_input.tif"}\NormalTok{),}
              \AttributeTok{egv\_template=} \StringTok{"./Templates/TemplateRasters/LV100m\_10km.tif"}\NormalTok{,}
              \AttributeTok{summary\_function =} \StringTok{"average"}\NormalTok{,}
              \AttributeTok{missing\_job =} \StringTok{"FillOutput"}\NormalTok{,}
              \AttributeTok{outlocation =} \StringTok{"./RasterGrids\_100m/2024/RAW/"}\NormalTok{,}
              \AttributeTok{outfilename =} \StringTok{"ForestsTrees\_Mixed\_cell.tif"}\NormalTok{,}
              \AttributeTok{layername =} \StringTok{"egv\_393"}\NormalTok{,}
              \AttributeTok{idw\_weight =} \DecValTok{2}\NormalTok{,}
              \AttributeTok{plot\_gaps =} \ConstantTok{FALSE}\NormalTok{,}\AttributeTok{plot\_final =} \ConstantTok{TRUE}\NormalTok{)}
\NormalTok{i2e\_rez}
\FunctionTok{rm}\NormalTok{(nogabali)}
\FunctionTok{rm}\NormalTok{(p2i\_rez)}
\FunctionTok{rm}\NormalTok{(i2e\_rez)}
\FunctionTok{unlink}\NormalTok{(}\StringTok{"./RasterGrids\_10m/2024/ForestsTrees\_Mixed\_input.tif"}\NormalTok{)}

\CommentTok{\# standardisation {-}{-}{-}{-}}
\ControlFlowTok{if}\NormalTok{(}\SpecialCharTok{!}\FunctionTok{require}\NormalTok{(terra)) \{}\FunctionTok{install.packages}\NormalTok{(}\StringTok{"terra"}\NormalTok{); }\FunctionTok{require}\NormalTok{(terra)\}}
\ControlFlowTok{if}\NormalTok{(}\SpecialCharTok{!}\FunctionTok{require}\NormalTok{(tidyverse)) \{}\FunctionTok{install.packages}\NormalTok{(}\StringTok{"tidyverse"}\NormalTok{); }\FunctionTok{require}\NormalTok{(tidyverse)\}}

\NormalTok{nosaukums}\OtherTok{=}\StringTok{"ForestsTrees\_Mixed\_cell.tif"}
\NormalTok{ielasisanas\_cels}\OtherTok{=}\FunctionTok{paste0}\NormalTok{(}\StringTok{"./RasterGrids\_100m/2024/RAW/"}\NormalTok{,nosaukums)}
\NormalTok{saglabasanas\_cels}\OtherTok{=}\FunctionTok{paste0}\NormalTok{(}\StringTok{"./RasterGrids\_100m/2024/Scaled/"}\NormalTok{,nosaukums)}
\NormalTok{slanis}\OtherTok{=}\FunctionTok{rast}\NormalTok{(ielasisanas\_cels)}
\NormalTok{videjais}\OtherTok{=}\FunctionTok{global}\NormalTok{(slanis,}\AttributeTok{fun=}\StringTok{"mean"}\NormalTok{,}\AttributeTok{na.rm=}\ConstantTok{TRUE}\NormalTok{)}
\NormalTok{centrets}\OtherTok{=}\NormalTok{slanis}\SpecialCharTok{{-}}\NormalTok{videjais[,}\DecValTok{1}\NormalTok{]}
\NormalTok{standartnovirze}\OtherTok{=}\NormalTok{terra}\SpecialCharTok{::}\FunctionTok{global}\NormalTok{(centrets,}\AttributeTok{fun=}\StringTok{"rms"}\NormalTok{,}\AttributeTok{na.rm=}\ConstantTok{TRUE}\NormalTok{)}
\NormalTok{merogots}\OtherTok{=}\NormalTok{centrets}\SpecialCharTok{/}\NormalTok{standartnovirze[,}\DecValTok{1}\NormalTok{]}
\FunctionTok{writeRaster}\NormalTok{(merogots,}
      \AttributeTok{filename=}\NormalTok{saglabasanas\_cels,}
      \AttributeTok{overwrite=}\ConstantTok{TRUE}\NormalTok{)}
\end{Highlighting}
\end{Shaded}

\section{ForestsTrees\_Mixed\_r500}\label{ch06.394}

\textbf{filename:} \texttt{ForestsTrees\_Mixed\_r500.tif}

\textbf{layername:} \texttt{egv\_394}

\textbf{English name:} Fractional cover of Mixed Forests within the 0.5 km landscape

\textbf{Latvian name:} Jauktu koku mežu platības īpatsvars 0,5 km ainavā

\textbf{Procedure:} The cover fraction within a radius of 500 m around the analysis grid cell is
calculated as the area-weighted sum of the \hyperref[ch06.393]{analysis cells} inside the
buffer, using the workflow \texttt{egvtools::radius\_function()}. During the calculation of the landscape metric,
inverse distance weighted (power = 2) gap filling on the output is applied
to ensure no missing values at the edges. Then the layer is rewritten to set
its name. Finally, the layer is standardised by subtracting the arithmetic
mean and dividing by the root mean squared error.

\begin{Shaded}
\begin{Highlighting}[]
\CommentTok{\# libs {-}{-}{-}{-}}
\ControlFlowTok{if}\NormalTok{(}\SpecialCharTok{!}\FunctionTok{require}\NormalTok{(terra)) \{}\FunctionTok{install.packages}\NormalTok{(}\StringTok{"terra"}\NormalTok{); }\FunctionTok{require}\NormalTok{(terra)\}}
\ControlFlowTok{if}\NormalTok{(}\SpecialCharTok{!}\FunctionTok{require}\NormalTok{(egvtools)) \{remotes}\SpecialCharTok{::}\FunctionTok{install\_github}\NormalTok{(}\StringTok{"aavotins/egvtools"}\NormalTok{); }\FunctionTok{require}\NormalTok{(egvtools)\}}


\CommentTok{\# Templates {-}{-}{-}{-}{-}}
\NormalTok{template100}\OtherTok{=}\FunctionTok{rast}\NormalTok{(}\StringTok{"./Templates/TemplateRasters/LV100m\_10km.tif"}\NormalTok{)}

\CommentTok{\# radii {-}{-}{-}{-}}
\FunctionTok{radius\_function}\NormalTok{(}
 \AttributeTok{kvadrati\_path =} \StringTok{"./Templates/TemplateGrids/tiles/"}\NormalTok{,}
 \AttributeTok{radii\_path   =} \StringTok{"./Templates/TemplateGridPoints/tiles/"}\NormalTok{,}
 \AttributeTok{tikls100\_path =} \StringTok{"./Templates/TemplateGrids/tikls100\_sauzeme.parquet"}\NormalTok{,}
 \AttributeTok{template\_path =} \StringTok{"./Templates/TemplateRasters/LV100m\_10km.tif"}\NormalTok{,}
 \AttributeTok{input\_layers  =} \FunctionTok{c}\NormalTok{(}\StringTok{"./RasterGrids\_100m/2024/RAW/ForestsTrees\_Mixed\_cell.tif"}\NormalTok{),}
 \AttributeTok{layer\_prefixes =} \FunctionTok{c}\NormalTok{(}\StringTok{"ForestsTrees\_Mixed"}\NormalTok{),}
 \AttributeTok{output\_dir   =} \StringTok{"./RasterGrids\_100m/2024/RAW/"}\NormalTok{,}
 \AttributeTok{n\_workers   =} \DecValTok{6}\NormalTok{,}
 \AttributeTok{radii     =} \FunctionTok{c}\NormalTok{(}\StringTok{"r500"}\NormalTok{),}
 \AttributeTok{radius\_mode  =} \StringTok{"sparse"}\NormalTok{,}
 \AttributeTok{extract\_fun  =} \StringTok{"mean"}\NormalTok{,}
 \AttributeTok{fill\_missing  =} \ConstantTok{TRUE}\NormalTok{,}
 \AttributeTok{IDW\_weight   =} \DecValTok{2}\NormalTok{,}
 \AttributeTok{future\_max\_size =} \DecValTok{40} \SpecialCharTok{*} \DecValTok{1024}\SpecialCharTok{\^{}}\DecValTok{3}\NormalTok{)}


\CommentTok{\# ForestsTrees\_Mixed\_r500.tif   egv\_394}
\NormalTok{slanis}\OtherTok{=}\FunctionTok{rast}\NormalTok{(}\StringTok{"./RasterGrids\_100m/2024/RAW/ForestsTrees\_Mixed\_r500.tif"}\NormalTok{)}
\FunctionTok{names}\NormalTok{(slanis)}\OtherTok{=}\StringTok{"egv\_394"}
\NormalTok{slanis2}\OtherTok{=}\FunctionTok{project}\NormalTok{(slanis,template100)}
\FunctionTok{writeRaster}\NormalTok{(slanis2,}
      \StringTok{"./RasterGrids\_100m/2024/RAW/ForestsTrees\_Mixed\_r500.tif"}\NormalTok{,}
      \AttributeTok{overwrite=}\ConstantTok{TRUE}\NormalTok{)}

\CommentTok{\# standardisation {-}{-}{-}{-}}
\ControlFlowTok{if}\NormalTok{(}\SpecialCharTok{!}\FunctionTok{require}\NormalTok{(terra)) \{}\FunctionTok{install.packages}\NormalTok{(}\StringTok{"terra"}\NormalTok{); }\FunctionTok{require}\NormalTok{(terra)\}}
\ControlFlowTok{if}\NormalTok{(}\SpecialCharTok{!}\FunctionTok{require}\NormalTok{(tidyverse)) \{}\FunctionTok{install.packages}\NormalTok{(}\StringTok{"tidyverse"}\NormalTok{); }\FunctionTok{require}\NormalTok{(tidyverse)\}}

\NormalTok{nosaukums}\OtherTok{=}\StringTok{"ForestsTrees\_Mixed\_r500.tif"}
\NormalTok{ielasisanas\_cels}\OtherTok{=}\FunctionTok{paste0}\NormalTok{(}\StringTok{"./RasterGrids\_100m/2024/RAW/"}\NormalTok{,nosaukums)}
\NormalTok{saglabasanas\_cels}\OtherTok{=}\FunctionTok{paste0}\NormalTok{(}\StringTok{"./RasterGrids\_100m/2024/Scaled/"}\NormalTok{,nosaukums)}
\NormalTok{slanis}\OtherTok{=}\FunctionTok{rast}\NormalTok{(ielasisanas\_cels)}
\NormalTok{videjais}\OtherTok{=}\FunctionTok{global}\NormalTok{(slanis,}\AttributeTok{fun=}\StringTok{"mean"}\NormalTok{,}\AttributeTok{na.rm=}\ConstantTok{TRUE}\NormalTok{)}
\NormalTok{centrets}\OtherTok{=}\NormalTok{slanis}\SpecialCharTok{{-}}\NormalTok{videjais[,}\DecValTok{1}\NormalTok{]}
\NormalTok{standartnovirze}\OtherTok{=}\NormalTok{terra}\SpecialCharTok{::}\FunctionTok{global}\NormalTok{(centrets,}\AttributeTok{fun=}\StringTok{"rms"}\NormalTok{,}\AttributeTok{na.rm=}\ConstantTok{TRUE}\NormalTok{)}
\NormalTok{merogots}\OtherTok{=}\NormalTok{centrets}\SpecialCharTok{/}\NormalTok{standartnovirze[,}\DecValTok{1}\NormalTok{]}
\FunctionTok{writeRaster}\NormalTok{(merogots,}
      \AttributeTok{filename=}\NormalTok{saglabasanas\_cels,}
      \AttributeTok{overwrite=}\ConstantTok{TRUE}\NormalTok{)}
\end{Highlighting}
\end{Shaded}

\section{ForestsTrees\_Mixed\_r1250}\label{ch06.395}

\textbf{filename:} \texttt{ForestsTrees\_Mixed\_r1250.tif}

\textbf{layername:} \texttt{egv\_395}

\textbf{English name:} Fractional cover of Mixed Forests within the 1.25 km landscape

\textbf{Latvian name:} Jauktu koku mežu platības īpatsvars 1,25 km ainavā

\textbf{Procedure:} The cover fraction within a radius of 1250 m around the analysis grid cell
is calculated as the area-weighted sum of the \hyperref[ch06.393]{analysis cells} inside
the buffer, using the workflow \texttt{egvtools::radius\_function()}. During the calculation of the landscape
metric, inverse distance weighted (power = 2) gap filling on the output is
applied to ensure no missing values at the edges. Then the layer is
rewritten to set its name. Finally, the layer is standardised by
subtracting the arithmetic mean and dividing by the root mean squared error.

\begin{Shaded}
\begin{Highlighting}[]
\CommentTok{\# libs {-}{-}{-}{-}}
\ControlFlowTok{if}\NormalTok{(}\SpecialCharTok{!}\FunctionTok{require}\NormalTok{(terra)) \{}\FunctionTok{install.packages}\NormalTok{(}\StringTok{"terra"}\NormalTok{); }\FunctionTok{require}\NormalTok{(terra)\}}
\ControlFlowTok{if}\NormalTok{(}\SpecialCharTok{!}\FunctionTok{require}\NormalTok{(egvtools)) \{remotes}\SpecialCharTok{::}\FunctionTok{install\_github}\NormalTok{(}\StringTok{"aavotins/egvtools"}\NormalTok{); }\FunctionTok{require}\NormalTok{(egvtools)\}}


\CommentTok{\# Templates {-}{-}{-}{-}{-}}
\NormalTok{template100}\OtherTok{=}\FunctionTok{rast}\NormalTok{(}\StringTok{"./Templates/TemplateRasters/LV100m\_10km.tif"}\NormalTok{)}

\CommentTok{\# radii {-}{-}{-}{-}}
\FunctionTok{radius\_function}\NormalTok{(}
 \AttributeTok{kvadrati\_path =} \StringTok{"./Templates/TemplateGrids/tiles/"}\NormalTok{,}
 \AttributeTok{radii\_path   =} \StringTok{"./Templates/TemplateGridPoints/tiles/"}\NormalTok{,}
 \AttributeTok{tikls100\_path =} \StringTok{"./Templates/TemplateGrids/tikls100\_sauzeme.parquet"}\NormalTok{,}
 \AttributeTok{template\_path =} \StringTok{"./Templates/TemplateRasters/LV100m\_10km.tif"}\NormalTok{,}
 \AttributeTok{input\_layers  =} \FunctionTok{c}\NormalTok{(}\StringTok{"./RasterGrids\_100m/2024/RAW/ForestsTrees\_Mixed\_cell.tif"}\NormalTok{),}
 \AttributeTok{layer\_prefixes =} \FunctionTok{c}\NormalTok{(}\StringTok{"ForestsTrees\_Mixed"}\NormalTok{),}
 \AttributeTok{output\_dir   =} \StringTok{"./RasterGrids\_100m/2024/RAW/"}\NormalTok{,}
 \AttributeTok{n\_workers   =} \DecValTok{6}\NormalTok{,}
 \AttributeTok{radii     =} \FunctionTok{c}\NormalTok{(}\StringTok{"r1250"}\NormalTok{),}
 \AttributeTok{radius\_mode  =} \StringTok{"sparse"}\NormalTok{,}
 \AttributeTok{extract\_fun  =} \StringTok{"mean"}\NormalTok{,}
 \AttributeTok{fill\_missing  =} \ConstantTok{TRUE}\NormalTok{,}
 \AttributeTok{IDW\_weight   =} \DecValTok{2}\NormalTok{,}
 \AttributeTok{future\_max\_size =} \DecValTok{40} \SpecialCharTok{*} \DecValTok{1024}\SpecialCharTok{\^{}}\DecValTok{3}\NormalTok{)}


\CommentTok{\# ForestsTrees\_Mixed\_r1250.tif  egv\_395}
\NormalTok{slanis}\OtherTok{=}\FunctionTok{rast}\NormalTok{(}\StringTok{"./RasterGrids\_100m/2024/RAW/ForestsTrees\_Mixed\_r1250.tif"}\NormalTok{)}
\FunctionTok{names}\NormalTok{(slanis)}\OtherTok{=}\StringTok{"egv\_395"}
\NormalTok{slanis2}\OtherTok{=}\FunctionTok{project}\NormalTok{(slanis,template100)}
\FunctionTok{writeRaster}\NormalTok{(slanis2,}
      \StringTok{"./RasterGrids\_100m/2024/RAW/ForestsTrees\_Mixed\_r1250.tif"}\NormalTok{,}
      \AttributeTok{overwrite=}\ConstantTok{TRUE}\NormalTok{)}

\CommentTok{\# standardisation {-}{-}{-}{-}}
\ControlFlowTok{if}\NormalTok{(}\SpecialCharTok{!}\FunctionTok{require}\NormalTok{(terra)) \{}\FunctionTok{install.packages}\NormalTok{(}\StringTok{"terra"}\NormalTok{); }\FunctionTok{require}\NormalTok{(terra)\}}
\ControlFlowTok{if}\NormalTok{(}\SpecialCharTok{!}\FunctionTok{require}\NormalTok{(tidyverse)) \{}\FunctionTok{install.packages}\NormalTok{(}\StringTok{"tidyverse"}\NormalTok{); }\FunctionTok{require}\NormalTok{(tidyverse)\}}

\NormalTok{nosaukums}\OtherTok{=}\StringTok{"ForestsTrees\_Mixed\_r1250.tif"}
\NormalTok{ielasisanas\_cels}\OtherTok{=}\FunctionTok{paste0}\NormalTok{(}\StringTok{"./RasterGrids\_100m/2024/RAW/"}\NormalTok{,nosaukums)}
\NormalTok{saglabasanas\_cels}\OtherTok{=}\FunctionTok{paste0}\NormalTok{(}\StringTok{"./RasterGrids\_100m/2024/Scaled/"}\NormalTok{,nosaukums)}
\NormalTok{slanis}\OtherTok{=}\FunctionTok{rast}\NormalTok{(ielasisanas\_cels)}
\NormalTok{videjais}\OtherTok{=}\FunctionTok{global}\NormalTok{(slanis,}\AttributeTok{fun=}\StringTok{"mean"}\NormalTok{,}\AttributeTok{na.rm=}\ConstantTok{TRUE}\NormalTok{)}
\NormalTok{centrets}\OtherTok{=}\NormalTok{slanis}\SpecialCharTok{{-}}\NormalTok{videjais[,}\DecValTok{1}\NormalTok{]}
\NormalTok{standartnovirze}\OtherTok{=}\NormalTok{terra}\SpecialCharTok{::}\FunctionTok{global}\NormalTok{(centrets,}\AttributeTok{fun=}\StringTok{"rms"}\NormalTok{,}\AttributeTok{na.rm=}\ConstantTok{TRUE}\NormalTok{)}
\NormalTok{merogots}\OtherTok{=}\NormalTok{centrets}\SpecialCharTok{/}\NormalTok{standartnovirze[,}\DecValTok{1}\NormalTok{]}
\FunctionTok{writeRaster}\NormalTok{(merogots,}
      \AttributeTok{filename=}\NormalTok{saglabasanas\_cels,}
      \AttributeTok{overwrite=}\ConstantTok{TRUE}\NormalTok{)}
\end{Highlighting}
\end{Shaded}

\section{ForestsTrees\_Mixed\_r3000}\label{ch06.396}

\textbf{filename:} \texttt{ForestsTrees\_Mixed\_r3000.tif}

\textbf{layername:} \texttt{egv\_396}

\textbf{English name:} Fractional cover of Mixed Forests within the 3 km landscape

\textbf{Latvian name:} Jauktu koku mežu platības īpatsvars 3 km ainavā

\textbf{Procedure:} The cover fraction within a radius of 3000 m around the analysis grid cell
is calculated as the area-weighted sum of the \hyperref[ch06.393]{analysis cells} inside
the buffer, using the workflow \texttt{egvtools::radius\_function()}. During the calculation of the landscape
metric, inverse distance weighted (power = 2) gap filling on the output is
applied to ensure no missing values at the edges. Then the layer is
rewritten to set its name. Finally, the layer is standardised by
subtracting the arithmetic mean and dividing by the root mean squared error.

\begin{Shaded}
\begin{Highlighting}[]
\CommentTok{\# libs {-}{-}{-}{-}}
\ControlFlowTok{if}\NormalTok{(}\SpecialCharTok{!}\FunctionTok{require}\NormalTok{(terra)) \{}\FunctionTok{install.packages}\NormalTok{(}\StringTok{"terra"}\NormalTok{); }\FunctionTok{require}\NormalTok{(terra)\}}
\ControlFlowTok{if}\NormalTok{(}\SpecialCharTok{!}\FunctionTok{require}\NormalTok{(egvtools)) \{remotes}\SpecialCharTok{::}\FunctionTok{install\_github}\NormalTok{(}\StringTok{"aavotins/egvtools"}\NormalTok{); }\FunctionTok{require}\NormalTok{(egvtools)\}}


\CommentTok{\# Templates {-}{-}{-}{-}{-}}
\NormalTok{template100}\OtherTok{=}\FunctionTok{rast}\NormalTok{(}\StringTok{"./Templates/TemplateRasters/LV100m\_10km.tif"}\NormalTok{)}

\CommentTok{\# radii {-}{-}{-}{-}}
\FunctionTok{radius\_function}\NormalTok{(}
 \AttributeTok{kvadrati\_path =} \StringTok{"./Templates/TemplateGrids/tiles/"}\NormalTok{,}
 \AttributeTok{radii\_path   =} \StringTok{"./Templates/TemplateGridPoints/tiles/"}\NormalTok{,}
 \AttributeTok{tikls100\_path =} \StringTok{"./Templates/TemplateGrids/tikls100\_sauzeme.parquet"}\NormalTok{,}
 \AttributeTok{template\_path =} \StringTok{"./Templates/TemplateRasters/LV100m\_10km.tif"}\NormalTok{,}
 \AttributeTok{input\_layers  =} \FunctionTok{c}\NormalTok{(}\StringTok{"./RasterGrids\_100m/2024/RAW/ForestsTrees\_Mixed\_cell.tif"}\NormalTok{),}
 \AttributeTok{layer\_prefixes =} \FunctionTok{c}\NormalTok{(}\StringTok{"ForestsTrees\_Mixed"}\NormalTok{),}
 \AttributeTok{output\_dir   =} \StringTok{"./RasterGrids\_100m/2024/RAW/"}\NormalTok{,}
 \AttributeTok{n\_workers   =} \DecValTok{6}\NormalTok{,}
 \AttributeTok{radii     =} \FunctionTok{c}\NormalTok{(}\StringTok{"r3000"}\NormalTok{),}
 \AttributeTok{radius\_mode  =} \StringTok{"sparse"}\NormalTok{,}
 \AttributeTok{extract\_fun  =} \StringTok{"mean"}\NormalTok{,}
 \AttributeTok{fill\_missing  =} \ConstantTok{TRUE}\NormalTok{,}
 \AttributeTok{IDW\_weight   =} \DecValTok{2}\NormalTok{,}
 \AttributeTok{future\_max\_size =} \DecValTok{40} \SpecialCharTok{*} \DecValTok{1024}\SpecialCharTok{\^{}}\DecValTok{3}\NormalTok{)}


\CommentTok{\# ForestsTrees\_Mixed\_r3000.tif  egv\_396}
\NormalTok{slanis}\OtherTok{=}\FunctionTok{rast}\NormalTok{(}\StringTok{"./RasterGrids\_100m/2024/RAW/ForestsTrees\_Mixed\_r3000.tif"}\NormalTok{)}
\FunctionTok{names}\NormalTok{(slanis)}\OtherTok{=}\StringTok{"egv\_396"}
\NormalTok{slanis2}\OtherTok{=}\FunctionTok{project}\NormalTok{(slanis,template100)}
\FunctionTok{writeRaster}\NormalTok{(slanis2,}
      \StringTok{"./RasterGrids\_100m/2024/RAW/ForestsTrees\_Mixed\_r3000.tif"}\NormalTok{,}
      \AttributeTok{overwrite=}\ConstantTok{TRUE}\NormalTok{)}

\CommentTok{\# standardisation {-}{-}{-}{-}}
\ControlFlowTok{if}\NormalTok{(}\SpecialCharTok{!}\FunctionTok{require}\NormalTok{(terra)) \{}\FunctionTok{install.packages}\NormalTok{(}\StringTok{"terra"}\NormalTok{); }\FunctionTok{require}\NormalTok{(terra)\}}
\ControlFlowTok{if}\NormalTok{(}\SpecialCharTok{!}\FunctionTok{require}\NormalTok{(tidyverse)) \{}\FunctionTok{install.packages}\NormalTok{(}\StringTok{"tidyverse"}\NormalTok{); }\FunctionTok{require}\NormalTok{(tidyverse)\}}

\NormalTok{nosaukums}\OtherTok{=}\StringTok{"ForestsTrees\_Mixed\_r3000.tif"}
\NormalTok{ielasisanas\_cels}\OtherTok{=}\FunctionTok{paste0}\NormalTok{(}\StringTok{"./RasterGrids\_100m/2024/RAW/"}\NormalTok{,nosaukums)}
\NormalTok{saglabasanas\_cels}\OtherTok{=}\FunctionTok{paste0}\NormalTok{(}\StringTok{"./RasterGrids\_100m/2024/Scaled/"}\NormalTok{,nosaukums)}
\NormalTok{slanis}\OtherTok{=}\FunctionTok{rast}\NormalTok{(ielasisanas\_cels)}
\NormalTok{videjais}\OtherTok{=}\FunctionTok{global}\NormalTok{(slanis,}\AttributeTok{fun=}\StringTok{"mean"}\NormalTok{,}\AttributeTok{na.rm=}\ConstantTok{TRUE}\NormalTok{)}
\NormalTok{centrets}\OtherTok{=}\NormalTok{slanis}\SpecialCharTok{{-}}\NormalTok{videjais[,}\DecValTok{1}\NormalTok{]}
\NormalTok{standartnovirze}\OtherTok{=}\NormalTok{terra}\SpecialCharTok{::}\FunctionTok{global}\NormalTok{(centrets,}\AttributeTok{fun=}\StringTok{"rms"}\NormalTok{,}\AttributeTok{na.rm=}\ConstantTok{TRUE}\NormalTok{)}
\NormalTok{merogots}\OtherTok{=}\NormalTok{centrets}\SpecialCharTok{/}\NormalTok{standartnovirze[,}\DecValTok{1}\NormalTok{]}
\FunctionTok{writeRaster}\NormalTok{(merogots,}
      \AttributeTok{filename=}\NormalTok{saglabasanas\_cels,}
      \AttributeTok{overwrite=}\ConstantTok{TRUE}\NormalTok{)}
\end{Highlighting}
\end{Shaded}

\section{ForestsTrees\_Mixed\_r10000}\label{ch06.397}

\textbf{filename:} \texttt{ForestsTrees\_Mixed\_r10000.tif}

\textbf{layername:} \texttt{egv\_397}

\textbf{English name:} Fractional cover of Mixed Forests within the 10 km landscape

\textbf{Latvian name:} Jauktu koku mežu platības īpatsvars 10 km ainavā

\textbf{Procedure:} The cover fraction within a radius of 10000 m around the analysis grid cell
is calculated as the area-weighted sum of the \hyperref[ch06.393]{analysis cells} inside
the buffer, using the workflow \texttt{egvtools::radius\_function()}. During the calculation of the landscape
metric, inverse distance weighted (power = 2) gap filling on the output is
applied to ensure no missing values at the edges. Then the layer is
rewritten to set its name. Finally, the layer is standardised by
subtracting the arithmetic mean and dividing by the root mean squared error.

\begin{Shaded}
\begin{Highlighting}[]
\CommentTok{\# libs {-}{-}{-}{-}}
\ControlFlowTok{if}\NormalTok{(}\SpecialCharTok{!}\FunctionTok{require}\NormalTok{(terra)) \{}\FunctionTok{install.packages}\NormalTok{(}\StringTok{"terra"}\NormalTok{); }\FunctionTok{require}\NormalTok{(terra)\}}
\ControlFlowTok{if}\NormalTok{(}\SpecialCharTok{!}\FunctionTok{require}\NormalTok{(egvtools)) \{remotes}\SpecialCharTok{::}\FunctionTok{install\_github}\NormalTok{(}\StringTok{"aavotins/egvtools"}\NormalTok{); }\FunctionTok{require}\NormalTok{(egvtools)\}}


\CommentTok{\# Templates {-}{-}{-}{-}{-}}
\NormalTok{template100}\OtherTok{=}\FunctionTok{rast}\NormalTok{(}\StringTok{"./Templates/TemplateRasters/LV100m\_10km.tif"}\NormalTok{)}

\CommentTok{\# radii {-}{-}{-}{-}}
\FunctionTok{radius\_function}\NormalTok{(}
 \AttributeTok{kvadrati\_path =} \StringTok{"./Templates/TemplateGrids/tiles/"}\NormalTok{,}
 \AttributeTok{radii\_path   =} \StringTok{"./Templates/TemplateGridPoints/tiles/"}\NormalTok{,}
 \AttributeTok{tikls100\_path =} \StringTok{"./Templates/TemplateGrids/tikls100\_sauzeme.parquet"}\NormalTok{,}
 \AttributeTok{template\_path =} \StringTok{"./Templates/TemplateRasters/LV100m\_10km.tif"}\NormalTok{,}
 \AttributeTok{input\_layers  =} \FunctionTok{c}\NormalTok{(}\StringTok{"./RasterGrids\_100m/2024/RAW/ForestsTrees\_Mixed\_cell.tif"}\NormalTok{),}
 \AttributeTok{layer\_prefixes =} \FunctionTok{c}\NormalTok{(}\StringTok{"ForestsTrees\_Mixed"}\NormalTok{),}
 \AttributeTok{output\_dir   =} \StringTok{"./RasterGrids\_100m/2024/RAW/"}\NormalTok{,}
 \AttributeTok{n\_workers   =} \DecValTok{6}\NormalTok{,}
 \AttributeTok{radii     =} \FunctionTok{c}\NormalTok{(}\StringTok{"r10000"}\NormalTok{),}
 \AttributeTok{radius\_mode  =} \StringTok{"sparse"}\NormalTok{,}
 \AttributeTok{extract\_fun  =} \StringTok{"mean"}\NormalTok{,}
 \AttributeTok{fill\_missing  =} \ConstantTok{TRUE}\NormalTok{,}
 \AttributeTok{IDW\_weight   =} \DecValTok{2}\NormalTok{,}
 \AttributeTok{future\_max\_size =} \DecValTok{40} \SpecialCharTok{*} \DecValTok{1024}\SpecialCharTok{\^{}}\DecValTok{3}\NormalTok{)}


\CommentTok{\# ForestsTrees\_Mixed\_r10000.tif egv\_397}
\NormalTok{slanis}\OtherTok{=}\FunctionTok{rast}\NormalTok{(}\StringTok{"./RasterGrids\_100m/2024/RAW/ForestsTrees\_Mixed\_r10000.tif"}\NormalTok{)}
\FunctionTok{names}\NormalTok{(slanis)}\OtherTok{=}\StringTok{"egv\_397"}
\NormalTok{slanis2}\OtherTok{=}\FunctionTok{project}\NormalTok{(slanis,template100)}
\FunctionTok{writeRaster}\NormalTok{(slanis2,}
      \StringTok{"./RasterGrids\_100m/2024/RAW/ForestsTrees\_Mixed\_r10000.tif"}\NormalTok{,}
      \AttributeTok{overwrite=}\ConstantTok{TRUE}\NormalTok{)}

\CommentTok{\# standardisation {-}{-}{-}{-}}
\ControlFlowTok{if}\NormalTok{(}\SpecialCharTok{!}\FunctionTok{require}\NormalTok{(terra)) \{}\FunctionTok{install.packages}\NormalTok{(}\StringTok{"terra"}\NormalTok{); }\FunctionTok{require}\NormalTok{(terra)\}}
\ControlFlowTok{if}\NormalTok{(}\SpecialCharTok{!}\FunctionTok{require}\NormalTok{(tidyverse)) \{}\FunctionTok{install.packages}\NormalTok{(}\StringTok{"tidyverse"}\NormalTok{); }\FunctionTok{require}\NormalTok{(tidyverse)\}}

\NormalTok{nosaukums}\OtherTok{=}\StringTok{"ForestsTrees\_Mixed\_r10000.tif"}
\NormalTok{ielasisanas\_cels}\OtherTok{=}\FunctionTok{paste0}\NormalTok{(}\StringTok{"./RasterGrids\_100m/2024/RAW/"}\NormalTok{,nosaukums)}
\NormalTok{saglabasanas\_cels}\OtherTok{=}\FunctionTok{paste0}\NormalTok{(}\StringTok{"./RasterGrids\_100m/2024/Scaled/"}\NormalTok{,nosaukums)}
\NormalTok{slanis}\OtherTok{=}\FunctionTok{rast}\NormalTok{(ielasisanas\_cels)}
\NormalTok{videjais}\OtherTok{=}\FunctionTok{global}\NormalTok{(slanis,}\AttributeTok{fun=}\StringTok{"mean"}\NormalTok{,}\AttributeTok{na.rm=}\ConstantTok{TRUE}\NormalTok{)}
\NormalTok{centrets}\OtherTok{=}\NormalTok{slanis}\SpecialCharTok{{-}}\NormalTok{videjais[,}\DecValTok{1}\NormalTok{]}
\NormalTok{standartnovirze}\OtherTok{=}\NormalTok{terra}\SpecialCharTok{::}\FunctionTok{global}\NormalTok{(centrets,}\AttributeTok{fun=}\StringTok{"rms"}\NormalTok{,}\AttributeTok{na.rm=}\ConstantTok{TRUE}\NormalTok{)}
\NormalTok{merogots}\OtherTok{=}\NormalTok{centrets}\SpecialCharTok{/}\NormalTok{standartnovirze[,}\DecValTok{1}\NormalTok{]}
\FunctionTok{writeRaster}\NormalTok{(merogots,}
      \AttributeTok{filename=}\NormalTok{saglabasanas\_cels,}
      \AttributeTok{overwrite=}\ConstantTok{TRUE}\NormalTok{)}
\end{Highlighting}
\end{Shaded}

\section{ForestsTrees\_TemperateDeciduous\_cell}\label{ch06.398}

\textbf{filename:} \texttt{ForestsTrees\_TemperateDeciduous\_cell.tif}

\textbf{layername:} \texttt{egv\_398}

\textbf{English name:} Fractional cover of Temperate Deciduous Forests within the
analysis cell (1 ha)

\textbf{Latvian name:} Platlapju mežu platības īpatsvars analīzes šūnā (1 ha)

\textbf{Procedure:} Most EGVs describing forests are spatially restricted to areas outside
of clearcuts and dead stands. This mask is created using a combination of
the \hyperref[Ch04.01]{State Forest Service's
State Forest Registry} land category 12 and 14, and \hyperref[Ch04.09]{The
Global Forest Watch} pixels classified as lost tree canopy cover since
2020 (raster layer matching input, presence = 1, absence = 0).

To prepare this EGV, stands from the \hyperref[Ch04.01]{State Forest Service's State Forest
Registry} are classified into (in order):

\begin{itemize}
\item
  coniferous (see \hyperref[Ch01]{Terminology and acronyms} for species codes) if
  timber volume of those species exceeded 75\%;
\item
  Boreal deciduous if timber volume of those species exceeded 75\%;
\item
  temperate deciduous if timber volume of those species exceeded 50\%;
\item
  mixed otherwise;
\end{itemize}

then temperate deciduous stands are selected and geometries are
rasterised (presence = 1, NA otherwise). Rasterisation is
performed using the workflow \texttt{egvtools::polygon2input()}, restricting to pixels outside clearcut
mask and covering background with value 0. The resulting layer
is then aggregated to EGV resolution using the workflow \texttt{egvtools::input2egv()}, which
calculates the arithmetic mean to determine the cover fraction. During
aggregation, inverse distance weighted (power = 2) gap filling on the output is
applied to ensure no missing values at the edges. Finally, the layer is
standardised by subtracting the arithmetic mean and dividing by the root mean squared
error.

\begin{Shaded}
\begin{Highlighting}[]
\CommentTok{\# libs {-}{-}{-}{-}}
\ControlFlowTok{if}\NormalTok{(}\SpecialCharTok{!}\FunctionTok{require}\NormalTok{(egvtools)) \{remotes}\SpecialCharTok{::}\FunctionTok{install\_github}\NormalTok{(}\StringTok{"aavotins/egvtools"}\NormalTok{); }\FunctionTok{require}\NormalTok{(egvtools)\}}
\ControlFlowTok{if}\NormalTok{(}\SpecialCharTok{!}\FunctionTok{require}\NormalTok{(terra)) \{}\FunctionTok{install.packages}\NormalTok{(}\StringTok{"terra"}\NormalTok{); }\FunctionTok{require}\NormalTok{(terra)\}}
\ControlFlowTok{if}\NormalTok{(}\SpecialCharTok{!}\FunctionTok{require}\NormalTok{(sf)) \{}\FunctionTok{install.packages}\NormalTok{(}\StringTok{"sf"}\NormalTok{); }\FunctionTok{require}\NormalTok{(sf)\}}
\ControlFlowTok{if}\NormalTok{(}\SpecialCharTok{!}\FunctionTok{require}\NormalTok{(tidyverse)) \{}\FunctionTok{install.packages}\NormalTok{(}\StringTok{"tidyverse"}\NormalTok{); }\FunctionTok{require}\NormalTok{(tidyverse)\}}
\ControlFlowTok{if}\NormalTok{(}\SpecialCharTok{!}\FunctionTok{require}\NormalTok{(sfarrow)) \{}\FunctionTok{install.packages}\NormalTok{(}\StringTok{"sfarrow"}\NormalTok{); }\FunctionTok{require}\NormalTok{(sfarrow)\}}
\ControlFlowTok{if}\NormalTok{(}\SpecialCharTok{!}\FunctionTok{require}\NormalTok{(readxl)) \{}\FunctionTok{install.packages}\NormalTok{(}\StringTok{"readxl"}\NormalTok{); }\FunctionTok{require}\NormalTok{(readxl)\}}
\ControlFlowTok{if}\NormalTok{(}\SpecialCharTok{!}\FunctionTok{require}\NormalTok{(raster)) \{}\FunctionTok{install.packages}\NormalTok{(}\StringTok{"raster"}\NormalTok{); }\FunctionTok{require}\NormalTok{(raster)\}}
\ControlFlowTok{if}\NormalTok{(}\SpecialCharTok{!}\FunctionTok{require}\NormalTok{(fasterize)) \{}\FunctionTok{install.packages}\NormalTok{(}\StringTok{"fasterize"}\NormalTok{); }\FunctionTok{require}\NormalTok{(fasterize)\}}

\CommentTok{\# templates {-}{-}{-}{-}}
\NormalTok{template100}\OtherTok{=}\FunctionTok{rast}\NormalTok{(}\StringTok{"./Templates/TemplateRasters/LV100m\_10km.tif"}\NormalTok{)}
\NormalTok{template10}\OtherTok{=}\FunctionTok{rast}\NormalTok{(}\StringTok{"./Templates/TemplateRasters/LV10m\_10km.tif"}\NormalTok{)}
\NormalTok{rastrs10}\OtherTok{=}\FunctionTok{raster}\NormalTok{(template10)}

\NormalTok{nulls10}\OtherTok{=}\FunctionTok{rast}\NormalTok{(}\StringTok{"./Templates/TemplateRasters/nulls\_LV10m\_10km.tif"}\NormalTok{)}
\NormalTok{nulls100}\OtherTok{=}\FunctionTok{rast}\NormalTok{(}\StringTok{"./Templates/TemplateRasters/nulls\_LV100m\_10km.tif"}\NormalTok{)}


\CommentTok{\# simple landscape {-}{-}{-}{-}}
\NormalTok{simple\_landscape}\OtherTok{=}\FunctionTok{rast}\NormalTok{(}\StringTok{"RasterGrids\_10m/2024/Ainava\_vienk\_mask.tif"}\NormalTok{)}

\CommentTok{\# mvr {-}{-}{-}{-}}
\NormalTok{mvr}\OtherTok{=}\FunctionTok{st\_read\_parquet}\NormalTok{(}\StringTok{"./Geodata/2024/MVR/nogabali\_2024janv.parquet"}\NormalTok{)}
\NormalTok{mvr}\SpecialCharTok{$}\NormalTok{yes}\OtherTok{=}\DecValTok{1}

\CommentTok{\# clear cut mask {-}{-}{-}{-}}
\NormalTok{izcirtumi}\OtherTok{=}\NormalTok{mvr }\SpecialCharTok{\%\textgreater{}\%} 
 \FunctionTok{filter}\NormalTok{(zkat }\SpecialCharTok{\%in\%} \FunctionTok{c}\NormalTok{(}\StringTok{"12"}\NormalTok{,}\StringTok{"14"}\NormalTok{)) }\SpecialCharTok{\%\textgreater{}\%} 
\NormalTok{ dplyr}\SpecialCharTok{::}\FunctionTok{select}\NormalTok{(yes)}
\NormalTok{r\_izcirtumi\_mvr}\OtherTok{=}\FunctionTok{fasterize}\NormalTok{(izcirtumi,rastrs10,}\AttributeTok{field=}\StringTok{"yes"}\NormalTok{)}
\NormalTok{t\_izcirtumi\_mvr}\OtherTok{=}\FunctionTok{rast}\NormalTok{(r\_izcirtumi\_mvr)}
\FunctionTok{plot}\NormalTok{(t\_izcirtumi\_mvr)}

\NormalTok{tcl}\OtherTok{=}\FunctionTok{rast}\NormalTok{(}\StringTok{"./Geodata/2024/Trees/GFW/TreeCoverLoss\_v1\_12.tif"}\NormalTok{)}
\NormalTok{tcl2}\OtherTok{=}\FunctionTok{ifel}\NormalTok{(tcl}\SpecialCharTok{\textless{}}\DecValTok{20}\NormalTok{,}\DecValTok{0}\NormalTok{,}\DecValTok{1}\NormalTok{)}
\NormalTok{tclX}\OtherTok{=}\FunctionTok{cover}\NormalTok{(tcl2,nulls10)}
\FunctionTok{plot}\NormalTok{(tclX)}

\NormalTok{clearcut\_mask}\OtherTok{=}\FunctionTok{cover}\NormalTok{(t\_izcirtumi\_mvr,tclX,}
          \AttributeTok{filename=}\StringTok{"./RasterGrids\_10m/2024/Mask\_clearcuts.tif"}\NormalTok{,}
          \AttributeTok{overwrite=}\ConstantTok{TRUE}\NormalTok{)}
\FunctionTok{plot}\NormalTok{(clearcut\_mask)}

\FunctionTok{rm}\NormalTok{(izcirtumi)}
\FunctionTok{rm}\NormalTok{(r\_izcirtumi\_mvr)}
\FunctionTok{rm}\NormalTok{(t\_izcirtumi\_mvr)}
\FunctionTok{rm}\NormalTok{(tcl)}
\FunctionTok{rm}\NormalTok{(tcl2)}
\FunctionTok{rm}\NormalTok{(tclX)}

\CommentTok{\# ForestsTrees\_TemperateDeciduous\_cell.tif  egv\_398 {-}{-}{-}{-}}
\NormalTok{skujkoki}\OtherTok{=}\FunctionTok{c}\NormalTok{(}\StringTok{"1"}\NormalTok{,}\StringTok{"3"}\NormalTok{,}\StringTok{"13"}\NormalTok{,}\StringTok{"14"}\NormalTok{,}\StringTok{"15"}\NormalTok{,}\StringTok{"22"}\NormalTok{,}\StringTok{"23"}\NormalTok{,}\StringTok{"28"}\NormalTok{) }\CommentTok{\# 8}
\NormalTok{saurlapji}\OtherTok{=}\FunctionTok{c}\NormalTok{(}\StringTok{"4"}\NormalTok{,}\StringTok{"6"}\NormalTok{,}\StringTok{"8"}\NormalTok{,}\StringTok{"9"}\NormalTok{,}\StringTok{"19"}\NormalTok{,}\StringTok{"20"}\NormalTok{,}\StringTok{"21"}\NormalTok{,}\StringTok{"32"}\NormalTok{,}\StringTok{"35"}\NormalTok{,}\StringTok{"68"}\NormalTok{) }\CommentTok{\# 10}
\NormalTok{platlapji}\OtherTok{=}\FunctionTok{c}\NormalTok{(}\StringTok{"10"}\NormalTok{,}\StringTok{"11"}\NormalTok{,}\StringTok{"12"}\NormalTok{,}\StringTok{"16"}\NormalTok{,}\StringTok{"17"}\NormalTok{,}\StringTok{"18"}\NormalTok{,}\StringTok{"24"}\NormalTok{,}\StringTok{"25"}\NormalTok{,}\StringTok{"26"}\NormalTok{,}\StringTok{"27"}\NormalTok{,}\StringTok{"28"}\NormalTok{,}\StringTok{"29"}\NormalTok{,}\StringTok{"50"}\NormalTok{,}
      \StringTok{"61"}\NormalTok{,}\StringTok{"62"}\NormalTok{,}\StringTok{"63"}\NormalTok{,}\StringTok{"64"}\NormalTok{,}\StringTok{"65"}\NormalTok{,}\StringTok{"66"}\NormalTok{,}\StringTok{"67"}\NormalTok{,}\StringTok{"69"}\NormalTok{) }\CommentTok{\# 21}
\NormalTok{mvr}\OtherTok{=}\NormalTok{mvr }\SpecialCharTok{\%\textgreater{}\%} 
 \FunctionTok{mutate}\NormalTok{(}\AttributeTok{kraja\_skujkoku=}\FunctionTok{ifelse}\NormalTok{(s10 }\SpecialCharTok{\%in\%}\NormalTok{ skujkoki,v10,}\DecValTok{0}\NormalTok{)}\SpecialCharTok{+}
      \FunctionTok{ifelse}\NormalTok{(s11 }\SpecialCharTok{\%in\%}\NormalTok{ skujkoki,v11,}\DecValTok{0}\NormalTok{)}\SpecialCharTok{+}\FunctionTok{ifelse}\NormalTok{(s12 }\SpecialCharTok{\%in\%}\NormalTok{ skujkoki,v12,}\DecValTok{0}\NormalTok{)}\SpecialCharTok{+}
      \FunctionTok{ifelse}\NormalTok{(s13 }\SpecialCharTok{\%in\%}\NormalTok{ skujkoki,v13,}\DecValTok{0}\NormalTok{)}\SpecialCharTok{+}\FunctionTok{ifelse}\NormalTok{(s14 }\SpecialCharTok{\%in\%}\NormalTok{ skujkoki,v14,}\DecValTok{0}\NormalTok{),}
     \AttributeTok{kraja\_saurlapju=}\FunctionTok{ifelse}\NormalTok{(s10 }\SpecialCharTok{\%in\%}\NormalTok{ saurlapji,v10,}\DecValTok{0}\NormalTok{)}\SpecialCharTok{+}
      \FunctionTok{ifelse}\NormalTok{(s11 }\SpecialCharTok{\%in\%}\NormalTok{ saurlapji,v11,}\DecValTok{0}\NormalTok{)}\SpecialCharTok{+}\FunctionTok{ifelse}\NormalTok{(s12 }\SpecialCharTok{\%in\%}\NormalTok{ saurlapji,v12,}\DecValTok{0}\NormalTok{)}\SpecialCharTok{+}
      \FunctionTok{ifelse}\NormalTok{(s13 }\SpecialCharTok{\%in\%}\NormalTok{ saurlapji,v13,}\DecValTok{0}\NormalTok{)}\SpecialCharTok{+}\FunctionTok{ifelse}\NormalTok{(s14 }\SpecialCharTok{\%in\%}\NormalTok{ saurlapji,v14,}\DecValTok{0}\NormalTok{),}
     \AttributeTok{kraja\_platlapju=}\FunctionTok{ifelse}\NormalTok{(s10 }\SpecialCharTok{\%in\%}\NormalTok{ platlapji,v10,}\DecValTok{0}\NormalTok{)}\SpecialCharTok{+}
      \FunctionTok{ifelse}\NormalTok{(s11 }\SpecialCharTok{\%in\%}\NormalTok{ platlapji,v11,}\DecValTok{0}\NormalTok{)}\SpecialCharTok{+}\FunctionTok{ifelse}\NormalTok{(s12 }\SpecialCharTok{\%in\%}\NormalTok{ platlapji,v12,}\DecValTok{0}\NormalTok{)}\SpecialCharTok{+}
      \FunctionTok{ifelse}\NormalTok{(s13 }\SpecialCharTok{\%in\%}\NormalTok{ platlapji,v13,}\DecValTok{0}\NormalTok{)}\SpecialCharTok{+}\FunctionTok{ifelse}\NormalTok{(s14 }\SpecialCharTok{\%in\%}\NormalTok{ platlapji,v14,}\DecValTok{0}\NormalTok{)) }\SpecialCharTok{\%\textgreater{}\%} 
 \FunctionTok{mutate}\NormalTok{(}\AttributeTok{kopeja\_kraja=}\NormalTok{kraja\_skujkoku}\SpecialCharTok{+}\NormalTok{kraja\_platlapju}\SpecialCharTok{+}\NormalTok{kraja\_saurlapju) }\SpecialCharTok{\%\textgreater{}\%} 
 \FunctionTok{mutate}\NormalTok{(}\AttributeTok{tips=}\FunctionTok{ifelse}\NormalTok{(kraja\_skujkoku}\SpecialCharTok{/}\NormalTok{kopeja\_kraja}\SpecialCharTok{\textgreater{}=}\FloatTok{0.75}\NormalTok{,}\StringTok{"skujkoku"}\NormalTok{,}
           \FunctionTok{ifelse}\NormalTok{(kraja\_saurlapju}\SpecialCharTok{/}\NormalTok{kopeja\_kraja}\SpecialCharTok{\textgreater{}=}\FloatTok{0.75}\NormalTok{,}\StringTok{"saurlapju"}\NormalTok{,}
              \FunctionTok{ifelse}\NormalTok{(kraja\_platlapju}\SpecialCharTok{/}\NormalTok{kopeja\_kraja}\SpecialCharTok{\textgreater{}}\FloatTok{0.5}\NormalTok{,}\StringTok{"platlapju"}\NormalTok{,}
                  \StringTok{"jauktu koku"}\NormalTok{))))}
\NormalTok{nogabali}\OtherTok{=}\NormalTok{mvr }\SpecialCharTok{\%\textgreater{}\%} 
 \FunctionTok{filter}\NormalTok{(zkat}\SpecialCharTok{==}\StringTok{"10"}\SpecialCharTok{\&}\NormalTok{tips}\SpecialCharTok{==}\StringTok{"platlapju"}\NormalTok{)}

\NormalTok{p2i\_rez}\OtherTok{=}\NormalTok{egvtools}\SpecialCharTok{::}\FunctionTok{polygon2input}\NormalTok{(}\AttributeTok{vector\_data =}\NormalTok{ nogabali,}
                \AttributeTok{template\_path =} \StringTok{"./Templates/TemplateRasters/LV10m\_10km.tif"}\NormalTok{,}
                \AttributeTok{out\_path =} \StringTok{"./RasterGrids\_10m/2024/"}\NormalTok{,}
                \AttributeTok{file\_name =} \StringTok{"ForestsTrees\_TemperateDeciduous\_input.tif"}\NormalTok{,}
                \AttributeTok{value\_field =} \StringTok{"yes"}\NormalTok{,}
                \AttributeTok{restrict\_to =}\NormalTok{ clearcut\_mask,}
                \AttributeTok{restrict\_values =} \DecValTok{0}\NormalTok{,}
                \AttributeTok{prepare=}\ConstantTok{FALSE}\NormalTok{,}
                \AttributeTok{background\_raster =} \StringTok{"./Templates/TemplateRasters/nulls\_LV10m\_10km.tif"}\NormalTok{,}
                \AttributeTok{plot\_result =} \ConstantTok{TRUE}\NormalTok{)}
\NormalTok{p2i\_rez}
\NormalTok{i2e\_rez}\OtherTok{=}\NormalTok{egvtools}\SpecialCharTok{::}\FunctionTok{input2egv}\NormalTok{(}\AttributeTok{input=}\FunctionTok{paste0}\NormalTok{(}\StringTok{"./RasterGrids\_10m/2024/"}\NormalTok{,}
                     \StringTok{"ForestsTrees\_TemperateDeciduous\_input.tif"}\NormalTok{),}
              \AttributeTok{egv\_template=} \StringTok{"./Templates/TemplateRasters/LV100m\_10km.tif"}\NormalTok{,}
              \AttributeTok{summary\_function =} \StringTok{"average"}\NormalTok{,}
              \AttributeTok{missing\_job =} \StringTok{"FillOutput"}\NormalTok{,}
              \AttributeTok{outlocation =} \StringTok{"./RasterGrids\_100m/2024/RAW/"}\NormalTok{,}
              \AttributeTok{outfilename =} \StringTok{"ForestsTrees\_TemperateDeciduous\_cell.tif"}\NormalTok{,}
              \AttributeTok{layername =} \StringTok{"egv\_398"}\NormalTok{,}
              \AttributeTok{idw\_weight =} \DecValTok{2}\NormalTok{,}
              \AttributeTok{plot\_gaps =} \ConstantTok{FALSE}\NormalTok{,}\AttributeTok{plot\_final =} \ConstantTok{TRUE}\NormalTok{)}
\NormalTok{i2e\_rez}
\FunctionTok{rm}\NormalTok{(nogabali)}
\FunctionTok{rm}\NormalTok{(p2i\_rez)}
\FunctionTok{rm}\NormalTok{(i2e\_rez)}
\FunctionTok{unlink}\NormalTok{(}\StringTok{"./RasterGrids\_10m/2024/ForestsTrees\_TemperateDeciduous\_input.tif"}\NormalTok{)}

\CommentTok{\# standardisation {-}{-}{-}{-}}
\ControlFlowTok{if}\NormalTok{(}\SpecialCharTok{!}\FunctionTok{require}\NormalTok{(terra)) \{}\FunctionTok{install.packages}\NormalTok{(}\StringTok{"terra"}\NormalTok{); }\FunctionTok{require}\NormalTok{(terra)\}}
\ControlFlowTok{if}\NormalTok{(}\SpecialCharTok{!}\FunctionTok{require}\NormalTok{(tidyverse)) \{}\FunctionTok{install.packages}\NormalTok{(}\StringTok{"tidyverse"}\NormalTok{); }\FunctionTok{require}\NormalTok{(tidyverse)\}}

\NormalTok{nosaukums}\OtherTok{=}\StringTok{"ForestsTrees\_TemperateDeciduous\_cell.tif"}
\NormalTok{ielasisanas\_cels}\OtherTok{=}\FunctionTok{paste0}\NormalTok{(}\StringTok{"./RasterGrids\_100m/2024/RAW/"}\NormalTok{,nosaukums)}
\NormalTok{saglabasanas\_cels}\OtherTok{=}\FunctionTok{paste0}\NormalTok{(}\StringTok{"./RasterGrids\_100m/2024/Scaled/"}\NormalTok{,nosaukums)}
\NormalTok{slanis}\OtherTok{=}\FunctionTok{rast}\NormalTok{(ielasisanas\_cels)}
\NormalTok{videjais}\OtherTok{=}\FunctionTok{global}\NormalTok{(slanis,}\AttributeTok{fun=}\StringTok{"mean"}\NormalTok{,}\AttributeTok{na.rm=}\ConstantTok{TRUE}\NormalTok{)}
\NormalTok{centrets}\OtherTok{=}\NormalTok{slanis}\SpecialCharTok{{-}}\NormalTok{videjais[,}\DecValTok{1}\NormalTok{]}
\NormalTok{standartnovirze}\OtherTok{=}\NormalTok{terra}\SpecialCharTok{::}\FunctionTok{global}\NormalTok{(centrets,}\AttributeTok{fun=}\StringTok{"rms"}\NormalTok{,}\AttributeTok{na.rm=}\ConstantTok{TRUE}\NormalTok{)}
\NormalTok{merogots}\OtherTok{=}\NormalTok{centrets}\SpecialCharTok{/}\NormalTok{standartnovirze[,}\DecValTok{1}\NormalTok{]}
\FunctionTok{writeRaster}\NormalTok{(merogots,}
      \AttributeTok{filename=}\NormalTok{saglabasanas\_cels,}
      \AttributeTok{overwrite=}\ConstantTok{TRUE}\NormalTok{)}
\end{Highlighting}
\end{Shaded}

\section{ForestsTrees\_TemperateDeciduous\_r500}\label{ch06.399}

\textbf{filename:} \texttt{ForestsTrees\_TemperateDeciduous\_r500.tif}

\textbf{layername:} \texttt{egv\_399}

\textbf{English name:} Fractional cover of Temperate Deciduous Forests within the 0.5
km landscape

\textbf{Latvian name:} Platlapju mežu platības īpatsvars 0,5 km ainavā

\textbf{Procedure:} The cover fraction within a radius of 500 m around the analysis grid cell is
calculated as the area-weighted sum of the \hyperref[ch06.398]{analysis cells} inside the
buffer, using the workflow \texttt{egvtools::radius\_function()}. During the calculation of the landscape metric,
inverse distance weighted (power = 2) gap filling on the output is applied
to ensure no missing values at the edges. Then the layer is rewritten to set
its name. Finally, the layer is standardised by subtracting the arithmetic
mean and dividing by the root mean squared error.

\begin{Shaded}
\begin{Highlighting}[]
\CommentTok{\# libs {-}{-}{-}{-}}
\ControlFlowTok{if}\NormalTok{(}\SpecialCharTok{!}\FunctionTok{require}\NormalTok{(terra)) \{}\FunctionTok{install.packages}\NormalTok{(}\StringTok{"terra"}\NormalTok{); }\FunctionTok{require}\NormalTok{(terra)\}}
\ControlFlowTok{if}\NormalTok{(}\SpecialCharTok{!}\FunctionTok{require}\NormalTok{(egvtools)) \{remotes}\SpecialCharTok{::}\FunctionTok{install\_github}\NormalTok{(}\StringTok{"aavotins/egvtools"}\NormalTok{); }\FunctionTok{require}\NormalTok{(egvtools)\}}


\CommentTok{\# Templates {-}{-}{-}{-}{-}}
\NormalTok{template100}\OtherTok{=}\FunctionTok{rast}\NormalTok{(}\StringTok{"./Templates/TemplateRasters/LV100m\_10km.tif"}\NormalTok{)}

\CommentTok{\# radii {-}{-}{-}{-}}
\FunctionTok{radius\_function}\NormalTok{(}
 \AttributeTok{kvadrati\_path =} \StringTok{"./Templates/TemplateGrids/tiles/"}\NormalTok{,}
 \AttributeTok{radii\_path   =} \StringTok{"./Templates/TemplateGridPoints/tiles/"}\NormalTok{,}
 \AttributeTok{tikls100\_path =} \StringTok{"./Templates/TemplateGrids/tikls100\_sauzeme.parquet"}\NormalTok{,}
 \AttributeTok{template\_path =} \StringTok{"./Templates/TemplateRasters/LV100m\_10km.tif"}\NormalTok{,}
 \AttributeTok{input\_layers  =} \FunctionTok{c}\NormalTok{(}\StringTok{"./RasterGrids\_100m/2024/RAW/ForestsTrees\_TemperateDeciduous\_cell.tif"}\NormalTok{),}
 \AttributeTok{layer\_prefixes =} \FunctionTok{c}\NormalTok{(}\StringTok{"ForestsTrees\_TemperateDeciduous"}\NormalTok{),}
 \AttributeTok{output\_dir   =} \StringTok{"./RasterGrids\_100m/2024/RAW/"}\NormalTok{,}
 \AttributeTok{n\_workers   =} \DecValTok{6}\NormalTok{,}
 \AttributeTok{radii     =} \FunctionTok{c}\NormalTok{(}\StringTok{"r500"}\NormalTok{),}
 \AttributeTok{radius\_mode  =} \StringTok{"sparse"}\NormalTok{,}
 \AttributeTok{extract\_fun  =} \StringTok{"mean"}\NormalTok{,}
 \AttributeTok{fill\_missing  =} \ConstantTok{TRUE}\NormalTok{,}
 \AttributeTok{IDW\_weight   =} \DecValTok{2}\NormalTok{,}
 \AttributeTok{future\_max\_size =} \DecValTok{40} \SpecialCharTok{*} \DecValTok{1024}\SpecialCharTok{\^{}}\DecValTok{3}\NormalTok{)}


\CommentTok{\# ForestsTrees\_TemperateDeciduous\_r500.tif  egv\_399}
\NormalTok{slanis}\OtherTok{=}\FunctionTok{rast}\NormalTok{(}\StringTok{"./RasterGrids\_100m/2024/RAW/ForestsTrees\_TemperateDeciduous\_r500.tif"}\NormalTok{)}
\FunctionTok{names}\NormalTok{(slanis)}\OtherTok{=}\StringTok{"egv\_399"}
\NormalTok{slanis2}\OtherTok{=}\FunctionTok{project}\NormalTok{(slanis,template100)}
\FunctionTok{writeRaster}\NormalTok{(slanis2,}
      \StringTok{"./RasterGrids\_100m/2024/RAW/ForestsTrees\_TemperateDeciduous\_r500.tif"}\NormalTok{,}
      \AttributeTok{overwrite=}\ConstantTok{TRUE}\NormalTok{)}

\CommentTok{\# standardisation {-}{-}{-}{-}}
\ControlFlowTok{if}\NormalTok{(}\SpecialCharTok{!}\FunctionTok{require}\NormalTok{(terra)) \{}\FunctionTok{install.packages}\NormalTok{(}\StringTok{"terra"}\NormalTok{); }\FunctionTok{require}\NormalTok{(terra)\}}
\ControlFlowTok{if}\NormalTok{(}\SpecialCharTok{!}\FunctionTok{require}\NormalTok{(tidyverse)) \{}\FunctionTok{install.packages}\NormalTok{(}\StringTok{"tidyverse"}\NormalTok{); }\FunctionTok{require}\NormalTok{(tidyverse)\}}

\NormalTok{nosaukums}\OtherTok{=}\StringTok{"ForestsTrees\_TemperateDeciduous\_r500.tif"}
\NormalTok{ielasisanas\_cels}\OtherTok{=}\FunctionTok{paste0}\NormalTok{(}\StringTok{"./RasterGrids\_100m/2024/RAW/"}\NormalTok{,nosaukums)}
\NormalTok{saglabasanas\_cels}\OtherTok{=}\FunctionTok{paste0}\NormalTok{(}\StringTok{"./RasterGrids\_100m/2024/Scaled/"}\NormalTok{,nosaukums)}
\NormalTok{slanis}\OtherTok{=}\FunctionTok{rast}\NormalTok{(ielasisanas\_cels)}
\NormalTok{videjais}\OtherTok{=}\FunctionTok{global}\NormalTok{(slanis,}\AttributeTok{fun=}\StringTok{"mean"}\NormalTok{,}\AttributeTok{na.rm=}\ConstantTok{TRUE}\NormalTok{)}
\NormalTok{centrets}\OtherTok{=}\NormalTok{slanis}\SpecialCharTok{{-}}\NormalTok{videjais[,}\DecValTok{1}\NormalTok{]}
\NormalTok{standartnovirze}\OtherTok{=}\NormalTok{terra}\SpecialCharTok{::}\FunctionTok{global}\NormalTok{(centrets,}\AttributeTok{fun=}\StringTok{"rms"}\NormalTok{,}\AttributeTok{na.rm=}\ConstantTok{TRUE}\NormalTok{)}
\NormalTok{merogots}\OtherTok{=}\NormalTok{centrets}\SpecialCharTok{/}\NormalTok{standartnovirze[,}\DecValTok{1}\NormalTok{]}
\FunctionTok{writeRaster}\NormalTok{(merogots,}
      \AttributeTok{filename=}\NormalTok{saglabasanas\_cels,}
      \AttributeTok{overwrite=}\ConstantTok{TRUE}\NormalTok{)}
\end{Highlighting}
\end{Shaded}

\section{ForestsTrees\_TemperateDeciduous\_r1250}\label{ch06.400}

\textbf{filename:} \texttt{ForestsTrees\_TemperateDeciduous\_r1250.tif}

\textbf{layername:} \texttt{egv\_400}

\textbf{English name:} Fractional cover of Temperate Deciduous Forests within the
1.25 km landscape

\textbf{Latvian name:} Platlapju mežu platības īpatsvars 1,25 km ainavā

\textbf{Procedure:} The cover fraction within a radius of 1250 m around the analysis grid cell
is calculated as the area-weighted sum of the \hyperref[ch06.398]{analysis cells} inside
the buffer, using the workflow \texttt{egvtools::radius\_function()}. During the calculation of the landscape
metric, inverse distance weighted (power = 2) gap filling on the output is
applied to ensure no missing values at the edges. Then the layer is
rewritten to set its name. Finally, the layer is standardised by
subtracting the arithmetic mean and dividing by the root mean squared error.

\begin{Shaded}
\begin{Highlighting}[]
\CommentTok{\# libs {-}{-}{-}{-}}
\ControlFlowTok{if}\NormalTok{(}\SpecialCharTok{!}\FunctionTok{require}\NormalTok{(terra)) \{}\FunctionTok{install.packages}\NormalTok{(}\StringTok{"terra"}\NormalTok{); }\FunctionTok{require}\NormalTok{(terra)\}}
\ControlFlowTok{if}\NormalTok{(}\SpecialCharTok{!}\FunctionTok{require}\NormalTok{(egvtools)) \{remotes}\SpecialCharTok{::}\FunctionTok{install\_github}\NormalTok{(}\StringTok{"aavotins/egvtools"}\NormalTok{); }\FunctionTok{require}\NormalTok{(egvtools)\}}


\CommentTok{\# Templates {-}{-}{-}{-}{-}}
\NormalTok{template100}\OtherTok{=}\FunctionTok{rast}\NormalTok{(}\StringTok{"./Templates/TemplateRasters/LV100m\_10km.tif"}\NormalTok{)}

\CommentTok{\# radii {-}{-}{-}{-}}
\FunctionTok{radius\_function}\NormalTok{(}
 \AttributeTok{kvadrati\_path =} \StringTok{"./Templates/TemplateGrids/tiles/"}\NormalTok{,}
 \AttributeTok{radii\_path   =} \StringTok{"./Templates/TemplateGridPoints/tiles/"}\NormalTok{,}
 \AttributeTok{tikls100\_path =} \StringTok{"./Templates/TemplateGrids/tikls100\_sauzeme.parquet"}\NormalTok{,}
 \AttributeTok{template\_path =} \StringTok{"./Templates/TemplateRasters/LV100m\_10km.tif"}\NormalTok{,}
 \AttributeTok{input\_layers  =} \FunctionTok{c}\NormalTok{(}\StringTok{"./RasterGrids\_100m/2024/RAW/ForestsTrees\_TemperateDeciduous\_cell.tif"}\NormalTok{),}
 \AttributeTok{layer\_prefixes =} \FunctionTok{c}\NormalTok{(}\StringTok{"ForestsTrees\_TemperateDeciduous"}\NormalTok{),}
 \AttributeTok{output\_dir   =} \StringTok{"./RasterGrids\_100m/2024/RAW/"}\NormalTok{,}
 \AttributeTok{n\_workers   =} \DecValTok{6}\NormalTok{,}
 \AttributeTok{radii     =} \FunctionTok{c}\NormalTok{(}\StringTok{"r1250"}\NormalTok{),}
 \AttributeTok{radius\_mode  =} \StringTok{"sparse"}\NormalTok{,}
 \AttributeTok{extract\_fun  =} \StringTok{"mean"}\NormalTok{,}
 \AttributeTok{fill\_missing  =} \ConstantTok{TRUE}\NormalTok{,}
 \AttributeTok{IDW\_weight   =} \DecValTok{2}\NormalTok{,}
 \AttributeTok{future\_max\_size =} \DecValTok{40} \SpecialCharTok{*} \DecValTok{1024}\SpecialCharTok{\^{}}\DecValTok{3}\NormalTok{)}


\CommentTok{\# ForestsTrees\_TemperateDeciduous\_r1250.tif egv\_400}
\NormalTok{slanis}\OtherTok{=}\FunctionTok{rast}\NormalTok{(}\StringTok{"./RasterGrids\_100m/2024/RAW/ForestsTrees\_TemperateDeciduous\_r1250.tif"}\NormalTok{)}
\FunctionTok{names}\NormalTok{(slanis)}\OtherTok{=}\StringTok{"egv\_400"}
\NormalTok{slanis2}\OtherTok{=}\FunctionTok{project}\NormalTok{(slanis,template100)}
\FunctionTok{writeRaster}\NormalTok{(slanis2,}
      \StringTok{"./RasterGrids\_100m/2024/RAW/ForestsTrees\_TemperateDeciduous\_r1250.tif"}\NormalTok{,}
      \AttributeTok{overwrite=}\ConstantTok{TRUE}\NormalTok{)}

\CommentTok{\# standardisation {-}{-}{-}{-}}
\ControlFlowTok{if}\NormalTok{(}\SpecialCharTok{!}\FunctionTok{require}\NormalTok{(terra)) \{}\FunctionTok{install.packages}\NormalTok{(}\StringTok{"terra"}\NormalTok{); }\FunctionTok{require}\NormalTok{(terra)\}}
\ControlFlowTok{if}\NormalTok{(}\SpecialCharTok{!}\FunctionTok{require}\NormalTok{(tidyverse)) \{}\FunctionTok{install.packages}\NormalTok{(}\StringTok{"tidyverse"}\NormalTok{); }\FunctionTok{require}\NormalTok{(tidyverse)\}}

\NormalTok{nosaukums}\OtherTok{=}\StringTok{"ForestsTrees\_TemperateDeciduous\_r1250.tif"}
\NormalTok{ielasisanas\_cels}\OtherTok{=}\FunctionTok{paste0}\NormalTok{(}\StringTok{"./RasterGrids\_100m/2024/RAW/"}\NormalTok{,nosaukums)}
\NormalTok{saglabasanas\_cels}\OtherTok{=}\FunctionTok{paste0}\NormalTok{(}\StringTok{"./RasterGrids\_100m/2024/Scaled/"}\NormalTok{,nosaukums)}
\NormalTok{slanis}\OtherTok{=}\FunctionTok{rast}\NormalTok{(ielasisanas\_cels)}
\NormalTok{videjais}\OtherTok{=}\FunctionTok{global}\NormalTok{(slanis,}\AttributeTok{fun=}\StringTok{"mean"}\NormalTok{,}\AttributeTok{na.rm=}\ConstantTok{TRUE}\NormalTok{)}
\NormalTok{centrets}\OtherTok{=}\NormalTok{slanis}\SpecialCharTok{{-}}\NormalTok{videjais[,}\DecValTok{1}\NormalTok{]}
\NormalTok{standartnovirze}\OtherTok{=}\NormalTok{terra}\SpecialCharTok{::}\FunctionTok{global}\NormalTok{(centrets,}\AttributeTok{fun=}\StringTok{"rms"}\NormalTok{,}\AttributeTok{na.rm=}\ConstantTok{TRUE}\NormalTok{)}
\NormalTok{merogots}\OtherTok{=}\NormalTok{centrets}\SpecialCharTok{/}\NormalTok{standartnovirze[,}\DecValTok{1}\NormalTok{]}
\FunctionTok{writeRaster}\NormalTok{(merogots,}
      \AttributeTok{filename=}\NormalTok{saglabasanas\_cels,}
      \AttributeTok{overwrite=}\ConstantTok{TRUE}\NormalTok{)}
\end{Highlighting}
\end{Shaded}

\section{ForestsTrees\_TemperateDeciduous\_r3000}\label{ch06.401}

\textbf{filename:} \texttt{ForestsTrees\_TemperateDeciduous\_r3000.tif}

\textbf{layername:} \texttt{egv\_401}

\textbf{English name:} Fractional cover of Temperate Deciduous Forests within the 3
km landscape

\textbf{Latvian name:} Platlapju mežu platības īpatsvars 3 km ainavā

\textbf{Procedure:} The cover fraction within a radius of 3000 m around the analysis grid cell
is calculated as the area-weighted sum of the \hyperref[ch06.398]{analysis cells} inside
the buffer, using the workflow \texttt{egvtools::radius\_function()}. During the calculation of the landscape
metric, inverse distance weighted (power = 2) gap filling on the output is
applied to ensure no missing values at the edges. Then the layer is
rewritten to set its name. Finally, the layer is standardised by
subtracting the arithmetic mean and dividing by the root mean squared error.

\begin{Shaded}
\begin{Highlighting}[]
\CommentTok{\# libs {-}{-}{-}{-}}
\ControlFlowTok{if}\NormalTok{(}\SpecialCharTok{!}\FunctionTok{require}\NormalTok{(terra)) \{}\FunctionTok{install.packages}\NormalTok{(}\StringTok{"terra"}\NormalTok{); }\FunctionTok{require}\NormalTok{(terra)\}}
\ControlFlowTok{if}\NormalTok{(}\SpecialCharTok{!}\FunctionTok{require}\NormalTok{(egvtools)) \{remotes}\SpecialCharTok{::}\FunctionTok{install\_github}\NormalTok{(}\StringTok{"aavotins/egvtools"}\NormalTok{); }\FunctionTok{require}\NormalTok{(egvtools)\}}


\CommentTok{\# Templates {-}{-}{-}{-}{-}}
\NormalTok{template100}\OtherTok{=}\FunctionTok{rast}\NormalTok{(}\StringTok{"./Templates/TemplateRasters/LV100m\_10km.tif"}\NormalTok{)}

\CommentTok{\# radii {-}{-}{-}{-}}
\FunctionTok{radius\_function}\NormalTok{(}
 \AttributeTok{kvadrati\_path =} \StringTok{"./Templates/TemplateGrids/tiles/"}\NormalTok{,}
 \AttributeTok{radii\_path   =} \StringTok{"./Templates/TemplateGridPoints/tiles/"}\NormalTok{,}
 \AttributeTok{tikls100\_path =} \StringTok{"./Templates/TemplateGrids/tikls100\_sauzeme.parquet"}\NormalTok{,}
 \AttributeTok{template\_path =} \StringTok{"./Templates/TemplateRasters/LV100m\_10km.tif"}\NormalTok{,}
 \AttributeTok{input\_layers  =} \FunctionTok{c}\NormalTok{(}\StringTok{"./RasterGrids\_100m/2024/RAW/ForestsTrees\_TemperateDeciduous\_cell.tif"}\NormalTok{),}
 \AttributeTok{layer\_prefixes =} \FunctionTok{c}\NormalTok{(}\StringTok{"ForestsTrees\_TemperateDeciduous"}\NormalTok{),}
 \AttributeTok{output\_dir   =} \StringTok{"./RasterGrids\_100m/2024/RAW/"}\NormalTok{,}
 \AttributeTok{n\_workers   =} \DecValTok{6}\NormalTok{,}
 \AttributeTok{radii     =} \FunctionTok{c}\NormalTok{(}\StringTok{"r3000"}\NormalTok{),}
 \AttributeTok{radius\_mode  =} \StringTok{"sparse"}\NormalTok{,}
 \AttributeTok{extract\_fun  =} \StringTok{"mean"}\NormalTok{,}
 \AttributeTok{fill\_missing  =} \ConstantTok{TRUE}\NormalTok{,}
 \AttributeTok{IDW\_weight   =} \DecValTok{2}\NormalTok{,}
 \AttributeTok{future\_max\_size =} \DecValTok{40} \SpecialCharTok{*} \DecValTok{1024}\SpecialCharTok{\^{}}\DecValTok{3}\NormalTok{)}


\CommentTok{\# ForestsTrees\_TemperateDeciduous\_r3000.tif egv\_401}
\NormalTok{slanis}\OtherTok{=}\FunctionTok{rast}\NormalTok{(}\StringTok{"./RasterGrids\_100m/2024/RAW/ForestsTrees\_TemperateDeciduous\_r3000.tif"}\NormalTok{)}
\FunctionTok{names}\NormalTok{(slanis)}\OtherTok{=}\StringTok{"egv\_401"}
\NormalTok{slanis2}\OtherTok{=}\FunctionTok{project}\NormalTok{(slanis,template100)}
\FunctionTok{writeRaster}\NormalTok{(slanis2,}
      \StringTok{"./RasterGrids\_100m/2024/RAW/ForestsTrees\_TemperateDeciduous\_r3000.tif"}\NormalTok{,}
      \AttributeTok{overwrite=}\ConstantTok{TRUE}\NormalTok{)}

\CommentTok{\# standardisation {-}{-}{-}{-}}
\ControlFlowTok{if}\NormalTok{(}\SpecialCharTok{!}\FunctionTok{require}\NormalTok{(terra)) \{}\FunctionTok{install.packages}\NormalTok{(}\StringTok{"terra"}\NormalTok{); }\FunctionTok{require}\NormalTok{(terra)\}}
\ControlFlowTok{if}\NormalTok{(}\SpecialCharTok{!}\FunctionTok{require}\NormalTok{(tidyverse)) \{}\FunctionTok{install.packages}\NormalTok{(}\StringTok{"tidyverse"}\NormalTok{); }\FunctionTok{require}\NormalTok{(tidyverse)\}}

\NormalTok{nosaukums}\OtherTok{=}\StringTok{"ForestsTrees\_TemperateDeciduous\_r3000.tif"}
\NormalTok{ielasisanas\_cels}\OtherTok{=}\FunctionTok{paste0}\NormalTok{(}\StringTok{"./RasterGrids\_100m/2024/RAW/"}\NormalTok{,nosaukums)}
\NormalTok{saglabasanas\_cels}\OtherTok{=}\FunctionTok{paste0}\NormalTok{(}\StringTok{"./RasterGrids\_100m/2024/Scaled/"}\NormalTok{,nosaukums)}
\NormalTok{slanis}\OtherTok{=}\FunctionTok{rast}\NormalTok{(ielasisanas\_cels)}
\NormalTok{videjais}\OtherTok{=}\FunctionTok{global}\NormalTok{(slanis,}\AttributeTok{fun=}\StringTok{"mean"}\NormalTok{,}\AttributeTok{na.rm=}\ConstantTok{TRUE}\NormalTok{)}
\NormalTok{centrets}\OtherTok{=}\NormalTok{slanis}\SpecialCharTok{{-}}\NormalTok{videjais[,}\DecValTok{1}\NormalTok{]}
\NormalTok{standartnovirze}\OtherTok{=}\NormalTok{terra}\SpecialCharTok{::}\FunctionTok{global}\NormalTok{(centrets,}\AttributeTok{fun=}\StringTok{"rms"}\NormalTok{,}\AttributeTok{na.rm=}\ConstantTok{TRUE}\NormalTok{)}
\NormalTok{merogots}\OtherTok{=}\NormalTok{centrets}\SpecialCharTok{/}\NormalTok{standartnovirze[,}\DecValTok{1}\NormalTok{]}
\FunctionTok{writeRaster}\NormalTok{(merogots,}
      \AttributeTok{filename=}\NormalTok{saglabasanas\_cels,}
      \AttributeTok{overwrite=}\ConstantTok{TRUE}\NormalTok{)}
\end{Highlighting}
\end{Shaded}

\section{ForestsTrees\_TemperateDeciduous\_r10000}\label{ch06.402}

\textbf{filename:} \texttt{ForestsTrees\_TemperateDeciduous\_r10000.tif}

\textbf{layername:} \texttt{egv\_402}

\textbf{English name:} Fractional cover of Temperate Deciduous Forests within the 10
km landscape

\textbf{Latvian name:} Platlapju mežu platības īpatsvars 10 km ainavā

\textbf{Procedure:} The cover fraction within a radius of 10000 m around the analysis grid cell
is calculated as the area-weighted sum of the \hyperref[ch06.398]{analysis cells} inside
the buffer, using the workflow \texttt{egvtools::radius\_function()}. During the calculation of the landscape
metric, inverse distance weighted (power = 2) gap filling on the output is
applied to ensure no missing values at the edges. Then the layer is
rewritten to set its name. Finally, the layer is standardised by
subtracting the arithmetic mean and dividing by the root mean squared error.

\begin{Shaded}
\begin{Highlighting}[]
\CommentTok{\# libs {-}{-}{-}{-}}
\ControlFlowTok{if}\NormalTok{(}\SpecialCharTok{!}\FunctionTok{require}\NormalTok{(terra)) \{}\FunctionTok{install.packages}\NormalTok{(}\StringTok{"terra"}\NormalTok{); }\FunctionTok{require}\NormalTok{(terra)\}}
\ControlFlowTok{if}\NormalTok{(}\SpecialCharTok{!}\FunctionTok{require}\NormalTok{(egvtools)) \{remotes}\SpecialCharTok{::}\FunctionTok{install\_github}\NormalTok{(}\StringTok{"aavotins/egvtools"}\NormalTok{); }\FunctionTok{require}\NormalTok{(egvtools)\}}


\CommentTok{\# Templates {-}{-}{-}{-}{-}}
\NormalTok{template100}\OtherTok{=}\FunctionTok{rast}\NormalTok{(}\StringTok{"./Templates/TemplateRasters/LV100m\_10km.tif"}\NormalTok{)}

\CommentTok{\# radii {-}{-}{-}{-}}
\FunctionTok{radius\_function}\NormalTok{(}
 \AttributeTok{kvadrati\_path =} \StringTok{"./Templates/TemplateGrids/tiles/"}\NormalTok{,}
 \AttributeTok{radii\_path   =} \StringTok{"./Templates/TemplateGridPoints/tiles/"}\NormalTok{,}
 \AttributeTok{tikls100\_path =} \StringTok{"./Templates/TemplateGrids/tikls100\_sauzeme.parquet"}\NormalTok{,}
 \AttributeTok{template\_path =} \StringTok{"./Templates/TemplateRasters/LV100m\_10km.tif"}\NormalTok{,}
 \AttributeTok{input\_layers  =} \FunctionTok{c}\NormalTok{(}\StringTok{"./RasterGrids\_100m/2024/RAW/ForestsTrees\_TemperateDeciduous\_cell.tif"}\NormalTok{),}
 \AttributeTok{layer\_prefixes =} \FunctionTok{c}\NormalTok{(}\StringTok{"ForestsTrees\_TemperateDeciduous"}\NormalTok{),}
 \AttributeTok{output\_dir   =} \StringTok{"./RasterGrids\_100m/2024/RAW/"}\NormalTok{,}
 \AttributeTok{n\_workers   =} \DecValTok{6}\NormalTok{,}
 \AttributeTok{radii     =} \FunctionTok{c}\NormalTok{(}\StringTok{"r10000"}\NormalTok{),}
 \AttributeTok{radius\_mode  =} \StringTok{"sparse"}\NormalTok{,}
 \AttributeTok{extract\_fun  =} \StringTok{"mean"}\NormalTok{,}
 \AttributeTok{fill\_missing  =} \ConstantTok{TRUE}\NormalTok{,}
 \AttributeTok{IDW\_weight   =} \DecValTok{2}\NormalTok{,}
 \AttributeTok{future\_max\_size =} \DecValTok{40} \SpecialCharTok{*} \DecValTok{1024}\SpecialCharTok{\^{}}\DecValTok{3}\NormalTok{)}


\CommentTok{\# ForestsTrees\_TemperateDeciduous\_r10000.tif    egv\_402}
\NormalTok{slanis}\OtherTok{=}\FunctionTok{rast}\NormalTok{(}\StringTok{"./RasterGrids\_100m/2024/RAW/ForestsTrees\_TemperateDeciduous\_r10000.tif"}\NormalTok{)}
\FunctionTok{names}\NormalTok{(slanis)}\OtherTok{=}\StringTok{"egv\_402"}
\NormalTok{slanis2}\OtherTok{=}\FunctionTok{project}\NormalTok{(slanis,template100)}
\FunctionTok{writeRaster}\NormalTok{(slanis2,}
      \StringTok{"./RasterGrids\_100m/2024/RAW/ForestsTrees\_TemperateDeciduous\_r10000.tif"}\NormalTok{,}
      \AttributeTok{overwrite=}\ConstantTok{TRUE}\NormalTok{)}

\CommentTok{\# standardisation {-}{-}{-}{-}}
\ControlFlowTok{if}\NormalTok{(}\SpecialCharTok{!}\FunctionTok{require}\NormalTok{(terra)) \{}\FunctionTok{install.packages}\NormalTok{(}\StringTok{"terra"}\NormalTok{); }\FunctionTok{require}\NormalTok{(terra)\}}
\ControlFlowTok{if}\NormalTok{(}\SpecialCharTok{!}\FunctionTok{require}\NormalTok{(tidyverse)) \{}\FunctionTok{install.packages}\NormalTok{(}\StringTok{"tidyverse"}\NormalTok{); }\FunctionTok{require}\NormalTok{(tidyverse)\}}

\NormalTok{nosaukums}\OtherTok{=}\StringTok{"ForestsTrees\_TemperateDeciduous\_r10000.tif"}
\NormalTok{ielasisanas\_cels}\OtherTok{=}\FunctionTok{paste0}\NormalTok{(}\StringTok{"./RasterGrids\_100m/2024/RAW/"}\NormalTok{,nosaukums)}
\NormalTok{saglabasanas\_cels}\OtherTok{=}\FunctionTok{paste0}\NormalTok{(}\StringTok{"./RasterGrids\_100m/2024/Scaled/"}\NormalTok{,nosaukums)}
\NormalTok{slanis}\OtherTok{=}\FunctionTok{rast}\NormalTok{(ielasisanas\_cels)}
\NormalTok{videjais}\OtherTok{=}\FunctionTok{global}\NormalTok{(slanis,}\AttributeTok{fun=}\StringTok{"mean"}\NormalTok{,}\AttributeTok{na.rm=}\ConstantTok{TRUE}\NormalTok{)}
\NormalTok{centrets}\OtherTok{=}\NormalTok{slanis}\SpecialCharTok{{-}}\NormalTok{videjais[,}\DecValTok{1}\NormalTok{]}
\NormalTok{standartnovirze}\OtherTok{=}\NormalTok{terra}\SpecialCharTok{::}\FunctionTok{global}\NormalTok{(centrets,}\AttributeTok{fun=}\StringTok{"rms"}\NormalTok{,}\AttributeTok{na.rm=}\ConstantTok{TRUE}\NormalTok{)}
\NormalTok{merogots}\OtherTok{=}\NormalTok{centrets}\SpecialCharTok{/}\NormalTok{standartnovirze[,}\DecValTok{1}\NormalTok{]}
\FunctionTok{writeRaster}\NormalTok{(merogots,}
      \AttributeTok{filename=}\NormalTok{saglabasanas\_cels,}
      \AttributeTok{overwrite=}\ConstantTok{TRUE}\NormalTok{)}
\end{Highlighting}
\end{Shaded}

\section{General\_AllotmentGardens\_cell}\label{ch06.403}

\textbf{filename:} \texttt{General\_AllotmentGardens\_cell.tif}

\textbf{layername:} \texttt{egv\_403}

\textbf{English name:} Fractional cover of Allotment gardens within the analysis cell
(1 ha)

\textbf{Latvian name:} Vasarnīcu kompleksu platības īpatsvars analīzes šūnā (1 ha)

\textbf{Procedure:} First, the allotment gardens and farmsteads from the \hyperref[Ch05.03]{Landscape
classification} are selected (value 410 is reclassified to value 1;
all others are set to 0). The resulting layer
is then aggregated to EGV resolution using the workflow \texttt{egvtools::input2egv()}, which
calculates the arithmetic mean to determine the cover fraction. During
aggregation, inverse distance weighted (power = 2) gap filling on the output is
applied to ensure no missing values at the edges. Finally, the layer is
standardised by subtracting the arithmetic mean and dividing by the root mean squared
error.

\begin{Shaded}
\begin{Highlighting}[]
\CommentTok{\# libs {-}{-}{-}{-}}
\ControlFlowTok{if}\NormalTok{(}\SpecialCharTok{!}\FunctionTok{require}\NormalTok{(egvtools)) \{remotes}\SpecialCharTok{::}\FunctionTok{install\_github}\NormalTok{(}\StringTok{"aavotins/egvtools"}\NormalTok{); }\FunctionTok{require}\NormalTok{(egvtools)\}}
\ControlFlowTok{if}\NormalTok{(}\SpecialCharTok{!}\FunctionTok{require}\NormalTok{(terra)) \{}\FunctionTok{install.packages}\NormalTok{(}\StringTok{"terra"}\NormalTok{); }\FunctionTok{require}\NormalTok{(terra)\}}
\ControlFlowTok{if}\NormalTok{(}\SpecialCharTok{!}\FunctionTok{require}\NormalTok{(sf)) \{}\FunctionTok{install.packages}\NormalTok{(}\StringTok{"sf"}\NormalTok{); }\FunctionTok{require}\NormalTok{(sf)\}}
\ControlFlowTok{if}\NormalTok{(}\SpecialCharTok{!}\FunctionTok{require}\NormalTok{(tidyverse)) \{}\FunctionTok{install.packages}\NormalTok{(}\StringTok{"tidyverse"}\NormalTok{); }\FunctionTok{require}\NormalTok{(tidyverse)\}}
\ControlFlowTok{if}\NormalTok{(}\SpecialCharTok{!}\FunctionTok{require}\NormalTok{(sfarrow)) \{}\FunctionTok{install.packages}\NormalTok{(}\StringTok{"sfarrow"}\NormalTok{); }\FunctionTok{require}\NormalTok{(sfarrow)\}}
\ControlFlowTok{if}\NormalTok{(}\SpecialCharTok{!}\FunctionTok{require}\NormalTok{(readxl)) \{}\FunctionTok{install.packages}\NormalTok{(}\StringTok{"readxl"}\NormalTok{); }\FunctionTok{require}\NormalTok{(readxl)\}}
\ControlFlowTok{if}\NormalTok{(}\SpecialCharTok{!}\FunctionTok{require}\NormalTok{(raster)) \{}\FunctionTok{install.packages}\NormalTok{(}\StringTok{"raster"}\NormalTok{); }\FunctionTok{require}\NormalTok{(raster)\}}
\ControlFlowTok{if}\NormalTok{(}\SpecialCharTok{!}\FunctionTok{require}\NormalTok{(fasterize)) \{}\FunctionTok{install.packages}\NormalTok{(}\StringTok{"fasterize"}\NormalTok{); }\FunctionTok{require}\NormalTok{(fasterize)\}}

\CommentTok{\# templates {-}{-}{-}{-}}
\NormalTok{template100}\OtherTok{=}\FunctionTok{rast}\NormalTok{(}\StringTok{"./Templates/TemplateRasters/LV100m\_10km.tif"}\NormalTok{)}
\NormalTok{template10}\OtherTok{=}\FunctionTok{rast}\NormalTok{(}\StringTok{"./Templates/TemplateRasters/LV10m\_10km.tif"}\NormalTok{)}
\NormalTok{rastrs10}\OtherTok{=}\FunctionTok{raster}\NormalTok{(template10)}

\NormalTok{nulls10}\OtherTok{=}\FunctionTok{rast}\NormalTok{(}\StringTok{"./Templates/TemplateRasters/nulls\_LV10m\_10km.tif"}\NormalTok{)}
\NormalTok{nulls100}\OtherTok{=}\FunctionTok{rast}\NormalTok{(}\StringTok{"./Templates/TemplateRasters/nulls\_LV100m\_10km.tif"}\NormalTok{)}

\CommentTok{\# simple landscape {-}{-}{-}{-}}
\NormalTok{simple\_landscape}\OtherTok{=}\FunctionTok{rast}\NormalTok{(}\StringTok{"RasterGrids\_10m/2024/Ainava\_vienk\_mask.tif"}\NormalTok{)}


\CommentTok{\# General\_AllotmentGardens\_cell.tif egv\_403 {-}{-}{-}{-}}
\NormalTok{allotmentgardens}\OtherTok{=}\FunctionTok{ifel}\NormalTok{(simple\_landscape}\SpecialCharTok{==}\DecValTok{410}\NormalTok{,}\DecValTok{1}\NormalTok{,}\DecValTok{0}\NormalTok{)}
\NormalTok{i2e\_rez}\OtherTok{=}\NormalTok{egvtools}\SpecialCharTok{::}\FunctionTok{input2egv}\NormalTok{(}\AttributeTok{input=}\NormalTok{allotmentgardens,}
              \AttributeTok{egv\_template=} \StringTok{"./Templates/TemplateRasters/LV100m\_10km.tif"}\NormalTok{,}
              \AttributeTok{summary\_function =} \StringTok{"average"}\NormalTok{,}
              \AttributeTok{missing\_job =} \StringTok{"FillOutput"}\NormalTok{,}
              \AttributeTok{outlocation =} \StringTok{"./RasterGrids\_100m/2024/RAW/"}\NormalTok{,}
              \AttributeTok{outfilename =} \StringTok{"General\_AllotmentGardens\_cell.tif"}\NormalTok{,}
              \AttributeTok{layername =} \StringTok{"egv\_403"}\NormalTok{,}
              \AttributeTok{idw\_weight =} \DecValTok{2}\NormalTok{,}
              \AttributeTok{plot\_gaps =} \ConstantTok{FALSE}\NormalTok{,}\AttributeTok{plot\_final =} \ConstantTok{TRUE}\NormalTok{)}
\NormalTok{i2e\_rez}
\FunctionTok{rm}\NormalTok{(allotmentgardens)}
\FunctionTok{rm}\NormalTok{(i2e\_rez)}

\CommentTok{\# standardisation {-}{-}{-}{-}}
\ControlFlowTok{if}\NormalTok{(}\SpecialCharTok{!}\FunctionTok{require}\NormalTok{(terra)) \{}\FunctionTok{install.packages}\NormalTok{(}\StringTok{"terra"}\NormalTok{); }\FunctionTok{require}\NormalTok{(terra)\}}
\ControlFlowTok{if}\NormalTok{(}\SpecialCharTok{!}\FunctionTok{require}\NormalTok{(tidyverse)) \{}\FunctionTok{install.packages}\NormalTok{(}\StringTok{"tidyverse"}\NormalTok{); }\FunctionTok{require}\NormalTok{(tidyverse)\}}

\NormalTok{nosaukums}\OtherTok{=}\StringTok{"General\_AllotmentGardens\_cell.tif"}
\NormalTok{ielasisanas\_cels}\OtherTok{=}\FunctionTok{paste0}\NormalTok{(}\StringTok{"./RasterGrids\_100m/2024/RAW/"}\NormalTok{,nosaukums)}
\NormalTok{saglabasanas\_cels}\OtherTok{=}\FunctionTok{paste0}\NormalTok{(}\StringTok{"./RasterGrids\_100m/2024/Scaled/"}\NormalTok{,nosaukums)}
\NormalTok{slanis}\OtherTok{=}\FunctionTok{rast}\NormalTok{(ielasisanas\_cels)}
\NormalTok{videjais}\OtherTok{=}\FunctionTok{global}\NormalTok{(slanis,}\AttributeTok{fun=}\StringTok{"mean"}\NormalTok{,}\AttributeTok{na.rm=}\ConstantTok{TRUE}\NormalTok{)}
\NormalTok{centrets}\OtherTok{=}\NormalTok{slanis}\SpecialCharTok{{-}}\NormalTok{videjais[,}\DecValTok{1}\NormalTok{]}
\NormalTok{standartnovirze}\OtherTok{=}\NormalTok{terra}\SpecialCharTok{::}\FunctionTok{global}\NormalTok{(centrets,}\AttributeTok{fun=}\StringTok{"rms"}\NormalTok{,}\AttributeTok{na.rm=}\ConstantTok{TRUE}\NormalTok{)}
\NormalTok{merogots}\OtherTok{=}\NormalTok{centrets}\SpecialCharTok{/}\NormalTok{standartnovirze[,}\DecValTok{1}\NormalTok{]}
\FunctionTok{writeRaster}\NormalTok{(merogots,}
      \AttributeTok{filename=}\NormalTok{saglabasanas\_cels,}
      \AttributeTok{overwrite=}\ConstantTok{TRUE}\NormalTok{)}
\end{Highlighting}
\end{Shaded}

\section{General\_AllotmentGardens\_r500}\label{ch06.404}

\textbf{filename:} \texttt{General\_AllotmentGardens\_r500.tif}

\textbf{layername:} \texttt{egv\_404}

\textbf{English name:} Fractional cover of Allotment gardens within the 0.5 km
landscape

\textbf{Latvian name:} Vasarnīcu kompleksu platības īpatsvars 0,5 km ainavā

\textbf{Procedure:} The cover fraction within a radius of 500 m around the analysis grid cell is
calculated as the area-weighted sum of the \hyperref[ch06.403]{analysis cells} inside the
buffer, using the workflow \texttt{egvtools::radius\_function()}. During the calculation of the landscape metric,
inverse distance weighted (power = 2) gap filling on the output is applied
to ensure no missing values at the edges. Then the layer is rewritten to set
its name. Finally, the layer is standardised by subtracting the arithmetic
mean and dividing by the root mean squared error.

\begin{Shaded}
\begin{Highlighting}[]
\CommentTok{\# libs {-}{-}{-}{-}}
\ControlFlowTok{if}\NormalTok{(}\SpecialCharTok{!}\FunctionTok{require}\NormalTok{(terra)) \{}\FunctionTok{install.packages}\NormalTok{(}\StringTok{"terra"}\NormalTok{); }\FunctionTok{require}\NormalTok{(terra)\}}
\ControlFlowTok{if}\NormalTok{(}\SpecialCharTok{!}\FunctionTok{require}\NormalTok{(egvtools)) \{remotes}\SpecialCharTok{::}\FunctionTok{install\_github}\NormalTok{(}\StringTok{"aavotins/egvtools"}\NormalTok{); }\FunctionTok{require}\NormalTok{(egvtools)\}}


\CommentTok{\# Templates {-}{-}{-}{-}{-}}
\NormalTok{template100}\OtherTok{=}\FunctionTok{rast}\NormalTok{(}\StringTok{"./Templates/TemplateRasters/LV100m\_10km.tif"}\NormalTok{)}

\CommentTok{\# radii {-}{-}{-}{-}}
\FunctionTok{radius\_function}\NormalTok{(}
 \AttributeTok{kvadrati\_path =} \StringTok{"./Templates/TemplateGrids/tiles/"}\NormalTok{,}
 \AttributeTok{radii\_path   =} \StringTok{"./Templates/TemplateGridPoints/tiles/"}\NormalTok{,}
 \AttributeTok{tikls100\_path =} \StringTok{"./Templates/TemplateGrids/tikls100\_sauzeme.parquet"}\NormalTok{,}
 \AttributeTok{template\_path =} \StringTok{"./Templates/TemplateRasters/LV100m\_10km.tif"}\NormalTok{,}
 \AttributeTok{input\_layers  =} \FunctionTok{c}\NormalTok{(}\StringTok{"./RasterGrids\_100m/2024/RAW/General\_AllotmentGardens\_cell.tif"}\NormalTok{),}
 \AttributeTok{layer\_prefixes =} \FunctionTok{c}\NormalTok{(}\StringTok{"General\_AllotmentGardens"}\NormalTok{),}
 \AttributeTok{output\_dir   =} \StringTok{"./RasterGrids\_100m/2024/RAW/"}\NormalTok{,}
 \AttributeTok{n\_workers   =} \DecValTok{6}\NormalTok{,}
 \AttributeTok{radii     =} \FunctionTok{c}\NormalTok{(}\StringTok{"r500"}\NormalTok{),}
 \AttributeTok{radius\_mode  =} \StringTok{"sparse"}\NormalTok{,}
 \AttributeTok{extract\_fun  =} \StringTok{"mean"}\NormalTok{,}
 \AttributeTok{fill\_missing  =} \ConstantTok{TRUE}\NormalTok{,}
 \AttributeTok{IDW\_weight   =} \DecValTok{2}\NormalTok{,}
 \AttributeTok{future\_max\_size =} \DecValTok{40} \SpecialCharTok{*} \DecValTok{1024}\SpecialCharTok{\^{}}\DecValTok{3}\NormalTok{)}


\CommentTok{\# General\_AllotmentGardens\_r500.tif egv\_404}
\NormalTok{slanis}\OtherTok{=}\FunctionTok{rast}\NormalTok{(}\StringTok{"./RasterGrids\_100m/2024/RAW/General\_AllotmentGardens\_r500.tif"}\NormalTok{)}
\FunctionTok{names}\NormalTok{(slanis)}\OtherTok{=}\StringTok{"egv\_404"}
\NormalTok{slanis2}\OtherTok{=}\FunctionTok{project}\NormalTok{(slanis,template100)}
\FunctionTok{writeRaster}\NormalTok{(slanis2,}
      \StringTok{"./RasterGrids\_100m/2024/RAW/General\_AllotmentGardens\_r500.tif"}\NormalTok{,}
      \AttributeTok{overwrite=}\ConstantTok{TRUE}\NormalTok{)}

\CommentTok{\# standardisation {-}{-}{-}{-}}
\ControlFlowTok{if}\NormalTok{(}\SpecialCharTok{!}\FunctionTok{require}\NormalTok{(terra)) \{}\FunctionTok{install.packages}\NormalTok{(}\StringTok{"terra"}\NormalTok{); }\FunctionTok{require}\NormalTok{(terra)\}}
\ControlFlowTok{if}\NormalTok{(}\SpecialCharTok{!}\FunctionTok{require}\NormalTok{(tidyverse)) \{}\FunctionTok{install.packages}\NormalTok{(}\StringTok{"tidyverse"}\NormalTok{); }\FunctionTok{require}\NormalTok{(tidyverse)\}}

\NormalTok{nosaukums}\OtherTok{=}\StringTok{"General\_AllotmentGardens\_r500.tif"}
\NormalTok{ielasisanas\_cels}\OtherTok{=}\FunctionTok{paste0}\NormalTok{(}\StringTok{"./RasterGrids\_100m/2024/RAW/"}\NormalTok{,nosaukums)}
\NormalTok{saglabasanas\_cels}\OtherTok{=}\FunctionTok{paste0}\NormalTok{(}\StringTok{"./RasterGrids\_100m/2024/Scaled/"}\NormalTok{,nosaukums)}
\NormalTok{slanis}\OtherTok{=}\FunctionTok{rast}\NormalTok{(ielasisanas\_cels)}
\NormalTok{videjais}\OtherTok{=}\FunctionTok{global}\NormalTok{(slanis,}\AttributeTok{fun=}\StringTok{"mean"}\NormalTok{,}\AttributeTok{na.rm=}\ConstantTok{TRUE}\NormalTok{)}
\NormalTok{centrets}\OtherTok{=}\NormalTok{slanis}\SpecialCharTok{{-}}\NormalTok{videjais[,}\DecValTok{1}\NormalTok{]}
\NormalTok{standartnovirze}\OtherTok{=}\NormalTok{terra}\SpecialCharTok{::}\FunctionTok{global}\NormalTok{(centrets,}\AttributeTok{fun=}\StringTok{"rms"}\NormalTok{,}\AttributeTok{na.rm=}\ConstantTok{TRUE}\NormalTok{)}
\NormalTok{merogots}\OtherTok{=}\NormalTok{centrets}\SpecialCharTok{/}\NormalTok{standartnovirze[,}\DecValTok{1}\NormalTok{]}
\FunctionTok{writeRaster}\NormalTok{(merogots,}
      \AttributeTok{filename=}\NormalTok{saglabasanas\_cels,}
      \AttributeTok{overwrite=}\ConstantTok{TRUE}\NormalTok{)}
\end{Highlighting}
\end{Shaded}

\section{General\_AllotmentGardens\_r1250}\label{ch06.405}

\textbf{filename:} \texttt{General\_AllotmentGardens\_r1250.tif}

\textbf{layername:} \texttt{egv\_405}

\textbf{English name:} Fractional cover of Allotment gardens within the 1.25 km
landscape

\textbf{Latvian name:} Vasarnīcu kompleksu platības īpatsvars 1,25 km ainavā

\textbf{Procedure:} The cover fraction within a radius of 1250 m around the analysis grid cell
is calculated as the area-weighted sum of the \hyperref[ch06.403]{analysis cells} inside
the buffer, using the workflow \texttt{egvtools::radius\_function()}. During the calculation of the landscape
metric, inverse distance weighted (power = 2) gap filling on the output is
applied to ensure no missing values at the edges. Then the layer is
rewritten to set its name. Finally, the layer is standardised by
subtracting the arithmetic mean and dividing by the root mean squared error.

\begin{Shaded}
\begin{Highlighting}[]
\CommentTok{\# libs {-}{-}{-}{-}}
\ControlFlowTok{if}\NormalTok{(}\SpecialCharTok{!}\FunctionTok{require}\NormalTok{(terra)) \{}\FunctionTok{install.packages}\NormalTok{(}\StringTok{"terra"}\NormalTok{); }\FunctionTok{require}\NormalTok{(terra)\}}
\ControlFlowTok{if}\NormalTok{(}\SpecialCharTok{!}\FunctionTok{require}\NormalTok{(egvtools)) \{remotes}\SpecialCharTok{::}\FunctionTok{install\_github}\NormalTok{(}\StringTok{"aavotins/egvtools"}\NormalTok{); }\FunctionTok{require}\NormalTok{(egvtools)\}}


\CommentTok{\# Templates {-}{-}{-}{-}{-}}
\NormalTok{template100}\OtherTok{=}\FunctionTok{rast}\NormalTok{(}\StringTok{"./Templates/TemplateRasters/LV100m\_10km.tif"}\NormalTok{)}

\CommentTok{\# radii {-}{-}{-}{-}}
\FunctionTok{radius\_function}\NormalTok{(}
 \AttributeTok{kvadrati\_path =} \StringTok{"./Templates/TemplateGrids/tiles/"}\NormalTok{,}
 \AttributeTok{radii\_path   =} \StringTok{"./Templates/TemplateGridPoints/tiles/"}\NormalTok{,}
 \AttributeTok{tikls100\_path =} \StringTok{"./Templates/TemplateGrids/tikls100\_sauzeme.parquet"}\NormalTok{,}
 \AttributeTok{template\_path =} \StringTok{"./Templates/TemplateRasters/LV100m\_10km.tif"}\NormalTok{,}
 \AttributeTok{input\_layers  =} \FunctionTok{c}\NormalTok{(}\StringTok{"./RasterGrids\_100m/2024/RAW/General\_AllotmentGardens\_cell.tif"}\NormalTok{),}
 \AttributeTok{layer\_prefixes =} \FunctionTok{c}\NormalTok{(}\StringTok{"General\_AllotmentGardens"}\NormalTok{),}
 \AttributeTok{output\_dir   =} \StringTok{"./RasterGrids\_100m/2024/RAW/"}\NormalTok{,}
 \AttributeTok{n\_workers   =} \DecValTok{6}\NormalTok{,}
 \AttributeTok{radii     =} \FunctionTok{c}\NormalTok{(}\StringTok{"r1250"}\NormalTok{),}
 \AttributeTok{radius\_mode  =} \StringTok{"sparse"}\NormalTok{,}
 \AttributeTok{extract\_fun  =} \StringTok{"mean"}\NormalTok{,}
 \AttributeTok{fill\_missing  =} \ConstantTok{TRUE}\NormalTok{,}
 \AttributeTok{IDW\_weight   =} \DecValTok{2}\NormalTok{,}
 \AttributeTok{future\_max\_size =} \DecValTok{40} \SpecialCharTok{*} \DecValTok{1024}\SpecialCharTok{\^{}}\DecValTok{3}\NormalTok{)}


\CommentTok{\# General\_AllotmentGardens\_r1250.tif    egv\_405}
\NormalTok{slanis}\OtherTok{=}\FunctionTok{rast}\NormalTok{(}\StringTok{"./RasterGrids\_100m/2024/RAW/General\_AllotmentGardens\_r1250.tif"}\NormalTok{)}
\FunctionTok{names}\NormalTok{(slanis)}\OtherTok{=}\StringTok{"egv\_405"}
\NormalTok{slanis2}\OtherTok{=}\FunctionTok{project}\NormalTok{(slanis,template100)}
\FunctionTok{writeRaster}\NormalTok{(slanis2,}
      \StringTok{"./RasterGrids\_100m/2024/RAW/General\_AllotmentGardens\_r1250.tif"}\NormalTok{,}
      \AttributeTok{overwrite=}\ConstantTok{TRUE}\NormalTok{)}

\CommentTok{\# standardisation {-}{-}{-}{-}}
\ControlFlowTok{if}\NormalTok{(}\SpecialCharTok{!}\FunctionTok{require}\NormalTok{(terra)) \{}\FunctionTok{install.packages}\NormalTok{(}\StringTok{"terra"}\NormalTok{); }\FunctionTok{require}\NormalTok{(terra)\}}
\ControlFlowTok{if}\NormalTok{(}\SpecialCharTok{!}\FunctionTok{require}\NormalTok{(tidyverse)) \{}\FunctionTok{install.packages}\NormalTok{(}\StringTok{"tidyverse"}\NormalTok{); }\FunctionTok{require}\NormalTok{(tidyverse)\}}

\NormalTok{nosaukums}\OtherTok{=}\StringTok{"General\_AllotmentGardens\_r1250.tif"}
\NormalTok{ielasisanas\_cels}\OtherTok{=}\FunctionTok{paste0}\NormalTok{(}\StringTok{"./RasterGrids\_100m/2024/RAW/"}\NormalTok{,nosaukums)}
\NormalTok{saglabasanas\_cels}\OtherTok{=}\FunctionTok{paste0}\NormalTok{(}\StringTok{"./RasterGrids\_100m/2024/Scaled/"}\NormalTok{,nosaukums)}
\NormalTok{slanis}\OtherTok{=}\FunctionTok{rast}\NormalTok{(ielasisanas\_cels)}
\NormalTok{videjais}\OtherTok{=}\FunctionTok{global}\NormalTok{(slanis,}\AttributeTok{fun=}\StringTok{"mean"}\NormalTok{,}\AttributeTok{na.rm=}\ConstantTok{TRUE}\NormalTok{)}
\NormalTok{centrets}\OtherTok{=}\NormalTok{slanis}\SpecialCharTok{{-}}\NormalTok{videjais[,}\DecValTok{1}\NormalTok{]}
\NormalTok{standartnovirze}\OtherTok{=}\NormalTok{terra}\SpecialCharTok{::}\FunctionTok{global}\NormalTok{(centrets,}\AttributeTok{fun=}\StringTok{"rms"}\NormalTok{,}\AttributeTok{na.rm=}\ConstantTok{TRUE}\NormalTok{)}
\NormalTok{merogots}\OtherTok{=}\NormalTok{centrets}\SpecialCharTok{/}\NormalTok{standartnovirze[,}\DecValTok{1}\NormalTok{]}
\FunctionTok{writeRaster}\NormalTok{(merogots,}
      \AttributeTok{filename=}\NormalTok{saglabasanas\_cels,}
      \AttributeTok{overwrite=}\ConstantTok{TRUE}\NormalTok{)}
\end{Highlighting}
\end{Shaded}

\section{General\_AllotmentGardens\_r3000}\label{ch06.406}

\textbf{filename:} \texttt{General\_AllotmentGardens\_r3000.tif}

\textbf{layername:} \texttt{egv\_406}

\textbf{English name:} Fractional cover of Allotment gardens within the 3 km
landscape

\textbf{Latvian name:} Vasarnīcu kompleksu platības īpatsvars 3 km ainavā

\textbf{Procedure:} The cover fraction within a radius of 3000 m around the analysis grid cell
is calculated as the area-weighted sum of the \hyperref[ch06.403]{analysis cells} inside
the buffer, using the workflow \texttt{egvtools::radius\_function()}. During the calculation of the landscape
metric, inverse distance weighted (power = 2) gap filling on the output is
applied to ensure no missing values at the edges. Then the layer is
rewritten to set its name. Finally, the layer is standardised by
subtracting the arithmetic mean and dividing by the root mean squared error.

\begin{Shaded}
\begin{Highlighting}[]
\CommentTok{\# libs {-}{-}{-}{-}}
\ControlFlowTok{if}\NormalTok{(}\SpecialCharTok{!}\FunctionTok{require}\NormalTok{(terra)) \{}\FunctionTok{install.packages}\NormalTok{(}\StringTok{"terra"}\NormalTok{); }\FunctionTok{require}\NormalTok{(terra)\}}
\ControlFlowTok{if}\NormalTok{(}\SpecialCharTok{!}\FunctionTok{require}\NormalTok{(egvtools)) \{remotes}\SpecialCharTok{::}\FunctionTok{install\_github}\NormalTok{(}\StringTok{"aavotins/egvtools"}\NormalTok{); }\FunctionTok{require}\NormalTok{(egvtools)\}}


\CommentTok{\# Templates {-}{-}{-}{-}{-}}
\NormalTok{template100}\OtherTok{=}\FunctionTok{rast}\NormalTok{(}\StringTok{"./Templates/TemplateRasters/LV100m\_10km.tif"}\NormalTok{)}

\CommentTok{\# radii {-}{-}{-}{-}}
\FunctionTok{radius\_function}\NormalTok{(}
 \AttributeTok{kvadrati\_path =} \StringTok{"./Templates/TemplateGrids/tiles/"}\NormalTok{,}
 \AttributeTok{radii\_path   =} \StringTok{"./Templates/TemplateGridPoints/tiles/"}\NormalTok{,}
 \AttributeTok{tikls100\_path =} \StringTok{"./Templates/TemplateGrids/tikls100\_sauzeme.parquet"}\NormalTok{,}
 \AttributeTok{template\_path =} \StringTok{"./Templates/TemplateRasters/LV100m\_10km.tif"}\NormalTok{,}
 \AttributeTok{input\_layers  =} \FunctionTok{c}\NormalTok{(}\StringTok{"./RasterGrids\_100m/2024/RAW/General\_AllotmentGardens\_cell.tif"}\NormalTok{),}
 \AttributeTok{layer\_prefixes =} \FunctionTok{c}\NormalTok{(}\StringTok{"General\_AllotmentGardens"}\NormalTok{),}
 \AttributeTok{output\_dir   =} \StringTok{"./RasterGrids\_100m/2024/RAW/"}\NormalTok{,}
 \AttributeTok{n\_workers   =} \DecValTok{6}\NormalTok{,}
 \AttributeTok{radii     =} \FunctionTok{c}\NormalTok{(}\StringTok{"r3000"}\NormalTok{),}
 \AttributeTok{radius\_mode  =} \StringTok{"sparse"}\NormalTok{,}
 \AttributeTok{extract\_fun  =} \StringTok{"mean"}\NormalTok{,}
 \AttributeTok{fill\_missing  =} \ConstantTok{TRUE}\NormalTok{,}
 \AttributeTok{IDW\_weight   =} \DecValTok{2}\NormalTok{,}
 \AttributeTok{future\_max\_size =} \DecValTok{40} \SpecialCharTok{*} \DecValTok{1024}\SpecialCharTok{\^{}}\DecValTok{3}\NormalTok{)}


\CommentTok{\# General\_AllotmentGardens\_r3000.tif    egv\_406}
\NormalTok{slanis}\OtherTok{=}\FunctionTok{rast}\NormalTok{(}\StringTok{"./RasterGrids\_100m/2024/RAW/General\_AllotmentGardens\_r3000.tif"}\NormalTok{)}
\FunctionTok{names}\NormalTok{(slanis)}\OtherTok{=}\StringTok{"egv\_406"}
\NormalTok{slanis2}\OtherTok{=}\FunctionTok{project}\NormalTok{(slanis,template100)}
\FunctionTok{writeRaster}\NormalTok{(slanis2,}
      \StringTok{"./RasterGrids\_100m/2024/RAW/General\_AllotmentGardens\_r3000.tif"}\NormalTok{,}
      \AttributeTok{overwrite=}\ConstantTok{TRUE}\NormalTok{)}

\CommentTok{\# standardisation {-}{-}{-}{-}}
\ControlFlowTok{if}\NormalTok{(}\SpecialCharTok{!}\FunctionTok{require}\NormalTok{(terra)) \{}\FunctionTok{install.packages}\NormalTok{(}\StringTok{"terra"}\NormalTok{); }\FunctionTok{require}\NormalTok{(terra)\}}
\ControlFlowTok{if}\NormalTok{(}\SpecialCharTok{!}\FunctionTok{require}\NormalTok{(tidyverse)) \{}\FunctionTok{install.packages}\NormalTok{(}\StringTok{"tidyverse"}\NormalTok{); }\FunctionTok{require}\NormalTok{(tidyverse)\}}

\NormalTok{nosaukums}\OtherTok{=}\StringTok{"General\_AllotmentGardens\_r3000.tif"}
\NormalTok{ielasisanas\_cels}\OtherTok{=}\FunctionTok{paste0}\NormalTok{(}\StringTok{"./RasterGrids\_100m/2024/RAW/"}\NormalTok{,nosaukums)}
\NormalTok{saglabasanas\_cels}\OtherTok{=}\FunctionTok{paste0}\NormalTok{(}\StringTok{"./RasterGrids\_100m/2024/Scaled/"}\NormalTok{,nosaukums)}
\NormalTok{slanis}\OtherTok{=}\FunctionTok{rast}\NormalTok{(ielasisanas\_cels)}
\NormalTok{videjais}\OtherTok{=}\FunctionTok{global}\NormalTok{(slanis,}\AttributeTok{fun=}\StringTok{"mean"}\NormalTok{,}\AttributeTok{na.rm=}\ConstantTok{TRUE}\NormalTok{)}
\NormalTok{centrets}\OtherTok{=}\NormalTok{slanis}\SpecialCharTok{{-}}\NormalTok{videjais[,}\DecValTok{1}\NormalTok{]}
\NormalTok{standartnovirze}\OtherTok{=}\NormalTok{terra}\SpecialCharTok{::}\FunctionTok{global}\NormalTok{(centrets,}\AttributeTok{fun=}\StringTok{"rms"}\NormalTok{,}\AttributeTok{na.rm=}\ConstantTok{TRUE}\NormalTok{)}
\NormalTok{merogots}\OtherTok{=}\NormalTok{centrets}\SpecialCharTok{/}\NormalTok{standartnovirze[,}\DecValTok{1}\NormalTok{]}
\FunctionTok{writeRaster}\NormalTok{(merogots,}
      \AttributeTok{filename=}\NormalTok{saglabasanas\_cels,}
      \AttributeTok{overwrite=}\ConstantTok{TRUE}\NormalTok{)}
\end{Highlighting}
\end{Shaded}

\section{General\_AllotmentGardens\_r10000}\label{ch06.407}

\textbf{filename:} \texttt{General\_AllotmentGardens\_r10000.tif}

\textbf{layername:} \texttt{egv\_407}

\textbf{English name:} Fractional cover of Allotment gardens within the 10 km
landscape

\textbf{Latvian name:} Vasarnīcu kompleksu platības īpatsvars 10 km ainavā

\textbf{Procedure:} The cover fraction within a radius of 10000 m around the analysis grid cell
is calculated as the area-weighted sum of the \hyperref[ch06.403]{analysis cells} inside
the buffer, using the workflow \texttt{egvtools::radius\_function()}. During the calculation of the landscape
metric, inverse distance weighted (power = 2) gap filling on the output is
applied to ensure no missing values at the edges. Then the layer is
rewritten to set its name. Finally, the layer is standardised by
subtracting the arithmetic mean and dividing by the root mean squared error.

\begin{Shaded}
\begin{Highlighting}[]
\CommentTok{\# libs {-}{-}{-}{-}}
\ControlFlowTok{if}\NormalTok{(}\SpecialCharTok{!}\FunctionTok{require}\NormalTok{(terra)) \{}\FunctionTok{install.packages}\NormalTok{(}\StringTok{"terra"}\NormalTok{); }\FunctionTok{require}\NormalTok{(terra)\}}
\ControlFlowTok{if}\NormalTok{(}\SpecialCharTok{!}\FunctionTok{require}\NormalTok{(egvtools)) \{remotes}\SpecialCharTok{::}\FunctionTok{install\_github}\NormalTok{(}\StringTok{"aavotins/egvtools"}\NormalTok{); }\FunctionTok{require}\NormalTok{(egvtools)\}}


\CommentTok{\# Templates {-}{-}{-}{-}{-}}
\NormalTok{template100}\OtherTok{=}\FunctionTok{rast}\NormalTok{(}\StringTok{"./Templates/TemplateRasters/LV100m\_10km.tif"}\NormalTok{)}

\CommentTok{\# radii {-}{-}{-}{-}}
\FunctionTok{radius\_function}\NormalTok{(}
 \AttributeTok{kvadrati\_path =} \StringTok{"./Templates/TemplateGrids/tiles/"}\NormalTok{,}
 \AttributeTok{radii\_path   =} \StringTok{"./Templates/TemplateGridPoints/tiles/"}\NormalTok{,}
 \AttributeTok{tikls100\_path =} \StringTok{"./Templates/TemplateGrids/tikls100\_sauzeme.parquet"}\NormalTok{,}
 \AttributeTok{template\_path =} \StringTok{"./Templates/TemplateRasters/LV100m\_10km.tif"}\NormalTok{,}
 \AttributeTok{input\_layers  =} \FunctionTok{c}\NormalTok{(}\StringTok{"./RasterGrids\_100m/2024/RAW/General\_AllotmentGardens\_cell.tif"}\NormalTok{),}
 \AttributeTok{layer\_prefixes =} \FunctionTok{c}\NormalTok{(}\StringTok{"General\_AllotmentGardens"}\NormalTok{),}
 \AttributeTok{output\_dir   =} \StringTok{"./RasterGrids\_100m/2024/RAW/"}\NormalTok{,}
 \AttributeTok{n\_workers   =} \DecValTok{6}\NormalTok{,}
 \AttributeTok{radii     =} \FunctionTok{c}\NormalTok{(}\StringTok{"r10000"}\NormalTok{),}
 \AttributeTok{radius\_mode  =} \StringTok{"sparse"}\NormalTok{,}
 \AttributeTok{extract\_fun  =} \StringTok{"mean"}\NormalTok{,}
 \AttributeTok{fill\_missing  =} \ConstantTok{TRUE}\NormalTok{,}
 \AttributeTok{IDW\_weight   =} \DecValTok{2}\NormalTok{,}
 \AttributeTok{future\_max\_size =} \DecValTok{40} \SpecialCharTok{*} \DecValTok{1024}\SpecialCharTok{\^{}}\DecValTok{3}\NormalTok{)}


\CommentTok{\# General\_AllotmentGardens\_r10000.tif   egv\_407}
\NormalTok{slanis}\OtherTok{=}\FunctionTok{rast}\NormalTok{(}\StringTok{"./RasterGrids\_100m/2024/RAW/General\_AllotmentGardens\_r10000.tif"}\NormalTok{)}
\FunctionTok{names}\NormalTok{(slanis)}\OtherTok{=}\StringTok{"egv\_407"}
\NormalTok{slanis2}\OtherTok{=}\FunctionTok{project}\NormalTok{(slanis,template100)}
\FunctionTok{writeRaster}\NormalTok{(slanis2,}
      \StringTok{"./RasterGrids\_100m/2024/RAW/General\_AllotmentGardens\_r10000.tif"}\NormalTok{,}
      \AttributeTok{overwrite=}\ConstantTok{TRUE}\NormalTok{)}

\CommentTok{\# standardisation {-}{-}{-}{-}}
\ControlFlowTok{if}\NormalTok{(}\SpecialCharTok{!}\FunctionTok{require}\NormalTok{(terra)) \{}\FunctionTok{install.packages}\NormalTok{(}\StringTok{"terra"}\NormalTok{); }\FunctionTok{require}\NormalTok{(terra)\}}
\ControlFlowTok{if}\NormalTok{(}\SpecialCharTok{!}\FunctionTok{require}\NormalTok{(tidyverse)) \{}\FunctionTok{install.packages}\NormalTok{(}\StringTok{"tidyverse"}\NormalTok{); }\FunctionTok{require}\NormalTok{(tidyverse)\}}

\NormalTok{nosaukums}\OtherTok{=}\StringTok{"General\_AllotmentGardens\_r10000.tif"}
\NormalTok{ielasisanas\_cels}\OtherTok{=}\FunctionTok{paste0}\NormalTok{(}\StringTok{"./RasterGrids\_100m/2024/RAW/"}\NormalTok{,nosaukums)}
\NormalTok{saglabasanas\_cels}\OtherTok{=}\FunctionTok{paste0}\NormalTok{(}\StringTok{"./RasterGrids\_100m/2024/Scaled/"}\NormalTok{,nosaukums)}
\NormalTok{slanis}\OtherTok{=}\FunctionTok{rast}\NormalTok{(ielasisanas\_cels)}
\NormalTok{videjais}\OtherTok{=}\FunctionTok{global}\NormalTok{(slanis,}\AttributeTok{fun=}\StringTok{"mean"}\NormalTok{,}\AttributeTok{na.rm=}\ConstantTok{TRUE}\NormalTok{)}
\NormalTok{centrets}\OtherTok{=}\NormalTok{slanis}\SpecialCharTok{{-}}\NormalTok{videjais[,}\DecValTok{1}\NormalTok{]}
\NormalTok{standartnovirze}\OtherTok{=}\NormalTok{terra}\SpecialCharTok{::}\FunctionTok{global}\NormalTok{(centrets,}\AttributeTok{fun=}\StringTok{"rms"}\NormalTok{,}\AttributeTok{na.rm=}\ConstantTok{TRUE}\NormalTok{)}
\NormalTok{merogots}\OtherTok{=}\NormalTok{centrets}\SpecialCharTok{/}\NormalTok{standartnovirze[,}\DecValTok{1}\NormalTok{]}
\FunctionTok{writeRaster}\NormalTok{(merogots,}
      \AttributeTok{filename=}\NormalTok{saglabasanas\_cels,}
      \AttributeTok{overwrite=}\ConstantTok{TRUE}\NormalTok{)}
\end{Highlighting}
\end{Shaded}

\section{General\_BareSoilQuarry\_cell}\label{ch06.408}

\textbf{filename:} \texttt{General\_BareSoilQuarry\_cell.tif}

\textbf{layername:} \texttt{egv\_408}

\textbf{English name:} Fractional cover of areas with Bare Soil, Quarries within the
analysis cell (1 ha)

\textbf{Latvian name:} Atklātas augsnes un karjeru platības īpatsvars analīzes šūnā
(1 ha)

\textbf{Procedure:} First, the bare soil and quarry areas from the \hyperref[Ch05.03]{Landscape
classification} are selected (value 800 is reclassified to value 1;
all others are set to 0). The resulting layer
is then aggregated to EGV resolution using the workflow \texttt{egvtools::input2egv()}, which
calculates the arithmetic mean to determine the cover fraction. During
aggregation, inverse distance weighted (power = 2) gap filling on the output is
applied to ensure no missing values at the edges. Finally, the layer is
standardised by subtracting the arithmetic mean and dividing by the root mean squared
error.

\begin{Shaded}
\begin{Highlighting}[]
\CommentTok{\# libs {-}{-}{-}{-}}
\ControlFlowTok{if}\NormalTok{(}\SpecialCharTok{!}\FunctionTok{require}\NormalTok{(egvtools)) \{remotes}\SpecialCharTok{::}\FunctionTok{install\_github}\NormalTok{(}\StringTok{"aavotins/egvtools"}\NormalTok{); }\FunctionTok{require}\NormalTok{(egvtools)\}}
\ControlFlowTok{if}\NormalTok{(}\SpecialCharTok{!}\FunctionTok{require}\NormalTok{(terra)) \{}\FunctionTok{install.packages}\NormalTok{(}\StringTok{"terra"}\NormalTok{); }\FunctionTok{require}\NormalTok{(terra)\}}
\ControlFlowTok{if}\NormalTok{(}\SpecialCharTok{!}\FunctionTok{require}\NormalTok{(sf)) \{}\FunctionTok{install.packages}\NormalTok{(}\StringTok{"sf"}\NormalTok{); }\FunctionTok{require}\NormalTok{(sf)\}}
\ControlFlowTok{if}\NormalTok{(}\SpecialCharTok{!}\FunctionTok{require}\NormalTok{(tidyverse)) \{}\FunctionTok{install.packages}\NormalTok{(}\StringTok{"tidyverse"}\NormalTok{); }\FunctionTok{require}\NormalTok{(tidyverse)\}}
\ControlFlowTok{if}\NormalTok{(}\SpecialCharTok{!}\FunctionTok{require}\NormalTok{(sfarrow)) \{}\FunctionTok{install.packages}\NormalTok{(}\StringTok{"sfarrow"}\NormalTok{); }\FunctionTok{require}\NormalTok{(sfarrow)\}}
\ControlFlowTok{if}\NormalTok{(}\SpecialCharTok{!}\FunctionTok{require}\NormalTok{(readxl)) \{}\FunctionTok{install.packages}\NormalTok{(}\StringTok{"readxl"}\NormalTok{); }\FunctionTok{require}\NormalTok{(readxl)\}}
\ControlFlowTok{if}\NormalTok{(}\SpecialCharTok{!}\FunctionTok{require}\NormalTok{(raster)) \{}\FunctionTok{install.packages}\NormalTok{(}\StringTok{"raster"}\NormalTok{); }\FunctionTok{require}\NormalTok{(raster)\}}
\ControlFlowTok{if}\NormalTok{(}\SpecialCharTok{!}\FunctionTok{require}\NormalTok{(fasterize)) \{}\FunctionTok{install.packages}\NormalTok{(}\StringTok{"fasterize"}\NormalTok{); }\FunctionTok{require}\NormalTok{(fasterize)\}}

\CommentTok{\# templates {-}{-}{-}{-}}
\NormalTok{template100}\OtherTok{=}\FunctionTok{rast}\NormalTok{(}\StringTok{"./Templates/TemplateRasters/LV100m\_10km.tif"}\NormalTok{)}
\NormalTok{template10}\OtherTok{=}\FunctionTok{rast}\NormalTok{(}\StringTok{"./Templates/TemplateRasters/LV10m\_10km.tif"}\NormalTok{)}
\NormalTok{rastrs10}\OtherTok{=}\FunctionTok{raster}\NormalTok{(template10)}

\NormalTok{nulls10}\OtherTok{=}\FunctionTok{rast}\NormalTok{(}\StringTok{"./Templates/TemplateRasters/nulls\_LV10m\_10km.tif"}\NormalTok{)}
\NormalTok{nulls100}\OtherTok{=}\FunctionTok{rast}\NormalTok{(}\StringTok{"./Templates/TemplateRasters/nulls\_LV100m\_10km.tif"}\NormalTok{)}

\CommentTok{\# simple landscape {-}{-}{-}{-}}
\NormalTok{simple\_landscape}\OtherTok{=}\FunctionTok{rast}\NormalTok{(}\StringTok{"RasterGrids\_10m/2024/Ainava\_vienk\_mask.tif"}\NormalTok{)}


\CommentTok{\# General\_BareSoilQuarry\_cell.tif   egv\_408 {-}{-}{-}{-}}
\NormalTok{baresoilquerry}\OtherTok{=}\FunctionTok{ifel}\NormalTok{(simple\_landscape}\SpecialCharTok{==}\DecValTok{800}\NormalTok{,}\DecValTok{1}\NormalTok{,}\DecValTok{0}\NormalTok{)}
\NormalTok{i2e\_rez}\OtherTok{=}\NormalTok{egvtools}\SpecialCharTok{::}\FunctionTok{input2egv}\NormalTok{(}\AttributeTok{input=}\NormalTok{baresoilquerry,}
              \AttributeTok{egv\_template=} \StringTok{"./Templates/TemplateRasters/LV100m\_10km.tif"}\NormalTok{,}
              \AttributeTok{summary\_function =} \StringTok{"average"}\NormalTok{,}
              \AttributeTok{missing\_job =} \StringTok{"FillOutput"}\NormalTok{,}
              \AttributeTok{outlocation =} \StringTok{"./RasterGrids\_100m/2024/RAW/"}\NormalTok{,}
              \AttributeTok{outfilename =} \StringTok{"General\_BareSoilQuarry\_cell.tif"}\NormalTok{,}
              \AttributeTok{layername =} \StringTok{"egv\_408"}\NormalTok{,}
              \AttributeTok{idw\_weight =} \DecValTok{2}\NormalTok{,}
              \AttributeTok{plot\_gaps =} \ConstantTok{FALSE}\NormalTok{,}\AttributeTok{plot\_final =} \ConstantTok{TRUE}\NormalTok{)}
\NormalTok{i2e\_rez}
\FunctionTok{rm}\NormalTok{(baresoilquerry)}
\FunctionTok{rm}\NormalTok{(i2e\_rez)}

\CommentTok{\# standardisation {-}{-}{-}{-}}
\ControlFlowTok{if}\NormalTok{(}\SpecialCharTok{!}\FunctionTok{require}\NormalTok{(terra)) \{}\FunctionTok{install.packages}\NormalTok{(}\StringTok{"terra"}\NormalTok{); }\FunctionTok{require}\NormalTok{(terra)\}}
\ControlFlowTok{if}\NormalTok{(}\SpecialCharTok{!}\FunctionTok{require}\NormalTok{(tidyverse)) \{}\FunctionTok{install.packages}\NormalTok{(}\StringTok{"tidyverse"}\NormalTok{); }\FunctionTok{require}\NormalTok{(tidyverse)\}}

\NormalTok{nosaukums}\OtherTok{=}\StringTok{"General\_BareSoilQuarry\_cell.tif"}
\NormalTok{ielasisanas\_cels}\OtherTok{=}\FunctionTok{paste0}\NormalTok{(}\StringTok{"./RasterGrids\_100m/2024/RAW/"}\NormalTok{,nosaukums)}
\NormalTok{saglabasanas\_cels}\OtherTok{=}\FunctionTok{paste0}\NormalTok{(}\StringTok{"./RasterGrids\_100m/2024/Scaled/"}\NormalTok{,nosaukums)}
\NormalTok{slanis}\OtherTok{=}\FunctionTok{rast}\NormalTok{(ielasisanas\_cels)}
\NormalTok{videjais}\OtherTok{=}\FunctionTok{global}\NormalTok{(slanis,}\AttributeTok{fun=}\StringTok{"mean"}\NormalTok{,}\AttributeTok{na.rm=}\ConstantTok{TRUE}\NormalTok{)}
\NormalTok{centrets}\OtherTok{=}\NormalTok{slanis}\SpecialCharTok{{-}}\NormalTok{videjais[,}\DecValTok{1}\NormalTok{]}
\NormalTok{standartnovirze}\OtherTok{=}\NormalTok{terra}\SpecialCharTok{::}\FunctionTok{global}\NormalTok{(centrets,}\AttributeTok{fun=}\StringTok{"rms"}\NormalTok{,}\AttributeTok{na.rm=}\ConstantTok{TRUE}\NormalTok{)}
\NormalTok{merogots}\OtherTok{=}\NormalTok{centrets}\SpecialCharTok{/}\NormalTok{standartnovirze[,}\DecValTok{1}\NormalTok{]}
\FunctionTok{writeRaster}\NormalTok{(merogots,}
      \AttributeTok{filename=}\NormalTok{saglabasanas\_cels,}
      \AttributeTok{overwrite=}\ConstantTok{TRUE}\NormalTok{)}
\end{Highlighting}
\end{Shaded}

\section{General\_BareSoilQuarry\_r500}\label{ch06.409}

\textbf{filename:} \texttt{General\_BareSoilQuarry\_r500.tif}

\textbf{layername:} \texttt{egv\_409}

\textbf{English name:} Fractional cover of areas with Bare Soil, Quarries within the
0.5 km landscape

\textbf{Latvian name:} Atklātas augsnes un karjeru platības īpatsvars 0,5 km ainavā

\textbf{Procedure:} The cover fraction within a radius of 500 m around the analysis grid cell is
calculated as the area-weighted sum of the \hyperref[ch06.408]{analysis cells} inside the
buffer, using the workflow \texttt{egvtools::radius\_function()}. During the calculation of the landscape metric,
inverse distance weighted (power = 2) gap filling on the output is applied
to ensure no missing values at the edges. Then the layer is rewritten to set
its name. Finally, the layer is standardised by subtracting the arithmetic
mean and dividing by the root mean squared error.

\begin{Shaded}
\begin{Highlighting}[]
\CommentTok{\# libs {-}{-}{-}{-}}
\ControlFlowTok{if}\NormalTok{(}\SpecialCharTok{!}\FunctionTok{require}\NormalTok{(terra)) \{}\FunctionTok{install.packages}\NormalTok{(}\StringTok{"terra"}\NormalTok{); }\FunctionTok{require}\NormalTok{(terra)\}}
\ControlFlowTok{if}\NormalTok{(}\SpecialCharTok{!}\FunctionTok{require}\NormalTok{(egvtools)) \{remotes}\SpecialCharTok{::}\FunctionTok{install\_github}\NormalTok{(}\StringTok{"aavotins/egvtools"}\NormalTok{); }\FunctionTok{require}\NormalTok{(egvtools)\}}


\CommentTok{\# Templates {-}{-}{-}{-}{-}}
\NormalTok{template100}\OtherTok{=}\FunctionTok{rast}\NormalTok{(}\StringTok{"./Templates/TemplateRasters/LV100m\_10km.tif"}\NormalTok{)}

\CommentTok{\# radii {-}{-}{-}{-}}
\FunctionTok{radius\_function}\NormalTok{(}
 \AttributeTok{kvadrati\_path =} \StringTok{"./Templates/TemplateGrids/tiles/"}\NormalTok{,}
 \AttributeTok{radii\_path   =} \StringTok{"./Templates/TemplateGridPoints/tiles/"}\NormalTok{,}
 \AttributeTok{tikls100\_path =} \StringTok{"./Templates/TemplateGrids/tikls100\_sauzeme.parquet"}\NormalTok{,}
 \AttributeTok{template\_path =} \StringTok{"./Templates/TemplateRasters/LV100m\_10km.tif"}\NormalTok{,}
 \AttributeTok{input\_layers  =} \FunctionTok{c}\NormalTok{(}\StringTok{"./RasterGrids\_100m/2024/RAW/General\_BareSoilQuarry\_cell.tif"}\NormalTok{),}
 \AttributeTok{layer\_prefixes =} \FunctionTok{c}\NormalTok{(}\StringTok{"General\_BareSoilQuarry"}\NormalTok{),}
 \AttributeTok{output\_dir   =} \StringTok{"./RasterGrids\_100m/2024/RAW/"}\NormalTok{,}
 \AttributeTok{n\_workers   =} \DecValTok{6}\NormalTok{,}
 \AttributeTok{radii     =} \FunctionTok{c}\NormalTok{(}\StringTok{"r500"}\NormalTok{),}
 \AttributeTok{radius\_mode  =} \StringTok{"sparse"}\NormalTok{,}
 \AttributeTok{extract\_fun  =} \StringTok{"mean"}\NormalTok{,}
 \AttributeTok{fill\_missing  =} \ConstantTok{TRUE}\NormalTok{,}
 \AttributeTok{IDW\_weight   =} \DecValTok{2}\NormalTok{,}
 \AttributeTok{future\_max\_size =} \DecValTok{40} \SpecialCharTok{*} \DecValTok{1024}\SpecialCharTok{\^{}}\DecValTok{3}\NormalTok{)}


\CommentTok{\# General\_BareSoilQuarry\_r500.tif   egv\_409}
\NormalTok{slanis}\OtherTok{=}\FunctionTok{rast}\NormalTok{(}\StringTok{"./RasterGrids\_100m/2024/RAW/General\_BareSoilQuarry\_r500.tif"}\NormalTok{)}
\FunctionTok{names}\NormalTok{(slanis)}\OtherTok{=}\StringTok{"egv\_409"}
\NormalTok{slanis2}\OtherTok{=}\FunctionTok{project}\NormalTok{(slanis,template100)}
\FunctionTok{writeRaster}\NormalTok{(slanis2,}
      \StringTok{"./RasterGrids\_100m/2024/RAW/General\_BareSoilQuarry\_r500.tif"}\NormalTok{,}
      \AttributeTok{overwrite=}\ConstantTok{TRUE}\NormalTok{)}

\CommentTok{\# standardisation {-}{-}{-}{-}}
\ControlFlowTok{if}\NormalTok{(}\SpecialCharTok{!}\FunctionTok{require}\NormalTok{(terra)) \{}\FunctionTok{install.packages}\NormalTok{(}\StringTok{"terra"}\NormalTok{); }\FunctionTok{require}\NormalTok{(terra)\}}
\ControlFlowTok{if}\NormalTok{(}\SpecialCharTok{!}\FunctionTok{require}\NormalTok{(tidyverse)) \{}\FunctionTok{install.packages}\NormalTok{(}\StringTok{"tidyverse"}\NormalTok{); }\FunctionTok{require}\NormalTok{(tidyverse)\}}

\NormalTok{nosaukums}\OtherTok{=}\StringTok{"General\_BareSoilQuarry\_r500.tif"}
\NormalTok{ielasisanas\_cels}\OtherTok{=}\FunctionTok{paste0}\NormalTok{(}\StringTok{"./RasterGrids\_100m/2024/RAW/"}\NormalTok{,nosaukums)}
\NormalTok{saglabasanas\_cels}\OtherTok{=}\FunctionTok{paste0}\NormalTok{(}\StringTok{"./RasterGrids\_100m/2024/Scaled/"}\NormalTok{,nosaukums)}
\NormalTok{slanis}\OtherTok{=}\FunctionTok{rast}\NormalTok{(ielasisanas\_cels)}
\NormalTok{videjais}\OtherTok{=}\FunctionTok{global}\NormalTok{(slanis,}\AttributeTok{fun=}\StringTok{"mean"}\NormalTok{,}\AttributeTok{na.rm=}\ConstantTok{TRUE}\NormalTok{)}
\NormalTok{centrets}\OtherTok{=}\NormalTok{slanis}\SpecialCharTok{{-}}\NormalTok{videjais[,}\DecValTok{1}\NormalTok{]}
\NormalTok{standartnovirze}\OtherTok{=}\NormalTok{terra}\SpecialCharTok{::}\FunctionTok{global}\NormalTok{(centrets,}\AttributeTok{fun=}\StringTok{"rms"}\NormalTok{,}\AttributeTok{na.rm=}\ConstantTok{TRUE}\NormalTok{)}
\NormalTok{merogots}\OtherTok{=}\NormalTok{centrets}\SpecialCharTok{/}\NormalTok{standartnovirze[,}\DecValTok{1}\NormalTok{]}
\FunctionTok{writeRaster}\NormalTok{(merogots,}
      \AttributeTok{filename=}\NormalTok{saglabasanas\_cels,}
      \AttributeTok{overwrite=}\ConstantTok{TRUE}\NormalTok{)}
\end{Highlighting}
\end{Shaded}

\section{General\_BareSoilQuarry\_r1250}\label{ch06.410}

\textbf{filename:} \texttt{General\_BareSoilQuarry\_r1250.tif}

\textbf{layername:} \texttt{egv\_410}

\textbf{English name:} Fractional cover of areas with Bare Soil, Quarries within the
1.25 km landscape

\textbf{Latvian name:} Atklātas augsnes un karjeru platības īpatsvars 1,25 km ainavā

\textbf{Procedure:} The cover fraction within a radius of 1250 m around the analysis grid cell
is calculated as the area-weighted sum of the \hyperref[ch06.408]{analysis cells} inside
the buffer, using the workflow \texttt{egvtools::radius\_function()}. During the calculation of the landscape
metric, inverse distance weighted (power = 2) gap filling on the output is
applied to ensure no missing values at the edges. Then the layer is
rewritten to set its name. Finally, the layer is standardised by
subtracting the arithmetic mean and dividing by the root mean squared error.

\begin{Shaded}
\begin{Highlighting}[]
\CommentTok{\# libs {-}{-}{-}{-}}
\ControlFlowTok{if}\NormalTok{(}\SpecialCharTok{!}\FunctionTok{require}\NormalTok{(terra)) \{}\FunctionTok{install.packages}\NormalTok{(}\StringTok{"terra"}\NormalTok{); }\FunctionTok{require}\NormalTok{(terra)\}}
\ControlFlowTok{if}\NormalTok{(}\SpecialCharTok{!}\FunctionTok{require}\NormalTok{(egvtools)) \{remotes}\SpecialCharTok{::}\FunctionTok{install\_github}\NormalTok{(}\StringTok{"aavotins/egvtools"}\NormalTok{); }\FunctionTok{require}\NormalTok{(egvtools)\}}


\CommentTok{\# Templates {-}{-}{-}{-}{-}}
\NormalTok{template100}\OtherTok{=}\FunctionTok{rast}\NormalTok{(}\StringTok{"./Templates/TemplateRasters/LV100m\_10km.tif"}\NormalTok{)}

\CommentTok{\# radii {-}{-}{-}{-}}
\FunctionTok{radius\_function}\NormalTok{(}
 \AttributeTok{kvadrati\_path =} \StringTok{"./Templates/TemplateGrids/tiles/"}\NormalTok{,}
 \AttributeTok{radii\_path   =} \StringTok{"./Templates/TemplateGridPoints/tiles/"}\NormalTok{,}
 \AttributeTok{tikls100\_path =} \StringTok{"./Templates/TemplateGrids/tikls100\_sauzeme.parquet"}\NormalTok{,}
 \AttributeTok{template\_path =} \StringTok{"./Templates/TemplateRasters/LV100m\_10km.tif"}\NormalTok{,}
 \AttributeTok{input\_layers  =} \FunctionTok{c}\NormalTok{(}\StringTok{"./RasterGrids\_100m/2024/RAW/General\_BareSoilQuarry\_cell.tif"}\NormalTok{),}
 \AttributeTok{layer\_prefixes =} \FunctionTok{c}\NormalTok{(}\StringTok{"General\_BareSoilQuarry"}\NormalTok{),}
 \AttributeTok{output\_dir   =} \StringTok{"./RasterGrids\_100m/2024/RAW/"}\NormalTok{,}
 \AttributeTok{n\_workers   =} \DecValTok{6}\NormalTok{,}
 \AttributeTok{radii     =} \FunctionTok{c}\NormalTok{(}\StringTok{"r1250"}\NormalTok{),}
 \AttributeTok{radius\_mode  =} \StringTok{"sparse"}\NormalTok{,}
 \AttributeTok{extract\_fun  =} \StringTok{"mean"}\NormalTok{,}
 \AttributeTok{fill\_missing  =} \ConstantTok{TRUE}\NormalTok{,}
 \AttributeTok{IDW\_weight   =} \DecValTok{2}\NormalTok{,}
 \AttributeTok{future\_max\_size =} \DecValTok{40} \SpecialCharTok{*} \DecValTok{1024}\SpecialCharTok{\^{}}\DecValTok{3}\NormalTok{)}


\CommentTok{\# General\_BareSoilQuarry\_r1250.tif  egv\_410}
\NormalTok{slanis}\OtherTok{=}\FunctionTok{rast}\NormalTok{(}\StringTok{"./RasterGrids\_100m/2024/RAW/General\_BareSoilQuarry\_r1250.tif"}\NormalTok{)}
\FunctionTok{names}\NormalTok{(slanis)}\OtherTok{=}\StringTok{"egv\_410"}
\NormalTok{slanis2}\OtherTok{=}\FunctionTok{project}\NormalTok{(slanis,template100)}
\FunctionTok{writeRaster}\NormalTok{(slanis2,}
      \StringTok{"./RasterGrids\_100m/2024/RAW/General\_BareSoilQuarry\_r1250.tif"}\NormalTok{,}
      \AttributeTok{overwrite=}\ConstantTok{TRUE}\NormalTok{)}

\CommentTok{\# standardisation {-}{-}{-}{-}}
\ControlFlowTok{if}\NormalTok{(}\SpecialCharTok{!}\FunctionTok{require}\NormalTok{(terra)) \{}\FunctionTok{install.packages}\NormalTok{(}\StringTok{"terra"}\NormalTok{); }\FunctionTok{require}\NormalTok{(terra)\}}
\ControlFlowTok{if}\NormalTok{(}\SpecialCharTok{!}\FunctionTok{require}\NormalTok{(tidyverse)) \{}\FunctionTok{install.packages}\NormalTok{(}\StringTok{"tidyverse"}\NormalTok{); }\FunctionTok{require}\NormalTok{(tidyverse)\}}

\NormalTok{nosaukums}\OtherTok{=}\StringTok{"General\_BareSoilQuarry\_r1250.tif"}
\NormalTok{ielasisanas\_cels}\OtherTok{=}\FunctionTok{paste0}\NormalTok{(}\StringTok{"./RasterGrids\_100m/2024/RAW/"}\NormalTok{,nosaukums)}
\NormalTok{saglabasanas\_cels}\OtherTok{=}\FunctionTok{paste0}\NormalTok{(}\StringTok{"./RasterGrids\_100m/2024/Scaled/"}\NormalTok{,nosaukums)}
\NormalTok{slanis}\OtherTok{=}\FunctionTok{rast}\NormalTok{(ielasisanas\_cels)}
\NormalTok{videjais}\OtherTok{=}\FunctionTok{global}\NormalTok{(slanis,}\AttributeTok{fun=}\StringTok{"mean"}\NormalTok{,}\AttributeTok{na.rm=}\ConstantTok{TRUE}\NormalTok{)}
\NormalTok{centrets}\OtherTok{=}\NormalTok{slanis}\SpecialCharTok{{-}}\NormalTok{videjais[,}\DecValTok{1}\NormalTok{]}
\NormalTok{standartnovirze}\OtherTok{=}\NormalTok{terra}\SpecialCharTok{::}\FunctionTok{global}\NormalTok{(centrets,}\AttributeTok{fun=}\StringTok{"rms"}\NormalTok{,}\AttributeTok{na.rm=}\ConstantTok{TRUE}\NormalTok{)}
\NormalTok{merogots}\OtherTok{=}\NormalTok{centrets}\SpecialCharTok{/}\NormalTok{standartnovirze[,}\DecValTok{1}\NormalTok{]}
\FunctionTok{writeRaster}\NormalTok{(merogots,}
      \AttributeTok{filename=}\NormalTok{saglabasanas\_cels,}
      \AttributeTok{overwrite=}\ConstantTok{TRUE}\NormalTok{)}
\end{Highlighting}
\end{Shaded}

\section{General\_BareSoilQuarry\_r3000}\label{ch06.411}

\textbf{filename:} \texttt{General\_BareSoilQuarry\_r3000.tif}

\textbf{layername:} \texttt{egv\_411}

\textbf{English name:} Fractional cover of areas with Bare Soil, Quarries within the
3 km landscape

\textbf{Latvian name:} Atklātas augsnes un karjeru platības īpatsvars 3 km ainavā

\textbf{Procedure:} The cover fraction within a radius of 3000 m around the analysis grid cell
is calculated as the area-weighted sum of the \hyperref[ch06.408]{analysis cells} inside
the buffer, using the workflow \texttt{egvtools::radius\_function()}. During the calculation of the landscape
metric, inverse distance weighted (power = 2) gap filling on the output is
applied to ensure no missing values at the edges. Then the layer is
rewritten to set its name. Finally, the layer is standardised by
subtracting the arithmetic mean and dividing by the root mean squared error.

\begin{Shaded}
\begin{Highlighting}[]
\CommentTok{\# libs {-}{-}{-}{-}}
\ControlFlowTok{if}\NormalTok{(}\SpecialCharTok{!}\FunctionTok{require}\NormalTok{(terra)) \{}\FunctionTok{install.packages}\NormalTok{(}\StringTok{"terra"}\NormalTok{); }\FunctionTok{require}\NormalTok{(terra)\}}
\ControlFlowTok{if}\NormalTok{(}\SpecialCharTok{!}\FunctionTok{require}\NormalTok{(egvtools)) \{remotes}\SpecialCharTok{::}\FunctionTok{install\_github}\NormalTok{(}\StringTok{"aavotins/egvtools"}\NormalTok{); }\FunctionTok{require}\NormalTok{(egvtools)\}}


\CommentTok{\# Templates {-}{-}{-}{-}{-}}
\NormalTok{template100}\OtherTok{=}\FunctionTok{rast}\NormalTok{(}\StringTok{"./Templates/TemplateRasters/LV100m\_10km.tif"}\NormalTok{)}

\CommentTok{\# radii {-}{-}{-}{-}}
\FunctionTok{radius\_function}\NormalTok{(}
 \AttributeTok{kvadrati\_path =} \StringTok{"./Templates/TemplateGrids/tiles/"}\NormalTok{,}
 \AttributeTok{radii\_path   =} \StringTok{"./Templates/TemplateGridPoints/tiles/"}\NormalTok{,}
 \AttributeTok{tikls100\_path =} \StringTok{"./Templates/TemplateGrids/tikls100\_sauzeme.parquet"}\NormalTok{,}
 \AttributeTok{template\_path =} \StringTok{"./Templates/TemplateRasters/LV100m\_10km.tif"}\NormalTok{,}
 \AttributeTok{input\_layers  =} \FunctionTok{c}\NormalTok{(}\StringTok{"./RasterGrids\_100m/2024/RAW/General\_BareSoilQuarry\_cell.tif"}\NormalTok{),}
 \AttributeTok{layer\_prefixes =} \FunctionTok{c}\NormalTok{(}\StringTok{"General\_BareSoilQuarry"}\NormalTok{),}
 \AttributeTok{output\_dir   =} \StringTok{"./RasterGrids\_100m/2024/RAW/"}\NormalTok{,}
 \AttributeTok{n\_workers   =} \DecValTok{6}\NormalTok{,}
 \AttributeTok{radii     =} \FunctionTok{c}\NormalTok{(}\StringTok{"r3000"}\NormalTok{),}
 \AttributeTok{radius\_mode  =} \StringTok{"sparse"}\NormalTok{,}
 \AttributeTok{extract\_fun  =} \StringTok{"mean"}\NormalTok{,}
 \AttributeTok{fill\_missing  =} \ConstantTok{TRUE}\NormalTok{,}
 \AttributeTok{IDW\_weight   =} \DecValTok{2}\NormalTok{,}
 \AttributeTok{future\_max\_size =} \DecValTok{40} \SpecialCharTok{*} \DecValTok{1024}\SpecialCharTok{\^{}}\DecValTok{3}\NormalTok{)}


\CommentTok{\# General\_BareSoilQuarry\_r3000.tif  egv\_411}
\NormalTok{slanis}\OtherTok{=}\FunctionTok{rast}\NormalTok{(}\StringTok{"./RasterGrids\_100m/2024/RAW/General\_BareSoilQuarry\_r3000.tif"}\NormalTok{)}
\FunctionTok{names}\NormalTok{(slanis)}\OtherTok{=}\StringTok{"egv\_411"}
\NormalTok{slanis2}\OtherTok{=}\FunctionTok{project}\NormalTok{(slanis,template100)}
\FunctionTok{writeRaster}\NormalTok{(slanis2,}
      \StringTok{"./RasterGrids\_100m/2024/RAW/General\_BareSoilQuarry\_r3000.tif"}\NormalTok{,}
      \AttributeTok{overwrite=}\ConstantTok{TRUE}\NormalTok{)}

\CommentTok{\# standardisation {-}{-}{-}{-}}
\ControlFlowTok{if}\NormalTok{(}\SpecialCharTok{!}\FunctionTok{require}\NormalTok{(terra)) \{}\FunctionTok{install.packages}\NormalTok{(}\StringTok{"terra"}\NormalTok{); }\FunctionTok{require}\NormalTok{(terra)\}}
\ControlFlowTok{if}\NormalTok{(}\SpecialCharTok{!}\FunctionTok{require}\NormalTok{(tidyverse)) \{}\FunctionTok{install.packages}\NormalTok{(}\StringTok{"tidyverse"}\NormalTok{); }\FunctionTok{require}\NormalTok{(tidyverse)\}}

\NormalTok{nosaukums}\OtherTok{=}\StringTok{"General\_BareSoilQuarry\_r3000.tif"}
\NormalTok{ielasisanas\_cels}\OtherTok{=}\FunctionTok{paste0}\NormalTok{(}\StringTok{"./RasterGrids\_100m/2024/RAW/"}\NormalTok{,nosaukums)}
\NormalTok{saglabasanas\_cels}\OtherTok{=}\FunctionTok{paste0}\NormalTok{(}\StringTok{"./RasterGrids\_100m/2024/Scaled/"}\NormalTok{,nosaukums)}
\NormalTok{slanis}\OtherTok{=}\FunctionTok{rast}\NormalTok{(ielasisanas\_cels)}
\NormalTok{videjais}\OtherTok{=}\FunctionTok{global}\NormalTok{(slanis,}\AttributeTok{fun=}\StringTok{"mean"}\NormalTok{,}\AttributeTok{na.rm=}\ConstantTok{TRUE}\NormalTok{)}
\NormalTok{centrets}\OtherTok{=}\NormalTok{slanis}\SpecialCharTok{{-}}\NormalTok{videjais[,}\DecValTok{1}\NormalTok{]}
\NormalTok{standartnovirze}\OtherTok{=}\NormalTok{terra}\SpecialCharTok{::}\FunctionTok{global}\NormalTok{(centrets,}\AttributeTok{fun=}\StringTok{"rms"}\NormalTok{,}\AttributeTok{na.rm=}\ConstantTok{TRUE}\NormalTok{)}
\NormalTok{merogots}\OtherTok{=}\NormalTok{centrets}\SpecialCharTok{/}\NormalTok{standartnovirze[,}\DecValTok{1}\NormalTok{]}
\FunctionTok{writeRaster}\NormalTok{(merogots,}
      \AttributeTok{filename=}\NormalTok{saglabasanas\_cels,}
      \AttributeTok{overwrite=}\ConstantTok{TRUE}\NormalTok{)}
\end{Highlighting}
\end{Shaded}

\section{General\_BareSoilQuarry\_r10000}\label{ch06.412}

\textbf{filename:} \texttt{General\_BareSoilQuarry\_r10000.tif}

\textbf{layername:} \texttt{egv\_412}

\textbf{English name:} Fractional cover of areas with Bare Soil, Quarries within the
10 km landscape

\textbf{Latvian name:} Atklātas augsnes un karjeru platības īpatsvars 10 km ainavā

\textbf{Procedure:} The cover fraction within a radius of 10000 m around the analysis grid cell
is calculated as the area-weighted sum of the \hyperref[ch06.408]{analysis cells} inside
the buffer, using the workflow \texttt{egvtools::radius\_function()}. During the calculation of the landscape
metric, inverse distance weighted (power = 2) gap filling on the output is
applied to ensure no missing values at the edges. Then the layer is
rewritten to set its name. Finally, the layer is standardised by
subtracting the arithmetic mean and dividing by the root mean squared error.

\begin{Shaded}
\begin{Highlighting}[]
\CommentTok{\# libs {-}{-}{-}{-}}
\ControlFlowTok{if}\NormalTok{(}\SpecialCharTok{!}\FunctionTok{require}\NormalTok{(terra)) \{}\FunctionTok{install.packages}\NormalTok{(}\StringTok{"terra"}\NormalTok{); }\FunctionTok{require}\NormalTok{(terra)\}}
\ControlFlowTok{if}\NormalTok{(}\SpecialCharTok{!}\FunctionTok{require}\NormalTok{(egvtools)) \{remotes}\SpecialCharTok{::}\FunctionTok{install\_github}\NormalTok{(}\StringTok{"aavotins/egvtools"}\NormalTok{); }\FunctionTok{require}\NormalTok{(egvtools)\}}


\CommentTok{\# Templates {-}{-}{-}{-}{-}}
\NormalTok{template100}\OtherTok{=}\FunctionTok{rast}\NormalTok{(}\StringTok{"./Templates/TemplateRasters/LV100m\_10km.tif"}\NormalTok{)}

\CommentTok{\# radii {-}{-}{-}{-}}
\FunctionTok{radius\_function}\NormalTok{(}
 \AttributeTok{kvadrati\_path =} \StringTok{"./Templates/TemplateGrids/tiles/"}\NormalTok{,}
 \AttributeTok{radii\_path   =} \StringTok{"./Templates/TemplateGridPoints/tiles/"}\NormalTok{,}
 \AttributeTok{tikls100\_path =} \StringTok{"./Templates/TemplateGrids/tikls100\_sauzeme.parquet"}\NormalTok{,}
 \AttributeTok{template\_path =} \StringTok{"./Templates/TemplateRasters/LV100m\_10km.tif"}\NormalTok{,}
 \AttributeTok{input\_layers  =} \FunctionTok{c}\NormalTok{(}\StringTok{"./RasterGrids\_100m/2024/RAW/General\_BareSoilQuarry\_cell.tif"}\NormalTok{),}
 \AttributeTok{layer\_prefixes =} \FunctionTok{c}\NormalTok{(}\StringTok{"General\_BareSoilQuarry"}\NormalTok{),}
 \AttributeTok{output\_dir   =} \StringTok{"./RasterGrids\_100m/2024/RAW/"}\NormalTok{,}
 \AttributeTok{n\_workers   =} \DecValTok{6}\NormalTok{,}
 \AttributeTok{radii     =} \FunctionTok{c}\NormalTok{(}\StringTok{"r10000"}\NormalTok{),}
 \AttributeTok{radius\_mode  =} \StringTok{"sparse"}\NormalTok{,}
 \AttributeTok{extract\_fun  =} \StringTok{"mean"}\NormalTok{,}
 \AttributeTok{fill\_missing  =} \ConstantTok{TRUE}\NormalTok{,}
 \AttributeTok{IDW\_weight   =} \DecValTok{2}\NormalTok{,}
 \AttributeTok{future\_max\_size =} \DecValTok{40} \SpecialCharTok{*} \DecValTok{1024}\SpecialCharTok{\^{}}\DecValTok{3}\NormalTok{)}


\CommentTok{\# General\_BareSoilQuarry\_r10000.tif egv\_412}
\NormalTok{slanis}\OtherTok{=}\FunctionTok{rast}\NormalTok{(}\StringTok{"./RasterGrids\_100m/2024/RAW/General\_BareSoilQuarry\_r10000.tif"}\NormalTok{)}
\FunctionTok{names}\NormalTok{(slanis)}\OtherTok{=}\StringTok{"egv\_412"}
\NormalTok{slanis2}\OtherTok{=}\FunctionTok{project}\NormalTok{(slanis,template100)}
\FunctionTok{writeRaster}\NormalTok{(slanis2,}
      \StringTok{"./RasterGrids\_100m/2024/RAW/General\_BareSoilQuarry\_r10000.tif"}\NormalTok{,}
      \AttributeTok{overwrite=}\ConstantTok{TRUE}\NormalTok{)}

\CommentTok{\# standardisation {-}{-}{-}{-}}
\ControlFlowTok{if}\NormalTok{(}\SpecialCharTok{!}\FunctionTok{require}\NormalTok{(terra)) \{}\FunctionTok{install.packages}\NormalTok{(}\StringTok{"terra"}\NormalTok{); }\FunctionTok{require}\NormalTok{(terra)\}}
\ControlFlowTok{if}\NormalTok{(}\SpecialCharTok{!}\FunctionTok{require}\NormalTok{(tidyverse)) \{}\FunctionTok{install.packages}\NormalTok{(}\StringTok{"tidyverse"}\NormalTok{); }\FunctionTok{require}\NormalTok{(tidyverse)\}}

\NormalTok{nosaukums}\OtherTok{=}\StringTok{"General\_BareSoilQuarry\_r10000.tif"}
\NormalTok{ielasisanas\_cels}\OtherTok{=}\FunctionTok{paste0}\NormalTok{(}\StringTok{"./RasterGrids\_100m/2024/RAW/"}\NormalTok{,nosaukums)}
\NormalTok{saglabasanas\_cels}\OtherTok{=}\FunctionTok{paste0}\NormalTok{(}\StringTok{"./RasterGrids\_100m/2024/Scaled/"}\NormalTok{,nosaukums)}
\NormalTok{slanis}\OtherTok{=}\FunctionTok{rast}\NormalTok{(ielasisanas\_cels)}
\NormalTok{videjais}\OtherTok{=}\FunctionTok{global}\NormalTok{(slanis,}\AttributeTok{fun=}\StringTok{"mean"}\NormalTok{,}\AttributeTok{na.rm=}\ConstantTok{TRUE}\NormalTok{)}
\NormalTok{centrets}\OtherTok{=}\NormalTok{slanis}\SpecialCharTok{{-}}\NormalTok{videjais[,}\DecValTok{1}\NormalTok{]}
\NormalTok{standartnovirze}\OtherTok{=}\NormalTok{terra}\SpecialCharTok{::}\FunctionTok{global}\NormalTok{(centrets,}\AttributeTok{fun=}\StringTok{"rms"}\NormalTok{,}\AttributeTok{na.rm=}\ConstantTok{TRUE}\NormalTok{)}
\NormalTok{merogots}\OtherTok{=}\NormalTok{centrets}\SpecialCharTok{/}\NormalTok{standartnovirze[,}\DecValTok{1}\NormalTok{]}
\FunctionTok{writeRaster}\NormalTok{(merogots,}
      \AttributeTok{filename=}\NormalTok{saglabasanas\_cels,}
      \AttributeTok{overwrite=}\ConstantTok{TRUE}\NormalTok{)}
\end{Highlighting}
\end{Shaded}

\section{General\_Builtup\_cell}\label{ch06.413}

\textbf{filename:} \texttt{General\_Builtup\_cell.tif}

\textbf{layername:} \texttt{egv\_413}

\textbf{English name:} Fractional cover of Built-Up areas within the analysis cell (1
ha)

\textbf{Latvian name:} Apbūves platības īpatsvars analīzes šūnā (1 ha)

\textbf{Procedure:} First, the built-up areas from the \hyperref[Ch05.03]{Landscape classification}
are selected (value 500 is reclassified to value 1; all others are set to 0). The resulting layer
is then aggregated to EGV resolution using the workflow \texttt{egvtools::input2egv()}, which
calculates the arithmetic mean to determine the cover fraction. During
aggregation, inverse distance weighted (power = 2) gap filling on the output is
applied to ensure no missing values at the edges. Finally, the layer is
standardised by subtracting the arithmetic mean and dividing by the root mean squared
error.

\begin{Shaded}
\begin{Highlighting}[]
\CommentTok{\# libs {-}{-}{-}{-}}
\ControlFlowTok{if}\NormalTok{(}\SpecialCharTok{!}\FunctionTok{require}\NormalTok{(egvtools)) \{remotes}\SpecialCharTok{::}\FunctionTok{install\_github}\NormalTok{(}\StringTok{"aavotins/egvtools"}\NormalTok{); }\FunctionTok{require}\NormalTok{(egvtools)\}}
\ControlFlowTok{if}\NormalTok{(}\SpecialCharTok{!}\FunctionTok{require}\NormalTok{(terra)) \{}\FunctionTok{install.packages}\NormalTok{(}\StringTok{"terra"}\NormalTok{); }\FunctionTok{require}\NormalTok{(terra)\}}
\ControlFlowTok{if}\NormalTok{(}\SpecialCharTok{!}\FunctionTok{require}\NormalTok{(sf)) \{}\FunctionTok{install.packages}\NormalTok{(}\StringTok{"sf"}\NormalTok{); }\FunctionTok{require}\NormalTok{(sf)\}}
\ControlFlowTok{if}\NormalTok{(}\SpecialCharTok{!}\FunctionTok{require}\NormalTok{(tidyverse)) \{}\FunctionTok{install.packages}\NormalTok{(}\StringTok{"tidyverse"}\NormalTok{); }\FunctionTok{require}\NormalTok{(tidyverse)\}}
\ControlFlowTok{if}\NormalTok{(}\SpecialCharTok{!}\FunctionTok{require}\NormalTok{(sfarrow)) \{}\FunctionTok{install.packages}\NormalTok{(}\StringTok{"sfarrow"}\NormalTok{); }\FunctionTok{require}\NormalTok{(sfarrow)\}}
\ControlFlowTok{if}\NormalTok{(}\SpecialCharTok{!}\FunctionTok{require}\NormalTok{(readxl)) \{}\FunctionTok{install.packages}\NormalTok{(}\StringTok{"readxl"}\NormalTok{); }\FunctionTok{require}\NormalTok{(readxl)\}}
\ControlFlowTok{if}\NormalTok{(}\SpecialCharTok{!}\FunctionTok{require}\NormalTok{(raster)) \{}\FunctionTok{install.packages}\NormalTok{(}\StringTok{"raster"}\NormalTok{); }\FunctionTok{require}\NormalTok{(raster)\}}
\ControlFlowTok{if}\NormalTok{(}\SpecialCharTok{!}\FunctionTok{require}\NormalTok{(fasterize)) \{}\FunctionTok{install.packages}\NormalTok{(}\StringTok{"fasterize"}\NormalTok{); }\FunctionTok{require}\NormalTok{(fasterize)\}}

\CommentTok{\# templates {-}{-}{-}{-}}
\NormalTok{template100}\OtherTok{=}\FunctionTok{rast}\NormalTok{(}\StringTok{"./Templates/TemplateRasters/LV100m\_10km.tif"}\NormalTok{)}
\NormalTok{template10}\OtherTok{=}\FunctionTok{rast}\NormalTok{(}\StringTok{"./Templates/TemplateRasters/LV10m\_10km.tif"}\NormalTok{)}
\NormalTok{rastrs10}\OtherTok{=}\FunctionTok{raster}\NormalTok{(template10)}

\NormalTok{nulls10}\OtherTok{=}\FunctionTok{rast}\NormalTok{(}\StringTok{"./Templates/TemplateRasters/nulls\_LV10m\_10km.tif"}\NormalTok{)}
\NormalTok{nulls100}\OtherTok{=}\FunctionTok{rast}\NormalTok{(}\StringTok{"./Templates/TemplateRasters/nulls\_LV100m\_10km.tif"}\NormalTok{)}

\CommentTok{\# simple landscape {-}{-}{-}{-}}
\NormalTok{simple\_landscape}\OtherTok{=}\FunctionTok{rast}\NormalTok{(}\StringTok{"RasterGrids\_10m/2024/Ainava\_vienk\_mask.tif"}\NormalTok{)}


\CommentTok{\# General\_Builtup\_cell.tif  egv\_413 {-}{-}{-}{-}}
\NormalTok{builtup}\OtherTok{=}\FunctionTok{ifel}\NormalTok{(simple\_landscape}\SpecialCharTok{==}\DecValTok{500}\NormalTok{,}\DecValTok{1}\NormalTok{,}\DecValTok{0}\NormalTok{)}
\NormalTok{i2e\_rez}\OtherTok{=}\NormalTok{egvtools}\SpecialCharTok{::}\FunctionTok{input2egv}\NormalTok{(}\AttributeTok{input=}\NormalTok{builtup,}
              \AttributeTok{egv\_template=} \StringTok{"./Templates/TemplateRasters/LV100m\_10km.tif"}\NormalTok{,}
              \AttributeTok{summary\_function =} \StringTok{"average"}\NormalTok{,}
              \AttributeTok{missing\_job =} \StringTok{"FillOutput"}\NormalTok{,}
              \AttributeTok{outlocation =} \StringTok{"./RasterGrids\_100m/2024/RAW/"}\NormalTok{,}
              \AttributeTok{outfilename =} \StringTok{"General\_Builtup\_cell.tif"}\NormalTok{,}
              \AttributeTok{layername =} \StringTok{"egv\_413"}\NormalTok{,}
              \AttributeTok{idw\_weight =} \DecValTok{2}\NormalTok{,}
              \AttributeTok{plot\_gaps =} \ConstantTok{FALSE}\NormalTok{,}\AttributeTok{plot\_final =} \ConstantTok{TRUE}\NormalTok{)}
\NormalTok{i2e\_rez}
\FunctionTok{rm}\NormalTok{(builtup)}
\FunctionTok{rm}\NormalTok{(i2e\_rez)}

\CommentTok{\# standardisation {-}{-}{-}{-}}
\ControlFlowTok{if}\NormalTok{(}\SpecialCharTok{!}\FunctionTok{require}\NormalTok{(terra)) \{}\FunctionTok{install.packages}\NormalTok{(}\StringTok{"terra"}\NormalTok{); }\FunctionTok{require}\NormalTok{(terra)\}}
\ControlFlowTok{if}\NormalTok{(}\SpecialCharTok{!}\FunctionTok{require}\NormalTok{(tidyverse)) \{}\FunctionTok{install.packages}\NormalTok{(}\StringTok{"tidyverse"}\NormalTok{); }\FunctionTok{require}\NormalTok{(tidyverse)\}}

\NormalTok{nosaukums}\OtherTok{=}\StringTok{"General\_Builtup\_cell.tif"}
\NormalTok{ielasisanas\_cels}\OtherTok{=}\FunctionTok{paste0}\NormalTok{(}\StringTok{"./RasterGrids\_100m/2024/RAW/"}\NormalTok{,nosaukums)}
\NormalTok{saglabasanas\_cels}\OtherTok{=}\FunctionTok{paste0}\NormalTok{(}\StringTok{"./RasterGrids\_100m/2024/Scaled/"}\NormalTok{,nosaukums)}
\NormalTok{slanis}\OtherTok{=}\FunctionTok{rast}\NormalTok{(ielasisanas\_cels)}
\NormalTok{videjais}\OtherTok{=}\FunctionTok{global}\NormalTok{(slanis,}\AttributeTok{fun=}\StringTok{"mean"}\NormalTok{,}\AttributeTok{na.rm=}\ConstantTok{TRUE}\NormalTok{)}
\NormalTok{centrets}\OtherTok{=}\NormalTok{slanis}\SpecialCharTok{{-}}\NormalTok{videjais[,}\DecValTok{1}\NormalTok{]}
\NormalTok{standartnovirze}\OtherTok{=}\NormalTok{terra}\SpecialCharTok{::}\FunctionTok{global}\NormalTok{(centrets,}\AttributeTok{fun=}\StringTok{"rms"}\NormalTok{,}\AttributeTok{na.rm=}\ConstantTok{TRUE}\NormalTok{)}
\NormalTok{merogots}\OtherTok{=}\NormalTok{centrets}\SpecialCharTok{/}\NormalTok{standartnovirze[,}\DecValTok{1}\NormalTok{]}
\FunctionTok{writeRaster}\NormalTok{(merogots,}
      \AttributeTok{filename=}\NormalTok{saglabasanas\_cels,}
      \AttributeTok{overwrite=}\ConstantTok{TRUE}\NormalTok{)}
\end{Highlighting}
\end{Shaded}

\section{General\_Builtup\_r500}\label{ch06.414}

\textbf{filename:} \texttt{General\_Builtup\_r500.tif}

\textbf{layername:} \texttt{egv\_414}

\textbf{English name:} Fractional cover of Built-Up areas within the 0.5 km landscape

\textbf{Latvian name:} Apbūves platības īpatsvars 0,5 km ainavā

\textbf{Procedure:} The cover fraction within a radius of 500 m around the analysis grid cell is
calculated as the area-weighted sum of the \hyperref[ch06.413]{analysis cells} inside the
buffer, using the workflow \texttt{egvtools::radius\_function()}. During the calculation of the landscape metric,
inverse distance weighted (power = 2) gap filling on the output is applied
to ensure no missing values at the edges. Then the layer is rewritten to set
its name. Finally, the layer is standardised by subtracting the arithmetic
mean and dividing by the root mean squared error.

\begin{Shaded}
\begin{Highlighting}[]
\CommentTok{\# libs {-}{-}{-}{-}}
\ControlFlowTok{if}\NormalTok{(}\SpecialCharTok{!}\FunctionTok{require}\NormalTok{(terra)) \{}\FunctionTok{install.packages}\NormalTok{(}\StringTok{"terra"}\NormalTok{); }\FunctionTok{require}\NormalTok{(terra)\}}
\ControlFlowTok{if}\NormalTok{(}\SpecialCharTok{!}\FunctionTok{require}\NormalTok{(egvtools)) \{remotes}\SpecialCharTok{::}\FunctionTok{install\_github}\NormalTok{(}\StringTok{"aavotins/egvtools"}\NormalTok{); }\FunctionTok{require}\NormalTok{(egvtools)\}}


\CommentTok{\# Templates {-}{-}{-}{-}{-}}
\NormalTok{template100}\OtherTok{=}\FunctionTok{rast}\NormalTok{(}\StringTok{"./Templates/TemplateRasters/LV100m\_10km.tif"}\NormalTok{)}

\CommentTok{\# radii {-}{-}{-}{-}}
\FunctionTok{radius\_function}\NormalTok{(}
 \AttributeTok{kvadrati\_path =} \StringTok{"./Templates/TemplateGrids/tiles/"}\NormalTok{,}
 \AttributeTok{radii\_path   =} \StringTok{"./Templates/TemplateGridPoints/tiles/"}\NormalTok{,}
 \AttributeTok{tikls100\_path =} \StringTok{"./Templates/TemplateGrids/tikls100\_sauzeme.parquet"}\NormalTok{,}
 \AttributeTok{template\_path =} \StringTok{"./Templates/TemplateRasters/LV100m\_10km.tif"}\NormalTok{,}
 \AttributeTok{input\_layers  =} \FunctionTok{c}\NormalTok{(}\StringTok{"./RasterGrids\_100m/2024/RAW/General\_Builtup\_cell.tif"}\NormalTok{),}
 \AttributeTok{layer\_prefixes =} \FunctionTok{c}\NormalTok{(}\StringTok{"General\_Builtup"}\NormalTok{),}
 \AttributeTok{output\_dir   =} \StringTok{"./RasterGrids\_100m/2024/RAW/"}\NormalTok{,}
 \AttributeTok{n\_workers   =} \DecValTok{6}\NormalTok{,}
 \AttributeTok{radii     =} \FunctionTok{c}\NormalTok{(}\StringTok{"r500"}\NormalTok{),}
 \AttributeTok{radius\_mode  =} \StringTok{"sparse"}\NormalTok{,}
 \AttributeTok{extract\_fun  =} \StringTok{"mean"}\NormalTok{,}
 \AttributeTok{fill\_missing  =} \ConstantTok{TRUE}\NormalTok{,}
 \AttributeTok{IDW\_weight   =} \DecValTok{2}\NormalTok{,}
 \AttributeTok{future\_max\_size =} \DecValTok{40} \SpecialCharTok{*} \DecValTok{1024}\SpecialCharTok{\^{}}\DecValTok{3}\NormalTok{)}


\CommentTok{\# General\_Builtup\_r500.tif  egv\_414}
\NormalTok{slanis}\OtherTok{=}\FunctionTok{rast}\NormalTok{(}\StringTok{"./RasterGrids\_100m/2024/RAW/General\_Builtup\_r500.tif"}\NormalTok{)}
\FunctionTok{names}\NormalTok{(slanis)}\OtherTok{=}\StringTok{"egv\_414"}
\NormalTok{slanis2}\OtherTok{=}\FunctionTok{project}\NormalTok{(slanis,template100)}
\FunctionTok{writeRaster}\NormalTok{(slanis2,}
      \StringTok{"./RasterGrids\_100m/2024/RAW/General\_Builtup\_r500.tif"}\NormalTok{,}
      \AttributeTok{overwrite=}\ConstantTok{TRUE}\NormalTok{)}

\CommentTok{\# standardisation {-}{-}{-}{-}}
\ControlFlowTok{if}\NormalTok{(}\SpecialCharTok{!}\FunctionTok{require}\NormalTok{(terra)) \{}\FunctionTok{install.packages}\NormalTok{(}\StringTok{"terra"}\NormalTok{); }\FunctionTok{require}\NormalTok{(terra)\}}
\ControlFlowTok{if}\NormalTok{(}\SpecialCharTok{!}\FunctionTok{require}\NormalTok{(tidyverse)) \{}\FunctionTok{install.packages}\NormalTok{(}\StringTok{"tidyverse"}\NormalTok{); }\FunctionTok{require}\NormalTok{(tidyverse)\}}

\NormalTok{nosaukums}\OtherTok{=}\StringTok{"General\_Builtup\_r500.tif"}
\NormalTok{ielasisanas\_cels}\OtherTok{=}\FunctionTok{paste0}\NormalTok{(}\StringTok{"./RasterGrids\_100m/2024/RAW/"}\NormalTok{,nosaukums)}
\NormalTok{saglabasanas\_cels}\OtherTok{=}\FunctionTok{paste0}\NormalTok{(}\StringTok{"./RasterGrids\_100m/2024/Scaled/"}\NormalTok{,nosaukums)}
\NormalTok{slanis}\OtherTok{=}\FunctionTok{rast}\NormalTok{(ielasisanas\_cels)}
\NormalTok{videjais}\OtherTok{=}\FunctionTok{global}\NormalTok{(slanis,}\AttributeTok{fun=}\StringTok{"mean"}\NormalTok{,}\AttributeTok{na.rm=}\ConstantTok{TRUE}\NormalTok{)}
\NormalTok{centrets}\OtherTok{=}\NormalTok{slanis}\SpecialCharTok{{-}}\NormalTok{videjais[,}\DecValTok{1}\NormalTok{]}
\NormalTok{standartnovirze}\OtherTok{=}\NormalTok{terra}\SpecialCharTok{::}\FunctionTok{global}\NormalTok{(centrets,}\AttributeTok{fun=}\StringTok{"rms"}\NormalTok{,}\AttributeTok{na.rm=}\ConstantTok{TRUE}\NormalTok{)}
\NormalTok{merogots}\OtherTok{=}\NormalTok{centrets}\SpecialCharTok{/}\NormalTok{standartnovirze[,}\DecValTok{1}\NormalTok{]}
\FunctionTok{writeRaster}\NormalTok{(merogots,}
      \AttributeTok{filename=}\NormalTok{saglabasanas\_cels,}
      \AttributeTok{overwrite=}\ConstantTok{TRUE}\NormalTok{)}
\end{Highlighting}
\end{Shaded}

\section{General\_Builtup\_r1250}\label{ch06.415}

\textbf{filename:} \texttt{General\_Builtup\_r1250.tif}

\textbf{layername:} \texttt{egv\_415}

\textbf{English name:} Fractional cover of Built-Up areas within the 1.25 km
landscape

\textbf{Latvian name:} Apbūves platības īpatsvars 1,25 km ainavā

\textbf{Procedure:} The cover fraction within a radius of 1250 m around the analysis grid cell
is calculated as the area-weighted sum of the \hyperref[ch06.413]{analysis cells} inside
the buffer, using the workflow \texttt{egvtools::radius\_function()}. During the calculation of the landscape
metric, inverse distance weighted (power = 2) gap filling on the output is
applied to ensure no missing values at the edges. Then the layer is
rewritten to set its name. Finally, the layer is standardised by
subtracting the arithmetic mean and dividing by the root mean squared error.

\begin{Shaded}
\begin{Highlighting}[]
\CommentTok{\# libs {-}{-}{-}{-}}
\ControlFlowTok{if}\NormalTok{(}\SpecialCharTok{!}\FunctionTok{require}\NormalTok{(terra)) \{}\FunctionTok{install.packages}\NormalTok{(}\StringTok{"terra"}\NormalTok{); }\FunctionTok{require}\NormalTok{(terra)\}}
\ControlFlowTok{if}\NormalTok{(}\SpecialCharTok{!}\FunctionTok{require}\NormalTok{(egvtools)) \{remotes}\SpecialCharTok{::}\FunctionTok{install\_github}\NormalTok{(}\StringTok{"aavotins/egvtools"}\NormalTok{); }\FunctionTok{require}\NormalTok{(egvtools)\}}


\CommentTok{\# Templates {-}{-}{-}{-}{-}}
\NormalTok{template100}\OtherTok{=}\FunctionTok{rast}\NormalTok{(}\StringTok{"./Templates/TemplateRasters/LV100m\_10km.tif"}\NormalTok{)}

\CommentTok{\# radii {-}{-}{-}{-}}
\FunctionTok{radius\_function}\NormalTok{(}
 \AttributeTok{kvadrati\_path =} \StringTok{"./Templates/TemplateGrids/tiles/"}\NormalTok{,}
 \AttributeTok{radii\_path   =} \StringTok{"./Templates/TemplateGridPoints/tiles/"}\NormalTok{,}
 \AttributeTok{tikls100\_path =} \StringTok{"./Templates/TemplateGrids/tikls100\_sauzeme.parquet"}\NormalTok{,}
 \AttributeTok{template\_path =} \StringTok{"./Templates/TemplateRasters/LV100m\_10km.tif"}\NormalTok{,}
 \AttributeTok{input\_layers  =} \FunctionTok{c}\NormalTok{(}\StringTok{"./RasterGrids\_100m/2024/RAW/General\_Builtup\_cell.tif"}\NormalTok{),}
 \AttributeTok{layer\_prefixes =} \FunctionTok{c}\NormalTok{(}\StringTok{"General\_Builtup"}\NormalTok{),}
 \AttributeTok{output\_dir   =} \StringTok{"./RasterGrids\_100m/2024/RAW/"}\NormalTok{,}
 \AttributeTok{n\_workers   =} \DecValTok{6}\NormalTok{,}
 \AttributeTok{radii     =} \FunctionTok{c}\NormalTok{(}\StringTok{"r1250"}\NormalTok{),}
 \AttributeTok{radius\_mode  =} \StringTok{"sparse"}\NormalTok{,}
 \AttributeTok{extract\_fun  =} \StringTok{"mean"}\NormalTok{,}
 \AttributeTok{fill\_missing  =} \ConstantTok{TRUE}\NormalTok{,}
 \AttributeTok{IDW\_weight   =} \DecValTok{2}\NormalTok{,}
 \AttributeTok{future\_max\_size =} \DecValTok{40} \SpecialCharTok{*} \DecValTok{1024}\SpecialCharTok{\^{}}\DecValTok{3}\NormalTok{)}


\CommentTok{\# General\_Builtup\_r1250.tif egv\_415}
\NormalTok{slanis}\OtherTok{=}\FunctionTok{rast}\NormalTok{(}\StringTok{"./RasterGrids\_100m/2024/RAW/General\_Builtup\_r1250.tif"}\NormalTok{)}
\FunctionTok{names}\NormalTok{(slanis)}\OtherTok{=}\StringTok{"egv\_415"}
\NormalTok{slanis2}\OtherTok{=}\FunctionTok{project}\NormalTok{(slanis,template100)}
\FunctionTok{writeRaster}\NormalTok{(slanis2,}
      \StringTok{"./RasterGrids\_100m/2024/RAW/General\_Builtup\_r1250.tif"}\NormalTok{,}
      \AttributeTok{overwrite=}\ConstantTok{TRUE}\NormalTok{)}

\CommentTok{\# standardisation {-}{-}{-}{-}}
\ControlFlowTok{if}\NormalTok{(}\SpecialCharTok{!}\FunctionTok{require}\NormalTok{(terra)) \{}\FunctionTok{install.packages}\NormalTok{(}\StringTok{"terra"}\NormalTok{); }\FunctionTok{require}\NormalTok{(terra)\}}
\ControlFlowTok{if}\NormalTok{(}\SpecialCharTok{!}\FunctionTok{require}\NormalTok{(tidyverse)) \{}\FunctionTok{install.packages}\NormalTok{(}\StringTok{"tidyverse"}\NormalTok{); }\FunctionTok{require}\NormalTok{(tidyverse)\}}

\NormalTok{nosaukums}\OtherTok{=}\StringTok{"General\_Builtup\_r1250.tif"}
\NormalTok{ielasisanas\_cels}\OtherTok{=}\FunctionTok{paste0}\NormalTok{(}\StringTok{"./RasterGrids\_100m/2024/RAW/"}\NormalTok{,nosaukums)}
\NormalTok{saglabasanas\_cels}\OtherTok{=}\FunctionTok{paste0}\NormalTok{(}\StringTok{"./RasterGrids\_100m/2024/Scaled/"}\NormalTok{,nosaukums)}
\NormalTok{slanis}\OtherTok{=}\FunctionTok{rast}\NormalTok{(ielasisanas\_cels)}
\NormalTok{videjais}\OtherTok{=}\FunctionTok{global}\NormalTok{(slanis,}\AttributeTok{fun=}\StringTok{"mean"}\NormalTok{,}\AttributeTok{na.rm=}\ConstantTok{TRUE}\NormalTok{)}
\NormalTok{centrets}\OtherTok{=}\NormalTok{slanis}\SpecialCharTok{{-}}\NormalTok{videjais[,}\DecValTok{1}\NormalTok{]}
\NormalTok{standartnovirze}\OtherTok{=}\NormalTok{terra}\SpecialCharTok{::}\FunctionTok{global}\NormalTok{(centrets,}\AttributeTok{fun=}\StringTok{"rms"}\NormalTok{,}\AttributeTok{na.rm=}\ConstantTok{TRUE}\NormalTok{)}
\NormalTok{merogots}\OtherTok{=}\NormalTok{centrets}\SpecialCharTok{/}\NormalTok{standartnovirze[,}\DecValTok{1}\NormalTok{]}
\FunctionTok{writeRaster}\NormalTok{(merogots,}
      \AttributeTok{filename=}\NormalTok{saglabasanas\_cels,}
      \AttributeTok{overwrite=}\ConstantTok{TRUE}\NormalTok{)}
\end{Highlighting}
\end{Shaded}

\section{General\_Builtup\_r3000}\label{ch06.416}

\textbf{filename:} \texttt{General\_Builtup\_r3000.tif}

\textbf{layername:} \texttt{egv\_416}

\textbf{English name:} Fractional cover of Built-Up areas within the 3 km landscape

\textbf{Latvian name:} Apbūves platības īpatsvars 3 km ainavā

\textbf{Procedure:} The cover fraction within a radius of 3000 m around the analysis grid cell
is calculated as the area-weighted sum of the \hyperref[ch06.413]{analysis cells} inside
the buffer, using the workflow \texttt{egvtools::radius\_function()}. During the calculation of the landscape
metric, inverse distance weighted (power = 2) gap filling on the output is
applied to ensure no missing values at the edges. Then the layer is
rewritten to set its name. Finally, the layer is standardised by
subtracting the arithmetic mean and dividing by the root mean squared error.

\begin{Shaded}
\begin{Highlighting}[]
\CommentTok{\# libs {-}{-}{-}{-}}
\ControlFlowTok{if}\NormalTok{(}\SpecialCharTok{!}\FunctionTok{require}\NormalTok{(terra)) \{}\FunctionTok{install.packages}\NormalTok{(}\StringTok{"terra"}\NormalTok{); }\FunctionTok{require}\NormalTok{(terra)\}}
\ControlFlowTok{if}\NormalTok{(}\SpecialCharTok{!}\FunctionTok{require}\NormalTok{(egvtools)) \{remotes}\SpecialCharTok{::}\FunctionTok{install\_github}\NormalTok{(}\StringTok{"aavotins/egvtools"}\NormalTok{); }\FunctionTok{require}\NormalTok{(egvtools)\}}


\CommentTok{\# Templates {-}{-}{-}{-}{-}}
\NormalTok{template100}\OtherTok{=}\FunctionTok{rast}\NormalTok{(}\StringTok{"./Templates/TemplateRasters/LV100m\_10km.tif"}\NormalTok{)}

\CommentTok{\# radii {-}{-}{-}{-}}
\FunctionTok{radius\_function}\NormalTok{(}
 \AttributeTok{kvadrati\_path =} \StringTok{"./Templates/TemplateGrids/tiles/"}\NormalTok{,}
 \AttributeTok{radii\_path   =} \StringTok{"./Templates/TemplateGridPoints/tiles/"}\NormalTok{,}
 \AttributeTok{tikls100\_path =} \StringTok{"./Templates/TemplateGrids/tikls100\_sauzeme.parquet"}\NormalTok{,}
 \AttributeTok{template\_path =} \StringTok{"./Templates/TemplateRasters/LV100m\_10km.tif"}\NormalTok{,}
 \AttributeTok{input\_layers  =} \FunctionTok{c}\NormalTok{(}\StringTok{"./RasterGrids\_100m/2024/RAW/General\_Builtup\_cell.tif"}\NormalTok{),}
 \AttributeTok{layer\_prefixes =} \FunctionTok{c}\NormalTok{(}\StringTok{"General\_Builtup"}\NormalTok{),}
 \AttributeTok{output\_dir   =} \StringTok{"./RasterGrids\_100m/2024/RAW/"}\NormalTok{,}
 \AttributeTok{n\_workers   =} \DecValTok{6}\NormalTok{,}
 \AttributeTok{radii     =} \FunctionTok{c}\NormalTok{(}\StringTok{"r3000"}\NormalTok{),}
 \AttributeTok{radius\_mode  =} \StringTok{"sparse"}\NormalTok{,}
 \AttributeTok{extract\_fun  =} \StringTok{"mean"}\NormalTok{,}
 \AttributeTok{fill\_missing  =} \ConstantTok{TRUE}\NormalTok{,}
 \AttributeTok{IDW\_weight   =} \DecValTok{2}\NormalTok{,}
 \AttributeTok{future\_max\_size =} \DecValTok{40} \SpecialCharTok{*} \DecValTok{1024}\SpecialCharTok{\^{}}\DecValTok{3}\NormalTok{)}


\CommentTok{\# General\_Builtup\_r3000.tif egv\_416}
\NormalTok{slanis}\OtherTok{=}\FunctionTok{rast}\NormalTok{(}\StringTok{"./RasterGrids\_100m/2024/RAW/General\_Builtup\_r3000.tif"}\NormalTok{)}
\FunctionTok{names}\NormalTok{(slanis)}\OtherTok{=}\StringTok{"egv\_416"}
\NormalTok{slanis2}\OtherTok{=}\FunctionTok{project}\NormalTok{(slanis,template100)}
\FunctionTok{writeRaster}\NormalTok{(slanis2,}
      \StringTok{"./RasterGrids\_100m/2024/RAW/General\_Builtup\_r3000.tif"}\NormalTok{,}
      \AttributeTok{overwrite=}\ConstantTok{TRUE}\NormalTok{)}

\CommentTok{\# standardisation {-}{-}{-}{-}}
\ControlFlowTok{if}\NormalTok{(}\SpecialCharTok{!}\FunctionTok{require}\NormalTok{(terra)) \{}\FunctionTok{install.packages}\NormalTok{(}\StringTok{"terra"}\NormalTok{); }\FunctionTok{require}\NormalTok{(terra)\}}
\ControlFlowTok{if}\NormalTok{(}\SpecialCharTok{!}\FunctionTok{require}\NormalTok{(tidyverse)) \{}\FunctionTok{install.packages}\NormalTok{(}\StringTok{"tidyverse"}\NormalTok{); }\FunctionTok{require}\NormalTok{(tidyverse)\}}

\NormalTok{nosaukums}\OtherTok{=}\StringTok{"General\_Builtup\_r3000.tif"}
\NormalTok{ielasisanas\_cels}\OtherTok{=}\FunctionTok{paste0}\NormalTok{(}\StringTok{"./RasterGrids\_100m/2024/RAW/"}\NormalTok{,nosaukums)}
\NormalTok{saglabasanas\_cels}\OtherTok{=}\FunctionTok{paste0}\NormalTok{(}\StringTok{"./RasterGrids\_100m/2024/Scaled/"}\NormalTok{,nosaukums)}
\NormalTok{slanis}\OtherTok{=}\FunctionTok{rast}\NormalTok{(ielasisanas\_cels)}
\NormalTok{videjais}\OtherTok{=}\FunctionTok{global}\NormalTok{(slanis,}\AttributeTok{fun=}\StringTok{"mean"}\NormalTok{,}\AttributeTok{na.rm=}\ConstantTok{TRUE}\NormalTok{)}
\NormalTok{centrets}\OtherTok{=}\NormalTok{slanis}\SpecialCharTok{{-}}\NormalTok{videjais[,}\DecValTok{1}\NormalTok{]}
\NormalTok{standartnovirze}\OtherTok{=}\NormalTok{terra}\SpecialCharTok{::}\FunctionTok{global}\NormalTok{(centrets,}\AttributeTok{fun=}\StringTok{"rms"}\NormalTok{,}\AttributeTok{na.rm=}\ConstantTok{TRUE}\NormalTok{)}
\NormalTok{merogots}\OtherTok{=}\NormalTok{centrets}\SpecialCharTok{/}\NormalTok{standartnovirze[,}\DecValTok{1}\NormalTok{]}
\FunctionTok{writeRaster}\NormalTok{(merogots,}
      \AttributeTok{filename=}\NormalTok{saglabasanas\_cels,}
      \AttributeTok{overwrite=}\ConstantTok{TRUE}\NormalTok{)}
\end{Highlighting}
\end{Shaded}

\section{General\_Builtup\_r10000}\label{ch06.417}

\textbf{filename:} \texttt{General\_Builtup\_r10000.tif}

\textbf{layername:} \texttt{egv\_417}

\textbf{English name:} Fractional cover of Built-Up areas within the 10 km landscape

\textbf{Latvian name:} Apbūves platības īpatsvars 10 km ainavā

\textbf{Procedure:} The cover fraction within a radius of 10000 m around the analysis grid cell
is calculated as the area-weighted sum of the \hyperref[ch06.413]{analysis cells} inside
the buffer, using the workflow \texttt{egvtools::radius\_function()}. During the calculation of the landscape
metric, inverse distance weighted (power = 2) gap filling on the output is
applied to ensure no missing values at the edges. Then the layer is
rewritten to set its name. Finally, the layer is standardised by
subtracting the arithmetic mean and dividing by the root mean squared error.

\begin{Shaded}
\begin{Highlighting}[]
\CommentTok{\# libs {-}{-}{-}{-}}
\ControlFlowTok{if}\NormalTok{(}\SpecialCharTok{!}\FunctionTok{require}\NormalTok{(terra)) \{}\FunctionTok{install.packages}\NormalTok{(}\StringTok{"terra"}\NormalTok{); }\FunctionTok{require}\NormalTok{(terra)\}}
\ControlFlowTok{if}\NormalTok{(}\SpecialCharTok{!}\FunctionTok{require}\NormalTok{(egvtools)) \{remotes}\SpecialCharTok{::}\FunctionTok{install\_github}\NormalTok{(}\StringTok{"aavotins/egvtools"}\NormalTok{); }\FunctionTok{require}\NormalTok{(egvtools)\}}


\CommentTok{\# Templates {-}{-}{-}{-}{-}}
\NormalTok{template100}\OtherTok{=}\FunctionTok{rast}\NormalTok{(}\StringTok{"./Templates/TemplateRasters/LV100m\_10km.tif"}\NormalTok{)}

\CommentTok{\# radii {-}{-}{-}{-}}
\FunctionTok{radius\_function}\NormalTok{(}
 \AttributeTok{kvadrati\_path =} \StringTok{"./Templates/TemplateGrids/tiles/"}\NormalTok{,}
 \AttributeTok{radii\_path   =} \StringTok{"./Templates/TemplateGridPoints/tiles/"}\NormalTok{,}
 \AttributeTok{tikls100\_path =} \StringTok{"./Templates/TemplateGrids/tikls100\_sauzeme.parquet"}\NormalTok{,}
 \AttributeTok{template\_path =} \StringTok{"./Templates/TemplateRasters/LV100m\_10km.tif"}\NormalTok{,}
 \AttributeTok{input\_layers  =} \FunctionTok{c}\NormalTok{(}\StringTok{"./RasterGrids\_100m/2024/RAW/General\_Builtup\_cell.tif"}\NormalTok{),}
 \AttributeTok{layer\_prefixes =} \FunctionTok{c}\NormalTok{(}\StringTok{"General\_Builtup"}\NormalTok{),}
 \AttributeTok{output\_dir   =} \StringTok{"./RasterGrids\_100m/2024/RAW/"}\NormalTok{,}
 \AttributeTok{n\_workers   =} \DecValTok{6}\NormalTok{,}
 \AttributeTok{radii     =} \FunctionTok{c}\NormalTok{(}\StringTok{"r10000"}\NormalTok{),}
 \AttributeTok{radius\_mode  =} \StringTok{"sparse"}\NormalTok{,}
 \AttributeTok{extract\_fun  =} \StringTok{"mean"}\NormalTok{,}
 \AttributeTok{fill\_missing  =} \ConstantTok{TRUE}\NormalTok{,}
 \AttributeTok{IDW\_weight   =} \DecValTok{2}\NormalTok{,}
 \AttributeTok{future\_max\_size =} \DecValTok{40} \SpecialCharTok{*} \DecValTok{1024}\SpecialCharTok{\^{}}\DecValTok{3}\NormalTok{)}


\CommentTok{\# General\_Builtup\_r10000.tif    egv\_417}
\NormalTok{slanis}\OtherTok{=}\FunctionTok{rast}\NormalTok{(}\StringTok{"./RasterGrids\_100m/2024/RAW/General\_Builtup\_r10000.tif"}\NormalTok{)}
\FunctionTok{names}\NormalTok{(slanis)}\OtherTok{=}\StringTok{"egv\_417"}
\NormalTok{slanis2}\OtherTok{=}\FunctionTok{project}\NormalTok{(slanis,template100)}
\FunctionTok{writeRaster}\NormalTok{(slanis2,}
      \StringTok{"./RasterGrids\_100m/2024/RAW/General\_Builtup\_r10000.tif"}\NormalTok{,}
      \AttributeTok{overwrite=}\ConstantTok{TRUE}\NormalTok{)}

\CommentTok{\# standardisation {-}{-}{-}{-}}
\ControlFlowTok{if}\NormalTok{(}\SpecialCharTok{!}\FunctionTok{require}\NormalTok{(terra)) \{}\FunctionTok{install.packages}\NormalTok{(}\StringTok{"terra"}\NormalTok{); }\FunctionTok{require}\NormalTok{(terra)\}}
\ControlFlowTok{if}\NormalTok{(}\SpecialCharTok{!}\FunctionTok{require}\NormalTok{(tidyverse)) \{}\FunctionTok{install.packages}\NormalTok{(}\StringTok{"tidyverse"}\NormalTok{); }\FunctionTok{require}\NormalTok{(tidyverse)\}}

\NormalTok{nosaukums}\OtherTok{=}\StringTok{"General\_Builtup\_r10000.tif"}
\NormalTok{ielasisanas\_cels}\OtherTok{=}\FunctionTok{paste0}\NormalTok{(}\StringTok{"./RasterGrids\_100m/2024/RAW/"}\NormalTok{,nosaukums)}
\NormalTok{saglabasanas\_cels}\OtherTok{=}\FunctionTok{paste0}\NormalTok{(}\StringTok{"./RasterGrids\_100m/2024/Scaled/"}\NormalTok{,nosaukums)}
\NormalTok{slanis}\OtherTok{=}\FunctionTok{rast}\NormalTok{(ielasisanas\_cels)}
\NormalTok{videjais}\OtherTok{=}\FunctionTok{global}\NormalTok{(slanis,}\AttributeTok{fun=}\StringTok{"mean"}\NormalTok{,}\AttributeTok{na.rm=}\ConstantTok{TRUE}\NormalTok{)}
\NormalTok{centrets}\OtherTok{=}\NormalTok{slanis}\SpecialCharTok{{-}}\NormalTok{videjais[,}\DecValTok{1}\NormalTok{]}
\NormalTok{standartnovirze}\OtherTok{=}\NormalTok{terra}\SpecialCharTok{::}\FunctionTok{global}\NormalTok{(centrets,}\AttributeTok{fun=}\StringTok{"rms"}\NormalTok{,}\AttributeTok{na.rm=}\ConstantTok{TRUE}\NormalTok{)}
\NormalTok{merogots}\OtherTok{=}\NormalTok{centrets}\SpecialCharTok{/}\NormalTok{standartnovirze[,}\DecValTok{1}\NormalTok{]}
\FunctionTok{writeRaster}\NormalTok{(merogots,}
      \AttributeTok{filename=}\NormalTok{saglabasanas\_cels,}
      \AttributeTok{overwrite=}\ConstantTok{TRUE}\NormalTok{)}
\end{Highlighting}
\end{Shaded}

\section{General\_Farmland\_cell}\label{ch06.418}

\textbf{filename:} \texttt{General\_Farmland\_cell.tif}

\textbf{layername:} \texttt{egv\_418}

\textbf{English name:} Fractional cover of Farmland within the analysis cell (1 ha)

\textbf{Latvian name:} Lauksaimniecībā izmantojamo zemju platības īpatsvars analīzes
šūnā (1 ha)

\textbf{Procedure:} First, the farmlands from the \hyperref[Ch05.03]{Landscape classification} are
selected (values between 300 and 400 are reclassified to value 1; all others are set to 0).
The resulting layer
is then aggregated to EGV resolution using the workflow \texttt{egvtools::input2egv()}, which
calculates the arithmetic mean to determine the cover fraction. During
aggregation, inverse distance weighted (power = 2) gap filling on the output is
applied to ensure no missing values at the edges. Finally, the layer is
standardised by subtracting the arithmetic mean and dividing by the root mean squared
error.

\begin{Shaded}
\begin{Highlighting}[]
\CommentTok{\# libs {-}{-}{-}{-}}
\ControlFlowTok{if}\NormalTok{(}\SpecialCharTok{!}\FunctionTok{require}\NormalTok{(egvtools)) \{remotes}\SpecialCharTok{::}\FunctionTok{install\_github}\NormalTok{(}\StringTok{"aavotins/egvtools"}\NormalTok{); }\FunctionTok{require}\NormalTok{(egvtools)\}}
\ControlFlowTok{if}\NormalTok{(}\SpecialCharTok{!}\FunctionTok{require}\NormalTok{(terra)) \{}\FunctionTok{install.packages}\NormalTok{(}\StringTok{"terra"}\NormalTok{); }\FunctionTok{require}\NormalTok{(terra)\}}
\ControlFlowTok{if}\NormalTok{(}\SpecialCharTok{!}\FunctionTok{require}\NormalTok{(sf)) \{}\FunctionTok{install.packages}\NormalTok{(}\StringTok{"sf"}\NormalTok{); }\FunctionTok{require}\NormalTok{(sf)\}}
\ControlFlowTok{if}\NormalTok{(}\SpecialCharTok{!}\FunctionTok{require}\NormalTok{(tidyverse)) \{}\FunctionTok{install.packages}\NormalTok{(}\StringTok{"tidyverse"}\NormalTok{); }\FunctionTok{require}\NormalTok{(tidyverse)\}}
\ControlFlowTok{if}\NormalTok{(}\SpecialCharTok{!}\FunctionTok{require}\NormalTok{(sfarrow)) \{}\FunctionTok{install.packages}\NormalTok{(}\StringTok{"sfarrow"}\NormalTok{); }\FunctionTok{require}\NormalTok{(sfarrow)\}}
\ControlFlowTok{if}\NormalTok{(}\SpecialCharTok{!}\FunctionTok{require}\NormalTok{(readxl)) \{}\FunctionTok{install.packages}\NormalTok{(}\StringTok{"readxl"}\NormalTok{); }\FunctionTok{require}\NormalTok{(readxl)\}}
\ControlFlowTok{if}\NormalTok{(}\SpecialCharTok{!}\FunctionTok{require}\NormalTok{(raster)) \{}\FunctionTok{install.packages}\NormalTok{(}\StringTok{"raster"}\NormalTok{); }\FunctionTok{require}\NormalTok{(raster)\}}
\ControlFlowTok{if}\NormalTok{(}\SpecialCharTok{!}\FunctionTok{require}\NormalTok{(fasterize)) \{}\FunctionTok{install.packages}\NormalTok{(}\StringTok{"fasterize"}\NormalTok{); }\FunctionTok{require}\NormalTok{(fasterize)\}}

\CommentTok{\# templates {-}{-}{-}{-}}
\NormalTok{template100}\OtherTok{=}\FunctionTok{rast}\NormalTok{(}\StringTok{"./Templates/TemplateRasters/LV100m\_10km.tif"}\NormalTok{)}
\NormalTok{template10}\OtherTok{=}\FunctionTok{rast}\NormalTok{(}\StringTok{"./Templates/TemplateRasters/LV10m\_10km.tif"}\NormalTok{)}
\NormalTok{rastrs10}\OtherTok{=}\FunctionTok{raster}\NormalTok{(template10)}

\NormalTok{nulls10}\OtherTok{=}\FunctionTok{rast}\NormalTok{(}\StringTok{"./Templates/TemplateRasters/nulls\_LV10m\_10km.tif"}\NormalTok{)}
\NormalTok{nulls100}\OtherTok{=}\FunctionTok{rast}\NormalTok{(}\StringTok{"./Templates/TemplateRasters/nulls\_LV100m\_10km.tif"}\NormalTok{)}

\CommentTok{\# simple landscape {-}{-}{-}{-}}
\NormalTok{simple\_landscape}\OtherTok{=}\FunctionTok{rast}\NormalTok{(}\StringTok{"RasterGrids\_10m/2024/Ainava\_vienk\_mask.tif"}\NormalTok{)}


\CommentTok{\# General\_Farmland\_cell.tif egv\_418 {-}{-}{-}{-}}
\NormalTok{farmland}\OtherTok{=}\FunctionTok{ifel}\NormalTok{(simple\_landscape}\SpecialCharTok{\textgreater{}=}\DecValTok{300}\SpecialCharTok{\&}\NormalTok{simple\_landscape}\SpecialCharTok{\textless{}}\DecValTok{400}\NormalTok{,}\DecValTok{1}\NormalTok{,}\DecValTok{0}\NormalTok{)}
\NormalTok{i2e\_rez}\OtherTok{=}\NormalTok{egvtools}\SpecialCharTok{::}\FunctionTok{input2egv}\NormalTok{(}\AttributeTok{input=}\NormalTok{farmland,}
              \AttributeTok{egv\_template=} \StringTok{"./Templates/TemplateRasters/LV100m\_10km.tif"}\NormalTok{,}
              \AttributeTok{summary\_function =} \StringTok{"average"}\NormalTok{,}
              \AttributeTok{missing\_job =} \StringTok{"FillOutput"}\NormalTok{,}
              \AttributeTok{outlocation =} \StringTok{"./RasterGrids\_100m/2024/RAW/"}\NormalTok{,}
              \AttributeTok{outfilename =} \StringTok{"General\_Farmland\_cell.tif"}\NormalTok{,}
              \AttributeTok{layername =} \StringTok{"egv\_418"}\NormalTok{,}
              \AttributeTok{idw\_weight =} \DecValTok{2}\NormalTok{,}
              \AttributeTok{plot\_gaps =} \ConstantTok{FALSE}\NormalTok{,}\AttributeTok{plot\_final =} \ConstantTok{TRUE}\NormalTok{)}
\NormalTok{i2e\_rez}
\FunctionTok{rm}\NormalTok{(farmland)}
\FunctionTok{rm}\NormalTok{(i2e\_rez)}

\CommentTok{\# standardisation {-}{-}{-}{-}}
\ControlFlowTok{if}\NormalTok{(}\SpecialCharTok{!}\FunctionTok{require}\NormalTok{(terra)) \{}\FunctionTok{install.packages}\NormalTok{(}\StringTok{"terra"}\NormalTok{); }\FunctionTok{require}\NormalTok{(terra)\}}
\ControlFlowTok{if}\NormalTok{(}\SpecialCharTok{!}\FunctionTok{require}\NormalTok{(tidyverse)) \{}\FunctionTok{install.packages}\NormalTok{(}\StringTok{"tidyverse"}\NormalTok{); }\FunctionTok{require}\NormalTok{(tidyverse)\}}

\NormalTok{nosaukums}\OtherTok{=}\StringTok{"General\_Farmland\_cell.tif"}
\NormalTok{ielasisanas\_cels}\OtherTok{=}\FunctionTok{paste0}\NormalTok{(}\StringTok{"./RasterGrids\_100m/2024/RAW/"}\NormalTok{,nosaukums)}
\NormalTok{saglabasanas\_cels}\OtherTok{=}\FunctionTok{paste0}\NormalTok{(}\StringTok{"./RasterGrids\_100m/2024/Scaled/"}\NormalTok{,nosaukums)}
\NormalTok{slanis}\OtherTok{=}\FunctionTok{rast}\NormalTok{(ielasisanas\_cels)}
\NormalTok{videjais}\OtherTok{=}\FunctionTok{global}\NormalTok{(slanis,}\AttributeTok{fun=}\StringTok{"mean"}\NormalTok{,}\AttributeTok{na.rm=}\ConstantTok{TRUE}\NormalTok{)}
\NormalTok{centrets}\OtherTok{=}\NormalTok{slanis}\SpecialCharTok{{-}}\NormalTok{videjais[,}\DecValTok{1}\NormalTok{]}
\NormalTok{standartnovirze}\OtherTok{=}\NormalTok{terra}\SpecialCharTok{::}\FunctionTok{global}\NormalTok{(centrets,}\AttributeTok{fun=}\StringTok{"rms"}\NormalTok{,}\AttributeTok{na.rm=}\ConstantTok{TRUE}\NormalTok{)}
\NormalTok{merogots}\OtherTok{=}\NormalTok{centrets}\SpecialCharTok{/}\NormalTok{standartnovirze[,}\DecValTok{1}\NormalTok{]}
\FunctionTok{writeRaster}\NormalTok{(merogots,}
      \AttributeTok{filename=}\NormalTok{saglabasanas\_cels,}
      \AttributeTok{overwrite=}\ConstantTok{TRUE}\NormalTok{)}
\end{Highlighting}
\end{Shaded}

\section{General\_Farmland\_r500}\label{ch06.419}

\textbf{filename:} \texttt{General\_Farmland\_r500.tif}

\textbf{layername:} \texttt{egv\_419}

\textbf{English name:} Fractional cover of Farmland within the 0.5 km landscape

\textbf{Latvian name:} Lauksaimniecībā izmantojamo zemju platības īpatsvars 0,5 km
ainavā

\textbf{Procedure:} The cover fraction within a radius of 500 m around the analysis grid cell is
calculated as the area-weighted sum of the \hyperref[ch06.418]{analysis cells} inside the
buffer, using the workflow \texttt{egvtools::radius\_function()}. During the calculation of the landscape metric,
inverse distance weighted (power = 2) gap filling on the output is applied
to ensure no missing values at the edges. Then the layer is rewritten to set
its name. Finally, the layer is standardised by subtracting the arithmetic
mean and dividing by the root mean squared error.

\begin{Shaded}
\begin{Highlighting}[]
\CommentTok{\# libs {-}{-}{-}{-}}
\ControlFlowTok{if}\NormalTok{(}\SpecialCharTok{!}\FunctionTok{require}\NormalTok{(terra)) \{}\FunctionTok{install.packages}\NormalTok{(}\StringTok{"terra"}\NormalTok{); }\FunctionTok{require}\NormalTok{(terra)\}}
\ControlFlowTok{if}\NormalTok{(}\SpecialCharTok{!}\FunctionTok{require}\NormalTok{(egvtools)) \{remotes}\SpecialCharTok{::}\FunctionTok{install\_github}\NormalTok{(}\StringTok{"aavotins/egvtools"}\NormalTok{); }\FunctionTok{require}\NormalTok{(egvtools)\}}


\CommentTok{\# Templates {-}{-}{-}{-}{-}}
\NormalTok{template100}\OtherTok{=}\FunctionTok{rast}\NormalTok{(}\StringTok{"./Templates/TemplateRasters/LV100m\_10km.tif"}\NormalTok{)}

\CommentTok{\# radii {-}{-}{-}{-}}
\FunctionTok{radius\_function}\NormalTok{(}
 \AttributeTok{kvadrati\_path =} \StringTok{"./Templates/TemplateGrids/tiles/"}\NormalTok{,}
 \AttributeTok{radii\_path   =} \StringTok{"./Templates/TemplateGridPoints/tiles/"}\NormalTok{,}
 \AttributeTok{tikls100\_path =} \StringTok{"./Templates/TemplateGrids/tikls100\_sauzeme.parquet"}\NormalTok{,}
 \AttributeTok{template\_path =} \StringTok{"./Templates/TemplateRasters/LV100m\_10km.tif"}\NormalTok{,}
 \AttributeTok{input\_layers  =} \FunctionTok{c}\NormalTok{(}\StringTok{"./RasterGrids\_100m/2024/RAW/General\_Farmland\_cell.tif"}\NormalTok{),}
 \AttributeTok{layer\_prefixes =} \FunctionTok{c}\NormalTok{(}\StringTok{"General\_Farmland"}\NormalTok{),}
 \AttributeTok{output\_dir   =} \StringTok{"./RasterGrids\_100m/2024/RAW/"}\NormalTok{,}
 \AttributeTok{n\_workers   =} \DecValTok{6}\NormalTok{,}
 \AttributeTok{radii     =} \FunctionTok{c}\NormalTok{(}\StringTok{"r500"}\NormalTok{),}
 \AttributeTok{radius\_mode  =} \StringTok{"sparse"}\NormalTok{,}
 \AttributeTok{extract\_fun  =} \StringTok{"mean"}\NormalTok{,}
 \AttributeTok{fill\_missing  =} \ConstantTok{TRUE}\NormalTok{,}
 \AttributeTok{IDW\_weight   =} \DecValTok{2}\NormalTok{,}
 \AttributeTok{future\_max\_size =} \DecValTok{40} \SpecialCharTok{*} \DecValTok{1024}\SpecialCharTok{\^{}}\DecValTok{3}\NormalTok{)}


\CommentTok{\# General\_Farmland\_r500.tif egv\_419}
\NormalTok{slanis}\OtherTok{=}\FunctionTok{rast}\NormalTok{(}\StringTok{"./RasterGrids\_100m/2024/RAW/General\_Farmland\_r500.tif"}\NormalTok{)}
\FunctionTok{names}\NormalTok{(slanis)}\OtherTok{=}\StringTok{"egv\_419"}
\NormalTok{slanis2}\OtherTok{=}\FunctionTok{project}\NormalTok{(slanis,template100)}
\FunctionTok{writeRaster}\NormalTok{(slanis2,}
      \StringTok{"./RasterGrids\_100m/2024/RAW/General\_Farmland\_r500.tif"}\NormalTok{,}
      \AttributeTok{overwrite=}\ConstantTok{TRUE}\NormalTok{)}

\CommentTok{\# standardisation {-}{-}{-}{-}}
\ControlFlowTok{if}\NormalTok{(}\SpecialCharTok{!}\FunctionTok{require}\NormalTok{(terra)) \{}\FunctionTok{install.packages}\NormalTok{(}\StringTok{"terra"}\NormalTok{); }\FunctionTok{require}\NormalTok{(terra)\}}
\ControlFlowTok{if}\NormalTok{(}\SpecialCharTok{!}\FunctionTok{require}\NormalTok{(tidyverse)) \{}\FunctionTok{install.packages}\NormalTok{(}\StringTok{"tidyverse"}\NormalTok{); }\FunctionTok{require}\NormalTok{(tidyverse)\}}

\NormalTok{nosaukums}\OtherTok{=}\StringTok{"General\_Farmland\_r500.tif"}
\NormalTok{ielasisanas\_cels}\OtherTok{=}\FunctionTok{paste0}\NormalTok{(}\StringTok{"./RasterGrids\_100m/2024/RAW/"}\NormalTok{,nosaukums)}
\NormalTok{saglabasanas\_cels}\OtherTok{=}\FunctionTok{paste0}\NormalTok{(}\StringTok{"./RasterGrids\_100m/2024/Scaled/"}\NormalTok{,nosaukums)}
\NormalTok{slanis}\OtherTok{=}\FunctionTok{rast}\NormalTok{(ielasisanas\_cels)}
\NormalTok{videjais}\OtherTok{=}\FunctionTok{global}\NormalTok{(slanis,}\AttributeTok{fun=}\StringTok{"mean"}\NormalTok{,}\AttributeTok{na.rm=}\ConstantTok{TRUE}\NormalTok{)}
\NormalTok{centrets}\OtherTok{=}\NormalTok{slanis}\SpecialCharTok{{-}}\NormalTok{videjais[,}\DecValTok{1}\NormalTok{]}
\NormalTok{standartnovirze}\OtherTok{=}\NormalTok{terra}\SpecialCharTok{::}\FunctionTok{global}\NormalTok{(centrets,}\AttributeTok{fun=}\StringTok{"rms"}\NormalTok{,}\AttributeTok{na.rm=}\ConstantTok{TRUE}\NormalTok{)}
\NormalTok{merogots}\OtherTok{=}\NormalTok{centrets}\SpecialCharTok{/}\NormalTok{standartnovirze[,}\DecValTok{1}\NormalTok{]}
\FunctionTok{writeRaster}\NormalTok{(merogots,}
      \AttributeTok{filename=}\NormalTok{saglabasanas\_cels,}
      \AttributeTok{overwrite=}\ConstantTok{TRUE}\NormalTok{)}
\end{Highlighting}
\end{Shaded}

\section{General\_Farmland\_r1250}\label{ch06.420}

\textbf{filename:} \texttt{General\_Farmland\_r1250.tif}

\textbf{layername:} \texttt{egv\_420}

\textbf{English name:} Fractional cover of Farmland within the 1.25 km landscape

\textbf{Latvian name:} Lauksaimniecībā izmantojamo zemju platības īpatsvars 1,25 km
ainavā

\textbf{Procedure:} The cover fraction within a radius of 1250 m around the analysis grid cell
is calculated as the area-weighted sum of the \hyperref[ch06.418]{analysis cells} inside
the buffer, using the workflow \texttt{egvtools::radius\_function()}. During the calculation of the landscape
metric, inverse distance weighted (power = 2) gap filling on the output is
applied to ensure no missing values at the edges. Then the layer is
rewritten to set its name. Finally, the layer is standardised by
subtracting the arithmetic mean and dividing by the root mean squared error.

\begin{Shaded}
\begin{Highlighting}[]
\CommentTok{\# libs {-}{-}{-}{-}}
\ControlFlowTok{if}\NormalTok{(}\SpecialCharTok{!}\FunctionTok{require}\NormalTok{(terra)) \{}\FunctionTok{install.packages}\NormalTok{(}\StringTok{"terra"}\NormalTok{); }\FunctionTok{require}\NormalTok{(terra)\}}
\ControlFlowTok{if}\NormalTok{(}\SpecialCharTok{!}\FunctionTok{require}\NormalTok{(egvtools)) \{remotes}\SpecialCharTok{::}\FunctionTok{install\_github}\NormalTok{(}\StringTok{"aavotins/egvtools"}\NormalTok{); }\FunctionTok{require}\NormalTok{(egvtools)\}}


\CommentTok{\# Templates {-}{-}{-}{-}{-}}
\NormalTok{template100}\OtherTok{=}\FunctionTok{rast}\NormalTok{(}\StringTok{"./Templates/TemplateRasters/LV100m\_10km.tif"}\NormalTok{)}

\CommentTok{\# radii {-}{-}{-}{-}}
\FunctionTok{radius\_function}\NormalTok{(}
 \AttributeTok{kvadrati\_path =} \StringTok{"./Templates/TemplateGrids/tiles/"}\NormalTok{,}
 \AttributeTok{radii\_path   =} \StringTok{"./Templates/TemplateGridPoints/tiles/"}\NormalTok{,}
 \AttributeTok{tikls100\_path =} \StringTok{"./Templates/TemplateGrids/tikls100\_sauzeme.parquet"}\NormalTok{,}
 \AttributeTok{template\_path =} \StringTok{"./Templates/TemplateRasters/LV100m\_10km.tif"}\NormalTok{,}
 \AttributeTok{input\_layers  =} \FunctionTok{c}\NormalTok{(}\StringTok{"./RasterGrids\_100m/2024/RAW/General\_Farmland\_cell.tif"}\NormalTok{),}
 \AttributeTok{layer\_prefixes =} \FunctionTok{c}\NormalTok{(}\StringTok{"General\_Farmland"}\NormalTok{),}
 \AttributeTok{output\_dir   =} \StringTok{"./RasterGrids\_100m/2024/RAW/"}\NormalTok{,}
 \AttributeTok{n\_workers   =} \DecValTok{6}\NormalTok{,}
 \AttributeTok{radii     =} \FunctionTok{c}\NormalTok{(}\StringTok{"r1250"}\NormalTok{),}
 \AttributeTok{radius\_mode  =} \StringTok{"sparse"}\NormalTok{,}
 \AttributeTok{extract\_fun  =} \StringTok{"mean"}\NormalTok{,}
 \AttributeTok{fill\_missing  =} \ConstantTok{TRUE}\NormalTok{,}
 \AttributeTok{IDW\_weight   =} \DecValTok{2}\NormalTok{,}
 \AttributeTok{future\_max\_size =} \DecValTok{40} \SpecialCharTok{*} \DecValTok{1024}\SpecialCharTok{\^{}}\DecValTok{3}\NormalTok{)}


\CommentTok{\# General\_Farmland\_r1250.tif    egv\_420}
\NormalTok{slanis}\OtherTok{=}\FunctionTok{rast}\NormalTok{(}\StringTok{"./RasterGrids\_100m/2024/RAW/General\_Farmland\_r1250.tif"}\NormalTok{)}
\FunctionTok{names}\NormalTok{(slanis)}\OtherTok{=}\StringTok{"egv\_420"}
\NormalTok{slanis2}\OtherTok{=}\FunctionTok{project}\NormalTok{(slanis,template100)}
\FunctionTok{writeRaster}\NormalTok{(slanis2,}
      \StringTok{"./RasterGrids\_100m/2024/RAW/General\_Farmland\_r1250.tif"}\NormalTok{,}
      \AttributeTok{overwrite=}\ConstantTok{TRUE}\NormalTok{)}

\CommentTok{\# standardisation {-}{-}{-}{-}}
\ControlFlowTok{if}\NormalTok{(}\SpecialCharTok{!}\FunctionTok{require}\NormalTok{(terra)) \{}\FunctionTok{install.packages}\NormalTok{(}\StringTok{"terra"}\NormalTok{); }\FunctionTok{require}\NormalTok{(terra)\}}
\ControlFlowTok{if}\NormalTok{(}\SpecialCharTok{!}\FunctionTok{require}\NormalTok{(tidyverse)) \{}\FunctionTok{install.packages}\NormalTok{(}\StringTok{"tidyverse"}\NormalTok{); }\FunctionTok{require}\NormalTok{(tidyverse)\}}

\NormalTok{nosaukums}\OtherTok{=}\StringTok{"General\_Farmland\_r1250.tif"}
\NormalTok{ielasisanas\_cels}\OtherTok{=}\FunctionTok{paste0}\NormalTok{(}\StringTok{"./RasterGrids\_100m/2024/RAW/"}\NormalTok{,nosaukums)}
\NormalTok{saglabasanas\_cels}\OtherTok{=}\FunctionTok{paste0}\NormalTok{(}\StringTok{"./RasterGrids\_100m/2024/Scaled/"}\NormalTok{,nosaukums)}
\NormalTok{slanis}\OtherTok{=}\FunctionTok{rast}\NormalTok{(ielasisanas\_cels)}
\NormalTok{videjais}\OtherTok{=}\FunctionTok{global}\NormalTok{(slanis,}\AttributeTok{fun=}\StringTok{"mean"}\NormalTok{,}\AttributeTok{na.rm=}\ConstantTok{TRUE}\NormalTok{)}
\NormalTok{centrets}\OtherTok{=}\NormalTok{slanis}\SpecialCharTok{{-}}\NormalTok{videjais[,}\DecValTok{1}\NormalTok{]}
\NormalTok{standartnovirze}\OtherTok{=}\NormalTok{terra}\SpecialCharTok{::}\FunctionTok{global}\NormalTok{(centrets,}\AttributeTok{fun=}\StringTok{"rms"}\NormalTok{,}\AttributeTok{na.rm=}\ConstantTok{TRUE}\NormalTok{)}
\NormalTok{merogots}\OtherTok{=}\NormalTok{centrets}\SpecialCharTok{/}\NormalTok{standartnovirze[,}\DecValTok{1}\NormalTok{]}
\FunctionTok{writeRaster}\NormalTok{(merogots,}
      \AttributeTok{filename=}\NormalTok{saglabasanas\_cels,}
      \AttributeTok{overwrite=}\ConstantTok{TRUE}\NormalTok{)}
\end{Highlighting}
\end{Shaded}

\section{General\_Farmland\_r3000}\label{ch06.421}

\textbf{filename:} \texttt{General\_Farmland\_r3000.tif}

\textbf{layername:} \texttt{egv\_421}

\textbf{English name:} Fractional cover of Farmland within the 3 km landscape

\textbf{Latvian name:} Lauksaimniecībā izmantojamo zemju platības īpatsvars 3 km
ainavā

\textbf{Procedure:} The cover fraction within a radius of 3000 m around the analysis grid cell
is calculated as the area-weighted sum of the \hyperref[ch06.418]{analysis cells} inside
the buffer, using the workflow \texttt{egvtools::radius\_function()}. During the calculation of the landscape
metric, inverse distance weighted (power = 2) gap filling on the output is
applied to ensure no missing values at the edges. Then the layer is
rewritten to set its name. Finally, the layer is standardised by
subtracting the arithmetic mean and dividing by the root mean squared error.

\begin{Shaded}
\begin{Highlighting}[]
\CommentTok{\# libs {-}{-}{-}{-}}
\ControlFlowTok{if}\NormalTok{(}\SpecialCharTok{!}\FunctionTok{require}\NormalTok{(terra)) \{}\FunctionTok{install.packages}\NormalTok{(}\StringTok{"terra"}\NormalTok{); }\FunctionTok{require}\NormalTok{(terra)\}}
\ControlFlowTok{if}\NormalTok{(}\SpecialCharTok{!}\FunctionTok{require}\NormalTok{(egvtools)) \{remotes}\SpecialCharTok{::}\FunctionTok{install\_github}\NormalTok{(}\StringTok{"aavotins/egvtools"}\NormalTok{); }\FunctionTok{require}\NormalTok{(egvtools)\}}


\CommentTok{\# Templates {-}{-}{-}{-}{-}}
\NormalTok{template100}\OtherTok{=}\FunctionTok{rast}\NormalTok{(}\StringTok{"./Templates/TemplateRasters/LV100m\_10km.tif"}\NormalTok{)}

\CommentTok{\# radii {-}{-}{-}{-}}
\FunctionTok{radius\_function}\NormalTok{(}
 \AttributeTok{kvadrati\_path =} \StringTok{"./Templates/TemplateGrids/tiles/"}\NormalTok{,}
 \AttributeTok{radii\_path   =} \StringTok{"./Templates/TemplateGridPoints/tiles/"}\NormalTok{,}
 \AttributeTok{tikls100\_path =} \StringTok{"./Templates/TemplateGrids/tikls100\_sauzeme.parquet"}\NormalTok{,}
 \AttributeTok{template\_path =} \StringTok{"./Templates/TemplateRasters/LV100m\_10km.tif"}\NormalTok{,}
 \AttributeTok{input\_layers  =} \FunctionTok{c}\NormalTok{(}\StringTok{"./RasterGrids\_100m/2024/RAW/General\_Farmland\_cell.tif"}\NormalTok{),}
 \AttributeTok{layer\_prefixes =} \FunctionTok{c}\NormalTok{(}\StringTok{"General\_Farmland"}\NormalTok{),}
 \AttributeTok{output\_dir   =} \StringTok{"./RasterGrids\_100m/2024/RAW/"}\NormalTok{,}
 \AttributeTok{n\_workers   =} \DecValTok{6}\NormalTok{,}
 \AttributeTok{radii     =} \FunctionTok{c}\NormalTok{(}\StringTok{"r3000"}\NormalTok{),}
 \AttributeTok{radius\_mode  =} \StringTok{"sparse"}\NormalTok{,}
 \AttributeTok{extract\_fun  =} \StringTok{"mean"}\NormalTok{,}
 \AttributeTok{fill\_missing  =} \ConstantTok{TRUE}\NormalTok{,}
 \AttributeTok{IDW\_weight   =} \DecValTok{2}\NormalTok{,}
 \AttributeTok{future\_max\_size =} \DecValTok{40} \SpecialCharTok{*} \DecValTok{1024}\SpecialCharTok{\^{}}\DecValTok{3}\NormalTok{)}


\CommentTok{\# General\_Farmland\_r3000.tif    egv\_421}
\NormalTok{slanis}\OtherTok{=}\FunctionTok{rast}\NormalTok{(}\StringTok{"./RasterGrids\_100m/2024/RAW/General\_Farmland\_r3000.tif"}\NormalTok{)}
\FunctionTok{names}\NormalTok{(slanis)}\OtherTok{=}\StringTok{"egv\_421"}
\NormalTok{slanis2}\OtherTok{=}\FunctionTok{project}\NormalTok{(slanis,template100)}
\FunctionTok{writeRaster}\NormalTok{(slanis2,}
      \StringTok{"./RasterGrids\_100m/2024/RAW/General\_Farmland\_r3000.tif"}\NormalTok{,}
      \AttributeTok{overwrite=}\ConstantTok{TRUE}\NormalTok{)}

\CommentTok{\# standardisation {-}{-}{-}{-}}
\ControlFlowTok{if}\NormalTok{(}\SpecialCharTok{!}\FunctionTok{require}\NormalTok{(terra)) \{}\FunctionTok{install.packages}\NormalTok{(}\StringTok{"terra"}\NormalTok{); }\FunctionTok{require}\NormalTok{(terra)\}}
\ControlFlowTok{if}\NormalTok{(}\SpecialCharTok{!}\FunctionTok{require}\NormalTok{(tidyverse)) \{}\FunctionTok{install.packages}\NormalTok{(}\StringTok{"tidyverse"}\NormalTok{); }\FunctionTok{require}\NormalTok{(tidyverse)\}}

\NormalTok{nosaukums}\OtherTok{=}\StringTok{"General\_Farmland\_r3000.tif"}
\NormalTok{ielasisanas\_cels}\OtherTok{=}\FunctionTok{paste0}\NormalTok{(}\StringTok{"./RasterGrids\_100m/2024/RAW/"}\NormalTok{,nosaukums)}
\NormalTok{saglabasanas\_cels}\OtherTok{=}\FunctionTok{paste0}\NormalTok{(}\StringTok{"./RasterGrids\_100m/2024/Scaled/"}\NormalTok{,nosaukums)}
\NormalTok{slanis}\OtherTok{=}\FunctionTok{rast}\NormalTok{(ielasisanas\_cels)}
\NormalTok{videjais}\OtherTok{=}\FunctionTok{global}\NormalTok{(slanis,}\AttributeTok{fun=}\StringTok{"mean"}\NormalTok{,}\AttributeTok{na.rm=}\ConstantTok{TRUE}\NormalTok{)}
\NormalTok{centrets}\OtherTok{=}\NormalTok{slanis}\SpecialCharTok{{-}}\NormalTok{videjais[,}\DecValTok{1}\NormalTok{]}
\NormalTok{standartnovirze}\OtherTok{=}\NormalTok{terra}\SpecialCharTok{::}\FunctionTok{global}\NormalTok{(centrets,}\AttributeTok{fun=}\StringTok{"rms"}\NormalTok{,}\AttributeTok{na.rm=}\ConstantTok{TRUE}\NormalTok{)}
\NormalTok{merogots}\OtherTok{=}\NormalTok{centrets}\SpecialCharTok{/}\NormalTok{standartnovirze[,}\DecValTok{1}\NormalTok{]}
\FunctionTok{writeRaster}\NormalTok{(merogots,}
      \AttributeTok{filename=}\NormalTok{saglabasanas\_cels,}
      \AttributeTok{overwrite=}\ConstantTok{TRUE}\NormalTok{)}
\end{Highlighting}
\end{Shaded}

\section{General\_Farmland\_r10000}\label{ch06.422}

\textbf{filename:} \texttt{General\_Farmland\_r10000.tif}

\textbf{layername:} \texttt{egv\_422}

\textbf{English name:} Fractional cover of Farmland within the 10 km landscape

\textbf{Latvian name:} Lauksaimniecībā izmantojamo zemju platības īpatsvars 10 km
ainavā

\textbf{Procedure:} The cover fraction within a radius of 10000 m around the analysis grid cell
is calculated as the area-weighted sum of the \hyperref[ch06.418]{analysis cells} inside
the buffer, using the workflow \texttt{egvtools::radius\_function()}. During the calculation of the landscape
metric, inverse distance weighted (power = 2) gap filling on the output is
applied to ensure no missing values at the edges. Then the layer is
rewritten to set its name. Finally, the layer is standardised by
subtracting the arithmetic mean and dividing by the root mean squared error.

\begin{Shaded}
\begin{Highlighting}[]
\CommentTok{\# libs {-}{-}{-}{-}}
\ControlFlowTok{if}\NormalTok{(}\SpecialCharTok{!}\FunctionTok{require}\NormalTok{(terra)) \{}\FunctionTok{install.packages}\NormalTok{(}\StringTok{"terra"}\NormalTok{); }\FunctionTok{require}\NormalTok{(terra)\}}
\ControlFlowTok{if}\NormalTok{(}\SpecialCharTok{!}\FunctionTok{require}\NormalTok{(egvtools)) \{remotes}\SpecialCharTok{::}\FunctionTok{install\_github}\NormalTok{(}\StringTok{"aavotins/egvtools"}\NormalTok{); }\FunctionTok{require}\NormalTok{(egvtools)\}}


\CommentTok{\# Templates {-}{-}{-}{-}{-}}
\NormalTok{template100}\OtherTok{=}\FunctionTok{rast}\NormalTok{(}\StringTok{"./Templates/TemplateRasters/LV100m\_10km.tif"}\NormalTok{)}

\CommentTok{\# radii {-}{-}{-}{-}}
\FunctionTok{radius\_function}\NormalTok{(}
 \AttributeTok{kvadrati\_path =} \StringTok{"./Templates/TemplateGrids/tiles/"}\NormalTok{,}
 \AttributeTok{radii\_path   =} \StringTok{"./Templates/TemplateGridPoints/tiles/"}\NormalTok{,}
 \AttributeTok{tikls100\_path =} \StringTok{"./Templates/TemplateGrids/tikls100\_sauzeme.parquet"}\NormalTok{,}
 \AttributeTok{template\_path =} \StringTok{"./Templates/TemplateRasters/LV100m\_10km.tif"}\NormalTok{,}
 \AttributeTok{input\_layers  =} \FunctionTok{c}\NormalTok{(}\StringTok{"./RasterGrids\_100m/2024/RAW/General\_Farmland\_cell.tif"}\NormalTok{),}
 \AttributeTok{layer\_prefixes =} \FunctionTok{c}\NormalTok{(}\StringTok{"General\_Farmland"}\NormalTok{),}
 \AttributeTok{output\_dir   =} \StringTok{"./RasterGrids\_100m/2024/RAW/"}\NormalTok{,}
 \AttributeTok{n\_workers   =} \DecValTok{6}\NormalTok{,}
 \AttributeTok{radii     =} \FunctionTok{c}\NormalTok{(}\StringTok{"r10000"}\NormalTok{),}
 \AttributeTok{radius\_mode  =} \StringTok{"sparse"}\NormalTok{,}
 \AttributeTok{extract\_fun  =} \StringTok{"mean"}\NormalTok{,}
 \AttributeTok{fill\_missing  =} \ConstantTok{TRUE}\NormalTok{,}
 \AttributeTok{IDW\_weight   =} \DecValTok{2}\NormalTok{,}
 \AttributeTok{future\_max\_size =} \DecValTok{40} \SpecialCharTok{*} \DecValTok{1024}\SpecialCharTok{\^{}}\DecValTok{3}\NormalTok{)}


\CommentTok{\# General\_Farmland\_r10000.tif   egv\_422}
\NormalTok{slanis}\OtherTok{=}\FunctionTok{rast}\NormalTok{(}\StringTok{"./RasterGrids\_100m/2024/RAW/General\_Farmland\_r10000.tif"}\NormalTok{)}
\FunctionTok{names}\NormalTok{(slanis)}\OtherTok{=}\StringTok{"egv\_422"}
\NormalTok{slanis2}\OtherTok{=}\FunctionTok{project}\NormalTok{(slanis,template100)}
\FunctionTok{writeRaster}\NormalTok{(slanis2,}
      \StringTok{"./RasterGrids\_100m/2024/RAW/General\_Farmland\_r10000.tif"}\NormalTok{,}
      \AttributeTok{overwrite=}\ConstantTok{TRUE}\NormalTok{)}

\CommentTok{\# standardisation {-}{-}{-}{-}}
\ControlFlowTok{if}\NormalTok{(}\SpecialCharTok{!}\FunctionTok{require}\NormalTok{(terra)) \{}\FunctionTok{install.packages}\NormalTok{(}\StringTok{"terra"}\NormalTok{); }\FunctionTok{require}\NormalTok{(terra)\}}
\ControlFlowTok{if}\NormalTok{(}\SpecialCharTok{!}\FunctionTok{require}\NormalTok{(tidyverse)) \{}\FunctionTok{install.packages}\NormalTok{(}\StringTok{"tidyverse"}\NormalTok{); }\FunctionTok{require}\NormalTok{(tidyverse)\}}

\NormalTok{nosaukums}\OtherTok{=}\StringTok{"General\_Farmland\_r10000.tif"}
\NormalTok{ielasisanas\_cels}\OtherTok{=}\FunctionTok{paste0}\NormalTok{(}\StringTok{"./RasterGrids\_100m/2024/RAW/"}\NormalTok{,nosaukums)}
\NormalTok{saglabasanas\_cels}\OtherTok{=}\FunctionTok{paste0}\NormalTok{(}\StringTok{"./RasterGrids\_100m/2024/Scaled/"}\NormalTok{,nosaukums)}
\NormalTok{slanis}\OtherTok{=}\FunctionTok{rast}\NormalTok{(ielasisanas\_cels)}
\NormalTok{videjais}\OtherTok{=}\FunctionTok{global}\NormalTok{(slanis,}\AttributeTok{fun=}\StringTok{"mean"}\NormalTok{,}\AttributeTok{na.rm=}\ConstantTok{TRUE}\NormalTok{)}
\NormalTok{centrets}\OtherTok{=}\NormalTok{slanis}\SpecialCharTok{{-}}\NormalTok{videjais[,}\DecValTok{1}\NormalTok{]}
\NormalTok{standartnovirze}\OtherTok{=}\NormalTok{terra}\SpecialCharTok{::}\FunctionTok{global}\NormalTok{(centrets,}\AttributeTok{fun=}\StringTok{"rms"}\NormalTok{,}\AttributeTok{na.rm=}\ConstantTok{TRUE}\NormalTok{)}
\NormalTok{merogots}\OtherTok{=}\NormalTok{centrets}\SpecialCharTok{/}\NormalTok{standartnovirze[,}\DecValTok{1}\NormalTok{]}
\FunctionTok{writeRaster}\NormalTok{(merogots,}
      \AttributeTok{filename=}\NormalTok{saglabasanas\_cels,}
      \AttributeTok{overwrite=}\ConstantTok{TRUE}\NormalTok{)}
\end{Highlighting}
\end{Shaded}

\section{General\_ForestsWithoutInventory\_cell}\label{ch06.423}

\textbf{filename:} \texttt{General\_ForestsWithoutInventory\_cell.tif}

\textbf{layername:} \texttt{egv\_423}

\textbf{English name:} Fractional cover of Forests Without Inventory within the
analysis cell (1 ha)

\textbf{Latvian name:} Netaksēto mežu platības īpatsvars analīzes šūnā (1 ha)

\textbf{Procedure:} First, clearcuts and forest stands from the \hyperref[Ch04.01]{State Forest Service's
State Forest Registry} are rasterised to match inputs (presence as value 1; NA
elsewhere). Then, from the \hyperref[Ch05.03]{Landscape classification} class 630 is
reclassified to value 1, others to 0). These layers are then combined so that
values 1 from the second layer, where spatially matching NA values in the first
layer and classified as 1; otherwise 0. The resulting layer
is then aggregated to EGV resolution using the workflow \texttt{egvtools::input2egv()}, which
calculates the arithmetic mean to determine the cover fraction. During
aggregation, inverse distance weighted (power = 2) gap filling on the output is
applied to ensure no missing values at the edges. Finally, the layer is
standardised by subtracting the arithmetic mean and dividing by the root mean squared
error.

\begin{Shaded}
\begin{Highlighting}[]
\CommentTok{\# libs {-}{-}{-}{-}}
\ControlFlowTok{if}\NormalTok{(}\SpecialCharTok{!}\FunctionTok{require}\NormalTok{(egvtools)) \{remotes}\SpecialCharTok{::}\FunctionTok{install\_github}\NormalTok{(}\StringTok{"aavotins/egvtools"}\NormalTok{); }\FunctionTok{require}\NormalTok{(egvtools)\}}
\ControlFlowTok{if}\NormalTok{(}\SpecialCharTok{!}\FunctionTok{require}\NormalTok{(terra)) \{}\FunctionTok{install.packages}\NormalTok{(}\StringTok{"terra"}\NormalTok{); }\FunctionTok{require}\NormalTok{(terra)\}}
\ControlFlowTok{if}\NormalTok{(}\SpecialCharTok{!}\FunctionTok{require}\NormalTok{(sf)) \{}\FunctionTok{install.packages}\NormalTok{(}\StringTok{"sf"}\NormalTok{); }\FunctionTok{require}\NormalTok{(sf)\}}
\ControlFlowTok{if}\NormalTok{(}\SpecialCharTok{!}\FunctionTok{require}\NormalTok{(tidyverse)) \{}\FunctionTok{install.packages}\NormalTok{(}\StringTok{"tidyverse"}\NormalTok{); }\FunctionTok{require}\NormalTok{(tidyverse)\}}
\ControlFlowTok{if}\NormalTok{(}\SpecialCharTok{!}\FunctionTok{require}\NormalTok{(sfarrow)) \{}\FunctionTok{install.packages}\NormalTok{(}\StringTok{"sfarrow"}\NormalTok{); }\FunctionTok{require}\NormalTok{(sfarrow)\}}
\ControlFlowTok{if}\NormalTok{(}\SpecialCharTok{!}\FunctionTok{require}\NormalTok{(readxl)) \{}\FunctionTok{install.packages}\NormalTok{(}\StringTok{"readxl"}\NormalTok{); }\FunctionTok{require}\NormalTok{(readxl)\}}
\ControlFlowTok{if}\NormalTok{(}\SpecialCharTok{!}\FunctionTok{require}\NormalTok{(raster)) \{}\FunctionTok{install.packages}\NormalTok{(}\StringTok{"raster"}\NormalTok{); }\FunctionTok{require}\NormalTok{(raster)\}}
\ControlFlowTok{if}\NormalTok{(}\SpecialCharTok{!}\FunctionTok{require}\NormalTok{(fasterize)) \{}\FunctionTok{install.packages}\NormalTok{(}\StringTok{"fasterize"}\NormalTok{); }\FunctionTok{require}\NormalTok{(fasterize)\}}

\CommentTok{\# templates {-}{-}{-}{-}}
\NormalTok{template100}\OtherTok{=}\FunctionTok{rast}\NormalTok{(}\StringTok{"./Templates/TemplateRasters/LV100m\_10km.tif"}\NormalTok{)}
\NormalTok{template10}\OtherTok{=}\FunctionTok{rast}\NormalTok{(}\StringTok{"./Templates/TemplateRasters/LV10m\_10km.tif"}\NormalTok{)}
\NormalTok{rastrs10}\OtherTok{=}\FunctionTok{raster}\NormalTok{(template10)}

\NormalTok{nulls10}\OtherTok{=}\FunctionTok{rast}\NormalTok{(}\StringTok{"./Templates/TemplateRasters/nulls\_LV10m\_10km.tif"}\NormalTok{)}
\NormalTok{nulls100}\OtherTok{=}\FunctionTok{rast}\NormalTok{(}\StringTok{"./Templates/TemplateRasters/nulls\_LV100m\_10km.tif"}\NormalTok{)}

\CommentTok{\# simple landscape {-}{-}{-}{-}}
\NormalTok{simple\_landscape}\OtherTok{=}\FunctionTok{rast}\NormalTok{(}\StringTok{"RasterGrids\_10m/2024/Ainava\_vienk\_mask.tif"}\NormalTok{)}


\CommentTok{\# General\_ForestsWithoutInventory\_cell.tif  egv\_423 {-}{-}{-}{-}}
\NormalTok{mvr}\OtherTok{=}\FunctionTok{st\_read\_parquet}\NormalTok{(}\StringTok{"./Geodata/2024/MVR/nogabali\_2024janv.parquet"}\NormalTok{)}
\NormalTok{tksetie}\OtherTok{=}\NormalTok{mvr }\SpecialCharTok{\%\textgreater{}\%} 
 \FunctionTok{mutate}\NormalTok{(}\AttributeTok{yes=}\DecValTok{1}\NormalTok{) }\SpecialCharTok{\%\textgreater{}\%} 
 \FunctionTok{filter}\NormalTok{(zkat }\SpecialCharTok{\%in\%} \FunctionTok{c}\NormalTok{(}\StringTok{"10"}\NormalTok{,}\StringTok{"12"}\NormalTok{,}\StringTok{"14"}\NormalTok{,}\StringTok{"16"}\NormalTok{))}
\NormalTok{taksetie\_r}\OtherTok{=}\FunctionTok{fasterize}\NormalTok{(tksetie,rastrs10,}\AttributeTok{field=}\StringTok{"yes"}\NormalTok{,}\AttributeTok{fun=}\StringTok{"first"}\NormalTok{)}
\NormalTok{taksetie\_t}\OtherTok{=}\FunctionTok{rast}\NormalTok{(taksetie\_r)}
\NormalTok{visi\_mezi}\OtherTok{=}\FunctionTok{ifel}\NormalTok{(simple\_landscape}\SpecialCharTok{==}\DecValTok{630}\NormalTok{,}\DecValTok{1}\NormalTok{,}\DecValTok{0}\NormalTok{)}
\NormalTok{netaksetie}\OtherTok{=}\FunctionTok{ifel}\NormalTok{(}\FunctionTok{is.na}\NormalTok{(taksetie\_t)}\SpecialCharTok{\&}\NormalTok{visi\_mezi}\SpecialCharTok{==}\DecValTok{1}\NormalTok{,}\DecValTok{1}\NormalTok{,}\DecValTok{0}\NormalTok{)}
\FunctionTok{plot}\NormalTok{(netaksetie)}

\NormalTok{i2e\_rez}\OtherTok{=}\NormalTok{egvtools}\SpecialCharTok{::}\FunctionTok{input2egv}\NormalTok{(}\AttributeTok{input=}\NormalTok{netaksetie,}
              \AttributeTok{egv\_template=} \StringTok{"./Templates/TemplateRasters/LV100m\_10km.tif"}\NormalTok{,}
              \AttributeTok{summary\_function =} \StringTok{"average"}\NormalTok{,}
              \AttributeTok{missing\_job =} \StringTok{"FillOutput"}\NormalTok{,}
              \AttributeTok{outlocation =} \StringTok{"./RasterGrids\_100m/2024/RAW/"}\NormalTok{,}
              \AttributeTok{outfilename =} \StringTok{"General\_ForestsWithoutInventory\_cell.tif"}\NormalTok{,}
              \AttributeTok{layername =} \StringTok{"egv\_423"}\NormalTok{,}
              \AttributeTok{idw\_weight =} \DecValTok{2}\NormalTok{,}
              \AttributeTok{plot\_gaps =} \ConstantTok{FALSE}\NormalTok{,}\AttributeTok{plot\_final =} \ConstantTok{TRUE}\NormalTok{)}
\NormalTok{i2e\_rez}
\FunctionTok{rm}\NormalTok{(netaksetie)}
\FunctionTok{rm}\NormalTok{(i2e\_rez)}
\FunctionTok{rm}\NormalTok{(mvr)}
\FunctionTok{rm}\NormalTok{(tksetie)}
\FunctionTok{rm}\NormalTok{(taksetie\_r)}
\FunctionTok{rm}\NormalTok{(taksetie\_t)}
\FunctionTok{rm}\NormalTok{(visi\_mezi)}

\CommentTok{\# standardisation {-}{-}{-}{-}}
\ControlFlowTok{if}\NormalTok{(}\SpecialCharTok{!}\FunctionTok{require}\NormalTok{(terra)) \{}\FunctionTok{install.packages}\NormalTok{(}\StringTok{"terra"}\NormalTok{); }\FunctionTok{require}\NormalTok{(terra)\}}
\ControlFlowTok{if}\NormalTok{(}\SpecialCharTok{!}\FunctionTok{require}\NormalTok{(tidyverse)) \{}\FunctionTok{install.packages}\NormalTok{(}\StringTok{"tidyverse"}\NormalTok{); }\FunctionTok{require}\NormalTok{(tidyverse)\}}

\NormalTok{nosaukums}\OtherTok{=}\StringTok{"General\_ForestsWithoutInventory\_cell.tif"}
\NormalTok{ielasisanas\_cels}\OtherTok{=}\FunctionTok{paste0}\NormalTok{(}\StringTok{"./RasterGrids\_100m/2024/RAW/"}\NormalTok{,nosaukums)}
\NormalTok{saglabasanas\_cels}\OtherTok{=}\FunctionTok{paste0}\NormalTok{(}\StringTok{"./RasterGrids\_100m/2024/Scaled/"}\NormalTok{,nosaukums)}
\NormalTok{slanis}\OtherTok{=}\FunctionTok{rast}\NormalTok{(ielasisanas\_cels)}
\NormalTok{videjais}\OtherTok{=}\FunctionTok{global}\NormalTok{(slanis,}\AttributeTok{fun=}\StringTok{"mean"}\NormalTok{,}\AttributeTok{na.rm=}\ConstantTok{TRUE}\NormalTok{)}
\NormalTok{centrets}\OtherTok{=}\NormalTok{slanis}\SpecialCharTok{{-}}\NormalTok{videjais[,}\DecValTok{1}\NormalTok{]}
\NormalTok{standartnovirze}\OtherTok{=}\NormalTok{terra}\SpecialCharTok{::}\FunctionTok{global}\NormalTok{(centrets,}\AttributeTok{fun=}\StringTok{"rms"}\NormalTok{,}\AttributeTok{na.rm=}\ConstantTok{TRUE}\NormalTok{)}
\NormalTok{merogots}\OtherTok{=}\NormalTok{centrets}\SpecialCharTok{/}\NormalTok{standartnovirze[,}\DecValTok{1}\NormalTok{]}
\FunctionTok{writeRaster}\NormalTok{(merogots,}
      \AttributeTok{filename=}\NormalTok{saglabasanas\_cels,}
      \AttributeTok{overwrite=}\ConstantTok{TRUE}\NormalTok{)}
\end{Highlighting}
\end{Shaded}

\section{General\_ForestsWithoutInventory\_r500}\label{ch06.424}

\textbf{filename:} \texttt{General\_ForestsWithoutInventory\_r500.tif}

\textbf{layername:} \texttt{egv\_424}

\textbf{English name:} Fractional cover of Forests Without Inventory within the 0.5
km landscape

\textbf{Latvian name:} Netaksēto mežu platības īpatsvars 0,5 km ainavā

\textbf{Procedure:} The cover fraction within a radius of 500 m around the analysis grid cell is
calculated as the area-weighted sum of the \hyperref[ch06.423]{analysis cells} inside the
buffer, using the workflow \texttt{egvtools::radius\_function()}. During the calculation of the landscape metric,
inverse distance weighted (power = 2) gap filling on the output is applied
to ensure no missing values at the edges. Then the layer is rewritten to set
its name. Finally, the layer is standardised by subtracting the arithmetic
mean and dividing by the root mean squared error.

\begin{Shaded}
\begin{Highlighting}[]
\CommentTok{\# libs {-}{-}{-}{-}}
\ControlFlowTok{if}\NormalTok{(}\SpecialCharTok{!}\FunctionTok{require}\NormalTok{(terra)) \{}\FunctionTok{install.packages}\NormalTok{(}\StringTok{"terra"}\NormalTok{); }\FunctionTok{require}\NormalTok{(terra)\}}
\ControlFlowTok{if}\NormalTok{(}\SpecialCharTok{!}\FunctionTok{require}\NormalTok{(egvtools)) \{remotes}\SpecialCharTok{::}\FunctionTok{install\_github}\NormalTok{(}\StringTok{"aavotins/egvtools"}\NormalTok{); }\FunctionTok{require}\NormalTok{(egvtools)\}}


\CommentTok{\# Templates {-}{-}{-}{-}{-}}
\NormalTok{template100}\OtherTok{=}\FunctionTok{rast}\NormalTok{(}\StringTok{"./Templates/TemplateRasters/LV100m\_10km.tif"}\NormalTok{)}

\CommentTok{\# radii {-}{-}{-}{-}}
\FunctionTok{radius\_function}\NormalTok{(}
 \AttributeTok{kvadrati\_path =} \StringTok{"./Templates/TemplateGrids/tiles/"}\NormalTok{,}
 \AttributeTok{radii\_path   =} \StringTok{"./Templates/TemplateGridPoints/tiles/"}\NormalTok{,}
 \AttributeTok{tikls100\_path =} \StringTok{"./Templates/TemplateGrids/tikls100\_sauzeme.parquet"}\NormalTok{,}
 \AttributeTok{template\_path =} \StringTok{"./Templates/TemplateRasters/LV100m\_10km.tif"}\NormalTok{,}
 \AttributeTok{input\_layers  =} \FunctionTok{c}\NormalTok{(}\StringTok{"./RasterGrids\_100m/2024/RAW/General\_ForestsWithoutInventory\_cell.tif"}\NormalTok{),}
 \AttributeTok{layer\_prefixes =} \FunctionTok{c}\NormalTok{(}\StringTok{"General\_ForestsWithoutInventory"}\NormalTok{),}
 \AttributeTok{output\_dir   =} \StringTok{"./RasterGrids\_100m/2024/RAW/"}\NormalTok{,}
 \AttributeTok{n\_workers   =} \DecValTok{6}\NormalTok{,}
 \AttributeTok{radii     =} \FunctionTok{c}\NormalTok{(}\StringTok{"r500"}\NormalTok{),}
 \AttributeTok{radius\_mode  =} \StringTok{"sparse"}\NormalTok{,}
 \AttributeTok{extract\_fun  =} \StringTok{"mean"}\NormalTok{,}
 \AttributeTok{fill\_missing  =} \ConstantTok{TRUE}\NormalTok{,}
 \AttributeTok{IDW\_weight   =} \DecValTok{2}\NormalTok{,}
 \AttributeTok{future\_max\_size =} \DecValTok{40} \SpecialCharTok{*} \DecValTok{1024}\SpecialCharTok{\^{}}\DecValTok{3}\NormalTok{)}


\CommentTok{\# General\_ForestsWithoutInventory\_r500.tif  egv\_424}
\NormalTok{slanis}\OtherTok{=}\FunctionTok{rast}\NormalTok{(}\StringTok{"./RasterGrids\_100m/2024/RAW/General\_ForestsWithoutInventory\_r500.tif"}\NormalTok{)}
\FunctionTok{names}\NormalTok{(slanis)}\OtherTok{=}\StringTok{"egv\_424"}
\NormalTok{slanis2}\OtherTok{=}\FunctionTok{project}\NormalTok{(slanis,template100)}
\FunctionTok{writeRaster}\NormalTok{(slanis2,}
      \StringTok{"./RasterGrids\_100m/2024/RAW/General\_ForestsWithoutInventory\_r500.tif"}\NormalTok{,}
      \AttributeTok{overwrite=}\ConstantTok{TRUE}\NormalTok{)}

\CommentTok{\# standardisation {-}{-}{-}{-}}
\ControlFlowTok{if}\NormalTok{(}\SpecialCharTok{!}\FunctionTok{require}\NormalTok{(terra)) \{}\FunctionTok{install.packages}\NormalTok{(}\StringTok{"terra"}\NormalTok{); }\FunctionTok{require}\NormalTok{(terra)\}}
\ControlFlowTok{if}\NormalTok{(}\SpecialCharTok{!}\FunctionTok{require}\NormalTok{(tidyverse)) \{}\FunctionTok{install.packages}\NormalTok{(}\StringTok{"tidyverse"}\NormalTok{); }\FunctionTok{require}\NormalTok{(tidyverse)\}}

\NormalTok{nosaukums}\OtherTok{=}\StringTok{"General\_ForestsWithoutInventory\_r500.tif"}
\NormalTok{ielasisanas\_cels}\OtherTok{=}\FunctionTok{paste0}\NormalTok{(}\StringTok{"./RasterGrids\_100m/2024/RAW/"}\NormalTok{,nosaukums)}
\NormalTok{saglabasanas\_cels}\OtherTok{=}\FunctionTok{paste0}\NormalTok{(}\StringTok{"./RasterGrids\_100m/2024/Scaled/"}\NormalTok{,nosaukums)}
\NormalTok{slanis}\OtherTok{=}\FunctionTok{rast}\NormalTok{(ielasisanas\_cels)}
\NormalTok{videjais}\OtherTok{=}\FunctionTok{global}\NormalTok{(slanis,}\AttributeTok{fun=}\StringTok{"mean"}\NormalTok{,}\AttributeTok{na.rm=}\ConstantTok{TRUE}\NormalTok{)}
\NormalTok{centrets}\OtherTok{=}\NormalTok{slanis}\SpecialCharTok{{-}}\NormalTok{videjais[,}\DecValTok{1}\NormalTok{]}
\NormalTok{standartnovirze}\OtherTok{=}\NormalTok{terra}\SpecialCharTok{::}\FunctionTok{global}\NormalTok{(centrets,}\AttributeTok{fun=}\StringTok{"rms"}\NormalTok{,}\AttributeTok{na.rm=}\ConstantTok{TRUE}\NormalTok{)}
\NormalTok{merogots}\OtherTok{=}\NormalTok{centrets}\SpecialCharTok{/}\NormalTok{standartnovirze[,}\DecValTok{1}\NormalTok{]}
\FunctionTok{writeRaster}\NormalTok{(merogots,}
      \AttributeTok{filename=}\NormalTok{saglabasanas\_cels,}
      \AttributeTok{overwrite=}\ConstantTok{TRUE}\NormalTok{)}
\end{Highlighting}
\end{Shaded}

\section{General\_ForestsWithoutInventory\_r1250}\label{ch06.425}

\textbf{filename:} \texttt{General\_ForestsWithoutInventory\_r1250.tif}

\textbf{layername:} \texttt{egv\_425}

\textbf{English name:} Fractional cover of Forests Without Inventory within the 1.25
km landscape

\textbf{Latvian name:} Netaksēto mežu platības īpatsvars 1,25 km ainavā

\textbf{Procedure:} The cover fraction within a radius of 1250 m around the analysis grid cell
is calculated as the area-weighted sum of the \hyperref[ch06.423]{analysis cells} inside
the buffer, using the workflow \texttt{egvtools::radius\_function()}. During the calculation of the landscape
metric, inverse distance weighted (power = 2) gap filling on the output is
applied to ensure no missing values at the edges. Then the layer is
rewritten to set its name. Finally, the layer is standardised by
subtracting the arithmetic mean and dividing by the root mean squared error.

\begin{Shaded}
\begin{Highlighting}[]
\CommentTok{\# libs {-}{-}{-}{-}}
\ControlFlowTok{if}\NormalTok{(}\SpecialCharTok{!}\FunctionTok{require}\NormalTok{(terra)) \{}\FunctionTok{install.packages}\NormalTok{(}\StringTok{"terra"}\NormalTok{); }\FunctionTok{require}\NormalTok{(terra)\}}
\ControlFlowTok{if}\NormalTok{(}\SpecialCharTok{!}\FunctionTok{require}\NormalTok{(egvtools)) \{remotes}\SpecialCharTok{::}\FunctionTok{install\_github}\NormalTok{(}\StringTok{"aavotins/egvtools"}\NormalTok{); }\FunctionTok{require}\NormalTok{(egvtools)\}}


\CommentTok{\# Templates {-}{-}{-}{-}{-}}
\NormalTok{template100}\OtherTok{=}\FunctionTok{rast}\NormalTok{(}\StringTok{"./Templates/TemplateRasters/LV100m\_10km.tif"}\NormalTok{)}

\CommentTok{\# radii {-}{-}{-}{-}}
\FunctionTok{radius\_function}\NormalTok{(}
 \AttributeTok{kvadrati\_path =} \StringTok{"./Templates/TemplateGrids/tiles/"}\NormalTok{,}
 \AttributeTok{radii\_path   =} \StringTok{"./Templates/TemplateGridPoints/tiles/"}\NormalTok{,}
 \AttributeTok{tikls100\_path =} \StringTok{"./Templates/TemplateGrids/tikls100\_sauzeme.parquet"}\NormalTok{,}
 \AttributeTok{template\_path =} \StringTok{"./Templates/TemplateRasters/LV100m\_10km.tif"}\NormalTok{,}
 \AttributeTok{input\_layers  =} \FunctionTok{c}\NormalTok{(}\StringTok{"./RasterGrids\_100m/2024/RAW/General\_ForestsWithoutInventory\_cell.tif"}\NormalTok{),}
 \AttributeTok{layer\_prefixes =} \FunctionTok{c}\NormalTok{(}\StringTok{"General\_ForestsWithoutInventory"}\NormalTok{),}
 \AttributeTok{output\_dir   =} \StringTok{"./RasterGrids\_100m/2024/RAW/"}\NormalTok{,}
 \AttributeTok{n\_workers   =} \DecValTok{6}\NormalTok{,}
 \AttributeTok{radii     =} \FunctionTok{c}\NormalTok{(}\StringTok{"r1250"}\NormalTok{),}
 \AttributeTok{radius\_mode  =} \StringTok{"sparse"}\NormalTok{,}
 \AttributeTok{extract\_fun  =} \StringTok{"mean"}\NormalTok{,}
 \AttributeTok{fill\_missing  =} \ConstantTok{TRUE}\NormalTok{,}
 \AttributeTok{IDW\_weight   =} \DecValTok{2}\NormalTok{,}
 \AttributeTok{future\_max\_size =} \DecValTok{40} \SpecialCharTok{*} \DecValTok{1024}\SpecialCharTok{\^{}}\DecValTok{3}\NormalTok{)}


\CommentTok{\# General\_ForestsWithoutInventory\_r1250.tif egv\_425}
\NormalTok{slanis}\OtherTok{=}\FunctionTok{rast}\NormalTok{(}\StringTok{"./RasterGrids\_100m/2024/RAW/General\_ForestsWithoutInventory\_r1250.tif"}\NormalTok{)}
\FunctionTok{names}\NormalTok{(slanis)}\OtherTok{=}\StringTok{"egv\_425"}
\NormalTok{slanis2}\OtherTok{=}\FunctionTok{project}\NormalTok{(slanis,template100)}
\FunctionTok{writeRaster}\NormalTok{(slanis2,}
      \StringTok{"./RasterGrids\_100m/2024/RAW/General\_ForestsWithoutInventory\_r1250.tif"}\NormalTok{,}
      \AttributeTok{overwrite=}\ConstantTok{TRUE}\NormalTok{)}

\CommentTok{\# standardisation {-}{-}{-}{-}}
\ControlFlowTok{if}\NormalTok{(}\SpecialCharTok{!}\FunctionTok{require}\NormalTok{(terra)) \{}\FunctionTok{install.packages}\NormalTok{(}\StringTok{"terra"}\NormalTok{); }\FunctionTok{require}\NormalTok{(terra)\}}
\ControlFlowTok{if}\NormalTok{(}\SpecialCharTok{!}\FunctionTok{require}\NormalTok{(tidyverse)) \{}\FunctionTok{install.packages}\NormalTok{(}\StringTok{"tidyverse"}\NormalTok{); }\FunctionTok{require}\NormalTok{(tidyverse)\}}

\NormalTok{nosaukums}\OtherTok{=}\StringTok{"General\_ForestsWithoutInventory\_r1250.tif"}
\NormalTok{ielasisanas\_cels}\OtherTok{=}\FunctionTok{paste0}\NormalTok{(}\StringTok{"./RasterGrids\_100m/2024/RAW/"}\NormalTok{,nosaukums)}
\NormalTok{saglabasanas\_cels}\OtherTok{=}\FunctionTok{paste0}\NormalTok{(}\StringTok{"./RasterGrids\_100m/2024/Scaled/"}\NormalTok{,nosaukums)}
\NormalTok{slanis}\OtherTok{=}\FunctionTok{rast}\NormalTok{(ielasisanas\_cels)}
\NormalTok{videjais}\OtherTok{=}\FunctionTok{global}\NormalTok{(slanis,}\AttributeTok{fun=}\StringTok{"mean"}\NormalTok{,}\AttributeTok{na.rm=}\ConstantTok{TRUE}\NormalTok{)}
\NormalTok{centrets}\OtherTok{=}\NormalTok{slanis}\SpecialCharTok{{-}}\NormalTok{videjais[,}\DecValTok{1}\NormalTok{]}
\NormalTok{standartnovirze}\OtherTok{=}\NormalTok{terra}\SpecialCharTok{::}\FunctionTok{global}\NormalTok{(centrets,}\AttributeTok{fun=}\StringTok{"rms"}\NormalTok{,}\AttributeTok{na.rm=}\ConstantTok{TRUE}\NormalTok{)}
\NormalTok{merogots}\OtherTok{=}\NormalTok{centrets}\SpecialCharTok{/}\NormalTok{standartnovirze[,}\DecValTok{1}\NormalTok{]}
\FunctionTok{writeRaster}\NormalTok{(merogots,}
      \AttributeTok{filename=}\NormalTok{saglabasanas\_cels,}
      \AttributeTok{overwrite=}\ConstantTok{TRUE}\NormalTok{)}
\end{Highlighting}
\end{Shaded}

\section{General\_ForestsWithoutInventory\_r3000}\label{ch06.426}

\textbf{filename:} \texttt{General\_ForestsWithoutInventory\_r3000.tif}

\textbf{layername:} \texttt{egv\_426}

\textbf{English name:} Fractional cover of Forests Without Inventory within the 3 km
landscape

\textbf{Latvian name:} Netaksēto mežu platības īpatsvars 3 km ainavā

\textbf{Procedure:} The cover fraction within a radius of 3000 m around the analysis grid cell
is calculated as the area-weighted sum of the \hyperref[ch06.423]{analysis cells} inside
the buffer, using the workflow \texttt{egvtools::radius\_function()}. During the calculation of the landscape
metric, inverse distance weighted (power = 2) gap filling on the output is
applied to ensure no missing values at the edges. Then the layer is
rewritten to set its name. Finally, the layer is standardised by
subtracting the arithmetic mean and dividing by the root mean squared error.

\begin{Shaded}
\begin{Highlighting}[]
\CommentTok{\# libs {-}{-}{-}{-}}
\ControlFlowTok{if}\NormalTok{(}\SpecialCharTok{!}\FunctionTok{require}\NormalTok{(terra)) \{}\FunctionTok{install.packages}\NormalTok{(}\StringTok{"terra"}\NormalTok{); }\FunctionTok{require}\NormalTok{(terra)\}}
\ControlFlowTok{if}\NormalTok{(}\SpecialCharTok{!}\FunctionTok{require}\NormalTok{(egvtools)) \{remotes}\SpecialCharTok{::}\FunctionTok{install\_github}\NormalTok{(}\StringTok{"aavotins/egvtools"}\NormalTok{); }\FunctionTok{require}\NormalTok{(egvtools)\}}


\CommentTok{\# Templates {-}{-}{-}{-}{-}}
\NormalTok{template100}\OtherTok{=}\FunctionTok{rast}\NormalTok{(}\StringTok{"./Templates/TemplateRasters/LV100m\_10km.tif"}\NormalTok{)}

\CommentTok{\# radii {-}{-}{-}{-}}
\FunctionTok{radius\_function}\NormalTok{(}
 \AttributeTok{kvadrati\_path =} \StringTok{"./Templates/TemplateGrids/tiles/"}\NormalTok{,}
 \AttributeTok{radii\_path   =} \StringTok{"./Templates/TemplateGridPoints/tiles/"}\NormalTok{,}
 \AttributeTok{tikls100\_path =} \StringTok{"./Templates/TemplateGrids/tikls100\_sauzeme.parquet"}\NormalTok{,}
 \AttributeTok{template\_path =} \StringTok{"./Templates/TemplateRasters/LV100m\_10km.tif"}\NormalTok{,}
 \AttributeTok{input\_layers  =} \FunctionTok{c}\NormalTok{(}\StringTok{"./RasterGrids\_100m/2024/RAW/General\_ForestsWithoutInventory\_cell.tif"}\NormalTok{),}
 \AttributeTok{layer\_prefixes =} \FunctionTok{c}\NormalTok{(}\StringTok{"General\_ForestsWithoutInventory"}\NormalTok{),}
 \AttributeTok{output\_dir   =} \StringTok{"./RasterGrids\_100m/2024/RAW/"}\NormalTok{,}
 \AttributeTok{n\_workers   =} \DecValTok{6}\NormalTok{,}
 \AttributeTok{radii     =} \FunctionTok{c}\NormalTok{(}\StringTok{"r3000"}\NormalTok{),}
 \AttributeTok{radius\_mode  =} \StringTok{"sparse"}\NormalTok{,}
 \AttributeTok{extract\_fun  =} \StringTok{"mean"}\NormalTok{,}
 \AttributeTok{fill\_missing  =} \ConstantTok{TRUE}\NormalTok{,}
 \AttributeTok{IDW\_weight   =} \DecValTok{2}\NormalTok{,}
 \AttributeTok{future\_max\_size =} \DecValTok{40} \SpecialCharTok{*} \DecValTok{1024}\SpecialCharTok{\^{}}\DecValTok{3}\NormalTok{)}


\CommentTok{\# General\_ForestsWithoutInventory\_r3000.tif egv\_426}
\NormalTok{slanis}\OtherTok{=}\FunctionTok{rast}\NormalTok{(}\StringTok{"./RasterGrids\_100m/2024/RAW/General\_ForestsWithoutInventory\_r3000.tif"}\NormalTok{)}
\FunctionTok{names}\NormalTok{(slanis)}\OtherTok{=}\StringTok{"egv\_426"}
\NormalTok{slanis2}\OtherTok{=}\FunctionTok{project}\NormalTok{(slanis,template100)}
\FunctionTok{writeRaster}\NormalTok{(slanis2,}
      \StringTok{"./RasterGrids\_100m/2024/RAW/General\_ForestsWithoutInventory\_r3000.tif"}\NormalTok{,}
      \AttributeTok{overwrite=}\ConstantTok{TRUE}\NormalTok{)}

\CommentTok{\# standardisation {-}{-}{-}{-}}
\ControlFlowTok{if}\NormalTok{(}\SpecialCharTok{!}\FunctionTok{require}\NormalTok{(terra)) \{}\FunctionTok{install.packages}\NormalTok{(}\StringTok{"terra"}\NormalTok{); }\FunctionTok{require}\NormalTok{(terra)\}}
\ControlFlowTok{if}\NormalTok{(}\SpecialCharTok{!}\FunctionTok{require}\NormalTok{(tidyverse)) \{}\FunctionTok{install.packages}\NormalTok{(}\StringTok{"tidyverse"}\NormalTok{); }\FunctionTok{require}\NormalTok{(tidyverse)\}}

\NormalTok{nosaukums}\OtherTok{=}\StringTok{"General\_ForestsWithoutInventory\_r3000.tif"}
\NormalTok{ielasisanas\_cels}\OtherTok{=}\FunctionTok{paste0}\NormalTok{(}\StringTok{"./RasterGrids\_100m/2024/RAW/"}\NormalTok{,nosaukums)}
\NormalTok{saglabasanas\_cels}\OtherTok{=}\FunctionTok{paste0}\NormalTok{(}\StringTok{"./RasterGrids\_100m/2024/Scaled/"}\NormalTok{,nosaukums)}
\NormalTok{slanis}\OtherTok{=}\FunctionTok{rast}\NormalTok{(ielasisanas\_cels)}
\NormalTok{videjais}\OtherTok{=}\FunctionTok{global}\NormalTok{(slanis,}\AttributeTok{fun=}\StringTok{"mean"}\NormalTok{,}\AttributeTok{na.rm=}\ConstantTok{TRUE}\NormalTok{)}
\NormalTok{centrets}\OtherTok{=}\NormalTok{slanis}\SpecialCharTok{{-}}\NormalTok{videjais[,}\DecValTok{1}\NormalTok{]}
\NormalTok{standartnovirze}\OtherTok{=}\NormalTok{terra}\SpecialCharTok{::}\FunctionTok{global}\NormalTok{(centrets,}\AttributeTok{fun=}\StringTok{"rms"}\NormalTok{,}\AttributeTok{na.rm=}\ConstantTok{TRUE}\NormalTok{)}
\NormalTok{merogots}\OtherTok{=}\NormalTok{centrets}\SpecialCharTok{/}\NormalTok{standartnovirze[,}\DecValTok{1}\NormalTok{]}
\FunctionTok{writeRaster}\NormalTok{(merogots,}
      \AttributeTok{filename=}\NormalTok{saglabasanas\_cels,}
      \AttributeTok{overwrite=}\ConstantTok{TRUE}\NormalTok{)}
\end{Highlighting}
\end{Shaded}

\section{General\_ForestsWithoutInventory\_r10000}\label{ch06.427}

\textbf{filename:} \texttt{General\_ForestsWithoutInventory\_r10000.tif}

\textbf{layername:} \texttt{egv\_427}

\textbf{English name:} Fractional cover of Forests Without Inventory within the 10 km
landscape

\textbf{Latvian name:} Netaksēto mežu platības īpatsvars 10 km ainavā

\textbf{Procedure:} The cover fraction within a radius of 10000 m around the analysis grid cell
is calculated as the area-weighted sum of the \hyperref[ch06.423]{analysis cells} inside
the buffer, using the workflow \texttt{egvtools::radius\_function()}. During the calculation of the landscape
metric, inverse distance weighted (power = 2) gap filling on the output is
applied to ensure no missing values at the edges. Then the layer is
rewritten to set its name. Finally, the layer is standardised by
subtracting the arithmetic mean and dividing by the root mean squared error.

\begin{Shaded}
\begin{Highlighting}[]
\CommentTok{\# libs {-}{-}{-}{-}}
\ControlFlowTok{if}\NormalTok{(}\SpecialCharTok{!}\FunctionTok{require}\NormalTok{(terra)) \{}\FunctionTok{install.packages}\NormalTok{(}\StringTok{"terra"}\NormalTok{); }\FunctionTok{require}\NormalTok{(terra)\}}
\ControlFlowTok{if}\NormalTok{(}\SpecialCharTok{!}\FunctionTok{require}\NormalTok{(egvtools)) \{remotes}\SpecialCharTok{::}\FunctionTok{install\_github}\NormalTok{(}\StringTok{"aavotins/egvtools"}\NormalTok{); }\FunctionTok{require}\NormalTok{(egvtools)\}}


\CommentTok{\# Templates {-}{-}{-}{-}{-}}
\NormalTok{template100}\OtherTok{=}\FunctionTok{rast}\NormalTok{(}\StringTok{"./Templates/TemplateRasters/LV100m\_10km.tif"}\NormalTok{)}

\CommentTok{\# radii {-}{-}{-}{-}}
\FunctionTok{radius\_function}\NormalTok{(}
 \AttributeTok{kvadrati\_path =} \StringTok{"./Templates/TemplateGrids/tiles/"}\NormalTok{,}
 \AttributeTok{radii\_path   =} \StringTok{"./Templates/TemplateGridPoints/tiles/"}\NormalTok{,}
 \AttributeTok{tikls100\_path =} \StringTok{"./Templates/TemplateGrids/tikls100\_sauzeme.parquet"}\NormalTok{,}
 \AttributeTok{template\_path =} \StringTok{"./Templates/TemplateRasters/LV100m\_10km.tif"}\NormalTok{,}
 \AttributeTok{input\_layers  =} \FunctionTok{c}\NormalTok{(}\StringTok{"./RasterGrids\_100m/2024/RAW/General\_ForestsWithoutInventory\_cell.tif"}\NormalTok{),}
 \AttributeTok{layer\_prefixes =} \FunctionTok{c}\NormalTok{(}\StringTok{"General\_ForestsWithoutInventory"}\NormalTok{),}
 \AttributeTok{output\_dir   =} \StringTok{"./RasterGrids\_100m/2024/RAW/"}\NormalTok{,}
 \AttributeTok{n\_workers   =} \DecValTok{6}\NormalTok{,}
 \AttributeTok{radii     =} \FunctionTok{c}\NormalTok{(}\StringTok{"r10000"}\NormalTok{),}
 \AttributeTok{radius\_mode  =} \StringTok{"sparse"}\NormalTok{,}
 \AttributeTok{extract\_fun  =} \StringTok{"mean"}\NormalTok{,}
 \AttributeTok{fill\_missing  =} \ConstantTok{TRUE}\NormalTok{,}
 \AttributeTok{IDW\_weight   =} \DecValTok{2}\NormalTok{,}
 \AttributeTok{future\_max\_size =} \DecValTok{40} \SpecialCharTok{*} \DecValTok{1024}\SpecialCharTok{\^{}}\DecValTok{3}\NormalTok{)}


\CommentTok{\# General\_ForestsWithoutInventory\_r10000.tif    egv\_427}
\NormalTok{slanis}\OtherTok{=}\FunctionTok{rast}\NormalTok{(}\StringTok{"./RasterGrids\_100m/2024/RAW/General\_ForestsWithoutInventory\_r10000.tif"}\NormalTok{)}
\FunctionTok{names}\NormalTok{(slanis)}\OtherTok{=}\StringTok{"egv\_427"}
\NormalTok{slanis2}\OtherTok{=}\FunctionTok{project}\NormalTok{(slanis,template100)}
\FunctionTok{writeRaster}\NormalTok{(slanis2,}
      \StringTok{"./RasterGrids\_100m/2024/RAW/General\_ForestsWithoutInventory\_r10000.tif"}\NormalTok{,}
      \AttributeTok{overwrite=}\ConstantTok{TRUE}\NormalTok{)}

\CommentTok{\# standardisation {-}{-}{-}{-}}
\ControlFlowTok{if}\NormalTok{(}\SpecialCharTok{!}\FunctionTok{require}\NormalTok{(terra)) \{}\FunctionTok{install.packages}\NormalTok{(}\StringTok{"terra"}\NormalTok{); }\FunctionTok{require}\NormalTok{(terra)\}}
\ControlFlowTok{if}\NormalTok{(}\SpecialCharTok{!}\FunctionTok{require}\NormalTok{(tidyverse)) \{}\FunctionTok{install.packages}\NormalTok{(}\StringTok{"tidyverse"}\NormalTok{); }\FunctionTok{require}\NormalTok{(tidyverse)\}}

\NormalTok{nosaukums}\OtherTok{=}\StringTok{"General\_ForestsWithoutInventory\_r10000.tif"}
\NormalTok{ielasisanas\_cels}\OtherTok{=}\FunctionTok{paste0}\NormalTok{(}\StringTok{"./RasterGrids\_100m/2024/RAW/"}\NormalTok{,nosaukums)}
\NormalTok{saglabasanas\_cels}\OtherTok{=}\FunctionTok{paste0}\NormalTok{(}\StringTok{"./RasterGrids\_100m/2024/Scaled/"}\NormalTok{,nosaukums)}
\NormalTok{slanis}\OtherTok{=}\FunctionTok{rast}\NormalTok{(ielasisanas\_cels)}
\NormalTok{videjais}\OtherTok{=}\FunctionTok{global}\NormalTok{(slanis,}\AttributeTok{fun=}\StringTok{"mean"}\NormalTok{,}\AttributeTok{na.rm=}\ConstantTok{TRUE}\NormalTok{)}
\NormalTok{centrets}\OtherTok{=}\NormalTok{slanis}\SpecialCharTok{{-}}\NormalTok{videjais[,}\DecValTok{1}\NormalTok{]}
\NormalTok{standartnovirze}\OtherTok{=}\NormalTok{terra}\SpecialCharTok{::}\FunctionTok{global}\NormalTok{(centrets,}\AttributeTok{fun=}\StringTok{"rms"}\NormalTok{,}\AttributeTok{na.rm=}\ConstantTok{TRUE}\NormalTok{)}
\NormalTok{merogots}\OtherTok{=}\NormalTok{centrets}\SpecialCharTok{/}\NormalTok{standartnovirze[,}\DecValTok{1}\NormalTok{]}
\FunctionTok{writeRaster}\NormalTok{(merogots,}
      \AttributeTok{filename=}\NormalTok{saglabasanas\_cels,}
      \AttributeTok{overwrite=}\ConstantTok{TRUE}\NormalTok{)}
\end{Highlighting}
\end{Shaded}

\section{General\_GardensOrchards\_cell}\label{ch06.428}

\textbf{filename:} \texttt{General\_GardensOrchards\_cell.tif}

\textbf{layername:} \texttt{egv\_428}

\textbf{English name:} Fractional cover of Allotment gardens, Orchards within the
analysis cell (1 ha)

\textbf{Latvian name:} Vasarnīcu kompleksu un augļudārzu platības īpatsvars analīzes
šūnā (1 ha)

\textbf{Procedure:} First, the allotment gardens and orchards from the \hyperref[Ch05.03]{Landscape
classification} are selected (values between 400 and 500 are
reclassified to value 1; all others are set to 0). The resulting layer
is then aggregated to EGV resolution using the workflow \texttt{egvtools::input2egv()}, which
calculates the arithmetic mean to determine the cover fraction. During
aggregation, inverse distance weighted (power = 2) gap filling on the output is
applied to ensure no missing values at the edges. Finally, the layer is
standardised by subtracting the arithmetic mean and dividing by the root mean squared
error.

\begin{Shaded}
\begin{Highlighting}[]
\CommentTok{\# libs {-}{-}{-}{-}}
\ControlFlowTok{if}\NormalTok{(}\SpecialCharTok{!}\FunctionTok{require}\NormalTok{(egvtools)) \{remotes}\SpecialCharTok{::}\FunctionTok{install\_github}\NormalTok{(}\StringTok{"aavotins/egvtools"}\NormalTok{); }\FunctionTok{require}\NormalTok{(egvtools)\}}
\ControlFlowTok{if}\NormalTok{(}\SpecialCharTok{!}\FunctionTok{require}\NormalTok{(terra)) \{}\FunctionTok{install.packages}\NormalTok{(}\StringTok{"terra"}\NormalTok{); }\FunctionTok{require}\NormalTok{(terra)\}}
\ControlFlowTok{if}\NormalTok{(}\SpecialCharTok{!}\FunctionTok{require}\NormalTok{(sf)) \{}\FunctionTok{install.packages}\NormalTok{(}\StringTok{"sf"}\NormalTok{); }\FunctionTok{require}\NormalTok{(sf)\}}
\ControlFlowTok{if}\NormalTok{(}\SpecialCharTok{!}\FunctionTok{require}\NormalTok{(tidyverse)) \{}\FunctionTok{install.packages}\NormalTok{(}\StringTok{"tidyverse"}\NormalTok{); }\FunctionTok{require}\NormalTok{(tidyverse)\}}
\ControlFlowTok{if}\NormalTok{(}\SpecialCharTok{!}\FunctionTok{require}\NormalTok{(sfarrow)) \{}\FunctionTok{install.packages}\NormalTok{(}\StringTok{"sfarrow"}\NormalTok{); }\FunctionTok{require}\NormalTok{(sfarrow)\}}
\ControlFlowTok{if}\NormalTok{(}\SpecialCharTok{!}\FunctionTok{require}\NormalTok{(readxl)) \{}\FunctionTok{install.packages}\NormalTok{(}\StringTok{"readxl"}\NormalTok{); }\FunctionTok{require}\NormalTok{(readxl)\}}
\ControlFlowTok{if}\NormalTok{(}\SpecialCharTok{!}\FunctionTok{require}\NormalTok{(raster)) \{}\FunctionTok{install.packages}\NormalTok{(}\StringTok{"raster"}\NormalTok{); }\FunctionTok{require}\NormalTok{(raster)\}}
\ControlFlowTok{if}\NormalTok{(}\SpecialCharTok{!}\FunctionTok{require}\NormalTok{(fasterize)) \{}\FunctionTok{install.packages}\NormalTok{(}\StringTok{"fasterize"}\NormalTok{); }\FunctionTok{require}\NormalTok{(fasterize)\}}

\CommentTok{\# templates {-}{-}{-}{-}}
\NormalTok{template100}\OtherTok{=}\FunctionTok{rast}\NormalTok{(}\StringTok{"./Templates/TemplateRasters/LV100m\_10km.tif"}\NormalTok{)}
\NormalTok{template10}\OtherTok{=}\FunctionTok{rast}\NormalTok{(}\StringTok{"./Templates/TemplateRasters/LV10m\_10km.tif"}\NormalTok{)}
\NormalTok{rastrs10}\OtherTok{=}\FunctionTok{raster}\NormalTok{(template10)}

\NormalTok{nulls10}\OtherTok{=}\FunctionTok{rast}\NormalTok{(}\StringTok{"./Templates/TemplateRasters/nulls\_LV10m\_10km.tif"}\NormalTok{)}
\NormalTok{nulls100}\OtherTok{=}\FunctionTok{rast}\NormalTok{(}\StringTok{"./Templates/TemplateRasters/nulls\_LV100m\_10km.tif"}\NormalTok{)}

\CommentTok{\# simple landscape {-}{-}{-}{-}}
\NormalTok{simple\_landscape}\OtherTok{=}\FunctionTok{rast}\NormalTok{(}\StringTok{"RasterGrids\_10m/2024/Ainava\_vienk\_mask.tif"}\NormalTok{)}


\CommentTok{\# General\_GardensOrchards\_cell.tif  egv\_428 {-}{-}{-}{-}}
\NormalTok{parejie}\OtherTok{=}\FunctionTok{ifel}\NormalTok{(simple\_landscape}\SpecialCharTok{\textgreater{}=}\DecValTok{400}\SpecialCharTok{\&}\NormalTok{simple\_landscape}\SpecialCharTok{\textless{}}\DecValTok{500}\NormalTok{,}\DecValTok{1}\NormalTok{,}\DecValTok{0}\NormalTok{)}
\NormalTok{i2e\_rez}\OtherTok{=}\NormalTok{egvtools}\SpecialCharTok{::}\FunctionTok{input2egv}\NormalTok{(}\AttributeTok{input=}\NormalTok{parejie,}
              \AttributeTok{egv\_template=} \StringTok{"./Templates/TemplateRasters/LV100m\_10km.tif"}\NormalTok{,}
              \AttributeTok{summary\_function =} \StringTok{"average"}\NormalTok{,}
              \AttributeTok{missing\_job =} \StringTok{"FillOutput"}\NormalTok{,}
              \AttributeTok{outlocation =} \StringTok{"./RasterGrids\_100m/2024/RAW/"}\NormalTok{,}
              \AttributeTok{outfilename =} \StringTok{"General\_GardensOrchards\_cell.tif"}\NormalTok{,}
              \AttributeTok{layername =} \StringTok{"egv\_428"}\NormalTok{,}
              \AttributeTok{idw\_weight =} \DecValTok{2}\NormalTok{,}
              \AttributeTok{plot\_gaps =} \ConstantTok{FALSE}\NormalTok{,}\AttributeTok{plot\_final =} \ConstantTok{TRUE}\NormalTok{)}
\NormalTok{i2e\_rez}
\FunctionTok{rm}\NormalTok{(parejie)}
\FunctionTok{rm}\NormalTok{(i2e\_rez)}

\CommentTok{\# standardisation {-}{-}{-}{-}}
\ControlFlowTok{if}\NormalTok{(}\SpecialCharTok{!}\FunctionTok{require}\NormalTok{(terra)) \{}\FunctionTok{install.packages}\NormalTok{(}\StringTok{"terra"}\NormalTok{); }\FunctionTok{require}\NormalTok{(terra)\}}
\ControlFlowTok{if}\NormalTok{(}\SpecialCharTok{!}\FunctionTok{require}\NormalTok{(tidyverse)) \{}\FunctionTok{install.packages}\NormalTok{(}\StringTok{"tidyverse"}\NormalTok{); }\FunctionTok{require}\NormalTok{(tidyverse)\}}

\NormalTok{nosaukums}\OtherTok{=}\StringTok{"General\_GardensOrchards\_cell.tif"}
\NormalTok{ielasisanas\_cels}\OtherTok{=}\FunctionTok{paste0}\NormalTok{(}\StringTok{"./RasterGrids\_100m/2024/RAW/"}\NormalTok{,nosaukums)}
\NormalTok{saglabasanas\_cels}\OtherTok{=}\FunctionTok{paste0}\NormalTok{(}\StringTok{"./RasterGrids\_100m/2024/Scaled/"}\NormalTok{,nosaukums)}
\NormalTok{slanis}\OtherTok{=}\FunctionTok{rast}\NormalTok{(ielasisanas\_cels)}
\NormalTok{videjais}\OtherTok{=}\FunctionTok{global}\NormalTok{(slanis,}\AttributeTok{fun=}\StringTok{"mean"}\NormalTok{,}\AttributeTok{na.rm=}\ConstantTok{TRUE}\NormalTok{)}
\NormalTok{centrets}\OtherTok{=}\NormalTok{slanis}\SpecialCharTok{{-}}\NormalTok{videjais[,}\DecValTok{1}\NormalTok{]}
\NormalTok{standartnovirze}\OtherTok{=}\NormalTok{terra}\SpecialCharTok{::}\FunctionTok{global}\NormalTok{(centrets,}\AttributeTok{fun=}\StringTok{"rms"}\NormalTok{,}\AttributeTok{na.rm=}\ConstantTok{TRUE}\NormalTok{)}
\NormalTok{merogots}\OtherTok{=}\NormalTok{centrets}\SpecialCharTok{/}\NormalTok{standartnovirze[,}\DecValTok{1}\NormalTok{]}
\FunctionTok{writeRaster}\NormalTok{(merogots,}
      \AttributeTok{filename=}\NormalTok{saglabasanas\_cels,}
      \AttributeTok{overwrite=}\ConstantTok{TRUE}\NormalTok{)}
\end{Highlighting}
\end{Shaded}

\section{General\_GardensOrchards\_r500}\label{ch06.429}

\textbf{filename:} \texttt{General\_GardensOrchards\_r500.tif}

\textbf{layername:} \texttt{egv\_429}

\textbf{English name:} Fractional cover of Allotment gardens, Orchards within the 0.5
km landscape

\textbf{Latvian name:} Vasarnīcu kompleksu un augļudārzu platības īpatsvars 0,5 km
ainavā

\textbf{Procedure:} The cover fraction within a radius of 500 m around the analysis grid cell is
calculated as the area-weighted sum of the \hyperref[ch06.428]{analysis cells} inside the
buffer, using the workflow \texttt{egvtools::radius\_function()}. During the calculation of the landscape metric,
inverse distance weighted (power = 2) gap filling on the output is applied
to ensure no missing values at the edges. Then the layer is rewritten to set
its name. Finally, the layer is standardised by subtracting the arithmetic
mean and dividing by the root mean squared error.

\begin{Shaded}
\begin{Highlighting}[]
\CommentTok{\# libs {-}{-}{-}{-}}
\ControlFlowTok{if}\NormalTok{(}\SpecialCharTok{!}\FunctionTok{require}\NormalTok{(terra)) \{}\FunctionTok{install.packages}\NormalTok{(}\StringTok{"terra"}\NormalTok{); }\FunctionTok{require}\NormalTok{(terra)\}}
\ControlFlowTok{if}\NormalTok{(}\SpecialCharTok{!}\FunctionTok{require}\NormalTok{(egvtools)) \{remotes}\SpecialCharTok{::}\FunctionTok{install\_github}\NormalTok{(}\StringTok{"aavotins/egvtools"}\NormalTok{); }\FunctionTok{require}\NormalTok{(egvtools)\}}


\CommentTok{\# Templates {-}{-}{-}{-}{-}}
\NormalTok{template100}\OtherTok{=}\FunctionTok{rast}\NormalTok{(}\StringTok{"./Templates/TemplateRasters/LV100m\_10km.tif"}\NormalTok{)}

\CommentTok{\# radii {-}{-}{-}{-}}
\FunctionTok{radius\_function}\NormalTok{(}
 \AttributeTok{kvadrati\_path =} \StringTok{"./Templates/TemplateGrids/tiles/"}\NormalTok{,}
 \AttributeTok{radii\_path   =} \StringTok{"./Templates/TemplateGridPoints/tiles/"}\NormalTok{,}
 \AttributeTok{tikls100\_path =} \StringTok{"./Templates/TemplateGrids/tikls100\_sauzeme.parquet"}\NormalTok{,}
 \AttributeTok{template\_path =} \StringTok{"./Templates/TemplateRasters/LV100m\_10km.tif"}\NormalTok{,}
 \AttributeTok{input\_layers  =} \FunctionTok{c}\NormalTok{(}\StringTok{"./RasterGrids\_100m/2024/RAW/General\_GardensOrchards\_cell.tif"}\NormalTok{),}
 \AttributeTok{layer\_prefixes =} \FunctionTok{c}\NormalTok{(}\StringTok{"General\_GardensOrchards"}\NormalTok{),}
 \AttributeTok{output\_dir   =} \StringTok{"./RasterGrids\_100m/2024/RAW/"}\NormalTok{,}
 \AttributeTok{n\_workers   =} \DecValTok{6}\NormalTok{,}
 \AttributeTok{radii     =} \FunctionTok{c}\NormalTok{(}\StringTok{"r500"}\NormalTok{),}
 \AttributeTok{radius\_mode  =} \StringTok{"sparse"}\NormalTok{,}
 \AttributeTok{extract\_fun  =} \StringTok{"mean"}\NormalTok{,}
 \AttributeTok{fill\_missing  =} \ConstantTok{TRUE}\NormalTok{,}
 \AttributeTok{IDW\_weight   =} \DecValTok{2}\NormalTok{,}
 \AttributeTok{future\_max\_size =} \DecValTok{40} \SpecialCharTok{*} \DecValTok{1024}\SpecialCharTok{\^{}}\DecValTok{3}\NormalTok{)}


\CommentTok{\# General\_GardensOrchards\_r500.tif  egv\_429}
\NormalTok{slanis}\OtherTok{=}\FunctionTok{rast}\NormalTok{(}\StringTok{"./RasterGrids\_100m/2024/RAW/General\_GardensOrchards\_r500.tif"}\NormalTok{)}
\FunctionTok{names}\NormalTok{(slanis)}\OtherTok{=}\StringTok{"egv\_429"}
\NormalTok{slanis2}\OtherTok{=}\FunctionTok{project}\NormalTok{(slanis,template100)}
\FunctionTok{writeRaster}\NormalTok{(slanis2,}
      \StringTok{"./RasterGrids\_100m/2024/RAW/General\_GardensOrchards\_r500.tif"}\NormalTok{,}
      \AttributeTok{overwrite=}\ConstantTok{TRUE}\NormalTok{)}

\CommentTok{\# standardisation {-}{-}{-}{-}}
\ControlFlowTok{if}\NormalTok{(}\SpecialCharTok{!}\FunctionTok{require}\NormalTok{(terra)) \{}\FunctionTok{install.packages}\NormalTok{(}\StringTok{"terra"}\NormalTok{); }\FunctionTok{require}\NormalTok{(terra)\}}
\ControlFlowTok{if}\NormalTok{(}\SpecialCharTok{!}\FunctionTok{require}\NormalTok{(tidyverse)) \{}\FunctionTok{install.packages}\NormalTok{(}\StringTok{"tidyverse"}\NormalTok{); }\FunctionTok{require}\NormalTok{(tidyverse)\}}

\NormalTok{nosaukums}\OtherTok{=}\StringTok{"General\_GardensOrchards\_r500.tif"}
\NormalTok{ielasisanas\_cels}\OtherTok{=}\FunctionTok{paste0}\NormalTok{(}\StringTok{"./RasterGrids\_100m/2024/RAW/"}\NormalTok{,nosaukums)}
\NormalTok{saglabasanas\_cels}\OtherTok{=}\FunctionTok{paste0}\NormalTok{(}\StringTok{"./RasterGrids\_100m/2024/Scaled/"}\NormalTok{,nosaukums)}
\NormalTok{slanis}\OtherTok{=}\FunctionTok{rast}\NormalTok{(ielasisanas\_cels)}
\NormalTok{videjais}\OtherTok{=}\FunctionTok{global}\NormalTok{(slanis,}\AttributeTok{fun=}\StringTok{"mean"}\NormalTok{,}\AttributeTok{na.rm=}\ConstantTok{TRUE}\NormalTok{)}
\NormalTok{centrets}\OtherTok{=}\NormalTok{slanis}\SpecialCharTok{{-}}\NormalTok{videjais[,}\DecValTok{1}\NormalTok{]}
\NormalTok{standartnovirze}\OtherTok{=}\NormalTok{terra}\SpecialCharTok{::}\FunctionTok{global}\NormalTok{(centrets,}\AttributeTok{fun=}\StringTok{"rms"}\NormalTok{,}\AttributeTok{na.rm=}\ConstantTok{TRUE}\NormalTok{)}
\NormalTok{merogots}\OtherTok{=}\NormalTok{centrets}\SpecialCharTok{/}\NormalTok{standartnovirze[,}\DecValTok{1}\NormalTok{]}
\FunctionTok{writeRaster}\NormalTok{(merogots,}
      \AttributeTok{filename=}\NormalTok{saglabasanas\_cels,}
      \AttributeTok{overwrite=}\ConstantTok{TRUE}\NormalTok{)}
\end{Highlighting}
\end{Shaded}

\section{General\_GardensOrchards\_r1250}\label{ch06.430}

\textbf{filename:} \texttt{General\_GardensOrchards\_r1250.tif}

\textbf{layername:} \texttt{egv\_430}

\textbf{English name:} Fractional cover of Allotment gardens, Orchards within the
1.25 km landscape

\textbf{Latvian name:} Vasarnīcu kompleksu un augļudārzu platības īpatsvars 1,25 km
ainavā

\textbf{Procedure:} The cover fraction within a radius of 1250 m around the analysis grid cell
is calculated as the area-weighted sum of the \hyperref[ch06.428]{analysis cells} inside
the buffer, using the workflow \texttt{egvtools::radius\_function()}. During the calculation of the landscape
metric, inverse distance weighted (power = 2) gap filling on the output is
applied to ensure no missing values at the edges. Then the layer is
rewritten to set its name. Finally, the layer is standardised by
subtracting the arithmetic mean and dividing by the root mean squared error.

\begin{Shaded}
\begin{Highlighting}[]
\CommentTok{\# libs {-}{-}{-}{-}}
\ControlFlowTok{if}\NormalTok{(}\SpecialCharTok{!}\FunctionTok{require}\NormalTok{(terra)) \{}\FunctionTok{install.packages}\NormalTok{(}\StringTok{"terra"}\NormalTok{); }\FunctionTok{require}\NormalTok{(terra)\}}
\ControlFlowTok{if}\NormalTok{(}\SpecialCharTok{!}\FunctionTok{require}\NormalTok{(egvtools)) \{remotes}\SpecialCharTok{::}\FunctionTok{install\_github}\NormalTok{(}\StringTok{"aavotins/egvtools"}\NormalTok{); }\FunctionTok{require}\NormalTok{(egvtools)\}}


\CommentTok{\# Templates {-}{-}{-}{-}{-}}
\NormalTok{template100}\OtherTok{=}\FunctionTok{rast}\NormalTok{(}\StringTok{"./Templates/TemplateRasters/LV100m\_10km.tif"}\NormalTok{)}

\CommentTok{\# radii {-}{-}{-}{-}}
\FunctionTok{radius\_function}\NormalTok{(}
 \AttributeTok{kvadrati\_path =} \StringTok{"./Templates/TemplateGrids/tiles/"}\NormalTok{,}
 \AttributeTok{radii\_path   =} \StringTok{"./Templates/TemplateGridPoints/tiles/"}\NormalTok{,}
 \AttributeTok{tikls100\_path =} \StringTok{"./Templates/TemplateGrids/tikls100\_sauzeme.parquet"}\NormalTok{,}
 \AttributeTok{template\_path =} \StringTok{"./Templates/TemplateRasters/LV100m\_10km.tif"}\NormalTok{,}
 \AttributeTok{input\_layers  =} \FunctionTok{c}\NormalTok{(}\StringTok{"./RasterGrids\_100m/2024/RAW/General\_GardensOrchards\_cell.tif"}\NormalTok{),}
 \AttributeTok{layer\_prefixes =} \FunctionTok{c}\NormalTok{(}\StringTok{"General\_GardensOrchards"}\NormalTok{),}
 \AttributeTok{output\_dir   =} \StringTok{"./RasterGrids\_100m/2024/RAW/"}\NormalTok{,}
 \AttributeTok{n\_workers   =} \DecValTok{6}\NormalTok{,}
 \AttributeTok{radii     =} \FunctionTok{c}\NormalTok{(}\StringTok{"r1250"}\NormalTok{),}
 \AttributeTok{radius\_mode  =} \StringTok{"sparse"}\NormalTok{,}
 \AttributeTok{extract\_fun  =} \StringTok{"mean"}\NormalTok{,}
 \AttributeTok{fill\_missing  =} \ConstantTok{TRUE}\NormalTok{,}
 \AttributeTok{IDW\_weight   =} \DecValTok{2}\NormalTok{,}
 \AttributeTok{future\_max\_size =} \DecValTok{40} \SpecialCharTok{*} \DecValTok{1024}\SpecialCharTok{\^{}}\DecValTok{3}\NormalTok{)}


\CommentTok{\# General\_GardensOrchards\_r1250.tif egv\_430}
\NormalTok{slanis}\OtherTok{=}\FunctionTok{rast}\NormalTok{(}\StringTok{"./RasterGrids\_100m/2024/RAW/General\_GardensOrchards\_r1250.tif"}\NormalTok{)}
\FunctionTok{names}\NormalTok{(slanis)}\OtherTok{=}\StringTok{"egv\_430"}
\NormalTok{slanis2}\OtherTok{=}\FunctionTok{project}\NormalTok{(slanis,template100)}
\FunctionTok{writeRaster}\NormalTok{(slanis2,}
      \StringTok{"./RasterGrids\_100m/2024/RAW/General\_GardensOrchards\_r1250.tif"}\NormalTok{,}
      \AttributeTok{overwrite=}\ConstantTok{TRUE}\NormalTok{)}

\CommentTok{\# standardisation {-}{-}{-}{-}}
\ControlFlowTok{if}\NormalTok{(}\SpecialCharTok{!}\FunctionTok{require}\NormalTok{(terra)) \{}\FunctionTok{install.packages}\NormalTok{(}\StringTok{"terra"}\NormalTok{); }\FunctionTok{require}\NormalTok{(terra)\}}
\ControlFlowTok{if}\NormalTok{(}\SpecialCharTok{!}\FunctionTok{require}\NormalTok{(tidyverse)) \{}\FunctionTok{install.packages}\NormalTok{(}\StringTok{"tidyverse"}\NormalTok{); }\FunctionTok{require}\NormalTok{(tidyverse)\}}

\NormalTok{nosaukums}\OtherTok{=}\StringTok{"General\_GardensOrchards\_r1250.tif"}
\NormalTok{ielasisanas\_cels}\OtherTok{=}\FunctionTok{paste0}\NormalTok{(}\StringTok{"./RasterGrids\_100m/2024/RAW/"}\NormalTok{,nosaukums)}
\NormalTok{saglabasanas\_cels}\OtherTok{=}\FunctionTok{paste0}\NormalTok{(}\StringTok{"./RasterGrids\_100m/2024/Scaled/"}\NormalTok{,nosaukums)}
\NormalTok{slanis}\OtherTok{=}\FunctionTok{rast}\NormalTok{(ielasisanas\_cels)}
\NormalTok{videjais}\OtherTok{=}\FunctionTok{global}\NormalTok{(slanis,}\AttributeTok{fun=}\StringTok{"mean"}\NormalTok{,}\AttributeTok{na.rm=}\ConstantTok{TRUE}\NormalTok{)}
\NormalTok{centrets}\OtherTok{=}\NormalTok{slanis}\SpecialCharTok{{-}}\NormalTok{videjais[,}\DecValTok{1}\NormalTok{]}
\NormalTok{standartnovirze}\OtherTok{=}\NormalTok{terra}\SpecialCharTok{::}\FunctionTok{global}\NormalTok{(centrets,}\AttributeTok{fun=}\StringTok{"rms"}\NormalTok{,}\AttributeTok{na.rm=}\ConstantTok{TRUE}\NormalTok{)}
\NormalTok{merogots}\OtherTok{=}\NormalTok{centrets}\SpecialCharTok{/}\NormalTok{standartnovirze[,}\DecValTok{1}\NormalTok{]}
\FunctionTok{writeRaster}\NormalTok{(merogots,}
      \AttributeTok{filename=}\NormalTok{saglabasanas\_cels,}
      \AttributeTok{overwrite=}\ConstantTok{TRUE}\NormalTok{)}
\end{Highlighting}
\end{Shaded}

\section{General\_GardensOrchards\_r3000}\label{ch06.431}

\textbf{filename:} \texttt{General\_GardensOrchards\_r3000.tif}

\textbf{layername:} \texttt{egv\_431}

\textbf{English name:} Fractional cover of Allotment gardens, Orchards within the 3
km landscape

\textbf{Latvian name:} Vasarnīcu kompleksu un augļudārzu platības īpatsvars 3 km
ainavā

\textbf{Procedure:} The cover fraction within a radius of 3000 m around the analysis grid cell
is calculated as the area-weighted sum of the \hyperref[ch06.428]{analysis cells} inside
the buffer, using the workflow \texttt{egvtools::radius\_function()}. During the calculation of the landscape
metric, inverse distance weighted (power = 2) gap filling on the output is
applied to ensure no missing values at the edges. Then the layer is
rewritten to set its name. Finally, the layer is standardised by
subtracting the arithmetic mean and dividing by the root mean squared error.

\begin{Shaded}
\begin{Highlighting}[]
\CommentTok{\# libs {-}{-}{-}{-}}
\ControlFlowTok{if}\NormalTok{(}\SpecialCharTok{!}\FunctionTok{require}\NormalTok{(terra)) \{}\FunctionTok{install.packages}\NormalTok{(}\StringTok{"terra"}\NormalTok{); }\FunctionTok{require}\NormalTok{(terra)\}}
\ControlFlowTok{if}\NormalTok{(}\SpecialCharTok{!}\FunctionTok{require}\NormalTok{(egvtools)) \{remotes}\SpecialCharTok{::}\FunctionTok{install\_github}\NormalTok{(}\StringTok{"aavotins/egvtools"}\NormalTok{); }\FunctionTok{require}\NormalTok{(egvtools)\}}


\CommentTok{\# Templates {-}{-}{-}{-}{-}}
\NormalTok{template100}\OtherTok{=}\FunctionTok{rast}\NormalTok{(}\StringTok{"./Templates/TemplateRasters/LV100m\_10km.tif"}\NormalTok{)}

\CommentTok{\# radii {-}{-}{-}{-}}
\FunctionTok{radius\_function}\NormalTok{(}
 \AttributeTok{kvadrati\_path =} \StringTok{"./Templates/TemplateGrids/tiles/"}\NormalTok{,}
 \AttributeTok{radii\_path   =} \StringTok{"./Templates/TemplateGridPoints/tiles/"}\NormalTok{,}
 \AttributeTok{tikls100\_path =} \StringTok{"./Templates/TemplateGrids/tikls100\_sauzeme.parquet"}\NormalTok{,}
 \AttributeTok{template\_path =} \StringTok{"./Templates/TemplateRasters/LV100m\_10km.tif"}\NormalTok{,}
 \AttributeTok{input\_layers  =} \FunctionTok{c}\NormalTok{(}\StringTok{"./RasterGrids\_100m/2024/RAW/General\_GardensOrchards\_cell.tif"}\NormalTok{),}
 \AttributeTok{layer\_prefixes =} \FunctionTok{c}\NormalTok{(}\StringTok{"General\_GardensOrchards"}\NormalTok{),}
 \AttributeTok{output\_dir   =} \StringTok{"./RasterGrids\_100m/2024/RAW/"}\NormalTok{,}
 \AttributeTok{n\_workers   =} \DecValTok{6}\NormalTok{,}
 \AttributeTok{radii     =} \FunctionTok{c}\NormalTok{(}\StringTok{"r3000"}\NormalTok{),}
 \AttributeTok{radius\_mode  =} \StringTok{"sparse"}\NormalTok{,}
 \AttributeTok{extract\_fun  =} \StringTok{"mean"}\NormalTok{,}
 \AttributeTok{fill\_missing  =} \ConstantTok{TRUE}\NormalTok{,}
 \AttributeTok{IDW\_weight   =} \DecValTok{2}\NormalTok{,}
 \AttributeTok{future\_max\_size =} \DecValTok{40} \SpecialCharTok{*} \DecValTok{1024}\SpecialCharTok{\^{}}\DecValTok{3}\NormalTok{)}


\CommentTok{\# General\_GardensOrchards\_r3000.tif egv\_431}
\NormalTok{slanis}\OtherTok{=}\FunctionTok{rast}\NormalTok{(}\StringTok{"./RasterGrids\_100m/2024/RAW/General\_GardensOrchards\_r3000.tif"}\NormalTok{)}
\FunctionTok{names}\NormalTok{(slanis)}\OtherTok{=}\StringTok{"egv\_431"}
\NormalTok{slanis2}\OtherTok{=}\FunctionTok{project}\NormalTok{(slanis,template100)}
\FunctionTok{writeRaster}\NormalTok{(slanis2,}
      \StringTok{"./RasterGrids\_100m/2024/RAW/General\_GardensOrchards\_r3000.tif"}\NormalTok{,}
      \AttributeTok{overwrite=}\ConstantTok{TRUE}\NormalTok{)}

\CommentTok{\# standardisation {-}{-}{-}{-}}
\ControlFlowTok{if}\NormalTok{(}\SpecialCharTok{!}\FunctionTok{require}\NormalTok{(terra)) \{}\FunctionTok{install.packages}\NormalTok{(}\StringTok{"terra"}\NormalTok{); }\FunctionTok{require}\NormalTok{(terra)\}}
\ControlFlowTok{if}\NormalTok{(}\SpecialCharTok{!}\FunctionTok{require}\NormalTok{(tidyverse)) \{}\FunctionTok{install.packages}\NormalTok{(}\StringTok{"tidyverse"}\NormalTok{); }\FunctionTok{require}\NormalTok{(tidyverse)\}}

\NormalTok{nosaukums}\OtherTok{=}\StringTok{"General\_GardensOrchards\_r3000.tif"}
\NormalTok{ielasisanas\_cels}\OtherTok{=}\FunctionTok{paste0}\NormalTok{(}\StringTok{"./RasterGrids\_100m/2024/RAW/"}\NormalTok{,nosaukums)}
\NormalTok{saglabasanas\_cels}\OtherTok{=}\FunctionTok{paste0}\NormalTok{(}\StringTok{"./RasterGrids\_100m/2024/Scaled/"}\NormalTok{,nosaukums)}
\NormalTok{slanis}\OtherTok{=}\FunctionTok{rast}\NormalTok{(ielasisanas\_cels)}
\NormalTok{videjais}\OtherTok{=}\FunctionTok{global}\NormalTok{(slanis,}\AttributeTok{fun=}\StringTok{"mean"}\NormalTok{,}\AttributeTok{na.rm=}\ConstantTok{TRUE}\NormalTok{)}
\NormalTok{centrets}\OtherTok{=}\NormalTok{slanis}\SpecialCharTok{{-}}\NormalTok{videjais[,}\DecValTok{1}\NormalTok{]}
\NormalTok{standartnovirze}\OtherTok{=}\NormalTok{terra}\SpecialCharTok{::}\FunctionTok{global}\NormalTok{(centrets,}\AttributeTok{fun=}\StringTok{"rms"}\NormalTok{,}\AttributeTok{na.rm=}\ConstantTok{TRUE}\NormalTok{)}
\NormalTok{merogots}\OtherTok{=}\NormalTok{centrets}\SpecialCharTok{/}\NormalTok{standartnovirze[,}\DecValTok{1}\NormalTok{]}
\FunctionTok{writeRaster}\NormalTok{(merogots,}
      \AttributeTok{filename=}\NormalTok{saglabasanas\_cels,}
      \AttributeTok{overwrite=}\ConstantTok{TRUE}\NormalTok{)}
\end{Highlighting}
\end{Shaded}

\section{General\_GardensOrchards\_r10000}\label{ch06.432}

\textbf{filename:} \texttt{General\_GardensOrchards\_r10000.tif}

\textbf{layername:} \texttt{egv\_432}

\textbf{English name:} Fractional cover of Allotment gardens, Orchards within the 10
km landscape

\textbf{Latvian name:} Vasarnīcu kompleksu un augļudārzu platības īpatsvars 10 km
ainavā

\textbf{Procedure:} The cover fraction within a radius of 10000 m around the analysis grid cell
is calculated as the area-weighted sum of the \hyperref[ch06.428]{analysis cells} inside
the buffer, using the workflow \texttt{egvtools::radius\_function()}. During the calculation of the landscape
metric, inverse distance weighted (power = 2) gap filling on the output is
applied to ensure no missing values at the edges. Then the layer is
rewritten to set its name. Finally, the layer is standardised by
subtracting the arithmetic mean and dividing by the root mean squared error.

\begin{Shaded}
\begin{Highlighting}[]
\CommentTok{\# libs {-}{-}{-}{-}}
\ControlFlowTok{if}\NormalTok{(}\SpecialCharTok{!}\FunctionTok{require}\NormalTok{(terra)) \{}\FunctionTok{install.packages}\NormalTok{(}\StringTok{"terra"}\NormalTok{); }\FunctionTok{require}\NormalTok{(terra)\}}
\ControlFlowTok{if}\NormalTok{(}\SpecialCharTok{!}\FunctionTok{require}\NormalTok{(egvtools)) \{remotes}\SpecialCharTok{::}\FunctionTok{install\_github}\NormalTok{(}\StringTok{"aavotins/egvtools"}\NormalTok{); }\FunctionTok{require}\NormalTok{(egvtools)\}}


\CommentTok{\# Templates {-}{-}{-}{-}{-}}
\NormalTok{template100}\OtherTok{=}\FunctionTok{rast}\NormalTok{(}\StringTok{"./Templates/TemplateRasters/LV100m\_10km.tif"}\NormalTok{)}

\CommentTok{\# radii {-}{-}{-}{-}}
\FunctionTok{radius\_function}\NormalTok{(}
 \AttributeTok{kvadrati\_path =} \StringTok{"./Templates/TemplateGrids/tiles/"}\NormalTok{,}
 \AttributeTok{radii\_path   =} \StringTok{"./Templates/TemplateGridPoints/tiles/"}\NormalTok{,}
 \AttributeTok{tikls100\_path =} \StringTok{"./Templates/TemplateGrids/tikls100\_sauzeme.parquet"}\NormalTok{,}
 \AttributeTok{template\_path =} \StringTok{"./Templates/TemplateRasters/LV100m\_10km.tif"}\NormalTok{,}
 \AttributeTok{input\_layers  =} \FunctionTok{c}\NormalTok{(}\StringTok{"./RasterGrids\_100m/2024/RAW/General\_GardensOrchards\_cell.tif"}\NormalTok{),}
 \AttributeTok{layer\_prefixes =} \FunctionTok{c}\NormalTok{(}\StringTok{"General\_GardensOrchards"}\NormalTok{),}
 \AttributeTok{output\_dir   =} \StringTok{"./RasterGrids\_100m/2024/RAW/"}\NormalTok{,}
 \AttributeTok{n\_workers   =} \DecValTok{6}\NormalTok{,}
 \AttributeTok{radii     =} \FunctionTok{c}\NormalTok{(}\StringTok{"r10000"}\NormalTok{),}
 \AttributeTok{radius\_mode  =} \StringTok{"sparse"}\NormalTok{,}
 \AttributeTok{extract\_fun  =} \StringTok{"mean"}\NormalTok{,}
 \AttributeTok{fill\_missing  =} \ConstantTok{TRUE}\NormalTok{,}
 \AttributeTok{IDW\_weight   =} \DecValTok{2}\NormalTok{,}
 \AttributeTok{future\_max\_size =} \DecValTok{40} \SpecialCharTok{*} \DecValTok{1024}\SpecialCharTok{\^{}}\DecValTok{3}\NormalTok{)}


\CommentTok{\# General\_GardensOrchards\_r10000.tif    egv\_432}
\NormalTok{slanis}\OtherTok{=}\FunctionTok{rast}\NormalTok{(}\StringTok{"./RasterGrids\_100m/2024/RAW/General\_GardensOrchards\_r10000.tif"}\NormalTok{)}
\FunctionTok{names}\NormalTok{(slanis)}\OtherTok{=}\StringTok{"egv\_432"}
\NormalTok{slanis2}\OtherTok{=}\FunctionTok{project}\NormalTok{(slanis,template100)}
\FunctionTok{writeRaster}\NormalTok{(slanis2,}
      \StringTok{"./RasterGrids\_100m/2024/RAW/General\_GardensOrchards\_r10000.tif"}\NormalTok{,}
      \AttributeTok{overwrite=}\ConstantTok{TRUE}\NormalTok{)}

\CommentTok{\# standardisation {-}{-}{-}{-}}
\ControlFlowTok{if}\NormalTok{(}\SpecialCharTok{!}\FunctionTok{require}\NormalTok{(terra)) \{}\FunctionTok{install.packages}\NormalTok{(}\StringTok{"terra"}\NormalTok{); }\FunctionTok{require}\NormalTok{(terra)\}}
\ControlFlowTok{if}\NormalTok{(}\SpecialCharTok{!}\FunctionTok{require}\NormalTok{(tidyverse)) \{}\FunctionTok{install.packages}\NormalTok{(}\StringTok{"tidyverse"}\NormalTok{); }\FunctionTok{require}\NormalTok{(tidyverse)\}}

\NormalTok{nosaukums}\OtherTok{=}\StringTok{"General\_GardensOrchards\_r10000.tif"}
\NormalTok{ielasisanas\_cels}\OtherTok{=}\FunctionTok{paste0}\NormalTok{(}\StringTok{"./RasterGrids\_100m/2024/RAW/"}\NormalTok{,nosaukums)}
\NormalTok{saglabasanas\_cels}\OtherTok{=}\FunctionTok{paste0}\NormalTok{(}\StringTok{"./RasterGrids\_100m/2024/Scaled/"}\NormalTok{,nosaukums)}
\NormalTok{slanis}\OtherTok{=}\FunctionTok{rast}\NormalTok{(ielasisanas\_cels)}
\NormalTok{videjais}\OtherTok{=}\FunctionTok{global}\NormalTok{(slanis,}\AttributeTok{fun=}\StringTok{"mean"}\NormalTok{,}\AttributeTok{na.rm=}\ConstantTok{TRUE}\NormalTok{)}
\NormalTok{centrets}\OtherTok{=}\NormalTok{slanis}\SpecialCharTok{{-}}\NormalTok{videjais[,}\DecValTok{1}\NormalTok{]}
\NormalTok{standartnovirze}\OtherTok{=}\NormalTok{terra}\SpecialCharTok{::}\FunctionTok{global}\NormalTok{(centrets,}\AttributeTok{fun=}\StringTok{"rms"}\NormalTok{,}\AttributeTok{na.rm=}\ConstantTok{TRUE}\NormalTok{)}
\NormalTok{merogots}\OtherTok{=}\NormalTok{centrets}\SpecialCharTok{/}\NormalTok{standartnovirze[,}\DecValTok{1}\NormalTok{]}
\FunctionTok{writeRaster}\NormalTok{(merogots,}
      \AttributeTok{filename=}\NormalTok{saglabasanas\_cels,}
      \AttributeTok{overwrite=}\ConstantTok{TRUE}\NormalTok{)}
\end{Highlighting}
\end{Shaded}

\section{General\_Roads\_cell}\label{ch06.433}

\textbf{filename:} \texttt{General\_Roads\_cell.tif}

\textbf{layername:} \texttt{egv\_433}

\textbf{English name:} Fractional cover of Roads within the analysis cell (1 ha)

\textbf{Latvian name:} Ceļu platības īpatsvars analīzes šūnā (1 ha)

\textbf{Procedure:} First, the roads from the \hyperref[Ch05.03]{Landscape classification} are
selected (value 100 is reclassified to value 1; all others are set to 0). The resulting layer
is then aggregated to EGV resolution using the workflow \texttt{egvtools::input2egv()}, which
calculates the arithmetic mean to determine the cover fraction. During
aggregation, inverse distance weighted (power = 2) gap filling on the output is
applied to ensure no missing values at the edges. Finally, the layer is
standardised by subtracting the arithmetic mean and dividing by the root mean squared
error.

\begin{Shaded}
\begin{Highlighting}[]
\CommentTok{\# libs {-}{-}{-}{-}}
\ControlFlowTok{if}\NormalTok{(}\SpecialCharTok{!}\FunctionTok{require}\NormalTok{(egvtools)) \{remotes}\SpecialCharTok{::}\FunctionTok{install\_github}\NormalTok{(}\StringTok{"aavotins/egvtools"}\NormalTok{); }\FunctionTok{require}\NormalTok{(egvtools)\}}
\ControlFlowTok{if}\NormalTok{(}\SpecialCharTok{!}\FunctionTok{require}\NormalTok{(terra)) \{}\FunctionTok{install.packages}\NormalTok{(}\StringTok{"terra"}\NormalTok{); }\FunctionTok{require}\NormalTok{(terra)\}}
\ControlFlowTok{if}\NormalTok{(}\SpecialCharTok{!}\FunctionTok{require}\NormalTok{(sf)) \{}\FunctionTok{install.packages}\NormalTok{(}\StringTok{"sf"}\NormalTok{); }\FunctionTok{require}\NormalTok{(sf)\}}
\ControlFlowTok{if}\NormalTok{(}\SpecialCharTok{!}\FunctionTok{require}\NormalTok{(tidyverse)) \{}\FunctionTok{install.packages}\NormalTok{(}\StringTok{"tidyverse"}\NormalTok{); }\FunctionTok{require}\NormalTok{(tidyverse)\}}
\ControlFlowTok{if}\NormalTok{(}\SpecialCharTok{!}\FunctionTok{require}\NormalTok{(sfarrow)) \{}\FunctionTok{install.packages}\NormalTok{(}\StringTok{"sfarrow"}\NormalTok{); }\FunctionTok{require}\NormalTok{(sfarrow)\}}
\ControlFlowTok{if}\NormalTok{(}\SpecialCharTok{!}\FunctionTok{require}\NormalTok{(readxl)) \{}\FunctionTok{install.packages}\NormalTok{(}\StringTok{"readxl"}\NormalTok{); }\FunctionTok{require}\NormalTok{(readxl)\}}
\ControlFlowTok{if}\NormalTok{(}\SpecialCharTok{!}\FunctionTok{require}\NormalTok{(raster)) \{}\FunctionTok{install.packages}\NormalTok{(}\StringTok{"raster"}\NormalTok{); }\FunctionTok{require}\NormalTok{(raster)\}}
\ControlFlowTok{if}\NormalTok{(}\SpecialCharTok{!}\FunctionTok{require}\NormalTok{(fasterize)) \{}\FunctionTok{install.packages}\NormalTok{(}\StringTok{"fasterize"}\NormalTok{); }\FunctionTok{require}\NormalTok{(fasterize)\}}

\CommentTok{\# templates {-}{-}{-}{-}}
\NormalTok{template100}\OtherTok{=}\FunctionTok{rast}\NormalTok{(}\StringTok{"./Templates/TemplateRasters/LV100m\_10km.tif"}\NormalTok{)}
\NormalTok{template10}\OtherTok{=}\FunctionTok{rast}\NormalTok{(}\StringTok{"./Templates/TemplateRasters/LV10m\_10km.tif"}\NormalTok{)}
\NormalTok{rastrs10}\OtherTok{=}\FunctionTok{raster}\NormalTok{(template10)}

\NormalTok{nulls10}\OtherTok{=}\FunctionTok{rast}\NormalTok{(}\StringTok{"./Templates/TemplateRasters/nulls\_LV10m\_10km.tif"}\NormalTok{)}
\NormalTok{nulls100}\OtherTok{=}\FunctionTok{rast}\NormalTok{(}\StringTok{"./Templates/TemplateRasters/nulls\_LV100m\_10km.tif"}\NormalTok{)}

\CommentTok{\# simple landscape {-}{-}{-}{-}}
\NormalTok{simple\_landscape}\OtherTok{=}\FunctionTok{rast}\NormalTok{(}\StringTok{"RasterGrids\_10m/2024/Ainava\_vienk\_mask.tif"}\NormalTok{)}


\CommentTok{\# General\_Roads\_cell.tif    egv\_433 {-}{-}{-}{-}}
\NormalTok{celi}\OtherTok{=}\FunctionTok{ifel}\NormalTok{(simple\_landscape}\SpecialCharTok{==}\DecValTok{100}\NormalTok{,}\DecValTok{1}\NormalTok{,}\DecValTok{0}\NormalTok{)}
\NormalTok{i2e\_rez}\OtherTok{=}\NormalTok{egvtools}\SpecialCharTok{::}\FunctionTok{input2egv}\NormalTok{(}\AttributeTok{input=}\NormalTok{celi,}
              \AttributeTok{egv\_template=} \StringTok{"./Templates/TemplateRasters/LV100m\_10km.tif"}\NormalTok{,}
              \AttributeTok{summary\_function =} \StringTok{"average"}\NormalTok{,}
              \AttributeTok{missing\_job =} \StringTok{"FillOutput"}\NormalTok{,}
              \AttributeTok{outlocation =} \StringTok{"./RasterGrids\_100m/2024/RAW/"}\NormalTok{,}
              \AttributeTok{outfilename =} \StringTok{"General\_Roads\_cell.tif"}\NormalTok{,}
              \AttributeTok{layername =} \StringTok{"egv\_433"}\NormalTok{,}
              \AttributeTok{idw\_weight =} \DecValTok{2}\NormalTok{,}
              \AttributeTok{plot\_gaps =} \ConstantTok{FALSE}\NormalTok{,}\AttributeTok{plot\_final =} \ConstantTok{TRUE}\NormalTok{)}
\NormalTok{i2e\_rez}
\FunctionTok{rm}\NormalTok{(celi)}
\FunctionTok{rm}\NormalTok{(i2e\_rez)}

\CommentTok{\# standardisation {-}{-}{-}{-}}
\ControlFlowTok{if}\NormalTok{(}\SpecialCharTok{!}\FunctionTok{require}\NormalTok{(terra)) \{}\FunctionTok{install.packages}\NormalTok{(}\StringTok{"terra"}\NormalTok{); }\FunctionTok{require}\NormalTok{(terra)\}}
\ControlFlowTok{if}\NormalTok{(}\SpecialCharTok{!}\FunctionTok{require}\NormalTok{(tidyverse)) \{}\FunctionTok{install.packages}\NormalTok{(}\StringTok{"tidyverse"}\NormalTok{); }\FunctionTok{require}\NormalTok{(tidyverse)\}}

\NormalTok{nosaukums}\OtherTok{=}\StringTok{"General\_Roads\_cell.tif"}
\NormalTok{ielasisanas\_cels}\OtherTok{=}\FunctionTok{paste0}\NormalTok{(}\StringTok{"./RasterGrids\_100m/2024/RAW/"}\NormalTok{,nosaukums)}
\NormalTok{saglabasanas\_cels}\OtherTok{=}\FunctionTok{paste0}\NormalTok{(}\StringTok{"./RasterGrids\_100m/2024/Scaled/"}\NormalTok{,nosaukums)}
\NormalTok{slanis}\OtherTok{=}\FunctionTok{rast}\NormalTok{(ielasisanas\_cels)}
\NormalTok{videjais}\OtherTok{=}\FunctionTok{global}\NormalTok{(slanis,}\AttributeTok{fun=}\StringTok{"mean"}\NormalTok{,}\AttributeTok{na.rm=}\ConstantTok{TRUE}\NormalTok{)}
\NormalTok{centrets}\OtherTok{=}\NormalTok{slanis}\SpecialCharTok{{-}}\NormalTok{videjais[,}\DecValTok{1}\NormalTok{]}
\NormalTok{standartnovirze}\OtherTok{=}\NormalTok{terra}\SpecialCharTok{::}\FunctionTok{global}\NormalTok{(centrets,}\AttributeTok{fun=}\StringTok{"rms"}\NormalTok{,}\AttributeTok{na.rm=}\ConstantTok{TRUE}\NormalTok{)}
\NormalTok{merogots}\OtherTok{=}\NormalTok{centrets}\SpecialCharTok{/}\NormalTok{standartnovirze[,}\DecValTok{1}\NormalTok{]}
\FunctionTok{writeRaster}\NormalTok{(merogots,}
      \AttributeTok{filename=}\NormalTok{saglabasanas\_cels,}
      \AttributeTok{overwrite=}\ConstantTok{TRUE}\NormalTok{)}
\end{Highlighting}
\end{Shaded}

\section{General\_ShrubsOrchards\_cell}\label{ch06.434}

\textbf{filename:} \texttt{General\_ShrubsOrchards\_cell.tif}

\textbf{layername:} \texttt{egv\_434}

\textbf{English name:} Fractional cover of Shrubs, Young stands, Orchards within the
analysis cell (1 ha)

\textbf{Latvian name:} Krūmāju, jaunaudžu un augļudārzu platības īpatsvars analīzes
šūnā (1 ha)

\textbf{Procedure:} First, agricultural parcels declared as short term coppice are
selected from the \hyperref[Ch04.02]{Rural Support Service's information on declared fields}
and rasterised to match inputs. Then orchards and shrubs-low forest stands from
\hyperref[Ch05.03]{Landscape classification} are selected (values equal to 420 or 620
are reclassified to value 1, others as 0). The first layer is then covered
over the second. The resulting layer
is then aggregated to EGV resolution using the workflow \texttt{egvtools::input2egv()}, which
calculates the arithmetic mean to determine the cover fraction. During
aggregation, inverse distance weighted (power = 2) gap filling on the output is
applied to ensure no missing values at the edges. Finally, the layer is
standardised by subtracting the arithmetic mean and dividing by the root mean squared
error.

\begin{Shaded}
\begin{Highlighting}[]
\CommentTok{\# libs {-}{-}{-}{-}}
\ControlFlowTok{if}\NormalTok{(}\SpecialCharTok{!}\FunctionTok{require}\NormalTok{(egvtools)) \{remotes}\SpecialCharTok{::}\FunctionTok{install\_github}\NormalTok{(}\StringTok{"aavotins/egvtools"}\NormalTok{); }\FunctionTok{require}\NormalTok{(egvtools)\}}
\ControlFlowTok{if}\NormalTok{(}\SpecialCharTok{!}\FunctionTok{require}\NormalTok{(terra)) \{}\FunctionTok{install.packages}\NormalTok{(}\StringTok{"terra"}\NormalTok{); }\FunctionTok{require}\NormalTok{(terra)\}}
\ControlFlowTok{if}\NormalTok{(}\SpecialCharTok{!}\FunctionTok{require}\NormalTok{(sf)) \{}\FunctionTok{install.packages}\NormalTok{(}\StringTok{"sf"}\NormalTok{); }\FunctionTok{require}\NormalTok{(sf)\}}
\ControlFlowTok{if}\NormalTok{(}\SpecialCharTok{!}\FunctionTok{require}\NormalTok{(tidyverse)) \{}\FunctionTok{install.packages}\NormalTok{(}\StringTok{"tidyverse"}\NormalTok{); }\FunctionTok{require}\NormalTok{(tidyverse)\}}
\ControlFlowTok{if}\NormalTok{(}\SpecialCharTok{!}\FunctionTok{require}\NormalTok{(sfarrow)) \{}\FunctionTok{install.packages}\NormalTok{(}\StringTok{"sfarrow"}\NormalTok{); }\FunctionTok{require}\NormalTok{(sfarrow)\}}
\ControlFlowTok{if}\NormalTok{(}\SpecialCharTok{!}\FunctionTok{require}\NormalTok{(readxl)) \{}\FunctionTok{install.packages}\NormalTok{(}\StringTok{"readxl"}\NormalTok{); }\FunctionTok{require}\NormalTok{(readxl)\}}
\ControlFlowTok{if}\NormalTok{(}\SpecialCharTok{!}\FunctionTok{require}\NormalTok{(raster)) \{}\FunctionTok{install.packages}\NormalTok{(}\StringTok{"raster"}\NormalTok{); }\FunctionTok{require}\NormalTok{(raster)\}}
\ControlFlowTok{if}\NormalTok{(}\SpecialCharTok{!}\FunctionTok{require}\NormalTok{(fasterize)) \{}\FunctionTok{install.packages}\NormalTok{(}\StringTok{"fasterize"}\NormalTok{); }\FunctionTok{require}\NormalTok{(fasterize)\}}

\CommentTok{\# templates {-}{-}{-}{-}}
\NormalTok{template100}\OtherTok{=}\FunctionTok{rast}\NormalTok{(}\StringTok{"./Templates/TemplateRasters/LV100m\_10km.tif"}\NormalTok{)}
\NormalTok{template10}\OtherTok{=}\FunctionTok{rast}\NormalTok{(}\StringTok{"./Templates/TemplateRasters/LV10m\_10km.tif"}\NormalTok{)}
\NormalTok{rastrs10}\OtherTok{=}\FunctionTok{raster}\NormalTok{(template10)}

\NormalTok{nulls10}\OtherTok{=}\FunctionTok{rast}\NormalTok{(}\StringTok{"./Templates/TemplateRasters/nulls\_LV10m\_10km.tif"}\NormalTok{)}
\NormalTok{nulls100}\OtherTok{=}\FunctionTok{rast}\NormalTok{(}\StringTok{"./Templates/TemplateRasters/nulls\_LV100m\_10km.tif"}\NormalTok{)}

\CommentTok{\# simple landscape {-}{-}{-}{-}}
\NormalTok{simple\_landscape}\OtherTok{=}\FunctionTok{rast}\NormalTok{(}\StringTok{"RasterGrids\_10m/2024/Ainava\_vienk\_mask.tif"}\NormalTok{)}


\CommentTok{\# General\_ShrubsOrchards\_cell.tif   egv\_434 {-}{-}{-}{-}}
\NormalTok{kodi}\OtherTok{=}\FunctionTok{read\_excel}\NormalTok{(}\StringTok{"./Geodata/2024/LAD/KulturuKodi\_2024.xlsx"}\NormalTok{)}
\NormalTok{kodi}\SpecialCharTok{$}\NormalTok{kods}\OtherTok{=}\FunctionTok{as.character}\NormalTok{(kodi}\SpecialCharTok{$}\NormalTok{kods)}
\FunctionTok{table}\NormalTok{(kodi}\SpecialCharTok{$}\NormalTok{SDM\_grupa\_sakums,}\AttributeTok{useNA=}\StringTok{"always"}\NormalTok{)}
\NormalTok{lad}\OtherTok{=}\NormalTok{sfarrow}\SpecialCharTok{::}\FunctionTok{st\_read\_parquet}\NormalTok{(}\StringTok{"./Geodata/2024/LAD/Lauki\_2024.parquet"}\NormalTok{)}
\NormalTok{lad}\SpecialCharTok{$}\NormalTok{yes}\OtherTok{=}\DecValTok{1}
\NormalTok{lad}\OtherTok{=}\NormalTok{lad }\SpecialCharTok{\%\textgreater{}\%} 
 \FunctionTok{left\_join}\NormalTok{(kodi,}\AttributeTok{by=}\FunctionTok{c}\NormalTok{(}\StringTok{"PRODUCT\_CODE"}\OtherTok{=}\StringTok{"kods"}\NormalTok{))}
\NormalTok{ilggadigiekrumveida}\OtherTok{=}\NormalTok{lad }\SpecialCharTok{\%\textgreater{}\%} 
 \FunctionTok{filter}\NormalTok{(SDM\_grupa\_sakums }\SpecialCharTok{==} \StringTok{"krūmveida ilggadīgie stādījumi"}\NormalTok{)}
\NormalTok{krumveida\_r}\OtherTok{=}\FunctionTok{fasterize}\NormalTok{(ilggadigiekrumveida,rastrs10,}\AttributeTok{field=}\StringTok{"yes"}\NormalTok{,}\AttributeTok{fun=}\StringTok{"first"}\NormalTok{)}
\NormalTok{krumveida\_t}\OtherTok{=}\FunctionTok{rast}\NormalTok{(krumveida\_r)}
\NormalTok{augludarzi}\OtherTok{=}\FunctionTok{ifel}\NormalTok{(simple\_landscape}\SpecialCharTok{==}\DecValTok{420}\SpecialCharTok{|}\NormalTok{simple\_landscape}\SpecialCharTok{==}\DecValTok{620}\NormalTok{,}\DecValTok{1}\NormalTok{,}\DecValTok{0}\NormalTok{)}
\NormalTok{apvienoti}\OtherTok{=}\FunctionTok{cover}\NormalTok{(krumveida\_t,augludarzi)}
\FunctionTok{plot}\NormalTok{(apvienoti)}

\NormalTok{i2e\_rez}\OtherTok{=}\NormalTok{egvtools}\SpecialCharTok{::}\FunctionTok{input2egv}\NormalTok{(}\AttributeTok{input=}\NormalTok{apvienoti,}
              \AttributeTok{egv\_template=} \StringTok{"./Templates/TemplateRasters/LV100m\_10km.tif"}\NormalTok{,}
              \AttributeTok{summary\_function =} \StringTok{"average"}\NormalTok{,}
              \AttributeTok{missing\_job =} \StringTok{"FillOutput"}\NormalTok{,}
              \AttributeTok{outlocation =} \StringTok{"./RasterGrids\_100m/2024/RAW/"}\NormalTok{,}
              \AttributeTok{outfilename =} \StringTok{"General\_ShrubsOrchards\_cell.tif"}\NormalTok{,}
              \AttributeTok{layername =} \StringTok{"egv\_434"}\NormalTok{,}
              \AttributeTok{idw\_weight =} \DecValTok{2}\NormalTok{,}
              \AttributeTok{plot\_gaps =} \ConstantTok{FALSE}\NormalTok{,}\AttributeTok{plot\_final =} \ConstantTok{TRUE}\NormalTok{)}
\NormalTok{i2e\_rez}
\FunctionTok{rm}\NormalTok{(apvienoti)}
\FunctionTok{rm}\NormalTok{(i2e\_rez)}
\FunctionTok{rm}\NormalTok{(ilggadigiekrumveida)}
\FunctionTok{rm}\NormalTok{(krumveida\_r)}
\FunctionTok{rm}\NormalTok{(krumveida\_t)}
\FunctionTok{rm}\NormalTok{(augludarzi)}

\FunctionTok{rm}\NormalTok{(kodi)}
\FunctionTok{rm}\NormalTok{(lad)}

\CommentTok{\# standardisation {-}{-}{-}{-}}
\ControlFlowTok{if}\NormalTok{(}\SpecialCharTok{!}\FunctionTok{require}\NormalTok{(terra)) \{}\FunctionTok{install.packages}\NormalTok{(}\StringTok{"terra"}\NormalTok{); }\FunctionTok{require}\NormalTok{(terra)\}}
\ControlFlowTok{if}\NormalTok{(}\SpecialCharTok{!}\FunctionTok{require}\NormalTok{(tidyverse)) \{}\FunctionTok{install.packages}\NormalTok{(}\StringTok{"tidyverse"}\NormalTok{); }\FunctionTok{require}\NormalTok{(tidyverse)\}}

\NormalTok{nosaukums}\OtherTok{=}\StringTok{"General\_ShrubsOrchards\_cell.tif"}
\NormalTok{ielasisanas\_cels}\OtherTok{=}\FunctionTok{paste0}\NormalTok{(}\StringTok{"./RasterGrids\_100m/2024/RAW/"}\NormalTok{,nosaukums)}
\NormalTok{saglabasanas\_cels}\OtherTok{=}\FunctionTok{paste0}\NormalTok{(}\StringTok{"./RasterGrids\_100m/2024/Scaled/"}\NormalTok{,nosaukums)}
\NormalTok{slanis}\OtherTok{=}\FunctionTok{rast}\NormalTok{(ielasisanas\_cels)}
\NormalTok{videjais}\OtherTok{=}\FunctionTok{global}\NormalTok{(slanis,}\AttributeTok{fun=}\StringTok{"mean"}\NormalTok{,}\AttributeTok{na.rm=}\ConstantTok{TRUE}\NormalTok{)}
\NormalTok{centrets}\OtherTok{=}\NormalTok{slanis}\SpecialCharTok{{-}}\NormalTok{videjais[,}\DecValTok{1}\NormalTok{]}
\NormalTok{standartnovirze}\OtherTok{=}\NormalTok{terra}\SpecialCharTok{::}\FunctionTok{global}\NormalTok{(centrets,}\AttributeTok{fun=}\StringTok{"rms"}\NormalTok{,}\AttributeTok{na.rm=}\ConstantTok{TRUE}\NormalTok{)}
\NormalTok{merogots}\OtherTok{=}\NormalTok{centrets}\SpecialCharTok{/}\NormalTok{standartnovirze[,}\DecValTok{1}\NormalTok{]}
\FunctionTok{writeRaster}\NormalTok{(merogots,}
      \AttributeTok{filename=}\NormalTok{saglabasanas\_cels,}
      \AttributeTok{overwrite=}\ConstantTok{TRUE}\NormalTok{)}
\end{Highlighting}
\end{Shaded}

\section{General\_ShrubsOrchards\_r500}\label{ch06.435}

\textbf{filename:} \texttt{General\_ShrubsOrchards\_r500.tif}

\textbf{layername:} \texttt{egv\_435}

\textbf{English name:} Fractional cover of Shrubs, Young stands, Orchards within the
0.5 km landscape

\textbf{Latvian name:} Krūmāju, jaunaudžu un augļudārzu platības īpatsvars 0,5 km
ainavā

\textbf{Procedure:} The cover fraction within a radius of 500 m around the analysis grid cell is
calculated as the area-weighted sum of the \hyperref[ch06.434]{analysis cells} inside the
buffer, using the workflow \texttt{egvtools::radius\_function()}. During the calculation of the landscape metric,
inverse distance weighted (power = 2) gap filling on the output is applied
to ensure no missing values at the edges. Then the layer is rewritten to set
its name. Finally, the layer is standardised by subtracting the arithmetic
mean and dividing by the root mean squared error.

\begin{Shaded}
\begin{Highlighting}[]
\CommentTok{\# libs {-}{-}{-}{-}}
\ControlFlowTok{if}\NormalTok{(}\SpecialCharTok{!}\FunctionTok{require}\NormalTok{(terra)) \{}\FunctionTok{install.packages}\NormalTok{(}\StringTok{"terra"}\NormalTok{); }\FunctionTok{require}\NormalTok{(terra)\}}
\ControlFlowTok{if}\NormalTok{(}\SpecialCharTok{!}\FunctionTok{require}\NormalTok{(egvtools)) \{remotes}\SpecialCharTok{::}\FunctionTok{install\_github}\NormalTok{(}\StringTok{"aavotins/egvtools"}\NormalTok{); }\FunctionTok{require}\NormalTok{(egvtools)\}}


\CommentTok{\# Templates {-}{-}{-}{-}{-}}
\NormalTok{template100}\OtherTok{=}\FunctionTok{rast}\NormalTok{(}\StringTok{"./Templates/TemplateRasters/LV100m\_10km.tif"}\NormalTok{)}

\CommentTok{\# radii {-}{-}{-}{-}}
\FunctionTok{radius\_function}\NormalTok{(}
 \AttributeTok{kvadrati\_path =} \StringTok{"./Templates/TemplateGrids/tiles/"}\NormalTok{,}
 \AttributeTok{radii\_path   =} \StringTok{"./Templates/TemplateGridPoints/tiles/"}\NormalTok{,}
 \AttributeTok{tikls100\_path =} \StringTok{"./Templates/TemplateGrids/tikls100\_sauzeme.parquet"}\NormalTok{,}
 \AttributeTok{template\_path =} \StringTok{"./Templates/TemplateRasters/LV100m\_10km.tif"}\NormalTok{,}
 \AttributeTok{input\_layers  =} \FunctionTok{c}\NormalTok{(}\StringTok{"./RasterGrids\_100m/2024/RAW/General\_ShrubsOrchards\_cell.tif"}\NormalTok{),}
 \AttributeTok{layer\_prefixes =} \FunctionTok{c}\NormalTok{(}\StringTok{"General\_ShrubsOrchards"}\NormalTok{),}
 \AttributeTok{output\_dir   =} \StringTok{"./RasterGrids\_100m/2024/RAW/"}\NormalTok{,}
 \AttributeTok{n\_workers   =} \DecValTok{6}\NormalTok{,}
 \AttributeTok{radii     =} \FunctionTok{c}\NormalTok{(}\StringTok{"r500"}\NormalTok{),}
 \AttributeTok{radius\_mode  =} \StringTok{"sparse"}\NormalTok{,}
 \AttributeTok{extract\_fun  =} \StringTok{"mean"}\NormalTok{,}
 \AttributeTok{fill\_missing  =} \ConstantTok{TRUE}\NormalTok{,}
 \AttributeTok{IDW\_weight   =} \DecValTok{2}\NormalTok{,}
 \AttributeTok{future\_max\_size =} \DecValTok{40} \SpecialCharTok{*} \DecValTok{1024}\SpecialCharTok{\^{}}\DecValTok{3}\NormalTok{)}


\CommentTok{\# General\_ShrubsOrchards\_r500.tif   egv\_435}
\NormalTok{slanis}\OtherTok{=}\FunctionTok{rast}\NormalTok{(}\StringTok{"./RasterGrids\_100m/2024/RAW/General\_ShrubsOrchards\_r500.tif"}\NormalTok{)}
\FunctionTok{names}\NormalTok{(slanis)}\OtherTok{=}\StringTok{"egv\_435"}
\NormalTok{slanis2}\OtherTok{=}\FunctionTok{project}\NormalTok{(slanis,template100)}
\FunctionTok{writeRaster}\NormalTok{(slanis2,}
      \StringTok{"./RasterGrids\_100m/2024/RAW/General\_ShrubsOrchards\_r500.tif"}\NormalTok{,}
      \AttributeTok{overwrite=}\ConstantTok{TRUE}\NormalTok{)}

\CommentTok{\# standardisation {-}{-}{-}{-}}
\ControlFlowTok{if}\NormalTok{(}\SpecialCharTok{!}\FunctionTok{require}\NormalTok{(terra)) \{}\FunctionTok{install.packages}\NormalTok{(}\StringTok{"terra"}\NormalTok{); }\FunctionTok{require}\NormalTok{(terra)\}}
\ControlFlowTok{if}\NormalTok{(}\SpecialCharTok{!}\FunctionTok{require}\NormalTok{(tidyverse)) \{}\FunctionTok{install.packages}\NormalTok{(}\StringTok{"tidyverse"}\NormalTok{); }\FunctionTok{require}\NormalTok{(tidyverse)\}}

\NormalTok{nosaukums}\OtherTok{=}\StringTok{"General\_ShrubsOrchards\_r500.tif"}
\NormalTok{ielasisanas\_cels}\OtherTok{=}\FunctionTok{paste0}\NormalTok{(}\StringTok{"./RasterGrids\_100m/2024/RAW/"}\NormalTok{,nosaukums)}
\NormalTok{saglabasanas\_cels}\OtherTok{=}\FunctionTok{paste0}\NormalTok{(}\StringTok{"./RasterGrids\_100m/2024/Scaled/"}\NormalTok{,nosaukums)}
\NormalTok{slanis}\OtherTok{=}\FunctionTok{rast}\NormalTok{(ielasisanas\_cels)}
\NormalTok{videjais}\OtherTok{=}\FunctionTok{global}\NormalTok{(slanis,}\AttributeTok{fun=}\StringTok{"mean"}\NormalTok{,}\AttributeTok{na.rm=}\ConstantTok{TRUE}\NormalTok{)}
\NormalTok{centrets}\OtherTok{=}\NormalTok{slanis}\SpecialCharTok{{-}}\NormalTok{videjais[,}\DecValTok{1}\NormalTok{]}
\NormalTok{standartnovirze}\OtherTok{=}\NormalTok{terra}\SpecialCharTok{::}\FunctionTok{global}\NormalTok{(centrets,}\AttributeTok{fun=}\StringTok{"rms"}\NormalTok{,}\AttributeTok{na.rm=}\ConstantTok{TRUE}\NormalTok{)}
\NormalTok{merogots}\OtherTok{=}\NormalTok{centrets}\SpecialCharTok{/}\NormalTok{standartnovirze[,}\DecValTok{1}\NormalTok{]}
\FunctionTok{writeRaster}\NormalTok{(merogots,}
      \AttributeTok{filename=}\NormalTok{saglabasanas\_cels,}
      \AttributeTok{overwrite=}\ConstantTok{TRUE}\NormalTok{)}
\end{Highlighting}
\end{Shaded}

\section{General\_ShrubsOrchards\_r1250}\label{ch06.436}

\textbf{filename:} \texttt{General\_ShrubsOrchards\_r1250.tif}

\textbf{layername:} \texttt{egv\_436}

\textbf{English name:} Fractional cover of Shrubs, Young stands, Orchards within the
1.25 km landscape

\textbf{Latvian name:} Krūmāju, jaunaudžu un augļudārzu platības īpatsvars 1,25 km
ainavā

\textbf{Procedure:} The cover fraction within a radius of 1250 m around the analysis grid cell
is calculated as the area-weighted sum of the \hyperref[ch06.434]{analysis cells} inside
the buffer, using the workflow \texttt{egvtools::radius\_function()}. During the calculation of the landscape
metric, inverse distance weighted (power = 2) gap filling on the output is
applied to ensure no missing values at the edges. Then the layer is
rewritten to set its name. Finally, the layer is standardised by
subtracting the arithmetic mean and dividing by the root mean squared error.

\begin{Shaded}
\begin{Highlighting}[]
\CommentTok{\# libs {-}{-}{-}{-}}
\ControlFlowTok{if}\NormalTok{(}\SpecialCharTok{!}\FunctionTok{require}\NormalTok{(terra)) \{}\FunctionTok{install.packages}\NormalTok{(}\StringTok{"terra"}\NormalTok{); }\FunctionTok{require}\NormalTok{(terra)\}}
\ControlFlowTok{if}\NormalTok{(}\SpecialCharTok{!}\FunctionTok{require}\NormalTok{(egvtools)) \{remotes}\SpecialCharTok{::}\FunctionTok{install\_github}\NormalTok{(}\StringTok{"aavotins/egvtools"}\NormalTok{); }\FunctionTok{require}\NormalTok{(egvtools)\}}


\CommentTok{\# Templates {-}{-}{-}{-}{-}}
\NormalTok{template100}\OtherTok{=}\FunctionTok{rast}\NormalTok{(}\StringTok{"./Templates/TemplateRasters/LV100m\_10km.tif"}\NormalTok{)}

\CommentTok{\# radii {-}{-}{-}{-}}
\FunctionTok{radius\_function}\NormalTok{(}
 \AttributeTok{kvadrati\_path =} \StringTok{"./Templates/TemplateGrids/tiles/"}\NormalTok{,}
 \AttributeTok{radii\_path   =} \StringTok{"./Templates/TemplateGridPoints/tiles/"}\NormalTok{,}
 \AttributeTok{tikls100\_path =} \StringTok{"./Templates/TemplateGrids/tikls100\_sauzeme.parquet"}\NormalTok{,}
 \AttributeTok{template\_path =} \StringTok{"./Templates/TemplateRasters/LV100m\_10km.tif"}\NormalTok{,}
 \AttributeTok{input\_layers  =} \FunctionTok{c}\NormalTok{(}\StringTok{"./RasterGrids\_100m/2024/RAW/General\_ShrubsOrchards\_cell.tif"}\NormalTok{),}
 \AttributeTok{layer\_prefixes =} \FunctionTok{c}\NormalTok{(}\StringTok{"General\_ShrubsOrchards"}\NormalTok{),}
 \AttributeTok{output\_dir   =} \StringTok{"./RasterGrids\_100m/2024/RAW/"}\NormalTok{,}
 \AttributeTok{n\_workers   =} \DecValTok{6}\NormalTok{,}
 \AttributeTok{radii     =} \FunctionTok{c}\NormalTok{(}\StringTok{"r1250"}\NormalTok{),}
 \AttributeTok{radius\_mode  =} \StringTok{"sparse"}\NormalTok{,}
 \AttributeTok{extract\_fun  =} \StringTok{"mean"}\NormalTok{,}
 \AttributeTok{fill\_missing  =} \ConstantTok{TRUE}\NormalTok{,}
 \AttributeTok{IDW\_weight   =} \DecValTok{2}\NormalTok{,}
 \AttributeTok{future\_max\_size =} \DecValTok{40} \SpecialCharTok{*} \DecValTok{1024}\SpecialCharTok{\^{}}\DecValTok{3}\NormalTok{)}


\CommentTok{\# General\_ShrubsOrchards\_r1250.tif  egv\_436}
\NormalTok{slanis}\OtherTok{=}\FunctionTok{rast}\NormalTok{(}\StringTok{"./RasterGrids\_100m/2024/RAW/General\_ShrubsOrchards\_r1250.tif"}\NormalTok{)}
\FunctionTok{names}\NormalTok{(slanis)}\OtherTok{=}\StringTok{"egv\_436"}
\NormalTok{slanis2}\OtherTok{=}\FunctionTok{project}\NormalTok{(slanis,template100)}
\FunctionTok{writeRaster}\NormalTok{(slanis2,}
      \StringTok{"./RasterGrids\_100m/2024/RAW/General\_ShrubsOrchards\_r1250.tif"}\NormalTok{,}
      \AttributeTok{overwrite=}\ConstantTok{TRUE}\NormalTok{)}

\CommentTok{\# standardisation {-}{-}{-}{-}}
\ControlFlowTok{if}\NormalTok{(}\SpecialCharTok{!}\FunctionTok{require}\NormalTok{(terra)) \{}\FunctionTok{install.packages}\NormalTok{(}\StringTok{"terra"}\NormalTok{); }\FunctionTok{require}\NormalTok{(terra)\}}
\ControlFlowTok{if}\NormalTok{(}\SpecialCharTok{!}\FunctionTok{require}\NormalTok{(tidyverse)) \{}\FunctionTok{install.packages}\NormalTok{(}\StringTok{"tidyverse"}\NormalTok{); }\FunctionTok{require}\NormalTok{(tidyverse)\}}

\NormalTok{nosaukums}\OtherTok{=}\StringTok{"General\_ShrubsOrchards\_r1250.tif"}
\NormalTok{ielasisanas\_cels}\OtherTok{=}\FunctionTok{paste0}\NormalTok{(}\StringTok{"./RasterGrids\_100m/2024/RAW/"}\NormalTok{,nosaukums)}
\NormalTok{saglabasanas\_cels}\OtherTok{=}\FunctionTok{paste0}\NormalTok{(}\StringTok{"./RasterGrids\_100m/2024/Scaled/"}\NormalTok{,nosaukums)}
\NormalTok{slanis}\OtherTok{=}\FunctionTok{rast}\NormalTok{(ielasisanas\_cels)}
\NormalTok{videjais}\OtherTok{=}\FunctionTok{global}\NormalTok{(slanis,}\AttributeTok{fun=}\StringTok{"mean"}\NormalTok{,}\AttributeTok{na.rm=}\ConstantTok{TRUE}\NormalTok{)}
\NormalTok{centrets}\OtherTok{=}\NormalTok{slanis}\SpecialCharTok{{-}}\NormalTok{videjais[,}\DecValTok{1}\NormalTok{]}
\NormalTok{standartnovirze}\OtherTok{=}\NormalTok{terra}\SpecialCharTok{::}\FunctionTok{global}\NormalTok{(centrets,}\AttributeTok{fun=}\StringTok{"rms"}\NormalTok{,}\AttributeTok{na.rm=}\ConstantTok{TRUE}\NormalTok{)}
\NormalTok{merogots}\OtherTok{=}\NormalTok{centrets}\SpecialCharTok{/}\NormalTok{standartnovirze[,}\DecValTok{1}\NormalTok{]}
\FunctionTok{writeRaster}\NormalTok{(merogots,}
      \AttributeTok{filename=}\NormalTok{saglabasanas\_cels,}
      \AttributeTok{overwrite=}\ConstantTok{TRUE}\NormalTok{)}
\end{Highlighting}
\end{Shaded}

\section{General\_ShrubsOrchards\_r3000}\label{ch06.437}

\textbf{filename:} \texttt{General\_ShrubsOrchards\_r3000.tif}

\textbf{layername:} \texttt{egv\_437}

\textbf{English name:} Fractional cover of Shrubs, Young stands, Orchards within the
3 km landscape

\textbf{Latvian name:} Krūmāju, jaunaudžu un augļudārzu platības īpatsvars 3 km
ainavā

\textbf{Procedure:} The cover fraction within a radius of 3000 m around the analysis grid cell
is calculated as the area-weighted sum of the \hyperref[ch06.434]{analysis cells} inside
the buffer, using the workflow \texttt{egvtools::radius\_function()}. During the calculation of the landscape
metric, inverse distance weighted (power = 2) gap filling on the output is
applied to ensure no missing values at the edges. Then the layer is
rewritten to set its name. Finally, the layer is standardised by
subtracting the arithmetic mean and dividing by the root mean squared error.

\begin{Shaded}
\begin{Highlighting}[]
\CommentTok{\# libs {-}{-}{-}{-}}
\ControlFlowTok{if}\NormalTok{(}\SpecialCharTok{!}\FunctionTok{require}\NormalTok{(terra)) \{}\FunctionTok{install.packages}\NormalTok{(}\StringTok{"terra"}\NormalTok{); }\FunctionTok{require}\NormalTok{(terra)\}}
\ControlFlowTok{if}\NormalTok{(}\SpecialCharTok{!}\FunctionTok{require}\NormalTok{(egvtools)) \{remotes}\SpecialCharTok{::}\FunctionTok{install\_github}\NormalTok{(}\StringTok{"aavotins/egvtools"}\NormalTok{); }\FunctionTok{require}\NormalTok{(egvtools)\}}


\CommentTok{\# Templates {-}{-}{-}{-}{-}}
\NormalTok{template100}\OtherTok{=}\FunctionTok{rast}\NormalTok{(}\StringTok{"./Templates/TemplateRasters/LV100m\_10km.tif"}\NormalTok{)}

\CommentTok{\# radii {-}{-}{-}{-}}
\FunctionTok{radius\_function}\NormalTok{(}
 \AttributeTok{kvadrati\_path =} \StringTok{"./Templates/TemplateGrids/tiles/"}\NormalTok{,}
 \AttributeTok{radii\_path   =} \StringTok{"./Templates/TemplateGridPoints/tiles/"}\NormalTok{,}
 \AttributeTok{tikls100\_path =} \StringTok{"./Templates/TemplateGrids/tikls100\_sauzeme.parquet"}\NormalTok{,}
 \AttributeTok{template\_path =} \StringTok{"./Templates/TemplateRasters/LV100m\_10km.tif"}\NormalTok{,}
 \AttributeTok{input\_layers  =} \FunctionTok{c}\NormalTok{(}\StringTok{"./RasterGrids\_100m/2024/RAW/General\_ShrubsOrchards\_cell.tif"}\NormalTok{),}
 \AttributeTok{layer\_prefixes =} \FunctionTok{c}\NormalTok{(}\StringTok{"General\_ShrubsOrchards"}\NormalTok{),}
 \AttributeTok{output\_dir   =} \StringTok{"./RasterGrids\_100m/2024/RAW/"}\NormalTok{,}
 \AttributeTok{n\_workers   =} \DecValTok{6}\NormalTok{,}
 \AttributeTok{radii     =} \FunctionTok{c}\NormalTok{(}\StringTok{"r3000"}\NormalTok{),}
 \AttributeTok{radius\_mode  =} \StringTok{"sparse"}\NormalTok{,}
 \AttributeTok{extract\_fun  =} \StringTok{"mean"}\NormalTok{,}
 \AttributeTok{fill\_missing  =} \ConstantTok{TRUE}\NormalTok{,}
 \AttributeTok{IDW\_weight   =} \DecValTok{2}\NormalTok{,}
 \AttributeTok{future\_max\_size =} \DecValTok{40} \SpecialCharTok{*} \DecValTok{1024}\SpecialCharTok{\^{}}\DecValTok{3}\NormalTok{)}


\CommentTok{\# General\_ShrubsOrchards\_r3000.tif  egv\_437}
\NormalTok{slanis}\OtherTok{=}\FunctionTok{rast}\NormalTok{(}\StringTok{"./RasterGrids\_100m/2024/RAW/General\_ShrubsOrchards\_r3000.tif"}\NormalTok{)}
\FunctionTok{names}\NormalTok{(slanis)}\OtherTok{=}\StringTok{"egv\_437"}
\NormalTok{slanis2}\OtherTok{=}\FunctionTok{project}\NormalTok{(slanis,template100)}
\FunctionTok{writeRaster}\NormalTok{(slanis2,}
      \StringTok{"./RasterGrids\_100m/2024/RAW/General\_ShrubsOrchards\_r3000.tif"}\NormalTok{,}
      \AttributeTok{overwrite=}\ConstantTok{TRUE}\NormalTok{)}

\CommentTok{\# standardisation {-}{-}{-}{-}}
\ControlFlowTok{if}\NormalTok{(}\SpecialCharTok{!}\FunctionTok{require}\NormalTok{(terra)) \{}\FunctionTok{install.packages}\NormalTok{(}\StringTok{"terra"}\NormalTok{); }\FunctionTok{require}\NormalTok{(terra)\}}
\ControlFlowTok{if}\NormalTok{(}\SpecialCharTok{!}\FunctionTok{require}\NormalTok{(tidyverse)) \{}\FunctionTok{install.packages}\NormalTok{(}\StringTok{"tidyverse"}\NormalTok{); }\FunctionTok{require}\NormalTok{(tidyverse)\}}

\NormalTok{nosaukums}\OtherTok{=}\StringTok{"General\_ShrubsOrchards\_r3000.tif"}
\NormalTok{ielasisanas\_cels}\OtherTok{=}\FunctionTok{paste0}\NormalTok{(}\StringTok{"./RasterGrids\_100m/2024/RAW/"}\NormalTok{,nosaukums)}
\NormalTok{saglabasanas\_cels}\OtherTok{=}\FunctionTok{paste0}\NormalTok{(}\StringTok{"./RasterGrids\_100m/2024/Scaled/"}\NormalTok{,nosaukums)}
\NormalTok{slanis}\OtherTok{=}\FunctionTok{rast}\NormalTok{(ielasisanas\_cels)}
\NormalTok{videjais}\OtherTok{=}\FunctionTok{global}\NormalTok{(slanis,}\AttributeTok{fun=}\StringTok{"mean"}\NormalTok{,}\AttributeTok{na.rm=}\ConstantTok{TRUE}\NormalTok{)}
\NormalTok{centrets}\OtherTok{=}\NormalTok{slanis}\SpecialCharTok{{-}}\NormalTok{videjais[,}\DecValTok{1}\NormalTok{]}
\NormalTok{standartnovirze}\OtherTok{=}\NormalTok{terra}\SpecialCharTok{::}\FunctionTok{global}\NormalTok{(centrets,}\AttributeTok{fun=}\StringTok{"rms"}\NormalTok{,}\AttributeTok{na.rm=}\ConstantTok{TRUE}\NormalTok{)}
\NormalTok{merogots}\OtherTok{=}\NormalTok{centrets}\SpecialCharTok{/}\NormalTok{standartnovirze[,}\DecValTok{1}\NormalTok{]}
\FunctionTok{writeRaster}\NormalTok{(merogots,}
      \AttributeTok{filename=}\NormalTok{saglabasanas\_cels,}
      \AttributeTok{overwrite=}\ConstantTok{TRUE}\NormalTok{)}
\end{Highlighting}
\end{Shaded}

\section{General\_ShrubsOrchards\_r10000}\label{ch06.438}

\textbf{filename:} \texttt{General\_ShrubsOrchards\_r10000.tif}

\textbf{layername:} \texttt{egv\_438}

\textbf{English name:} Fractional cover of Shrubs, Young stands, Orchards within the
10 km landscape

\textbf{Latvian name:} Krūmāju, jaunaudžu un augļudārzu platības īpatsvars 10 km
ainavā

\textbf{Procedure:} The cover fraction within a radius of 10000 m around the analysis grid cell
is calculated as the area-weighted sum of the \hyperref[ch06.434]{analysis cells} inside
the buffer, using the workflow \texttt{egvtools::radius\_function()}. During the calculation of the landscape
metric, inverse distance weighted (power = 2) gap filling on the output is
applied to ensure no missing values at the edges. Then the layer is
rewritten to set its name. Finally, the layer is standardised by
subtracting the arithmetic mean and dividing by the root mean squared error.

\begin{Shaded}
\begin{Highlighting}[]
\CommentTok{\# libs {-}{-}{-}{-}}
\ControlFlowTok{if}\NormalTok{(}\SpecialCharTok{!}\FunctionTok{require}\NormalTok{(terra)) \{}\FunctionTok{install.packages}\NormalTok{(}\StringTok{"terra"}\NormalTok{); }\FunctionTok{require}\NormalTok{(terra)\}}
\ControlFlowTok{if}\NormalTok{(}\SpecialCharTok{!}\FunctionTok{require}\NormalTok{(egvtools)) \{remotes}\SpecialCharTok{::}\FunctionTok{install\_github}\NormalTok{(}\StringTok{"aavotins/egvtools"}\NormalTok{); }\FunctionTok{require}\NormalTok{(egvtools)\}}


\CommentTok{\# Templates {-}{-}{-}{-}{-}}
\NormalTok{template100}\OtherTok{=}\FunctionTok{rast}\NormalTok{(}\StringTok{"./Templates/TemplateRasters/LV100m\_10km.tif"}\NormalTok{)}

\CommentTok{\# radii {-}{-}{-}{-}}
\FunctionTok{radius\_function}\NormalTok{(}
 \AttributeTok{kvadrati\_path =} \StringTok{"./Templates/TemplateGrids/tiles/"}\NormalTok{,}
 \AttributeTok{radii\_path   =} \StringTok{"./Templates/TemplateGridPoints/tiles/"}\NormalTok{,}
 \AttributeTok{tikls100\_path =} \StringTok{"./Templates/TemplateGrids/tikls100\_sauzeme.parquet"}\NormalTok{,}
 \AttributeTok{template\_path =} \StringTok{"./Templates/TemplateRasters/LV100m\_10km.tif"}\NormalTok{,}
 \AttributeTok{input\_layers  =} \FunctionTok{c}\NormalTok{(}\StringTok{"./RasterGrids\_100m/2024/RAW/General\_ShrubsOrchards\_cell.tif"}\NormalTok{),}
 \AttributeTok{layer\_prefixes =} \FunctionTok{c}\NormalTok{(}\StringTok{"General\_ShrubsOrchards"}\NormalTok{),}
 \AttributeTok{output\_dir   =} \StringTok{"./RasterGrids\_100m/2024/RAW/"}\NormalTok{,}
 \AttributeTok{n\_workers   =} \DecValTok{6}\NormalTok{,}
 \AttributeTok{radii     =} \FunctionTok{c}\NormalTok{(}\StringTok{"r10000"}\NormalTok{),}
 \AttributeTok{radius\_mode  =} \StringTok{"sparse"}\NormalTok{,}
 \AttributeTok{extract\_fun  =} \StringTok{"mean"}\NormalTok{,}
 \AttributeTok{fill\_missing  =} \ConstantTok{TRUE}\NormalTok{,}
 \AttributeTok{IDW\_weight   =} \DecValTok{2}\NormalTok{,}
 \AttributeTok{future\_max\_size =} \DecValTok{40} \SpecialCharTok{*} \DecValTok{1024}\SpecialCharTok{\^{}}\DecValTok{3}\NormalTok{)}


\CommentTok{\# General\_ShrubsOrchards\_r10000.tif egv\_438}
\NormalTok{slanis}\OtherTok{=}\FunctionTok{rast}\NormalTok{(}\StringTok{"./RasterGrids\_100m/2024/RAW/General\_ShrubsOrchards\_r10000.tif"}\NormalTok{)}
\FunctionTok{names}\NormalTok{(slanis)}\OtherTok{=}\StringTok{"egv\_438"}
\NormalTok{slanis2}\OtherTok{=}\FunctionTok{project}\NormalTok{(slanis,template100)}
\FunctionTok{writeRaster}\NormalTok{(slanis2,}
      \StringTok{"./RasterGrids\_100m/2024/RAW/General\_ShrubsOrchards\_r10000.tif"}\NormalTok{,}
      \AttributeTok{overwrite=}\ConstantTok{TRUE}\NormalTok{)}

\CommentTok{\# standardisation {-}{-}{-}{-}}
\ControlFlowTok{if}\NormalTok{(}\SpecialCharTok{!}\FunctionTok{require}\NormalTok{(terra)) \{}\FunctionTok{install.packages}\NormalTok{(}\StringTok{"terra"}\NormalTok{); }\FunctionTok{require}\NormalTok{(terra)\}}
\ControlFlowTok{if}\NormalTok{(}\SpecialCharTok{!}\FunctionTok{require}\NormalTok{(tidyverse)) \{}\FunctionTok{install.packages}\NormalTok{(}\StringTok{"tidyverse"}\NormalTok{); }\FunctionTok{require}\NormalTok{(tidyverse)\}}

\NormalTok{nosaukums}\OtherTok{=}\StringTok{"General\_ShrubsOrchards\_r10000.tif"}
\NormalTok{ielasisanas\_cels}\OtherTok{=}\FunctionTok{paste0}\NormalTok{(}\StringTok{"./RasterGrids\_100m/2024/RAW/"}\NormalTok{,nosaukums)}
\NormalTok{saglabasanas\_cels}\OtherTok{=}\FunctionTok{paste0}\NormalTok{(}\StringTok{"./RasterGrids\_100m/2024/Scaled/"}\NormalTok{,nosaukums)}
\NormalTok{slanis}\OtherTok{=}\FunctionTok{rast}\NormalTok{(ielasisanas\_cels)}
\NormalTok{videjais}\OtherTok{=}\FunctionTok{global}\NormalTok{(slanis,}\AttributeTok{fun=}\StringTok{"mean"}\NormalTok{,}\AttributeTok{na.rm=}\ConstantTok{TRUE}\NormalTok{)}
\NormalTok{centrets}\OtherTok{=}\NormalTok{slanis}\SpecialCharTok{{-}}\NormalTok{videjais[,}\DecValTok{1}\NormalTok{]}
\NormalTok{standartnovirze}\OtherTok{=}\NormalTok{terra}\SpecialCharTok{::}\FunctionTok{global}\NormalTok{(centrets,}\AttributeTok{fun=}\StringTok{"rms"}\NormalTok{,}\AttributeTok{na.rm=}\ConstantTok{TRUE}\NormalTok{)}
\NormalTok{merogots}\OtherTok{=}\NormalTok{centrets}\SpecialCharTok{/}\NormalTok{standartnovirze[,}\DecValTok{1}\NormalTok{]}
\FunctionTok{writeRaster}\NormalTok{(merogots,}
      \AttributeTok{filename=}\NormalTok{saglabasanas\_cels,}
      \AttributeTok{overwrite=}\ConstantTok{TRUE}\NormalTok{)}
\end{Highlighting}
\end{Shaded}

\section{General\_ShrubsOrchardsGardens\_cell}\label{ch06.439}

\textbf{filename:} \texttt{General\_ShrubsOrchardsGardens\_cell.tif}

\textbf{layername:} \texttt{egv\_439}

\textbf{English name:} Fractional cover of Shrubs, Young stands, Orchards, Allotment
gardens within the analysis cell (1 ha)

\textbf{Latvian name:} Krūmāju, jaunaudžu, augļudārzu un vasarnīcu kompleksu platības
īpatsvars analīzes šūnā (1 ha)

\textbf{Procedure:} First, agricultural parcels declared as short term coppice are
selected from the \hyperref[Ch04.02]{Rural Support Service's information on declared fields}
and rasterised to match inputs. Then orchards, allotment gardens and shrubs-low
forest stands from the \hyperref[Ch05.03]{Landscape classification} are selected (values
between 400 and 500 or equal to 620 are reclassified to value 1, others as 0).
The first layer is then covered over the second. The resulting layer
is then aggregated to EGV resolution using the workflow \texttt{egvtools::input2egv()}, which
calculates the arithmetic mean to determine the cover fraction. During
aggregation, inverse distance weighted (power = 2) gap filling on the output is
applied to ensure no missing values at the edges. Finally, the layer is
standardised by subtracting the arithmetic mean and dividing by the root mean squared
error.

\begin{Shaded}
\begin{Highlighting}[]
\CommentTok{\# libs {-}{-}{-}{-}}
\ControlFlowTok{if}\NormalTok{(}\SpecialCharTok{!}\FunctionTok{require}\NormalTok{(egvtools)) \{remotes}\SpecialCharTok{::}\FunctionTok{install\_github}\NormalTok{(}\StringTok{"aavotins/egvtools"}\NormalTok{); }\FunctionTok{require}\NormalTok{(egvtools)\}}
\ControlFlowTok{if}\NormalTok{(}\SpecialCharTok{!}\FunctionTok{require}\NormalTok{(terra)) \{}\FunctionTok{install.packages}\NormalTok{(}\StringTok{"terra"}\NormalTok{); }\FunctionTok{require}\NormalTok{(terra)\}}
\ControlFlowTok{if}\NormalTok{(}\SpecialCharTok{!}\FunctionTok{require}\NormalTok{(sf)) \{}\FunctionTok{install.packages}\NormalTok{(}\StringTok{"sf"}\NormalTok{); }\FunctionTok{require}\NormalTok{(sf)\}}
\ControlFlowTok{if}\NormalTok{(}\SpecialCharTok{!}\FunctionTok{require}\NormalTok{(tidyverse)) \{}\FunctionTok{install.packages}\NormalTok{(}\StringTok{"tidyverse"}\NormalTok{); }\FunctionTok{require}\NormalTok{(tidyverse)\}}
\ControlFlowTok{if}\NormalTok{(}\SpecialCharTok{!}\FunctionTok{require}\NormalTok{(sfarrow)) \{}\FunctionTok{install.packages}\NormalTok{(}\StringTok{"sfarrow"}\NormalTok{); }\FunctionTok{require}\NormalTok{(sfarrow)\}}
\ControlFlowTok{if}\NormalTok{(}\SpecialCharTok{!}\FunctionTok{require}\NormalTok{(readxl)) \{}\FunctionTok{install.packages}\NormalTok{(}\StringTok{"readxl"}\NormalTok{); }\FunctionTok{require}\NormalTok{(readxl)\}}
\ControlFlowTok{if}\NormalTok{(}\SpecialCharTok{!}\FunctionTok{require}\NormalTok{(raster)) \{}\FunctionTok{install.packages}\NormalTok{(}\StringTok{"raster"}\NormalTok{); }\FunctionTok{require}\NormalTok{(raster)\}}
\ControlFlowTok{if}\NormalTok{(}\SpecialCharTok{!}\FunctionTok{require}\NormalTok{(fasterize)) \{}\FunctionTok{install.packages}\NormalTok{(}\StringTok{"fasterize"}\NormalTok{); }\FunctionTok{require}\NormalTok{(fasterize)\}}

\CommentTok{\# templates {-}{-}{-}{-}}
\NormalTok{template100}\OtherTok{=}\FunctionTok{rast}\NormalTok{(}\StringTok{"./Templates/TemplateRasters/LV100m\_10km.tif"}\NormalTok{)}
\NormalTok{template10}\OtherTok{=}\FunctionTok{rast}\NormalTok{(}\StringTok{"./Templates/TemplateRasters/LV10m\_10km.tif"}\NormalTok{)}
\NormalTok{rastrs10}\OtherTok{=}\FunctionTok{raster}\NormalTok{(template10)}

\NormalTok{nulls10}\OtherTok{=}\FunctionTok{rast}\NormalTok{(}\StringTok{"./Templates/TemplateRasters/nulls\_LV10m\_10km.tif"}\NormalTok{)}
\NormalTok{nulls100}\OtherTok{=}\FunctionTok{rast}\NormalTok{(}\StringTok{"./Templates/TemplateRasters/nulls\_LV100m\_10km.tif"}\NormalTok{)}

\CommentTok{\# simple landscape {-}{-}{-}{-}}
\NormalTok{simple\_landscape}\OtherTok{=}\FunctionTok{rast}\NormalTok{(}\StringTok{"RasterGrids\_10m/2024/Ainava\_vienk\_mask.tif"}\NormalTok{)}


\CommentTok{\# General\_ShrubsOrchardsGardens\_cell.tif    egv\_439 {-}{-}{-}{-}}
\NormalTok{kodi}\OtherTok{=}\FunctionTok{read\_excel}\NormalTok{(}\StringTok{"./Geodata/2024/LAD/KulturuKodi\_2024.xlsx"}\NormalTok{)}
\NormalTok{kodi}\SpecialCharTok{$}\NormalTok{kods}\OtherTok{=}\FunctionTok{as.character}\NormalTok{(kodi}\SpecialCharTok{$}\NormalTok{kods)}
\FunctionTok{table}\NormalTok{(kodi}\SpecialCharTok{$}\NormalTok{SDM\_grupa\_sakums,}\AttributeTok{useNA=}\StringTok{"always"}\NormalTok{)}
\NormalTok{lad}\OtherTok{=}\NormalTok{sfarrow}\SpecialCharTok{::}\FunctionTok{st\_read\_parquet}\NormalTok{(}\StringTok{"./Geodata/2024/LAD/Lauki\_2024.parquet"}\NormalTok{)}
\NormalTok{lad}\SpecialCharTok{$}\NormalTok{yes}\OtherTok{=}\DecValTok{1}
\NormalTok{lad}\OtherTok{=}\NormalTok{lad }\SpecialCharTok{\%\textgreater{}\%} 
 \FunctionTok{left\_join}\NormalTok{(kodi,}\AttributeTok{by=}\FunctionTok{c}\NormalTok{(}\StringTok{"PRODUCT\_CODE"}\OtherTok{=}\StringTok{"kods"}\NormalTok{))}
\NormalTok{ilggadigiekrumveida}\OtherTok{=}\NormalTok{lad }\SpecialCharTok{\%\textgreater{}\%} 
 \FunctionTok{filter}\NormalTok{(SDM\_grupa\_sakums }\SpecialCharTok{==} \StringTok{"krūmveida ilggadīgie stādījumi"}\NormalTok{)}
\NormalTok{krumveida\_r}\OtherTok{=}\FunctionTok{fasterize}\NormalTok{(ilggadigiekrumveida,rastrs10,}\AttributeTok{field=}\StringTok{"yes"}\NormalTok{,}\AttributeTok{fun=}\StringTok{"first"}\NormalTok{)}
\NormalTok{krumveida\_t}\OtherTok{=}\FunctionTok{rast}\NormalTok{(krumveida\_r)}
\NormalTok{parejie}\OtherTok{=}\FunctionTok{ifel}\NormalTok{((simple\_landscape}\SpecialCharTok{\textgreater{}=}\DecValTok{400}\SpecialCharTok{\&}\NormalTok{simple\_landscape}\SpecialCharTok{\textless{}}\DecValTok{500}\NormalTok{)}\SpecialCharTok{|}
\NormalTok{        simple\_landscape}\SpecialCharTok{==}\DecValTok{620}\NormalTok{,}\DecValTok{1}\NormalTok{,}\DecValTok{0}\NormalTok{)}
\NormalTok{apvienoti}\OtherTok{=}\FunctionTok{cover}\NormalTok{(krumveida\_t,parejie)}
\FunctionTok{plot}\NormalTok{(apvienoti)}

\NormalTok{i2e\_rez}\OtherTok{=}\NormalTok{egvtools}\SpecialCharTok{::}\FunctionTok{input2egv}\NormalTok{(}\AttributeTok{input=}\NormalTok{apvienoti,}
              \AttributeTok{egv\_template=} \StringTok{"./Templates/TemplateRasters/LV100m\_10km.tif"}\NormalTok{,}
              \AttributeTok{summary\_function =} \StringTok{"average"}\NormalTok{,}
              \AttributeTok{missing\_job =} \StringTok{"FillOutput"}\NormalTok{,}
              \AttributeTok{outlocation =} \StringTok{"./RasterGrids\_100m/2024/RAW/"}\NormalTok{,}
              \AttributeTok{outfilename =} \StringTok{"General\_ShrubsOrchardsGardens\_cell.tif"}\NormalTok{,}
              \AttributeTok{layername =} \StringTok{"egv\_439"}\NormalTok{,}
              \AttributeTok{idw\_weight =} \DecValTok{2}\NormalTok{,}
              \AttributeTok{plot\_gaps =} \ConstantTok{FALSE}\NormalTok{,}\AttributeTok{plot\_final =} \ConstantTok{TRUE}\NormalTok{)}
\NormalTok{i2e\_rez}
\FunctionTok{rm}\NormalTok{(apvienoti)}
\FunctionTok{rm}\NormalTok{(i2e\_rez)}
\FunctionTok{rm}\NormalTok{(ilggadigiekrumveida)}
\FunctionTok{rm}\NormalTok{(krumveida\_r)}
\FunctionTok{rm}\NormalTok{(krumveida\_t)}
\FunctionTok{rm}\NormalTok{(parejie)}

\CommentTok{\# standardisation {-}{-}{-}{-}}
\ControlFlowTok{if}\NormalTok{(}\SpecialCharTok{!}\FunctionTok{require}\NormalTok{(terra)) \{}\FunctionTok{install.packages}\NormalTok{(}\StringTok{"terra"}\NormalTok{); }\FunctionTok{require}\NormalTok{(terra)\}}
\ControlFlowTok{if}\NormalTok{(}\SpecialCharTok{!}\FunctionTok{require}\NormalTok{(tidyverse)) \{}\FunctionTok{install.packages}\NormalTok{(}\StringTok{"tidyverse"}\NormalTok{); }\FunctionTok{require}\NormalTok{(tidyverse)\}}

\NormalTok{nosaukums}\OtherTok{=}\StringTok{"General\_ShrubsOrchardsGardens\_cell.tif"}
\NormalTok{ielasisanas\_cels}\OtherTok{=}\FunctionTok{paste0}\NormalTok{(}\StringTok{"./RasterGrids\_100m/2024/RAW/"}\NormalTok{,nosaukums)}
\NormalTok{saglabasanas\_cels}\OtherTok{=}\FunctionTok{paste0}\NormalTok{(}\StringTok{"./RasterGrids\_100m/2024/Scaled/"}\NormalTok{,nosaukums)}
\NormalTok{slanis}\OtherTok{=}\FunctionTok{rast}\NormalTok{(ielasisanas\_cels)}
\NormalTok{videjais}\OtherTok{=}\FunctionTok{global}\NormalTok{(slanis,}\AttributeTok{fun=}\StringTok{"mean"}\NormalTok{,}\AttributeTok{na.rm=}\ConstantTok{TRUE}\NormalTok{)}
\NormalTok{centrets}\OtherTok{=}\NormalTok{slanis}\SpecialCharTok{{-}}\NormalTok{videjais[,}\DecValTok{1}\NormalTok{]}
\NormalTok{standartnovirze}\OtherTok{=}\NormalTok{terra}\SpecialCharTok{::}\FunctionTok{global}\NormalTok{(centrets,}\AttributeTok{fun=}\StringTok{"rms"}\NormalTok{,}\AttributeTok{na.rm=}\ConstantTok{TRUE}\NormalTok{)}
\NormalTok{merogots}\OtherTok{=}\NormalTok{centrets}\SpecialCharTok{/}\NormalTok{standartnovirze[,}\DecValTok{1}\NormalTok{]}
\FunctionTok{writeRaster}\NormalTok{(merogots,}
      \AttributeTok{filename=}\NormalTok{saglabasanas\_cels,}
      \AttributeTok{overwrite=}\ConstantTok{TRUE}\NormalTok{)}
\end{Highlighting}
\end{Shaded}

\section{General\_ShrubsOrchardsGardens\_r500}\label{ch06.440}

\textbf{filename:} \texttt{General\_ShrubsOrchardsGardens\_r500.tif}

\textbf{layername:} \texttt{egv\_440}

\textbf{English name:} Fractional cover of Shrubs, Young stands, Orchards, Allotment
gardens within the 0.5 km landscape

\textbf{Latvian name:} Krūmāju, jaunaudžu, augļudārzu un vasarnīcu kompleksu platības
īpatsvars 0,5 km ainavā

\textbf{Procedure:} The cover fraction within a radius of 500 m around the analysis grid cell is
calculated as the area-weighted sum of the \hyperref[ch06.439]{analysis cells} inside the
buffer, using the workflow \texttt{egvtools::radius\_function()}. During the calculation of the landscape metric,
inverse distance weighted (power = 2) gap filling on the output is applied
to ensure no missing values at the edges. Then the layer is rewritten to set
its name. Finally, the layer is standardised by subtracting the arithmetic
mean and dividing by the root mean squared error.

\begin{Shaded}
\begin{Highlighting}[]
\CommentTok{\# libs {-}{-}{-}{-}}
\ControlFlowTok{if}\NormalTok{(}\SpecialCharTok{!}\FunctionTok{require}\NormalTok{(terra)) \{}\FunctionTok{install.packages}\NormalTok{(}\StringTok{"terra"}\NormalTok{); }\FunctionTok{require}\NormalTok{(terra)\}}
\ControlFlowTok{if}\NormalTok{(}\SpecialCharTok{!}\FunctionTok{require}\NormalTok{(egvtools)) \{remotes}\SpecialCharTok{::}\FunctionTok{install\_github}\NormalTok{(}\StringTok{"aavotins/egvtools"}\NormalTok{); }\FunctionTok{require}\NormalTok{(egvtools)\}}


\CommentTok{\# Templates {-}{-}{-}{-}{-}}
\NormalTok{template100}\OtherTok{=}\FunctionTok{rast}\NormalTok{(}\StringTok{"./Templates/TemplateRasters/LV100m\_10km.tif"}\NormalTok{)}

\CommentTok{\# radii {-}{-}{-}{-}}
\FunctionTok{radius\_function}\NormalTok{(}
 \AttributeTok{kvadrati\_path =} \StringTok{"./Templates/TemplateGrids/tiles/"}\NormalTok{,}
 \AttributeTok{radii\_path   =} \StringTok{"./Templates/TemplateGridPoints/tiles/"}\NormalTok{,}
 \AttributeTok{tikls100\_path =} \StringTok{"./Templates/TemplateGrids/tikls100\_sauzeme.parquet"}\NormalTok{,}
 \AttributeTok{template\_path =} \StringTok{"./Templates/TemplateRasters/LV100m\_10km.tif"}\NormalTok{,}
 \AttributeTok{input\_layers  =} \FunctionTok{c}\NormalTok{(}\StringTok{"./RasterGrids\_100m/2024/RAW/General\_ShrubsOrchardsGardens\_cell.tif"}\NormalTok{),}
 \AttributeTok{layer\_prefixes =} \FunctionTok{c}\NormalTok{(}\StringTok{"General\_ShrubsOrchardsGardens"}\NormalTok{),}
 \AttributeTok{output\_dir   =} \StringTok{"./RasterGrids\_100m/2024/RAW/"}\NormalTok{,}
 \AttributeTok{n\_workers   =} \DecValTok{6}\NormalTok{,}
 \AttributeTok{radii     =} \FunctionTok{c}\NormalTok{(}\StringTok{"r500"}\NormalTok{),}
 \AttributeTok{radius\_mode  =} \StringTok{"sparse"}\NormalTok{,}
 \AttributeTok{extract\_fun  =} \StringTok{"mean"}\NormalTok{,}
 \AttributeTok{fill\_missing  =} \ConstantTok{TRUE}\NormalTok{,}
 \AttributeTok{IDW\_weight   =} \DecValTok{2}\NormalTok{,}
 \AttributeTok{future\_max\_size =} \DecValTok{40} \SpecialCharTok{*} \DecValTok{1024}\SpecialCharTok{\^{}}\DecValTok{3}\NormalTok{)}


\CommentTok{\# General\_ShrubsOrchardsGardens\_r500.tif    egv\_440}
\NormalTok{slanis}\OtherTok{=}\FunctionTok{rast}\NormalTok{(}\StringTok{"./RasterGrids\_100m/2024/RAW/General\_ShrubsOrchardsGardens\_r500.tif"}\NormalTok{)}
\FunctionTok{names}\NormalTok{(slanis)}\OtherTok{=}\StringTok{"egv\_440"}
\NormalTok{slanis2}\OtherTok{=}\FunctionTok{project}\NormalTok{(slanis,template100)}
\FunctionTok{writeRaster}\NormalTok{(slanis2,}
      \StringTok{"./RasterGrids\_100m/2024/RAW/General\_ShrubsOrchardsGardens\_r500.tif"}\NormalTok{,}
      \AttributeTok{overwrite=}\ConstantTok{TRUE}\NormalTok{)}

\CommentTok{\# standardisation {-}{-}{-}{-}}
\ControlFlowTok{if}\NormalTok{(}\SpecialCharTok{!}\FunctionTok{require}\NormalTok{(terra)) \{}\FunctionTok{install.packages}\NormalTok{(}\StringTok{"terra"}\NormalTok{); }\FunctionTok{require}\NormalTok{(terra)\}}
\ControlFlowTok{if}\NormalTok{(}\SpecialCharTok{!}\FunctionTok{require}\NormalTok{(tidyverse)) \{}\FunctionTok{install.packages}\NormalTok{(}\StringTok{"tidyverse"}\NormalTok{); }\FunctionTok{require}\NormalTok{(tidyverse)\}}

\NormalTok{nosaukums}\OtherTok{=}\StringTok{"General\_ShrubsOrchardsGardens\_r500.tif"}
\NormalTok{ielasisanas\_cels}\OtherTok{=}\FunctionTok{paste0}\NormalTok{(}\StringTok{"./RasterGrids\_100m/2024/RAW/"}\NormalTok{,nosaukums)}
\NormalTok{saglabasanas\_cels}\OtherTok{=}\FunctionTok{paste0}\NormalTok{(}\StringTok{"./RasterGrids\_100m/2024/Scaled/"}\NormalTok{,nosaukums)}
\NormalTok{slanis}\OtherTok{=}\FunctionTok{rast}\NormalTok{(ielasisanas\_cels)}
\NormalTok{videjais}\OtherTok{=}\FunctionTok{global}\NormalTok{(slanis,}\AttributeTok{fun=}\StringTok{"mean"}\NormalTok{,}\AttributeTok{na.rm=}\ConstantTok{TRUE}\NormalTok{)}
\NormalTok{centrets}\OtherTok{=}\NormalTok{slanis}\SpecialCharTok{{-}}\NormalTok{videjais[,}\DecValTok{1}\NormalTok{]}
\NormalTok{standartnovirze}\OtherTok{=}\NormalTok{terra}\SpecialCharTok{::}\FunctionTok{global}\NormalTok{(centrets,}\AttributeTok{fun=}\StringTok{"rms"}\NormalTok{,}\AttributeTok{na.rm=}\ConstantTok{TRUE}\NormalTok{)}
\NormalTok{merogots}\OtherTok{=}\NormalTok{centrets}\SpecialCharTok{/}\NormalTok{standartnovirze[,}\DecValTok{1}\NormalTok{]}
\FunctionTok{writeRaster}\NormalTok{(merogots,}
      \AttributeTok{filename=}\NormalTok{saglabasanas\_cels,}
      \AttributeTok{overwrite=}\ConstantTok{TRUE}\NormalTok{)}
\end{Highlighting}
\end{Shaded}

\section{General\_ShrubsOrchardsGardens\_r1250}\label{ch06.441}

\textbf{filename:} \texttt{General\_ShrubsOrchardsGardens\_r1250.tif}

\textbf{layername:} \texttt{egv\_441}

\textbf{English name:} Fractional cover of Shrubs, Young stands, Orchards, Allotment
gardens within the 1.25 km landscape

\textbf{Latvian name:} Krūmāju, jaunaudžu, augļudārzu un vasarnīcu kompleksu platības
īpatsvars 1,25 km ainavā

\textbf{Procedure:} The cover fraction within a radius of 1250 m around the analysis grid cell
is calculated as the area-weighted sum of the \hyperref[ch06.439]{analysis cells} inside
the buffer, using the workflow \texttt{egvtools::radius\_function()}. During the calculation of the landscape
metric, inverse distance weighted (power = 2) gap filling on the output is
applied to ensure no missing values at the edges. Then the layer is
rewritten to set its name. Finally, the layer is standardised by
subtracting the arithmetic mean and dividing by the root mean squared error.

\begin{Shaded}
\begin{Highlighting}[]
\CommentTok{\# libs {-}{-}{-}{-}}
\ControlFlowTok{if}\NormalTok{(}\SpecialCharTok{!}\FunctionTok{require}\NormalTok{(terra)) \{}\FunctionTok{install.packages}\NormalTok{(}\StringTok{"terra"}\NormalTok{); }\FunctionTok{require}\NormalTok{(terra)\}}
\ControlFlowTok{if}\NormalTok{(}\SpecialCharTok{!}\FunctionTok{require}\NormalTok{(egvtools)) \{remotes}\SpecialCharTok{::}\FunctionTok{install\_github}\NormalTok{(}\StringTok{"aavotins/egvtools"}\NormalTok{); }\FunctionTok{require}\NormalTok{(egvtools)\}}


\CommentTok{\# Templates {-}{-}{-}{-}{-}}
\NormalTok{template100}\OtherTok{=}\FunctionTok{rast}\NormalTok{(}\StringTok{"./Templates/TemplateRasters/LV100m\_10km.tif"}\NormalTok{)}

\CommentTok{\# radii {-}{-}{-}{-}}
\FunctionTok{radius\_function}\NormalTok{(}
 \AttributeTok{kvadrati\_path =} \StringTok{"./Templates/TemplateGrids/tiles/"}\NormalTok{,}
 \AttributeTok{radii\_path   =} \StringTok{"./Templates/TemplateGridPoints/tiles/"}\NormalTok{,}
 \AttributeTok{tikls100\_path =} \StringTok{"./Templates/TemplateGrids/tikls100\_sauzeme.parquet"}\NormalTok{,}
 \AttributeTok{template\_path =} \StringTok{"./Templates/TemplateRasters/LV100m\_10km.tif"}\NormalTok{,}
 \AttributeTok{input\_layers  =} \FunctionTok{c}\NormalTok{(}\StringTok{"./RasterGrids\_100m/2024/RAW/General\_ShrubsOrchardsGardens\_cell.tif"}\NormalTok{),}
 \AttributeTok{layer\_prefixes =} \FunctionTok{c}\NormalTok{(}\StringTok{"General\_ShrubsOrchardsGardens"}\NormalTok{),}
 \AttributeTok{output\_dir   =} \StringTok{"./RasterGrids\_100m/2024/RAW/"}\NormalTok{,}
 \AttributeTok{n\_workers   =} \DecValTok{6}\NormalTok{,}
 \AttributeTok{radii     =} \FunctionTok{c}\NormalTok{(}\StringTok{"r1250"}\NormalTok{),}
 \AttributeTok{radius\_mode  =} \StringTok{"sparse"}\NormalTok{,}
 \AttributeTok{extract\_fun  =} \StringTok{"mean"}\NormalTok{,}
 \AttributeTok{fill\_missing  =} \ConstantTok{TRUE}\NormalTok{,}
 \AttributeTok{IDW\_weight   =} \DecValTok{2}\NormalTok{,}
 \AttributeTok{future\_max\_size =} \DecValTok{40} \SpecialCharTok{*} \DecValTok{1024}\SpecialCharTok{\^{}}\DecValTok{3}\NormalTok{)}


\CommentTok{\# General\_ShrubsOrchardsGardens\_r1250.tif   egv\_441}
\NormalTok{slanis}\OtherTok{=}\FunctionTok{rast}\NormalTok{(}\StringTok{"./RasterGrids\_100m/2024/RAW/General\_ShrubsOrchardsGardens\_r1250.tif"}\NormalTok{)}
\FunctionTok{names}\NormalTok{(slanis)}\OtherTok{=}\StringTok{"egv\_441"}
\NormalTok{slanis2}\OtherTok{=}\FunctionTok{project}\NormalTok{(slanis,template100)}
\FunctionTok{writeRaster}\NormalTok{(slanis2,}
      \StringTok{"./RasterGrids\_100m/2024/RAW/General\_ShrubsOrchardsGardens\_r1250.tif"}\NormalTok{,}
      \AttributeTok{overwrite=}\ConstantTok{TRUE}\NormalTok{)}

\CommentTok{\# standardisation {-}{-}{-}{-}}
\ControlFlowTok{if}\NormalTok{(}\SpecialCharTok{!}\FunctionTok{require}\NormalTok{(terra)) \{}\FunctionTok{install.packages}\NormalTok{(}\StringTok{"terra"}\NormalTok{); }\FunctionTok{require}\NormalTok{(terra)\}}
\ControlFlowTok{if}\NormalTok{(}\SpecialCharTok{!}\FunctionTok{require}\NormalTok{(tidyverse)) \{}\FunctionTok{install.packages}\NormalTok{(}\StringTok{"tidyverse"}\NormalTok{); }\FunctionTok{require}\NormalTok{(tidyverse)\}}

\NormalTok{nosaukums}\OtherTok{=}\StringTok{"General\_ShrubsOrchardsGardens\_r1250.tif"}
\NormalTok{ielasisanas\_cels}\OtherTok{=}\FunctionTok{paste0}\NormalTok{(}\StringTok{"./RasterGrids\_100m/2024/RAW/"}\NormalTok{,nosaukums)}
\NormalTok{saglabasanas\_cels}\OtherTok{=}\FunctionTok{paste0}\NormalTok{(}\StringTok{"./RasterGrids\_100m/2024/Scaled/"}\NormalTok{,nosaukums)}
\NormalTok{slanis}\OtherTok{=}\FunctionTok{rast}\NormalTok{(ielasisanas\_cels)}
\NormalTok{videjais}\OtherTok{=}\FunctionTok{global}\NormalTok{(slanis,}\AttributeTok{fun=}\StringTok{"mean"}\NormalTok{,}\AttributeTok{na.rm=}\ConstantTok{TRUE}\NormalTok{)}
\NormalTok{centrets}\OtherTok{=}\NormalTok{slanis}\SpecialCharTok{{-}}\NormalTok{videjais[,}\DecValTok{1}\NormalTok{]}
\NormalTok{standartnovirze}\OtherTok{=}\NormalTok{terra}\SpecialCharTok{::}\FunctionTok{global}\NormalTok{(centrets,}\AttributeTok{fun=}\StringTok{"rms"}\NormalTok{,}\AttributeTok{na.rm=}\ConstantTok{TRUE}\NormalTok{)}
\NormalTok{merogots}\OtherTok{=}\NormalTok{centrets}\SpecialCharTok{/}\NormalTok{standartnovirze[,}\DecValTok{1}\NormalTok{]}
\FunctionTok{writeRaster}\NormalTok{(merogots,}
      \AttributeTok{filename=}\NormalTok{saglabasanas\_cels,}
      \AttributeTok{overwrite=}\ConstantTok{TRUE}\NormalTok{)}
\end{Highlighting}
\end{Shaded}

\section{General\_ShrubsOrchardsGardens\_r3000}\label{ch06.442}

\textbf{filename:} \texttt{General\_ShrubsOrchardsGardens\_r3000.tif}

\textbf{layername:} \texttt{egv\_442}

\textbf{English name:} Fractional cover of Shrubs, Young stands, Orchards, Allotment
gardens within the 3 km landscape

\textbf{Latvian name:} Krūmāju, jaunaudžu, augļudārzu un vasarnīcu kompleksu platības
īpatsvars 3 km ainavā

\textbf{Procedure:} The cover fraction within a radius of 3000 m around the analysis grid cell
is calculated as the area-weighted sum of the \hyperref[ch06.439]{analysis cells} inside
the buffer, using the workflow \texttt{egvtools::radius\_function()}. During the calculation of the landscape
metric, inverse distance weighted (power = 2) gap filling on the output is
applied to ensure no missing values at the edges. Then the layer is
rewritten to set its name. Finally, the layer is standardised by
subtracting the arithmetic mean and dividing by the root mean squared error.

\begin{Shaded}
\begin{Highlighting}[]
\CommentTok{\# libs {-}{-}{-}{-}}
\ControlFlowTok{if}\NormalTok{(}\SpecialCharTok{!}\FunctionTok{require}\NormalTok{(terra)) \{}\FunctionTok{install.packages}\NormalTok{(}\StringTok{"terra"}\NormalTok{); }\FunctionTok{require}\NormalTok{(terra)\}}
\ControlFlowTok{if}\NormalTok{(}\SpecialCharTok{!}\FunctionTok{require}\NormalTok{(egvtools)) \{remotes}\SpecialCharTok{::}\FunctionTok{install\_github}\NormalTok{(}\StringTok{"aavotins/egvtools"}\NormalTok{); }\FunctionTok{require}\NormalTok{(egvtools)\}}


\CommentTok{\# Templates {-}{-}{-}{-}{-}}
\NormalTok{template100}\OtherTok{=}\FunctionTok{rast}\NormalTok{(}\StringTok{"./Templates/TemplateRasters/LV100m\_10km.tif"}\NormalTok{)}

\CommentTok{\# radii {-}{-}{-}{-}}
\FunctionTok{radius\_function}\NormalTok{(}
 \AttributeTok{kvadrati\_path =} \StringTok{"./Templates/TemplateGrids/tiles/"}\NormalTok{,}
 \AttributeTok{radii\_path   =} \StringTok{"./Templates/TemplateGridPoints/tiles/"}\NormalTok{,}
 \AttributeTok{tikls100\_path =} \StringTok{"./Templates/TemplateGrids/tikls100\_sauzeme.parquet"}\NormalTok{,}
 \AttributeTok{template\_path =} \StringTok{"./Templates/TemplateRasters/LV100m\_10km.tif"}\NormalTok{,}
 \AttributeTok{input\_layers  =} \FunctionTok{c}\NormalTok{(}\StringTok{"./RasterGrids\_100m/2024/RAW/General\_ShrubsOrchardsGardens\_cell.tif"}\NormalTok{),}
 \AttributeTok{layer\_prefixes =} \FunctionTok{c}\NormalTok{(}\StringTok{"General\_ShrubsOrchardsGardens"}\NormalTok{),}
 \AttributeTok{output\_dir   =} \StringTok{"./RasterGrids\_100m/2024/RAW/"}\NormalTok{,}
 \AttributeTok{n\_workers   =} \DecValTok{6}\NormalTok{,}
 \AttributeTok{radii     =} \FunctionTok{c}\NormalTok{(}\StringTok{"r3000"}\NormalTok{),}
 \AttributeTok{radius\_mode  =} \StringTok{"sparse"}\NormalTok{,}
 \AttributeTok{extract\_fun  =} \StringTok{"mean"}\NormalTok{,}
 \AttributeTok{fill\_missing  =} \ConstantTok{TRUE}\NormalTok{,}
 \AttributeTok{IDW\_weight   =} \DecValTok{2}\NormalTok{,}
 \AttributeTok{future\_max\_size =} \DecValTok{40} \SpecialCharTok{*} \DecValTok{1024}\SpecialCharTok{\^{}}\DecValTok{3}\NormalTok{)}


\CommentTok{\# General\_ShrubsOrchardsGardens\_r3000.tif   egv\_442}
\NormalTok{slanis}\OtherTok{=}\FunctionTok{rast}\NormalTok{(}\StringTok{"./RasterGrids\_100m/2024/RAW/General\_ShrubsOrchardsGardens\_r3000.tif"}\NormalTok{)}
\FunctionTok{names}\NormalTok{(slanis)}\OtherTok{=}\StringTok{"egv\_442"}
\NormalTok{slanis2}\OtherTok{=}\FunctionTok{project}\NormalTok{(slanis,template100)}
\FunctionTok{writeRaster}\NormalTok{(slanis2,}
      \StringTok{"./RasterGrids\_100m/2024/RAW/General\_ShrubsOrchardsGardens\_r3000.tif"}\NormalTok{,}
      \AttributeTok{overwrite=}\ConstantTok{TRUE}\NormalTok{)}

\CommentTok{\# standardisation {-}{-}{-}{-}}
\ControlFlowTok{if}\NormalTok{(}\SpecialCharTok{!}\FunctionTok{require}\NormalTok{(terra)) \{}\FunctionTok{install.packages}\NormalTok{(}\StringTok{"terra"}\NormalTok{); }\FunctionTok{require}\NormalTok{(terra)\}}
\ControlFlowTok{if}\NormalTok{(}\SpecialCharTok{!}\FunctionTok{require}\NormalTok{(tidyverse)) \{}\FunctionTok{install.packages}\NormalTok{(}\StringTok{"tidyverse"}\NormalTok{); }\FunctionTok{require}\NormalTok{(tidyverse)\}}

\NormalTok{nosaukums}\OtherTok{=}\StringTok{"General\_ShrubsOrchardsGardens\_r3000.tif"}
\NormalTok{ielasisanas\_cels}\OtherTok{=}\FunctionTok{paste0}\NormalTok{(}\StringTok{"./RasterGrids\_100m/2024/RAW/"}\NormalTok{,nosaukums)}
\NormalTok{saglabasanas\_cels}\OtherTok{=}\FunctionTok{paste0}\NormalTok{(}\StringTok{"./RasterGrids\_100m/2024/Scaled/"}\NormalTok{,nosaukums)}
\NormalTok{slanis}\OtherTok{=}\FunctionTok{rast}\NormalTok{(ielasisanas\_cels)}
\NormalTok{videjais}\OtherTok{=}\FunctionTok{global}\NormalTok{(slanis,}\AttributeTok{fun=}\StringTok{"mean"}\NormalTok{,}\AttributeTok{na.rm=}\ConstantTok{TRUE}\NormalTok{)}
\NormalTok{centrets}\OtherTok{=}\NormalTok{slanis}\SpecialCharTok{{-}}\NormalTok{videjais[,}\DecValTok{1}\NormalTok{]}
\NormalTok{standartnovirze}\OtherTok{=}\NormalTok{terra}\SpecialCharTok{::}\FunctionTok{global}\NormalTok{(centrets,}\AttributeTok{fun=}\StringTok{"rms"}\NormalTok{,}\AttributeTok{na.rm=}\ConstantTok{TRUE}\NormalTok{)}
\NormalTok{merogots}\OtherTok{=}\NormalTok{centrets}\SpecialCharTok{/}\NormalTok{standartnovirze[,}\DecValTok{1}\NormalTok{]}
\FunctionTok{writeRaster}\NormalTok{(merogots,}
      \AttributeTok{filename=}\NormalTok{saglabasanas\_cels,}
      \AttributeTok{overwrite=}\ConstantTok{TRUE}\NormalTok{)}
\end{Highlighting}
\end{Shaded}

\section{General\_ShrubsOrchardsGardens\_r10000}\label{ch06.443}

\textbf{filename:} \texttt{General\_ShrubsOrchardsGardens\_r10000.tif}

\textbf{layername:} \texttt{egv\_443}

\textbf{English name:} Fractional cover of Shrubs, Young stands, Orchards, Allotment
gardens within the 10 km landscape

\textbf{Latvian name:} Krūmāju, jaunaudžu, augļudārzu un vasarnīcu kompleksu platības
īpatsvars 10 km ainavā

\textbf{Procedure:} The cover fraction within a radius of 10000 m around the analysis grid cell
is calculated as the area-weighted sum of the \hyperref[ch06.439]{analysis cells} inside
the buffer, using the workflow \texttt{egvtools::radius\_function()}. During the calculation of the landscape
metric, inverse distance weighted (power = 2) gap filling on the output is
applied to ensure no missing values at the edges. Then the layer is
rewritten to set its name. Finally, the layer is standardised by
subtracting the arithmetic mean and dividing by the root mean squared error.

\begin{Shaded}
\begin{Highlighting}[]
\CommentTok{\# libs {-}{-}{-}{-}}
\ControlFlowTok{if}\NormalTok{(}\SpecialCharTok{!}\FunctionTok{require}\NormalTok{(terra)) \{}\FunctionTok{install.packages}\NormalTok{(}\StringTok{"terra"}\NormalTok{); }\FunctionTok{require}\NormalTok{(terra)\}}
\ControlFlowTok{if}\NormalTok{(}\SpecialCharTok{!}\FunctionTok{require}\NormalTok{(egvtools)) \{remotes}\SpecialCharTok{::}\FunctionTok{install\_github}\NormalTok{(}\StringTok{"aavotins/egvtools"}\NormalTok{); }\FunctionTok{require}\NormalTok{(egvtools)\}}


\CommentTok{\# Templates {-}{-}{-}{-}{-}}
\NormalTok{template100}\OtherTok{=}\FunctionTok{rast}\NormalTok{(}\StringTok{"./Templates/TemplateRasters/LV100m\_10km.tif"}\NormalTok{)}

\CommentTok{\# radii {-}{-}{-}{-}}
\FunctionTok{radius\_function}\NormalTok{(}
 \AttributeTok{kvadrati\_path =} \StringTok{"./Templates/TemplateGrids/tiles/"}\NormalTok{,}
 \AttributeTok{radii\_path   =} \StringTok{"./Templates/TemplateGridPoints/tiles/"}\NormalTok{,}
 \AttributeTok{tikls100\_path =} \StringTok{"./Templates/TemplateGrids/tikls100\_sauzeme.parquet"}\NormalTok{,}
 \AttributeTok{template\_path =} \StringTok{"./Templates/TemplateRasters/LV100m\_10km.tif"}\NormalTok{,}
 \AttributeTok{input\_layers  =} \FunctionTok{c}\NormalTok{(}\StringTok{"./RasterGrids\_100m/2024/RAW/General\_ShrubsOrchardsGardens\_cell.tif"}\NormalTok{),}
 \AttributeTok{layer\_prefixes =} \FunctionTok{c}\NormalTok{(}\StringTok{"General\_ShrubsOrchardsGardens"}\NormalTok{),}
 \AttributeTok{output\_dir   =} \StringTok{"./RasterGrids\_100m/2024/RAW/"}\NormalTok{,}
 \AttributeTok{n\_workers   =} \DecValTok{6}\NormalTok{,}
 \AttributeTok{radii     =} \FunctionTok{c}\NormalTok{(}\StringTok{"r10000"}\NormalTok{),}
 \AttributeTok{radius\_mode  =} \StringTok{"sparse"}\NormalTok{,}
 \AttributeTok{extract\_fun  =} \StringTok{"mean"}\NormalTok{,}
 \AttributeTok{fill\_missing  =} \ConstantTok{TRUE}\NormalTok{,}
 \AttributeTok{IDW\_weight   =} \DecValTok{2}\NormalTok{,}
 \AttributeTok{future\_max\_size =} \DecValTok{40} \SpecialCharTok{*} \DecValTok{1024}\SpecialCharTok{\^{}}\DecValTok{3}\NormalTok{)}


\CommentTok{\# General\_ShrubsOrchardsGardens\_r10000.tif  egv\_443}
\NormalTok{slanis}\OtherTok{=}\FunctionTok{rast}\NormalTok{(}\StringTok{"./RasterGrids\_100m/2024/RAW/General\_ShrubsOrchardsGardens\_r10000.tif"}\NormalTok{)}
\FunctionTok{names}\NormalTok{(slanis)}\OtherTok{=}\StringTok{"egv\_443"}
\NormalTok{slanis2}\OtherTok{=}\FunctionTok{project}\NormalTok{(slanis,template100)}
\FunctionTok{writeRaster}\NormalTok{(slanis2,}
      \StringTok{"./RasterGrids\_100m/2024/RAW/General\_ShrubsOrchardsGardens\_r10000.tif"}\NormalTok{,}
      \AttributeTok{overwrite=}\ConstantTok{TRUE}\NormalTok{)}

\CommentTok{\# standardisation {-}{-}{-}{-}}
\ControlFlowTok{if}\NormalTok{(}\SpecialCharTok{!}\FunctionTok{require}\NormalTok{(terra)) \{}\FunctionTok{install.packages}\NormalTok{(}\StringTok{"terra"}\NormalTok{); }\FunctionTok{require}\NormalTok{(terra)\}}
\ControlFlowTok{if}\NormalTok{(}\SpecialCharTok{!}\FunctionTok{require}\NormalTok{(tidyverse)) \{}\FunctionTok{install.packages}\NormalTok{(}\StringTok{"tidyverse"}\NormalTok{); }\FunctionTok{require}\NormalTok{(tidyverse)\}}

\NormalTok{nosaukums}\OtherTok{=}\StringTok{"General\_ShrubsOrchardsGardens\_r10000.tif"}
\NormalTok{ielasisanas\_cels}\OtherTok{=}\FunctionTok{paste0}\NormalTok{(}\StringTok{"./RasterGrids\_100m/2024/RAW/"}\NormalTok{,nosaukums)}
\NormalTok{saglabasanas\_cels}\OtherTok{=}\FunctionTok{paste0}\NormalTok{(}\StringTok{"./RasterGrids\_100m/2024/Scaled/"}\NormalTok{,nosaukums)}
\NormalTok{slanis}\OtherTok{=}\FunctionTok{rast}\NormalTok{(ielasisanas\_cels)}
\NormalTok{videjais}\OtherTok{=}\FunctionTok{global}\NormalTok{(slanis,}\AttributeTok{fun=}\StringTok{"mean"}\NormalTok{,}\AttributeTok{na.rm=}\ConstantTok{TRUE}\NormalTok{)}
\NormalTok{centrets}\OtherTok{=}\NormalTok{slanis}\SpecialCharTok{{-}}\NormalTok{videjais[,}\DecValTok{1}\NormalTok{]}
\NormalTok{standartnovirze}\OtherTok{=}\NormalTok{terra}\SpecialCharTok{::}\FunctionTok{global}\NormalTok{(centrets,}\AttributeTok{fun=}\StringTok{"rms"}\NormalTok{,}\AttributeTok{na.rm=}\ConstantTok{TRUE}\NormalTok{)}
\NormalTok{merogots}\OtherTok{=}\NormalTok{centrets}\SpecialCharTok{/}\NormalTok{standartnovirze[,}\DecValTok{1}\NormalTok{]}
\FunctionTok{writeRaster}\NormalTok{(merogots,}
      \AttributeTok{filename=}\NormalTok{saglabasanas\_cels,}
      \AttributeTok{overwrite=}\ConstantTok{TRUE}\NormalTok{)}
\end{Highlighting}
\end{Shaded}

\section{General\_SwampsMiresBogsHelophytes\_cell}\label{ch06.444}

\textbf{filename:} \texttt{General\_SwampsMiresBogsHelophytes\_cell.tif}

\textbf{layername:} \texttt{egv\_444}

\textbf{English name:} Fractional cover of Swamps, Mires, Bogs, Reed-, Sedge-, Rush-
Beds within the analysis cell (1 ha)

\textbf{Latvian name:} Purvu, niedrāju, grīslāju, meldrāju platības īpatsvars
analīzes šūnā (1 ha)

\textbf{Procedure:} First, the swamps, mires, bogs and reed, sedge, rush beds from the
\hyperref[Ch05.03]{Landscape classification} are selected (values between 700 and 800 are
reclassified to value 1; all others are set to 0). The resulting layer
is then aggregated to EGV resolution using the workflow \texttt{egvtools::input2egv()}, which
calculates the arithmetic mean to determine the cover fraction. During
aggregation, inverse distance weighted (power = 2) gap filling on the output is
applied to ensure no missing values at the edges. Finally, the layer is
standardised by subtracting the arithmetic mean and dividing by the root mean squared
error.

\begin{Shaded}
\begin{Highlighting}[]
\CommentTok{\# libs {-}{-}{-}{-}}
\ControlFlowTok{if}\NormalTok{(}\SpecialCharTok{!}\FunctionTok{require}\NormalTok{(egvtools)) \{remotes}\SpecialCharTok{::}\FunctionTok{install\_github}\NormalTok{(}\StringTok{"aavotins/egvtools"}\NormalTok{); }\FunctionTok{require}\NormalTok{(egvtools)\}}
\ControlFlowTok{if}\NormalTok{(}\SpecialCharTok{!}\FunctionTok{require}\NormalTok{(terra)) \{}\FunctionTok{install.packages}\NormalTok{(}\StringTok{"terra"}\NormalTok{); }\FunctionTok{require}\NormalTok{(terra)\}}
\ControlFlowTok{if}\NormalTok{(}\SpecialCharTok{!}\FunctionTok{require}\NormalTok{(sf)) \{}\FunctionTok{install.packages}\NormalTok{(}\StringTok{"sf"}\NormalTok{); }\FunctionTok{require}\NormalTok{(sf)\}}
\ControlFlowTok{if}\NormalTok{(}\SpecialCharTok{!}\FunctionTok{require}\NormalTok{(tidyverse)) \{}\FunctionTok{install.packages}\NormalTok{(}\StringTok{"tidyverse"}\NormalTok{); }\FunctionTok{require}\NormalTok{(tidyverse)\}}
\ControlFlowTok{if}\NormalTok{(}\SpecialCharTok{!}\FunctionTok{require}\NormalTok{(sfarrow)) \{}\FunctionTok{install.packages}\NormalTok{(}\StringTok{"sfarrow"}\NormalTok{); }\FunctionTok{require}\NormalTok{(sfarrow)\}}
\ControlFlowTok{if}\NormalTok{(}\SpecialCharTok{!}\FunctionTok{require}\NormalTok{(readxl)) \{}\FunctionTok{install.packages}\NormalTok{(}\StringTok{"readxl"}\NormalTok{); }\FunctionTok{require}\NormalTok{(readxl)\}}
\ControlFlowTok{if}\NormalTok{(}\SpecialCharTok{!}\FunctionTok{require}\NormalTok{(raster)) \{}\FunctionTok{install.packages}\NormalTok{(}\StringTok{"raster"}\NormalTok{); }\FunctionTok{require}\NormalTok{(raster)\}}
\ControlFlowTok{if}\NormalTok{(}\SpecialCharTok{!}\FunctionTok{require}\NormalTok{(fasterize)) \{}\FunctionTok{install.packages}\NormalTok{(}\StringTok{"fasterize"}\NormalTok{); }\FunctionTok{require}\NormalTok{(fasterize)\}}

\CommentTok{\# templates {-}{-}{-}{-}}
\NormalTok{template100}\OtherTok{=}\FunctionTok{rast}\NormalTok{(}\StringTok{"./Templates/TemplateRasters/LV100m\_10km.tif"}\NormalTok{)}
\NormalTok{template10}\OtherTok{=}\FunctionTok{rast}\NormalTok{(}\StringTok{"./Templates/TemplateRasters/LV10m\_10km.tif"}\NormalTok{)}
\NormalTok{rastrs10}\OtherTok{=}\FunctionTok{raster}\NormalTok{(template10)}

\NormalTok{nulls10}\OtherTok{=}\FunctionTok{rast}\NormalTok{(}\StringTok{"./Templates/TemplateRasters/nulls\_LV10m\_10km.tif"}\NormalTok{)}
\NormalTok{nulls100}\OtherTok{=}\FunctionTok{rast}\NormalTok{(}\StringTok{"./Templates/TemplateRasters/nulls\_LV100m\_10km.tif"}\NormalTok{)}

\CommentTok{\# simple landscape {-}{-}{-}{-}}
\NormalTok{simple\_landscape}\OtherTok{=}\FunctionTok{rast}\NormalTok{(}\StringTok{"RasterGrids\_10m/2024/Ainava\_vienk\_mask.tif"}\NormalTok{)}


\CommentTok{\# General\_SwampsMiresBogsHelophytes\_cell.tif    egv\_444 {-}{-}{-}{-}}
\NormalTok{purvi}\OtherTok{=}\FunctionTok{ifel}\NormalTok{(simple\_landscape}\SpecialCharTok{\textgreater{}=}\DecValTok{700}\SpecialCharTok{\&}\NormalTok{simple\_landscape}\SpecialCharTok{\textless{}}\DecValTok{800}\NormalTok{,}\DecValTok{1}\NormalTok{,}\DecValTok{0}\NormalTok{)}
\NormalTok{i2e\_rez}\OtherTok{=}\NormalTok{egvtools}\SpecialCharTok{::}\FunctionTok{input2egv}\NormalTok{(}\AttributeTok{input=}\NormalTok{purvi,}
              \AttributeTok{egv\_template=} \StringTok{"./Templates/TemplateRasters/LV100m\_10km.tif"}\NormalTok{,}
              \AttributeTok{summary\_function =} \StringTok{"average"}\NormalTok{,}
              \AttributeTok{missing\_job =} \StringTok{"FillOutput"}\NormalTok{,}
              \AttributeTok{outlocation =} \StringTok{"./RasterGrids\_100m/2024/RAW/"}\NormalTok{,}
              \AttributeTok{outfilename =} \StringTok{"General\_SwampsMiresBogsHelophytes\_cell.tif"}\NormalTok{,}
              \AttributeTok{layername =} \StringTok{"egv\_444"}\NormalTok{,}
              \AttributeTok{idw\_weight =} \DecValTok{2}\NormalTok{,}
              \AttributeTok{plot\_gaps =} \ConstantTok{FALSE}\NormalTok{,}\AttributeTok{plot\_final =} \ConstantTok{TRUE}\NormalTok{)}
\NormalTok{i2e\_rez}
\FunctionTok{rm}\NormalTok{(purvi)}
\FunctionTok{rm}\NormalTok{(i2e\_rez)}

\CommentTok{\# standardisation {-}{-}{-}{-}}
\ControlFlowTok{if}\NormalTok{(}\SpecialCharTok{!}\FunctionTok{require}\NormalTok{(terra)) \{}\FunctionTok{install.packages}\NormalTok{(}\StringTok{"terra"}\NormalTok{); }\FunctionTok{require}\NormalTok{(terra)\}}
\ControlFlowTok{if}\NormalTok{(}\SpecialCharTok{!}\FunctionTok{require}\NormalTok{(tidyverse)) \{}\FunctionTok{install.packages}\NormalTok{(}\StringTok{"tidyverse"}\NormalTok{); }\FunctionTok{require}\NormalTok{(tidyverse)\}}

\NormalTok{nosaukums}\OtherTok{=}\StringTok{"General\_SwampsMiresBogsHelophytes\_cell.tif"}
\NormalTok{ielasisanas\_cels}\OtherTok{=}\FunctionTok{paste0}\NormalTok{(}\StringTok{"./RasterGrids\_100m/2024/RAW/"}\NormalTok{,nosaukums)}
\NormalTok{saglabasanas\_cels}\OtherTok{=}\FunctionTok{paste0}\NormalTok{(}\StringTok{"./RasterGrids\_100m/2024/Scaled/"}\NormalTok{,nosaukums)}
\NormalTok{slanis}\OtherTok{=}\FunctionTok{rast}\NormalTok{(ielasisanas\_cels)}
\NormalTok{videjais}\OtherTok{=}\FunctionTok{global}\NormalTok{(slanis,}\AttributeTok{fun=}\StringTok{"mean"}\NormalTok{,}\AttributeTok{na.rm=}\ConstantTok{TRUE}\NormalTok{)}
\NormalTok{centrets}\OtherTok{=}\NormalTok{slanis}\SpecialCharTok{{-}}\NormalTok{videjais[,}\DecValTok{1}\NormalTok{]}
\NormalTok{standartnovirze}\OtherTok{=}\NormalTok{terra}\SpecialCharTok{::}\FunctionTok{global}\NormalTok{(centrets,}\AttributeTok{fun=}\StringTok{"rms"}\NormalTok{,}\AttributeTok{na.rm=}\ConstantTok{TRUE}\NormalTok{)}
\NormalTok{merogots}\OtherTok{=}\NormalTok{centrets}\SpecialCharTok{/}\NormalTok{standartnovirze[,}\DecValTok{1}\NormalTok{]}
\FunctionTok{writeRaster}\NormalTok{(merogots,}
      \AttributeTok{filename=}\NormalTok{saglabasanas\_cels,}
      \AttributeTok{overwrite=}\ConstantTok{TRUE}\NormalTok{)}
\end{Highlighting}
\end{Shaded}

\section{General\_SwampsMiresBogsHelophytes\_r500}\label{ch06.445}

\textbf{filename:} \texttt{General\_SwampsMiresBogsHelophytes\_r500.tif}

\textbf{layername:} \texttt{egv\_445}

\textbf{English name:} Fractional cover of Swamps, Mires, Bogs, Reed-, Sedge-, Rush-
Beds within the 0.5 km landscape

\textbf{Latvian name:} Purvu, niedrāju, grīslāju, meldrāju platības īpatsvars 0,5 km
ainavā

\textbf{Procedure:} The cover fraction within a radius of 500 m around the analysis grid cell is
calculated as the area-weighted sum of the \hyperref[ch06.444]{analysis cells} inside the
buffer, using the workflow \texttt{egvtools::radius\_function()}. During the calculation of the landscape metric,
inverse distance weighted (power = 2) gap filling on the output is applied
to ensure no missing values at the edges. Then the layer is rewritten to set
its name. Finally, the layer is standardised by subtracting the arithmetic
mean and dividing by the root mean squared error.

\begin{Shaded}
\begin{Highlighting}[]
\CommentTok{\# libs {-}{-}{-}{-}}
\ControlFlowTok{if}\NormalTok{(}\SpecialCharTok{!}\FunctionTok{require}\NormalTok{(terra)) \{}\FunctionTok{install.packages}\NormalTok{(}\StringTok{"terra"}\NormalTok{); }\FunctionTok{require}\NormalTok{(terra)\}}
\ControlFlowTok{if}\NormalTok{(}\SpecialCharTok{!}\FunctionTok{require}\NormalTok{(egvtools)) \{remotes}\SpecialCharTok{::}\FunctionTok{install\_github}\NormalTok{(}\StringTok{"aavotins/egvtools"}\NormalTok{); }\FunctionTok{require}\NormalTok{(egvtools)\}}


\CommentTok{\# Templates {-}{-}{-}{-}{-}}
\NormalTok{template100}\OtherTok{=}\FunctionTok{rast}\NormalTok{(}\StringTok{"./Templates/TemplateRasters/LV100m\_10km.tif"}\NormalTok{)}

\CommentTok{\# radii {-}{-}{-}{-}}
\FunctionTok{radius\_function}\NormalTok{(}
 \AttributeTok{kvadrati\_path =} \StringTok{"./Templates/TemplateGrids/tiles/"}\NormalTok{,}
 \AttributeTok{radii\_path   =} \StringTok{"./Templates/TemplateGridPoints/tiles/"}\NormalTok{,}
 \AttributeTok{tikls100\_path =} \StringTok{"./Templates/TemplateGrids/tikls100\_sauzeme.parquet"}\NormalTok{,}
 \AttributeTok{template\_path =} \StringTok{"./Templates/TemplateRasters/LV100m\_10km.tif"}\NormalTok{,}
 \AttributeTok{input\_layers  =} \FunctionTok{c}\NormalTok{(}\StringTok{"./RasterGrids\_100m/2024/RAW/General\_SwampsMiresBogsHelophytes\_cell.tif"}\NormalTok{),}
 \AttributeTok{layer\_prefixes =} \FunctionTok{c}\NormalTok{(}\StringTok{"General\_SwampsMiresBogsHelophytes"}\NormalTok{),}
 \AttributeTok{output\_dir   =} \StringTok{"./RasterGrids\_100m/2024/RAW/"}\NormalTok{,}
 \AttributeTok{n\_workers   =} \DecValTok{6}\NormalTok{,}
 \AttributeTok{radii     =} \FunctionTok{c}\NormalTok{(}\StringTok{"r500"}\NormalTok{),}
 \AttributeTok{radius\_mode  =} \StringTok{"sparse"}\NormalTok{,}
 \AttributeTok{extract\_fun  =} \StringTok{"mean"}\NormalTok{,}
 \AttributeTok{fill\_missing  =} \ConstantTok{TRUE}\NormalTok{,}
 \AttributeTok{IDW\_weight   =} \DecValTok{2}\NormalTok{,}
 \AttributeTok{future\_max\_size =} \DecValTok{40} \SpecialCharTok{*} \DecValTok{1024}\SpecialCharTok{\^{}}\DecValTok{3}\NormalTok{)}


\CommentTok{\# General\_SwampsMiresBogsHelophytes\_r500.tif    egv\_445}
\NormalTok{slanis}\OtherTok{=}\FunctionTok{rast}\NormalTok{(}\StringTok{"./RasterGrids\_100m/2024/RAW/General\_SwampsMiresBogsHelophytes\_r500.tif"}\NormalTok{)}
\FunctionTok{names}\NormalTok{(slanis)}\OtherTok{=}\StringTok{"egv\_445"}
\NormalTok{slanis2}\OtherTok{=}\FunctionTok{project}\NormalTok{(slanis,template100)}
\FunctionTok{writeRaster}\NormalTok{(slanis2,}
      \StringTok{"./RasterGrids\_100m/2024/RAW/General\_SwampsMiresBogsHelophytes\_r500.tif"}\NormalTok{,}
      \AttributeTok{overwrite=}\ConstantTok{TRUE}\NormalTok{)}

\CommentTok{\# standardisation {-}{-}{-}{-}}
\ControlFlowTok{if}\NormalTok{(}\SpecialCharTok{!}\FunctionTok{require}\NormalTok{(terra)) \{}\FunctionTok{install.packages}\NormalTok{(}\StringTok{"terra"}\NormalTok{); }\FunctionTok{require}\NormalTok{(terra)\}}
\ControlFlowTok{if}\NormalTok{(}\SpecialCharTok{!}\FunctionTok{require}\NormalTok{(tidyverse)) \{}\FunctionTok{install.packages}\NormalTok{(}\StringTok{"tidyverse"}\NormalTok{); }\FunctionTok{require}\NormalTok{(tidyverse)\}}

\NormalTok{nosaukums}\OtherTok{=}\StringTok{"General\_SwampsMiresBogsHelophytes\_r500.tif"}
\NormalTok{ielasisanas\_cels}\OtherTok{=}\FunctionTok{paste0}\NormalTok{(}\StringTok{"./RasterGrids\_100m/2024/RAW/"}\NormalTok{,nosaukums)}
\NormalTok{saglabasanas\_cels}\OtherTok{=}\FunctionTok{paste0}\NormalTok{(}\StringTok{"./RasterGrids\_100m/2024/Scaled/"}\NormalTok{,nosaukums)}
\NormalTok{slanis}\OtherTok{=}\FunctionTok{rast}\NormalTok{(ielasisanas\_cels)}
\NormalTok{videjais}\OtherTok{=}\FunctionTok{global}\NormalTok{(slanis,}\AttributeTok{fun=}\StringTok{"mean"}\NormalTok{,}\AttributeTok{na.rm=}\ConstantTok{TRUE}\NormalTok{)}
\NormalTok{centrets}\OtherTok{=}\NormalTok{slanis}\SpecialCharTok{{-}}\NormalTok{videjais[,}\DecValTok{1}\NormalTok{]}
\NormalTok{standartnovirze}\OtherTok{=}\NormalTok{terra}\SpecialCharTok{::}\FunctionTok{global}\NormalTok{(centrets,}\AttributeTok{fun=}\StringTok{"rms"}\NormalTok{,}\AttributeTok{na.rm=}\ConstantTok{TRUE}\NormalTok{)}
\NormalTok{merogots}\OtherTok{=}\NormalTok{centrets}\SpecialCharTok{/}\NormalTok{standartnovirze[,}\DecValTok{1}\NormalTok{]}
\FunctionTok{writeRaster}\NormalTok{(merogots,}
      \AttributeTok{filename=}\NormalTok{saglabasanas\_cels,}
      \AttributeTok{overwrite=}\ConstantTok{TRUE}\NormalTok{)}
\end{Highlighting}
\end{Shaded}

\section{General\_SwampsMiresBogsHelophytes\_r1250}\label{ch06.446}

\textbf{filename:} \texttt{General\_SwampsMiresBogsHelophytes\_r1250.tif}

\textbf{layername:} \texttt{egv\_446}

\textbf{English name:} Fractional cover of Swamps, Mires, Bogs, Reed-, Sedge-, Rush-
Beds within the 1.25 km landscape

\textbf{Latvian name:} Purvu, niedrāju, grīslāju, meldrāju platības īpatsvars 1,25 km
ainavā

\textbf{Procedure:} The cover fraction within a radius of 1250 m around the analysis grid cell
is calculated as the area-weighted sum of the \hyperref[ch06.444]{analysis cells} inside
the buffer, using the workflow \texttt{egvtools::radius\_function()}. During the calculation of the landscape
metric, inverse distance weighted (power = 2) gap filling on the output is
applied to ensure no missing values at the edges. Then the layer is
rewritten to set its name. Finally, the layer is standardised by
subtracting the arithmetic mean and dividing by the root mean squared error.

\begin{Shaded}
\begin{Highlighting}[]
\CommentTok{\# libs {-}{-}{-}{-}}
\ControlFlowTok{if}\NormalTok{(}\SpecialCharTok{!}\FunctionTok{require}\NormalTok{(terra)) \{}\FunctionTok{install.packages}\NormalTok{(}\StringTok{"terra"}\NormalTok{); }\FunctionTok{require}\NormalTok{(terra)\}}
\ControlFlowTok{if}\NormalTok{(}\SpecialCharTok{!}\FunctionTok{require}\NormalTok{(egvtools)) \{remotes}\SpecialCharTok{::}\FunctionTok{install\_github}\NormalTok{(}\StringTok{"aavotins/egvtools"}\NormalTok{); }\FunctionTok{require}\NormalTok{(egvtools)\}}


\CommentTok{\# Templates {-}{-}{-}{-}{-}}
\NormalTok{template100}\OtherTok{=}\FunctionTok{rast}\NormalTok{(}\StringTok{"./Templates/TemplateRasters/LV100m\_10km.tif"}\NormalTok{)}

\CommentTok{\# radii {-}{-}{-}{-}}
\FunctionTok{radius\_function}\NormalTok{(}
 \AttributeTok{kvadrati\_path =} \StringTok{"./Templates/TemplateGrids/tiles/"}\NormalTok{,}
 \AttributeTok{radii\_path   =} \StringTok{"./Templates/TemplateGridPoints/tiles/"}\NormalTok{,}
 \AttributeTok{tikls100\_path =} \StringTok{"./Templates/TemplateGrids/tikls100\_sauzeme.parquet"}\NormalTok{,}
 \AttributeTok{template\_path =} \StringTok{"./Templates/TemplateRasters/LV100m\_10km.tif"}\NormalTok{,}
 \AttributeTok{input\_layers  =} \FunctionTok{c}\NormalTok{(}\StringTok{"./RasterGrids\_100m/2024/RAW/General\_SwampsMiresBogsHelophytes\_cell.tif"}\NormalTok{),}
 \AttributeTok{layer\_prefixes =} \FunctionTok{c}\NormalTok{(}\StringTok{"General\_SwampsMiresBogsHelophytes"}\NormalTok{),}
 \AttributeTok{output\_dir   =} \StringTok{"./RasterGrids\_100m/2024/RAW/"}\NormalTok{,}
 \AttributeTok{n\_workers   =} \DecValTok{6}\NormalTok{,}
 \AttributeTok{radii     =} \FunctionTok{c}\NormalTok{(}\StringTok{"r1250"}\NormalTok{),}
 \AttributeTok{radius\_mode  =} \StringTok{"sparse"}\NormalTok{,}
 \AttributeTok{extract\_fun  =} \StringTok{"mean"}\NormalTok{,}
 \AttributeTok{fill\_missing  =} \ConstantTok{TRUE}\NormalTok{,}
 \AttributeTok{IDW\_weight   =} \DecValTok{2}\NormalTok{,}
 \AttributeTok{future\_max\_size =} \DecValTok{40} \SpecialCharTok{*} \DecValTok{1024}\SpecialCharTok{\^{}}\DecValTok{3}\NormalTok{)}


\CommentTok{\# General\_SwampsMiresBogsHelophytes\_r1250.tif   egv\_446}
\NormalTok{slanis}\OtherTok{=}\FunctionTok{rast}\NormalTok{(}\StringTok{"./RasterGrids\_100m/2024/RAW/General\_SwampsMiresBogsHelophytes\_r1250.tif"}\NormalTok{)}
\FunctionTok{names}\NormalTok{(slanis)}\OtherTok{=}\StringTok{"egv\_446"}
\NormalTok{slanis2}\OtherTok{=}\FunctionTok{project}\NormalTok{(slanis,template100)}
\FunctionTok{writeRaster}\NormalTok{(slanis2,}
      \StringTok{"./RasterGrids\_100m/2024/RAW/General\_SwampsMiresBogsHelophytes\_r1250.tif"}\NormalTok{,}
      \AttributeTok{overwrite=}\ConstantTok{TRUE}\NormalTok{)}

\CommentTok{\# standardisation {-}{-}{-}{-}}
\ControlFlowTok{if}\NormalTok{(}\SpecialCharTok{!}\FunctionTok{require}\NormalTok{(terra)) \{}\FunctionTok{install.packages}\NormalTok{(}\StringTok{"terra"}\NormalTok{); }\FunctionTok{require}\NormalTok{(terra)\}}
\ControlFlowTok{if}\NormalTok{(}\SpecialCharTok{!}\FunctionTok{require}\NormalTok{(tidyverse)) \{}\FunctionTok{install.packages}\NormalTok{(}\StringTok{"tidyverse"}\NormalTok{); }\FunctionTok{require}\NormalTok{(tidyverse)\}}

\NormalTok{nosaukums}\OtherTok{=}\StringTok{"General\_SwampsMiresBogsHelophytes\_r1250.tif"}
\NormalTok{ielasisanas\_cels}\OtherTok{=}\FunctionTok{paste0}\NormalTok{(}\StringTok{"./RasterGrids\_100m/2024/RAW/"}\NormalTok{,nosaukums)}
\NormalTok{saglabasanas\_cels}\OtherTok{=}\FunctionTok{paste0}\NormalTok{(}\StringTok{"./RasterGrids\_100m/2024/Scaled/"}\NormalTok{,nosaukums)}
\NormalTok{slanis}\OtherTok{=}\FunctionTok{rast}\NormalTok{(ielasisanas\_cels)}
\NormalTok{videjais}\OtherTok{=}\FunctionTok{global}\NormalTok{(slanis,}\AttributeTok{fun=}\StringTok{"mean"}\NormalTok{,}\AttributeTok{na.rm=}\ConstantTok{TRUE}\NormalTok{)}
\NormalTok{centrets}\OtherTok{=}\NormalTok{slanis}\SpecialCharTok{{-}}\NormalTok{videjais[,}\DecValTok{1}\NormalTok{]}
\NormalTok{standartnovirze}\OtherTok{=}\NormalTok{terra}\SpecialCharTok{::}\FunctionTok{global}\NormalTok{(centrets,}\AttributeTok{fun=}\StringTok{"rms"}\NormalTok{,}\AttributeTok{na.rm=}\ConstantTok{TRUE}\NormalTok{)}
\NormalTok{merogots}\OtherTok{=}\NormalTok{centrets}\SpecialCharTok{/}\NormalTok{standartnovirze[,}\DecValTok{1}\NormalTok{]}
\FunctionTok{writeRaster}\NormalTok{(merogots,}
      \AttributeTok{filename=}\NormalTok{saglabasanas\_cels,}
      \AttributeTok{overwrite=}\ConstantTok{TRUE}\NormalTok{)}
\end{Highlighting}
\end{Shaded}

\section{General\_SwampsMiresBogsHelophytes\_r3000}\label{ch06.447}

\textbf{filename:} \texttt{General\_SwampsMiresBogsHelophytes\_r3000.tif}

\textbf{layername:} \texttt{egv\_447}

\textbf{English name:} Fractional cover of Swamps, Mires, Bogs, Reed-, Sedge-, Rush-
Beds within the 3 km landscape

\textbf{Latvian name:} Purvu, niedrāju, grīslāju, meldrāju platības īpatsvars 3 km
ainavā

\textbf{Procedure:} The cover fraction within a radius of 3000 m around the analysis grid cell
is calculated as the area-weighted sum of the \hyperref[ch06.444]{analysis cells} inside
the buffer, using the workflow \texttt{egvtools::radius\_function()}. During the calculation of the landscape
metric, inverse distance weighted (power = 2) gap filling on the output is
applied to ensure no missing values at the edges. Then the layer is
rewritten to set its name. Finally, the layer is standardised by
subtracting the arithmetic mean and dividing by the root mean squared error.

\begin{Shaded}
\begin{Highlighting}[]
\CommentTok{\# libs {-}{-}{-}{-}}
\ControlFlowTok{if}\NormalTok{(}\SpecialCharTok{!}\FunctionTok{require}\NormalTok{(terra)) \{}\FunctionTok{install.packages}\NormalTok{(}\StringTok{"terra"}\NormalTok{); }\FunctionTok{require}\NormalTok{(terra)\}}
\ControlFlowTok{if}\NormalTok{(}\SpecialCharTok{!}\FunctionTok{require}\NormalTok{(egvtools)) \{remotes}\SpecialCharTok{::}\FunctionTok{install\_github}\NormalTok{(}\StringTok{"aavotins/egvtools"}\NormalTok{); }\FunctionTok{require}\NormalTok{(egvtools)\}}


\CommentTok{\# Templates {-}{-}{-}{-}{-}}
\NormalTok{template100}\OtherTok{=}\FunctionTok{rast}\NormalTok{(}\StringTok{"./Templates/TemplateRasters/LV100m\_10km.tif"}\NormalTok{)}

\CommentTok{\# radii {-}{-}{-}{-}}
\FunctionTok{radius\_function}\NormalTok{(}
 \AttributeTok{kvadrati\_path =} \StringTok{"./Templates/TemplateGrids/tiles/"}\NormalTok{,}
 \AttributeTok{radii\_path   =} \StringTok{"./Templates/TemplateGridPoints/tiles/"}\NormalTok{,}
 \AttributeTok{tikls100\_path =} \StringTok{"./Templates/TemplateGrids/tikls100\_sauzeme.parquet"}\NormalTok{,}
 \AttributeTok{template\_path =} \StringTok{"./Templates/TemplateRasters/LV100m\_10km.tif"}\NormalTok{,}
 \AttributeTok{input\_layers  =} \FunctionTok{c}\NormalTok{(}\StringTok{"./RasterGrids\_100m/2024/RAW/General\_SwampsMiresBogsHelophytes\_cell.tif"}\NormalTok{),}
 \AttributeTok{layer\_prefixes =} \FunctionTok{c}\NormalTok{(}\StringTok{"General\_SwampsMiresBogsHelophytes"}\NormalTok{),}
 \AttributeTok{output\_dir   =} \StringTok{"./RasterGrids\_100m/2024/RAW/"}\NormalTok{,}
 \AttributeTok{n\_workers   =} \DecValTok{6}\NormalTok{,}
 \AttributeTok{radii     =} \FunctionTok{c}\NormalTok{(}\StringTok{"r3000"}\NormalTok{),}
 \AttributeTok{radius\_mode  =} \StringTok{"sparse"}\NormalTok{,}
 \AttributeTok{extract\_fun  =} \StringTok{"mean"}\NormalTok{,}
 \AttributeTok{fill\_missing  =} \ConstantTok{TRUE}\NormalTok{,}
 \AttributeTok{IDW\_weight   =} \DecValTok{2}\NormalTok{,}
 \AttributeTok{future\_max\_size =} \DecValTok{40} \SpecialCharTok{*} \DecValTok{1024}\SpecialCharTok{\^{}}\DecValTok{3}\NormalTok{)}


\CommentTok{\# General\_SwampsMiresBogsHelophytes\_r3000.tif   egv\_447}
\NormalTok{slanis}\OtherTok{=}\FunctionTok{rast}\NormalTok{(}\StringTok{"./RasterGrids\_100m/2024/RAW/General\_SwampsMiresBogsHelophytes\_r3000.tif"}\NormalTok{)}
\FunctionTok{names}\NormalTok{(slanis)}\OtherTok{=}\StringTok{"egv\_447"}
\NormalTok{slanis2}\OtherTok{=}\FunctionTok{project}\NormalTok{(slanis,template100)}
\FunctionTok{writeRaster}\NormalTok{(slanis2,}
      \StringTok{"./RasterGrids\_100m/2024/RAW/General\_SwampsMiresBogsHelophytes\_r3000.tif"}\NormalTok{,}
      \AttributeTok{overwrite=}\ConstantTok{TRUE}\NormalTok{)}

\CommentTok{\# standardisation {-}{-}{-}{-}}
\ControlFlowTok{if}\NormalTok{(}\SpecialCharTok{!}\FunctionTok{require}\NormalTok{(terra)) \{}\FunctionTok{install.packages}\NormalTok{(}\StringTok{"terra"}\NormalTok{); }\FunctionTok{require}\NormalTok{(terra)\}}
\ControlFlowTok{if}\NormalTok{(}\SpecialCharTok{!}\FunctionTok{require}\NormalTok{(tidyverse)) \{}\FunctionTok{install.packages}\NormalTok{(}\StringTok{"tidyverse"}\NormalTok{); }\FunctionTok{require}\NormalTok{(tidyverse)\}}

\NormalTok{nosaukums}\OtherTok{=}\StringTok{"General\_SwampsMiresBogsHelophytes\_r3000.tif"}
\NormalTok{ielasisanas\_cels}\OtherTok{=}\FunctionTok{paste0}\NormalTok{(}\StringTok{"./RasterGrids\_100m/2024/RAW/"}\NormalTok{,nosaukums)}
\NormalTok{saglabasanas\_cels}\OtherTok{=}\FunctionTok{paste0}\NormalTok{(}\StringTok{"./RasterGrids\_100m/2024/Scaled/"}\NormalTok{,nosaukums)}
\NormalTok{slanis}\OtherTok{=}\FunctionTok{rast}\NormalTok{(ielasisanas\_cels)}
\NormalTok{videjais}\OtherTok{=}\FunctionTok{global}\NormalTok{(slanis,}\AttributeTok{fun=}\StringTok{"mean"}\NormalTok{,}\AttributeTok{na.rm=}\ConstantTok{TRUE}\NormalTok{)}
\NormalTok{centrets}\OtherTok{=}\NormalTok{slanis}\SpecialCharTok{{-}}\NormalTok{videjais[,}\DecValTok{1}\NormalTok{]}
\NormalTok{standartnovirze}\OtherTok{=}\NormalTok{terra}\SpecialCharTok{::}\FunctionTok{global}\NormalTok{(centrets,}\AttributeTok{fun=}\StringTok{"rms"}\NormalTok{,}\AttributeTok{na.rm=}\ConstantTok{TRUE}\NormalTok{)}
\NormalTok{merogots}\OtherTok{=}\NormalTok{centrets}\SpecialCharTok{/}\NormalTok{standartnovirze[,}\DecValTok{1}\NormalTok{]}
\FunctionTok{writeRaster}\NormalTok{(merogots,}
      \AttributeTok{filename=}\NormalTok{saglabasanas\_cels,}
      \AttributeTok{overwrite=}\ConstantTok{TRUE}\NormalTok{)}
\end{Highlighting}
\end{Shaded}

\section{General\_SwampsMiresBogsHelophytes\_r10000}\label{ch06.448}

\textbf{filename:} \texttt{General\_SwampsMiresBogsHelophytes\_r10000.tif}

\textbf{layername:} \texttt{egv\_448}

\textbf{English name:} Fractional cover of Swamps, Mires, Bogs, Reed-, Sedge-, Rush-
Beds within the 10 km landscape

\textbf{Latvian name:} Purvu, niedrāju, grīslāju, meldrāju platības īpatsvars 10 km
ainavā

\textbf{Procedure:} The cover fraction within a radius of 10000 m around the analysis grid cell
is calculated as the area-weighted sum of the \hyperref[ch06.444]{analysis cells} inside
the buffer, using the workflow \texttt{egvtools::radius\_function()}. During the calculation of the landscape
metric, inverse distance weighted (power = 2) gap filling on the output is
applied to ensure no missing values at the edges. Then the layer is
rewritten to set its name. Finally, the layer is standardised by
subtracting the arithmetic mean and dividing by the root mean squared error.

\begin{Shaded}
\begin{Highlighting}[]
\CommentTok{\# libs {-}{-}{-}{-}}
\ControlFlowTok{if}\NormalTok{(}\SpecialCharTok{!}\FunctionTok{require}\NormalTok{(terra)) \{}\FunctionTok{install.packages}\NormalTok{(}\StringTok{"terra"}\NormalTok{); }\FunctionTok{require}\NormalTok{(terra)\}}
\ControlFlowTok{if}\NormalTok{(}\SpecialCharTok{!}\FunctionTok{require}\NormalTok{(egvtools)) \{remotes}\SpecialCharTok{::}\FunctionTok{install\_github}\NormalTok{(}\StringTok{"aavotins/egvtools"}\NormalTok{); }\FunctionTok{require}\NormalTok{(egvtools)\}}


\CommentTok{\# Templates {-}{-}{-}{-}{-}}
\NormalTok{template100}\OtherTok{=}\FunctionTok{rast}\NormalTok{(}\StringTok{"./Templates/TemplateRasters/LV100m\_10km.tif"}\NormalTok{)}

\CommentTok{\# radii {-}{-}{-}{-}}
\FunctionTok{radius\_function}\NormalTok{(}
 \AttributeTok{kvadrati\_path =} \StringTok{"./Templates/TemplateGrids/tiles/"}\NormalTok{,}
 \AttributeTok{radii\_path   =} \StringTok{"./Templates/TemplateGridPoints/tiles/"}\NormalTok{,}
 \AttributeTok{tikls100\_path =} \StringTok{"./Templates/TemplateGrids/tikls100\_sauzeme.parquet"}\NormalTok{,}
 \AttributeTok{template\_path =} \StringTok{"./Templates/TemplateRasters/LV100m\_10km.tif"}\NormalTok{,}
 \AttributeTok{input\_layers  =} \FunctionTok{c}\NormalTok{(}\StringTok{"./RasterGrids\_100m/2024/RAW/General\_SwampsMiresBogsHelophytes\_cell.tif"}\NormalTok{),}
 \AttributeTok{layer\_prefixes =} \FunctionTok{c}\NormalTok{(}\StringTok{"General\_SwampsMiresBogsHelophytes"}\NormalTok{),}
 \AttributeTok{output\_dir   =} \StringTok{"./RasterGrids\_100m/2024/RAW/"}\NormalTok{,}
 \AttributeTok{n\_workers   =} \DecValTok{6}\NormalTok{,}
 \AttributeTok{radii     =} \FunctionTok{c}\NormalTok{(}\StringTok{"r10000"}\NormalTok{),}
 \AttributeTok{radius\_mode  =} \StringTok{"sparse"}\NormalTok{,}
 \AttributeTok{extract\_fun  =} \StringTok{"mean"}\NormalTok{,}
 \AttributeTok{fill\_missing  =} \ConstantTok{TRUE}\NormalTok{,}
 \AttributeTok{IDW\_weight   =} \DecValTok{2}\NormalTok{,}
 \AttributeTok{future\_max\_size =} \DecValTok{40} \SpecialCharTok{*} \DecValTok{1024}\SpecialCharTok{\^{}}\DecValTok{3}\NormalTok{)}


\CommentTok{\# General\_SwampsMiresBogsHelophytes\_r10000.tif  egv\_448}
\NormalTok{slanis}\OtherTok{=}\FunctionTok{rast}\NormalTok{(}\StringTok{"./RasterGrids\_100m/2024/RAW/General\_SwampsMiresBogsHelophytes\_r10000.tif"}\NormalTok{)}
\FunctionTok{names}\NormalTok{(slanis)}\OtherTok{=}\StringTok{"egv\_448"}
\NormalTok{slanis2}\OtherTok{=}\FunctionTok{project}\NormalTok{(slanis,template100)}
\FunctionTok{writeRaster}\NormalTok{(slanis2,}
      \StringTok{"./RasterGrids\_100m/2024/RAW/General\_SwampsMiresBogsHelophytes\_r10000.tif"}\NormalTok{,}
      \AttributeTok{overwrite=}\ConstantTok{TRUE}\NormalTok{)}

\CommentTok{\# standardisation {-}{-}{-}{-}}
\ControlFlowTok{if}\NormalTok{(}\SpecialCharTok{!}\FunctionTok{require}\NormalTok{(terra)) \{}\FunctionTok{install.packages}\NormalTok{(}\StringTok{"terra"}\NormalTok{); }\FunctionTok{require}\NormalTok{(terra)\}}
\ControlFlowTok{if}\NormalTok{(}\SpecialCharTok{!}\FunctionTok{require}\NormalTok{(tidyverse)) \{}\FunctionTok{install.packages}\NormalTok{(}\StringTok{"tidyverse"}\NormalTok{); }\FunctionTok{require}\NormalTok{(tidyverse)\}}

\NormalTok{nosaukums}\OtherTok{=}\StringTok{"General\_SwampsMiresBogsHelophytes\_r10000.tif"}
\NormalTok{ielasisanas\_cels}\OtherTok{=}\FunctionTok{paste0}\NormalTok{(}\StringTok{"./RasterGrids\_100m/2024/RAW/"}\NormalTok{,nosaukums)}
\NormalTok{saglabasanas\_cels}\OtherTok{=}\FunctionTok{paste0}\NormalTok{(}\StringTok{"./RasterGrids\_100m/2024/Scaled/"}\NormalTok{,nosaukums)}
\NormalTok{slanis}\OtherTok{=}\FunctionTok{rast}\NormalTok{(ielasisanas\_cels)}
\NormalTok{videjais}\OtherTok{=}\FunctionTok{global}\NormalTok{(slanis,}\AttributeTok{fun=}\StringTok{"mean"}\NormalTok{,}\AttributeTok{na.rm=}\ConstantTok{TRUE}\NormalTok{)}
\NormalTok{centrets}\OtherTok{=}\NormalTok{slanis}\SpecialCharTok{{-}}\NormalTok{videjais[,}\DecValTok{1}\NormalTok{]}
\NormalTok{standartnovirze}\OtherTok{=}\NormalTok{terra}\SpecialCharTok{::}\FunctionTok{global}\NormalTok{(centrets,}\AttributeTok{fun=}\StringTok{"rms"}\NormalTok{,}\AttributeTok{na.rm=}\ConstantTok{TRUE}\NormalTok{)}
\NormalTok{merogots}\OtherTok{=}\NormalTok{centrets}\SpecialCharTok{/}\NormalTok{standartnovirze[,}\DecValTok{1}\NormalTok{]}
\FunctionTok{writeRaster}\NormalTok{(merogots,}
      \AttributeTok{filename=}\NormalTok{saglabasanas\_cels,}
      \AttributeTok{overwrite=}\ConstantTok{TRUE}\NormalTok{)}
\end{Highlighting}
\end{Shaded}

\section{General\_Trees\_cell}\label{ch06.449}

\textbf{filename:} \texttt{General\_Trees\_cell.tif}

\textbf{layername:} \texttt{egv\_449}

\textbf{English name:} Fractional cover of Trees, Shrubs, Clear-cuts within the
analysis cell (1 ha)

\textbf{Latvian name:} Koku, krūmu un izcirtumu platības īpatsvars analīzes šūnā (1
ha)

\textbf{Procedure:} First, the trees, shrubs and clear cuts from the \hyperref[Ch05.03]{Landscape
classification} are selected (values between 600 and 700 are
reclassified to value 1; all others are set to 0). The resulting layer
is then aggregated to EGV resolution using the workflow \texttt{egvtools::input2egv()}, which
calculates the arithmetic mean to determine the cover fraction. During
aggregation, inverse distance weighted (power = 2) gap filling on the output is
applied to ensure no missing values at the edges. Finally, the layer is
standardised by subtracting the arithmetic mean and dividing by the root mean squared
error.

\begin{Shaded}
\begin{Highlighting}[]
\CommentTok{\# libs {-}{-}{-}{-}}
\ControlFlowTok{if}\NormalTok{(}\SpecialCharTok{!}\FunctionTok{require}\NormalTok{(egvtools)) \{remotes}\SpecialCharTok{::}\FunctionTok{install\_github}\NormalTok{(}\StringTok{"aavotins/egvtools"}\NormalTok{); }\FunctionTok{require}\NormalTok{(egvtools)\}}
\ControlFlowTok{if}\NormalTok{(}\SpecialCharTok{!}\FunctionTok{require}\NormalTok{(terra)) \{}\FunctionTok{install.packages}\NormalTok{(}\StringTok{"terra"}\NormalTok{); }\FunctionTok{require}\NormalTok{(terra)\}}
\ControlFlowTok{if}\NormalTok{(}\SpecialCharTok{!}\FunctionTok{require}\NormalTok{(sf)) \{}\FunctionTok{install.packages}\NormalTok{(}\StringTok{"sf"}\NormalTok{); }\FunctionTok{require}\NormalTok{(sf)\}}
\ControlFlowTok{if}\NormalTok{(}\SpecialCharTok{!}\FunctionTok{require}\NormalTok{(tidyverse)) \{}\FunctionTok{install.packages}\NormalTok{(}\StringTok{"tidyverse"}\NormalTok{); }\FunctionTok{require}\NormalTok{(tidyverse)\}}
\ControlFlowTok{if}\NormalTok{(}\SpecialCharTok{!}\FunctionTok{require}\NormalTok{(sfarrow)) \{}\FunctionTok{install.packages}\NormalTok{(}\StringTok{"sfarrow"}\NormalTok{); }\FunctionTok{require}\NormalTok{(sfarrow)\}}
\ControlFlowTok{if}\NormalTok{(}\SpecialCharTok{!}\FunctionTok{require}\NormalTok{(readxl)) \{}\FunctionTok{install.packages}\NormalTok{(}\StringTok{"readxl"}\NormalTok{); }\FunctionTok{require}\NormalTok{(readxl)\}}
\ControlFlowTok{if}\NormalTok{(}\SpecialCharTok{!}\FunctionTok{require}\NormalTok{(raster)) \{}\FunctionTok{install.packages}\NormalTok{(}\StringTok{"raster"}\NormalTok{); }\FunctionTok{require}\NormalTok{(raster)\}}
\ControlFlowTok{if}\NormalTok{(}\SpecialCharTok{!}\FunctionTok{require}\NormalTok{(fasterize)) \{}\FunctionTok{install.packages}\NormalTok{(}\StringTok{"fasterize"}\NormalTok{); }\FunctionTok{require}\NormalTok{(fasterize)\}}

\CommentTok{\# templates {-}{-}{-}{-}}
\NormalTok{template100}\OtherTok{=}\FunctionTok{rast}\NormalTok{(}\StringTok{"./Templates/TemplateRasters/LV100m\_10km.tif"}\NormalTok{)}
\NormalTok{template10}\OtherTok{=}\FunctionTok{rast}\NormalTok{(}\StringTok{"./Templates/TemplateRasters/LV10m\_10km.tif"}\NormalTok{)}
\NormalTok{rastrs10}\OtherTok{=}\FunctionTok{raster}\NormalTok{(template10)}

\NormalTok{nulls10}\OtherTok{=}\FunctionTok{rast}\NormalTok{(}\StringTok{"./Templates/TemplateRasters/nulls\_LV10m\_10km.tif"}\NormalTok{)}
\NormalTok{nulls100}\OtherTok{=}\FunctionTok{rast}\NormalTok{(}\StringTok{"./Templates/TemplateRasters/nulls\_LV100m\_10km.tif"}\NormalTok{)}

\CommentTok{\# simple landscape {-}{-}{-}{-}}
\NormalTok{simple\_landscape}\OtherTok{=}\FunctionTok{rast}\NormalTok{(}\StringTok{"RasterGrids\_10m/2024/Ainava\_vienk\_mask.tif"}\NormalTok{)}


\CommentTok{\# General\_Trees\_cell.tif    egv\_449 {-}{-}{-}{-}}
\NormalTok{kokimezi}\OtherTok{=}\FunctionTok{ifel}\NormalTok{(simple\_landscape}\SpecialCharTok{\textgreater{}=}\DecValTok{600}\SpecialCharTok{\&}\NormalTok{simple\_landscape}\SpecialCharTok{\textless{}}\DecValTok{700}\NormalTok{,}\DecValTok{1}\NormalTok{,}\DecValTok{0}\NormalTok{)}
\NormalTok{i2e\_rez}\OtherTok{=}\NormalTok{egvtools}\SpecialCharTok{::}\FunctionTok{input2egv}\NormalTok{(}\AttributeTok{input=}\NormalTok{kokimezi,}
              \AttributeTok{egv\_template=} \StringTok{"./Templates/TemplateRasters/LV100m\_10km.tif"}\NormalTok{,}
              \AttributeTok{summary\_function =} \StringTok{"average"}\NormalTok{,}
              \AttributeTok{missing\_job =} \StringTok{"FillOutput"}\NormalTok{,}
              \AttributeTok{outlocation =} \StringTok{"./RasterGrids\_100m/2024/RAW/"}\NormalTok{,}
              \AttributeTok{outfilename =} \StringTok{"General\_Trees\_cell.tif"}\NormalTok{,}
              \AttributeTok{layername =} \StringTok{"egv\_449"}\NormalTok{,}
              \AttributeTok{idw\_weight =} \DecValTok{2}\NormalTok{,}
              \AttributeTok{plot\_gaps =} \ConstantTok{FALSE}\NormalTok{,}\AttributeTok{plot\_final =} \ConstantTok{TRUE}\NormalTok{)}
\NormalTok{i2e\_rez}
\FunctionTok{rm}\NormalTok{(kokimezi)}
\FunctionTok{rm}\NormalTok{(i2e\_rez)}

\CommentTok{\# standardisation {-}{-}{-}{-}}
\ControlFlowTok{if}\NormalTok{(}\SpecialCharTok{!}\FunctionTok{require}\NormalTok{(terra)) \{}\FunctionTok{install.packages}\NormalTok{(}\StringTok{"terra"}\NormalTok{); }\FunctionTok{require}\NormalTok{(terra)\}}
\ControlFlowTok{if}\NormalTok{(}\SpecialCharTok{!}\FunctionTok{require}\NormalTok{(tidyverse)) \{}\FunctionTok{install.packages}\NormalTok{(}\StringTok{"tidyverse"}\NormalTok{); }\FunctionTok{require}\NormalTok{(tidyverse)\}}

\NormalTok{nosaukums}\OtherTok{=}\StringTok{"General\_Trees\_cell.tif"}
\NormalTok{ielasisanas\_cels}\OtherTok{=}\FunctionTok{paste0}\NormalTok{(}\StringTok{"./RasterGrids\_100m/2024/RAW/"}\NormalTok{,nosaukums)}
\NormalTok{saglabasanas\_cels}\OtherTok{=}\FunctionTok{paste0}\NormalTok{(}\StringTok{"./RasterGrids\_100m/2024/Scaled/"}\NormalTok{,nosaukums)}
\NormalTok{slanis}\OtherTok{=}\FunctionTok{rast}\NormalTok{(ielasisanas\_cels)}
\NormalTok{videjais}\OtherTok{=}\FunctionTok{global}\NormalTok{(slanis,}\AttributeTok{fun=}\StringTok{"mean"}\NormalTok{,}\AttributeTok{na.rm=}\ConstantTok{TRUE}\NormalTok{)}
\NormalTok{centrets}\OtherTok{=}\NormalTok{slanis}\SpecialCharTok{{-}}\NormalTok{videjais[,}\DecValTok{1}\NormalTok{]}
\NormalTok{standartnovirze}\OtherTok{=}\NormalTok{terra}\SpecialCharTok{::}\FunctionTok{global}\NormalTok{(centrets,}\AttributeTok{fun=}\StringTok{"rms"}\NormalTok{,}\AttributeTok{na.rm=}\ConstantTok{TRUE}\NormalTok{)}
\NormalTok{merogots}\OtherTok{=}\NormalTok{centrets}\SpecialCharTok{/}\NormalTok{standartnovirze[,}\DecValTok{1}\NormalTok{]}
\FunctionTok{writeRaster}\NormalTok{(merogots,}
      \AttributeTok{filename=}\NormalTok{saglabasanas\_cels,}
      \AttributeTok{overwrite=}\ConstantTok{TRUE}\NormalTok{)}
\end{Highlighting}
\end{Shaded}

\section{General\_Trees\_r500}\label{ch06.450}

\textbf{filename:} \texttt{General\_Trees\_r500.tif}

\textbf{layername:} \texttt{egv\_450}

\textbf{English name:} Fractional cover of Trees, Shrubs, Clear-cuts within the 0.5
km landscape

\textbf{Latvian name:} Koku, krūmu un izcirtumu platības īpatsvars 0,5 km ainavā

\textbf{Procedure:} The cover fraction within a radius of 500 m around the analysis grid cell is
calculated as the area-weighted sum of the \hyperref[ch06.449]{analysis cells} inside the
buffer, using the workflow \texttt{egvtools::radius\_function()}. During the calculation of the landscape metric,
inverse distance weighted (power = 2) gap filling on the output is applied
to ensure no missing values at the edges. Then the layer is rewritten to set
its name. Finally, the layer is standardised by subtracting the arithmetic
mean and dividing by the root mean squared error.

\begin{Shaded}
\begin{Highlighting}[]
\CommentTok{\# libs {-}{-}{-}{-}}
\ControlFlowTok{if}\NormalTok{(}\SpecialCharTok{!}\FunctionTok{require}\NormalTok{(terra)) \{}\FunctionTok{install.packages}\NormalTok{(}\StringTok{"terra"}\NormalTok{); }\FunctionTok{require}\NormalTok{(terra)\}}
\ControlFlowTok{if}\NormalTok{(}\SpecialCharTok{!}\FunctionTok{require}\NormalTok{(egvtools)) \{remotes}\SpecialCharTok{::}\FunctionTok{install\_github}\NormalTok{(}\StringTok{"aavotins/egvtools"}\NormalTok{); }\FunctionTok{require}\NormalTok{(egvtools)\}}


\CommentTok{\# Templates {-}{-}{-}{-}{-}}
\NormalTok{template100}\OtherTok{=}\FunctionTok{rast}\NormalTok{(}\StringTok{"./Templates/TemplateRasters/LV100m\_10km.tif"}\NormalTok{)}

\CommentTok{\# radii {-}{-}{-}{-}}
\FunctionTok{radius\_function}\NormalTok{(}
 \AttributeTok{kvadrati\_path =} \StringTok{"./Templates/TemplateGrids/tiles/"}\NormalTok{,}
 \AttributeTok{radii\_path   =} \StringTok{"./Templates/TemplateGridPoints/tiles/"}\NormalTok{,}
 \AttributeTok{tikls100\_path =} \StringTok{"./Templates/TemplateGrids/tikls100\_sauzeme.parquet"}\NormalTok{,}
 \AttributeTok{template\_path =} \StringTok{"./Templates/TemplateRasters/LV100m\_10km.tif"}\NormalTok{,}
 \AttributeTok{input\_layers  =} \FunctionTok{c}\NormalTok{(}\StringTok{"./RasterGrids\_100m/2024/RAW/General\_Trees\_cell.tif"}\NormalTok{),}
 \AttributeTok{layer\_prefixes =} \FunctionTok{c}\NormalTok{(}\StringTok{"General\_Trees"}\NormalTok{),}
 \AttributeTok{output\_dir   =} \StringTok{"./RasterGrids\_100m/2024/RAW/"}\NormalTok{,}
 \AttributeTok{n\_workers   =} \DecValTok{6}\NormalTok{,}
 \AttributeTok{radii     =} \FunctionTok{c}\NormalTok{(}\StringTok{"r500"}\NormalTok{),}
 \AttributeTok{radius\_mode  =} \StringTok{"sparse"}\NormalTok{,}
 \AttributeTok{extract\_fun  =} \StringTok{"mean"}\NormalTok{,}
 \AttributeTok{fill\_missing  =} \ConstantTok{TRUE}\NormalTok{,}
 \AttributeTok{IDW\_weight   =} \DecValTok{2}\NormalTok{,}
 \AttributeTok{future\_max\_size =} \DecValTok{40} \SpecialCharTok{*} \DecValTok{1024}\SpecialCharTok{\^{}}\DecValTok{3}\NormalTok{)}


\CommentTok{\# General\_Trees\_r500.tif    egv\_450}
\NormalTok{slanis}\OtherTok{=}\FunctionTok{rast}\NormalTok{(}\StringTok{"./RasterGrids\_100m/2024/RAW/General\_Trees\_r500.tif"}\NormalTok{)}
\FunctionTok{names}\NormalTok{(slanis)}\OtherTok{=}\StringTok{"egv\_450"}
\NormalTok{slanis2}\OtherTok{=}\FunctionTok{project}\NormalTok{(slanis,template100)}
\FunctionTok{writeRaster}\NormalTok{(slanis2,}
      \StringTok{"./RasterGrids\_100m/2024/RAW/General\_Trees\_r500.tif"}\NormalTok{,}
      \AttributeTok{overwrite=}\ConstantTok{TRUE}\NormalTok{)}

\CommentTok{\# standardisation {-}{-}{-}{-}}
\ControlFlowTok{if}\NormalTok{(}\SpecialCharTok{!}\FunctionTok{require}\NormalTok{(terra)) \{}\FunctionTok{install.packages}\NormalTok{(}\StringTok{"terra"}\NormalTok{); }\FunctionTok{require}\NormalTok{(terra)\}}
\ControlFlowTok{if}\NormalTok{(}\SpecialCharTok{!}\FunctionTok{require}\NormalTok{(tidyverse)) \{}\FunctionTok{install.packages}\NormalTok{(}\StringTok{"tidyverse"}\NormalTok{); }\FunctionTok{require}\NormalTok{(tidyverse)\}}

\NormalTok{nosaukums}\OtherTok{=}\StringTok{"General\_Trees\_r500.tif"}
\NormalTok{ielasisanas\_cels}\OtherTok{=}\FunctionTok{paste0}\NormalTok{(}\StringTok{"./RasterGrids\_100m/2024/RAW/"}\NormalTok{,nosaukums)}
\NormalTok{saglabasanas\_cels}\OtherTok{=}\FunctionTok{paste0}\NormalTok{(}\StringTok{"./RasterGrids\_100m/2024/Scaled/"}\NormalTok{,nosaukums)}
\NormalTok{slanis}\OtherTok{=}\FunctionTok{rast}\NormalTok{(ielasisanas\_cels)}
\NormalTok{videjais}\OtherTok{=}\FunctionTok{global}\NormalTok{(slanis,}\AttributeTok{fun=}\StringTok{"mean"}\NormalTok{,}\AttributeTok{na.rm=}\ConstantTok{TRUE}\NormalTok{)}
\NormalTok{centrets}\OtherTok{=}\NormalTok{slanis}\SpecialCharTok{{-}}\NormalTok{videjais[,}\DecValTok{1}\NormalTok{]}
\NormalTok{standartnovirze}\OtherTok{=}\NormalTok{terra}\SpecialCharTok{::}\FunctionTok{global}\NormalTok{(centrets,}\AttributeTok{fun=}\StringTok{"rms"}\NormalTok{,}\AttributeTok{na.rm=}\ConstantTok{TRUE}\NormalTok{)}
\NormalTok{merogots}\OtherTok{=}\NormalTok{centrets}\SpecialCharTok{/}\NormalTok{standartnovirze[,}\DecValTok{1}\NormalTok{]}
\FunctionTok{writeRaster}\NormalTok{(merogots,}
      \AttributeTok{filename=}\NormalTok{saglabasanas\_cels,}
      \AttributeTok{overwrite=}\ConstantTok{TRUE}\NormalTok{)}
\end{Highlighting}
\end{Shaded}

\section{General\_Trees\_r1250}\label{ch06.451}

\textbf{filename:} \texttt{General\_Trees\_r1250.tif}

\textbf{layername:} \texttt{egv\_451}

\textbf{English name:} Fractional cover of Trees, Shrubs, Clear-cuts within the 1.25
km landscape

\textbf{Latvian name:} Koku, krūmu un izcirtumu platības īpatsvars 1,25 km ainavā

\textbf{Procedure:} The cover fraction within a radius of 1250 m around the analysis grid cell
is calculated as the area-weighted sum of the \hyperref[ch06.449]{analysis cells} inside
the buffer, using the workflow \texttt{egvtools::radius\_function()}. During the calculation of the landscape
metric, inverse distance weighted (power = 2) gap filling on the output is
applied to ensure no missing values at the edges. Then the layer is
rewritten to set its name. Finally, the layer is standardised by
subtracting the arithmetic mean and dividing by the root mean squared error.

\begin{Shaded}
\begin{Highlighting}[]
\CommentTok{\# libs {-}{-}{-}{-}}
\ControlFlowTok{if}\NormalTok{(}\SpecialCharTok{!}\FunctionTok{require}\NormalTok{(terra)) \{}\FunctionTok{install.packages}\NormalTok{(}\StringTok{"terra"}\NormalTok{); }\FunctionTok{require}\NormalTok{(terra)\}}
\ControlFlowTok{if}\NormalTok{(}\SpecialCharTok{!}\FunctionTok{require}\NormalTok{(egvtools)) \{remotes}\SpecialCharTok{::}\FunctionTok{install\_github}\NormalTok{(}\StringTok{"aavotins/egvtools"}\NormalTok{); }\FunctionTok{require}\NormalTok{(egvtools)\}}


\CommentTok{\# Templates {-}{-}{-}{-}{-}}
\NormalTok{template100}\OtherTok{=}\FunctionTok{rast}\NormalTok{(}\StringTok{"./Templates/TemplateRasters/LV100m\_10km.tif"}\NormalTok{)}

\CommentTok{\# radii {-}{-}{-}{-}}
\FunctionTok{radius\_function}\NormalTok{(}
 \AttributeTok{kvadrati\_path =} \StringTok{"./Templates/TemplateGrids/tiles/"}\NormalTok{,}
 \AttributeTok{radii\_path   =} \StringTok{"./Templates/TemplateGridPoints/tiles/"}\NormalTok{,}
 \AttributeTok{tikls100\_path =} \StringTok{"./Templates/TemplateGrids/tikls100\_sauzeme.parquet"}\NormalTok{,}
 \AttributeTok{template\_path =} \StringTok{"./Templates/TemplateRasters/LV100m\_10km.tif"}\NormalTok{,}
 \AttributeTok{input\_layers  =} \FunctionTok{c}\NormalTok{(}\StringTok{"./RasterGrids\_100m/2024/RAW/General\_Trees\_cell.tif"}\NormalTok{),}
 \AttributeTok{layer\_prefixes =} \FunctionTok{c}\NormalTok{(}\StringTok{"General\_Trees"}\NormalTok{),}
 \AttributeTok{output\_dir   =} \StringTok{"./RasterGrids\_100m/2024/RAW/"}\NormalTok{,}
 \AttributeTok{n\_workers   =} \DecValTok{6}\NormalTok{,}
 \AttributeTok{radii     =} \FunctionTok{c}\NormalTok{(}\StringTok{"r1250"}\NormalTok{),}
 \AttributeTok{radius\_mode  =} \StringTok{"sparse"}\NormalTok{,}
 \AttributeTok{extract\_fun  =} \StringTok{"mean"}\NormalTok{,}
 \AttributeTok{fill\_missing  =} \ConstantTok{TRUE}\NormalTok{,}
 \AttributeTok{IDW\_weight   =} \DecValTok{2}\NormalTok{,}
 \AttributeTok{future\_max\_size =} \DecValTok{40} \SpecialCharTok{*} \DecValTok{1024}\SpecialCharTok{\^{}}\DecValTok{3}\NormalTok{)}


\CommentTok{\# General\_Trees\_r1250.tif   egv\_451}
\NormalTok{slanis}\OtherTok{=}\FunctionTok{rast}\NormalTok{(}\StringTok{"./RasterGrids\_100m/2024/RAW/General\_Trees\_r1250.tif"}\NormalTok{)}
\FunctionTok{names}\NormalTok{(slanis)}\OtherTok{=}\StringTok{"egv\_451"}
\NormalTok{slanis2}\OtherTok{=}\FunctionTok{project}\NormalTok{(slanis,template100)}
\FunctionTok{writeRaster}\NormalTok{(slanis2,}
      \StringTok{"./RasterGrids\_100m/2024/RAW/General\_Trees\_r1250.tif"}\NormalTok{,}
      \AttributeTok{overwrite=}\ConstantTok{TRUE}\NormalTok{)}

\CommentTok{\# standardisation {-}{-}{-}{-}}
\ControlFlowTok{if}\NormalTok{(}\SpecialCharTok{!}\FunctionTok{require}\NormalTok{(terra)) \{}\FunctionTok{install.packages}\NormalTok{(}\StringTok{"terra"}\NormalTok{); }\FunctionTok{require}\NormalTok{(terra)\}}
\ControlFlowTok{if}\NormalTok{(}\SpecialCharTok{!}\FunctionTok{require}\NormalTok{(tidyverse)) \{}\FunctionTok{install.packages}\NormalTok{(}\StringTok{"tidyverse"}\NormalTok{); }\FunctionTok{require}\NormalTok{(tidyverse)\}}

\NormalTok{nosaukums}\OtherTok{=}\StringTok{"General\_Trees\_r1250.tif"}
\NormalTok{ielasisanas\_cels}\OtherTok{=}\FunctionTok{paste0}\NormalTok{(}\StringTok{"./RasterGrids\_100m/2024/RAW/"}\NormalTok{,nosaukums)}
\NormalTok{saglabasanas\_cels}\OtherTok{=}\FunctionTok{paste0}\NormalTok{(}\StringTok{"./RasterGrids\_100m/2024/Scaled/"}\NormalTok{,nosaukums)}
\NormalTok{slanis}\OtherTok{=}\FunctionTok{rast}\NormalTok{(ielasisanas\_cels)}
\NormalTok{videjais}\OtherTok{=}\FunctionTok{global}\NormalTok{(slanis,}\AttributeTok{fun=}\StringTok{"mean"}\NormalTok{,}\AttributeTok{na.rm=}\ConstantTok{TRUE}\NormalTok{)}
\NormalTok{centrets}\OtherTok{=}\NormalTok{slanis}\SpecialCharTok{{-}}\NormalTok{videjais[,}\DecValTok{1}\NormalTok{]}
\NormalTok{standartnovirze}\OtherTok{=}\NormalTok{terra}\SpecialCharTok{::}\FunctionTok{global}\NormalTok{(centrets,}\AttributeTok{fun=}\StringTok{"rms"}\NormalTok{,}\AttributeTok{na.rm=}\ConstantTok{TRUE}\NormalTok{)}
\NormalTok{merogots}\OtherTok{=}\NormalTok{centrets}\SpecialCharTok{/}\NormalTok{standartnovirze[,}\DecValTok{1}\NormalTok{]}
\FunctionTok{writeRaster}\NormalTok{(merogots,}
      \AttributeTok{filename=}\NormalTok{saglabasanas\_cels,}
      \AttributeTok{overwrite=}\ConstantTok{TRUE}\NormalTok{)}
\end{Highlighting}
\end{Shaded}

\section{General\_Trees\_r3000}\label{ch06.452}

\textbf{filename:} \texttt{General\_Trees\_r3000.tif}

\textbf{layername:} \texttt{egv\_452}

\textbf{English name:} Fractional cover of Trees, Shrubs, Clear-cuts within the 3 km
landscape

\textbf{Latvian name:} Koku, krūmu un izcirtumu platības īpatsvars 3 km ainavā

\textbf{Procedure:} The cover fraction within a radius of 3000 m around the analysis grid cell
is calculated as the area-weighted sum of the \hyperref[ch06.449]{analysis cells} inside
the buffer, using the workflow \texttt{egvtools::radius\_function()}. During the calculation of the landscape
metric, inverse distance weighted (power = 2) gap filling on the output is
applied to ensure no missing values at the edges. Then the layer is
rewritten to set its name. Finally, the layer is standardised by
subtracting the arithmetic mean and dividing by the root mean squared error.

\begin{Shaded}
\begin{Highlighting}[]
\CommentTok{\# libs {-}{-}{-}{-}}
\ControlFlowTok{if}\NormalTok{(}\SpecialCharTok{!}\FunctionTok{require}\NormalTok{(terra)) \{}\FunctionTok{install.packages}\NormalTok{(}\StringTok{"terra"}\NormalTok{); }\FunctionTok{require}\NormalTok{(terra)\}}
\ControlFlowTok{if}\NormalTok{(}\SpecialCharTok{!}\FunctionTok{require}\NormalTok{(egvtools)) \{remotes}\SpecialCharTok{::}\FunctionTok{install\_github}\NormalTok{(}\StringTok{"aavotins/egvtools"}\NormalTok{); }\FunctionTok{require}\NormalTok{(egvtools)\}}


\CommentTok{\# Templates {-}{-}{-}{-}{-}}
\NormalTok{template100}\OtherTok{=}\FunctionTok{rast}\NormalTok{(}\StringTok{"./Templates/TemplateRasters/LV100m\_10km.tif"}\NormalTok{)}

\CommentTok{\# radii {-}{-}{-}{-}}
\FunctionTok{radius\_function}\NormalTok{(}
 \AttributeTok{kvadrati\_path =} \StringTok{"./Templates/TemplateGrids/tiles/"}\NormalTok{,}
 \AttributeTok{radii\_path   =} \StringTok{"./Templates/TemplateGridPoints/tiles/"}\NormalTok{,}
 \AttributeTok{tikls100\_path =} \StringTok{"./Templates/TemplateGrids/tikls100\_sauzeme.parquet"}\NormalTok{,}
 \AttributeTok{template\_path =} \StringTok{"./Templates/TemplateRasters/LV100m\_10km.tif"}\NormalTok{,}
 \AttributeTok{input\_layers  =} \FunctionTok{c}\NormalTok{(}\StringTok{"./RasterGrids\_100m/2024/RAW/General\_Trees\_cell.tif"}\NormalTok{),}
 \AttributeTok{layer\_prefixes =} \FunctionTok{c}\NormalTok{(}\StringTok{"General\_Trees"}\NormalTok{),}
 \AttributeTok{output\_dir   =} \StringTok{"./RasterGrids\_100m/2024/RAW/"}\NormalTok{,}
 \AttributeTok{n\_workers   =} \DecValTok{6}\NormalTok{,}
 \AttributeTok{radii     =} \FunctionTok{c}\NormalTok{(}\StringTok{"r3000"}\NormalTok{),}
 \AttributeTok{radius\_mode  =} \StringTok{"sparse"}\NormalTok{,}
 \AttributeTok{extract\_fun  =} \StringTok{"mean"}\NormalTok{,}
 \AttributeTok{fill\_missing  =} \ConstantTok{TRUE}\NormalTok{,}
 \AttributeTok{IDW\_weight   =} \DecValTok{2}\NormalTok{,}
 \AttributeTok{future\_max\_size =} \DecValTok{40} \SpecialCharTok{*} \DecValTok{1024}\SpecialCharTok{\^{}}\DecValTok{3}\NormalTok{)}


\CommentTok{\# General\_Trees\_r3000.tif   egv\_452}
\NormalTok{slanis}\OtherTok{=}\FunctionTok{rast}\NormalTok{(}\StringTok{"./RasterGrids\_100m/2024/RAW/General\_Trees\_r3000.tif"}\NormalTok{)}
\FunctionTok{names}\NormalTok{(slanis)}\OtherTok{=}\StringTok{"egv\_452"}
\NormalTok{slanis2}\OtherTok{=}\FunctionTok{project}\NormalTok{(slanis,template100)}
\FunctionTok{writeRaster}\NormalTok{(slanis2,}
      \StringTok{"./RasterGrids\_100m/2024/RAW/General\_Trees\_r3000.tif"}\NormalTok{,}
      \AttributeTok{overwrite=}\ConstantTok{TRUE}\NormalTok{)}

\CommentTok{\# standardisation {-}{-}{-}{-}}
\ControlFlowTok{if}\NormalTok{(}\SpecialCharTok{!}\FunctionTok{require}\NormalTok{(terra)) \{}\FunctionTok{install.packages}\NormalTok{(}\StringTok{"terra"}\NormalTok{); }\FunctionTok{require}\NormalTok{(terra)\}}
\ControlFlowTok{if}\NormalTok{(}\SpecialCharTok{!}\FunctionTok{require}\NormalTok{(tidyverse)) \{}\FunctionTok{install.packages}\NormalTok{(}\StringTok{"tidyverse"}\NormalTok{); }\FunctionTok{require}\NormalTok{(tidyverse)\}}

\NormalTok{nosaukums}\OtherTok{=}\StringTok{"General\_Trees\_r3000.tif"}
\NormalTok{ielasisanas\_cels}\OtherTok{=}\FunctionTok{paste0}\NormalTok{(}\StringTok{"./RasterGrids\_100m/2024/RAW/"}\NormalTok{,nosaukums)}
\NormalTok{saglabasanas\_cels}\OtherTok{=}\FunctionTok{paste0}\NormalTok{(}\StringTok{"./RasterGrids\_100m/2024/Scaled/"}\NormalTok{,nosaukums)}
\NormalTok{slanis}\OtherTok{=}\FunctionTok{rast}\NormalTok{(ielasisanas\_cels)}
\NormalTok{videjais}\OtherTok{=}\FunctionTok{global}\NormalTok{(slanis,}\AttributeTok{fun=}\StringTok{"mean"}\NormalTok{,}\AttributeTok{na.rm=}\ConstantTok{TRUE}\NormalTok{)}
\NormalTok{centrets}\OtherTok{=}\NormalTok{slanis}\SpecialCharTok{{-}}\NormalTok{videjais[,}\DecValTok{1}\NormalTok{]}
\NormalTok{standartnovirze}\OtherTok{=}\NormalTok{terra}\SpecialCharTok{::}\FunctionTok{global}\NormalTok{(centrets,}\AttributeTok{fun=}\StringTok{"rms"}\NormalTok{,}\AttributeTok{na.rm=}\ConstantTok{TRUE}\NormalTok{)}
\NormalTok{merogots}\OtherTok{=}\NormalTok{centrets}\SpecialCharTok{/}\NormalTok{standartnovirze[,}\DecValTok{1}\NormalTok{]}
\FunctionTok{writeRaster}\NormalTok{(merogots,}
      \AttributeTok{filename=}\NormalTok{saglabasanas\_cels,}
      \AttributeTok{overwrite=}\ConstantTok{TRUE}\NormalTok{)}
\end{Highlighting}
\end{Shaded}

\section{General\_Trees\_r10000}\label{ch06.453}

\textbf{filename:} \texttt{General\_Trees\_r10000.tif}

\textbf{layername:} \texttt{egv\_453}

\textbf{English name:} Fractional cover of Trees, Shrubs, Clear-cuts within the 10 km
landscape

\textbf{Latvian name:} Koku, krūmu un izcirtumu platības īpatsvars 10 km ainavā

\textbf{Procedure:} The cover fraction within a radius of 10000 m around the analysis grid cell
is calculated as the area-weighted sum of the \hyperref[ch06.449]{analysis cells} inside
the buffer, using the workflow \texttt{egvtools::radius\_function()}. During the calculation of the landscape
metric, inverse distance weighted (power = 2) gap filling on the output is
applied to ensure no missing values at the edges. Then the layer is
rewritten to set its name. Finally, the layer is standardised by
subtracting the arithmetic mean and dividing by the root mean squared error.

\begin{Shaded}
\begin{Highlighting}[]
\CommentTok{\# libs {-}{-}{-}{-}}
\ControlFlowTok{if}\NormalTok{(}\SpecialCharTok{!}\FunctionTok{require}\NormalTok{(terra)) \{}\FunctionTok{install.packages}\NormalTok{(}\StringTok{"terra"}\NormalTok{); }\FunctionTok{require}\NormalTok{(terra)\}}
\ControlFlowTok{if}\NormalTok{(}\SpecialCharTok{!}\FunctionTok{require}\NormalTok{(egvtools)) \{remotes}\SpecialCharTok{::}\FunctionTok{install\_github}\NormalTok{(}\StringTok{"aavotins/egvtools"}\NormalTok{); }\FunctionTok{require}\NormalTok{(egvtools)\}}


\CommentTok{\# Templates {-}{-}{-}{-}{-}}
\NormalTok{template100}\OtherTok{=}\FunctionTok{rast}\NormalTok{(}\StringTok{"./Templates/TemplateRasters/LV100m\_10km.tif"}\NormalTok{)}

\CommentTok{\# radii {-}{-}{-}{-}}
\FunctionTok{radius\_function}\NormalTok{(}
 \AttributeTok{kvadrati\_path =} \StringTok{"./Templates/TemplateGrids/tiles/"}\NormalTok{,}
 \AttributeTok{radii\_path   =} \StringTok{"./Templates/TemplateGridPoints/tiles/"}\NormalTok{,}
 \AttributeTok{tikls100\_path =} \StringTok{"./Templates/TemplateGrids/tikls100\_sauzeme.parquet"}\NormalTok{,}
 \AttributeTok{template\_path =} \StringTok{"./Templates/TemplateRasters/LV100m\_10km.tif"}\NormalTok{,}
 \AttributeTok{input\_layers  =} \FunctionTok{c}\NormalTok{(}\StringTok{"./RasterGrids\_100m/2024/RAW/General\_Trees\_cell.tif"}\NormalTok{),}
 \AttributeTok{layer\_prefixes =} \FunctionTok{c}\NormalTok{(}\StringTok{"General\_Trees"}\NormalTok{),}
 \AttributeTok{output\_dir   =} \StringTok{"./RasterGrids\_100m/2024/RAW/"}\NormalTok{,}
 \AttributeTok{n\_workers   =} \DecValTok{6}\NormalTok{,}
 \AttributeTok{radii     =} \FunctionTok{c}\NormalTok{(}\StringTok{"r10000"}\NormalTok{),}
 \AttributeTok{radius\_mode  =} \StringTok{"sparse"}\NormalTok{,}
 \AttributeTok{extract\_fun  =} \StringTok{"mean"}\NormalTok{,}
 \AttributeTok{fill\_missing  =} \ConstantTok{TRUE}\NormalTok{,}
 \AttributeTok{IDW\_weight   =} \DecValTok{2}\NormalTok{,}
 \AttributeTok{future\_max\_size =} \DecValTok{40} \SpecialCharTok{*} \DecValTok{1024}\SpecialCharTok{\^{}}\DecValTok{3}\NormalTok{)}


\CommentTok{\# General\_Trees\_r10000.tif  egv\_453}
\NormalTok{slanis}\OtherTok{=}\FunctionTok{rast}\NormalTok{(}\StringTok{"./RasterGrids\_100m/2024/RAW/General\_Trees\_r10000.tif"}\NormalTok{)}
\FunctionTok{names}\NormalTok{(slanis)}\OtherTok{=}\StringTok{"egv\_453"}
\NormalTok{slanis2}\OtherTok{=}\FunctionTok{project}\NormalTok{(slanis,template100)}
\FunctionTok{writeRaster}\NormalTok{(slanis2,}
      \StringTok{"./RasterGrids\_100m/2024/RAW/General\_Trees\_r10000.tif"}\NormalTok{,}
      \AttributeTok{overwrite=}\ConstantTok{TRUE}\NormalTok{)}

\CommentTok{\# standardisation {-}{-}{-}{-}}
\ControlFlowTok{if}\NormalTok{(}\SpecialCharTok{!}\FunctionTok{require}\NormalTok{(terra)) \{}\FunctionTok{install.packages}\NormalTok{(}\StringTok{"terra"}\NormalTok{); }\FunctionTok{require}\NormalTok{(terra)\}}
\ControlFlowTok{if}\NormalTok{(}\SpecialCharTok{!}\FunctionTok{require}\NormalTok{(tidyverse)) \{}\FunctionTok{install.packages}\NormalTok{(}\StringTok{"tidyverse"}\NormalTok{); }\FunctionTok{require}\NormalTok{(tidyverse)\}}

\NormalTok{nosaukums}\OtherTok{=}\StringTok{"General\_Trees\_r10000.tif"}
\NormalTok{ielasisanas\_cels}\OtherTok{=}\FunctionTok{paste0}\NormalTok{(}\StringTok{"./RasterGrids\_100m/2024/RAW/"}\NormalTok{,nosaukums)}
\NormalTok{saglabasanas\_cels}\OtherTok{=}\FunctionTok{paste0}\NormalTok{(}\StringTok{"./RasterGrids\_100m/2024/Scaled/"}\NormalTok{,nosaukums)}
\NormalTok{slanis}\OtherTok{=}\FunctionTok{rast}\NormalTok{(ielasisanas\_cels)}
\NormalTok{videjais}\OtherTok{=}\FunctionTok{global}\NormalTok{(slanis,}\AttributeTok{fun=}\StringTok{"mean"}\NormalTok{,}\AttributeTok{na.rm=}\ConstantTok{TRUE}\NormalTok{)}
\NormalTok{centrets}\OtherTok{=}\NormalTok{slanis}\SpecialCharTok{{-}}\NormalTok{videjais[,}\DecValTok{1}\NormalTok{]}
\NormalTok{standartnovirze}\OtherTok{=}\NormalTok{terra}\SpecialCharTok{::}\FunctionTok{global}\NormalTok{(centrets,}\AttributeTok{fun=}\StringTok{"rms"}\NormalTok{,}\AttributeTok{na.rm=}\ConstantTok{TRUE}\NormalTok{)}
\NormalTok{merogots}\OtherTok{=}\NormalTok{centrets}\SpecialCharTok{/}\NormalTok{standartnovirze[,}\DecValTok{1}\NormalTok{]}
\FunctionTok{writeRaster}\NormalTok{(merogots,}
      \AttributeTok{filename=}\NormalTok{saglabasanas\_cels,}
      \AttributeTok{overwrite=}\ConstantTok{TRUE}\NormalTok{)}
\end{Highlighting}
\end{Shaded}

\section{General\_TreesOutsideForests\_cell}\label{ch06.454}

\textbf{filename:} \texttt{General\_TreesOutsideForests\_cell.tif}

\textbf{layername:} \texttt{egv\_454}

\textbf{English name:} Fractional cover of Tree covered areas Outside Forests within
the analysis cell (1 ha)

\textbf{Latvian name:} Ar kokiem klāto teritoriju ārpus mežiem platības īpatsvars
analīzes šūnā (1 ha)

\textbf{Procedure:} First, the tree covered areas outside forest stands from the \hyperref[Ch05.03]{Landscape
classification} are selected (value 640 is reclassified to value 1;
all others are set to 0). The resulting layer
is then aggregated to EGV resolution using the workflow \texttt{egvtools::input2egv()}, which
calculates the arithmetic mean to determine the cover fraction. During
aggregation, inverse distance weighted (power = 2) gap filling on the output is
applied to ensure no missing values at the edges. Finally, the layer is
standardised by subtracting the arithmetic mean and dividing by the root mean squared
error.

\begin{Shaded}
\begin{Highlighting}[]
\CommentTok{\# libs {-}{-}{-}{-}}
\ControlFlowTok{if}\NormalTok{(}\SpecialCharTok{!}\FunctionTok{require}\NormalTok{(egvtools)) \{remotes}\SpecialCharTok{::}\FunctionTok{install\_github}\NormalTok{(}\StringTok{"aavotins/egvtools"}\NormalTok{); }\FunctionTok{require}\NormalTok{(egvtools)\}}
\ControlFlowTok{if}\NormalTok{(}\SpecialCharTok{!}\FunctionTok{require}\NormalTok{(terra)) \{}\FunctionTok{install.packages}\NormalTok{(}\StringTok{"terra"}\NormalTok{); }\FunctionTok{require}\NormalTok{(terra)\}}
\ControlFlowTok{if}\NormalTok{(}\SpecialCharTok{!}\FunctionTok{require}\NormalTok{(sf)) \{}\FunctionTok{install.packages}\NormalTok{(}\StringTok{"sf"}\NormalTok{); }\FunctionTok{require}\NormalTok{(sf)\}}
\ControlFlowTok{if}\NormalTok{(}\SpecialCharTok{!}\FunctionTok{require}\NormalTok{(tidyverse)) \{}\FunctionTok{install.packages}\NormalTok{(}\StringTok{"tidyverse"}\NormalTok{); }\FunctionTok{require}\NormalTok{(tidyverse)\}}
\ControlFlowTok{if}\NormalTok{(}\SpecialCharTok{!}\FunctionTok{require}\NormalTok{(sfarrow)) \{}\FunctionTok{install.packages}\NormalTok{(}\StringTok{"sfarrow"}\NormalTok{); }\FunctionTok{require}\NormalTok{(sfarrow)\}}
\ControlFlowTok{if}\NormalTok{(}\SpecialCharTok{!}\FunctionTok{require}\NormalTok{(readxl)) \{}\FunctionTok{install.packages}\NormalTok{(}\StringTok{"readxl"}\NormalTok{); }\FunctionTok{require}\NormalTok{(readxl)\}}
\ControlFlowTok{if}\NormalTok{(}\SpecialCharTok{!}\FunctionTok{require}\NormalTok{(raster)) \{}\FunctionTok{install.packages}\NormalTok{(}\StringTok{"raster"}\NormalTok{); }\FunctionTok{require}\NormalTok{(raster)\}}
\ControlFlowTok{if}\NormalTok{(}\SpecialCharTok{!}\FunctionTok{require}\NormalTok{(fasterize)) \{}\FunctionTok{install.packages}\NormalTok{(}\StringTok{"fasterize"}\NormalTok{); }\FunctionTok{require}\NormalTok{(fasterize)\}}

\CommentTok{\# templates {-}{-}{-}{-}}
\NormalTok{template100}\OtherTok{=}\FunctionTok{rast}\NormalTok{(}\StringTok{"./Templates/TemplateRasters/LV100m\_10km.tif"}\NormalTok{)}
\NormalTok{template10}\OtherTok{=}\FunctionTok{rast}\NormalTok{(}\StringTok{"./Templates/TemplateRasters/LV10m\_10km.tif"}\NormalTok{)}
\NormalTok{rastrs10}\OtherTok{=}\FunctionTok{raster}\NormalTok{(template10)}

\NormalTok{nulls10}\OtherTok{=}\FunctionTok{rast}\NormalTok{(}\StringTok{"./Templates/TemplateRasters/nulls\_LV10m\_10km.tif"}\NormalTok{)}
\NormalTok{nulls100}\OtherTok{=}\FunctionTok{rast}\NormalTok{(}\StringTok{"./Templates/TemplateRasters/nulls\_LV100m\_10km.tif"}\NormalTok{)}

\CommentTok{\# simple landscape {-}{-}{-}{-}}
\NormalTok{simple\_landscape}\OtherTok{=}\FunctionTok{rast}\NormalTok{(}\StringTok{"RasterGrids\_10m/2024/Ainava\_vienk\_mask.tif"}\NormalTok{)}


\CommentTok{\# General\_TreesOutsideForests\_cell.tif  egv\_454 {-}{-}{-}{-}}
\NormalTok{kokiarpuse}\OtherTok{=}\FunctionTok{ifel}\NormalTok{(simple\_landscape}\SpecialCharTok{==}\DecValTok{640}\NormalTok{,}\DecValTok{1}\NormalTok{,}\DecValTok{0}\NormalTok{)}
\NormalTok{i2e\_rez}\OtherTok{=}\NormalTok{egvtools}\SpecialCharTok{::}\FunctionTok{input2egv}\NormalTok{(}\AttributeTok{input=}\NormalTok{kokiarpuse,}
              \AttributeTok{egv\_template=} \StringTok{"./Templates/TemplateRasters/LV100m\_10km.tif"}\NormalTok{,}
              \AttributeTok{summary\_function =} \StringTok{"average"}\NormalTok{,}
              \AttributeTok{missing\_job =} \StringTok{"FillOutput"}\NormalTok{,}
              \AttributeTok{outlocation =} \StringTok{"./RasterGrids\_100m/2024/RAW/"}\NormalTok{,}
              \AttributeTok{outfilename =} \StringTok{"General\_TreesOutsideForests\_cell.tif"}\NormalTok{,}
              \AttributeTok{layername =} \StringTok{"egv\_454"}\NormalTok{,}
              \AttributeTok{idw\_weight =} \DecValTok{2}\NormalTok{,}
              \AttributeTok{plot\_gaps =} \ConstantTok{FALSE}\NormalTok{,}\AttributeTok{plot\_final =} \ConstantTok{TRUE}\NormalTok{)}
\NormalTok{i2e\_rez}
\FunctionTok{rm}\NormalTok{(kokiarpuse)}
\FunctionTok{rm}\NormalTok{(i2e\_rez)}

\CommentTok{\# standardisation {-}{-}{-}{-}}
\ControlFlowTok{if}\NormalTok{(}\SpecialCharTok{!}\FunctionTok{require}\NormalTok{(terra)) \{}\FunctionTok{install.packages}\NormalTok{(}\StringTok{"terra"}\NormalTok{); }\FunctionTok{require}\NormalTok{(terra)\}}
\ControlFlowTok{if}\NormalTok{(}\SpecialCharTok{!}\FunctionTok{require}\NormalTok{(tidyverse)) \{}\FunctionTok{install.packages}\NormalTok{(}\StringTok{"tidyverse"}\NormalTok{); }\FunctionTok{require}\NormalTok{(tidyverse)\}}

\NormalTok{nosaukums}\OtherTok{=}\StringTok{"General\_TreesOutsideForests\_cell.tif"}
\NormalTok{ielasisanas\_cels}\OtherTok{=}\FunctionTok{paste0}\NormalTok{(}\StringTok{"./RasterGrids\_100m/2024/RAW/"}\NormalTok{,nosaukums)}
\NormalTok{saglabasanas\_cels}\OtherTok{=}\FunctionTok{paste0}\NormalTok{(}\StringTok{"./RasterGrids\_100m/2024/Scaled/"}\NormalTok{,nosaukums)}
\NormalTok{slanis}\OtherTok{=}\FunctionTok{rast}\NormalTok{(ielasisanas\_cels)}
\NormalTok{videjais}\OtherTok{=}\FunctionTok{global}\NormalTok{(slanis,}\AttributeTok{fun=}\StringTok{"mean"}\NormalTok{,}\AttributeTok{na.rm=}\ConstantTok{TRUE}\NormalTok{)}
\NormalTok{centrets}\OtherTok{=}\NormalTok{slanis}\SpecialCharTok{{-}}\NormalTok{videjais[,}\DecValTok{1}\NormalTok{]}
\NormalTok{standartnovirze}\OtherTok{=}\NormalTok{terra}\SpecialCharTok{::}\FunctionTok{global}\NormalTok{(centrets,}\AttributeTok{fun=}\StringTok{"rms"}\NormalTok{,}\AttributeTok{na.rm=}\ConstantTok{TRUE}\NormalTok{)}
\NormalTok{merogots}\OtherTok{=}\NormalTok{centrets}\SpecialCharTok{/}\NormalTok{standartnovirze[,}\DecValTok{1}\NormalTok{]}
\FunctionTok{writeRaster}\NormalTok{(merogots,}
      \AttributeTok{filename=}\NormalTok{saglabasanas\_cels,}
      \AttributeTok{overwrite=}\ConstantTok{TRUE}\NormalTok{)}
\end{Highlighting}
\end{Shaded}

\section{General\_TreesOutsideForests\_r500}\label{ch06.455}

\textbf{filename:} \texttt{General\_TreesOutsideForests\_r500.tif}

\textbf{layername:} \texttt{egv\_455}

\textbf{English name:} Fractional cover of Tree covered areas Outside Forests within
the 0.5 km landscape

\textbf{Latvian name:} Ar kokiem klāto teritoriju ārpus mežiem platības īpatsvars 0,5
km ainavā

\textbf{Procedure:} The cover fraction within a radius of 500 m around the analysis grid cell is
calculated as the area-weighted sum of the \hyperref[ch06.454]{analysis cells} inside the
buffer, using the workflow \texttt{egvtools::radius\_function()}. During the calculation of the landscape metric,
inverse distance weighted (power = 2) gap filling on the output is applied
to ensure no missing values at the edges. Then the layer is rewritten to set
its name. Finally, the layer is standardised by subtracting the arithmetic
mean and dividing by the root mean squared error.

\begin{Shaded}
\begin{Highlighting}[]
\CommentTok{\# libs {-}{-}{-}{-}}
\ControlFlowTok{if}\NormalTok{(}\SpecialCharTok{!}\FunctionTok{require}\NormalTok{(terra)) \{}\FunctionTok{install.packages}\NormalTok{(}\StringTok{"terra"}\NormalTok{); }\FunctionTok{require}\NormalTok{(terra)\}}
\ControlFlowTok{if}\NormalTok{(}\SpecialCharTok{!}\FunctionTok{require}\NormalTok{(egvtools)) \{remotes}\SpecialCharTok{::}\FunctionTok{install\_github}\NormalTok{(}\StringTok{"aavotins/egvtools"}\NormalTok{); }\FunctionTok{require}\NormalTok{(egvtools)\}}


\CommentTok{\# Templates {-}{-}{-}{-}{-}}
\NormalTok{template100}\OtherTok{=}\FunctionTok{rast}\NormalTok{(}\StringTok{"./Templates/TemplateRasters/LV100m\_10km.tif"}\NormalTok{)}

\CommentTok{\# radii {-}{-}{-}{-}}
\FunctionTok{radius\_function}\NormalTok{(}
 \AttributeTok{kvadrati\_path =} \StringTok{"./Templates/TemplateGrids/tiles/"}\NormalTok{,}
 \AttributeTok{radii\_path   =} \StringTok{"./Templates/TemplateGridPoints/tiles/"}\NormalTok{,}
 \AttributeTok{tikls100\_path =} \StringTok{"./Templates/TemplateGrids/tikls100\_sauzeme.parquet"}\NormalTok{,}
 \AttributeTok{template\_path =} \StringTok{"./Templates/TemplateRasters/LV100m\_10km.tif"}\NormalTok{,}
 \AttributeTok{input\_layers  =} \FunctionTok{c}\NormalTok{(}\StringTok{"./RasterGrids\_100m/2024/RAW/General\_TreesOutsideForests\_cell.tif"}\NormalTok{),}
 \AttributeTok{layer\_prefixes =} \FunctionTok{c}\NormalTok{(}\StringTok{"General\_TreesOutsideForests"}\NormalTok{),}
 \AttributeTok{output\_dir   =} \StringTok{"./RasterGrids\_100m/2024/RAW/"}\NormalTok{,}
 \AttributeTok{n\_workers   =} \DecValTok{6}\NormalTok{,}
 \AttributeTok{radii     =} \FunctionTok{c}\NormalTok{(}\StringTok{"r500"}\NormalTok{),}
 \AttributeTok{radius\_mode  =} \StringTok{"sparse"}\NormalTok{,}
 \AttributeTok{extract\_fun  =} \StringTok{"mean"}\NormalTok{,}
 \AttributeTok{fill\_missing  =} \ConstantTok{TRUE}\NormalTok{,}
 \AttributeTok{IDW\_weight   =} \DecValTok{2}\NormalTok{,}
 \AttributeTok{future\_max\_size =} \DecValTok{40} \SpecialCharTok{*} \DecValTok{1024}\SpecialCharTok{\^{}}\DecValTok{3}\NormalTok{)}


\CommentTok{\# General\_TreesOutsideForests\_r500.tif  egv\_455}
\NormalTok{slanis}\OtherTok{=}\FunctionTok{rast}\NormalTok{(}\StringTok{"./RasterGrids\_100m/2024/RAW/General\_TreesOutsideForests\_r500.tif"}\NormalTok{)}
\FunctionTok{names}\NormalTok{(slanis)}\OtherTok{=}\StringTok{"egv\_455"}
\NormalTok{slanis2}\OtherTok{=}\FunctionTok{project}\NormalTok{(slanis,template100)}
\FunctionTok{writeRaster}\NormalTok{(slanis2,}
      \StringTok{"./RasterGrids\_100m/2024/RAW/General\_TreesOutsideForests\_r500.tif"}\NormalTok{,}
      \AttributeTok{overwrite=}\ConstantTok{TRUE}\NormalTok{)}

\CommentTok{\# standardisation {-}{-}{-}{-}}
\ControlFlowTok{if}\NormalTok{(}\SpecialCharTok{!}\FunctionTok{require}\NormalTok{(terra)) \{}\FunctionTok{install.packages}\NormalTok{(}\StringTok{"terra"}\NormalTok{); }\FunctionTok{require}\NormalTok{(terra)\}}
\ControlFlowTok{if}\NormalTok{(}\SpecialCharTok{!}\FunctionTok{require}\NormalTok{(tidyverse)) \{}\FunctionTok{install.packages}\NormalTok{(}\StringTok{"tidyverse"}\NormalTok{); }\FunctionTok{require}\NormalTok{(tidyverse)\}}

\NormalTok{nosaukums}\OtherTok{=}\StringTok{"General\_TreesOutsideForests\_r500.tif"}
\NormalTok{ielasisanas\_cels}\OtherTok{=}\FunctionTok{paste0}\NormalTok{(}\StringTok{"./RasterGrids\_100m/2024/RAW/"}\NormalTok{,nosaukums)}
\NormalTok{saglabasanas\_cels}\OtherTok{=}\FunctionTok{paste0}\NormalTok{(}\StringTok{"./RasterGrids\_100m/2024/Scaled/"}\NormalTok{,nosaukums)}
\NormalTok{slanis}\OtherTok{=}\FunctionTok{rast}\NormalTok{(ielasisanas\_cels)}
\NormalTok{videjais}\OtherTok{=}\FunctionTok{global}\NormalTok{(slanis,}\AttributeTok{fun=}\StringTok{"mean"}\NormalTok{,}\AttributeTok{na.rm=}\ConstantTok{TRUE}\NormalTok{)}
\NormalTok{centrets}\OtherTok{=}\NormalTok{slanis}\SpecialCharTok{{-}}\NormalTok{videjais[,}\DecValTok{1}\NormalTok{]}
\NormalTok{standartnovirze}\OtherTok{=}\NormalTok{terra}\SpecialCharTok{::}\FunctionTok{global}\NormalTok{(centrets,}\AttributeTok{fun=}\StringTok{"rms"}\NormalTok{,}\AttributeTok{na.rm=}\ConstantTok{TRUE}\NormalTok{)}
\NormalTok{merogots}\OtherTok{=}\NormalTok{centrets}\SpecialCharTok{/}\NormalTok{standartnovirze[,}\DecValTok{1}\NormalTok{]}
\FunctionTok{writeRaster}\NormalTok{(merogots,}
      \AttributeTok{filename=}\NormalTok{saglabasanas\_cels,}
      \AttributeTok{overwrite=}\ConstantTok{TRUE}\NormalTok{)}
\end{Highlighting}
\end{Shaded}

\section{General\_TreesOutsideForests\_r1250}\label{ch06.456}

\textbf{filename:} \texttt{General\_TreesOutsideForests\_r1250.tif}

\textbf{layername:} \texttt{egv\_456}

\textbf{English name:} Fractional cover of Tree covered areas Outside Forests within
the 1.25 km landscape

\textbf{Latvian name:} Ar kokiem klāto teritoriju ārpus mežiem platības īpatsvars
1,25 km ainavā

\textbf{Procedure:} The cover fraction within a radius of 1250 m around the analysis grid cell
is calculated as the area-weighted sum of the \hyperref[ch06.454]{analysis cells} inside
the buffer, using the workflow \texttt{egvtools::radius\_function()}. During the calculation of the landscape
metric, inverse distance weighted (power = 2) gap filling on the output is
applied to ensure no missing values at the edges. Then the layer is
rewritten to set its name. Finally, the layer is standardised by
subtracting the arithmetic mean and dividing by the root mean squared error.

\begin{Shaded}
\begin{Highlighting}[]
\CommentTok{\# libs {-}{-}{-}{-}}
\ControlFlowTok{if}\NormalTok{(}\SpecialCharTok{!}\FunctionTok{require}\NormalTok{(terra)) \{}\FunctionTok{install.packages}\NormalTok{(}\StringTok{"terra"}\NormalTok{); }\FunctionTok{require}\NormalTok{(terra)\}}
\ControlFlowTok{if}\NormalTok{(}\SpecialCharTok{!}\FunctionTok{require}\NormalTok{(egvtools)) \{remotes}\SpecialCharTok{::}\FunctionTok{install\_github}\NormalTok{(}\StringTok{"aavotins/egvtools"}\NormalTok{); }\FunctionTok{require}\NormalTok{(egvtools)\}}


\CommentTok{\# Templates {-}{-}{-}{-}{-}}
\NormalTok{template100}\OtherTok{=}\FunctionTok{rast}\NormalTok{(}\StringTok{"./Templates/TemplateRasters/LV100m\_10km.tif"}\NormalTok{)}

\CommentTok{\# radii {-}{-}{-}{-}}
\FunctionTok{radius\_function}\NormalTok{(}
 \AttributeTok{kvadrati\_path =} \StringTok{"./Templates/TemplateGrids/tiles/"}\NormalTok{,}
 \AttributeTok{radii\_path   =} \StringTok{"./Templates/TemplateGridPoints/tiles/"}\NormalTok{,}
 \AttributeTok{tikls100\_path =} \StringTok{"./Templates/TemplateGrids/tikls100\_sauzeme.parquet"}\NormalTok{,}
 \AttributeTok{template\_path =} \StringTok{"./Templates/TemplateRasters/LV100m\_10km.tif"}\NormalTok{,}
 \AttributeTok{input\_layers  =} \FunctionTok{c}\NormalTok{(}\StringTok{"./RasterGrids\_100m/2024/RAW/General\_TreesOutsideForests\_cell.tif"}\NormalTok{),}
 \AttributeTok{layer\_prefixes =} \FunctionTok{c}\NormalTok{(}\StringTok{"General\_TreesOutsideForests"}\NormalTok{),}
 \AttributeTok{output\_dir   =} \StringTok{"./RasterGrids\_100m/2024/RAW/"}\NormalTok{,}
 \AttributeTok{n\_workers   =} \DecValTok{6}\NormalTok{,}
 \AttributeTok{radii     =} \FunctionTok{c}\NormalTok{(}\StringTok{"r1250"}\NormalTok{),}
 \AttributeTok{radius\_mode  =} \StringTok{"sparse"}\NormalTok{,}
 \AttributeTok{extract\_fun  =} \StringTok{"mean"}\NormalTok{,}
 \AttributeTok{fill\_missing  =} \ConstantTok{TRUE}\NormalTok{,}
 \AttributeTok{IDW\_weight   =} \DecValTok{2}\NormalTok{,}
 \AttributeTok{future\_max\_size =} \DecValTok{40} \SpecialCharTok{*} \DecValTok{1024}\SpecialCharTok{\^{}}\DecValTok{3}\NormalTok{)}


\CommentTok{\# General\_TreesOutsideForests\_r1250.tif egv\_456}
\NormalTok{slanis}\OtherTok{=}\FunctionTok{rast}\NormalTok{(}\StringTok{"./RasterGrids\_100m/2024/RAW/General\_TreesOutsideForests\_r1250.tif"}\NormalTok{)}
\FunctionTok{names}\NormalTok{(slanis)}\OtherTok{=}\StringTok{"egv\_456"}
\NormalTok{slanis2}\OtherTok{=}\FunctionTok{project}\NormalTok{(slanis,template100)}
\FunctionTok{writeRaster}\NormalTok{(slanis2,}
      \StringTok{"./RasterGrids\_100m/2024/RAW/General\_TreesOutsideForests\_r1250.tif"}\NormalTok{,}
      \AttributeTok{overwrite=}\ConstantTok{TRUE}\NormalTok{)}

\CommentTok{\# standardisation {-}{-}{-}{-}}
\ControlFlowTok{if}\NormalTok{(}\SpecialCharTok{!}\FunctionTok{require}\NormalTok{(terra)) \{}\FunctionTok{install.packages}\NormalTok{(}\StringTok{"terra"}\NormalTok{); }\FunctionTok{require}\NormalTok{(terra)\}}
\ControlFlowTok{if}\NormalTok{(}\SpecialCharTok{!}\FunctionTok{require}\NormalTok{(tidyverse)) \{}\FunctionTok{install.packages}\NormalTok{(}\StringTok{"tidyverse"}\NormalTok{); }\FunctionTok{require}\NormalTok{(tidyverse)\}}

\NormalTok{nosaukums}\OtherTok{=}\StringTok{"General\_TreesOutsideForests\_r1250.tif"}
\NormalTok{ielasisanas\_cels}\OtherTok{=}\FunctionTok{paste0}\NormalTok{(}\StringTok{"./RasterGrids\_100m/2024/RAW/"}\NormalTok{,nosaukums)}
\NormalTok{saglabasanas\_cels}\OtherTok{=}\FunctionTok{paste0}\NormalTok{(}\StringTok{"./RasterGrids\_100m/2024/Scaled/"}\NormalTok{,nosaukums)}
\NormalTok{slanis}\OtherTok{=}\FunctionTok{rast}\NormalTok{(ielasisanas\_cels)}
\NormalTok{videjais}\OtherTok{=}\FunctionTok{global}\NormalTok{(slanis,}\AttributeTok{fun=}\StringTok{"mean"}\NormalTok{,}\AttributeTok{na.rm=}\ConstantTok{TRUE}\NormalTok{)}
\NormalTok{centrets}\OtherTok{=}\NormalTok{slanis}\SpecialCharTok{{-}}\NormalTok{videjais[,}\DecValTok{1}\NormalTok{]}
\NormalTok{standartnovirze}\OtherTok{=}\NormalTok{terra}\SpecialCharTok{::}\FunctionTok{global}\NormalTok{(centrets,}\AttributeTok{fun=}\StringTok{"rms"}\NormalTok{,}\AttributeTok{na.rm=}\ConstantTok{TRUE}\NormalTok{)}
\NormalTok{merogots}\OtherTok{=}\NormalTok{centrets}\SpecialCharTok{/}\NormalTok{standartnovirze[,}\DecValTok{1}\NormalTok{]}
\FunctionTok{writeRaster}\NormalTok{(merogots,}
      \AttributeTok{filename=}\NormalTok{saglabasanas\_cels,}
      \AttributeTok{overwrite=}\ConstantTok{TRUE}\NormalTok{)}
\end{Highlighting}
\end{Shaded}

\section{General\_TreesOutsideForests\_r3000}\label{ch06.457}

\textbf{filename:} \texttt{General\_TreesOutsideForests\_r3000.tif}

\textbf{layername:} \texttt{egv\_457}

\textbf{English name:} Fractional cover of Tree covered areas Outside Forests within
the 3 km landscape

\textbf{Latvian name:} Ar kokiem klāto teritoriju ārpus mežiem platības īpatsvars 3
km ainavā

\textbf{Procedure:} The cover fraction within a radius of 3000 m around the analysis grid cell
is calculated as the area-weighted sum of the \hyperref[ch06.454]{analysis cells} inside
the buffer, using the workflow \texttt{egvtools::radius\_function()}. During the calculation of the landscape
metric, inverse distance weighted (power = 2) gap filling on the output is
applied to ensure no missing values at the edges. Then the layer is
rewritten to set its name. Finally, the layer is standardised by
subtracting the arithmetic mean and dividing by the root mean squared error.

\begin{Shaded}
\begin{Highlighting}[]
\CommentTok{\# libs {-}{-}{-}{-}}
\ControlFlowTok{if}\NormalTok{(}\SpecialCharTok{!}\FunctionTok{require}\NormalTok{(terra)) \{}\FunctionTok{install.packages}\NormalTok{(}\StringTok{"terra"}\NormalTok{); }\FunctionTok{require}\NormalTok{(terra)\}}
\ControlFlowTok{if}\NormalTok{(}\SpecialCharTok{!}\FunctionTok{require}\NormalTok{(egvtools)) \{remotes}\SpecialCharTok{::}\FunctionTok{install\_github}\NormalTok{(}\StringTok{"aavotins/egvtools"}\NormalTok{); }\FunctionTok{require}\NormalTok{(egvtools)\}}


\CommentTok{\# Templates {-}{-}{-}{-}{-}}
\NormalTok{template100}\OtherTok{=}\FunctionTok{rast}\NormalTok{(}\StringTok{"./Templates/TemplateRasters/LV100m\_10km.tif"}\NormalTok{)}

\CommentTok{\# radii {-}{-}{-}{-}}
\FunctionTok{radius\_function}\NormalTok{(}
 \AttributeTok{kvadrati\_path =} \StringTok{"./Templates/TemplateGrids/tiles/"}\NormalTok{,}
 \AttributeTok{radii\_path   =} \StringTok{"./Templates/TemplateGridPoints/tiles/"}\NormalTok{,}
 \AttributeTok{tikls100\_path =} \StringTok{"./Templates/TemplateGrids/tikls100\_sauzeme.parquet"}\NormalTok{,}
 \AttributeTok{template\_path =} \StringTok{"./Templates/TemplateRasters/LV100m\_10km.tif"}\NormalTok{,}
 \AttributeTok{input\_layers  =} \FunctionTok{c}\NormalTok{(}\StringTok{"./RasterGrids\_100m/2024/RAW/General\_TreesOutsideForests\_cell.tif"}\NormalTok{),}
 \AttributeTok{layer\_prefixes =} \FunctionTok{c}\NormalTok{(}\StringTok{"General\_TreesOutsideForests"}\NormalTok{),}
 \AttributeTok{output\_dir   =} \StringTok{"./RasterGrids\_100m/2024/RAW/"}\NormalTok{,}
 \AttributeTok{n\_workers   =} \DecValTok{6}\NormalTok{,}
 \AttributeTok{radii     =} \FunctionTok{c}\NormalTok{(}\StringTok{"r3000"}\NormalTok{),}
 \AttributeTok{radius\_mode  =} \StringTok{"sparse"}\NormalTok{,}
 \AttributeTok{extract\_fun  =} \StringTok{"mean"}\NormalTok{,}
 \AttributeTok{fill\_missing  =} \ConstantTok{TRUE}\NormalTok{,}
 \AttributeTok{IDW\_weight   =} \DecValTok{2}\NormalTok{,}
 \AttributeTok{future\_max\_size =} \DecValTok{40} \SpecialCharTok{*} \DecValTok{1024}\SpecialCharTok{\^{}}\DecValTok{3}\NormalTok{)}


\CommentTok{\# General\_TreesOutsideForests\_r3000.tif egv\_457}
\NormalTok{slanis}\OtherTok{=}\FunctionTok{rast}\NormalTok{(}\StringTok{"./RasterGrids\_100m/2024/RAW/General\_TreesOutsideForests\_r3000.tif"}\NormalTok{)}
\FunctionTok{names}\NormalTok{(slanis)}\OtherTok{=}\StringTok{"egv\_457"}
\NormalTok{slanis2}\OtherTok{=}\FunctionTok{project}\NormalTok{(slanis,template100)}
\FunctionTok{writeRaster}\NormalTok{(slanis2,}
      \StringTok{"./RasterGrids\_100m/2024/RAW/General\_TreesOutsideForests\_r3000.tif"}\NormalTok{,}
      \AttributeTok{overwrite=}\ConstantTok{TRUE}\NormalTok{)}

\CommentTok{\# standardisation {-}{-}{-}{-}}
\ControlFlowTok{if}\NormalTok{(}\SpecialCharTok{!}\FunctionTok{require}\NormalTok{(terra)) \{}\FunctionTok{install.packages}\NormalTok{(}\StringTok{"terra"}\NormalTok{); }\FunctionTok{require}\NormalTok{(terra)\}}
\ControlFlowTok{if}\NormalTok{(}\SpecialCharTok{!}\FunctionTok{require}\NormalTok{(tidyverse)) \{}\FunctionTok{install.packages}\NormalTok{(}\StringTok{"tidyverse"}\NormalTok{); }\FunctionTok{require}\NormalTok{(tidyverse)\}}

\NormalTok{nosaukums}\OtherTok{=}\StringTok{"General\_TreesOutsideForests\_r3000.tif"}
\NormalTok{ielasisanas\_cels}\OtherTok{=}\FunctionTok{paste0}\NormalTok{(}\StringTok{"./RasterGrids\_100m/2024/RAW/"}\NormalTok{,nosaukums)}
\NormalTok{saglabasanas\_cels}\OtherTok{=}\FunctionTok{paste0}\NormalTok{(}\StringTok{"./RasterGrids\_100m/2024/Scaled/"}\NormalTok{,nosaukums)}
\NormalTok{slanis}\OtherTok{=}\FunctionTok{rast}\NormalTok{(ielasisanas\_cels)}
\NormalTok{videjais}\OtherTok{=}\FunctionTok{global}\NormalTok{(slanis,}\AttributeTok{fun=}\StringTok{"mean"}\NormalTok{,}\AttributeTok{na.rm=}\ConstantTok{TRUE}\NormalTok{)}
\NormalTok{centrets}\OtherTok{=}\NormalTok{slanis}\SpecialCharTok{{-}}\NormalTok{videjais[,}\DecValTok{1}\NormalTok{]}
\NormalTok{standartnovirze}\OtherTok{=}\NormalTok{terra}\SpecialCharTok{::}\FunctionTok{global}\NormalTok{(centrets,}\AttributeTok{fun=}\StringTok{"rms"}\NormalTok{,}\AttributeTok{na.rm=}\ConstantTok{TRUE}\NormalTok{)}
\NormalTok{merogots}\OtherTok{=}\NormalTok{centrets}\SpecialCharTok{/}\NormalTok{standartnovirze[,}\DecValTok{1}\NormalTok{]}
\FunctionTok{writeRaster}\NormalTok{(merogots,}
      \AttributeTok{filename=}\NormalTok{saglabasanas\_cels,}
      \AttributeTok{overwrite=}\ConstantTok{TRUE}\NormalTok{)}
\end{Highlighting}
\end{Shaded}

\section{General\_TreesOutsideForests\_r10000}\label{ch06.458}

\textbf{filename:} \texttt{General\_TreesOutsideForests\_r10000.tif}

\textbf{layername:} \texttt{egv\_458}

\textbf{English name:} Fractional cover of Tree covered areas Outside Forests within
the 10 km landscape

\textbf{Latvian name:} Ar kokiem klāto teritoriju ārpus mežiem platības īpatsvars 10
km ainavā

\textbf{Procedure:} The cover fraction within a radius of 10000 m around the analysis grid cell
is calculated as the area-weighted sum of the \hyperref[ch06.454]{analysis cells} inside
the buffer, using the workflow \texttt{egvtools::radius\_function()}. During the calculation of the landscape
metric, inverse distance weighted (power = 2) gap filling on the output is
applied to ensure no missing values at the edges. Then the layer is
rewritten to set its name. Finally, the layer is standardised by
subtracting the arithmetic mean and dividing by the root mean squared error.

\begin{Shaded}
\begin{Highlighting}[]
\CommentTok{\# libs {-}{-}{-}{-}}
\ControlFlowTok{if}\NormalTok{(}\SpecialCharTok{!}\FunctionTok{require}\NormalTok{(terra)) \{}\FunctionTok{install.packages}\NormalTok{(}\StringTok{"terra"}\NormalTok{); }\FunctionTok{require}\NormalTok{(terra)\}}
\ControlFlowTok{if}\NormalTok{(}\SpecialCharTok{!}\FunctionTok{require}\NormalTok{(egvtools)) \{remotes}\SpecialCharTok{::}\FunctionTok{install\_github}\NormalTok{(}\StringTok{"aavotins/egvtools"}\NormalTok{); }\FunctionTok{require}\NormalTok{(egvtools)\}}


\CommentTok{\# Templates {-}{-}{-}{-}{-}}
\NormalTok{template100}\OtherTok{=}\FunctionTok{rast}\NormalTok{(}\StringTok{"./Templates/TemplateRasters/LV100m\_10km.tif"}\NormalTok{)}

\CommentTok{\# radii {-}{-}{-}{-}}
\FunctionTok{radius\_function}\NormalTok{(}
 \AttributeTok{kvadrati\_path =} \StringTok{"./Templates/TemplateGrids/tiles/"}\NormalTok{,}
 \AttributeTok{radii\_path   =} \StringTok{"./Templates/TemplateGridPoints/tiles/"}\NormalTok{,}
 \AttributeTok{tikls100\_path =} \StringTok{"./Templates/TemplateGrids/tikls100\_sauzeme.parquet"}\NormalTok{,}
 \AttributeTok{template\_path =} \StringTok{"./Templates/TemplateRasters/LV100m\_10km.tif"}\NormalTok{,}
 \AttributeTok{input\_layers  =} \FunctionTok{c}\NormalTok{(}\StringTok{"./RasterGrids\_100m/2024/RAW/General\_TreesOutsideForests\_cell.tif"}\NormalTok{),}
 \AttributeTok{layer\_prefixes =} \FunctionTok{c}\NormalTok{(}\StringTok{"General\_TreesOutsideForests"}\NormalTok{),}
 \AttributeTok{output\_dir   =} \StringTok{"./RasterGrids\_100m/2024/RAW/"}\NormalTok{,}
 \AttributeTok{n\_workers   =} \DecValTok{6}\NormalTok{,}
 \AttributeTok{radii     =} \FunctionTok{c}\NormalTok{(}\StringTok{"r10000"}\NormalTok{),}
 \AttributeTok{radius\_mode  =} \StringTok{"sparse"}\NormalTok{,}
 \AttributeTok{extract\_fun  =} \StringTok{"mean"}\NormalTok{,}
 \AttributeTok{fill\_missing  =} \ConstantTok{TRUE}\NormalTok{,}
 \AttributeTok{IDW\_weight   =} \DecValTok{2}\NormalTok{,}
 \AttributeTok{future\_max\_size =} \DecValTok{40} \SpecialCharTok{*} \DecValTok{1024}\SpecialCharTok{\^{}}\DecValTok{3}\NormalTok{)}


\CommentTok{\# General\_TreesOutsideForests\_r10000.tif    egv\_458}
\NormalTok{slanis}\OtherTok{=}\FunctionTok{rast}\NormalTok{(}\StringTok{"./RasterGrids\_100m/2024/RAW/General\_TreesOutsideForests\_r10000.tif"}\NormalTok{)}
\FunctionTok{names}\NormalTok{(slanis)}\OtherTok{=}\StringTok{"egv\_458"}
\NormalTok{slanis2}\OtherTok{=}\FunctionTok{project}\NormalTok{(slanis,template100)}
\FunctionTok{writeRaster}\NormalTok{(slanis2,}
      \StringTok{"./RasterGrids\_100m/2024/RAW/General\_TreesOutsideForests\_r10000.tif"}\NormalTok{,}
      \AttributeTok{overwrite=}\ConstantTok{TRUE}\NormalTok{)}

\CommentTok{\# standardisation {-}{-}{-}{-}}
\ControlFlowTok{if}\NormalTok{(}\SpecialCharTok{!}\FunctionTok{require}\NormalTok{(terra)) \{}\FunctionTok{install.packages}\NormalTok{(}\StringTok{"terra"}\NormalTok{); }\FunctionTok{require}\NormalTok{(terra)\}}
\ControlFlowTok{if}\NormalTok{(}\SpecialCharTok{!}\FunctionTok{require}\NormalTok{(tidyverse)) \{}\FunctionTok{install.packages}\NormalTok{(}\StringTok{"tidyverse"}\NormalTok{); }\FunctionTok{require}\NormalTok{(tidyverse)\}}

\NormalTok{nosaukums}\OtherTok{=}\StringTok{"General\_TreesOutsideForests\_r10000.tif"}
\NormalTok{ielasisanas\_cels}\OtherTok{=}\FunctionTok{paste0}\NormalTok{(}\StringTok{"./RasterGrids\_100m/2024/RAW/"}\NormalTok{,nosaukums)}
\NormalTok{saglabasanas\_cels}\OtherTok{=}\FunctionTok{paste0}\NormalTok{(}\StringTok{"./RasterGrids\_100m/2024/Scaled/"}\NormalTok{,nosaukums)}
\NormalTok{slanis}\OtherTok{=}\FunctionTok{rast}\NormalTok{(ielasisanas\_cels)}
\NormalTok{videjais}\OtherTok{=}\FunctionTok{global}\NormalTok{(slanis,}\AttributeTok{fun=}\StringTok{"mean"}\NormalTok{,}\AttributeTok{na.rm=}\ConstantTok{TRUE}\NormalTok{)}
\NormalTok{centrets}\OtherTok{=}\NormalTok{slanis}\SpecialCharTok{{-}}\NormalTok{videjais[,}\DecValTok{1}\NormalTok{]}
\NormalTok{standartnovirze}\OtherTok{=}\NormalTok{terra}\SpecialCharTok{::}\FunctionTok{global}\NormalTok{(centrets,}\AttributeTok{fun=}\StringTok{"rms"}\NormalTok{,}\AttributeTok{na.rm=}\ConstantTok{TRUE}\NormalTok{)}
\NormalTok{merogots}\OtherTok{=}\NormalTok{centrets}\SpecialCharTok{/}\NormalTok{standartnovirze[,}\DecValTok{1}\NormalTok{]}
\FunctionTok{writeRaster}\NormalTok{(merogots,}
      \AttributeTok{filename=}\NormalTok{saglabasanas\_cels,}
      \AttributeTok{overwrite=}\ConstantTok{TRUE}\NormalTok{)}
\end{Highlighting}
\end{Shaded}

\section{General\_Water\_cell}\label{ch06.459}

\textbf{filename:} \texttt{General\_Water\_cell.tif}

\textbf{layername:} \texttt{egv\_459}

\textbf{English name:} Fractional cover of Waterbodies within the analysis cell (1
ha)

\textbf{Latvian name:} Ūdenstilpju platības īpatsvars analīzes šūnā (1 ha)

\textbf{Procedure:} First, the waters from the \hyperref[Ch05.03]{Landscape classification} are
selected (value 200 is reclassified to value 1; all others are set to 0). The resulting layer
is then aggregated to EGV resolution using the workflow \texttt{egvtools::input2egv()}, which
calculates the arithmetic mean to determine the cover fraction. During
aggregation, inverse distance weighted (power = 2) gap filling on the output is
applied to ensure no missing values at the edges. Finally, the layer is
standardised by subtracting the arithmetic mean and dividing by the root mean squared
error.

\begin{Shaded}
\begin{Highlighting}[]
\CommentTok{\# libs {-}{-}{-}{-}}
\ControlFlowTok{if}\NormalTok{(}\SpecialCharTok{!}\FunctionTok{require}\NormalTok{(egvtools)) \{remotes}\SpecialCharTok{::}\FunctionTok{install\_github}\NormalTok{(}\StringTok{"aavotins/egvtools"}\NormalTok{); }\FunctionTok{require}\NormalTok{(egvtools)\}}
\ControlFlowTok{if}\NormalTok{(}\SpecialCharTok{!}\FunctionTok{require}\NormalTok{(terra)) \{}\FunctionTok{install.packages}\NormalTok{(}\StringTok{"terra"}\NormalTok{); }\FunctionTok{require}\NormalTok{(terra)\}}
\ControlFlowTok{if}\NormalTok{(}\SpecialCharTok{!}\FunctionTok{require}\NormalTok{(sf)) \{}\FunctionTok{install.packages}\NormalTok{(}\StringTok{"sf"}\NormalTok{); }\FunctionTok{require}\NormalTok{(sf)\}}
\ControlFlowTok{if}\NormalTok{(}\SpecialCharTok{!}\FunctionTok{require}\NormalTok{(tidyverse)) \{}\FunctionTok{install.packages}\NormalTok{(}\StringTok{"tidyverse"}\NormalTok{); }\FunctionTok{require}\NormalTok{(tidyverse)\}}
\ControlFlowTok{if}\NormalTok{(}\SpecialCharTok{!}\FunctionTok{require}\NormalTok{(sfarrow)) \{}\FunctionTok{install.packages}\NormalTok{(}\StringTok{"sfarrow"}\NormalTok{); }\FunctionTok{require}\NormalTok{(sfarrow)\}}
\ControlFlowTok{if}\NormalTok{(}\SpecialCharTok{!}\FunctionTok{require}\NormalTok{(readxl)) \{}\FunctionTok{install.packages}\NormalTok{(}\StringTok{"readxl"}\NormalTok{); }\FunctionTok{require}\NormalTok{(readxl)\}}
\ControlFlowTok{if}\NormalTok{(}\SpecialCharTok{!}\FunctionTok{require}\NormalTok{(raster)) \{}\FunctionTok{install.packages}\NormalTok{(}\StringTok{"raster"}\NormalTok{); }\FunctionTok{require}\NormalTok{(raster)\}}
\ControlFlowTok{if}\NormalTok{(}\SpecialCharTok{!}\FunctionTok{require}\NormalTok{(fasterize)) \{}\FunctionTok{install.packages}\NormalTok{(}\StringTok{"fasterize"}\NormalTok{); }\FunctionTok{require}\NormalTok{(fasterize)\}}

\CommentTok{\# templates {-}{-}{-}{-}}
\NormalTok{template100}\OtherTok{=}\FunctionTok{rast}\NormalTok{(}\StringTok{"./Templates/TemplateRasters/LV100m\_10km.tif"}\NormalTok{)}
\NormalTok{template10}\OtherTok{=}\FunctionTok{rast}\NormalTok{(}\StringTok{"./Templates/TemplateRasters/LV10m\_10km.tif"}\NormalTok{)}
\NormalTok{rastrs10}\OtherTok{=}\FunctionTok{raster}\NormalTok{(template10)}

\NormalTok{nulls10}\OtherTok{=}\FunctionTok{rast}\NormalTok{(}\StringTok{"./Templates/TemplateRasters/nulls\_LV10m\_10km.tif"}\NormalTok{)}
\NormalTok{nulls100}\OtherTok{=}\FunctionTok{rast}\NormalTok{(}\StringTok{"./Templates/TemplateRasters/nulls\_LV100m\_10km.tif"}\NormalTok{)}

\CommentTok{\# simple landscape {-}{-}{-}{-}}
\NormalTok{simple\_landscape}\OtherTok{=}\FunctionTok{rast}\NormalTok{(}\StringTok{"RasterGrids\_10m/2024/Ainava\_vienk\_mask.tif"}\NormalTok{)}


\CommentTok{\# General\_Water\_cell.tif    egv\_459 {-}{-}{-}{-}}
\NormalTok{udens}\OtherTok{=}\FunctionTok{ifel}\NormalTok{(simple\_landscape}\SpecialCharTok{==}\DecValTok{200}\NormalTok{,}\DecValTok{1}\NormalTok{,}\DecValTok{0}\NormalTok{)}
\NormalTok{i2e\_rez}\OtherTok{=}\NormalTok{egvtools}\SpecialCharTok{::}\FunctionTok{input2egv}\NormalTok{(}\AttributeTok{input=}\NormalTok{udens,}
              \AttributeTok{egv\_template=} \StringTok{"./Templates/TemplateRasters/LV100m\_10km.tif"}\NormalTok{,}
              \AttributeTok{summary\_function =} \StringTok{"average"}\NormalTok{,}
              \AttributeTok{missing\_job =} \StringTok{"FillOutput"}\NormalTok{,}
              \AttributeTok{outlocation =} \StringTok{"./RasterGrids\_100m/2024/RAW/"}\NormalTok{,}
              \AttributeTok{outfilename =} \StringTok{"General\_Water\_cell.tif"}\NormalTok{,}
              \AttributeTok{layername =} \StringTok{"egv\_459"}\NormalTok{,}
              \AttributeTok{idw\_weight =} \DecValTok{2}\NormalTok{,}
              \AttributeTok{plot\_gaps =} \ConstantTok{FALSE}\NormalTok{,}\AttributeTok{plot\_final =} \ConstantTok{TRUE}\NormalTok{)}
\NormalTok{i2e\_rez}
\FunctionTok{rm}\NormalTok{(udens)}
\FunctionTok{rm}\NormalTok{(i2e\_rez)}

\CommentTok{\# standardisation {-}{-}{-}{-}}
\ControlFlowTok{if}\NormalTok{(}\SpecialCharTok{!}\FunctionTok{require}\NormalTok{(terra)) \{}\FunctionTok{install.packages}\NormalTok{(}\StringTok{"terra"}\NormalTok{); }\FunctionTok{require}\NormalTok{(terra)\}}
\ControlFlowTok{if}\NormalTok{(}\SpecialCharTok{!}\FunctionTok{require}\NormalTok{(tidyverse)) \{}\FunctionTok{install.packages}\NormalTok{(}\StringTok{"tidyverse"}\NormalTok{); }\FunctionTok{require}\NormalTok{(tidyverse)\}}

\NormalTok{nosaukums}\OtherTok{=}\StringTok{"General\_Water\_cell.tif"}
\NormalTok{ielasisanas\_cels}\OtherTok{=}\FunctionTok{paste0}\NormalTok{(}\StringTok{"./RasterGrids\_100m/2024/RAW/"}\NormalTok{,nosaukums)}
\NormalTok{saglabasanas\_cels}\OtherTok{=}\FunctionTok{paste0}\NormalTok{(}\StringTok{"./RasterGrids\_100m/2024/Scaled/"}\NormalTok{,nosaukums)}
\NormalTok{slanis}\OtherTok{=}\FunctionTok{rast}\NormalTok{(ielasisanas\_cels)}
\NormalTok{videjais}\OtherTok{=}\FunctionTok{global}\NormalTok{(slanis,}\AttributeTok{fun=}\StringTok{"mean"}\NormalTok{,}\AttributeTok{na.rm=}\ConstantTok{TRUE}\NormalTok{)}
\NormalTok{centrets}\OtherTok{=}\NormalTok{slanis}\SpecialCharTok{{-}}\NormalTok{videjais[,}\DecValTok{1}\NormalTok{]}
\NormalTok{standartnovirze}\OtherTok{=}\NormalTok{terra}\SpecialCharTok{::}\FunctionTok{global}\NormalTok{(centrets,}\AttributeTok{fun=}\StringTok{"rms"}\NormalTok{,}\AttributeTok{na.rm=}\ConstantTok{TRUE}\NormalTok{)}
\NormalTok{merogots}\OtherTok{=}\NormalTok{centrets}\SpecialCharTok{/}\NormalTok{standartnovirze[,}\DecValTok{1}\NormalTok{]}
\FunctionTok{writeRaster}\NormalTok{(merogots,}
      \AttributeTok{filename=}\NormalTok{saglabasanas\_cels,}
      \AttributeTok{overwrite=}\ConstantTok{TRUE}\NormalTok{)}
\end{Highlighting}
\end{Shaded}

\section{General\_Water\_r500}\label{ch06.460}

\textbf{filename:} \texttt{General\_Water\_r500.tif}

\textbf{layername:} \texttt{egv\_460}

\textbf{English name:} Fractional cover of Waterbodies within the 0.5 km landscape

\textbf{Latvian name:} Ūdenstilpju platības īpatsvars 0,5 km ainavā

\textbf{Procedure:} The cover fraction within a radius of 500 m around the analysis grid cell is
calculated as the area-weighted sum of the \hyperref[ch06.459]{analysis cells} inside the
buffer, using the workflow \texttt{egvtools::radius\_function()}. During the calculation of the landscape metric,
inverse distance weighted (power = 2) gap filling on the output is applied
to ensure no missing values at the edges. Then the layer is rewritten to set
its name. Finally, the layer is standardised by subtracting the arithmetic
mean and dividing by the root mean squared error.

\begin{Shaded}
\begin{Highlighting}[]
\CommentTok{\# libs {-}{-}{-}{-}}
\ControlFlowTok{if}\NormalTok{(}\SpecialCharTok{!}\FunctionTok{require}\NormalTok{(terra)) \{}\FunctionTok{install.packages}\NormalTok{(}\StringTok{"terra"}\NormalTok{); }\FunctionTok{require}\NormalTok{(terra)\}}
\ControlFlowTok{if}\NormalTok{(}\SpecialCharTok{!}\FunctionTok{require}\NormalTok{(egvtools)) \{remotes}\SpecialCharTok{::}\FunctionTok{install\_github}\NormalTok{(}\StringTok{"aavotins/egvtools"}\NormalTok{); }\FunctionTok{require}\NormalTok{(egvtools)\}}


\CommentTok{\# Templates {-}{-}{-}{-}{-}}
\NormalTok{template100}\OtherTok{=}\FunctionTok{rast}\NormalTok{(}\StringTok{"./Templates/TemplateRasters/LV100m\_10km.tif"}\NormalTok{)}

\CommentTok{\# radii {-}{-}{-}{-}}
\FunctionTok{radius\_function}\NormalTok{(}
 \AttributeTok{kvadrati\_path =} \StringTok{"./Templates/TemplateGrids/tiles/"}\NormalTok{,}
 \AttributeTok{radii\_path   =} \StringTok{"./Templates/TemplateGridPoints/tiles/"}\NormalTok{,}
 \AttributeTok{tikls100\_path =} \StringTok{"./Templates/TemplateGrids/tikls100\_sauzeme.parquet"}\NormalTok{,}
 \AttributeTok{template\_path =} \StringTok{"./Templates/TemplateRasters/LV100m\_10km.tif"}\NormalTok{,}
 \AttributeTok{input\_layers  =} \FunctionTok{c}\NormalTok{(}\StringTok{"./RasterGrids\_100m/2024/RAW/General\_Water\_cell.tif"}\NormalTok{),}
 \AttributeTok{layer\_prefixes =} \FunctionTok{c}\NormalTok{(}\StringTok{"General\_Water"}\NormalTok{),}
 \AttributeTok{output\_dir   =} \StringTok{"./RasterGrids\_100m/2024/RAW/"}\NormalTok{,}
 \AttributeTok{n\_workers   =} \DecValTok{6}\NormalTok{,}
 \AttributeTok{radii     =} \FunctionTok{c}\NormalTok{(}\StringTok{"r500"}\NormalTok{),}
 \AttributeTok{radius\_mode  =} \StringTok{"sparse"}\NormalTok{,}
 \AttributeTok{extract\_fun  =} \StringTok{"mean"}\NormalTok{,}
 \AttributeTok{fill\_missing  =} \ConstantTok{TRUE}\NormalTok{,}
 \AttributeTok{IDW\_weight   =} \DecValTok{2}\NormalTok{,}
 \AttributeTok{future\_max\_size =} \DecValTok{40} \SpecialCharTok{*} \DecValTok{1024}\SpecialCharTok{\^{}}\DecValTok{3}\NormalTok{)}


\CommentTok{\# General\_Water\_r500.tif    egv\_460}
\NormalTok{slanis}\OtherTok{=}\FunctionTok{rast}\NormalTok{(}\StringTok{"./RasterGrids\_100m/2024/RAW/General\_Water\_r500.tif"}\NormalTok{)}
\FunctionTok{names}\NormalTok{(slanis)}\OtherTok{=}\StringTok{"egv\_460"}
\NormalTok{slanis2}\OtherTok{=}\FunctionTok{project}\NormalTok{(slanis,template100)}
\FunctionTok{writeRaster}\NormalTok{(slanis2,}
      \StringTok{"./RasterGrids\_100m/2024/RAW/General\_Water\_r500.tif"}\NormalTok{,}
      \AttributeTok{overwrite=}\ConstantTok{TRUE}\NormalTok{)}

\CommentTok{\# standardisation {-}{-}{-}{-}}
\ControlFlowTok{if}\NormalTok{(}\SpecialCharTok{!}\FunctionTok{require}\NormalTok{(terra)) \{}\FunctionTok{install.packages}\NormalTok{(}\StringTok{"terra"}\NormalTok{); }\FunctionTok{require}\NormalTok{(terra)\}}
\ControlFlowTok{if}\NormalTok{(}\SpecialCharTok{!}\FunctionTok{require}\NormalTok{(tidyverse)) \{}\FunctionTok{install.packages}\NormalTok{(}\StringTok{"tidyverse"}\NormalTok{); }\FunctionTok{require}\NormalTok{(tidyverse)\}}

\NormalTok{nosaukums}\OtherTok{=}\StringTok{"General\_Water\_r500.tif"}
\NormalTok{ielasisanas\_cels}\OtherTok{=}\FunctionTok{paste0}\NormalTok{(}\StringTok{"./RasterGrids\_100m/2024/RAW/"}\NormalTok{,nosaukums)}
\NormalTok{saglabasanas\_cels}\OtherTok{=}\FunctionTok{paste0}\NormalTok{(}\StringTok{"./RasterGrids\_100m/2024/Scaled/"}\NormalTok{,nosaukums)}
\NormalTok{slanis}\OtherTok{=}\FunctionTok{rast}\NormalTok{(ielasisanas\_cels)}
\NormalTok{videjais}\OtherTok{=}\FunctionTok{global}\NormalTok{(slanis,}\AttributeTok{fun=}\StringTok{"mean"}\NormalTok{,}\AttributeTok{na.rm=}\ConstantTok{TRUE}\NormalTok{)}
\NormalTok{centrets}\OtherTok{=}\NormalTok{slanis}\SpecialCharTok{{-}}\NormalTok{videjais[,}\DecValTok{1}\NormalTok{]}
\NormalTok{standartnovirze}\OtherTok{=}\NormalTok{terra}\SpecialCharTok{::}\FunctionTok{global}\NormalTok{(centrets,}\AttributeTok{fun=}\StringTok{"rms"}\NormalTok{,}\AttributeTok{na.rm=}\ConstantTok{TRUE}\NormalTok{)}
\NormalTok{merogots}\OtherTok{=}\NormalTok{centrets}\SpecialCharTok{/}\NormalTok{standartnovirze[,}\DecValTok{1}\NormalTok{]}
\FunctionTok{writeRaster}\NormalTok{(merogots,}
      \AttributeTok{filename=}\NormalTok{saglabasanas\_cels,}
      \AttributeTok{overwrite=}\ConstantTok{TRUE}\NormalTok{)}
\end{Highlighting}
\end{Shaded}

\section{General\_Water\_r1250}\label{ch06.461}

\textbf{filename:} \texttt{General\_Water\_r1250.tif}

\textbf{layername:} \texttt{egv\_461}

\textbf{English name:} Fractional cover of Waterbodies within the 1.25 km landscape

\textbf{Latvian name:} Ūdenstilpju platības īpatsvars 1,25 km ainavā

\textbf{Procedure:} The cover fraction within a radius of 1250 m around the analysis grid cell
is calculated as the area-weighted sum of the \hyperref[ch06.459]{analysis cells} inside
the buffer, using the workflow \texttt{egvtools::radius\_function()}. During the calculation of the landscape
metric, inverse distance weighted (power = 2) gap filling on the output is
applied to ensure no missing values at the edges. Then the layer is
rewritten to set its name. Finally, the layer is standardised by
subtracting the arithmetic mean and dividing by the root mean squared error.

\begin{Shaded}
\begin{Highlighting}[]
\CommentTok{\# libs {-}{-}{-}{-}}
\ControlFlowTok{if}\NormalTok{(}\SpecialCharTok{!}\FunctionTok{require}\NormalTok{(terra)) \{}\FunctionTok{install.packages}\NormalTok{(}\StringTok{"terra"}\NormalTok{); }\FunctionTok{require}\NormalTok{(terra)\}}
\ControlFlowTok{if}\NormalTok{(}\SpecialCharTok{!}\FunctionTok{require}\NormalTok{(egvtools)) \{remotes}\SpecialCharTok{::}\FunctionTok{install\_github}\NormalTok{(}\StringTok{"aavotins/egvtools"}\NormalTok{); }\FunctionTok{require}\NormalTok{(egvtools)\}}


\CommentTok{\# Templates {-}{-}{-}{-}{-}}
\NormalTok{template100}\OtherTok{=}\FunctionTok{rast}\NormalTok{(}\StringTok{"./Templates/TemplateRasters/LV100m\_10km.tif"}\NormalTok{)}

\CommentTok{\# radii {-}{-}{-}{-}}
\FunctionTok{radius\_function}\NormalTok{(}
 \AttributeTok{kvadrati\_path =} \StringTok{"./Templates/TemplateGrids/tiles/"}\NormalTok{,}
 \AttributeTok{radii\_path   =} \StringTok{"./Templates/TemplateGridPoints/tiles/"}\NormalTok{,}
 \AttributeTok{tikls100\_path =} \StringTok{"./Templates/TemplateGrids/tikls100\_sauzeme.parquet"}\NormalTok{,}
 \AttributeTok{template\_path =} \StringTok{"./Templates/TemplateRasters/LV100m\_10km.tif"}\NormalTok{,}
 \AttributeTok{input\_layers  =} \FunctionTok{c}\NormalTok{(}\StringTok{"./RasterGrids\_100m/2024/RAW/General\_Water\_cell.tif"}\NormalTok{),}
 \AttributeTok{layer\_prefixes =} \FunctionTok{c}\NormalTok{(}\StringTok{"General\_Water"}\NormalTok{),}
 \AttributeTok{output\_dir   =} \StringTok{"./RasterGrids\_100m/2024/RAW/"}\NormalTok{,}
 \AttributeTok{n\_workers   =} \DecValTok{6}\NormalTok{,}
 \AttributeTok{radii     =} \FunctionTok{c}\NormalTok{(}\StringTok{"r1250"}\NormalTok{),}
 \AttributeTok{radius\_mode  =} \StringTok{"sparse"}\NormalTok{,}
 \AttributeTok{extract\_fun  =} \StringTok{"mean"}\NormalTok{,}
 \AttributeTok{fill\_missing  =} \ConstantTok{TRUE}\NormalTok{,}
 \AttributeTok{IDW\_weight   =} \DecValTok{2}\NormalTok{,}
 \AttributeTok{future\_max\_size =} \DecValTok{40} \SpecialCharTok{*} \DecValTok{1024}\SpecialCharTok{\^{}}\DecValTok{3}\NormalTok{)}


\CommentTok{\# General\_Water\_r1250.tif   egv\_461}
\NormalTok{slanis}\OtherTok{=}\FunctionTok{rast}\NormalTok{(}\StringTok{"./RasterGrids\_100m/2024/RAW/General\_Water\_r1250.tif"}\NormalTok{)}
\FunctionTok{names}\NormalTok{(slanis)}\OtherTok{=}\StringTok{"egv\_461"}
\NormalTok{slanis2}\OtherTok{=}\FunctionTok{project}\NormalTok{(slanis,template100)}
\FunctionTok{writeRaster}\NormalTok{(slanis2,}
      \StringTok{"./RasterGrids\_100m/2024/RAW/General\_Water\_r1250.tif"}\NormalTok{,}
      \AttributeTok{overwrite=}\ConstantTok{TRUE}\NormalTok{)}

\CommentTok{\# standardisation {-}{-}{-}{-}}
\ControlFlowTok{if}\NormalTok{(}\SpecialCharTok{!}\FunctionTok{require}\NormalTok{(terra)) \{}\FunctionTok{install.packages}\NormalTok{(}\StringTok{"terra"}\NormalTok{); }\FunctionTok{require}\NormalTok{(terra)\}}
\ControlFlowTok{if}\NormalTok{(}\SpecialCharTok{!}\FunctionTok{require}\NormalTok{(tidyverse)) \{}\FunctionTok{install.packages}\NormalTok{(}\StringTok{"tidyverse"}\NormalTok{); }\FunctionTok{require}\NormalTok{(tidyverse)\}}

\NormalTok{nosaukums}\OtherTok{=}\StringTok{"General\_Water\_r1250.tif"}
\NormalTok{ielasisanas\_cels}\OtherTok{=}\FunctionTok{paste0}\NormalTok{(}\StringTok{"./RasterGrids\_100m/2024/RAW/"}\NormalTok{,nosaukums)}
\NormalTok{saglabasanas\_cels}\OtherTok{=}\FunctionTok{paste0}\NormalTok{(}\StringTok{"./RasterGrids\_100m/2024/Scaled/"}\NormalTok{,nosaukums)}
\NormalTok{slanis}\OtherTok{=}\FunctionTok{rast}\NormalTok{(ielasisanas\_cels)}
\NormalTok{videjais}\OtherTok{=}\FunctionTok{global}\NormalTok{(slanis,}\AttributeTok{fun=}\StringTok{"mean"}\NormalTok{,}\AttributeTok{na.rm=}\ConstantTok{TRUE}\NormalTok{)}
\NormalTok{centrets}\OtherTok{=}\NormalTok{slanis}\SpecialCharTok{{-}}\NormalTok{videjais[,}\DecValTok{1}\NormalTok{]}
\NormalTok{standartnovirze}\OtherTok{=}\NormalTok{terra}\SpecialCharTok{::}\FunctionTok{global}\NormalTok{(centrets,}\AttributeTok{fun=}\StringTok{"rms"}\NormalTok{,}\AttributeTok{na.rm=}\ConstantTok{TRUE}\NormalTok{)}
\NormalTok{merogots}\OtherTok{=}\NormalTok{centrets}\SpecialCharTok{/}\NormalTok{standartnovirze[,}\DecValTok{1}\NormalTok{]}
\FunctionTok{writeRaster}\NormalTok{(merogots,}
      \AttributeTok{filename=}\NormalTok{saglabasanas\_cels,}
      \AttributeTok{overwrite=}\ConstantTok{TRUE}\NormalTok{)}
\end{Highlighting}
\end{Shaded}

\section{General\_Water\_r3000}\label{ch06.462}

\textbf{filename:} \texttt{General\_Water\_r3000.tif}

\textbf{layername:} \texttt{egv\_462}

\textbf{English name:} Fractional cover of Waterbodies within the 3 km landscape

\textbf{Latvian name:} Ūdenstilpju platības īpatsvars 3 km ainavā

\textbf{Procedure:} The cover fraction within a radius of 3000 m around the analysis grid cell
is calculated as the area-weighted sum of the \hyperref[ch06.459]{analysis cells} inside
the buffer, using the workflow \texttt{egvtools::radius\_function()}. During the calculation of the landscape
metric, inverse distance weighted (power = 2) gap filling on the output is
applied to ensure no missing values at the edges. Then the layer is
rewritten to set its name. Finally, the layer is standardised by
subtracting the arithmetic mean and dividing by the root mean squared error.

\begin{Shaded}
\begin{Highlighting}[]
\CommentTok{\# libs {-}{-}{-}{-}}
\ControlFlowTok{if}\NormalTok{(}\SpecialCharTok{!}\FunctionTok{require}\NormalTok{(terra)) \{}\FunctionTok{install.packages}\NormalTok{(}\StringTok{"terra"}\NormalTok{); }\FunctionTok{require}\NormalTok{(terra)\}}
\ControlFlowTok{if}\NormalTok{(}\SpecialCharTok{!}\FunctionTok{require}\NormalTok{(egvtools)) \{remotes}\SpecialCharTok{::}\FunctionTok{install\_github}\NormalTok{(}\StringTok{"aavotins/egvtools"}\NormalTok{); }\FunctionTok{require}\NormalTok{(egvtools)\}}


\CommentTok{\# Templates {-}{-}{-}{-}{-}}
\NormalTok{template100}\OtherTok{=}\FunctionTok{rast}\NormalTok{(}\StringTok{"./Templates/TemplateRasters/LV100m\_10km.tif"}\NormalTok{)}

\CommentTok{\# radii {-}{-}{-}{-}}
\FunctionTok{radius\_function}\NormalTok{(}
 \AttributeTok{kvadrati\_path =} \StringTok{"./Templates/TemplateGrids/tiles/"}\NormalTok{,}
 \AttributeTok{radii\_path   =} \StringTok{"./Templates/TemplateGridPoints/tiles/"}\NormalTok{,}
 \AttributeTok{tikls100\_path =} \StringTok{"./Templates/TemplateGrids/tikls100\_sauzeme.parquet"}\NormalTok{,}
 \AttributeTok{template\_path =} \StringTok{"./Templates/TemplateRasters/LV100m\_10km.tif"}\NormalTok{,}
 \AttributeTok{input\_layers  =} \FunctionTok{c}\NormalTok{(}\StringTok{"./RasterGrids\_100m/2024/RAW/General\_Water\_cell.tif"}\NormalTok{),}
 \AttributeTok{layer\_prefixes =} \FunctionTok{c}\NormalTok{(}\StringTok{"General\_Water"}\NormalTok{),}
 \AttributeTok{output\_dir   =} \StringTok{"./RasterGrids\_100m/2024/RAW/"}\NormalTok{,}
 \AttributeTok{n\_workers   =} \DecValTok{6}\NormalTok{,}
 \AttributeTok{radii     =} \FunctionTok{c}\NormalTok{(}\StringTok{"r3000"}\NormalTok{),}
 \AttributeTok{radius\_mode  =} \StringTok{"sparse"}\NormalTok{,}
 \AttributeTok{extract\_fun  =} \StringTok{"mean"}\NormalTok{,}
 \AttributeTok{fill\_missing  =} \ConstantTok{TRUE}\NormalTok{,}
 \AttributeTok{IDW\_weight   =} \DecValTok{2}\NormalTok{,}
 \AttributeTok{future\_max\_size =} \DecValTok{40} \SpecialCharTok{*} \DecValTok{1024}\SpecialCharTok{\^{}}\DecValTok{3}\NormalTok{)}


\CommentTok{\# General\_Water\_r3000.tif   egv\_462}
\NormalTok{slanis}\OtherTok{=}\FunctionTok{rast}\NormalTok{(}\StringTok{"./RasterGrids\_100m/2024/RAW/General\_Water\_r3000.tif"}\NormalTok{)}
\FunctionTok{names}\NormalTok{(slanis)}\OtherTok{=}\StringTok{"egv\_462"}
\NormalTok{slanis2}\OtherTok{=}\FunctionTok{project}\NormalTok{(slanis,template100)}
\FunctionTok{writeRaster}\NormalTok{(slanis2,}
      \StringTok{"./RasterGrids\_100m/2024/RAW/General\_Water\_r3000.tif"}\NormalTok{,}
      \AttributeTok{overwrite=}\ConstantTok{TRUE}\NormalTok{)}

\CommentTok{\# standardisation {-}{-}{-}{-}}
\ControlFlowTok{if}\NormalTok{(}\SpecialCharTok{!}\FunctionTok{require}\NormalTok{(terra)) \{}\FunctionTok{install.packages}\NormalTok{(}\StringTok{"terra"}\NormalTok{); }\FunctionTok{require}\NormalTok{(terra)\}}
\ControlFlowTok{if}\NormalTok{(}\SpecialCharTok{!}\FunctionTok{require}\NormalTok{(tidyverse)) \{}\FunctionTok{install.packages}\NormalTok{(}\StringTok{"tidyverse"}\NormalTok{); }\FunctionTok{require}\NormalTok{(tidyverse)\}}

\NormalTok{nosaukums}\OtherTok{=}\StringTok{"General\_Water\_r3000.tif"}
\NormalTok{ielasisanas\_cels}\OtherTok{=}\FunctionTok{paste0}\NormalTok{(}\StringTok{"./RasterGrids\_100m/2024/RAW/"}\NormalTok{,nosaukums)}
\NormalTok{saglabasanas\_cels}\OtherTok{=}\FunctionTok{paste0}\NormalTok{(}\StringTok{"./RasterGrids\_100m/2024/Scaled/"}\NormalTok{,nosaukums)}
\NormalTok{slanis}\OtherTok{=}\FunctionTok{rast}\NormalTok{(ielasisanas\_cels)}
\NormalTok{videjais}\OtherTok{=}\FunctionTok{global}\NormalTok{(slanis,}\AttributeTok{fun=}\StringTok{"mean"}\NormalTok{,}\AttributeTok{na.rm=}\ConstantTok{TRUE}\NormalTok{)}
\NormalTok{centrets}\OtherTok{=}\NormalTok{slanis}\SpecialCharTok{{-}}\NormalTok{videjais[,}\DecValTok{1}\NormalTok{]}
\NormalTok{standartnovirze}\OtherTok{=}\NormalTok{terra}\SpecialCharTok{::}\FunctionTok{global}\NormalTok{(centrets,}\AttributeTok{fun=}\StringTok{"rms"}\NormalTok{,}\AttributeTok{na.rm=}\ConstantTok{TRUE}\NormalTok{)}
\NormalTok{merogots}\OtherTok{=}\NormalTok{centrets}\SpecialCharTok{/}\NormalTok{standartnovirze[,}\DecValTok{1}\NormalTok{]}
\FunctionTok{writeRaster}\NormalTok{(merogots,}
      \AttributeTok{filename=}\NormalTok{saglabasanas\_cels,}
      \AttributeTok{overwrite=}\ConstantTok{TRUE}\NormalTok{)}
\end{Highlighting}
\end{Shaded}

\section{General\_Water\_r10000}\label{ch06.463}

\textbf{filename:} \texttt{General\_Water\_r10000.tif}

\textbf{layername:} \texttt{egv\_463}

\textbf{English name:} Fractional cover of Waterbodies within the 10 km landscape

\textbf{Latvian name:} Ūdenstilpju platības īpatsvars 10 km ainavā

\textbf{Procedure:} The cover fraction within a radius of 10000 m around the analysis grid cell
is calculated as the area-weighted sum of the \hyperref[ch06.459]{analysis cells} inside
the buffer, using the workflow \texttt{egvtools::radius\_function()}. During the calculation of the landscape
metric, inverse distance weighted (power = 2) gap filling on the output is
applied to ensure no missing values at the edges. Then the layer is
rewritten to set its name. Finally, the layer is standardised by
subtracting the arithmetic mean and dividing by the root mean squared error.

\begin{Shaded}
\begin{Highlighting}[]
\CommentTok{\# libs {-}{-}{-}{-}}
\ControlFlowTok{if}\NormalTok{(}\SpecialCharTok{!}\FunctionTok{require}\NormalTok{(terra)) \{}\FunctionTok{install.packages}\NormalTok{(}\StringTok{"terra"}\NormalTok{); }\FunctionTok{require}\NormalTok{(terra)\}}
\ControlFlowTok{if}\NormalTok{(}\SpecialCharTok{!}\FunctionTok{require}\NormalTok{(egvtools)) \{remotes}\SpecialCharTok{::}\FunctionTok{install\_github}\NormalTok{(}\StringTok{"aavotins/egvtools"}\NormalTok{); }\FunctionTok{require}\NormalTok{(egvtools)\}}


\CommentTok{\# Templates {-}{-}{-}{-}{-}}
\NormalTok{template100}\OtherTok{=}\FunctionTok{rast}\NormalTok{(}\StringTok{"./Templates/TemplateRasters/LV100m\_10km.tif"}\NormalTok{)}

\CommentTok{\# radii {-}{-}{-}{-}}
\FunctionTok{radius\_function}\NormalTok{(}
 \AttributeTok{kvadrati\_path =} \StringTok{"./Templates/TemplateGrids/tiles/"}\NormalTok{,}
 \AttributeTok{radii\_path   =} \StringTok{"./Templates/TemplateGridPoints/tiles/"}\NormalTok{,}
 \AttributeTok{tikls100\_path =} \StringTok{"./Templates/TemplateGrids/tikls100\_sauzeme.parquet"}\NormalTok{,}
 \AttributeTok{template\_path =} \StringTok{"./Templates/TemplateRasters/LV100m\_10km.tif"}\NormalTok{,}
 \AttributeTok{input\_layers  =} \FunctionTok{c}\NormalTok{(}\StringTok{"./RasterGrids\_100m/2024/RAW/General\_Water\_cell.tif"}\NormalTok{),}
 \AttributeTok{layer\_prefixes =} \FunctionTok{c}\NormalTok{(}\StringTok{"General\_Water"}\NormalTok{),}
 \AttributeTok{output\_dir   =} \StringTok{"./RasterGrids\_100m/2024/RAW/"}\NormalTok{,}
 \AttributeTok{n\_workers   =} \DecValTok{6}\NormalTok{,}
 \AttributeTok{radii     =} \FunctionTok{c}\NormalTok{(}\StringTok{"r10000"}\NormalTok{),}
 \AttributeTok{radius\_mode  =} \StringTok{"sparse"}\NormalTok{,}
 \AttributeTok{extract\_fun  =} \StringTok{"mean"}\NormalTok{,}
 \AttributeTok{fill\_missing  =} \ConstantTok{TRUE}\NormalTok{,}
 \AttributeTok{IDW\_weight   =} \DecValTok{2}\NormalTok{,}
 \AttributeTok{future\_max\_size =} \DecValTok{40} \SpecialCharTok{*} \DecValTok{1024}\SpecialCharTok{\^{}}\DecValTok{3}\NormalTok{)}


\CommentTok{\# General\_Water\_r10000.tif  egv\_463}
\NormalTok{slanis}\OtherTok{=}\FunctionTok{rast}\NormalTok{(}\StringTok{"./RasterGrids\_100m/2024/RAW/General\_Water\_r10000.tif"}\NormalTok{)}
\FunctionTok{names}\NormalTok{(slanis)}\OtherTok{=}\StringTok{"egv\_463"}
\NormalTok{slanis2}\OtherTok{=}\FunctionTok{project}\NormalTok{(slanis,template100)}
\FunctionTok{writeRaster}\NormalTok{(slanis2,}
      \StringTok{"./RasterGrids\_100m/2024/RAW/General\_Water\_r10000.tif"}\NormalTok{,}
      \AttributeTok{overwrite=}\ConstantTok{TRUE}\NormalTok{)}

\CommentTok{\# standardisation {-}{-}{-}{-}}
\ControlFlowTok{if}\NormalTok{(}\SpecialCharTok{!}\FunctionTok{require}\NormalTok{(terra)) \{}\FunctionTok{install.packages}\NormalTok{(}\StringTok{"terra"}\NormalTok{); }\FunctionTok{require}\NormalTok{(terra)\}}
\ControlFlowTok{if}\NormalTok{(}\SpecialCharTok{!}\FunctionTok{require}\NormalTok{(tidyverse)) \{}\FunctionTok{install.packages}\NormalTok{(}\StringTok{"tidyverse"}\NormalTok{); }\FunctionTok{require}\NormalTok{(tidyverse)\}}

\NormalTok{nosaukums}\OtherTok{=}\StringTok{"General\_Water\_r10000.tif"}
\NormalTok{ielasisanas\_cels}\OtherTok{=}\FunctionTok{paste0}\NormalTok{(}\StringTok{"./RasterGrids\_100m/2024/RAW/"}\NormalTok{,nosaukums)}
\NormalTok{saglabasanas\_cels}\OtherTok{=}\FunctionTok{paste0}\NormalTok{(}\StringTok{"./RasterGrids\_100m/2024/Scaled/"}\NormalTok{,nosaukums)}
\NormalTok{slanis}\OtherTok{=}\FunctionTok{rast}\NormalTok{(ielasisanas\_cels)}
\NormalTok{videjais}\OtherTok{=}\FunctionTok{global}\NormalTok{(slanis,}\AttributeTok{fun=}\StringTok{"mean"}\NormalTok{,}\AttributeTok{na.rm=}\ConstantTok{TRUE}\NormalTok{)}
\NormalTok{centrets}\OtherTok{=}\NormalTok{slanis}\SpecialCharTok{{-}}\NormalTok{videjais[,}\DecValTok{1}\NormalTok{]}
\NormalTok{standartnovirze}\OtherTok{=}\NormalTok{terra}\SpecialCharTok{::}\FunctionTok{global}\NormalTok{(centrets,}\AttributeTok{fun=}\StringTok{"rms"}\NormalTok{,}\AttributeTok{na.rm=}\ConstantTok{TRUE}\NormalTok{)}
\NormalTok{merogots}\OtherTok{=}\NormalTok{centrets}\SpecialCharTok{/}\NormalTok{standartnovirze[,}\DecValTok{1}\NormalTok{]}
\FunctionTok{writeRaster}\NormalTok{(merogots,}
      \AttributeTok{filename=}\NormalTok{saglabasanas\_cels,}
      \AttributeTok{overwrite=}\ConstantTok{TRUE}\NormalTok{)}
\end{Highlighting}
\end{Shaded}

\section{Wetlands\_Bogs\_cell}\label{ch06.464}

\textbf{filename:} \texttt{Wetlands\_Bogs\_cell.tif}

\textbf{layername:} \texttt{egv\_464}

\textbf{English name:} Fractional cover of Raised Bogs within the analysis cell (1
ha)

\textbf{Latvian name:} Augsto purvu platības īpatsvars analīzes šūnā (1 ha)

\textbf{Procedure:} Derived from the \hyperref[Ch04.17]{Bogs and Mires: EDI}, where bogs are
classified as 1 with 0 elsewhere. The resulting layer
is then aggregated to EGV resolution using the workflow \texttt{egvtools::input2egv()}, which
calculates the arithmetic mean to determine the cover fraction. During
aggregation, inverse distance weighted (power = 2) gap filling on the output is
applied to ensure no missing values at the edges. Finally, the layer is
standardised by subtracting the arithmetic mean and dividing by the root mean squared
error.

\begin{Shaded}
\begin{Highlighting}[]
\CommentTok{\# libs {-}{-}{-}{-}}
\ControlFlowTok{if}\NormalTok{(}\SpecialCharTok{!}\FunctionTok{require}\NormalTok{(egvtools)) \{remotes}\SpecialCharTok{::}\FunctionTok{install\_github}\NormalTok{(}\StringTok{"aavotins/egvtools"}\NormalTok{); }\FunctionTok{require}\NormalTok{(egvtools)\}}
\ControlFlowTok{if}\NormalTok{(}\SpecialCharTok{!}\FunctionTok{require}\NormalTok{(terra)) \{}\FunctionTok{install.packages}\NormalTok{(}\StringTok{"terra"}\NormalTok{); }\FunctionTok{require}\NormalTok{(terra)\}}

\CommentTok{\# template {-}{-}{-}{-}}
\NormalTok{template100}\OtherTok{=}\FunctionTok{rast}\NormalTok{(}\StringTok{"./Templates/TemplateRasters/LV100m\_10km.tif"}\NormalTok{)}

\CommentTok{\# Wetlands\_Bogs\_cell.tif    egv\_464 {-}{-}{-}{-}}
\NormalTok{bogs}\OtherTok{=}\FunctionTok{rast}\NormalTok{(}\StringTok{"./RasterGrids\_10m/2024/EDI\_BogsYN.tif"}\NormalTok{)}
\NormalTok{i2e\_rez}\OtherTok{=}\NormalTok{egvtools}\SpecialCharTok{::}\FunctionTok{input2egv}\NormalTok{(}\AttributeTok{input=}\NormalTok{bogs,}
              \AttributeTok{egv\_template=} \StringTok{"./Templates/TemplateRasters/LV100m\_10km.tif"}\NormalTok{,}
              \AttributeTok{summary\_function =} \StringTok{"average"}\NormalTok{,}
              \AttributeTok{missing\_job =} \StringTok{"FillOutput"}\NormalTok{,}
              \AttributeTok{outlocation =} \StringTok{"./RasterGrids\_100m/2024/RAW/"}\NormalTok{,}
              \AttributeTok{outfilename =} \StringTok{"Wetlands\_Bogs\_cell.tif"}\NormalTok{,}
              \AttributeTok{layername =} \StringTok{"egv\_464"}\NormalTok{,}
              \AttributeTok{idw\_weight =} \DecValTok{2}\NormalTok{,}
              \AttributeTok{plot\_gaps =} \ConstantTok{FALSE}\NormalTok{,}\AttributeTok{plot\_final =} \ConstantTok{TRUE}\NormalTok{)}
\NormalTok{i2e\_rez}
\FunctionTok{rm}\NormalTok{(bogs)}
\FunctionTok{rm}\NormalTok{(i2e\_rez)}

\CommentTok{\# standardisation {-}{-}{-}{-}}
\ControlFlowTok{if}\NormalTok{(}\SpecialCharTok{!}\FunctionTok{require}\NormalTok{(terra)) \{}\FunctionTok{install.packages}\NormalTok{(}\StringTok{"terra"}\NormalTok{); }\FunctionTok{require}\NormalTok{(terra)\}}
\ControlFlowTok{if}\NormalTok{(}\SpecialCharTok{!}\FunctionTok{require}\NormalTok{(tidyverse)) \{}\FunctionTok{install.packages}\NormalTok{(}\StringTok{"tidyverse"}\NormalTok{); }\FunctionTok{require}\NormalTok{(tidyverse)\}}

\NormalTok{nosaukums}\OtherTok{=}\StringTok{"Wetlands\_Bogs\_cell.tif"}
\NormalTok{ielasisanas\_cels}\OtherTok{=}\FunctionTok{paste0}\NormalTok{(}\StringTok{"./RasterGrids\_100m/2024/RAW/"}\NormalTok{,nosaukums)}
\NormalTok{saglabasanas\_cels}\OtherTok{=}\FunctionTok{paste0}\NormalTok{(}\StringTok{"./RasterGrids\_100m/2024/Scaled/"}\NormalTok{,nosaukums)}
\NormalTok{slanis}\OtherTok{=}\FunctionTok{rast}\NormalTok{(ielasisanas\_cels)}
\NormalTok{videjais}\OtherTok{=}\FunctionTok{global}\NormalTok{(slanis,}\AttributeTok{fun=}\StringTok{"mean"}\NormalTok{,}\AttributeTok{na.rm=}\ConstantTok{TRUE}\NormalTok{)}
\NormalTok{centrets}\OtherTok{=}\NormalTok{slanis}\SpecialCharTok{{-}}\NormalTok{videjais[,}\DecValTok{1}\NormalTok{]}
\NormalTok{standartnovirze}\OtherTok{=}\NormalTok{terra}\SpecialCharTok{::}\FunctionTok{global}\NormalTok{(centrets,}\AttributeTok{fun=}\StringTok{"rms"}\NormalTok{,}\AttributeTok{na.rm=}\ConstantTok{TRUE}\NormalTok{)}
\NormalTok{merogots}\OtherTok{=}\NormalTok{centrets}\SpecialCharTok{/}\NormalTok{standartnovirze[,}\DecValTok{1}\NormalTok{]}
\FunctionTok{writeRaster}\NormalTok{(merogots,}
      \AttributeTok{filename=}\NormalTok{saglabasanas\_cels,}
      \AttributeTok{overwrite=}\ConstantTok{TRUE}\NormalTok{)}
\end{Highlighting}
\end{Shaded}

\section{Wetlands\_Bogs\_r500}\label{ch06.465}

\textbf{filename:} \texttt{Wetlands\_Bogs\_r500.tif}

\textbf{layername:} \texttt{egv\_465}

\textbf{English name:} Fractional cover of Raised Bogs within the 0.5 km landscape

\textbf{Latvian name:} Augsto purvu platības īpatsvars 0,5 km ainavā

\textbf{Procedure:} The cover fraction within a radius of 500 m around the analysis grid cell is
calculated as the area-weighted sum of the \hyperref[ch06.464]{analysis cells} inside the
buffer, using the workflow \texttt{egvtools::radius\_function()}. During the calculation of the landscape metric,
inverse distance weighted (power = 2) gap filling on the output is applied
to ensure no missing values at the edges. Then the layer is rewritten to set
its name. Finally, the layer is standardised by subtracting the arithmetic
mean and dividing by the root mean squared error.

\begin{Shaded}
\begin{Highlighting}[]
\CommentTok{\# libs {-}{-}{-}{-}}
\ControlFlowTok{if}\NormalTok{(}\SpecialCharTok{!}\FunctionTok{require}\NormalTok{(terra)) \{}\FunctionTok{install.packages}\NormalTok{(}\StringTok{"terra"}\NormalTok{); }\FunctionTok{require}\NormalTok{(terra)\}}
\ControlFlowTok{if}\NormalTok{(}\SpecialCharTok{!}\FunctionTok{require}\NormalTok{(egvtools)) \{remotes}\SpecialCharTok{::}\FunctionTok{install\_github}\NormalTok{(}\StringTok{"aavotins/egvtools"}\NormalTok{); }\FunctionTok{require}\NormalTok{(egvtools)\}}


\CommentTok{\# Templates {-}{-}{-}{-}{-}}
\NormalTok{template100}\OtherTok{=}\FunctionTok{rast}\NormalTok{(}\StringTok{"./Templates/TemplateRasters/LV100m\_10km.tif"}\NormalTok{)}

\CommentTok{\# radii {-}{-}{-}{-}}
\FunctionTok{radius\_function}\NormalTok{(}
 \AttributeTok{kvadrati\_path =} \StringTok{"./Templates/TemplateGrids/tiles/"}\NormalTok{,}
 \AttributeTok{radii\_path   =} \StringTok{"./Templates/TemplateGridPoints/tiles/"}\NormalTok{,}
 \AttributeTok{tikls100\_path =} \StringTok{"./Templates/TemplateGrids/tikls100\_sauzeme.parquet"}\NormalTok{,}
 \AttributeTok{template\_path =} \StringTok{"./Templates/TemplateRasters/LV100m\_10km.tif"}\NormalTok{,}
 \AttributeTok{input\_layers  =} \FunctionTok{c}\NormalTok{(}\StringTok{"./RasterGrids\_100m/2024/RAW/Wetlands\_Bogs\_cell.tif"}\NormalTok{),}
 \AttributeTok{layer\_prefixes =} \FunctionTok{c}\NormalTok{(}\StringTok{"Wetlands\_Bogs"}\NormalTok{),}
 \AttributeTok{output\_dir   =} \StringTok{"./RasterGrids\_100m/2024/RAW/"}\NormalTok{,}
 \AttributeTok{n\_workers   =} \DecValTok{12}\NormalTok{,}
 \AttributeTok{radii     =} \FunctionTok{c}\NormalTok{(}\StringTok{"r500"}\NormalTok{),}
 \AttributeTok{radius\_mode  =} \StringTok{"sparse"}\NormalTok{,}
 \AttributeTok{extract\_fun  =} \StringTok{"mean"}\NormalTok{,}
 \AttributeTok{fill\_missing  =} \ConstantTok{TRUE}\NormalTok{,}
 \AttributeTok{IDW\_weight   =} \DecValTok{2}\NormalTok{,}
 \AttributeTok{future\_max\_size =} \DecValTok{20} \SpecialCharTok{*} \DecValTok{1024}\SpecialCharTok{\^{}}\DecValTok{3}\NormalTok{)}


\CommentTok{\# Wetlands\_Bogs\_r500.tif    egv\_465}
\NormalTok{slanis}\OtherTok{=}\FunctionTok{rast}\NormalTok{(}\StringTok{"./RasterGrids\_100m/2024/RAW/Wetlands\_Bogs\_r500.tif"}\NormalTok{)}
\FunctionTok{names}\NormalTok{(slanis)}\OtherTok{=}\StringTok{"egv\_465"}
\NormalTok{slanis2}\OtherTok{=}\FunctionTok{project}\NormalTok{(slanis,template100)}
\FunctionTok{writeRaster}\NormalTok{(slanis2,}
      \StringTok{"./RasterGrids\_100m/2024/RAW/Wetlands\_Bogs\_r500.tif"}\NormalTok{,}
      \AttributeTok{overwrite=}\ConstantTok{TRUE}\NormalTok{)}

\CommentTok{\# standardisation {-}{-}{-}{-}}
\ControlFlowTok{if}\NormalTok{(}\SpecialCharTok{!}\FunctionTok{require}\NormalTok{(terra)) \{}\FunctionTok{install.packages}\NormalTok{(}\StringTok{"terra"}\NormalTok{); }\FunctionTok{require}\NormalTok{(terra)\}}
\ControlFlowTok{if}\NormalTok{(}\SpecialCharTok{!}\FunctionTok{require}\NormalTok{(tidyverse)) \{}\FunctionTok{install.packages}\NormalTok{(}\StringTok{"tidyverse"}\NormalTok{); }\FunctionTok{require}\NormalTok{(tidyverse)\}}

\NormalTok{nosaukums}\OtherTok{=}\StringTok{"Wetlands\_Bogs\_r500.tif"}
\NormalTok{ielasisanas\_cels}\OtherTok{=}\FunctionTok{paste0}\NormalTok{(}\StringTok{"./RasterGrids\_100m/2024/RAW/"}\NormalTok{,nosaukums)}
\NormalTok{saglabasanas\_cels}\OtherTok{=}\FunctionTok{paste0}\NormalTok{(}\StringTok{"./RasterGrids\_100m/2024/Scaled/"}\NormalTok{,nosaukums)}
\NormalTok{slanis}\OtherTok{=}\FunctionTok{rast}\NormalTok{(ielasisanas\_cels)}
\NormalTok{videjais}\OtherTok{=}\FunctionTok{global}\NormalTok{(slanis,}\AttributeTok{fun=}\StringTok{"mean"}\NormalTok{,}\AttributeTok{na.rm=}\ConstantTok{TRUE}\NormalTok{)}
\NormalTok{centrets}\OtherTok{=}\NormalTok{slanis}\SpecialCharTok{{-}}\NormalTok{videjais[,}\DecValTok{1}\NormalTok{]}
\NormalTok{standartnovirze}\OtherTok{=}\NormalTok{terra}\SpecialCharTok{::}\FunctionTok{global}\NormalTok{(centrets,}\AttributeTok{fun=}\StringTok{"rms"}\NormalTok{,}\AttributeTok{na.rm=}\ConstantTok{TRUE}\NormalTok{)}
\NormalTok{merogots}\OtherTok{=}\NormalTok{centrets}\SpecialCharTok{/}\NormalTok{standartnovirze[,}\DecValTok{1}\NormalTok{]}
\FunctionTok{writeRaster}\NormalTok{(merogots,}
      \AttributeTok{filename=}\NormalTok{saglabasanas\_cels,}
      \AttributeTok{overwrite=}\ConstantTok{TRUE}\NormalTok{)}
\end{Highlighting}
\end{Shaded}

\section{Wetlands\_Bogs\_r1250}\label{ch06.466}

\textbf{filename:} \texttt{Wetlands\_Bogs\_r1250.tif}

\textbf{layername:} \texttt{egv\_466}

\textbf{English name:} Fractional cover of Raised Bogs within the 1.25 km landscape

\textbf{Latvian name:} Augsto purvu platības īpatsvars 1,25 km ainavā

\textbf{Procedure:} The cover fraction within a radius of 1250 m around the analysis grid cell
is calculated as the area-weighted sum of the \hyperref[ch06.464]{analysis cells} inside
the buffer, using the workflow \texttt{egvtools::radius\_function()}. During the calculation of the landscape
metric, inverse distance weighted (power = 2) gap filling on the output is
applied to ensure no missing values at the edges. Then the layer is
rewritten to set its name. Finally, the layer is standardised by
subtracting the arithmetic mean and dividing by the root mean squared error.

\begin{Shaded}
\begin{Highlighting}[]
\CommentTok{\# libs {-}{-}{-}{-}}
\ControlFlowTok{if}\NormalTok{(}\SpecialCharTok{!}\FunctionTok{require}\NormalTok{(terra)) \{}\FunctionTok{install.packages}\NormalTok{(}\StringTok{"terra"}\NormalTok{); }\FunctionTok{require}\NormalTok{(terra)\}}
\ControlFlowTok{if}\NormalTok{(}\SpecialCharTok{!}\FunctionTok{require}\NormalTok{(egvtools)) \{remotes}\SpecialCharTok{::}\FunctionTok{install\_github}\NormalTok{(}\StringTok{"aavotins/egvtools"}\NormalTok{); }\FunctionTok{require}\NormalTok{(egvtools)\}}


\CommentTok{\# Templates {-}{-}{-}{-}{-}}
\NormalTok{template100}\OtherTok{=}\FunctionTok{rast}\NormalTok{(}\StringTok{"./Templates/TemplateRasters/LV100m\_10km.tif"}\NormalTok{)}

\CommentTok{\# radii {-}{-}{-}{-}}
\FunctionTok{radius\_function}\NormalTok{(}
 \AttributeTok{kvadrati\_path =} \StringTok{"./Templates/TemplateGrids/tiles/"}\NormalTok{,}
 \AttributeTok{radii\_path   =} \StringTok{"./Templates/TemplateGridPoints/tiles/"}\NormalTok{,}
 \AttributeTok{tikls100\_path =} \StringTok{"./Templates/TemplateGrids/tikls100\_sauzeme.parquet"}\NormalTok{,}
 \AttributeTok{template\_path =} \StringTok{"./Templates/TemplateRasters/LV100m\_10km.tif"}\NormalTok{,}
 \AttributeTok{input\_layers  =} \FunctionTok{c}\NormalTok{(}\StringTok{"./RasterGrids\_100m/2024/RAW/Wetlands\_Bogs\_cell.tif"}\NormalTok{),}
 \AttributeTok{layer\_prefixes =} \FunctionTok{c}\NormalTok{(}\StringTok{"Wetlands\_Bogs"}\NormalTok{),}
 \AttributeTok{output\_dir   =} \StringTok{"./RasterGrids\_100m/2024/RAW/"}\NormalTok{,}
 \AttributeTok{n\_workers   =} \DecValTok{12}\NormalTok{,}
 \AttributeTok{radii     =} \FunctionTok{c}\NormalTok{(}\StringTok{"r1250"}\NormalTok{),}
 \AttributeTok{radius\_mode  =} \StringTok{"sparse"}\NormalTok{,}
 \AttributeTok{extract\_fun  =} \StringTok{"mean"}\NormalTok{,}
 \AttributeTok{fill\_missing  =} \ConstantTok{TRUE}\NormalTok{,}
 \AttributeTok{IDW\_weight   =} \DecValTok{2}\NormalTok{,}
 \AttributeTok{future\_max\_size =} \DecValTok{20} \SpecialCharTok{*} \DecValTok{1024}\SpecialCharTok{\^{}}\DecValTok{3}\NormalTok{)}


\CommentTok{\# Wetlands\_Bogs\_r1250.tif   egv\_466}
\NormalTok{slanis}\OtherTok{=}\FunctionTok{rast}\NormalTok{(}\StringTok{"./RasterGrids\_100m/2024/RAW/Wetlands\_Bogs\_r1250.tif"}\NormalTok{)}
\FunctionTok{names}\NormalTok{(slanis)}\OtherTok{=}\StringTok{"egv\_466"}
\NormalTok{slanis2}\OtherTok{=}\FunctionTok{project}\NormalTok{(slanis,template100)}
\FunctionTok{writeRaster}\NormalTok{(slanis2,}
      \StringTok{"./RasterGrids\_100m/2024/RAW/Wetlands\_Bogs\_r1250.tif"}\NormalTok{,}
      \AttributeTok{overwrite=}\ConstantTok{TRUE}\NormalTok{)}

\CommentTok{\# standardisation {-}{-}{-}{-}}
\ControlFlowTok{if}\NormalTok{(}\SpecialCharTok{!}\FunctionTok{require}\NormalTok{(terra)) \{}\FunctionTok{install.packages}\NormalTok{(}\StringTok{"terra"}\NormalTok{); }\FunctionTok{require}\NormalTok{(terra)\}}
\ControlFlowTok{if}\NormalTok{(}\SpecialCharTok{!}\FunctionTok{require}\NormalTok{(tidyverse)) \{}\FunctionTok{install.packages}\NormalTok{(}\StringTok{"tidyverse"}\NormalTok{); }\FunctionTok{require}\NormalTok{(tidyverse)\}}

\NormalTok{nosaukums}\OtherTok{=}\StringTok{"Wetlands\_Bogs\_r1250.tif"}
\NormalTok{ielasisanas\_cels}\OtherTok{=}\FunctionTok{paste0}\NormalTok{(}\StringTok{"./RasterGrids\_100m/2024/RAW/"}\NormalTok{,nosaukums)}
\NormalTok{saglabasanas\_cels}\OtherTok{=}\FunctionTok{paste0}\NormalTok{(}\StringTok{"./RasterGrids\_100m/2024/Scaled/"}\NormalTok{,nosaukums)}
\NormalTok{slanis}\OtherTok{=}\FunctionTok{rast}\NormalTok{(ielasisanas\_cels)}
\NormalTok{videjais}\OtherTok{=}\FunctionTok{global}\NormalTok{(slanis,}\AttributeTok{fun=}\StringTok{"mean"}\NormalTok{,}\AttributeTok{na.rm=}\ConstantTok{TRUE}\NormalTok{)}
\NormalTok{centrets}\OtherTok{=}\NormalTok{slanis}\SpecialCharTok{{-}}\NormalTok{videjais[,}\DecValTok{1}\NormalTok{]}
\NormalTok{standartnovirze}\OtherTok{=}\NormalTok{terra}\SpecialCharTok{::}\FunctionTok{global}\NormalTok{(centrets,}\AttributeTok{fun=}\StringTok{"rms"}\NormalTok{,}\AttributeTok{na.rm=}\ConstantTok{TRUE}\NormalTok{)}
\NormalTok{merogots}\OtherTok{=}\NormalTok{centrets}\SpecialCharTok{/}\NormalTok{standartnovirze[,}\DecValTok{1}\NormalTok{]}
\FunctionTok{writeRaster}\NormalTok{(merogots,}
      \AttributeTok{filename=}\NormalTok{saglabasanas\_cels,}
      \AttributeTok{overwrite=}\ConstantTok{TRUE}\NormalTok{)}
\end{Highlighting}
\end{Shaded}

\section{Wetlands\_Bogs\_r3000}\label{ch06.467}

\textbf{filename:} \texttt{Wetlands\_Bogs\_r3000.tif}

\textbf{layername:} \texttt{egv\_467}

\textbf{English name:} Fractional cover of Raised Bogs within the 3 km landscape

\textbf{Latvian name:} Augsto purvu platības īpatsvars 3 km ainavā

\textbf{Procedure:} The cover fraction within a radius of 3000 m around the analysis grid cell
is calculated as the area-weighted sum of the \hyperref[ch06.464]{analysis cells} inside
the buffer, using the workflow \texttt{egvtools::radius\_function()}. During the calculation of the landscape
metric, inverse distance weighted (power = 2) gap filling on the output is
applied to ensure no missing values at the edges. Then the layer is
rewritten to set its name. Finally, the layer is standardised by
subtracting the arithmetic mean and dividing by the root mean squared error.

\begin{Shaded}
\begin{Highlighting}[]
\CommentTok{\# libs {-}{-}{-}{-}}
\ControlFlowTok{if}\NormalTok{(}\SpecialCharTok{!}\FunctionTok{require}\NormalTok{(terra)) \{}\FunctionTok{install.packages}\NormalTok{(}\StringTok{"terra"}\NormalTok{); }\FunctionTok{require}\NormalTok{(terra)\}}
\ControlFlowTok{if}\NormalTok{(}\SpecialCharTok{!}\FunctionTok{require}\NormalTok{(egvtools)) \{remotes}\SpecialCharTok{::}\FunctionTok{install\_github}\NormalTok{(}\StringTok{"aavotins/egvtools"}\NormalTok{); }\FunctionTok{require}\NormalTok{(egvtools)\}}


\CommentTok{\# Templates {-}{-}{-}{-}{-}}
\NormalTok{template100}\OtherTok{=}\FunctionTok{rast}\NormalTok{(}\StringTok{"./Templates/TemplateRasters/LV100m\_10km.tif"}\NormalTok{)}

\CommentTok{\# radii {-}{-}{-}{-}}
\FunctionTok{radius\_function}\NormalTok{(}
 \AttributeTok{kvadrati\_path =} \StringTok{"./Templates/TemplateGrids/tiles/"}\NormalTok{,}
 \AttributeTok{radii\_path   =} \StringTok{"./Templates/TemplateGridPoints/tiles/"}\NormalTok{,}
 \AttributeTok{tikls100\_path =} \StringTok{"./Templates/TemplateGrids/tikls100\_sauzeme.parquet"}\NormalTok{,}
 \AttributeTok{template\_path =} \StringTok{"./Templates/TemplateRasters/LV100m\_10km.tif"}\NormalTok{,}
 \AttributeTok{input\_layers  =} \FunctionTok{c}\NormalTok{(}\StringTok{"./RasterGrids\_100m/2024/RAW/Wetlands\_Bogs\_cell.tif"}\NormalTok{),}
 \AttributeTok{layer\_prefixes =} \FunctionTok{c}\NormalTok{(}\StringTok{"Wetlands\_Bogs"}\NormalTok{),}
 \AttributeTok{output\_dir   =} \StringTok{"./RasterGrids\_100m/2024/RAW/"}\NormalTok{,}
 \AttributeTok{n\_workers   =} \DecValTok{12}\NormalTok{,}
 \AttributeTok{radii     =} \FunctionTok{c}\NormalTok{(}\StringTok{"r3000"}\NormalTok{),}
 \AttributeTok{radius\_mode  =} \StringTok{"sparse"}\NormalTok{,}
 \AttributeTok{extract\_fun  =} \StringTok{"mean"}\NormalTok{,}
 \AttributeTok{fill\_missing  =} \ConstantTok{TRUE}\NormalTok{,}
 \AttributeTok{IDW\_weight   =} \DecValTok{2}\NormalTok{,}
 \AttributeTok{future\_max\_size =} \DecValTok{20} \SpecialCharTok{*} \DecValTok{1024}\SpecialCharTok{\^{}}\DecValTok{3}\NormalTok{)}


\CommentTok{\# Wetlands\_Bogs\_r3000.tif   egv\_467}
\NormalTok{slanis}\OtherTok{=}\FunctionTok{rast}\NormalTok{(}\StringTok{"./RasterGrids\_100m/2024/RAW/Wetlands\_Bogs\_r3000.tif"}\NormalTok{)}
\FunctionTok{names}\NormalTok{(slanis)}\OtherTok{=}\StringTok{"egv\_467"}
\NormalTok{slanis2}\OtherTok{=}\FunctionTok{project}\NormalTok{(slanis,template100)}
\FunctionTok{writeRaster}\NormalTok{(slanis2,}
      \StringTok{"./RasterGrids\_100m/2024/RAW/Wetlands\_Bogs\_r3000.tif"}\NormalTok{,}
      \AttributeTok{overwrite=}\ConstantTok{TRUE}\NormalTok{)}

\CommentTok{\# standardisation {-}{-}{-}{-}}
\ControlFlowTok{if}\NormalTok{(}\SpecialCharTok{!}\FunctionTok{require}\NormalTok{(terra)) \{}\FunctionTok{install.packages}\NormalTok{(}\StringTok{"terra"}\NormalTok{); }\FunctionTok{require}\NormalTok{(terra)\}}
\ControlFlowTok{if}\NormalTok{(}\SpecialCharTok{!}\FunctionTok{require}\NormalTok{(tidyverse)) \{}\FunctionTok{install.packages}\NormalTok{(}\StringTok{"tidyverse"}\NormalTok{); }\FunctionTok{require}\NormalTok{(tidyverse)\}}

\NormalTok{nosaukums}\OtherTok{=}\StringTok{"Wetlands\_Bogs\_r3000.tif"}
\NormalTok{ielasisanas\_cels}\OtherTok{=}\FunctionTok{paste0}\NormalTok{(}\StringTok{"./RasterGrids\_100m/2024/RAW/"}\NormalTok{,nosaukums)}
\NormalTok{saglabasanas\_cels}\OtherTok{=}\FunctionTok{paste0}\NormalTok{(}\StringTok{"./RasterGrids\_100m/2024/Scaled/"}\NormalTok{,nosaukums)}
\NormalTok{slanis}\OtherTok{=}\FunctionTok{rast}\NormalTok{(ielasisanas\_cels)}
\NormalTok{videjais}\OtherTok{=}\FunctionTok{global}\NormalTok{(slanis,}\AttributeTok{fun=}\StringTok{"mean"}\NormalTok{,}\AttributeTok{na.rm=}\ConstantTok{TRUE}\NormalTok{)}
\NormalTok{centrets}\OtherTok{=}\NormalTok{slanis}\SpecialCharTok{{-}}\NormalTok{videjais[,}\DecValTok{1}\NormalTok{]}
\NormalTok{standartnovirze}\OtherTok{=}\NormalTok{terra}\SpecialCharTok{::}\FunctionTok{global}\NormalTok{(centrets,}\AttributeTok{fun=}\StringTok{"rms"}\NormalTok{,}\AttributeTok{na.rm=}\ConstantTok{TRUE}\NormalTok{)}
\NormalTok{merogots}\OtherTok{=}\NormalTok{centrets}\SpecialCharTok{/}\NormalTok{standartnovirze[,}\DecValTok{1}\NormalTok{]}
\FunctionTok{writeRaster}\NormalTok{(merogots,}
      \AttributeTok{filename=}\NormalTok{saglabasanas\_cels,}
      \AttributeTok{overwrite=}\ConstantTok{TRUE}\NormalTok{)}
\end{Highlighting}
\end{Shaded}

\section{Wetlands\_Bogs\_r10000}\label{ch06.468}

\textbf{filename:} \texttt{Wetlands\_Bogs\_r10000.tif}

\textbf{layername:} \texttt{egv\_468}

\textbf{English name:} Fractional cover of Raised Bogs within the 10 km landscape

\textbf{Latvian name:} Augsto purvu platības īpatsvars 10 km ainavā

\textbf{Procedure:} The cover fraction within a radius of 10000 m around the analysis grid cell
is calculated as the area-weighted sum of the \hyperref[ch06.464]{analysis cells} inside
the buffer, using the workflow \texttt{egvtools::radius\_function()}. During the calculation of the landscape
metric, inverse distance weighted (power = 2) gap filling on the output is
applied to ensure no missing values at the edges. Then the layer is
rewritten to set its name. Finally, the layer is standardised by
subtracting the arithmetic mean and dividing by the root mean squared error.

\begin{Shaded}
\begin{Highlighting}[]
\CommentTok{\# libs {-}{-}{-}{-}}
\ControlFlowTok{if}\NormalTok{(}\SpecialCharTok{!}\FunctionTok{require}\NormalTok{(terra)) \{}\FunctionTok{install.packages}\NormalTok{(}\StringTok{"terra"}\NormalTok{); }\FunctionTok{require}\NormalTok{(terra)\}}
\ControlFlowTok{if}\NormalTok{(}\SpecialCharTok{!}\FunctionTok{require}\NormalTok{(egvtools)) \{remotes}\SpecialCharTok{::}\FunctionTok{install\_github}\NormalTok{(}\StringTok{"aavotins/egvtools"}\NormalTok{); }\FunctionTok{require}\NormalTok{(egvtools)\}}


\CommentTok{\# Templates {-}{-}{-}{-}{-}}
\NormalTok{template100}\OtherTok{=}\FunctionTok{rast}\NormalTok{(}\StringTok{"./Templates/TemplateRasters/LV100m\_10km.tif"}\NormalTok{)}

\CommentTok{\# radii {-}{-}{-}{-}}
\FunctionTok{radius\_function}\NormalTok{(}
 \AttributeTok{kvadrati\_path =} \StringTok{"./Templates/TemplateGrids/tiles/"}\NormalTok{,}
 \AttributeTok{radii\_path   =} \StringTok{"./Templates/TemplateGridPoints/tiles/"}\NormalTok{,}
 \AttributeTok{tikls100\_path =} \StringTok{"./Templates/TemplateGrids/tikls100\_sauzeme.parquet"}\NormalTok{,}
 \AttributeTok{template\_path =} \StringTok{"./Templates/TemplateRasters/LV100m\_10km.tif"}\NormalTok{,}
 \AttributeTok{input\_layers  =} \FunctionTok{c}\NormalTok{(}\StringTok{"./RasterGrids\_100m/2024/RAW/Wetlands\_Bogs\_cell.tif"}\NormalTok{),}
 \AttributeTok{layer\_prefixes =} \FunctionTok{c}\NormalTok{(}\StringTok{"Wetlands\_Bogs"}\NormalTok{),}
 \AttributeTok{output\_dir   =} \StringTok{"./RasterGrids\_100m/2024/RAW/"}\NormalTok{,}
 \AttributeTok{n\_workers   =} \DecValTok{12}\NormalTok{,}
 \AttributeTok{radii     =} \FunctionTok{c}\NormalTok{(}\StringTok{"r10000"}\NormalTok{),}
 \AttributeTok{radius\_mode  =} \StringTok{"sparse"}\NormalTok{,}
 \AttributeTok{extract\_fun  =} \StringTok{"mean"}\NormalTok{,}
 \AttributeTok{fill\_missing  =} \ConstantTok{TRUE}\NormalTok{,}
 \AttributeTok{IDW\_weight   =} \DecValTok{2}\NormalTok{,}
 \AttributeTok{future\_max\_size =} \DecValTok{20} \SpecialCharTok{*} \DecValTok{1024}\SpecialCharTok{\^{}}\DecValTok{3}\NormalTok{)}


\CommentTok{\# Wetlands\_Bogs\_r10000.tif  egv\_468}
\NormalTok{slanis}\OtherTok{=}\FunctionTok{rast}\NormalTok{(}\StringTok{"./RasterGrids\_100m/2024/RAW/Wetlands\_Bogs\_r10000.tif"}\NormalTok{)}
\FunctionTok{names}\NormalTok{(slanis)}\OtherTok{=}\StringTok{"egv\_468"}
\NormalTok{slanis2}\OtherTok{=}\FunctionTok{project}\NormalTok{(slanis,template100)}
\FunctionTok{writeRaster}\NormalTok{(slanis2,}
      \StringTok{"./RasterGrids\_100m/2024/RAW/Wetlands\_Bogs\_r10000.tif"}\NormalTok{,}
      \AttributeTok{overwrite=}\ConstantTok{TRUE}\NormalTok{)}

\CommentTok{\# standardisation {-}{-}{-}{-}}
\ControlFlowTok{if}\NormalTok{(}\SpecialCharTok{!}\FunctionTok{require}\NormalTok{(terra)) \{}\FunctionTok{install.packages}\NormalTok{(}\StringTok{"terra"}\NormalTok{); }\FunctionTok{require}\NormalTok{(terra)\}}
\ControlFlowTok{if}\NormalTok{(}\SpecialCharTok{!}\FunctionTok{require}\NormalTok{(tidyverse)) \{}\FunctionTok{install.packages}\NormalTok{(}\StringTok{"tidyverse"}\NormalTok{); }\FunctionTok{require}\NormalTok{(tidyverse)\}}

\NormalTok{nosaukums}\OtherTok{=}\StringTok{"Wetlands\_Bogs\_r10000.tif"}
\NormalTok{ielasisanas\_cels}\OtherTok{=}\FunctionTok{paste0}\NormalTok{(}\StringTok{"./RasterGrids\_100m/2024/RAW/"}\NormalTok{,nosaukums)}
\NormalTok{saglabasanas\_cels}\OtherTok{=}\FunctionTok{paste0}\NormalTok{(}\StringTok{"./RasterGrids\_100m/2024/Scaled/"}\NormalTok{,nosaukums)}
\NormalTok{slanis}\OtherTok{=}\FunctionTok{rast}\NormalTok{(ielasisanas\_cels)}
\NormalTok{videjais}\OtherTok{=}\FunctionTok{global}\NormalTok{(slanis,}\AttributeTok{fun=}\StringTok{"mean"}\NormalTok{,}\AttributeTok{na.rm=}\ConstantTok{TRUE}\NormalTok{)}
\NormalTok{centrets}\OtherTok{=}\NormalTok{slanis}\SpecialCharTok{{-}}\NormalTok{videjais[,}\DecValTok{1}\NormalTok{]}
\NormalTok{standartnovirze}\OtherTok{=}\NormalTok{terra}\SpecialCharTok{::}\FunctionTok{global}\NormalTok{(centrets,}\AttributeTok{fun=}\StringTok{"rms"}\NormalTok{,}\AttributeTok{na.rm=}\ConstantTok{TRUE}\NormalTok{)}
\NormalTok{merogots}\OtherTok{=}\NormalTok{centrets}\SpecialCharTok{/}\NormalTok{standartnovirze[,}\DecValTok{1}\NormalTok{]}
\FunctionTok{writeRaster}\NormalTok{(merogots,}
      \AttributeTok{filename=}\NormalTok{saglabasanas\_cels,}
      \AttributeTok{overwrite=}\ConstantTok{TRUE}\NormalTok{)}
\end{Highlighting}
\end{Shaded}

\section{Wetlands\_Mires\_cell}\label{ch06.469}

\textbf{filename:} \texttt{Wetlands\_Mires\_cell.tif}

\textbf{layername:} \texttt{egv\_469}

\textbf{English name:} Fractional cover of Transitional Mires within the analysis
cell (1 ha)

\textbf{Latvian name:} Pārejas purvu platības īpatsvars analīzes šūnā (1 ha)

\textbf{Procedure:} Derived from the \hyperref[Ch04.17]{Bogs and Mires: EDI}, where transitional
mires are classified as 1 with 0 elsewhere. The resulting layer
is then aggregated to EGV resolution using the workflow \texttt{egvtools::input2egv()}, which
calculates the arithmetic mean to determine the cover fraction. During
aggregation, inverse distance weighted (power = 2) gap filling on the output is
applied to ensure no missing values at the edges. Finally, the layer is
standardised by subtracting the arithmetic mean and dividing by the root mean squared
error.

\begin{Shaded}
\begin{Highlighting}[]
\CommentTok{\# libs {-}{-}{-}{-}}
\ControlFlowTok{if}\NormalTok{(}\SpecialCharTok{!}\FunctionTok{require}\NormalTok{(egvtools)) \{remotes}\SpecialCharTok{::}\FunctionTok{install\_github}\NormalTok{(}\StringTok{"aavotins/egvtools"}\NormalTok{); }\FunctionTok{require}\NormalTok{(egvtools)\}}
\ControlFlowTok{if}\NormalTok{(}\SpecialCharTok{!}\FunctionTok{require}\NormalTok{(terra)) \{}\FunctionTok{install.packages}\NormalTok{(}\StringTok{"terra"}\NormalTok{); }\FunctionTok{require}\NormalTok{(terra)\}}

\CommentTok{\# template {-}{-}{-}{-}}
\NormalTok{template100}\OtherTok{=}\FunctionTok{rast}\NormalTok{(}\StringTok{"./Templates/TemplateRasters/LV100m\_10km.tif"}\NormalTok{)}

\CommentTok{\# Wetlands\_Mires\_cell.tif   egv\_469 {-}{-}{-}{-}}
\NormalTok{mires}\OtherTok{=}\FunctionTok{rast}\NormalTok{(}\StringTok{"./RasterGrids\_10m/2024/EDI\_TransitionalMiresYN.tif"}\NormalTok{)}
\NormalTok{i2e\_rez}\OtherTok{=}\NormalTok{egvtools}\SpecialCharTok{::}\FunctionTok{input2egv}\NormalTok{(}\AttributeTok{input=}\NormalTok{mires,}
              \AttributeTok{egv\_template=} \StringTok{"./Templates/TemplateRasters/LV100m\_10km.tif"}\NormalTok{,}
              \AttributeTok{summary\_function =} \StringTok{"average"}\NormalTok{,}
              \AttributeTok{missing\_job =} \StringTok{"FillOutput"}\NormalTok{,}
              \AttributeTok{outlocation =} \StringTok{"./RasterGrids\_100m/2024/RAW/"}\NormalTok{,}
              \AttributeTok{outfilename =} \StringTok{"Wetlands\_Mires\_cell.tif"}\NormalTok{,}
              \AttributeTok{layername =} \StringTok{"egv\_469"}\NormalTok{,}
              \AttributeTok{idw\_weight =} \DecValTok{2}\NormalTok{,}
              \AttributeTok{plot\_gaps =} \ConstantTok{FALSE}\NormalTok{,}\AttributeTok{plot\_final =} \ConstantTok{TRUE}\NormalTok{)}
\NormalTok{i2e\_rez}
\FunctionTok{rm}\NormalTok{(mires)}
\FunctionTok{rm}\NormalTok{(i2e\_rez)}

\CommentTok{\# standardisation {-}{-}{-}{-}}
\ControlFlowTok{if}\NormalTok{(}\SpecialCharTok{!}\FunctionTok{require}\NormalTok{(terra)) \{}\FunctionTok{install.packages}\NormalTok{(}\StringTok{"terra"}\NormalTok{); }\FunctionTok{require}\NormalTok{(terra)\}}
\ControlFlowTok{if}\NormalTok{(}\SpecialCharTok{!}\FunctionTok{require}\NormalTok{(tidyverse)) \{}\FunctionTok{install.packages}\NormalTok{(}\StringTok{"tidyverse"}\NormalTok{); }\FunctionTok{require}\NormalTok{(tidyverse)\}}

\NormalTok{nosaukums}\OtherTok{=}\StringTok{"Wetlands\_Mires\_cell.tif"}
\NormalTok{ielasisanas\_cels}\OtherTok{=}\FunctionTok{paste0}\NormalTok{(}\StringTok{"./RasterGrids\_100m/2024/RAW/"}\NormalTok{,nosaukums)}
\NormalTok{saglabasanas\_cels}\OtherTok{=}\FunctionTok{paste0}\NormalTok{(}\StringTok{"./RasterGrids\_100m/2024/Scaled/"}\NormalTok{,nosaukums)}
\NormalTok{slanis}\OtherTok{=}\FunctionTok{rast}\NormalTok{(ielasisanas\_cels)}
\NormalTok{videjais}\OtherTok{=}\FunctionTok{global}\NormalTok{(slanis,}\AttributeTok{fun=}\StringTok{"mean"}\NormalTok{,}\AttributeTok{na.rm=}\ConstantTok{TRUE}\NormalTok{)}
\NormalTok{centrets}\OtherTok{=}\NormalTok{slanis}\SpecialCharTok{{-}}\NormalTok{videjais[,}\DecValTok{1}\NormalTok{]}
\NormalTok{standartnovirze}\OtherTok{=}\NormalTok{terra}\SpecialCharTok{::}\FunctionTok{global}\NormalTok{(centrets,}\AttributeTok{fun=}\StringTok{"rms"}\NormalTok{,}\AttributeTok{na.rm=}\ConstantTok{TRUE}\NormalTok{)}
\NormalTok{merogots}\OtherTok{=}\NormalTok{centrets}\SpecialCharTok{/}\NormalTok{standartnovirze[,}\DecValTok{1}\NormalTok{]}
\FunctionTok{writeRaster}\NormalTok{(merogots,}
      \AttributeTok{filename=}\NormalTok{saglabasanas\_cels,}
      \AttributeTok{overwrite=}\ConstantTok{TRUE}\NormalTok{)}
\end{Highlighting}
\end{Shaded}

\section{Wetlands\_Mires\_r500}\label{ch06.470}

\textbf{filename:} \texttt{Wetlands\_Mires\_r500.tif}

\textbf{layername:} \texttt{egv\_470}

\textbf{English name:} Fractional cover of Transitional Mires within the 0.5 km
landscape

\textbf{Latvian name:} Pārejas purvu platības īpatsvars 0,5 km ainavā

\textbf{Procedure:} The cover fraction within a radius of 500 m around the analysis grid cell is
calculated as the area-weighted sum of the \hyperref[ch06.469]{analysis cells} inside the
buffer, using the workflow \texttt{egvtools::radius\_function()}. During the calculation of the landscape metric,
inverse distance weighted (power = 2) gap filling on the output is applied
to ensure no missing values at the edges. Then the layer is rewritten to set
its name. Finally, the layer is standardised by subtracting the arithmetic
mean and dividing by the root mean squared error.

\begin{Shaded}
\begin{Highlighting}[]
\CommentTok{\# libs {-}{-}{-}{-}}
\ControlFlowTok{if}\NormalTok{(}\SpecialCharTok{!}\FunctionTok{require}\NormalTok{(terra)) \{}\FunctionTok{install.packages}\NormalTok{(}\StringTok{"terra"}\NormalTok{); }\FunctionTok{require}\NormalTok{(terra)\}}
\ControlFlowTok{if}\NormalTok{(}\SpecialCharTok{!}\FunctionTok{require}\NormalTok{(egvtools)) \{remotes}\SpecialCharTok{::}\FunctionTok{install\_github}\NormalTok{(}\StringTok{"aavotins/egvtools"}\NormalTok{); }\FunctionTok{require}\NormalTok{(egvtools)\}}


\CommentTok{\# Templates {-}{-}{-}{-}{-}}
\NormalTok{template100}\OtherTok{=}\FunctionTok{rast}\NormalTok{(}\StringTok{"./Templates/TemplateRasters/LV100m\_10km.tif"}\NormalTok{)}

\CommentTok{\# radii {-}{-}{-}{-}}
\FunctionTok{radius\_function}\NormalTok{(}
 \AttributeTok{kvadrati\_path =} \StringTok{"./Templates/TemplateGrids/tiles/"}\NormalTok{,}
 \AttributeTok{radii\_path   =} \StringTok{"./Templates/TemplateGridPoints/tiles/"}\NormalTok{,}
 \AttributeTok{tikls100\_path =} \StringTok{"./Templates/TemplateGrids/tikls100\_sauzeme.parquet"}\NormalTok{,}
 \AttributeTok{template\_path =} \StringTok{"./Templates/TemplateRasters/LV100m\_10km.tif"}\NormalTok{,}
 \AttributeTok{input\_layers  =} \FunctionTok{c}\NormalTok{(}\StringTok{"./RasterGrids\_100m/2024/RAW/Wetlands\_Mires\_cell.tif"}\NormalTok{),}
 \AttributeTok{layer\_prefixes =} \FunctionTok{c}\NormalTok{(}\StringTok{"Wetlands\_Mires"}\NormalTok{),}
 \AttributeTok{output\_dir   =} \StringTok{"./RasterGrids\_100m/2024/RAW/"}\NormalTok{,}
 \AttributeTok{n\_workers   =} \DecValTok{12}\NormalTok{,}
 \AttributeTok{radii     =} \FunctionTok{c}\NormalTok{(}\StringTok{"r500"}\NormalTok{),}
 \AttributeTok{radius\_mode  =} \StringTok{"sparse"}\NormalTok{,}
 \AttributeTok{extract\_fun  =} \StringTok{"mean"}\NormalTok{,}
 \AttributeTok{fill\_missing  =} \ConstantTok{TRUE}\NormalTok{,}
 \AttributeTok{IDW\_weight   =} \DecValTok{2}\NormalTok{,}
 \AttributeTok{future\_max\_size =} \DecValTok{20} \SpecialCharTok{*} \DecValTok{1024}\SpecialCharTok{\^{}}\DecValTok{3}\NormalTok{)}


\CommentTok{\# Wetlands\_Mires\_r500.tif   egv\_470}
\NormalTok{slanis}\OtherTok{=}\FunctionTok{rast}\NormalTok{(}\StringTok{"./RasterGrids\_100m/2024/RAW/Wetlands\_Mires\_r500.tif"}\NormalTok{)}
\FunctionTok{names}\NormalTok{(slanis)}\OtherTok{=}\StringTok{"egv\_470"}
\NormalTok{slanis2}\OtherTok{=}\FunctionTok{project}\NormalTok{(slanis,template100)}
\FunctionTok{writeRaster}\NormalTok{(slanis2,}
      \StringTok{"./RasterGrids\_100m/2024/RAW/Wetlands\_Mires\_r500.tif"}\NormalTok{,}
      \AttributeTok{overwrite=}\ConstantTok{TRUE}\NormalTok{)}

\CommentTok{\# standardisation {-}{-}{-}{-}}
\ControlFlowTok{if}\NormalTok{(}\SpecialCharTok{!}\FunctionTok{require}\NormalTok{(terra)) \{}\FunctionTok{install.packages}\NormalTok{(}\StringTok{"terra"}\NormalTok{); }\FunctionTok{require}\NormalTok{(terra)\}}
\ControlFlowTok{if}\NormalTok{(}\SpecialCharTok{!}\FunctionTok{require}\NormalTok{(tidyverse)) \{}\FunctionTok{install.packages}\NormalTok{(}\StringTok{"tidyverse"}\NormalTok{); }\FunctionTok{require}\NormalTok{(tidyverse)\}}

\NormalTok{nosaukums}\OtherTok{=}\StringTok{"Wetlands\_Mires\_r500.tif"}
\NormalTok{ielasisanas\_cels}\OtherTok{=}\FunctionTok{paste0}\NormalTok{(}\StringTok{"./RasterGrids\_100m/2024/RAW/"}\NormalTok{,nosaukums)}
\NormalTok{saglabasanas\_cels}\OtherTok{=}\FunctionTok{paste0}\NormalTok{(}\StringTok{"./RasterGrids\_100m/2024/Scaled/"}\NormalTok{,nosaukums)}
\NormalTok{slanis}\OtherTok{=}\FunctionTok{rast}\NormalTok{(ielasisanas\_cels)}
\NormalTok{videjais}\OtherTok{=}\FunctionTok{global}\NormalTok{(slanis,}\AttributeTok{fun=}\StringTok{"mean"}\NormalTok{,}\AttributeTok{na.rm=}\ConstantTok{TRUE}\NormalTok{)}
\NormalTok{centrets}\OtherTok{=}\NormalTok{slanis}\SpecialCharTok{{-}}\NormalTok{videjais[,}\DecValTok{1}\NormalTok{]}
\NormalTok{standartnovirze}\OtherTok{=}\NormalTok{terra}\SpecialCharTok{::}\FunctionTok{global}\NormalTok{(centrets,}\AttributeTok{fun=}\StringTok{"rms"}\NormalTok{,}\AttributeTok{na.rm=}\ConstantTok{TRUE}\NormalTok{)}
\NormalTok{merogots}\OtherTok{=}\NormalTok{centrets}\SpecialCharTok{/}\NormalTok{standartnovirze[,}\DecValTok{1}\NormalTok{]}
\FunctionTok{writeRaster}\NormalTok{(merogots,}
      \AttributeTok{filename=}\NormalTok{saglabasanas\_cels,}
      \AttributeTok{overwrite=}\ConstantTok{TRUE}\NormalTok{)}
\end{Highlighting}
\end{Shaded}

\section{Wetlands\_Mires\_r1250}\label{ch06.471}

\textbf{filename:} \texttt{Wetlands\_Mires\_r1250.tif}

\textbf{layername:} \texttt{egv\_471}

\textbf{English name:} Fractional cover of Transitional Mires within the 1.25 km
landscape

\textbf{Latvian name:} Pārejas purvu platības īpatsvars 1,25 km ainavā

\textbf{Procedure:} The cover fraction within a radius of 1250 m around the analysis grid cell
is calculated as the area-weighted sum of the \hyperref[ch06.469]{analysis cells} inside
the buffer, using the workflow \texttt{egvtools::radius\_function()}. During the calculation of the landscape
metric, inverse distance weighted (power = 2) gap filling on the output is
applied to ensure no missing values at the edges. Then the layer is
rewritten to set its name. Finally, the layer is standardised by
subtracting the arithmetic mean and dividing by the root mean squared error.

\begin{Shaded}
\begin{Highlighting}[]
\CommentTok{\# libs {-}{-}{-}{-}}
\ControlFlowTok{if}\NormalTok{(}\SpecialCharTok{!}\FunctionTok{require}\NormalTok{(terra)) \{}\FunctionTok{install.packages}\NormalTok{(}\StringTok{"terra"}\NormalTok{); }\FunctionTok{require}\NormalTok{(terra)\}}
\ControlFlowTok{if}\NormalTok{(}\SpecialCharTok{!}\FunctionTok{require}\NormalTok{(egvtools)) \{remotes}\SpecialCharTok{::}\FunctionTok{install\_github}\NormalTok{(}\StringTok{"aavotins/egvtools"}\NormalTok{); }\FunctionTok{require}\NormalTok{(egvtools)\}}


\CommentTok{\# Templates {-}{-}{-}{-}{-}}
\NormalTok{template100}\OtherTok{=}\FunctionTok{rast}\NormalTok{(}\StringTok{"./Templates/TemplateRasters/LV100m\_10km.tif"}\NormalTok{)}

\CommentTok{\# radii {-}{-}{-}{-}}
\FunctionTok{radius\_function}\NormalTok{(}
 \AttributeTok{kvadrati\_path =} \StringTok{"./Templates/TemplateGrids/tiles/"}\NormalTok{,}
 \AttributeTok{radii\_path   =} \StringTok{"./Templates/TemplateGridPoints/tiles/"}\NormalTok{,}
 \AttributeTok{tikls100\_path =} \StringTok{"./Templates/TemplateGrids/tikls100\_sauzeme.parquet"}\NormalTok{,}
 \AttributeTok{template\_path =} \StringTok{"./Templates/TemplateRasters/LV100m\_10km.tif"}\NormalTok{,}
 \AttributeTok{input\_layers  =} \FunctionTok{c}\NormalTok{(}\StringTok{"./RasterGrids\_100m/2024/RAW/Wetlands\_Mires\_cell.tif"}\NormalTok{),}
 \AttributeTok{layer\_prefixes =} \FunctionTok{c}\NormalTok{(}\StringTok{"Wetlands\_Mires"}\NormalTok{),}
 \AttributeTok{output\_dir   =} \StringTok{"./RasterGrids\_100m/2024/RAW/"}\NormalTok{,}
 \AttributeTok{n\_workers   =} \DecValTok{12}\NormalTok{,}
 \AttributeTok{radii     =} \FunctionTok{c}\NormalTok{(}\StringTok{"r1250"}\NormalTok{),}
 \AttributeTok{radius\_mode  =} \StringTok{"sparse"}\NormalTok{,}
 \AttributeTok{extract\_fun  =} \StringTok{"mean"}\NormalTok{,}
 \AttributeTok{fill\_missing  =} \ConstantTok{TRUE}\NormalTok{,}
 \AttributeTok{IDW\_weight   =} \DecValTok{2}\NormalTok{,}
 \AttributeTok{future\_max\_size =} \DecValTok{20} \SpecialCharTok{*} \DecValTok{1024}\SpecialCharTok{\^{}}\DecValTok{3}\NormalTok{)}


\CommentTok{\# Wetlands\_Mires\_r1250.tif  egv\_471}
\NormalTok{slanis}\OtherTok{=}\FunctionTok{rast}\NormalTok{(}\StringTok{"./RasterGrids\_100m/2024/RAW/Wetlands\_Mires\_r1250.tif"}\NormalTok{)}
\FunctionTok{names}\NormalTok{(slanis)}\OtherTok{=}\StringTok{"egv\_471"}
\NormalTok{slanis2}\OtherTok{=}\FunctionTok{project}\NormalTok{(slanis,template100)}
\FunctionTok{writeRaster}\NormalTok{(slanis2,}
      \StringTok{"./RasterGrids\_100m/2024/RAW/Wetlands\_Mires\_r1250.tif"}\NormalTok{,}
      \AttributeTok{overwrite=}\ConstantTok{TRUE}\NormalTok{)}

\CommentTok{\# standardisation {-}{-}{-}{-}}
\ControlFlowTok{if}\NormalTok{(}\SpecialCharTok{!}\FunctionTok{require}\NormalTok{(terra)) \{}\FunctionTok{install.packages}\NormalTok{(}\StringTok{"terra"}\NormalTok{); }\FunctionTok{require}\NormalTok{(terra)\}}
\ControlFlowTok{if}\NormalTok{(}\SpecialCharTok{!}\FunctionTok{require}\NormalTok{(tidyverse)) \{}\FunctionTok{install.packages}\NormalTok{(}\StringTok{"tidyverse"}\NormalTok{); }\FunctionTok{require}\NormalTok{(tidyverse)\}}

\NormalTok{nosaukums}\OtherTok{=}\StringTok{"Wetlands\_Mires\_r1250.tif"}
\NormalTok{ielasisanas\_cels}\OtherTok{=}\FunctionTok{paste0}\NormalTok{(}\StringTok{"./RasterGrids\_100m/2024/RAW/"}\NormalTok{,nosaukums)}
\NormalTok{saglabasanas\_cels}\OtherTok{=}\FunctionTok{paste0}\NormalTok{(}\StringTok{"./RasterGrids\_100m/2024/Scaled/"}\NormalTok{,nosaukums)}
\NormalTok{slanis}\OtherTok{=}\FunctionTok{rast}\NormalTok{(ielasisanas\_cels)}
\NormalTok{videjais}\OtherTok{=}\FunctionTok{global}\NormalTok{(slanis,}\AttributeTok{fun=}\StringTok{"mean"}\NormalTok{,}\AttributeTok{na.rm=}\ConstantTok{TRUE}\NormalTok{)}
\NormalTok{centrets}\OtherTok{=}\NormalTok{slanis}\SpecialCharTok{{-}}\NormalTok{videjais[,}\DecValTok{1}\NormalTok{]}
\NormalTok{standartnovirze}\OtherTok{=}\NormalTok{terra}\SpecialCharTok{::}\FunctionTok{global}\NormalTok{(centrets,}\AttributeTok{fun=}\StringTok{"rms"}\NormalTok{,}\AttributeTok{na.rm=}\ConstantTok{TRUE}\NormalTok{)}
\NormalTok{merogots}\OtherTok{=}\NormalTok{centrets}\SpecialCharTok{/}\NormalTok{standartnovirze[,}\DecValTok{1}\NormalTok{]}
\FunctionTok{writeRaster}\NormalTok{(merogots,}
      \AttributeTok{filename=}\NormalTok{saglabasanas\_cels,}
      \AttributeTok{overwrite=}\ConstantTok{TRUE}\NormalTok{)}
\end{Highlighting}
\end{Shaded}

\section{Wetlands\_Mires\_r3000}\label{ch06.472}

\textbf{filename:} \texttt{Wetlands\_Mires\_r3000.tif}

\textbf{layername:} \texttt{egv\_472}

\textbf{English name:} Fractional cover of Transitional Mires within the 3 km
landscape

\textbf{Latvian name:} Pārejas purvu platības īpatsvars 3 km ainavā

\textbf{Procedure:} The cover fraction within a radius of 3000 m around the analysis grid cell
is calculated as the area-weighted sum of the \hyperref[ch06.469]{analysis cells} inside
the buffer, using the workflow \texttt{egvtools::radius\_function()}. During the calculation of the landscape
metric, inverse distance weighted (power = 2) gap filling on the output is
applied to ensure no missing values at the edges. Then the layer is
rewritten to set its name. Finally, the layer is standardised by
subtracting the arithmetic mean and dividing by the root mean squared error.

\begin{Shaded}
\begin{Highlighting}[]
\CommentTok{\# libs {-}{-}{-}{-}}
\ControlFlowTok{if}\NormalTok{(}\SpecialCharTok{!}\FunctionTok{require}\NormalTok{(terra)) \{}\FunctionTok{install.packages}\NormalTok{(}\StringTok{"terra"}\NormalTok{); }\FunctionTok{require}\NormalTok{(terra)\}}
\ControlFlowTok{if}\NormalTok{(}\SpecialCharTok{!}\FunctionTok{require}\NormalTok{(egvtools)) \{remotes}\SpecialCharTok{::}\FunctionTok{install\_github}\NormalTok{(}\StringTok{"aavotins/egvtools"}\NormalTok{); }\FunctionTok{require}\NormalTok{(egvtools)\}}


\CommentTok{\# Templates {-}{-}{-}{-}{-}}
\NormalTok{template100}\OtherTok{=}\FunctionTok{rast}\NormalTok{(}\StringTok{"./Templates/TemplateRasters/LV100m\_10km.tif"}\NormalTok{)}

\CommentTok{\# radii {-}{-}{-}{-}}
\FunctionTok{radius\_function}\NormalTok{(}
 \AttributeTok{kvadrati\_path =} \StringTok{"./Templates/TemplateGrids/tiles/"}\NormalTok{,}
 \AttributeTok{radii\_path   =} \StringTok{"./Templates/TemplateGridPoints/tiles/"}\NormalTok{,}
 \AttributeTok{tikls100\_path =} \StringTok{"./Templates/TemplateGrids/tikls100\_sauzeme.parquet"}\NormalTok{,}
 \AttributeTok{template\_path =} \StringTok{"./Templates/TemplateRasters/LV100m\_10km.tif"}\NormalTok{,}
 \AttributeTok{input\_layers  =} \FunctionTok{c}\NormalTok{(}\StringTok{"./RasterGrids\_100m/2024/RAW/Wetlands\_Mires\_cell.tif"}\NormalTok{),}
 \AttributeTok{layer\_prefixes =} \FunctionTok{c}\NormalTok{(}\StringTok{"Wetlands\_Mires"}\NormalTok{),}
 \AttributeTok{output\_dir   =} \StringTok{"./RasterGrids\_100m/2024/RAW/"}\NormalTok{,}
 \AttributeTok{n\_workers   =} \DecValTok{12}\NormalTok{,}
 \AttributeTok{radii     =} \FunctionTok{c}\NormalTok{(}\StringTok{"r3000"}\NormalTok{),}
 \AttributeTok{radius\_mode  =} \StringTok{"sparse"}\NormalTok{,}
 \AttributeTok{extract\_fun  =} \StringTok{"mean"}\NormalTok{,}
 \AttributeTok{fill\_missing  =} \ConstantTok{TRUE}\NormalTok{,}
 \AttributeTok{IDW\_weight   =} \DecValTok{2}\NormalTok{,}
 \AttributeTok{future\_max\_size =} \DecValTok{20} \SpecialCharTok{*} \DecValTok{1024}\SpecialCharTok{\^{}}\DecValTok{3}\NormalTok{)}


\CommentTok{\# Wetlands\_Mires\_r3000.tif  egv\_472}
\NormalTok{slanis}\OtherTok{=}\FunctionTok{rast}\NormalTok{(}\StringTok{"./RasterGrids\_100m/2024/RAW/Wetlands\_Mires\_r3000.tif"}\NormalTok{)}
\FunctionTok{names}\NormalTok{(slanis)}\OtherTok{=}\StringTok{"egv\_472"}
\NormalTok{slanis2}\OtherTok{=}\FunctionTok{project}\NormalTok{(slanis,template100)}
\FunctionTok{writeRaster}\NormalTok{(slanis2,}
      \StringTok{"./RasterGrids\_100m/2024/RAW/Wetlands\_Mires\_r3000.tif"}\NormalTok{,}
      \AttributeTok{overwrite=}\ConstantTok{TRUE}\NormalTok{)}

\CommentTok{\# standardisation {-}{-}{-}{-}}
\ControlFlowTok{if}\NormalTok{(}\SpecialCharTok{!}\FunctionTok{require}\NormalTok{(terra)) \{}\FunctionTok{install.packages}\NormalTok{(}\StringTok{"terra"}\NormalTok{); }\FunctionTok{require}\NormalTok{(terra)\}}
\ControlFlowTok{if}\NormalTok{(}\SpecialCharTok{!}\FunctionTok{require}\NormalTok{(tidyverse)) \{}\FunctionTok{install.packages}\NormalTok{(}\StringTok{"tidyverse"}\NormalTok{); }\FunctionTok{require}\NormalTok{(tidyverse)\}}

\NormalTok{nosaukums}\OtherTok{=}\StringTok{"Wetlands\_Mires\_r3000.tif"}
\NormalTok{ielasisanas\_cels}\OtherTok{=}\FunctionTok{paste0}\NormalTok{(}\StringTok{"./RasterGrids\_100m/2024/RAW/"}\NormalTok{,nosaukums)}
\NormalTok{saglabasanas\_cels}\OtherTok{=}\FunctionTok{paste0}\NormalTok{(}\StringTok{"./RasterGrids\_100m/2024/Scaled/"}\NormalTok{,nosaukums)}
\NormalTok{slanis}\OtherTok{=}\FunctionTok{rast}\NormalTok{(ielasisanas\_cels)}
\NormalTok{videjais}\OtherTok{=}\FunctionTok{global}\NormalTok{(slanis,}\AttributeTok{fun=}\StringTok{"mean"}\NormalTok{,}\AttributeTok{na.rm=}\ConstantTok{TRUE}\NormalTok{)}
\NormalTok{centrets}\OtherTok{=}\NormalTok{slanis}\SpecialCharTok{{-}}\NormalTok{videjais[,}\DecValTok{1}\NormalTok{]}
\NormalTok{standartnovirze}\OtherTok{=}\NormalTok{terra}\SpecialCharTok{::}\FunctionTok{global}\NormalTok{(centrets,}\AttributeTok{fun=}\StringTok{"rms"}\NormalTok{,}\AttributeTok{na.rm=}\ConstantTok{TRUE}\NormalTok{)}
\NormalTok{merogots}\OtherTok{=}\NormalTok{centrets}\SpecialCharTok{/}\NormalTok{standartnovirze[,}\DecValTok{1}\NormalTok{]}
\FunctionTok{writeRaster}\NormalTok{(merogots,}
      \AttributeTok{filename=}\NormalTok{saglabasanas\_cels,}
      \AttributeTok{overwrite=}\ConstantTok{TRUE}\NormalTok{)}
\end{Highlighting}
\end{Shaded}

\section{Wetlands\_Mires\_r10000}\label{ch06.473}

\textbf{filename:} \texttt{Wetlands\_Mires\_r10000.tif}

\textbf{layername:} \texttt{egv\_473}

\textbf{English name:} Fractional cover of Transitional Mires within the 10 km
landscape

\textbf{Latvian name:} Pārejas purvu platības īpatsvars 10 km ainavā

\textbf{Procedure:} The cover fraction within a radius of 10000 m around the analysis grid cell
is calculated as the area-weighted sum of the \hyperref[ch06.469]{analysis cells} inside
the buffer, using the workflow \texttt{egvtools::radius\_function()}. During the calculation of the landscape
metric, inverse distance weighted (power = 2) gap filling on the output is
applied to ensure no missing values at the edges. Then the layer is
rewritten to set its name. Finally, the layer is standardised by
subtracting the arithmetic mean and dividing by the root mean squared error.

\begin{Shaded}
\begin{Highlighting}[]
\CommentTok{\# libs {-}{-}{-}{-}}
\ControlFlowTok{if}\NormalTok{(}\SpecialCharTok{!}\FunctionTok{require}\NormalTok{(terra)) \{}\FunctionTok{install.packages}\NormalTok{(}\StringTok{"terra"}\NormalTok{); }\FunctionTok{require}\NormalTok{(terra)\}}
\ControlFlowTok{if}\NormalTok{(}\SpecialCharTok{!}\FunctionTok{require}\NormalTok{(egvtools)) \{remotes}\SpecialCharTok{::}\FunctionTok{install\_github}\NormalTok{(}\StringTok{"aavotins/egvtools"}\NormalTok{); }\FunctionTok{require}\NormalTok{(egvtools)\}}


\CommentTok{\# Templates {-}{-}{-}{-}{-}}
\NormalTok{template100}\OtherTok{=}\FunctionTok{rast}\NormalTok{(}\StringTok{"./Templates/TemplateRasters/LV100m\_10km.tif"}\NormalTok{)}

\CommentTok{\# radii {-}{-}{-}{-}}
\FunctionTok{radius\_function}\NormalTok{(}
 \AttributeTok{kvadrati\_path =} \StringTok{"./Templates/TemplateGrids/tiles/"}\NormalTok{,}
 \AttributeTok{radii\_path   =} \StringTok{"./Templates/TemplateGridPoints/tiles/"}\NormalTok{,}
 \AttributeTok{tikls100\_path =} \StringTok{"./Templates/TemplateGrids/tikls100\_sauzeme.parquet"}\NormalTok{,}
 \AttributeTok{template\_path =} \StringTok{"./Templates/TemplateRasters/LV100m\_10km.tif"}\NormalTok{,}
 \AttributeTok{input\_layers  =} \FunctionTok{c}\NormalTok{(}\StringTok{"./RasterGrids\_100m/2024/RAW/Wetlands\_Mires\_cell.tif"}\NormalTok{),}
 \AttributeTok{layer\_prefixes =} \FunctionTok{c}\NormalTok{(}\StringTok{"Wetlands\_Mires"}\NormalTok{),}
 \AttributeTok{output\_dir   =} \StringTok{"./RasterGrids\_100m/2024/RAW/"}\NormalTok{,}
 \AttributeTok{n\_workers   =} \DecValTok{12}\NormalTok{,}
 \AttributeTok{radii     =} \FunctionTok{c}\NormalTok{(}\StringTok{"r10000"}\NormalTok{),}
 \AttributeTok{radius\_mode  =} \StringTok{"sparse"}\NormalTok{,}
 \AttributeTok{extract\_fun  =} \StringTok{"mean"}\NormalTok{,}
 \AttributeTok{fill\_missing  =} \ConstantTok{TRUE}\NormalTok{,}
 \AttributeTok{IDW\_weight   =} \DecValTok{2}\NormalTok{,}
 \AttributeTok{future\_max\_size =} \DecValTok{20} \SpecialCharTok{*} \DecValTok{1024}\SpecialCharTok{\^{}}\DecValTok{3}\NormalTok{)}


\CommentTok{\# Wetlands\_Mires\_r10000.tif egv\_473}
\NormalTok{slanis}\OtherTok{=}\FunctionTok{rast}\NormalTok{(}\StringTok{"./RasterGrids\_100m/2024/RAW/Wetlands\_Mires\_r10000.tif"}\NormalTok{)}
\FunctionTok{names}\NormalTok{(slanis)}\OtherTok{=}\StringTok{"egv\_473"}
\NormalTok{slanis2}\OtherTok{=}\FunctionTok{project}\NormalTok{(slanis,template100)}
\FunctionTok{writeRaster}\NormalTok{(slanis2,}
      \StringTok{"./RasterGrids\_100m/2024/RAW/Wetlands\_Mires\_r10000.tif"}\NormalTok{,}
      \AttributeTok{overwrite=}\ConstantTok{TRUE}\NormalTok{)}

\CommentTok{\# standardisation {-}{-}{-}{-}}
\ControlFlowTok{if}\NormalTok{(}\SpecialCharTok{!}\FunctionTok{require}\NormalTok{(terra)) \{}\FunctionTok{install.packages}\NormalTok{(}\StringTok{"terra"}\NormalTok{); }\FunctionTok{require}\NormalTok{(terra)\}}
\ControlFlowTok{if}\NormalTok{(}\SpecialCharTok{!}\FunctionTok{require}\NormalTok{(tidyverse)) \{}\FunctionTok{install.packages}\NormalTok{(}\StringTok{"tidyverse"}\NormalTok{); }\FunctionTok{require}\NormalTok{(tidyverse)\}}

\NormalTok{nosaukums}\OtherTok{=}\StringTok{"Wetlands\_Mires\_r10000.tif"}
\NormalTok{ielasisanas\_cels}\OtherTok{=}\FunctionTok{paste0}\NormalTok{(}\StringTok{"./RasterGrids\_100m/2024/RAW/"}\NormalTok{,nosaukums)}
\NormalTok{saglabasanas\_cels}\OtherTok{=}\FunctionTok{paste0}\NormalTok{(}\StringTok{"./RasterGrids\_100m/2024/Scaled/"}\NormalTok{,nosaukums)}
\NormalTok{slanis}\OtherTok{=}\FunctionTok{rast}\NormalTok{(ielasisanas\_cels)}
\NormalTok{videjais}\OtherTok{=}\FunctionTok{global}\NormalTok{(slanis,}\AttributeTok{fun=}\StringTok{"mean"}\NormalTok{,}\AttributeTok{na.rm=}\ConstantTok{TRUE}\NormalTok{)}
\NormalTok{centrets}\OtherTok{=}\NormalTok{slanis}\SpecialCharTok{{-}}\NormalTok{videjais[,}\DecValTok{1}\NormalTok{]}
\NormalTok{standartnovirze}\OtherTok{=}\NormalTok{terra}\SpecialCharTok{::}\FunctionTok{global}\NormalTok{(centrets,}\AttributeTok{fun=}\StringTok{"rms"}\NormalTok{,}\AttributeTok{na.rm=}\ConstantTok{TRUE}\NormalTok{)}
\NormalTok{merogots}\OtherTok{=}\NormalTok{centrets}\SpecialCharTok{/}\NormalTok{standartnovirze[,}\DecValTok{1}\NormalTok{]}
\FunctionTok{writeRaster}\NormalTok{(merogots,}
      \AttributeTok{filename=}\NormalTok{saglabasanas\_cels,}
      \AttributeTok{overwrite=}\ConstantTok{TRUE}\NormalTok{)}
\end{Highlighting}
\end{Shaded}

\section{Wetlands\_ReedSedgeRushBeds\_cell}\label{ch06.474}

\textbf{filename:} \texttt{Wetlands\_ReedSedgeRushBeds\_cell.tif}

\textbf{layername:} \texttt{egv\_474}

\textbf{English name:} Fractional cover of Reed-, Sedge-, Rush-, Beds within the
analysis cell (1 ha)

\textbf{Latvian name:} Niedrāju, grīslāju, meldrāju platības īpatsvars analīzes šūnā
(1 ha)

\textbf{Procedure:} First, the reed, sedge and rush beds from the \hyperref[Ch05.03]{Landscape
classification} are selected (value 720 is reclassified to value 1;
all others are set to 0). The resulting layer
is then aggregated to EGV resolution using the workflow \texttt{egvtools::input2egv()}, which
calculates the arithmetic mean to determine the cover fraction. During
aggregation, inverse distance weighted (power = 2) gap filling on the output is
applied to ensure no missing values at the edges. Finally, the layer is
standardised by subtracting the arithmetic mean and dividing by the root mean squared
error.

\begin{Shaded}
\begin{Highlighting}[]
\CommentTok{\# libs {-}{-}{-}{-}}
\ControlFlowTok{if}\NormalTok{(}\SpecialCharTok{!}\FunctionTok{require}\NormalTok{(egvtools)) \{remotes}\SpecialCharTok{::}\FunctionTok{install\_github}\NormalTok{(}\StringTok{"aavotins/egvtools"}\NormalTok{); }\FunctionTok{require}\NormalTok{(egvtools)\}}
\ControlFlowTok{if}\NormalTok{(}\SpecialCharTok{!}\FunctionTok{require}\NormalTok{(terra)) \{}\FunctionTok{install.packages}\NormalTok{(}\StringTok{"terra"}\NormalTok{); }\FunctionTok{require}\NormalTok{(terra)\}}
\ControlFlowTok{if}\NormalTok{(}\SpecialCharTok{!}\FunctionTok{require}\NormalTok{(sf)) \{}\FunctionTok{install.packages}\NormalTok{(}\StringTok{"sf"}\NormalTok{); }\FunctionTok{require}\NormalTok{(sf)\}}
\ControlFlowTok{if}\NormalTok{(}\SpecialCharTok{!}\FunctionTok{require}\NormalTok{(tidyverse)) \{}\FunctionTok{install.packages}\NormalTok{(}\StringTok{"tidyverse"}\NormalTok{); }\FunctionTok{require}\NormalTok{(tidyverse)\}}
\ControlFlowTok{if}\NormalTok{(}\SpecialCharTok{!}\FunctionTok{require}\NormalTok{(sfarrow)) \{}\FunctionTok{install.packages}\NormalTok{(}\StringTok{"sfarrow"}\NormalTok{); }\FunctionTok{require}\NormalTok{(sfarrow)\}}
\ControlFlowTok{if}\NormalTok{(}\SpecialCharTok{!}\FunctionTok{require}\NormalTok{(readxl)) \{}\FunctionTok{install.packages}\NormalTok{(}\StringTok{"readxl"}\NormalTok{); }\FunctionTok{require}\NormalTok{(readxl)\}}
\ControlFlowTok{if}\NormalTok{(}\SpecialCharTok{!}\FunctionTok{require}\NormalTok{(raster)) \{}\FunctionTok{install.packages}\NormalTok{(}\StringTok{"raster"}\NormalTok{); }\FunctionTok{require}\NormalTok{(raster)\}}
\ControlFlowTok{if}\NormalTok{(}\SpecialCharTok{!}\FunctionTok{require}\NormalTok{(fasterize)) \{}\FunctionTok{install.packages}\NormalTok{(}\StringTok{"fasterize"}\NormalTok{); }\FunctionTok{require}\NormalTok{(fasterize)\}}

\CommentTok{\# templates {-}{-}{-}{-}}
\NormalTok{template100}\OtherTok{=}\FunctionTok{rast}\NormalTok{(}\StringTok{"./Templates/TemplateRasters/LV100m\_10km.tif"}\NormalTok{)}
\NormalTok{template10}\OtherTok{=}\FunctionTok{rast}\NormalTok{(}\StringTok{"./Templates/TemplateRasters/LV10m\_10km.tif"}\NormalTok{)}
\NormalTok{rastrs10}\OtherTok{=}\FunctionTok{raster}\NormalTok{(template10)}

\NormalTok{nulls10}\OtherTok{=}\FunctionTok{rast}\NormalTok{(}\StringTok{"./Templates/TemplateRasters/nulls\_LV10m\_10km.tif"}\NormalTok{)}
\NormalTok{nulls100}\OtherTok{=}\FunctionTok{rast}\NormalTok{(}\StringTok{"./Templates/TemplateRasters/nulls\_LV100m\_10km.tif"}\NormalTok{)}

\CommentTok{\# codes {-}{-}{-}{-}}
\NormalTok{kodi}\OtherTok{=}\FunctionTok{read\_excel}\NormalTok{(}\StringTok{"./Geodata/2024/LAD/KulturuKodi\_2024.xlsx"}\NormalTok{)}
\NormalTok{kodi}\SpecialCharTok{$}\NormalTok{kods}\OtherTok{=}\FunctionTok{as.character}\NormalTok{(kodi}\SpecialCharTok{$}\NormalTok{kods)}
\CommentTok{\# LAD {-}{-}{-}{-}}
\NormalTok{lad}\OtherTok{=}\NormalTok{sfarrow}\SpecialCharTok{::}\FunctionTok{st\_read\_parquet}\NormalTok{(}\StringTok{"./Geodata/2024/LAD/Lauki\_2024.parquet"}\NormalTok{)}
\NormalTok{lad}\SpecialCharTok{$}\NormalTok{yes}\OtherTok{=}\DecValTok{1}
\NormalTok{lad}\OtherTok{=}\NormalTok{lad }\SpecialCharTok{\%\textgreater{}\%} 
 \FunctionTok{left\_join}\NormalTok{(kodi,}\AttributeTok{by=}\FunctionTok{c}\NormalTok{(}\StringTok{"PRODUCT\_CODE"}\OtherTok{=}\StringTok{"kods"}\NormalTok{))}

\CommentTok{\# simple landscape {-}{-}{-}{-}}
\NormalTok{simple\_landscape}\OtherTok{=}\FunctionTok{rast}\NormalTok{(}\StringTok{"RasterGrids\_10m/2024/Ainava\_vienk\_mask.tif"}\NormalTok{)}


\CommentTok{\# Wetlands\_ReedSedgeRushBeds\_cell.tif   egv\_474 {-}{-}{-}{-}}
\NormalTok{reedsedgerush}\OtherTok{=}\FunctionTok{ifel}\NormalTok{(simple\_landscape}\SpecialCharTok{==}\DecValTok{720}\NormalTok{,}\DecValTok{1}\NormalTok{,}\DecValTok{0}\NormalTok{)}

\NormalTok{i2e\_rez}\OtherTok{=}\NormalTok{egvtools}\SpecialCharTok{::}\FunctionTok{input2egv}\NormalTok{(}\AttributeTok{input=}\NormalTok{reedsedgerush,}
              \AttributeTok{egv\_template=} \StringTok{"./Templates/TemplateRasters/LV100m\_10km.tif"}\NormalTok{,}
              \AttributeTok{summary\_function =} \StringTok{"average"}\NormalTok{,}
              \AttributeTok{missing\_job =} \StringTok{"FillOutput"}\NormalTok{,}
              \AttributeTok{outlocation =} \StringTok{"./RasterGrids\_100m/2024/RAW/"}\NormalTok{,}
              \AttributeTok{outfilename =} \StringTok{"Wetlands\_ReedSedgeRushBeds\_cell.tif"}\NormalTok{,}
              \AttributeTok{layername =} \StringTok{"egv\_474"}\NormalTok{,}
              \AttributeTok{idw\_weight =} \DecValTok{2}\NormalTok{,}
              \AttributeTok{plot\_gaps =} \ConstantTok{FALSE}\NormalTok{,}\AttributeTok{plot\_final =} \ConstantTok{TRUE}\NormalTok{)}
\NormalTok{i2e\_rez}
\FunctionTok{rm}\NormalTok{(reedsedgerush)}
\FunctionTok{rm}\NormalTok{(i2e\_rez)}
\FunctionTok{rm}\NormalTok{(simple\_landscape)}

\CommentTok{\# standardisation {-}{-}{-}{-}}
\ControlFlowTok{if}\NormalTok{(}\SpecialCharTok{!}\FunctionTok{require}\NormalTok{(terra)) \{}\FunctionTok{install.packages}\NormalTok{(}\StringTok{"terra"}\NormalTok{); }\FunctionTok{require}\NormalTok{(terra)\}}
\ControlFlowTok{if}\NormalTok{(}\SpecialCharTok{!}\FunctionTok{require}\NormalTok{(tidyverse)) \{}\FunctionTok{install.packages}\NormalTok{(}\StringTok{"tidyverse"}\NormalTok{); }\FunctionTok{require}\NormalTok{(tidyverse)\}}

\NormalTok{nosaukums}\OtherTok{=}\StringTok{"Wetlands\_ReedSedgeRushBeds\_cell.tif"}
\NormalTok{ielasisanas\_cels}\OtherTok{=}\FunctionTok{paste0}\NormalTok{(}\StringTok{"./RasterGrids\_100m/2024/RAW/"}\NormalTok{,nosaukums)}
\NormalTok{saglabasanas\_cels}\OtherTok{=}\FunctionTok{paste0}\NormalTok{(}\StringTok{"./RasterGrids\_100m/2024/Scaled/"}\NormalTok{,nosaukums)}
\NormalTok{slanis}\OtherTok{=}\FunctionTok{rast}\NormalTok{(ielasisanas\_cels)}
\NormalTok{videjais}\OtherTok{=}\FunctionTok{global}\NormalTok{(slanis,}\AttributeTok{fun=}\StringTok{"mean"}\NormalTok{,}\AttributeTok{na.rm=}\ConstantTok{TRUE}\NormalTok{)}
\NormalTok{centrets}\OtherTok{=}\NormalTok{slanis}\SpecialCharTok{{-}}\NormalTok{videjais[,}\DecValTok{1}\NormalTok{]}
\NormalTok{standartnovirze}\OtherTok{=}\NormalTok{terra}\SpecialCharTok{::}\FunctionTok{global}\NormalTok{(centrets,}\AttributeTok{fun=}\StringTok{"rms"}\NormalTok{,}\AttributeTok{na.rm=}\ConstantTok{TRUE}\NormalTok{)}
\NormalTok{merogots}\OtherTok{=}\NormalTok{centrets}\SpecialCharTok{/}\NormalTok{standartnovirze[,}\DecValTok{1}\NormalTok{]}
\FunctionTok{writeRaster}\NormalTok{(merogots,}
      \AttributeTok{filename=}\NormalTok{saglabasanas\_cels,}
      \AttributeTok{overwrite=}\ConstantTok{TRUE}\NormalTok{)}
\end{Highlighting}
\end{Shaded}

\section{Wetlands\_ReedSedgeRushBeds\_r500}\label{ch06.475}

\textbf{filename:} \texttt{Wetlands\_ReedSedgeRushBeds\_r500.tif}

\textbf{layername:} \texttt{egv\_475}

\textbf{English name:} Fractional cover of Reed-, Sedge-, Rush-, Beds within the 0.5
km landscape

\textbf{Latvian name:} Niedrāju, grīslāju, meldrāju platības īpatsvars 0,5 km ainavā

\textbf{Procedure:} The cover fraction within a radius of 500 m around the analysis grid cell is
calculated as the area-weighted sum of the \hyperref[ch06.474]{analysis cells} inside the
buffer, using the workflow \texttt{egvtools::radius\_function()}. During the calculation of the landscape metric,
inverse distance weighted (power = 2) gap filling on the output is applied
to ensure no missing values at the edges. Then the layer is rewritten to set
its name. Finally, the layer is standardised by subtracting the arithmetic
mean and dividing by the root mean squared error.

\begin{Shaded}
\begin{Highlighting}[]
\CommentTok{\# libs {-}{-}{-}{-}}
\ControlFlowTok{if}\NormalTok{(}\SpecialCharTok{!}\FunctionTok{require}\NormalTok{(terra)) \{}\FunctionTok{install.packages}\NormalTok{(}\StringTok{"terra"}\NormalTok{); }\FunctionTok{require}\NormalTok{(terra)\}}
\ControlFlowTok{if}\NormalTok{(}\SpecialCharTok{!}\FunctionTok{require}\NormalTok{(egvtools)) \{remotes}\SpecialCharTok{::}\FunctionTok{install\_github}\NormalTok{(}\StringTok{"aavotins/egvtools"}\NormalTok{); }\FunctionTok{require}\NormalTok{(egvtools)\}}


\CommentTok{\# Templates {-}{-}{-}{-}{-}}
\NormalTok{template100}\OtherTok{=}\FunctionTok{rast}\NormalTok{(}\StringTok{"./Templates/TemplateRasters/LV100m\_10km.tif"}\NormalTok{)}

\CommentTok{\# radii {-}{-}{-}{-}}
\FunctionTok{radius\_function}\NormalTok{(}
 \AttributeTok{kvadrati\_path =} \StringTok{"./Templates/TemplateGrids/tiles/"}\NormalTok{,}
 \AttributeTok{radii\_path   =} \StringTok{"./Templates/TemplateGridPoints/tiles/"}\NormalTok{,}
 \AttributeTok{tikls100\_path =} \StringTok{"./Templates/TemplateGrids/tikls100\_sauzeme.parquet"}\NormalTok{,}
 \AttributeTok{template\_path =} \StringTok{"./Templates/TemplateRasters/LV100m\_10km.tif"}\NormalTok{,}
 \AttributeTok{input\_layers  =} \FunctionTok{c}\NormalTok{(}\StringTok{"./RasterGrids\_100m/2024/RAW/Wetlands\_ReedSedgeRushBeds\_cell.tif"}\NormalTok{),}
 \AttributeTok{layer\_prefixes =} \FunctionTok{c}\NormalTok{(}\StringTok{"Wetlands\_ReedSedgeRushBeds"}\NormalTok{),}
 \AttributeTok{output\_dir   =} \StringTok{"./RasterGrids\_100m/2024/RAW/"}\NormalTok{,}
 \AttributeTok{n\_workers   =} \DecValTok{12}\NormalTok{,}
 \AttributeTok{radii     =} \FunctionTok{c}\NormalTok{(}\StringTok{"r500"}\NormalTok{),}
 \AttributeTok{radius\_mode  =} \StringTok{"sparse"}\NormalTok{,}
 \AttributeTok{extract\_fun  =} \StringTok{"mean"}\NormalTok{,}
 \AttributeTok{fill\_missing  =} \ConstantTok{TRUE}\NormalTok{,}
 \AttributeTok{IDW\_weight   =} \DecValTok{2}\NormalTok{,}
 \AttributeTok{future\_max\_size =} \DecValTok{20} \SpecialCharTok{*} \DecValTok{1024}\SpecialCharTok{\^{}}\DecValTok{3}\NormalTok{)}


\CommentTok{\# Wetlands\_ReedSedgeRushBeds\_r500.tif   egv\_475}
\NormalTok{slanis}\OtherTok{=}\FunctionTok{rast}\NormalTok{(}\StringTok{"./RasterGrids\_100m/2024/RAW/Wetlands\_ReedSedgeRushBeds\_r500.tif"}\NormalTok{)}
\FunctionTok{names}\NormalTok{(slanis)}\OtherTok{=}\StringTok{"egv\_475"}
\NormalTok{slanis2}\OtherTok{=}\FunctionTok{project}\NormalTok{(slanis,template100)}
\FunctionTok{writeRaster}\NormalTok{(slanis2,}
      \StringTok{"./RasterGrids\_100m/2024/RAW/Wetlands\_ReedSedgeRushBeds\_r500.tif"}\NormalTok{,}
      \AttributeTok{overwrite=}\ConstantTok{TRUE}\NormalTok{)}

\CommentTok{\# standardisation {-}{-}{-}{-}}
\ControlFlowTok{if}\NormalTok{(}\SpecialCharTok{!}\FunctionTok{require}\NormalTok{(terra)) \{}\FunctionTok{install.packages}\NormalTok{(}\StringTok{"terra"}\NormalTok{); }\FunctionTok{require}\NormalTok{(terra)\}}
\ControlFlowTok{if}\NormalTok{(}\SpecialCharTok{!}\FunctionTok{require}\NormalTok{(tidyverse)) \{}\FunctionTok{install.packages}\NormalTok{(}\StringTok{"tidyverse"}\NormalTok{); }\FunctionTok{require}\NormalTok{(tidyverse)\}}

\NormalTok{nosaukums}\OtherTok{=}\StringTok{"Wetlands\_ReedSedgeRushBeds\_r500.tif"}
\NormalTok{ielasisanas\_cels}\OtherTok{=}\FunctionTok{paste0}\NormalTok{(}\StringTok{"./RasterGrids\_100m/2024/RAW/"}\NormalTok{,nosaukums)}
\NormalTok{saglabasanas\_cels}\OtherTok{=}\FunctionTok{paste0}\NormalTok{(}\StringTok{"./RasterGrids\_100m/2024/Scaled/"}\NormalTok{,nosaukums)}
\NormalTok{slanis}\OtherTok{=}\FunctionTok{rast}\NormalTok{(ielasisanas\_cels)}
\NormalTok{videjais}\OtherTok{=}\FunctionTok{global}\NormalTok{(slanis,}\AttributeTok{fun=}\StringTok{"mean"}\NormalTok{,}\AttributeTok{na.rm=}\ConstantTok{TRUE}\NormalTok{)}
\NormalTok{centrets}\OtherTok{=}\NormalTok{slanis}\SpecialCharTok{{-}}\NormalTok{videjais[,}\DecValTok{1}\NormalTok{]}
\NormalTok{standartnovirze}\OtherTok{=}\NormalTok{terra}\SpecialCharTok{::}\FunctionTok{global}\NormalTok{(centrets,}\AttributeTok{fun=}\StringTok{"rms"}\NormalTok{,}\AttributeTok{na.rm=}\ConstantTok{TRUE}\NormalTok{)}
\NormalTok{merogots}\OtherTok{=}\NormalTok{centrets}\SpecialCharTok{/}\NormalTok{standartnovirze[,}\DecValTok{1}\NormalTok{]}
\FunctionTok{writeRaster}\NormalTok{(merogots,}
      \AttributeTok{filename=}\NormalTok{saglabasanas\_cels,}
      \AttributeTok{overwrite=}\ConstantTok{TRUE}\NormalTok{)}
\end{Highlighting}
\end{Shaded}

\section{Wetlands\_ReedSedgeRushBeds\_r1250}\label{ch06.476}

\textbf{filename:} \texttt{Wetlands\_ReedSedgeRushBeds\_r1250.tif}

\textbf{layername:} \texttt{egv\_476}

\textbf{English name:} Fractional cover of Reed-, Sedge-, Rush-, Beds within the 1.25
km landscape

\textbf{Latvian name:} Niedrāju, grīslāju, meldrāju platības īpatsvars 1,25 km ainavā

\textbf{Procedure:} The cover fraction within a radius of 1250 m around the analysis grid cell
is calculated as the area-weighted sum of the \hyperref[ch06.474]{analysis cells} inside
the buffer, using the workflow \texttt{egvtools::radius\_function()}. During the calculation of the landscape
metric, inverse distance weighted (power = 2) gap filling on the output is
applied to ensure no missing values at the edges. Then the layer is
rewritten to set its name. Finally, the layer is standardised by
subtracting the arithmetic mean and dividing by the root mean squared error.

\begin{Shaded}
\begin{Highlighting}[]
\CommentTok{\# libs {-}{-}{-}{-}}
\ControlFlowTok{if}\NormalTok{(}\SpecialCharTok{!}\FunctionTok{require}\NormalTok{(terra)) \{}\FunctionTok{install.packages}\NormalTok{(}\StringTok{"terra"}\NormalTok{); }\FunctionTok{require}\NormalTok{(terra)\}}
\ControlFlowTok{if}\NormalTok{(}\SpecialCharTok{!}\FunctionTok{require}\NormalTok{(egvtools)) \{remotes}\SpecialCharTok{::}\FunctionTok{install\_github}\NormalTok{(}\StringTok{"aavotins/egvtools"}\NormalTok{); }\FunctionTok{require}\NormalTok{(egvtools)\}}


\CommentTok{\# Templates {-}{-}{-}{-}{-}}
\NormalTok{template100}\OtherTok{=}\FunctionTok{rast}\NormalTok{(}\StringTok{"./Templates/TemplateRasters/LV100m\_10km.tif"}\NormalTok{)}

\CommentTok{\# radii {-}{-}{-}{-}}
\FunctionTok{radius\_function}\NormalTok{(}
 \AttributeTok{kvadrati\_path =} \StringTok{"./Templates/TemplateGrids/tiles/"}\NormalTok{,}
 \AttributeTok{radii\_path   =} \StringTok{"./Templates/TemplateGridPoints/tiles/"}\NormalTok{,}
 \AttributeTok{tikls100\_path =} \StringTok{"./Templates/TemplateGrids/tikls100\_sauzeme.parquet"}\NormalTok{,}
 \AttributeTok{template\_path =} \StringTok{"./Templates/TemplateRasters/LV100m\_10km.tif"}\NormalTok{,}
 \AttributeTok{input\_layers  =} \FunctionTok{c}\NormalTok{(}\StringTok{"./RasterGrids\_100m/2024/RAW/Wetlands\_ReedSedgeRushBeds\_cell.tif"}\NormalTok{),}
 \AttributeTok{layer\_prefixes =} \FunctionTok{c}\NormalTok{(}\StringTok{"Wetlands\_ReedSedgeRushBeds"}\NormalTok{),}
 \AttributeTok{output\_dir   =} \StringTok{"./RasterGrids\_100m/2024/RAW/"}\NormalTok{,}
 \AttributeTok{n\_workers   =} \DecValTok{12}\NormalTok{,}
 \AttributeTok{radii     =} \FunctionTok{c}\NormalTok{(}\StringTok{"r1250"}\NormalTok{),}
 \AttributeTok{radius\_mode  =} \StringTok{"sparse"}\NormalTok{,}
 \AttributeTok{extract\_fun  =} \StringTok{"mean"}\NormalTok{,}
 \AttributeTok{fill\_missing  =} \ConstantTok{TRUE}\NormalTok{,}
 \AttributeTok{IDW\_weight   =} \DecValTok{2}\NormalTok{,}
 \AttributeTok{future\_max\_size =} \DecValTok{20} \SpecialCharTok{*} \DecValTok{1024}\SpecialCharTok{\^{}}\DecValTok{3}\NormalTok{)}


\CommentTok{\# Wetlands\_ReedSedgeRushBeds\_r1250.tif  egv\_476}
\NormalTok{slanis}\OtherTok{=}\FunctionTok{rast}\NormalTok{(}\StringTok{"./RasterGrids\_100m/2024/RAW/Wetlands\_ReedSedgeRushBeds\_r1250.tif"}\NormalTok{)}
\FunctionTok{names}\NormalTok{(slanis)}\OtherTok{=}\StringTok{"egv\_476"}
\NormalTok{slanis2}\OtherTok{=}\FunctionTok{project}\NormalTok{(slanis,template100)}
\FunctionTok{writeRaster}\NormalTok{(slanis2,}
      \StringTok{"./RasterGrids\_100m/2024/RAW/Wetlands\_ReedSedgeRushBeds\_r1250.tif"}\NormalTok{,}
      \AttributeTok{overwrite=}\ConstantTok{TRUE}\NormalTok{)}

\CommentTok{\# standardisation {-}{-}{-}{-}}
\ControlFlowTok{if}\NormalTok{(}\SpecialCharTok{!}\FunctionTok{require}\NormalTok{(terra)) \{}\FunctionTok{install.packages}\NormalTok{(}\StringTok{"terra"}\NormalTok{); }\FunctionTok{require}\NormalTok{(terra)\}}
\ControlFlowTok{if}\NormalTok{(}\SpecialCharTok{!}\FunctionTok{require}\NormalTok{(tidyverse)) \{}\FunctionTok{install.packages}\NormalTok{(}\StringTok{"tidyverse"}\NormalTok{); }\FunctionTok{require}\NormalTok{(tidyverse)\}}

\NormalTok{nosaukums}\OtherTok{=}\StringTok{"Wetlands\_ReedSedgeRushBeds\_r1250.tif"}
\NormalTok{ielasisanas\_cels}\OtherTok{=}\FunctionTok{paste0}\NormalTok{(}\StringTok{"./RasterGrids\_100m/2024/RAW/"}\NormalTok{,nosaukums)}
\NormalTok{saglabasanas\_cels}\OtherTok{=}\FunctionTok{paste0}\NormalTok{(}\StringTok{"./RasterGrids\_100m/2024/Scaled/"}\NormalTok{,nosaukums)}
\NormalTok{slanis}\OtherTok{=}\FunctionTok{rast}\NormalTok{(ielasisanas\_cels)}
\NormalTok{videjais}\OtherTok{=}\FunctionTok{global}\NormalTok{(slanis,}\AttributeTok{fun=}\StringTok{"mean"}\NormalTok{,}\AttributeTok{na.rm=}\ConstantTok{TRUE}\NormalTok{)}
\NormalTok{centrets}\OtherTok{=}\NormalTok{slanis}\SpecialCharTok{{-}}\NormalTok{videjais[,}\DecValTok{1}\NormalTok{]}
\NormalTok{standartnovirze}\OtherTok{=}\NormalTok{terra}\SpecialCharTok{::}\FunctionTok{global}\NormalTok{(centrets,}\AttributeTok{fun=}\StringTok{"rms"}\NormalTok{,}\AttributeTok{na.rm=}\ConstantTok{TRUE}\NormalTok{)}
\NormalTok{merogots}\OtherTok{=}\NormalTok{centrets}\SpecialCharTok{/}\NormalTok{standartnovirze[,}\DecValTok{1}\NormalTok{]}
\FunctionTok{writeRaster}\NormalTok{(merogots,}
      \AttributeTok{filename=}\NormalTok{saglabasanas\_cels,}
      \AttributeTok{overwrite=}\ConstantTok{TRUE}\NormalTok{)}
\end{Highlighting}
\end{Shaded}

\section{Wetlands\_ReedSedgeRushBeds\_r3000}\label{ch06.477}

\textbf{filename:} \texttt{Wetlands\_ReedSedgeRushBeds\_r3000.tif}

\textbf{layername:} \texttt{egv\_477}

\textbf{English name:} Fractional cover of Reed-, Sedge-, Rush-, Beds within the 3 km
landscape

\textbf{Latvian name:} Niedrāju, grīslāju, meldrāju platības īpatsvars 3 km ainavā

\textbf{Procedure:} The cover fraction within a radius of 3000 m around the analysis grid cell
is calculated as the area-weighted sum of the \hyperref[ch06.474]{analysis cells} inside
the buffer, using the workflow \texttt{egvtools::radius\_function()}. During the calculation of the landscape
metric, inverse distance weighted (power = 2) gap filling on the output is
applied to ensure no missing values at the edges. Then the layer is
rewritten to set its name. Finally, the layer is standardised by
subtracting the arithmetic mean and dividing by the root mean squared error.

\begin{Shaded}
\begin{Highlighting}[]
\CommentTok{\# libs {-}{-}{-}{-}}
\ControlFlowTok{if}\NormalTok{(}\SpecialCharTok{!}\FunctionTok{require}\NormalTok{(terra)) \{}\FunctionTok{install.packages}\NormalTok{(}\StringTok{"terra"}\NormalTok{); }\FunctionTok{require}\NormalTok{(terra)\}}
\ControlFlowTok{if}\NormalTok{(}\SpecialCharTok{!}\FunctionTok{require}\NormalTok{(egvtools)) \{remotes}\SpecialCharTok{::}\FunctionTok{install\_github}\NormalTok{(}\StringTok{"aavotins/egvtools"}\NormalTok{); }\FunctionTok{require}\NormalTok{(egvtools)\}}


\CommentTok{\# Templates {-}{-}{-}{-}{-}}
\NormalTok{template100}\OtherTok{=}\FunctionTok{rast}\NormalTok{(}\StringTok{"./Templates/TemplateRasters/LV100m\_10km.tif"}\NormalTok{)}

\CommentTok{\# radii {-}{-}{-}{-}}
\FunctionTok{radius\_function}\NormalTok{(}
 \AttributeTok{kvadrati\_path =} \StringTok{"./Templates/TemplateGrids/tiles/"}\NormalTok{,}
 \AttributeTok{radii\_path   =} \StringTok{"./Templates/TemplateGridPoints/tiles/"}\NormalTok{,}
 \AttributeTok{tikls100\_path =} \StringTok{"./Templates/TemplateGrids/tikls100\_sauzeme.parquet"}\NormalTok{,}
 \AttributeTok{template\_path =} \StringTok{"./Templates/TemplateRasters/LV100m\_10km.tif"}\NormalTok{,}
 \AttributeTok{input\_layers  =} \FunctionTok{c}\NormalTok{(}\StringTok{"./RasterGrids\_100m/2024/RAW/Wetlands\_ReedSedgeRushBeds\_cell.tif"}\NormalTok{),}
 \AttributeTok{layer\_prefixes =} \FunctionTok{c}\NormalTok{(}\StringTok{"Wetlands\_ReedSedgeRushBeds"}\NormalTok{),}
 \AttributeTok{output\_dir   =} \StringTok{"./RasterGrids\_100m/2024/RAW/"}\NormalTok{,}
 \AttributeTok{n\_workers   =} \DecValTok{12}\NormalTok{,}
 \AttributeTok{radii     =} \FunctionTok{c}\NormalTok{(}\StringTok{"r3000"}\NormalTok{),}
 \AttributeTok{radius\_mode  =} \StringTok{"sparse"}\NormalTok{,}
 \AttributeTok{extract\_fun  =} \StringTok{"mean"}\NormalTok{,}
 \AttributeTok{fill\_missing  =} \ConstantTok{TRUE}\NormalTok{,}
 \AttributeTok{IDW\_weight   =} \DecValTok{2}\NormalTok{,}
 \AttributeTok{future\_max\_size =} \DecValTok{20} \SpecialCharTok{*} \DecValTok{1024}\SpecialCharTok{\^{}}\DecValTok{3}\NormalTok{)}


\CommentTok{\# Wetlands\_ReedSedgeRushBeds\_r3000.tif  egv\_477}
\NormalTok{slanis}\OtherTok{=}\FunctionTok{rast}\NormalTok{(}\StringTok{"./RasterGrids\_100m/2024/RAW/Wetlands\_ReedSedgeRushBeds\_r3000.tif"}\NormalTok{)}
\FunctionTok{names}\NormalTok{(slanis)}\OtherTok{=}\StringTok{"egv\_477"}
\NormalTok{slanis2}\OtherTok{=}\FunctionTok{project}\NormalTok{(slanis,template100)}
\FunctionTok{writeRaster}\NormalTok{(slanis2,}
      \StringTok{"./RasterGrids\_100m/2024/RAW/Wetlands\_ReedSedgeRushBeds\_r3000.tif"}\NormalTok{,}
      \AttributeTok{overwrite=}\ConstantTok{TRUE}\NormalTok{)}

\CommentTok{\# standardisation {-}{-}{-}{-}}
\ControlFlowTok{if}\NormalTok{(}\SpecialCharTok{!}\FunctionTok{require}\NormalTok{(terra)) \{}\FunctionTok{install.packages}\NormalTok{(}\StringTok{"terra"}\NormalTok{); }\FunctionTok{require}\NormalTok{(terra)\}}
\ControlFlowTok{if}\NormalTok{(}\SpecialCharTok{!}\FunctionTok{require}\NormalTok{(tidyverse)) \{}\FunctionTok{install.packages}\NormalTok{(}\StringTok{"tidyverse"}\NormalTok{); }\FunctionTok{require}\NormalTok{(tidyverse)\}}

\NormalTok{nosaukums}\OtherTok{=}\StringTok{"Wetlands\_ReedSedgeRushBeds\_r3000.tif"}
\NormalTok{ielasisanas\_cels}\OtherTok{=}\FunctionTok{paste0}\NormalTok{(}\StringTok{"./RasterGrids\_100m/2024/RAW/"}\NormalTok{,nosaukums)}
\NormalTok{saglabasanas\_cels}\OtherTok{=}\FunctionTok{paste0}\NormalTok{(}\StringTok{"./RasterGrids\_100m/2024/Scaled/"}\NormalTok{,nosaukums)}
\NormalTok{slanis}\OtherTok{=}\FunctionTok{rast}\NormalTok{(ielasisanas\_cels)}
\NormalTok{videjais}\OtherTok{=}\FunctionTok{global}\NormalTok{(slanis,}\AttributeTok{fun=}\StringTok{"mean"}\NormalTok{,}\AttributeTok{na.rm=}\ConstantTok{TRUE}\NormalTok{)}
\NormalTok{centrets}\OtherTok{=}\NormalTok{slanis}\SpecialCharTok{{-}}\NormalTok{videjais[,}\DecValTok{1}\NormalTok{]}
\NormalTok{standartnovirze}\OtherTok{=}\NormalTok{terra}\SpecialCharTok{::}\FunctionTok{global}\NormalTok{(centrets,}\AttributeTok{fun=}\StringTok{"rms"}\NormalTok{,}\AttributeTok{na.rm=}\ConstantTok{TRUE}\NormalTok{)}
\NormalTok{merogots}\OtherTok{=}\NormalTok{centrets}\SpecialCharTok{/}\NormalTok{standartnovirze[,}\DecValTok{1}\NormalTok{]}
\FunctionTok{writeRaster}\NormalTok{(merogots,}
      \AttributeTok{filename=}\NormalTok{saglabasanas\_cels,}
      \AttributeTok{overwrite=}\ConstantTok{TRUE}\NormalTok{)}
\end{Highlighting}
\end{Shaded}

\section{Wetlands\_ReedSedgeRushBeds\_r10000}\label{ch06.478}

\textbf{filename:} \texttt{Wetlands\_ReedSedgeRushBeds\_r10000.tif}

\textbf{layername:} \texttt{egv\_478}

\textbf{English name:} Fractional cover of Reed-, Sedge-, Rush-, Beds within the 10
km landscape

\textbf{Latvian name:} Niedrāju, grīslāju, meldrāju platības īpatsvars 10 km ainavā

\textbf{Procedure:} The cover fraction within a radius of 10000 m around the analysis grid cell
is calculated as the area-weighted sum of the \hyperref[ch06.474]{analysis cells} inside
the buffer, using the workflow \texttt{egvtools::radius\_function()}. During the calculation of the landscape
metric, inverse distance weighted (power = 2) gap filling on the output is
applied to ensure no missing values at the edges. Then the layer is
rewritten to set its name. Finally, the layer is standardised by
subtracting the arithmetic mean and dividing by the root mean squared error.

\begin{Shaded}
\begin{Highlighting}[]
\CommentTok{\# libs {-}{-}{-}{-}}
\ControlFlowTok{if}\NormalTok{(}\SpecialCharTok{!}\FunctionTok{require}\NormalTok{(terra)) \{}\FunctionTok{install.packages}\NormalTok{(}\StringTok{"terra"}\NormalTok{); }\FunctionTok{require}\NormalTok{(terra)\}}
\ControlFlowTok{if}\NormalTok{(}\SpecialCharTok{!}\FunctionTok{require}\NormalTok{(egvtools)) \{remotes}\SpecialCharTok{::}\FunctionTok{install\_github}\NormalTok{(}\StringTok{"aavotins/egvtools"}\NormalTok{); }\FunctionTok{require}\NormalTok{(egvtools)\}}


\CommentTok{\# Templates {-}{-}{-}{-}{-}}
\NormalTok{template100}\OtherTok{=}\FunctionTok{rast}\NormalTok{(}\StringTok{"./Templates/TemplateRasters/LV100m\_10km.tif"}\NormalTok{)}

\CommentTok{\# radii {-}{-}{-}{-}}
\FunctionTok{radius\_function}\NormalTok{(}
 \AttributeTok{kvadrati\_path =} \StringTok{"./Templates/TemplateGrids/tiles/"}\NormalTok{,}
 \AttributeTok{radii\_path   =} \StringTok{"./Templates/TemplateGridPoints/tiles/"}\NormalTok{,}
 \AttributeTok{tikls100\_path =} \StringTok{"./Templates/TemplateGrids/tikls100\_sauzeme.parquet"}\NormalTok{,}
 \AttributeTok{template\_path =} \StringTok{"./Templates/TemplateRasters/LV100m\_10km.tif"}\NormalTok{,}
 \AttributeTok{input\_layers  =} \FunctionTok{c}\NormalTok{(}\StringTok{"./RasterGrids\_100m/2024/RAW/Wetlands\_ReedSedgeRushBeds\_cell.tif"}\NormalTok{),}
 \AttributeTok{layer\_prefixes =} \FunctionTok{c}\NormalTok{(}\StringTok{"Wetlands\_ReedSedgeRushBeds"}\NormalTok{),}
 \AttributeTok{output\_dir   =} \StringTok{"./RasterGrids\_100m/2024/RAW/"}\NormalTok{,}
 \AttributeTok{n\_workers   =} \DecValTok{12}\NormalTok{,}
 \AttributeTok{radii     =} \FunctionTok{c}\NormalTok{(}\StringTok{"r10000"}\NormalTok{),}
 \AttributeTok{radius\_mode  =} \StringTok{"sparse"}\NormalTok{,}
 \AttributeTok{extract\_fun  =} \StringTok{"mean"}\NormalTok{,}
 \AttributeTok{fill\_missing  =} \ConstantTok{TRUE}\NormalTok{,}
 \AttributeTok{IDW\_weight   =} \DecValTok{2}\NormalTok{,}
 \AttributeTok{future\_max\_size =} \DecValTok{20} \SpecialCharTok{*} \DecValTok{1024}\SpecialCharTok{\^{}}\DecValTok{3}\NormalTok{)}


\CommentTok{\# Wetlands\_ReedSedgeRushBeds\_r10000.tif egv\_478}
\NormalTok{slanis}\OtherTok{=}\FunctionTok{rast}\NormalTok{(}\StringTok{"./RasterGrids\_100m/2024/RAW/Wetlands\_ReedSedgeRushBeds\_r10000.tif"}\NormalTok{)}
\FunctionTok{names}\NormalTok{(slanis)}\OtherTok{=}\StringTok{"egv\_478"}
\NormalTok{slanis2}\OtherTok{=}\FunctionTok{project}\NormalTok{(slanis,template100)}
\FunctionTok{writeRaster}\NormalTok{(slanis2,}
      \StringTok{"./RasterGrids\_100m/2024/RAW/Wetlands\_ReedSedgeRushBeds\_r10000.tif"}\NormalTok{,}
      \AttributeTok{overwrite=}\ConstantTok{TRUE}\NormalTok{)}

\CommentTok{\# standardisation {-}{-}{-}{-}}
\ControlFlowTok{if}\NormalTok{(}\SpecialCharTok{!}\FunctionTok{require}\NormalTok{(terra)) \{}\FunctionTok{install.packages}\NormalTok{(}\StringTok{"terra"}\NormalTok{); }\FunctionTok{require}\NormalTok{(terra)\}}
\ControlFlowTok{if}\NormalTok{(}\SpecialCharTok{!}\FunctionTok{require}\NormalTok{(tidyverse)) \{}\FunctionTok{install.packages}\NormalTok{(}\StringTok{"tidyverse"}\NormalTok{); }\FunctionTok{require}\NormalTok{(tidyverse)\}}

\NormalTok{nosaukums}\OtherTok{=}\StringTok{"Wetlands\_ReedSedgeRushBeds\_r10000.tif"}
\NormalTok{ielasisanas\_cels}\OtherTok{=}\FunctionTok{paste0}\NormalTok{(}\StringTok{"./RasterGrids\_100m/2024/RAW/"}\NormalTok{,nosaukums)}
\NormalTok{saglabasanas\_cels}\OtherTok{=}\FunctionTok{paste0}\NormalTok{(}\StringTok{"./RasterGrids\_100m/2024/Scaled/"}\NormalTok{,nosaukums)}
\NormalTok{slanis}\OtherTok{=}\FunctionTok{rast}\NormalTok{(ielasisanas\_cels)}
\NormalTok{videjais}\OtherTok{=}\FunctionTok{global}\NormalTok{(slanis,}\AttributeTok{fun=}\StringTok{"mean"}\NormalTok{,}\AttributeTok{na.rm=}\ConstantTok{TRUE}\NormalTok{)}
\NormalTok{centrets}\OtherTok{=}\NormalTok{slanis}\SpecialCharTok{{-}}\NormalTok{videjais[,}\DecValTok{1}\NormalTok{]}
\NormalTok{standartnovirze}\OtherTok{=}\NormalTok{terra}\SpecialCharTok{::}\FunctionTok{global}\NormalTok{(centrets,}\AttributeTok{fun=}\StringTok{"rms"}\NormalTok{,}\AttributeTok{na.rm=}\ConstantTok{TRUE}\NormalTok{)}
\NormalTok{merogots}\OtherTok{=}\NormalTok{centrets}\SpecialCharTok{/}\NormalTok{standartnovirze[,}\DecValTok{1}\NormalTok{]}
\FunctionTok{writeRaster}\NormalTok{(merogots,}
      \AttributeTok{filename=}\NormalTok{saglabasanas\_cels,}
      \AttributeTok{overwrite=}\ConstantTok{TRUE}\NormalTok{)}
\end{Highlighting}
\end{Shaded}

\section{EO\_NDMI-LYmed-average\_cell}\label{ch06.479}

\textbf{filename:} \texttt{EO\_NDMI-LYmed-average\_cell.tif}

\textbf{layername:} \texttt{egv\_479}

\textbf{English name:} Median vegetation water content index (NDMI) for the last year
within the analysis cell (1 ha)

\textbf{Latvian name:} Mediānā pēdējā gada ūdens satura veģetācijā indeksa (NDMI)
vērtība analīzes šūnā (1 ha)

\textbf{Procedure:} Directly follows \hyperref[Ch04.13]{preprocessing}. The arithmetic mean value
at the analysis cell is calculated using the workflow \texttt{egvtools::input2egv()}. To
protect against potential data loss at edge cells, inverse distance
weighted (power = 2) gap filling is implemented. Finally, the layer is
standardised by subtracting the arithmetic mean and
dividing by the root mean squared error. The ``last year'' is 2024.

\begin{Shaded}
\begin{Highlighting}[]
\CommentTok{\# libs {-}{-}{-}{-}}
\ControlFlowTok{if}\NormalTok{(}\SpecialCharTok{!}\FunctionTok{require}\NormalTok{(egvtools)) \{remotes}\SpecialCharTok{::}\FunctionTok{install\_github}\NormalTok{(}\StringTok{"aavotins/egvtools"}\NormalTok{); }\FunctionTok{require}\NormalTok{(egvtools)\}}

\CommentTok{\# EO\_NDMI{-}LYmed{-}average\_cell.tif {-}{-}{-}{-}}
\NormalTok{egvrez}\OtherTok{=}\FunctionTok{input2egv}\NormalTok{(}\AttributeTok{input=}\StringTok{"./Geodata/2024/S2indices/Mosaics/EO\_NDMI{-}LYmedian.tif"}\NormalTok{,}
         \AttributeTok{egv\_template=} \StringTok{"./Templates/TemplateRasters/LV100m\_10km.tif"}\NormalTok{,}
         \AttributeTok{summary\_function =} \StringTok{"average"}\NormalTok{,}
         \AttributeTok{missing\_job =} \StringTok{"FillOutput"}\NormalTok{,}
         \AttributeTok{outlocation =} \StringTok{"./RasterGrids\_100m/2024/RAW/"}\NormalTok{,}
         \AttributeTok{outfilename =} \StringTok{"EO\_NDMI{-}LYmed{-}average\_cell.tif"}\NormalTok{,}
         \AttributeTok{layername =} \StringTok{"egv\_479"}\NormalTok{,}
         \AttributeTok{idw\_weight =} \DecValTok{2}\NormalTok{,}
         \AttributeTok{plot\_gaps =} \ConstantTok{FALSE}\NormalTok{,}
         \AttributeTok{plot\_final =} \ConstantTok{FALSE}\NormalTok{)}
\NormalTok{egvrez}

\CommentTok{\# standardisation {-}{-}{-}{-}}
\ControlFlowTok{if}\NormalTok{(}\SpecialCharTok{!}\FunctionTok{require}\NormalTok{(terra)) \{}\FunctionTok{install.packages}\NormalTok{(}\StringTok{"terra"}\NormalTok{); }\FunctionTok{require}\NormalTok{(terra)\}}
\ControlFlowTok{if}\NormalTok{(}\SpecialCharTok{!}\FunctionTok{require}\NormalTok{(tidyverse)) \{}\FunctionTok{install.packages}\NormalTok{(}\StringTok{"tidyverse"}\NormalTok{); }\FunctionTok{require}\NormalTok{(tidyverse)\}}

\NormalTok{nosaukums}\OtherTok{=}\StringTok{"EO\_NDMI{-}LYmed{-}average\_cell.tif"}
\NormalTok{ielasisanas\_cels}\OtherTok{=}\FunctionTok{paste0}\NormalTok{(}\StringTok{"./RasterGrids\_100m/2024/RAW/"}\NormalTok{,nosaukums)}
\NormalTok{saglabasanas\_cels}\OtherTok{=}\FunctionTok{paste0}\NormalTok{(}\StringTok{"./RasterGrids\_100m/2024/Scaled/"}\NormalTok{,nosaukums)}
\NormalTok{slanis}\OtherTok{=}\FunctionTok{rast}\NormalTok{(ielasisanas\_cels)}
\NormalTok{videjais}\OtherTok{=}\FunctionTok{global}\NormalTok{(slanis,}\AttributeTok{fun=}\StringTok{"mean"}\NormalTok{,}\AttributeTok{na.rm=}\ConstantTok{TRUE}\NormalTok{)}
\NormalTok{centrets}\OtherTok{=}\NormalTok{slanis}\SpecialCharTok{{-}}\NormalTok{videjais[,}\DecValTok{1}\NormalTok{]}
\NormalTok{standartnovirze}\OtherTok{=}\NormalTok{terra}\SpecialCharTok{::}\FunctionTok{global}\NormalTok{(centrets,}\AttributeTok{fun=}\StringTok{"rms"}\NormalTok{,}\AttributeTok{na.rm=}\ConstantTok{TRUE}\NormalTok{)}
\NormalTok{merogots}\OtherTok{=}\NormalTok{centrets}\SpecialCharTok{/}\NormalTok{standartnovirze[,}\DecValTok{1}\NormalTok{]}
\FunctionTok{writeRaster}\NormalTok{(merogots,}
      \AttributeTok{filename=}\NormalTok{saglabasanas\_cels,}
      \AttributeTok{overwrite=}\ConstantTok{TRUE}\NormalTok{)}
\end{Highlighting}
\end{Shaded}

\section{EO\_NDMI-LYmedian-iqr\_cell}\label{ch06.480}

\textbf{filename:} \texttt{EO\_NDMI-LYmedian-iqr\_cell.tif}

\textbf{layername:} \texttt{egv\_480}

\textbf{English name:} Spatial variability of last year's median vegetation water
content index (NDMI) within the analysis cell (1 ha)

\textbf{Latvian name:} Telpiskā variabilitāte pēdējā gada mediānajai ūdens satura
veģetācijā indeksa (NDMI) vērtībai analīzes šūnā (1 ha)

\textbf{Procedure:} Directly follows \hyperref[Ch04.13]{preprocessing}. The
workflow \texttt{egvtools::input2egv()} is used to calculate Q1 and Q3 for every cell.
To protect against potential data loss at the edges, inverse distance
weighted (power = 2) gap filling is implemented. Next, Q1 is subtracted from Q3.
Finally, the layer is standardised by subtracting the arithmetic mean and
dividing by the root mean squared error. The ``last year'' is 2024.

\begin{Shaded}
\begin{Highlighting}[]
\CommentTok{\# libs {-}{-}{-}{-}}
\ControlFlowTok{if}\NormalTok{(}\SpecialCharTok{!}\FunctionTok{require}\NormalTok{(egvtools)) \{remotes}\SpecialCharTok{::}\FunctionTok{install\_github}\NormalTok{(}\StringTok{"aavotins/egvtools"}\NormalTok{); }\FunctionTok{require}\NormalTok{(egvtools)\}}

\CommentTok{\# EO\_NDMI{-}LYmedian{-}iqr\_cell.tif {-}{-}{-}{-}}

\NormalTok{p25rez}\OtherTok{=}\FunctionTok{input2egv}\NormalTok{(}\AttributeTok{input=}\StringTok{"./Geodata/2024/S2indices/Mosaics/EO\_NDMI{-}LYmedian.tif"}\NormalTok{,}
         \AttributeTok{egv\_template=} \StringTok{"./Templates/TemplateRasters/LV100m\_10km.tif"}\NormalTok{,}
         \AttributeTok{summary\_function =} \StringTok{"q1"}\NormalTok{,}
         \AttributeTok{missing\_job =} \StringTok{"FillOutput"}\NormalTok{,}
         \AttributeTok{outlocation =} \StringTok{"./RasterGrids\_100m/2024/"}\NormalTok{,}
         \AttributeTok{outfilename =} \StringTok{"draza\_p25.tif"}\NormalTok{,}
         \AttributeTok{layername =} \StringTok{"egv\_480"}\NormalTok{,}
         \AttributeTok{idw\_weight =} \DecValTok{2}\NormalTok{,}
         \AttributeTok{plot\_gaps =} \ConstantTok{FALSE}\NormalTok{,}
         \AttributeTok{plot\_final =} \ConstantTok{FALSE}\NormalTok{)}
\NormalTok{p25rez\_r}\OtherTok{=}\FunctionTok{rast}\NormalTok{(}\StringTok{"./RasterGrids\_100m/2024/draza\_p25.tif"}\NormalTok{)}


\NormalTok{p75rez}\OtherTok{=}\FunctionTok{input2egv}\NormalTok{(}\AttributeTok{input=}\StringTok{"./Geodata/2024/S2indices/Mosaics/EO\_NDMI{-}LYmedian.tif"}\NormalTok{,}
         \AttributeTok{egv\_template=} \StringTok{"./Templates/TemplateRasters/LV100m\_10km.tif"}\NormalTok{,}
         \AttributeTok{summary\_function =} \StringTok{"q3"}\NormalTok{,}
         \AttributeTok{missing\_job =} \StringTok{"FillOutput"}\NormalTok{,}
         \AttributeTok{outlocation =} \StringTok{"./RasterGrids\_100m/2024/"}\NormalTok{,}
         \AttributeTok{outfilename =} \StringTok{"draza\_p75.tif"}\NormalTok{,}
         \AttributeTok{layername =} \StringTok{"egv\_480"}\NormalTok{,}
         \AttributeTok{idw\_weight =} \DecValTok{2}\NormalTok{,}
         \AttributeTok{plot\_gaps =} \ConstantTok{FALSE}\NormalTok{,}
         \AttributeTok{plot\_final =} \ConstantTok{FALSE}\NormalTok{)}
\NormalTok{p75rez\_r}\OtherTok{=}\FunctionTok{rast}\NormalTok{(}\StringTok{"./RasterGrids\_100m/2024/draza\_p75.tif"}\NormalTok{)}

\NormalTok{iqr\_rez}\OtherTok{=}\NormalTok{p75rez\_r}\SpecialCharTok{{-}}\NormalTok{p25rez\_r}
\NormalTok{iqr\_rez}
\FunctionTok{plot}\NormalTok{(iqr\_rez)}

\FunctionTok{writeRaster}\NormalTok{(iqr\_rez,}
      \StringTok{"./RasterGrids\_100m/2024/RAW/EO\_NDMI{-}LYmedian{-}iqr\_cell.tif"}\NormalTok{,}
      \AttributeTok{overwrite=}\ConstantTok{TRUE}\NormalTok{)}

\FunctionTok{unlink}\NormalTok{(}\StringTok{"./RasterGrids\_100m/2024/draza\_p75.tif"}\NormalTok{)}
\FunctionTok{unlink}\NormalTok{(}\StringTok{"./RasterGrids\_100m/2024/draza\_p25.tif"}\NormalTok{)}

\CommentTok{\# standardisation {-}{-}{-}{-}}
\ControlFlowTok{if}\NormalTok{(}\SpecialCharTok{!}\FunctionTok{require}\NormalTok{(terra)) \{}\FunctionTok{install.packages}\NormalTok{(}\StringTok{"terra"}\NormalTok{); }\FunctionTok{require}\NormalTok{(terra)\}}
\ControlFlowTok{if}\NormalTok{(}\SpecialCharTok{!}\FunctionTok{require}\NormalTok{(tidyverse)) \{}\FunctionTok{install.packages}\NormalTok{(}\StringTok{"tidyverse"}\NormalTok{); }\FunctionTok{require}\NormalTok{(tidyverse)\}}

\NormalTok{nosaukums}\OtherTok{=}\StringTok{"EO\_NDMI{-}LYmedian{-}iqr\_cell.tif"}
\NormalTok{ielasisanas\_cels}\OtherTok{=}\FunctionTok{paste0}\NormalTok{(}\StringTok{"./RasterGrids\_100m/2024/RAW/"}\NormalTok{,nosaukums)}
\NormalTok{saglabasanas\_cels}\OtherTok{=}\FunctionTok{paste0}\NormalTok{(}\StringTok{"./RasterGrids\_100m/2024/Scaled/"}\NormalTok{,nosaukums)}
\NormalTok{slanis}\OtherTok{=}\FunctionTok{rast}\NormalTok{(ielasisanas\_cels)}
\NormalTok{videjais}\OtherTok{=}\FunctionTok{global}\NormalTok{(slanis,}\AttributeTok{fun=}\StringTok{"mean"}\NormalTok{,}\AttributeTok{na.rm=}\ConstantTok{TRUE}\NormalTok{)}
\NormalTok{centrets}\OtherTok{=}\NormalTok{slanis}\SpecialCharTok{{-}}\NormalTok{videjais[,}\DecValTok{1}\NormalTok{]}
\NormalTok{standartnovirze}\OtherTok{=}\NormalTok{terra}\SpecialCharTok{::}\FunctionTok{global}\NormalTok{(centrets,}\AttributeTok{fun=}\StringTok{"rms"}\NormalTok{,}\AttributeTok{na.rm=}\ConstantTok{TRUE}\NormalTok{)}
\NormalTok{merogots}\OtherTok{=}\NormalTok{centrets}\SpecialCharTok{/}\NormalTok{standartnovirze[,}\DecValTok{1}\NormalTok{]}
\FunctionTok{writeRaster}\NormalTok{(merogots,}
      \AttributeTok{filename=}\NormalTok{saglabasanas\_cels,}
      \AttributeTok{overwrite=}\ConstantTok{TRUE}\NormalTok{)}
\end{Highlighting}
\end{Shaded}

\section{EO\_NDMI-STiqr-median\_cell}\label{ch06.481}

\textbf{filename:} \texttt{EO\_NDMI-STiqr-median\_cell.tif}

\textbf{layername:} \texttt{egv\_481}

\textbf{English name:} Average short-term seasonality of vegetation water content index
(NDMI) within the analysis cell (1 ha)

\textbf{Latvian name:} Sezonalitāte pēdējo gadu vidējai ūdens satura veģetācijā
indeksa (NDMI) vērtībai analīzes šūnā (1 ha)

\textbf{Procedure:} Directly follows \hyperref[Ch04.13]{preprocessing}. The arithmetic mean value
at the analysis cell is calculated using the workflow \texttt{egvtools::input2egv()}. To
protect against potential data loss at edge cells, inverse distance
weighted (power = 2) gap filling is implemented. Finally, the layer is
standardised by subtracting the arithmetic mean and dividing by the root mean
squared error. The ``short-term'' refers to the last five years (2020-2024).

\begin{Shaded}
\begin{Highlighting}[]
\CommentTok{\# libs {-}{-}{-}{-}}
\ControlFlowTok{if}\NormalTok{(}\SpecialCharTok{!}\FunctionTok{require}\NormalTok{(egvtools)) \{remotes}\SpecialCharTok{::}\FunctionTok{install\_github}\NormalTok{(}\StringTok{"aavotins/egvtools"}\NormalTok{); }\FunctionTok{require}\NormalTok{(egvtools)\}}


\CommentTok{\# EO\_NDMI{-}STiqr{-}median\_cell.tif {-}{-}{-}{-}}

\NormalTok{egvrez}\OtherTok{=}\FunctionTok{input2egv}\NormalTok{(}\AttributeTok{input=}\StringTok{"./Geodata/2024/S2indices/Mosaics/EO\_NDMI{-}STiqr.tif"}\NormalTok{,}
         \AttributeTok{egv\_template=} \StringTok{"./Templates/TemplateRasters/LV100m\_10km.tif"}\NormalTok{,}
         \AttributeTok{summary\_function =} \StringTok{"average"}\NormalTok{,}
         \AttributeTok{missing\_job =} \StringTok{"FillOutput"}\NormalTok{,}
         \AttributeTok{outlocation =} \StringTok{"./RasterGrids\_100m/2024/RAW/"}\NormalTok{,}
         \AttributeTok{outfilename =} \StringTok{"EO\_NDMI{-}STiqr{-}median\_cell.tif"}\NormalTok{,}
         \AttributeTok{layername =} \StringTok{"egv\_481"}\NormalTok{,}
         \AttributeTok{idw\_weight =} \DecValTok{2}\NormalTok{,}
         \AttributeTok{plot\_gaps =} \ConstantTok{FALSE}\NormalTok{,}
         \AttributeTok{plot\_final =} \ConstantTok{FALSE}\NormalTok{)}
\NormalTok{egvrez}

\CommentTok{\# standardisation {-}{-}{-}{-}}
\ControlFlowTok{if}\NormalTok{(}\SpecialCharTok{!}\FunctionTok{require}\NormalTok{(terra)) \{}\FunctionTok{install.packages}\NormalTok{(}\StringTok{"terra"}\NormalTok{); }\FunctionTok{require}\NormalTok{(terra)\}}
\ControlFlowTok{if}\NormalTok{(}\SpecialCharTok{!}\FunctionTok{require}\NormalTok{(tidyverse)) \{}\FunctionTok{install.packages}\NormalTok{(}\StringTok{"tidyverse"}\NormalTok{); }\FunctionTok{require}\NormalTok{(tidyverse)\}}

\NormalTok{nosaukums}\OtherTok{=}\StringTok{"EO\_NDMI{-}STiqr{-}median\_cell.tif"}
\NormalTok{ielasisanas\_cels}\OtherTok{=}\FunctionTok{paste0}\NormalTok{(}\StringTok{"./RasterGrids\_100m/2024/RAW/"}\NormalTok{,nosaukums)}
\NormalTok{saglabasanas\_cels}\OtherTok{=}\FunctionTok{paste0}\NormalTok{(}\StringTok{"./RasterGrids\_100m/2024/Scaled/"}\NormalTok{,nosaukums)}
\NormalTok{slanis}\OtherTok{=}\FunctionTok{rast}\NormalTok{(ielasisanas\_cels)}
\NormalTok{videjais}\OtherTok{=}\FunctionTok{global}\NormalTok{(slanis,}\AttributeTok{fun=}\StringTok{"mean"}\NormalTok{,}\AttributeTok{na.rm=}\ConstantTok{TRUE}\NormalTok{)}
\NormalTok{centrets}\OtherTok{=}\NormalTok{slanis}\SpecialCharTok{{-}}\NormalTok{videjais[,}\DecValTok{1}\NormalTok{]}
\NormalTok{standartnovirze}\OtherTok{=}\NormalTok{terra}\SpecialCharTok{::}\FunctionTok{global}\NormalTok{(centrets,}\AttributeTok{fun=}\StringTok{"rms"}\NormalTok{,}\AttributeTok{na.rm=}\ConstantTok{TRUE}\NormalTok{)}
\NormalTok{merogots}\OtherTok{=}\NormalTok{centrets}\SpecialCharTok{/}\NormalTok{standartnovirze[,}\DecValTok{1}\NormalTok{]}
\FunctionTok{writeRaster}\NormalTok{(merogots,}
      \AttributeTok{filename=}\NormalTok{saglabasanas\_cels,}
      \AttributeTok{overwrite=}\ConstantTok{TRUE}\NormalTok{)}
\end{Highlighting}
\end{Shaded}

\section{EO\_NDMI-STmedian-average\_cell}\label{ch06.482}

\textbf{filename:} \texttt{EO\_NDMI-STmedian-average\_cell.tif}

\textbf{layername:} \texttt{egv\_482}

\textbf{English name:} Median short-term vegetation water content index (NDMI) within the
analysis cell (1 ha)

\textbf{Latvian name:} Mediānā pēdējo gadu ūdens satura veģetācijā indeksa
(NDMI) vērtība analīzes šūnā (1 ha)

\textbf{Procedure:} Directly follows \hyperref[Ch04.13]{preprocessing}. The arithmetic mean value
at the analysis cell is calculated using the workflow \texttt{egvtools::input2egv()}. To
protect against potential data loss at edge cells, inverse distance
weighted (power = 2) gap filling is implemented. Finally, the layer is
standardised by subtracting the arithmetic mean and dividing by the root mean
squared error. The ``short-term'' refers to the last five years (2020-2024).

\begin{Shaded}
\begin{Highlighting}[]
\CommentTok{\# libs {-}{-}{-}{-}}
\ControlFlowTok{if}\NormalTok{(}\SpecialCharTok{!}\FunctionTok{require}\NormalTok{(egvtools)) \{remotes}\SpecialCharTok{::}\FunctionTok{install\_github}\NormalTok{(}\StringTok{"aavotins/egvtools"}\NormalTok{); }\FunctionTok{require}\NormalTok{(egvtools)\}}

\CommentTok{\# EO\_NDMI{-}STmedian{-}average\_cell.tif {-}{-}{-}{-}}

\NormalTok{egvrez}\OtherTok{=}\FunctionTok{input2egv}\NormalTok{(}\AttributeTok{input=}\StringTok{"./Geodata/2024/S2indices/Mosaics/EO\_NDMI{-}STmedian.tif"}\NormalTok{,}
         \AttributeTok{egv\_template=} \StringTok{"./Templates/TemplateRasters/LV100m\_10km.tif"}\NormalTok{,}
         \AttributeTok{summary\_function =} \StringTok{"average"}\NormalTok{,}
         \AttributeTok{missing\_job =} \StringTok{"FillOutput"}\NormalTok{,}
         \AttributeTok{outlocation =} \StringTok{"./RasterGrids\_100m/2024/RAW/"}\NormalTok{,}
         \AttributeTok{outfilename =} \StringTok{"EO\_NDMI{-}STmedian{-}average\_cell.tif"}\NormalTok{,}
         \AttributeTok{layername =} \StringTok{"egv\_482"}\NormalTok{,}
         \AttributeTok{idw\_weight =} \DecValTok{2}\NormalTok{,}
         \AttributeTok{plot\_gaps =} \ConstantTok{FALSE}\NormalTok{,}
         \AttributeTok{plot\_final =} \ConstantTok{FALSE}\NormalTok{)}
\NormalTok{egvrez}

\CommentTok{\# standardisation {-}{-}{-}{-}}
\ControlFlowTok{if}\NormalTok{(}\SpecialCharTok{!}\FunctionTok{require}\NormalTok{(terra)) \{}\FunctionTok{install.packages}\NormalTok{(}\StringTok{"terra"}\NormalTok{); }\FunctionTok{require}\NormalTok{(terra)\}}
\ControlFlowTok{if}\NormalTok{(}\SpecialCharTok{!}\FunctionTok{require}\NormalTok{(tidyverse)) \{}\FunctionTok{install.packages}\NormalTok{(}\StringTok{"tidyverse"}\NormalTok{); }\FunctionTok{require}\NormalTok{(tidyverse)\}}

\NormalTok{nosaukums}\OtherTok{=}\StringTok{"EO\_NDMI{-}STmedian{-}average\_cell.tif"}
\NormalTok{ielasisanas\_cels}\OtherTok{=}\FunctionTok{paste0}\NormalTok{(}\StringTok{"./RasterGrids\_100m/2024/RAW/"}\NormalTok{,nosaukums)}
\NormalTok{saglabasanas\_cels}\OtherTok{=}\FunctionTok{paste0}\NormalTok{(}\StringTok{"./RasterGrids\_100m/2024/Scaled/"}\NormalTok{,nosaukums)}
\NormalTok{slanis}\OtherTok{=}\FunctionTok{rast}\NormalTok{(ielasisanas\_cels)}
\NormalTok{videjais}\OtherTok{=}\FunctionTok{global}\NormalTok{(slanis,}\AttributeTok{fun=}\StringTok{"mean"}\NormalTok{,}\AttributeTok{na.rm=}\ConstantTok{TRUE}\NormalTok{)}
\NormalTok{centrets}\OtherTok{=}\NormalTok{slanis}\SpecialCharTok{{-}}\NormalTok{videjais[,}\DecValTok{1}\NormalTok{]}
\NormalTok{standartnovirze}\OtherTok{=}\NormalTok{terra}\SpecialCharTok{::}\FunctionTok{global}\NormalTok{(centrets,}\AttributeTok{fun=}\StringTok{"rms"}\NormalTok{,}\AttributeTok{na.rm=}\ConstantTok{TRUE}\NormalTok{)}
\NormalTok{merogots}\OtherTok{=}\NormalTok{centrets}\SpecialCharTok{/}\NormalTok{standartnovirze[,}\DecValTok{1}\NormalTok{]}
\FunctionTok{writeRaster}\NormalTok{(merogots,}
      \AttributeTok{filename=}\NormalTok{saglabasanas\_cels,}
      \AttributeTok{overwrite=}\ConstantTok{TRUE}\NormalTok{)}
\end{Highlighting}
\end{Shaded}

\section{EO\_NDMI-STmedian-iqr\_cell}\label{ch06.483}

\textbf{filename:} \texttt{EO\_NDMI-STmedian-iqr\_cell.tif}

\textbf{layername:} \texttt{egv\_483}

\textbf{English name:} Spatial variability of short-term median vegetation water
content index (NDMI) within the analysis cell (1 ha)

\textbf{Latvian name:} Telpiskā variabilitāte pēdējo gadu mediānajai ūdens
satura veģetācijā indeksa (NDMI) vērtībai analīzes šūnā
(1 ha)

\textbf{Procedure:} Directly follows \hyperref[Ch04.13]{preprocessing}. The
workflow \texttt{egvtools::input2egv()} is used to calculate Q1 and Q3 for every cell.
To protect against potential data loss at the edges, inverse distance
weighted (power = 2) gap filling is implemented. Next, Q1 is subtracted from Q3.
Finally, the layer is standardised by subtracting the arithmetic mean and
dividing by the root mean squared error. The ``short-term'' refers to the last
five years (2020-2024).

\begin{Shaded}
\begin{Highlighting}[]
\CommentTok{\# libs {-}{-}{-}{-}}
\ControlFlowTok{if}\NormalTok{(}\SpecialCharTok{!}\FunctionTok{require}\NormalTok{(egvtools)) \{remotes}\SpecialCharTok{::}\FunctionTok{install\_github}\NormalTok{(}\StringTok{"aavotins/egvtools"}\NormalTok{); }\FunctionTok{require}\NormalTok{(egvtools)\}}


\CommentTok{\# EO\_NDMI{-}STmedian{-}iqr\_cell.tif {-}{-}{-}{-}}


\NormalTok{p25rez}\OtherTok{=}\FunctionTok{input2egv}\NormalTok{(}\AttributeTok{input=}\StringTok{"./Geodata/2024/S2indices/Mosaics/EO\_NDMI{-}STmedian.tif"}\NormalTok{,}
         \AttributeTok{egv\_template=} \StringTok{"./Templates/TemplateRasters/LV100m\_10km.tif"}\NormalTok{,}
         \AttributeTok{summary\_function =} \StringTok{"q1"}\NormalTok{,}
         \AttributeTok{missing\_job =} \StringTok{"FillOutput"}\NormalTok{,}
         \AttributeTok{outlocation =} \StringTok{"./RasterGrids\_100m/2024/"}\NormalTok{,}
         \AttributeTok{outfilename =} \StringTok{"draza\_p25.tif"}\NormalTok{,}
         \AttributeTok{layername =} \StringTok{"egv\_483"}\NormalTok{,}
         \AttributeTok{idw\_weight =} \DecValTok{2}\NormalTok{,}
         \AttributeTok{plot\_gaps =} \ConstantTok{FALSE}\NormalTok{,}
         \AttributeTok{plot\_final =} \ConstantTok{FALSE}\NormalTok{)}
\NormalTok{p25rez\_r}\OtherTok{=}\FunctionTok{rast}\NormalTok{(}\StringTok{"./RasterGrids\_100m/2024/draza\_p25.tif"}\NormalTok{)}


\NormalTok{p75rez}\OtherTok{=}\FunctionTok{input2egv}\NormalTok{(}\AttributeTok{input=}\StringTok{"./Geodata/2024/S2indices/Mosaics/EO\_NDMI{-}STmedian.tif"}\NormalTok{,}
         \AttributeTok{egv\_template=} \StringTok{"./Templates/TemplateRasters/LV100m\_10km.tif"}\NormalTok{,}
         \AttributeTok{summary\_function =} \StringTok{"q3"}\NormalTok{,}
         \AttributeTok{missing\_job =} \StringTok{"FillOutput"}\NormalTok{,}
         \AttributeTok{outlocation =} \StringTok{"./RasterGrids\_100m/2024/"}\NormalTok{,}
         \AttributeTok{outfilename =} \StringTok{"draza\_p75.tif"}\NormalTok{,}
         \AttributeTok{layername =} \StringTok{"egv\_483"}\NormalTok{,}
         \AttributeTok{idw\_weight =} \DecValTok{2}\NormalTok{,}
         \AttributeTok{plot\_gaps =} \ConstantTok{FALSE}\NormalTok{,}
         \AttributeTok{plot\_final =} \ConstantTok{FALSE}\NormalTok{)}
\NormalTok{p75rez\_r}\OtherTok{=}\FunctionTok{rast}\NormalTok{(}\StringTok{"./RasterGrids\_100m/2024/draza\_p75.tif"}\NormalTok{)}

\NormalTok{iqr\_rez}\OtherTok{=}\NormalTok{p75rez\_r}\SpecialCharTok{{-}}\NormalTok{p25rez\_r}
\NormalTok{iqr\_rez}
\FunctionTok{plot}\NormalTok{(iqr\_rez)}

\FunctionTok{writeRaster}\NormalTok{(iqr\_rez,}
      \StringTok{"./RasterGrids\_100m/2024/RAW/EO\_NDMI{-}STmedian{-}iqr\_cell.tif"}\NormalTok{,}
      \AttributeTok{overwrite=}\ConstantTok{TRUE}\NormalTok{)}

\FunctionTok{unlink}\NormalTok{(}\StringTok{"./RasterGrids\_100m/2024/draza\_p75.tif"}\NormalTok{)}
\FunctionTok{unlink}\NormalTok{(}\StringTok{"./RasterGrids\_100m/2024/draza\_p25.tif"}\NormalTok{)}

\CommentTok{\# standardisation {-}{-}{-}{-}}
\ControlFlowTok{if}\NormalTok{(}\SpecialCharTok{!}\FunctionTok{require}\NormalTok{(terra)) \{}\FunctionTok{install.packages}\NormalTok{(}\StringTok{"terra"}\NormalTok{); }\FunctionTok{require}\NormalTok{(terra)\}}
\ControlFlowTok{if}\NormalTok{(}\SpecialCharTok{!}\FunctionTok{require}\NormalTok{(tidyverse)) \{}\FunctionTok{install.packages}\NormalTok{(}\StringTok{"tidyverse"}\NormalTok{); }\FunctionTok{require}\NormalTok{(tidyverse)\}}

\NormalTok{nosaukums}\OtherTok{=}\StringTok{"EO\_NDMI{-}STmedian{-}iqr\_cell.tif"}
\NormalTok{ielasisanas\_cels}\OtherTok{=}\FunctionTok{paste0}\NormalTok{(}\StringTok{"./RasterGrids\_100m/2024/RAW/"}\NormalTok{,nosaukums)}
\NormalTok{saglabasanas\_cels}\OtherTok{=}\FunctionTok{paste0}\NormalTok{(}\StringTok{"./RasterGrids\_100m/2024/Scaled/"}\NormalTok{,nosaukums)}
\NormalTok{slanis}\OtherTok{=}\FunctionTok{rast}\NormalTok{(ielasisanas\_cels)}
\NormalTok{videjais}\OtherTok{=}\FunctionTok{global}\NormalTok{(slanis,}\AttributeTok{fun=}\StringTok{"mean"}\NormalTok{,}\AttributeTok{na.rm=}\ConstantTok{TRUE}\NormalTok{)}
\NormalTok{centrets}\OtherTok{=}\NormalTok{slanis}\SpecialCharTok{{-}}\NormalTok{videjais[,}\DecValTok{1}\NormalTok{]}
\NormalTok{standartnovirze}\OtherTok{=}\NormalTok{terra}\SpecialCharTok{::}\FunctionTok{global}\NormalTok{(centrets,}\AttributeTok{fun=}\StringTok{"rms"}\NormalTok{,}\AttributeTok{na.rm=}\ConstantTok{TRUE}\NormalTok{)}
\NormalTok{merogots}\OtherTok{=}\NormalTok{centrets}\SpecialCharTok{/}\NormalTok{standartnovirze[,}\DecValTok{1}\NormalTok{]}
\FunctionTok{writeRaster}\NormalTok{(merogots,}
      \AttributeTok{filename=}\NormalTok{saglabasanas\_cels,}
      \AttributeTok{overwrite=}\ConstantTok{TRUE}\NormalTok{)}
\end{Highlighting}
\end{Shaded}

\section{EO\_NDMI-STp25-min\_cell}\label{ch06.484}

\textbf{filename:} \texttt{EO\_NDMI-STp25-min\_cell.tif}

\textbf{layername:} \texttt{egv\_484}

\textbf{English name:} Minimum short-term 25th percentile of vegetation water content index
(NDMI) within the analysis cell (1 ha)

\textbf{Latvian name:} Minimālā 25. procentiles pēdējo gadu ūdens satura
veģetācijā indeksa (NDMI) vērtība analīzes šūnā (1 ha)

\textbf{Procedure:} Directly follows \hyperref[Ch04.13]{preprocessing}. The minimum value
at the analysis cell is calculated using the workflow \texttt{egvtools::input2egv()}. To
protect against potential data loss at edge cells, inverse distance
weighted (power = 2) gap filling is implemented. Finally, the layer is
standardised by subtracting the arithmetic mean and dividing by the root mean
squared error. The ``short-term'' refers to the last five years (2020-2024).

\begin{Shaded}
\begin{Highlighting}[]
\CommentTok{\# libs {-}{-}{-}{-}}
\ControlFlowTok{if}\NormalTok{(}\SpecialCharTok{!}\FunctionTok{require}\NormalTok{(egvtools)) \{remotes}\SpecialCharTok{::}\FunctionTok{install\_github}\NormalTok{(}\StringTok{"aavotins/egvtools"}\NormalTok{); }\FunctionTok{require}\NormalTok{(egvtools)\}}


\CommentTok{\# EO\_NDMI{-}STp25{-}min\_cell.tif {-}{-}{-}{-}}


\NormalTok{egvrez}\OtherTok{=}\FunctionTok{input2egv}\NormalTok{(}\AttributeTok{input=}\StringTok{"./Geodata/2024/S2indices/Mosaics/EO\_NDMI{-}STp25.tif"}\NormalTok{,}
         \AttributeTok{egv\_template=} \StringTok{"./Templates/TemplateRasters/LV100m\_10km.tif"}\NormalTok{,}
         \AttributeTok{summary\_function =} \StringTok{"min"}\NormalTok{,}
         \AttributeTok{missing\_job =} \StringTok{"FillOutput"}\NormalTok{,}
         \AttributeTok{outlocation =} \StringTok{"./RasterGrids\_100m/2024/RAW/"}\NormalTok{,}
         \AttributeTok{outfilename =} \StringTok{"EO\_NDMI{-}STp25{-}min\_cell.tif"}\NormalTok{,}
         \AttributeTok{layername =} \StringTok{"egv\_484"}\NormalTok{,}
         \AttributeTok{idw\_weight =} \DecValTok{2}\NormalTok{,}
         \AttributeTok{plot\_gaps =} \ConstantTok{FALSE}\NormalTok{,}
         \AttributeTok{plot\_final =} \ConstantTok{FALSE}\NormalTok{)}
\NormalTok{egvrez}

\CommentTok{\# standardisation {-}{-}{-}{-}}
\ControlFlowTok{if}\NormalTok{(}\SpecialCharTok{!}\FunctionTok{require}\NormalTok{(terra)) \{}\FunctionTok{install.packages}\NormalTok{(}\StringTok{"terra"}\NormalTok{); }\FunctionTok{require}\NormalTok{(terra)\}}
\ControlFlowTok{if}\NormalTok{(}\SpecialCharTok{!}\FunctionTok{require}\NormalTok{(tidyverse)) \{}\FunctionTok{install.packages}\NormalTok{(}\StringTok{"tidyverse"}\NormalTok{); }\FunctionTok{require}\NormalTok{(tidyverse)\}}

\NormalTok{nosaukums}\OtherTok{=}\StringTok{"EO\_NDMI{-}STp25{-}min\_cell.tif"}
\NormalTok{ielasisanas\_cels}\OtherTok{=}\FunctionTok{paste0}\NormalTok{(}\StringTok{"./RasterGrids\_100m/2024/RAW/"}\NormalTok{,nosaukums)}
\NormalTok{saglabasanas\_cels}\OtherTok{=}\FunctionTok{paste0}\NormalTok{(}\StringTok{"./RasterGrids\_100m/2024/Scaled/"}\NormalTok{,nosaukums)}
\NormalTok{slanis}\OtherTok{=}\FunctionTok{rast}\NormalTok{(ielasisanas\_cels)}
\NormalTok{videjais}\OtherTok{=}\FunctionTok{global}\NormalTok{(slanis,}\AttributeTok{fun=}\StringTok{"mean"}\NormalTok{,}\AttributeTok{na.rm=}\ConstantTok{TRUE}\NormalTok{)}
\NormalTok{centrets}\OtherTok{=}\NormalTok{slanis}\SpecialCharTok{{-}}\NormalTok{videjais[,}\DecValTok{1}\NormalTok{]}
\NormalTok{standartnovirze}\OtherTok{=}\NormalTok{terra}\SpecialCharTok{::}\FunctionTok{global}\NormalTok{(centrets,}\AttributeTok{fun=}\StringTok{"rms"}\NormalTok{,}\AttributeTok{na.rm=}\ConstantTok{TRUE}\NormalTok{)}
\NormalTok{merogots}\OtherTok{=}\NormalTok{centrets}\SpecialCharTok{/}\NormalTok{standartnovirze[,}\DecValTok{1}\NormalTok{]}
\FunctionTok{writeRaster}\NormalTok{(merogots,}
      \AttributeTok{filename=}\NormalTok{saglabasanas\_cels,}
      \AttributeTok{overwrite=}\ConstantTok{TRUE}\NormalTok{)}
\end{Highlighting}
\end{Shaded}

\section{EO\_NDMI-STp75-max\_cell}\label{ch06.485}

\textbf{filename:} \texttt{EO\_NDMI-STp75-max\_cell.tif}

\textbf{layername:} \texttt{egv\_485}

\textbf{English name:} Maximum short-term 75th percentile of vegetation water content index
(NDMI) within the analysis cell (1 ha)

\textbf{Latvian name:} Maksimālā 75. procentiles pēdējo gadu ūdens satura
veģetācijā indeksa (NDMI) vērtība analīzes šūnā (1 ha)

\textbf{Procedure:} Directly follows \hyperref[Ch04.13]{preprocessing}. The maximum value
at the analysis cell is calculated using the workflow \texttt{egvtools::input2egv()}. To
protect against potential data loss at edge cells, inverse distance
weighted (power = 2) gap filling is implemented. Finally, the layer is
standardised by subtracting the arithmetic mean and dividing by the root mean
squared error. The ``short-term'' refers to the last five years (2020-2024).

\begin{Shaded}
\begin{Highlighting}[]
\CommentTok{\# libs {-}{-}{-}{-}}
\ControlFlowTok{if}\NormalTok{(}\SpecialCharTok{!}\FunctionTok{require}\NormalTok{(egvtools)) \{remotes}\SpecialCharTok{::}\FunctionTok{install\_github}\NormalTok{(}\StringTok{"aavotins/egvtools"}\NormalTok{); }\FunctionTok{require}\NormalTok{(egvtools)\}}

\CommentTok{\# EO\_NDMI{-}STp75{-}max\_cell.tif {-}{-}{-}{-}}

\NormalTok{egvrez}\OtherTok{=}\FunctionTok{input2egv}\NormalTok{(}\AttributeTok{input=}\StringTok{"./Geodata/2024/S2indices/Mosaics/EO\_NDMI{-}STp75.tif"}\NormalTok{,}
         \AttributeTok{egv\_template=} \StringTok{"./Templates/TemplateRasters/LV100m\_10km.tif"}\NormalTok{,}
         \AttributeTok{summary\_function =} \StringTok{"min"}\NormalTok{,}
         \AttributeTok{missing\_job =} \StringTok{"FillOutput"}\NormalTok{,}
         \AttributeTok{outlocation =} \StringTok{"./RasterGrids\_100m/2024/RAW/"}\NormalTok{,}
         \AttributeTok{outfilename =} \StringTok{"EO\_NDMI{-}STp75{-}max\_cell.tif"}\NormalTok{,}
         \AttributeTok{layername =} \StringTok{"egv\_485"}\NormalTok{,}
         \AttributeTok{idw\_weight =} \DecValTok{2}\NormalTok{,}
         \AttributeTok{plot\_gaps =} \ConstantTok{FALSE}\NormalTok{,}
         \AttributeTok{plot\_final =} \ConstantTok{FALSE}\NormalTok{)}
\NormalTok{egvrez}

\CommentTok{\# standardisation {-}{-}{-}{-}}
\ControlFlowTok{if}\NormalTok{(}\SpecialCharTok{!}\FunctionTok{require}\NormalTok{(terra)) \{}\FunctionTok{install.packages}\NormalTok{(}\StringTok{"terra"}\NormalTok{); }\FunctionTok{require}\NormalTok{(terra)\}}
\ControlFlowTok{if}\NormalTok{(}\SpecialCharTok{!}\FunctionTok{require}\NormalTok{(tidyverse)) \{}\FunctionTok{install.packages}\NormalTok{(}\StringTok{"tidyverse"}\NormalTok{); }\FunctionTok{require}\NormalTok{(tidyverse)\}}

\NormalTok{nosaukums}\OtherTok{=}\StringTok{"EO\_NDMI{-}STp75{-}max\_cell.tif"}
\NormalTok{ielasisanas\_cels}\OtherTok{=}\FunctionTok{paste0}\NormalTok{(}\StringTok{"./RasterGrids\_100m/2024/RAW/"}\NormalTok{,nosaukums)}
\NormalTok{saglabasanas\_cels}\OtherTok{=}\FunctionTok{paste0}\NormalTok{(}\StringTok{"./RasterGrids\_100m/2024/Scaled/"}\NormalTok{,nosaukums)}
\NormalTok{slanis}\OtherTok{=}\FunctionTok{rast}\NormalTok{(ielasisanas\_cels)}
\NormalTok{videjais}\OtherTok{=}\FunctionTok{global}\NormalTok{(slanis,}\AttributeTok{fun=}\StringTok{"mean"}\NormalTok{,}\AttributeTok{na.rm=}\ConstantTok{TRUE}\NormalTok{)}
\NormalTok{centrets}\OtherTok{=}\NormalTok{slanis}\SpecialCharTok{{-}}\NormalTok{videjais[,}\DecValTok{1}\NormalTok{]}
\NormalTok{standartnovirze}\OtherTok{=}\NormalTok{terra}\SpecialCharTok{::}\FunctionTok{global}\NormalTok{(centrets,}\AttributeTok{fun=}\StringTok{"rms"}\NormalTok{,}\AttributeTok{na.rm=}\ConstantTok{TRUE}\NormalTok{)}
\NormalTok{merogots}\OtherTok{=}\NormalTok{centrets}\SpecialCharTok{/}\NormalTok{standartnovirze[,}\DecValTok{1}\NormalTok{]}
\FunctionTok{writeRaster}\NormalTok{(merogots,}
      \AttributeTok{filename=}\NormalTok{saglabasanas\_cels,}
      \AttributeTok{overwrite=}\ConstantTok{TRUE}\NormalTok{)}
\end{Highlighting}
\end{Shaded}

\section{EO\_NDVI-LYmedian-average\_cell}\label{ch06.486}

\textbf{filename:} \texttt{EO\_NDVI-LYmedian-average\_cell.tif}

\textbf{layername:} \texttt{egv\_486}

\textbf{English name:} Median vegetation index (NDVI) for the last year within the
analysis cell (1 ha)

\textbf{Latvian name:} Mediānā pēdējā gada veģetācijas indeksa (NDVI) vērtība analīzes šūnā (1 ha)

\textbf{Procedure:} Directly follows \hyperref[Ch04.13]{preprocessing}. The arithmetic mean value
at the analysis cell is calculated using the workflow \texttt{egvtools::input2egv()}. To protect against
potential data loss at edge cells, inverse distance weighted (power = 2) gap
filling is implemented. Finally, the layer is
standardised by subtracting the arithmetic mean and
dividing by the root mean squared error. The ``last year'' is 2024.

\begin{Shaded}
\begin{Highlighting}[]
\CommentTok{\# libs {-}{-}{-}{-}}
\ControlFlowTok{if}\NormalTok{(}\SpecialCharTok{!}\FunctionTok{require}\NormalTok{(egvtools)) \{remotes}\SpecialCharTok{::}\FunctionTok{install\_github}\NormalTok{(}\StringTok{"aavotins/egvtools"}\NormalTok{); }\FunctionTok{require}\NormalTok{(egvtools)\}}

\CommentTok{\# EO\_NDVI{-}LYmedian{-}average\_cell.tif {-}{-}{-}{-}}

\NormalTok{egvrez}\OtherTok{=}\FunctionTok{input2egv}\NormalTok{(}\AttributeTok{input=}\StringTok{"./Geodata/2024/S2indices/Mosaics/EO\_NDVI{-}LYmedian.tif"}\NormalTok{,}
         \AttributeTok{egv\_template=} \StringTok{"./Templates/TemplateRasters/LV100m\_10km.tif"}\NormalTok{,}
         \AttributeTok{summary\_function =} \StringTok{"average"}\NormalTok{,}
         \AttributeTok{missing\_job =} \StringTok{"FillOutput"}\NormalTok{,}
         \AttributeTok{outlocation =} \StringTok{"./RasterGrids\_100m/2024/RAW/"}\NormalTok{,}
         \AttributeTok{outfilename =} \StringTok{"EO\_NDVI{-}LYmedian{-}average\_cell.tif"}\NormalTok{,}
         \AttributeTok{layername =} \StringTok{"egv\_486"}\NormalTok{,}
         \AttributeTok{idw\_weight =} \DecValTok{2}\NormalTok{,}
         \AttributeTok{plot\_gaps =} \ConstantTok{FALSE}\NormalTok{,}
         \AttributeTok{plot\_final =} \ConstantTok{FALSE}\NormalTok{)}
\NormalTok{egvrez}

\CommentTok{\# standardisation {-}{-}{-}{-}}
\ControlFlowTok{if}\NormalTok{(}\SpecialCharTok{!}\FunctionTok{require}\NormalTok{(terra)) \{}\FunctionTok{install.packages}\NormalTok{(}\StringTok{"terra"}\NormalTok{); }\FunctionTok{require}\NormalTok{(terra)\}}
\ControlFlowTok{if}\NormalTok{(}\SpecialCharTok{!}\FunctionTok{require}\NormalTok{(tidyverse)) \{}\FunctionTok{install.packages}\NormalTok{(}\StringTok{"tidyverse"}\NormalTok{); }\FunctionTok{require}\NormalTok{(tidyverse)\}}

\NormalTok{nosaukums}\OtherTok{=}\StringTok{"EO\_NDVI{-}LYmedian{-}average\_cell.tif"}
\NormalTok{ielasisanas\_cels}\OtherTok{=}\FunctionTok{paste0}\NormalTok{(}\StringTok{"./RasterGrids\_100m/2024/RAW/"}\NormalTok{,nosaukums)}
\NormalTok{saglabasanas\_cels}\OtherTok{=}\FunctionTok{paste0}\NormalTok{(}\StringTok{"./RasterGrids\_100m/2024/Scaled/"}\NormalTok{,nosaukums)}
\NormalTok{slanis}\OtherTok{=}\FunctionTok{rast}\NormalTok{(ielasisanas\_cels)}
\NormalTok{videjais}\OtherTok{=}\FunctionTok{global}\NormalTok{(slanis,}\AttributeTok{fun=}\StringTok{"mean"}\NormalTok{,}\AttributeTok{na.rm=}\ConstantTok{TRUE}\NormalTok{)}
\NormalTok{centrets}\OtherTok{=}\NormalTok{slanis}\SpecialCharTok{{-}}\NormalTok{videjais[,}\DecValTok{1}\NormalTok{]}
\NormalTok{standartnovirze}\OtherTok{=}\NormalTok{terra}\SpecialCharTok{::}\FunctionTok{global}\NormalTok{(centrets,}\AttributeTok{fun=}\StringTok{"rms"}\NormalTok{,}\AttributeTok{na.rm=}\ConstantTok{TRUE}\NormalTok{)}
\NormalTok{merogots}\OtherTok{=}\NormalTok{centrets}\SpecialCharTok{/}\NormalTok{standartnovirze[,}\DecValTok{1}\NormalTok{]}
\FunctionTok{writeRaster}\NormalTok{(merogots,}
      \AttributeTok{filename=}\NormalTok{saglabasanas\_cels,}
      \AttributeTok{overwrite=}\ConstantTok{TRUE}\NormalTok{)}
\end{Highlighting}
\end{Shaded}

\section{EO\_NDVI-LYmedian-iqr\_cell}\label{ch06.487}

\textbf{filename:} \texttt{EO\_NDVI-LYmedian-iqr\_cell.tif}

\textbf{layername:} \texttt{egv\_487}

\textbf{English name:} Spatial variability of last year's median vegetation index
(NDVI) within the analysis cell (1 ha)

\textbf{Latvian name:} Telpiskā variabilitāte pēdējā gada mediānajai veģetācijas
indeksa (NDVI) vērtībai analīzes šūnā (1 ha)

\textbf{Procedure:} Directly follows \hyperref[Ch04.13]{preprocessing}. The
workflow \texttt{egvtools::input2egv()} is used to calculate Q1 and Q3 for every cell.
To protect against potential data loss at the edges, inverse distance
weighted (power = 2) gap filling is implemented. Next, Q1 is subtracted from Q3.
Finally, the layer is standardised by subtracting the arithmetic mean and
dividing by the root mean squared error. The ``last year'' is 2024.

\begin{Shaded}
\begin{Highlighting}[]
\CommentTok{\# libs {-}{-}{-}{-}}
\ControlFlowTok{if}\NormalTok{(}\SpecialCharTok{!}\FunctionTok{require}\NormalTok{(egvtools)) \{remotes}\SpecialCharTok{::}\FunctionTok{install\_github}\NormalTok{(}\StringTok{"aavotins/egvtools"}\NormalTok{); }\FunctionTok{require}\NormalTok{(egvtools)\}}


\CommentTok{\# EO\_NDVI{-}LYmedian{-}iqr\_cell.tif {-}{-}{-}{-}}


\NormalTok{p25rez}\OtherTok{=}\FunctionTok{input2egv}\NormalTok{(}\AttributeTok{input=}\StringTok{"./Geodata/2024/S2indices/Mosaics/EO\_NDVI{-}LYmedian.tif"}\NormalTok{,}
         \AttributeTok{egv\_template=} \StringTok{"./Templates/TemplateRasters/LV100m\_10km.tif"}\NormalTok{,}
         \AttributeTok{summary\_function =} \StringTok{"q1"}\NormalTok{,}
         \AttributeTok{missing\_job =} \StringTok{"FillOutput"}\NormalTok{,}
         \AttributeTok{outlocation =} \StringTok{"./RasterGrids\_100m/2024/"}\NormalTok{,}
         \AttributeTok{outfilename =} \StringTok{"draza\_p25.tif"}\NormalTok{,}
         \AttributeTok{layername =} \StringTok{"egv\_487"}\NormalTok{,}
         \AttributeTok{idw\_weight =} \DecValTok{2}\NormalTok{,}
         \AttributeTok{plot\_gaps =} \ConstantTok{FALSE}\NormalTok{,}
         \AttributeTok{plot\_final =} \ConstantTok{FALSE}\NormalTok{)}
\NormalTok{p25rez\_r}\OtherTok{=}\FunctionTok{rast}\NormalTok{(}\StringTok{"./RasterGrids\_100m/2024/draza\_p25.tif"}\NormalTok{)}


\NormalTok{p75rez}\OtherTok{=}\FunctionTok{input2egv}\NormalTok{(}\AttributeTok{input=}\StringTok{"./Geodata/2024/S2indices/Mosaics/EO\_NDVI{-}LYmedian.tif"}\NormalTok{,}
         \AttributeTok{egv\_template=} \StringTok{"./Templates/TemplateRasters/LV100m\_10km.tif"}\NormalTok{,}
         \AttributeTok{summary\_function =} \StringTok{"q3"}\NormalTok{,}
         \AttributeTok{missing\_job =} \StringTok{"FillOutput"}\NormalTok{,}
         \AttributeTok{outlocation =} \StringTok{"./RasterGrids\_100m/2024/"}\NormalTok{,}
         \AttributeTok{outfilename =} \StringTok{"draza\_p75.tif"}\NormalTok{,}
         \AttributeTok{layername =} \StringTok{"egv\_487"}\NormalTok{,}
         \AttributeTok{idw\_weight =} \DecValTok{2}\NormalTok{,}
         \AttributeTok{plot\_gaps =} \ConstantTok{FALSE}\NormalTok{,}
         \AttributeTok{plot\_final =} \ConstantTok{FALSE}\NormalTok{)}
\NormalTok{p75rez\_r}\OtherTok{=}\FunctionTok{rast}\NormalTok{(}\StringTok{"./RasterGrids\_100m/2024/draza\_p75.tif"}\NormalTok{)}

\NormalTok{iqr\_rez}\OtherTok{=}\NormalTok{p75rez\_r}\SpecialCharTok{{-}}\NormalTok{p25rez\_r}
\NormalTok{iqr\_rez}
\FunctionTok{plot}\NormalTok{(iqr\_rez)}

\FunctionTok{writeRaster}\NormalTok{(iqr\_rez,}
      \StringTok{"./RasterGrids\_100m/2024/RAW/EO\_NDVI{-}LYmedian{-}iqr\_cell.tif"}\NormalTok{,}
      \AttributeTok{overwrite=}\ConstantTok{TRUE}\NormalTok{)}

\FunctionTok{unlink}\NormalTok{(}\StringTok{"./RasterGrids\_100m/2024/draza\_p75.tif"}\NormalTok{)}
\FunctionTok{unlink}\NormalTok{(}\StringTok{"./RasterGrids\_100m/2024/draza\_p25.tif"}\NormalTok{)}

\CommentTok{\# standardisation {-}{-}{-}{-}}
\ControlFlowTok{if}\NormalTok{(}\SpecialCharTok{!}\FunctionTok{require}\NormalTok{(terra)) \{}\FunctionTok{install.packages}\NormalTok{(}\StringTok{"terra"}\NormalTok{); }\FunctionTok{require}\NormalTok{(terra)\}}
\ControlFlowTok{if}\NormalTok{(}\SpecialCharTok{!}\FunctionTok{require}\NormalTok{(tidyverse)) \{}\FunctionTok{install.packages}\NormalTok{(}\StringTok{"tidyverse"}\NormalTok{); }\FunctionTok{require}\NormalTok{(tidyverse)\}}

\NormalTok{nosaukums}\OtherTok{=}\StringTok{"EO\_NDVI{-}LYmedian{-}iqr\_cell.tif"}
\NormalTok{ielasisanas\_cels}\OtherTok{=}\FunctionTok{paste0}\NormalTok{(}\StringTok{"./RasterGrids\_100m/2024/RAW/"}\NormalTok{,nosaukums)}
\NormalTok{saglabasanas\_cels}\OtherTok{=}\FunctionTok{paste0}\NormalTok{(}\StringTok{"./RasterGrids\_100m/2024/Scaled/"}\NormalTok{,nosaukums)}
\NormalTok{slanis}\OtherTok{=}\FunctionTok{rast}\NormalTok{(ielasisanas\_cels)}
\NormalTok{videjais}\OtherTok{=}\FunctionTok{global}\NormalTok{(slanis,}\AttributeTok{fun=}\StringTok{"mean"}\NormalTok{,}\AttributeTok{na.rm=}\ConstantTok{TRUE}\NormalTok{)}
\NormalTok{centrets}\OtherTok{=}\NormalTok{slanis}\SpecialCharTok{{-}}\NormalTok{videjais[,}\DecValTok{1}\NormalTok{]}
\NormalTok{standartnovirze}\OtherTok{=}\NormalTok{terra}\SpecialCharTok{::}\FunctionTok{global}\NormalTok{(centrets,}\AttributeTok{fun=}\StringTok{"rms"}\NormalTok{,}\AttributeTok{na.rm=}\ConstantTok{TRUE}\NormalTok{)}
\NormalTok{merogots}\OtherTok{=}\NormalTok{centrets}\SpecialCharTok{/}\NormalTok{standartnovirze[,}\DecValTok{1}\NormalTok{]}
\FunctionTok{writeRaster}\NormalTok{(merogots,}
      \AttributeTok{filename=}\NormalTok{saglabasanas\_cels,}
      \AttributeTok{overwrite=}\ConstantTok{TRUE}\NormalTok{)}
\end{Highlighting}
\end{Shaded}

\section{EO\_NDVI-STiqr-median\_cell}\label{ch06.488}

\textbf{filename:} \texttt{EO\_NDVI-STiqr-median\_cell.tif}

\textbf{layername:} \texttt{egv\_488}

\textbf{English name:} Average short-term seasonality of vegetation index (NDVI)
within the analysis cell (1 ha)

\textbf{Latvian name:} Sezonalitāte pēdējo gadu vidējai veģetācijas indeksa
(NDVI) vērtībai analīzes šūnā (1 ha)

\textbf{Procedure:} Directly follows \hyperref[Ch04.13]{preprocessing}. The arithmetic mean value
at the analysis cell is calculated using the workflow \texttt{egvtools::input2egv()}. To
protect against potential data loss at edge cells, inverse distance
weighted (power = 2) gap filling is implemented. Finally, the layer is
standardised by subtracting the arithmetic mean and dividing by the root mean
squared error. The ``short-term'' refers to the last five years (2020-2024).

\begin{Shaded}
\begin{Highlighting}[]
\CommentTok{\# libs {-}{-}{-}{-}}
\ControlFlowTok{if}\NormalTok{(}\SpecialCharTok{!}\FunctionTok{require}\NormalTok{(egvtools)) \{remotes}\SpecialCharTok{::}\FunctionTok{install\_github}\NormalTok{(}\StringTok{"aavotins/egvtools"}\NormalTok{); }\FunctionTok{require}\NormalTok{(egvtools)\}}

\CommentTok{\# EO\_NDVI{-}STiqr{-}median\_cell.tif {-}{-}{-}{-}}

\NormalTok{egvrez}\OtherTok{=}\FunctionTok{input2egv}\NormalTok{(}\AttributeTok{input=}\StringTok{"./Geodata/2024/S2indices/Mosaics/EO\_NDVI{-}STiqr.tif"}\NormalTok{,}
         \AttributeTok{egv\_template=} \StringTok{"./Templates/TemplateRasters/LV100m\_10km.tif"}\NormalTok{,}
         \AttributeTok{summary\_function =} \StringTok{"average"}\NormalTok{,}
         \AttributeTok{missing\_job =} \StringTok{"FillOutput"}\NormalTok{,}
         \AttributeTok{outlocation =} \StringTok{"./RasterGrids\_100m/2024/RAW/"}\NormalTok{,}
         \AttributeTok{outfilename =} \StringTok{"EO\_NDVI{-}STiqr{-}median\_cell.tif"}\NormalTok{,}
         \AttributeTok{layername =} \StringTok{"egv\_488"}\NormalTok{,}
         \AttributeTok{idw\_weight =} \DecValTok{2}\NormalTok{,}
         \AttributeTok{plot\_gaps =} \ConstantTok{FALSE}\NormalTok{,}
         \AttributeTok{plot\_final =} \ConstantTok{FALSE}\NormalTok{)}
\NormalTok{egvrez}

\CommentTok{\# standardisation {-}{-}{-}{-}}
\ControlFlowTok{if}\NormalTok{(}\SpecialCharTok{!}\FunctionTok{require}\NormalTok{(terra)) \{}\FunctionTok{install.packages}\NormalTok{(}\StringTok{"terra"}\NormalTok{); }\FunctionTok{require}\NormalTok{(terra)\}}
\ControlFlowTok{if}\NormalTok{(}\SpecialCharTok{!}\FunctionTok{require}\NormalTok{(tidyverse)) \{}\FunctionTok{install.packages}\NormalTok{(}\StringTok{"tidyverse"}\NormalTok{); }\FunctionTok{require}\NormalTok{(tidyverse)\}}

\NormalTok{nosaukums}\OtherTok{=}\StringTok{"EO\_NDVI{-}STiqr{-}median\_cell.tif"}
\NormalTok{ielasisanas\_cels}\OtherTok{=}\FunctionTok{paste0}\NormalTok{(}\StringTok{"./RasterGrids\_100m/2024/RAW/"}\NormalTok{,nosaukums)}
\NormalTok{saglabasanas\_cels}\OtherTok{=}\FunctionTok{paste0}\NormalTok{(}\StringTok{"./RasterGrids\_100m/2024/Scaled/"}\NormalTok{,nosaukums)}
\NormalTok{slanis}\OtherTok{=}\FunctionTok{rast}\NormalTok{(ielasisanas\_cels)}
\NormalTok{videjais}\OtherTok{=}\FunctionTok{global}\NormalTok{(slanis,}\AttributeTok{fun=}\StringTok{"mean"}\NormalTok{,}\AttributeTok{na.rm=}\ConstantTok{TRUE}\NormalTok{)}
\NormalTok{centrets}\OtherTok{=}\NormalTok{slanis}\SpecialCharTok{{-}}\NormalTok{videjais[,}\DecValTok{1}\NormalTok{]}
\NormalTok{standartnovirze}\OtherTok{=}\NormalTok{terra}\SpecialCharTok{::}\FunctionTok{global}\NormalTok{(centrets,}\AttributeTok{fun=}\StringTok{"rms"}\NormalTok{,}\AttributeTok{na.rm=}\ConstantTok{TRUE}\NormalTok{)}
\NormalTok{merogots}\OtherTok{=}\NormalTok{centrets}\SpecialCharTok{/}\NormalTok{standartnovirze[,}\DecValTok{1}\NormalTok{]}
\FunctionTok{writeRaster}\NormalTok{(merogots,}
      \AttributeTok{filename=}\NormalTok{saglabasanas\_cels,}
      \AttributeTok{overwrite=}\ConstantTok{TRUE}\NormalTok{)}
\end{Highlighting}
\end{Shaded}

\section{EO\_NDVI-STmedian-average\_cell}\label{ch06.489}

\textbf{filename:} \texttt{EO\_NDVI-STmedian-average\_cell.tif}

\textbf{layername:} \texttt{egv\_489}

\textbf{English name:} Median short-term vegetation index (NDVI) within the analysis
cell (1 ha)

\textbf{Latvian name:} Mediānā pēdējo gadu veģetācijas indeksa (NDVI) vērtība analīzes šūnā (1 ha)

\textbf{Procedure:} Directly follows \hyperref[Ch04.13]{preprocessing}. The arithmetic mean value
at the analysis cell is calculated using the workflow \texttt{egvtools::input2egv()}. To
protect against potential data loss at edge cells, inverse distance
weighted (power = 2) gap filling is implemented. Finally, the layer is
standardised by subtracting the arithmetic mean and dividing by the root mean
squared error. The ``short-term'' refers to the last five years (2020-2024).

\begin{Shaded}
\begin{Highlighting}[]
\CommentTok{\# libs {-}{-}{-}{-}}
\ControlFlowTok{if}\NormalTok{(}\SpecialCharTok{!}\FunctionTok{require}\NormalTok{(egvtools)) \{remotes}\SpecialCharTok{::}\FunctionTok{install\_github}\NormalTok{(}\StringTok{"aavotins/egvtools"}\NormalTok{); }\FunctionTok{require}\NormalTok{(egvtools)\}}

\CommentTok{\# EO\_NDVI{-}STmedian{-}average\_cell.tif {-}{-}{-}{-}}

\NormalTok{egvrez}\OtherTok{=}\FunctionTok{input2egv}\NormalTok{(}\AttributeTok{input=}\StringTok{"./Geodata/2024/S2indices/Mosaics/EO\_NDVI{-}STmedian.tif"}\NormalTok{,}
         \AttributeTok{egv\_template=} \StringTok{"./Templates/TemplateRasters/LV100m\_10km.tif"}\NormalTok{,}
         \AttributeTok{summary\_function =} \StringTok{"average"}\NormalTok{,}
         \AttributeTok{missing\_job =} \StringTok{"FillOutput"}\NormalTok{,}
         \AttributeTok{outlocation =} \StringTok{"./RasterGrids\_100m/2024/RAW/"}\NormalTok{,}
         \AttributeTok{outfilename =} \StringTok{"EO\_NDVI{-}STmedian{-}average\_cell.tif"}\NormalTok{,}
         \AttributeTok{layername =} \StringTok{"egv\_489"}\NormalTok{,}
         \AttributeTok{idw\_weight =} \DecValTok{2}\NormalTok{,}
         \AttributeTok{plot\_gaps =} \ConstantTok{FALSE}\NormalTok{,}
         \AttributeTok{plot\_final =} \ConstantTok{FALSE}\NormalTok{)}
\NormalTok{egvrez}

\CommentTok{\# standardisation {-}{-}{-}{-}}
\ControlFlowTok{if}\NormalTok{(}\SpecialCharTok{!}\FunctionTok{require}\NormalTok{(terra)) \{}\FunctionTok{install.packages}\NormalTok{(}\StringTok{"terra"}\NormalTok{); }\FunctionTok{require}\NormalTok{(terra)\}}
\ControlFlowTok{if}\NormalTok{(}\SpecialCharTok{!}\FunctionTok{require}\NormalTok{(tidyverse)) \{}\FunctionTok{install.packages}\NormalTok{(}\StringTok{"tidyverse"}\NormalTok{); }\FunctionTok{require}\NormalTok{(tidyverse)\}}

\NormalTok{nosaukums}\OtherTok{=}\StringTok{"EO\_NDVI{-}STmedian{-}average\_cell.tif"}
\NormalTok{ielasisanas\_cels}\OtherTok{=}\FunctionTok{paste0}\NormalTok{(}\StringTok{"./RasterGrids\_100m/2024/RAW/"}\NormalTok{,nosaukums)}
\NormalTok{saglabasanas\_cels}\OtherTok{=}\FunctionTok{paste0}\NormalTok{(}\StringTok{"./RasterGrids\_100m/2024/Scaled/"}\NormalTok{,nosaukums)}
\NormalTok{slanis}\OtherTok{=}\FunctionTok{rast}\NormalTok{(ielasisanas\_cels)}
\NormalTok{videjais}\OtherTok{=}\FunctionTok{global}\NormalTok{(slanis,}\AttributeTok{fun=}\StringTok{"mean"}\NormalTok{,}\AttributeTok{na.rm=}\ConstantTok{TRUE}\NormalTok{)}
\NormalTok{centrets}\OtherTok{=}\NormalTok{slanis}\SpecialCharTok{{-}}\NormalTok{videjais[,}\DecValTok{1}\NormalTok{]}
\NormalTok{standartnovirze}\OtherTok{=}\NormalTok{terra}\SpecialCharTok{::}\FunctionTok{global}\NormalTok{(centrets,}\AttributeTok{fun=}\StringTok{"rms"}\NormalTok{,}\AttributeTok{na.rm=}\ConstantTok{TRUE}\NormalTok{)}
\NormalTok{merogots}\OtherTok{=}\NormalTok{centrets}\SpecialCharTok{/}\NormalTok{standartnovirze[,}\DecValTok{1}\NormalTok{]}
\FunctionTok{writeRaster}\NormalTok{(merogots,}
      \AttributeTok{filename=}\NormalTok{saglabasanas\_cels,}
      \AttributeTok{overwrite=}\ConstantTok{TRUE}\NormalTok{)}
\end{Highlighting}
\end{Shaded}

\section{EO\_NDVI-STmedian-iqr\_cell}\label{ch06.490}

\textbf{filename:} \texttt{EO\_NDVI-STmedian-iqr\_cell.tif}

\textbf{layername:} \texttt{egv\_490}

\textbf{English name:} Spatial variability of short-term median vegetation index
(NDVI) within the analysis cell (1 ha)

\textbf{Latvian name:} Telpiskā variabilitāte pēdējo gadu mediānajai
veģetācijas indeksa (NDVI) vērtībai analīzes šūnā (1 ha)

\textbf{Procedure:} Directly follows \hyperref[Ch04.13]{preprocessing}. The
workflow \texttt{egvtools::input2egv()} is used to calculate Q1 and Q3 for every cell.
To protect against potential data loss at the edges, inverse distance
weighted (power = 2) gap filling is implemented. Next, Q1 is subtracted from Q3.
Finally, the layer is standardised by subtracting the arithmetic mean and
dividing by the root mean squared error. The ``short-term'' refers to the last
five years (2020-2024).

\begin{Shaded}
\begin{Highlighting}[]
\CommentTok{\# libs {-}{-}{-}{-}}
\ControlFlowTok{if}\NormalTok{(}\SpecialCharTok{!}\FunctionTok{require}\NormalTok{(egvtools)) \{remotes}\SpecialCharTok{::}\FunctionTok{install\_github}\NormalTok{(}\StringTok{"aavotins/egvtools"}\NormalTok{); }\FunctionTok{require}\NormalTok{(egvtools)\}}


\CommentTok{\# EO\_NDVI{-}STmedian{-}iqr\_cell.tif {-}{-}{-}{-}}


\NormalTok{p25rez}\OtherTok{=}\FunctionTok{input2egv}\NormalTok{(}\AttributeTok{input=}\StringTok{"./Geodata/2024/S2indices/Mosaics/EO\_NDVI{-}STmedian.tif"}\NormalTok{,}
         \AttributeTok{egv\_template=} \StringTok{"./Templates/TemplateRasters/LV100m\_10km.tif"}\NormalTok{,}
         \AttributeTok{summary\_function =} \StringTok{"q1"}\NormalTok{,}
         \AttributeTok{missing\_job =} \StringTok{"FillOutput"}\NormalTok{,}
         \AttributeTok{outlocation =} \StringTok{"./RasterGrids\_100m/2024/"}\NormalTok{,}
         \AttributeTok{outfilename =} \StringTok{"draza\_p25.tif"}\NormalTok{,}
         \AttributeTok{layername =} \StringTok{"egv\_490"}\NormalTok{,}
         \AttributeTok{idw\_weight =} \DecValTok{2}\NormalTok{,}
         \AttributeTok{plot\_gaps =} \ConstantTok{FALSE}\NormalTok{,}
         \AttributeTok{plot\_final =} \ConstantTok{FALSE}\NormalTok{)}
\NormalTok{p25rez\_r}\OtherTok{=}\FunctionTok{rast}\NormalTok{(}\StringTok{"./RasterGrids\_100m/2024/draza\_p25.tif"}\NormalTok{)}


\NormalTok{p75rez}\OtherTok{=}\FunctionTok{input2egv}\NormalTok{(}\AttributeTok{input=}\StringTok{"./Geodata/2024/S2indices/Mosaics/EO\_NDVI{-}STmedian.tif"}\NormalTok{,}
         \AttributeTok{egv\_template=} \StringTok{"./Templates/TemplateRasters/LV100m\_10km.tif"}\NormalTok{,}
         \AttributeTok{summary\_function =} \StringTok{"q3"}\NormalTok{,}
         \AttributeTok{missing\_job =} \StringTok{"FillOutput"}\NormalTok{,}
         \AttributeTok{outlocation =} \StringTok{"./RasterGrids\_100m/2024/"}\NormalTok{,}
         \AttributeTok{outfilename =} \StringTok{"draza\_p75.tif"}\NormalTok{,}
         \AttributeTok{layername =} \StringTok{"egv\_490"}\NormalTok{,}
         \AttributeTok{idw\_weight =} \DecValTok{2}\NormalTok{,}
         \AttributeTok{plot\_gaps =} \ConstantTok{FALSE}\NormalTok{,}
         \AttributeTok{plot\_final =} \ConstantTok{FALSE}\NormalTok{)}
\NormalTok{p75rez\_r}\OtherTok{=}\FunctionTok{rast}\NormalTok{(}\StringTok{"./RasterGrids\_100m/2024/draza\_p75.tif"}\NormalTok{)}

\NormalTok{iqr\_rez}\OtherTok{=}\NormalTok{p75rez\_r}\SpecialCharTok{{-}}\NormalTok{p25rez\_r}
\NormalTok{iqr\_rez}
\FunctionTok{plot}\NormalTok{(iqr\_rez)}

\FunctionTok{writeRaster}\NormalTok{(iqr\_rez,}
      \StringTok{"./RasterGrids\_100m/2024/RAW/EO\_NDVI{-}STmedian{-}iqr\_cell.tif"}\NormalTok{,}
      \AttributeTok{overwrite=}\ConstantTok{TRUE}\NormalTok{)}

\FunctionTok{unlink}\NormalTok{(}\StringTok{"./RasterGrids\_100m/2024/draza\_p75.tif"}\NormalTok{)}
\FunctionTok{unlink}\NormalTok{(}\StringTok{"./RasterGrids\_100m/2024/draza\_p25.tif"}\NormalTok{)}

\CommentTok{\# standardisation {-}{-}{-}{-}}
\ControlFlowTok{if}\NormalTok{(}\SpecialCharTok{!}\FunctionTok{require}\NormalTok{(terra)) \{}\FunctionTok{install.packages}\NormalTok{(}\StringTok{"terra"}\NormalTok{); }\FunctionTok{require}\NormalTok{(terra)\}}
\ControlFlowTok{if}\NormalTok{(}\SpecialCharTok{!}\FunctionTok{require}\NormalTok{(tidyverse)) \{}\FunctionTok{install.packages}\NormalTok{(}\StringTok{"tidyverse"}\NormalTok{); }\FunctionTok{require}\NormalTok{(tidyverse)\}}

\NormalTok{nosaukums}\OtherTok{=}\StringTok{"EO\_NDVI{-}STmedian{-}iqr\_cell.tif"}
\NormalTok{ielasisanas\_cels}\OtherTok{=}\FunctionTok{paste0}\NormalTok{(}\StringTok{"./RasterGrids\_100m/2024/RAW/"}\NormalTok{,nosaukums)}
\NormalTok{saglabasanas\_cels}\OtherTok{=}\FunctionTok{paste0}\NormalTok{(}\StringTok{"./RasterGrids\_100m/2024/Scaled/"}\NormalTok{,nosaukums)}
\NormalTok{slanis}\OtherTok{=}\FunctionTok{rast}\NormalTok{(ielasisanas\_cels)}
\NormalTok{videjais}\OtherTok{=}\FunctionTok{global}\NormalTok{(slanis,}\AttributeTok{fun=}\StringTok{"mean"}\NormalTok{,}\AttributeTok{na.rm=}\ConstantTok{TRUE}\NormalTok{)}
\NormalTok{centrets}\OtherTok{=}\NormalTok{slanis}\SpecialCharTok{{-}}\NormalTok{videjais[,}\DecValTok{1}\NormalTok{]}
\NormalTok{standartnovirze}\OtherTok{=}\NormalTok{terra}\SpecialCharTok{::}\FunctionTok{global}\NormalTok{(centrets,}\AttributeTok{fun=}\StringTok{"rms"}\NormalTok{,}\AttributeTok{na.rm=}\ConstantTok{TRUE}\NormalTok{)}
\NormalTok{merogots}\OtherTok{=}\NormalTok{centrets}\SpecialCharTok{/}\NormalTok{standartnovirze[,}\DecValTok{1}\NormalTok{]}
\FunctionTok{writeRaster}\NormalTok{(merogots,}
      \AttributeTok{filename=}\NormalTok{saglabasanas\_cels,}
      \AttributeTok{overwrite=}\ConstantTok{TRUE}\NormalTok{)}
\end{Highlighting}
\end{Shaded}

\section{EO\_NDVI-STp25-min\_cell}\label{ch06.491}

\textbf{filename:} \texttt{EO\_NDVI-STp25-min\_cell.tif}

\textbf{layername:} \texttt{egv\_491}

\textbf{English name:} Minimum short-term 25th percentile of vegetation index (NDVI)
within the analysis cell (1 ha)

\textbf{Latvian name:} Minimālā 25. procentiles pēdējo gadu veģetācijas indeksa
(NDVI) vērtība analīzes šūnā (1 ha)

\textbf{Procedure:} Directly follows \hyperref[Ch04.13]{preprocessing}. The minimum value
at the analysis cell is calculated using the workflow \texttt{egvtools::input2egv()}. To
protect against potential data loss at edge cells, inverse distance
weighted (power = 2) gap filling is implemented. Finally, the layer is
standardised by subtracting the arithmetic mean and dividing by the root mean
squared error. The ``short-term'' refers to the last five years (2020-2024).

\begin{Shaded}
\begin{Highlighting}[]
\CommentTok{\# libs {-}{-}{-}{-}}
\ControlFlowTok{if}\NormalTok{(}\SpecialCharTok{!}\FunctionTok{require}\NormalTok{(egvtools)) \{remotes}\SpecialCharTok{::}\FunctionTok{install\_github}\NormalTok{(}\StringTok{"aavotins/egvtools"}\NormalTok{); }\FunctionTok{require}\NormalTok{(egvtools)\}}

\CommentTok{\# EO\_NDVI{-}STp25{-}min\_cell.tif {-}{-}{-}{-}}

\NormalTok{egvrez}\OtherTok{=}\FunctionTok{input2egv}\NormalTok{(}\AttributeTok{input=}\StringTok{"./Geodata/2024/S2indices/Mosaics/EO\_NDVI{-}STp25.tif"}\NormalTok{,}
         \AttributeTok{egv\_template=} \StringTok{"./Templates/TemplateRasters/LV100m\_10km.tif"}\NormalTok{,}
         \AttributeTok{summary\_function =} \StringTok{"min"}\NormalTok{,}
         \AttributeTok{missing\_job =} \StringTok{"FillOutput"}\NormalTok{,}
         \AttributeTok{outlocation =} \StringTok{"./RasterGrids\_100m/2024/RAW/"}\NormalTok{,}
         \AttributeTok{outfilename =} \StringTok{"EO\_NDVI{-}STp25{-}min\_cell.tif"}\NormalTok{,}
         \AttributeTok{layername =} \StringTok{"egv\_491"}\NormalTok{,}
         \AttributeTok{idw\_weight =} \DecValTok{2}\NormalTok{,}
         \AttributeTok{plot\_gaps =} \ConstantTok{FALSE}\NormalTok{,}
         \AttributeTok{plot\_final =} \ConstantTok{FALSE}\NormalTok{)}
\NormalTok{egvrez}

\CommentTok{\# standardisation {-}{-}{-}{-}}
\ControlFlowTok{if}\NormalTok{(}\SpecialCharTok{!}\FunctionTok{require}\NormalTok{(terra)) \{}\FunctionTok{install.packages}\NormalTok{(}\StringTok{"terra"}\NormalTok{); }\FunctionTok{require}\NormalTok{(terra)\}}
\ControlFlowTok{if}\NormalTok{(}\SpecialCharTok{!}\FunctionTok{require}\NormalTok{(tidyverse)) \{}\FunctionTok{install.packages}\NormalTok{(}\StringTok{"tidyverse"}\NormalTok{); }\FunctionTok{require}\NormalTok{(tidyverse)\}}

\NormalTok{nosaukums}\OtherTok{=}\StringTok{"EO\_NDVI{-}STp25{-}min\_cell.tif"}
\NormalTok{ielasisanas\_cels}\OtherTok{=}\FunctionTok{paste0}\NormalTok{(}\StringTok{"./RasterGrids\_100m/2024/RAW/"}\NormalTok{,nosaukums)}
\NormalTok{saglabasanas\_cels}\OtherTok{=}\FunctionTok{paste0}\NormalTok{(}\StringTok{"./RasterGrids\_100m/2024/Scaled/"}\NormalTok{,nosaukums)}
\NormalTok{slanis}\OtherTok{=}\FunctionTok{rast}\NormalTok{(ielasisanas\_cels)}
\NormalTok{videjais}\OtherTok{=}\FunctionTok{global}\NormalTok{(slanis,}\AttributeTok{fun=}\StringTok{"mean"}\NormalTok{,}\AttributeTok{na.rm=}\ConstantTok{TRUE}\NormalTok{)}
\NormalTok{centrets}\OtherTok{=}\NormalTok{slanis}\SpecialCharTok{{-}}\NormalTok{videjais[,}\DecValTok{1}\NormalTok{]}
\NormalTok{standartnovirze}\OtherTok{=}\NormalTok{terra}\SpecialCharTok{::}\FunctionTok{global}\NormalTok{(centrets,}\AttributeTok{fun=}\StringTok{"rms"}\NormalTok{,}\AttributeTok{na.rm=}\ConstantTok{TRUE}\NormalTok{)}
\NormalTok{merogots}\OtherTok{=}\NormalTok{centrets}\SpecialCharTok{/}\NormalTok{standartnovirze[,}\DecValTok{1}\NormalTok{]}
\FunctionTok{writeRaster}\NormalTok{(merogots,}
      \AttributeTok{filename=}\NormalTok{saglabasanas\_cels,}
      \AttributeTok{overwrite=}\ConstantTok{TRUE}\NormalTok{)}
\end{Highlighting}
\end{Shaded}

\section{EO\_NDVI-STp75-max\_cell}\label{ch06.492}

\textbf{filename:} \texttt{EO\_NDVI-STp75-max\_cell.tif}

\textbf{layername:} \texttt{egv\_492}

\textbf{English name:} Maximum short-term 75th percentile of vegetation index (NDVI)
within the analysis cell (1 ha)

\textbf{Latvian name:} Maksimālā 75. procentiles pēdējo gadu veģetācijas
indeksa (NDVI) vērtība analīzes šūnā (1 ha)

\textbf{Procedure:} Directly follows \hyperref[Ch04.13]{preprocessing}. The maximum value
at the analysis cell is calculated using the workflow \texttt{egvtools::input2egv()}. To
protect against potential data loss at edge cells, inverse distance
weighted (power = 2) gap filling is implemented. Finally, the layer is
standardised by subtracting the arithmetic mean and dividing by the root mean
squared error. The ``short-term'' refers to the last five years (2020-2024).

\begin{Shaded}
\begin{Highlighting}[]
\CommentTok{\# libs {-}{-}{-}{-}}
\ControlFlowTok{if}\NormalTok{(}\SpecialCharTok{!}\FunctionTok{require}\NormalTok{(egvtools)) \{remotes}\SpecialCharTok{::}\FunctionTok{install\_github}\NormalTok{(}\StringTok{"aavotins/egvtools"}\NormalTok{); }\FunctionTok{require}\NormalTok{(egvtools)\}}

\CommentTok{\# EO\_NDVI{-}STp75{-}max\_cell.tif {-}{-}{-}{-}}

\NormalTok{egvrez}\OtherTok{=}\FunctionTok{input2egv}\NormalTok{(}\AttributeTok{input=}\StringTok{"./Geodata/2024/S2indices/Mosaics/EO\_NDVI{-}STp75.tif"}\NormalTok{,}
         \AttributeTok{egv\_template=} \StringTok{"./Templates/TemplateRasters/LV100m\_10km.tif"}\NormalTok{,}
         \AttributeTok{summary\_function =} \StringTok{"min"}\NormalTok{,}
         \AttributeTok{missing\_job =} \StringTok{"FillOutput"}\NormalTok{,}
         \AttributeTok{outlocation =} \StringTok{"./RasterGrids\_100m/2024/RAW/"}\NormalTok{,}
         \AttributeTok{outfilename =} \StringTok{"EO\_NDVI{-}STp75{-}max\_cell.tif"}\NormalTok{,}
         \AttributeTok{layername =} \StringTok{"egv\_492"}\NormalTok{,}
         \AttributeTok{idw\_weight =} \DecValTok{2}\NormalTok{,}
         \AttributeTok{plot\_gaps =} \ConstantTok{FALSE}\NormalTok{,}
         \AttributeTok{plot\_final =} \ConstantTok{FALSE}\NormalTok{)}
\NormalTok{egvrez}

\CommentTok{\# standardisation {-}{-}{-}{-}}
\ControlFlowTok{if}\NormalTok{(}\SpecialCharTok{!}\FunctionTok{require}\NormalTok{(terra)) \{}\FunctionTok{install.packages}\NormalTok{(}\StringTok{"terra"}\NormalTok{); }\FunctionTok{require}\NormalTok{(terra)\}}
\ControlFlowTok{if}\NormalTok{(}\SpecialCharTok{!}\FunctionTok{require}\NormalTok{(tidyverse)) \{}\FunctionTok{install.packages}\NormalTok{(}\StringTok{"tidyverse"}\NormalTok{); }\FunctionTok{require}\NormalTok{(tidyverse)\}}

\NormalTok{nosaukums}\OtherTok{=}\StringTok{"EO\_NDVI{-}STp75{-}max\_cell.tif"}
\NormalTok{ielasisanas\_cels}\OtherTok{=}\FunctionTok{paste0}\NormalTok{(}\StringTok{"./RasterGrids\_100m/2024/RAW/"}\NormalTok{,nosaukums)}
\NormalTok{saglabasanas\_cels}\OtherTok{=}\FunctionTok{paste0}\NormalTok{(}\StringTok{"./RasterGrids\_100m/2024/Scaled/"}\NormalTok{,nosaukums)}
\NormalTok{slanis}\OtherTok{=}\FunctionTok{rast}\NormalTok{(ielasisanas\_cels)}
\NormalTok{videjais}\OtherTok{=}\FunctionTok{global}\NormalTok{(slanis,}\AttributeTok{fun=}\StringTok{"mean"}\NormalTok{,}\AttributeTok{na.rm=}\ConstantTok{TRUE}\NormalTok{)}
\NormalTok{centrets}\OtherTok{=}\NormalTok{slanis}\SpecialCharTok{{-}}\NormalTok{videjais[,}\DecValTok{1}\NormalTok{]}
\NormalTok{standartnovirze}\OtherTok{=}\NormalTok{terra}\SpecialCharTok{::}\FunctionTok{global}\NormalTok{(centrets,}\AttributeTok{fun=}\StringTok{"rms"}\NormalTok{,}\AttributeTok{na.rm=}\ConstantTok{TRUE}\NormalTok{)}
\NormalTok{merogots}\OtherTok{=}\NormalTok{centrets}\SpecialCharTok{/}\NormalTok{standartnovirze[,}\DecValTok{1}\NormalTok{]}
\FunctionTok{writeRaster}\NormalTok{(merogots,}
      \AttributeTok{filename=}\NormalTok{saglabasanas\_cels,}
      \AttributeTok{overwrite=}\ConstantTok{TRUE}\NormalTok{)}
\end{Highlighting}
\end{Shaded}

\section{EO\_NDWI-LYmedian-average\_cell}\label{ch06.493}

\textbf{filename:} \texttt{EO\_NDWI-LYmedian-average\_cell.tif}

\textbf{layername:} \texttt{egv\_493}

\textbf{English name:} Median water index (NDWI) for the last year within the
analysis cell (1 ha)

\textbf{Latvian name:} Mediānā pēdējā gada ūdens indeksa (NDWI) vērtība analīzes šūnā (1 ha)

\textbf{Procedure:} Directly follows \hyperref[Ch04.13]{preprocessing}. The arithmetic mean value
at the analysis cell is calculated using the workflow \texttt{egvtools::input2egv()}. To protect against
potential data loss at edge cells, inverse distance weighted (power = 2) gap
filling is implemented. Finally, the layer is
standardised by subtracting the arithmetic mean and
dividing by the root mean squared error. The ``last year'' is 2024.

\begin{Shaded}
\begin{Highlighting}[]
\CommentTok{\# libs {-}{-}{-}{-}}
\ControlFlowTok{if}\NormalTok{(}\SpecialCharTok{!}\FunctionTok{require}\NormalTok{(egvtools)) \{remotes}\SpecialCharTok{::}\FunctionTok{install\_github}\NormalTok{(}\StringTok{"aavotins/egvtools"}\NormalTok{); }\FunctionTok{require}\NormalTok{(egvtools)\}}

\NormalTok{egvrez}\OtherTok{=}\FunctionTok{input2egv}\NormalTok{(}\AttributeTok{input=}\StringTok{"./Geodata/2024/S2indices/Mosaics/EO\_NDWI{-}LYmedian.tif"}\NormalTok{,}
         \AttributeTok{egv\_template=} \StringTok{"./Templates/TemplateRasters/LV100m\_10km.tif"}\NormalTok{,}
         \AttributeTok{summary\_function =} \StringTok{"average"}\NormalTok{,}
         \AttributeTok{missing\_job =} \StringTok{"FillOutput"}\NormalTok{,}
         \AttributeTok{outlocation =} \StringTok{"./RasterGrids\_100m/2024/RAW/"}\NormalTok{,}
         \AttributeTok{outfilename =} \StringTok{"EO\_NDWI{-}LYmedian{-}average\_cell.tif"}\NormalTok{,}
         \AttributeTok{layername =} \StringTok{"egv\_493"}\NormalTok{,}
         \AttributeTok{idw\_weight =} \DecValTok{2}\NormalTok{,}
         \AttributeTok{plot\_gaps =} \ConstantTok{FALSE}\NormalTok{,}
         \AttributeTok{plot\_final =} \ConstantTok{FALSE}\NormalTok{)}
\NormalTok{egvrez}

\CommentTok{\# standardisation {-}{-}{-}{-}}
\ControlFlowTok{if}\NormalTok{(}\SpecialCharTok{!}\FunctionTok{require}\NormalTok{(terra)) \{}\FunctionTok{install.packages}\NormalTok{(}\StringTok{"terra"}\NormalTok{); }\FunctionTok{require}\NormalTok{(terra)\}}
\ControlFlowTok{if}\NormalTok{(}\SpecialCharTok{!}\FunctionTok{require}\NormalTok{(tidyverse)) \{}\FunctionTok{install.packages}\NormalTok{(}\StringTok{"tidyverse"}\NormalTok{); }\FunctionTok{require}\NormalTok{(tidyverse)\}}

\NormalTok{nosaukums}\OtherTok{=}\StringTok{"EO\_NDWI{-}LYmedian{-}average\_cell.tif"}
\NormalTok{ielasisanas\_cels}\OtherTok{=}\FunctionTok{paste0}\NormalTok{(}\StringTok{"./RasterGrids\_100m/2024/RAW/"}\NormalTok{,nosaukums)}
\NormalTok{saglabasanas\_cels}\OtherTok{=}\FunctionTok{paste0}\NormalTok{(}\StringTok{"./RasterGrids\_100m/2024/Scaled/"}\NormalTok{,nosaukums)}
\NormalTok{slanis}\OtherTok{=}\FunctionTok{rast}\NormalTok{(ielasisanas\_cels)}
\NormalTok{videjais}\OtherTok{=}\FunctionTok{global}\NormalTok{(slanis,}\AttributeTok{fun=}\StringTok{"mean"}\NormalTok{,}\AttributeTok{na.rm=}\ConstantTok{TRUE}\NormalTok{)}
\NormalTok{centrets}\OtherTok{=}\NormalTok{slanis}\SpecialCharTok{{-}}\NormalTok{videjais[,}\DecValTok{1}\NormalTok{]}
\NormalTok{standartnovirze}\OtherTok{=}\NormalTok{terra}\SpecialCharTok{::}\FunctionTok{global}\NormalTok{(centrets,}\AttributeTok{fun=}\StringTok{"rms"}\NormalTok{,}\AttributeTok{na.rm=}\ConstantTok{TRUE}\NormalTok{)}
\NormalTok{merogots}\OtherTok{=}\NormalTok{centrets}\SpecialCharTok{/}\NormalTok{standartnovirze[,}\DecValTok{1}\NormalTok{]}
\FunctionTok{writeRaster}\NormalTok{(merogots,}
      \AttributeTok{filename=}\NormalTok{saglabasanas\_cels,}
      \AttributeTok{overwrite=}\ConstantTok{TRUE}\NormalTok{)}
\end{Highlighting}
\end{Shaded}

\section{EO\_NDWI-LYmedian-iqr\_cell}\label{ch06.494}

\textbf{filename:} \texttt{EO\_NDWI-LYmedian-iqr\_cell.tif}

\textbf{layername:} \texttt{egv\_494}

\textbf{English name:} Spatial variability of last year's median water index (NDWI)
within the analysis cell (1 ha)

\textbf{Latvian name:} Telpiskā variabilitāte pēdējā gada mediānajai ūdens indeksa
(NDWI) vērtībai analīzes šūnā (1 ha)

\textbf{Procedure:} Directly follows \hyperref[Ch04.13]{preprocessing}. The
workflow \texttt{egvtools::input2egv()} is used to calculate Q1 and Q3 for every cell.
To protect against potential data loss at the edges, inverse distance
weighted (power = 2) gap filling is implemented. Next, Q1 is subtracted from Q3.
Finally, the layer is standardised by subtracting the arithmetic mean and
dividing by the root mean squared error. The ``last year'' is 2024.

\begin{Shaded}
\begin{Highlighting}[]
\CommentTok{\# libs {-}{-}{-}{-}}
\ControlFlowTok{if}\NormalTok{(}\SpecialCharTok{!}\FunctionTok{require}\NormalTok{(egvtools)) \{remotes}\SpecialCharTok{::}\FunctionTok{install\_github}\NormalTok{(}\StringTok{"aavotins/egvtools"}\NormalTok{); }\FunctionTok{require}\NormalTok{(egvtools)\}}

\CommentTok{\# EO\_NDWI{-}LYmedian{-}iqr\_cell.tif {-}{-}{-}{-}}


\NormalTok{p25rez}\OtherTok{=}\FunctionTok{input2egv}\NormalTok{(}\AttributeTok{input=}\StringTok{"./Geodata/2024/S2indices/Mosaics/EO\_NDWI{-}LYmedian.tif"}\NormalTok{,}
         \AttributeTok{egv\_template=} \StringTok{"./Templates/TemplateRasters/LV100m\_10km.tif"}\NormalTok{,}
         \AttributeTok{summary\_function =} \StringTok{"q1"}\NormalTok{,}
         \AttributeTok{missing\_job =} \StringTok{"FillOutput"}\NormalTok{,}
         \AttributeTok{outlocation =} \StringTok{"./RasterGrids\_100m/2024/"}\NormalTok{,}
         \AttributeTok{outfilename =} \StringTok{"draza\_p25.tif"}\NormalTok{,}
         \AttributeTok{layername =} \StringTok{"egv\_494"}\NormalTok{,}
         \AttributeTok{idw\_weight =} \DecValTok{2}\NormalTok{,}
         \AttributeTok{plot\_gaps =} \ConstantTok{FALSE}\NormalTok{,}
         \AttributeTok{plot\_final =} \ConstantTok{FALSE}\NormalTok{)}
\NormalTok{p25rez\_r}\OtherTok{=}\FunctionTok{rast}\NormalTok{(}\StringTok{"./RasterGrids\_100m/2024/draza\_p25.tif"}\NormalTok{)}


\NormalTok{p75rez}\OtherTok{=}\FunctionTok{input2egv}\NormalTok{(}\AttributeTok{input=}\StringTok{"./Geodata/2024/S2indices/Mosaics/EO\_NDWI{-}LYmedian.tif"}\NormalTok{,}
         \AttributeTok{egv\_template=} \StringTok{"./Templates/TemplateRasters/LV100m\_10km.tif"}\NormalTok{,}
         \AttributeTok{summary\_function =} \StringTok{"q3"}\NormalTok{,}
         \AttributeTok{missing\_job =} \StringTok{"FillOutput"}\NormalTok{,}
         \AttributeTok{outlocation =} \StringTok{"./RasterGrids\_100m/2024/"}\NormalTok{,}
         \AttributeTok{outfilename =} \StringTok{"draza\_p75.tif"}\NormalTok{,}
         \AttributeTok{layername =} \StringTok{"egv\_494"}\NormalTok{,}
         \AttributeTok{idw\_weight =} \DecValTok{2}\NormalTok{,}
         \AttributeTok{plot\_gaps =} \ConstantTok{FALSE}\NormalTok{,}
         \AttributeTok{plot\_final =} \ConstantTok{FALSE}\NormalTok{)}
\NormalTok{p75rez\_r}\OtherTok{=}\FunctionTok{rast}\NormalTok{(}\StringTok{"./RasterGrids\_100m/2024/draza\_p75.tif"}\NormalTok{)}

\NormalTok{iqr\_rez}\OtherTok{=}\NormalTok{p75rez\_r}\SpecialCharTok{{-}}\NormalTok{p25rez\_r}
\NormalTok{iqr\_rez}
\FunctionTok{plot}\NormalTok{(iqr\_rez)}

\FunctionTok{writeRaster}\NormalTok{(iqr\_rez,}
      \StringTok{"./RasterGrids\_100m/2024/RAW/EO\_NDWI{-}LYmedian{-}iqr\_cell.tif"}\NormalTok{,}
      \AttributeTok{overwrite=}\ConstantTok{TRUE}\NormalTok{)}

\FunctionTok{unlink}\NormalTok{(}\StringTok{"./RasterGrids\_100m/2024/draza\_p75.tif"}\NormalTok{)}
\FunctionTok{unlink}\NormalTok{(}\StringTok{"./RasterGrids\_100m/2024/draza\_p25.tif"}\NormalTok{)}

\CommentTok{\# standardisation {-}{-}{-}{-}}
\ControlFlowTok{if}\NormalTok{(}\SpecialCharTok{!}\FunctionTok{require}\NormalTok{(terra)) \{}\FunctionTok{install.packages}\NormalTok{(}\StringTok{"terra"}\NormalTok{); }\FunctionTok{require}\NormalTok{(terra)\}}
\ControlFlowTok{if}\NormalTok{(}\SpecialCharTok{!}\FunctionTok{require}\NormalTok{(tidyverse)) \{}\FunctionTok{install.packages}\NormalTok{(}\StringTok{"tidyverse"}\NormalTok{); }\FunctionTok{require}\NormalTok{(tidyverse)\}}

\NormalTok{nosaukums}\OtherTok{=}\StringTok{"EO\_NDWI{-}LYmedian{-}iqr\_cell.tif"}
\NormalTok{ielasisanas\_cels}\OtherTok{=}\FunctionTok{paste0}\NormalTok{(}\StringTok{"./RasterGrids\_100m/2024/RAW/"}\NormalTok{,nosaukums)}
\NormalTok{saglabasanas\_cels}\OtherTok{=}\FunctionTok{paste0}\NormalTok{(}\StringTok{"./RasterGrids\_100m/2024/Scaled/"}\NormalTok{,nosaukums)}
\NormalTok{slanis}\OtherTok{=}\FunctionTok{rast}\NormalTok{(ielasisanas\_cels)}
\NormalTok{videjais}\OtherTok{=}\FunctionTok{global}\NormalTok{(slanis,}\AttributeTok{fun=}\StringTok{"mean"}\NormalTok{,}\AttributeTok{na.rm=}\ConstantTok{TRUE}\NormalTok{)}
\NormalTok{centrets}\OtherTok{=}\NormalTok{slanis}\SpecialCharTok{{-}}\NormalTok{videjais[,}\DecValTok{1}\NormalTok{]}
\NormalTok{standartnovirze}\OtherTok{=}\NormalTok{terra}\SpecialCharTok{::}\FunctionTok{global}\NormalTok{(centrets,}\AttributeTok{fun=}\StringTok{"rms"}\NormalTok{,}\AttributeTok{na.rm=}\ConstantTok{TRUE}\NormalTok{)}
\NormalTok{merogots}\OtherTok{=}\NormalTok{centrets}\SpecialCharTok{/}\NormalTok{standartnovirze[,}\DecValTok{1}\NormalTok{]}
\FunctionTok{writeRaster}\NormalTok{(merogots,}
      \AttributeTok{filename=}\NormalTok{saglabasanas\_cels,}
      \AttributeTok{overwrite=}\ConstantTok{TRUE}\NormalTok{)}
\end{Highlighting}
\end{Shaded}

\section{EO\_NDWI-STiqr-median\_cell}\label{ch06.495}

\textbf{filename:} \texttt{EO\_NDWI-STiqr-median\_cell.tif}

\textbf{layername:} \texttt{egv\_495}

\textbf{English name:} Average short-term seasonality of water index (NDWI) within
the analysis cell (1 ha)

\textbf{Latvian name:} Sezonalitāte pēdējo gadu vidējai ūdens indeksa (NDWI)
vērtībai analīzes šūnā (1 ha)

\textbf{Procedure:} Directly follows \hyperref[Ch04.13]{preprocessing}. The arithmetic mean value
at the analysis cell is calculated using the workflow \texttt{egvtools::input2egv()}. To
protect against potential data loss at edge cells, inverse distance
weighted (power = 2) gap filling is implemented. Finally, the layer is
standardised by subtracting the arithmetic mean and dividing by the root mean
squared error. The ``short-term'' refers to the last five years (2020-2024).

\begin{Shaded}
\begin{Highlighting}[]
\CommentTok{\# libs {-}{-}{-}{-}}
\ControlFlowTok{if}\NormalTok{(}\SpecialCharTok{!}\FunctionTok{require}\NormalTok{(egvtools)) \{remotes}\SpecialCharTok{::}\FunctionTok{install\_github}\NormalTok{(}\StringTok{"aavotins/egvtools"}\NormalTok{); }\FunctionTok{require}\NormalTok{(egvtools)\}}

\CommentTok{\# EO\_NDWI{-}STiqr{-}median\_cell.tif {-}{-}{-}{-}}

\NormalTok{egvrez}\OtherTok{=}\FunctionTok{input2egv}\NormalTok{(}\AttributeTok{input=}\StringTok{"./Geodata/2024/S2indices/Mosaics/EO\_NDWI{-}STiqr.tif"}\NormalTok{,}
         \AttributeTok{egv\_template=} \StringTok{"./Templates/TemplateRasters/LV100m\_10km.tif"}\NormalTok{,}
         \AttributeTok{summary\_function =} \StringTok{"average"}\NormalTok{,}
         \AttributeTok{missing\_job =} \StringTok{"FillOutput"}\NormalTok{,}
         \AttributeTok{outlocation =} \StringTok{"./RasterGrids\_100m/2024/RAW/"}\NormalTok{,}
         \AttributeTok{outfilename =} \StringTok{"EO\_NDWI{-}STiqr{-}median\_cell.tif"}\NormalTok{,}
         \AttributeTok{layername =} \StringTok{"egv\_495"}\NormalTok{,}
         \AttributeTok{idw\_weight =} \DecValTok{2}\NormalTok{,}
         \AttributeTok{plot\_gaps =} \ConstantTok{FALSE}\NormalTok{,}
         \AttributeTok{plot\_final =} \ConstantTok{FALSE}\NormalTok{)}
\NormalTok{egvrez}

\CommentTok{\# standardisation {-}{-}{-}{-}}
\ControlFlowTok{if}\NormalTok{(}\SpecialCharTok{!}\FunctionTok{require}\NormalTok{(terra)) \{}\FunctionTok{install.packages}\NormalTok{(}\StringTok{"terra"}\NormalTok{); }\FunctionTok{require}\NormalTok{(terra)\}}
\ControlFlowTok{if}\NormalTok{(}\SpecialCharTok{!}\FunctionTok{require}\NormalTok{(tidyverse)) \{}\FunctionTok{install.packages}\NormalTok{(}\StringTok{"tidyverse"}\NormalTok{); }\FunctionTok{require}\NormalTok{(tidyverse)\}}

\NormalTok{nosaukums}\OtherTok{=}\StringTok{"EO\_NDWI{-}STiqr{-}median\_cell.tif"}
\NormalTok{ielasisanas\_cels}\OtherTok{=}\FunctionTok{paste0}\NormalTok{(}\StringTok{"./RasterGrids\_100m/2024/RAW/"}\NormalTok{,nosaukums)}
\NormalTok{saglabasanas\_cels}\OtherTok{=}\FunctionTok{paste0}\NormalTok{(}\StringTok{"./RasterGrids\_100m/2024/Scaled/"}\NormalTok{,nosaukums)}
\NormalTok{slanis}\OtherTok{=}\FunctionTok{rast}\NormalTok{(ielasisanas\_cels)}
\NormalTok{videjais}\OtherTok{=}\FunctionTok{global}\NormalTok{(slanis,}\AttributeTok{fun=}\StringTok{"mean"}\NormalTok{,}\AttributeTok{na.rm=}\ConstantTok{TRUE}\NormalTok{)}
\NormalTok{centrets}\OtherTok{=}\NormalTok{slanis}\SpecialCharTok{{-}}\NormalTok{videjais[,}\DecValTok{1}\NormalTok{]}
\NormalTok{standartnovirze}\OtherTok{=}\NormalTok{terra}\SpecialCharTok{::}\FunctionTok{global}\NormalTok{(centrets,}\AttributeTok{fun=}\StringTok{"rms"}\NormalTok{,}\AttributeTok{na.rm=}\ConstantTok{TRUE}\NormalTok{)}
\NormalTok{merogots}\OtherTok{=}\NormalTok{centrets}\SpecialCharTok{/}\NormalTok{standartnovirze[,}\DecValTok{1}\NormalTok{]}
\FunctionTok{writeRaster}\NormalTok{(merogots,}
      \AttributeTok{filename=}\NormalTok{saglabasanas\_cels,}
      \AttributeTok{overwrite=}\ConstantTok{TRUE}\NormalTok{)}
\end{Highlighting}
\end{Shaded}

\section{EO\_NDWI-STmedian-average\_cell}\label{ch06.496}

\textbf{filename:} \texttt{EO\_NDWI-STmedian-average\_cell.tif}

\textbf{layername:} \texttt{egv\_496}

\textbf{English name:} Median short-term water index (NDWI) within the analysis cell
(1 ha)

\textbf{Latvian name:} Mediānā pēdējo gadu ūdens indeksa (NDWI) vērtība analīzes šūnā (1 ha)

\textbf{Procedure:} Directly follows \hyperref[Ch04.13]{preprocessing}. The arithmetic mean value
at the analysis cell is calculated using the workflow \texttt{egvtools::input2egv()}. To
protect against potential data loss at edge cells, inverse distance
weighted (power = 2) gap filling is implemented. Finally, the layer is
standardised by subtracting the arithmetic mean and dividing by the root mean
squared error. The ``short-term'' refers to the last five years (2020-2024).

\begin{Shaded}
\begin{Highlighting}[]
\CommentTok{\# libs {-}{-}{-}{-}}
\ControlFlowTok{if}\NormalTok{(}\SpecialCharTok{!}\FunctionTok{require}\NormalTok{(egvtools)) \{remotes}\SpecialCharTok{::}\FunctionTok{install\_github}\NormalTok{(}\StringTok{"aavotins/egvtools"}\NormalTok{); }\FunctionTok{require}\NormalTok{(egvtools)\}}

\CommentTok{\# EO\_NDWI{-}STmedian{-}average\_cell.tif {-}{-}{-}{-}}

\NormalTok{egvrez}\OtherTok{=}\FunctionTok{input2egv}\NormalTok{(}\AttributeTok{input=}\StringTok{"./Geodata/2024/S2indices/Mosaics/EO\_NDWI{-}STmedian.tif"}\NormalTok{,}
         \AttributeTok{egv\_template=} \StringTok{"./Templates/TemplateRasters/LV100m\_10km.tif"}\NormalTok{,}
         \AttributeTok{summary\_function =} \StringTok{"average"}\NormalTok{,}
         \AttributeTok{missing\_job =} \StringTok{"FillOutput"}\NormalTok{,}
         \AttributeTok{outlocation =} \StringTok{"./RasterGrids\_100m/2024/RAW/"}\NormalTok{,}
         \AttributeTok{outfilename =} \StringTok{"EO\_NDWI{-}STmedian{-}average\_cell.tif"}\NormalTok{,}
         \AttributeTok{layername =} \StringTok{"egv\_496"}\NormalTok{,}
         \AttributeTok{idw\_weight =} \DecValTok{2}\NormalTok{,}
         \AttributeTok{plot\_gaps =} \ConstantTok{FALSE}\NormalTok{,}
         \AttributeTok{plot\_final =} \ConstantTok{FALSE}\NormalTok{)}
\NormalTok{egvrez}

\CommentTok{\# standardisation {-}{-}{-}{-}}
\ControlFlowTok{if}\NormalTok{(}\SpecialCharTok{!}\FunctionTok{require}\NormalTok{(terra)) \{}\FunctionTok{install.packages}\NormalTok{(}\StringTok{"terra"}\NormalTok{); }\FunctionTok{require}\NormalTok{(terra)\}}
\ControlFlowTok{if}\NormalTok{(}\SpecialCharTok{!}\FunctionTok{require}\NormalTok{(tidyverse)) \{}\FunctionTok{install.packages}\NormalTok{(}\StringTok{"tidyverse"}\NormalTok{); }\FunctionTok{require}\NormalTok{(tidyverse)\}}

\NormalTok{nosaukums}\OtherTok{=}\StringTok{"EO\_NDWI{-}STmedian{-}average\_cell.tif"}
\NormalTok{ielasisanas\_cels}\OtherTok{=}\FunctionTok{paste0}\NormalTok{(}\StringTok{"./RasterGrids\_100m/2024/RAW/"}\NormalTok{,nosaukums)}
\NormalTok{saglabasanas\_cels}\OtherTok{=}\FunctionTok{paste0}\NormalTok{(}\StringTok{"./RasterGrids\_100m/2024/Scaled/"}\NormalTok{,nosaukums)}
\NormalTok{slanis}\OtherTok{=}\FunctionTok{rast}\NormalTok{(ielasisanas\_cels)}
\NormalTok{videjais}\OtherTok{=}\FunctionTok{global}\NormalTok{(slanis,}\AttributeTok{fun=}\StringTok{"mean"}\NormalTok{,}\AttributeTok{na.rm=}\ConstantTok{TRUE}\NormalTok{)}
\NormalTok{centrets}\OtherTok{=}\NormalTok{slanis}\SpecialCharTok{{-}}\NormalTok{videjais[,}\DecValTok{1}\NormalTok{]}
\NormalTok{standartnovirze}\OtherTok{=}\NormalTok{terra}\SpecialCharTok{::}\FunctionTok{global}\NormalTok{(centrets,}\AttributeTok{fun=}\StringTok{"rms"}\NormalTok{,}\AttributeTok{na.rm=}\ConstantTok{TRUE}\NormalTok{)}
\NormalTok{merogots}\OtherTok{=}\NormalTok{centrets}\SpecialCharTok{/}\NormalTok{standartnovirze[,}\DecValTok{1}\NormalTok{]}
\FunctionTok{writeRaster}\NormalTok{(merogots,}
      \AttributeTok{filename=}\NormalTok{saglabasanas\_cels,}
      \AttributeTok{overwrite=}\ConstantTok{TRUE}\NormalTok{)}
\end{Highlighting}
\end{Shaded}

\section{EO\_NDWI-STmedian-iqr\_cell}\label{ch06.497}

\textbf{filename:} \texttt{EO\_NDWI-STmedian-iqr\_cell.tif}

\textbf{layername:} \texttt{egv\_497}

\textbf{English name:} Spatial variability of short-term median water index (NDWI)
within the analysis cell (1 ha)

\textbf{Latvian name:} Telpiskā variabilitāte pēdējo gadu mediānajai ūdens
indeksa (NDWI) vērtībai analīzes šūnā (1 ha)

\textbf{Procedure:} Directly follows \hyperref[Ch04.13]{preprocessing}. The
workflow \texttt{egvtools::input2egv()} is used to calculate Q1 and Q3 for every cell.
To protect against potential data loss at the edges, inverse distance
weighted (power = 2) gap filling is implemented. Next, Q1 is subtracted from Q3.
Finally, the layer is standardised by subtracting the arithmetic mean and
dividing by the root mean squared error. The ``short-term'' refers to the last
five years (2020-2024).

\begin{Shaded}
\begin{Highlighting}[]
\CommentTok{\# libs {-}{-}{-}{-}}
\ControlFlowTok{if}\NormalTok{(}\SpecialCharTok{!}\FunctionTok{require}\NormalTok{(egvtools)) \{remotes}\SpecialCharTok{::}\FunctionTok{install\_github}\NormalTok{(}\StringTok{"aavotins/egvtools"}\NormalTok{); }\FunctionTok{require}\NormalTok{(egvtools)\}}


\CommentTok{\# EO\_NDWI{-}STmedian{-}iqr\_cell.tif {-}{-}{-}{-}}

\NormalTok{p25rez}\OtherTok{=}\FunctionTok{input2egv}\NormalTok{(}\AttributeTok{input=}\StringTok{"./Geodata/2024/S2indices/Mosaics/EO\_NDWI{-}STmedian.tif"}\NormalTok{,}
         \AttributeTok{egv\_template=} \StringTok{"./Templates/TemplateRasters/LV100m\_10km.tif"}\NormalTok{,}
         \AttributeTok{summary\_function =} \StringTok{"q1"}\NormalTok{,}
         \AttributeTok{missing\_job =} \StringTok{"FillOutput"}\NormalTok{,}
         \AttributeTok{outlocation =} \StringTok{"./RasterGrids\_100m/2024/"}\NormalTok{,}
         \AttributeTok{outfilename =} \StringTok{"draza\_p25.tif"}\NormalTok{,}
         \AttributeTok{layername =} \StringTok{"egv\_497"}\NormalTok{,}
         \AttributeTok{idw\_weight =} \DecValTok{2}\NormalTok{,}
         \AttributeTok{plot\_gaps =} \ConstantTok{FALSE}\NormalTok{,}
         \AttributeTok{plot\_final =} \ConstantTok{FALSE}\NormalTok{)}
\NormalTok{p25rez\_r}\OtherTok{=}\FunctionTok{rast}\NormalTok{(}\StringTok{"./RasterGrids\_100m/2024/draza\_p25.tif"}\NormalTok{)}


\NormalTok{p75rez}\OtherTok{=}\FunctionTok{input2egv}\NormalTok{(}\AttributeTok{input=}\StringTok{"./Geodata/2024/S2indices/Mosaics/EO\_NDWI{-}STmedian.tif"}\NormalTok{,}
         \AttributeTok{egv\_template=} \StringTok{"./Templates/TemplateRasters/LV100m\_10km.tif"}\NormalTok{,}
         \AttributeTok{summary\_function =} \StringTok{"q3"}\NormalTok{,}
         \AttributeTok{missing\_job =} \StringTok{"FillOutput"}\NormalTok{,}
         \AttributeTok{outlocation =} \StringTok{"./RasterGrids\_100m/2024/"}\NormalTok{,}
         \AttributeTok{outfilename =} \StringTok{"draza\_p75.tif"}\NormalTok{,}
         \AttributeTok{layername =} \StringTok{"egv\_497"}\NormalTok{,}
         \AttributeTok{idw\_weight =} \DecValTok{2}\NormalTok{,}
         \AttributeTok{plot\_gaps =} \ConstantTok{FALSE}\NormalTok{,}
         \AttributeTok{plot\_final =} \ConstantTok{FALSE}\NormalTok{)}
\NormalTok{p75rez\_r}\OtherTok{=}\FunctionTok{rast}\NormalTok{(}\StringTok{"./RasterGrids\_100m/2024/draza\_p75.tif"}\NormalTok{)}

\NormalTok{iqr\_rez}\OtherTok{=}\NormalTok{p75rez\_r}\SpecialCharTok{{-}}\NormalTok{p25rez\_r}
\NormalTok{iqr\_rez}
\FunctionTok{plot}\NormalTok{(iqr\_rez)}

\FunctionTok{writeRaster}\NormalTok{(iqr\_rez,}
      \StringTok{"./RasterGrids\_100m/2024/RAW/EO\_NDWI{-}STmedian{-}iqr\_cell.tif"}\NormalTok{,}
      \AttributeTok{overwrite=}\ConstantTok{TRUE}\NormalTok{)}

\FunctionTok{unlink}\NormalTok{(}\StringTok{"./RasterGrids\_100m/2024/draza\_p75.tif"}\NormalTok{)}
\FunctionTok{unlink}\NormalTok{(}\StringTok{"./RasterGrids\_100m/2024/draza\_p25.tif"}\NormalTok{)}

\CommentTok{\# standardisation {-}{-}{-}{-}}
\ControlFlowTok{if}\NormalTok{(}\SpecialCharTok{!}\FunctionTok{require}\NormalTok{(terra)) \{}\FunctionTok{install.packages}\NormalTok{(}\StringTok{"terra"}\NormalTok{); }\FunctionTok{require}\NormalTok{(terra)\}}
\ControlFlowTok{if}\NormalTok{(}\SpecialCharTok{!}\FunctionTok{require}\NormalTok{(tidyverse)) \{}\FunctionTok{install.packages}\NormalTok{(}\StringTok{"tidyverse"}\NormalTok{); }\FunctionTok{require}\NormalTok{(tidyverse)\}}

\NormalTok{nosaukums}\OtherTok{=}\StringTok{"EO\_NDWI{-}STmedian{-}iqr\_cell.tif"}
\NormalTok{ielasisanas\_cels}\OtherTok{=}\FunctionTok{paste0}\NormalTok{(}\StringTok{"./RasterGrids\_100m/2024/RAW/"}\NormalTok{,nosaukums)}
\NormalTok{saglabasanas\_cels}\OtherTok{=}\FunctionTok{paste0}\NormalTok{(}\StringTok{"./RasterGrids\_100m/2024/Scaled/"}\NormalTok{,nosaukums)}
\NormalTok{slanis}\OtherTok{=}\FunctionTok{rast}\NormalTok{(ielasisanas\_cels)}
\NormalTok{videjais}\OtherTok{=}\FunctionTok{global}\NormalTok{(slanis,}\AttributeTok{fun=}\StringTok{"mean"}\NormalTok{,}\AttributeTok{na.rm=}\ConstantTok{TRUE}\NormalTok{)}
\NormalTok{centrets}\OtherTok{=}\NormalTok{slanis}\SpecialCharTok{{-}}\NormalTok{videjais[,}\DecValTok{1}\NormalTok{]}
\NormalTok{standartnovirze}\OtherTok{=}\NormalTok{terra}\SpecialCharTok{::}\FunctionTok{global}\NormalTok{(centrets,}\AttributeTok{fun=}\StringTok{"rms"}\NormalTok{,}\AttributeTok{na.rm=}\ConstantTok{TRUE}\NormalTok{)}
\NormalTok{merogots}\OtherTok{=}\NormalTok{centrets}\SpecialCharTok{/}\NormalTok{standartnovirze[,}\DecValTok{1}\NormalTok{]}
\FunctionTok{writeRaster}\NormalTok{(merogots,}
      \AttributeTok{filename=}\NormalTok{saglabasanas\_cels,}
      \AttributeTok{overwrite=}\ConstantTok{TRUE}\NormalTok{)}
\end{Highlighting}
\end{Shaded}

\section{EO\_NDWI-STp25-min\_cell}\label{ch06.498}

\textbf{filename:} \texttt{EO\_NDWI-STp25-min\_cell.tif}

\textbf{layername:} \texttt{egv\_498}

\textbf{English name:} Minimum short-term 25th percentile of water index (NDWI)
within the analysis cell (1 ha)

\textbf{Latvian name:} Minimālā 25. procentiles pēdējo gadu ūdens indeksa
(NDWI) vērtība analīzes šūnā (1 ha)

\textbf{Procedure:} Directly follows \hyperref[Ch04.13]{preprocessing}. The minimum value
at the analysis cell is calculated using the workflow \texttt{egvtools::input2egv()}. To
protect against potential data loss at edge cells, inverse distance
weighted (power = 2) gap filling is implemented. Finally, the layer is
standardised by subtracting the arithmetic mean and dividing by the root mean
squared error. The ``short-term'' refers to the last five years (2020-2024).

\begin{Shaded}
\begin{Highlighting}[]
\CommentTok{\# libs {-}{-}{-}{-}}
\ControlFlowTok{if}\NormalTok{(}\SpecialCharTok{!}\FunctionTok{require}\NormalTok{(egvtools)) \{remotes}\SpecialCharTok{::}\FunctionTok{install\_github}\NormalTok{(}\StringTok{"aavotins/egvtools"}\NormalTok{); }\FunctionTok{require}\NormalTok{(egvtools)\}}

\CommentTok{\# EO\_NDWI{-}STp25{-}min\_cell.tif {-}{-}{-}{-}}

\NormalTok{egvrez}\OtherTok{=}\FunctionTok{input2egv}\NormalTok{(}\AttributeTok{input=}\StringTok{"./Geodata/2024/S2indices/Mosaics/EO\_NDWI{-}STp25.tif"}\NormalTok{,}
         \AttributeTok{egv\_template=} \StringTok{"./Templates/TemplateRasters/LV100m\_10km.tif"}\NormalTok{,}
         \AttributeTok{summary\_function =} \StringTok{"min"}\NormalTok{,}
         \AttributeTok{missing\_job =} \StringTok{"FillOutput"}\NormalTok{,}
         \AttributeTok{outlocation =} \StringTok{"./RasterGrids\_100m/2024/RAW/"}\NormalTok{,}
         \AttributeTok{outfilename =} \StringTok{"EO\_NDWI{-}STp25{-}min\_cell.tif"}\NormalTok{,}
         \AttributeTok{layername =} \StringTok{"egv\_498"}\NormalTok{,}
         \AttributeTok{idw\_weight =} \DecValTok{2}\NormalTok{,}
         \AttributeTok{plot\_gaps =} \ConstantTok{FALSE}\NormalTok{,}
         \AttributeTok{plot\_final =} \ConstantTok{FALSE}\NormalTok{)}
\NormalTok{egvrez}

\CommentTok{\# standardisation {-}{-}{-}{-}}
\ControlFlowTok{if}\NormalTok{(}\SpecialCharTok{!}\FunctionTok{require}\NormalTok{(terra)) \{}\FunctionTok{install.packages}\NormalTok{(}\StringTok{"terra"}\NormalTok{); }\FunctionTok{require}\NormalTok{(terra)\}}
\ControlFlowTok{if}\NormalTok{(}\SpecialCharTok{!}\FunctionTok{require}\NormalTok{(tidyverse)) \{}\FunctionTok{install.packages}\NormalTok{(}\StringTok{"tidyverse"}\NormalTok{); }\FunctionTok{require}\NormalTok{(tidyverse)\}}

\NormalTok{nosaukums}\OtherTok{=}\StringTok{"EO\_NDWI{-}STp25{-}min\_cell.tif"}
\NormalTok{ielasisanas\_cels}\OtherTok{=}\FunctionTok{paste0}\NormalTok{(}\StringTok{"./RasterGrids\_100m/2024/RAW/"}\NormalTok{,nosaukums)}
\NormalTok{saglabasanas\_cels}\OtherTok{=}\FunctionTok{paste0}\NormalTok{(}\StringTok{"./RasterGrids\_100m/2024/Scaled/"}\NormalTok{,nosaukums)}
\NormalTok{slanis}\OtherTok{=}\FunctionTok{rast}\NormalTok{(ielasisanas\_cels)}
\NormalTok{videjais}\OtherTok{=}\FunctionTok{global}\NormalTok{(slanis,}\AttributeTok{fun=}\StringTok{"mean"}\NormalTok{,}\AttributeTok{na.rm=}\ConstantTok{TRUE}\NormalTok{)}
\NormalTok{centrets}\OtherTok{=}\NormalTok{slanis}\SpecialCharTok{{-}}\NormalTok{videjais[,}\DecValTok{1}\NormalTok{]}
\NormalTok{standartnovirze}\OtherTok{=}\NormalTok{terra}\SpecialCharTok{::}\FunctionTok{global}\NormalTok{(centrets,}\AttributeTok{fun=}\StringTok{"rms"}\NormalTok{,}\AttributeTok{na.rm=}\ConstantTok{TRUE}\NormalTok{)}
\NormalTok{merogots}\OtherTok{=}\NormalTok{centrets}\SpecialCharTok{/}\NormalTok{standartnovirze[,}\DecValTok{1}\NormalTok{]}
\FunctionTok{writeRaster}\NormalTok{(merogots,}
      \AttributeTok{filename=}\NormalTok{saglabasanas\_cels,}
      \AttributeTok{overwrite=}\ConstantTok{TRUE}\NormalTok{)}
\end{Highlighting}
\end{Shaded}

\section{EO\_NDWI-STp75-max\_cell}\label{ch06.499}

\textbf{filename:} \texttt{EO\_NDWI-STp75-max\_cell.tif}

\textbf{layername:} \texttt{egv\_499}

\textbf{English name:} Maximum short-term 75th percentile of water index (NDWI)
within the analysis cell (1 ha)

\textbf{Latvian name:} Maksimālā 75. procentiles pēdējo gadu ūdens indeksa
(NDWI) vērtība analīzes šūnā (1 ha)

\textbf{Procedure:} Directly follows \hyperref[Ch04.13]{preprocessing}. The maximum value
at the analysis cell is calculated using the workflow \texttt{egvtools::input2egv()}. To
protect against potential data loss at edge cells, inverse distance
weighted (power = 2) gap filling is implemented. Finally, the layer is
standardised by subtracting the arithmetic mean and dividing by the root mean
squared error. The ``short-term'' refers to the last five years (2020-2024).

\begin{Shaded}
\begin{Highlighting}[]
\CommentTok{\# libs {-}{-}{-}{-}}
\ControlFlowTok{if}\NormalTok{(}\SpecialCharTok{!}\FunctionTok{require}\NormalTok{(egvtools)) \{remotes}\SpecialCharTok{::}\FunctionTok{install\_github}\NormalTok{(}\StringTok{"aavotins/egvtools"}\NormalTok{); }\FunctionTok{require}\NormalTok{(egvtools)\}}


\CommentTok{\# EO\_NDWI{-}STp75{-}max\_cell.tif {-}{-}{-}{-}}

\NormalTok{egvrez}\OtherTok{=}\FunctionTok{input2egv}\NormalTok{(}\AttributeTok{input=}\StringTok{"./Geodata/2024/S2indices/Mosaics/EO\_NDWI{-}STp75.tif"}\NormalTok{,}
         \AttributeTok{egv\_template=} \StringTok{"./Templates/TemplateRasters/LV100m\_10km.tif"}\NormalTok{,}
         \AttributeTok{summary\_function =} \StringTok{"min"}\NormalTok{,}
         \AttributeTok{missing\_job =} \StringTok{"FillOutput"}\NormalTok{,}
         \AttributeTok{outlocation =} \StringTok{"./RasterGrids\_100m/2024/RAW/"}\NormalTok{,}
         \AttributeTok{outfilename =} \StringTok{"EO\_NDWI{-}STp75{-}max\_cell.tif"}\NormalTok{,}
         \AttributeTok{layername =} \StringTok{"egv\_499"}\NormalTok{,}
         \AttributeTok{idw\_weight =} \DecValTok{2}\NormalTok{,}
         \AttributeTok{plot\_gaps =} \ConstantTok{FALSE}\NormalTok{,}
         \AttributeTok{plot\_final =} \ConstantTok{FALSE}\NormalTok{)}
\NormalTok{egvrez}

\CommentTok{\# standardisation {-}{-}{-}{-}}
\ControlFlowTok{if}\NormalTok{(}\SpecialCharTok{!}\FunctionTok{require}\NormalTok{(terra)) \{}\FunctionTok{install.packages}\NormalTok{(}\StringTok{"terra"}\NormalTok{); }\FunctionTok{require}\NormalTok{(terra)\}}
\ControlFlowTok{if}\NormalTok{(}\SpecialCharTok{!}\FunctionTok{require}\NormalTok{(tidyverse)) \{}\FunctionTok{install.packages}\NormalTok{(}\StringTok{"tidyverse"}\NormalTok{); }\FunctionTok{require}\NormalTok{(tidyverse)\}}

\NormalTok{nosaukums}\OtherTok{=}\StringTok{"EO\_NDWI{-}STp75{-}max\_cell.tif"}
\NormalTok{ielasisanas\_cels}\OtherTok{=}\FunctionTok{paste0}\NormalTok{(}\StringTok{"./RasterGrids\_100m/2024/RAW/"}\NormalTok{,nosaukums)}
\NormalTok{saglabasanas\_cels}\OtherTok{=}\FunctionTok{paste0}\NormalTok{(}\StringTok{"./RasterGrids\_100m/2024/Scaled/"}\NormalTok{,nosaukums)}
\NormalTok{slanis}\OtherTok{=}\FunctionTok{rast}\NormalTok{(ielasisanas\_cels)}
\NormalTok{videjais}\OtherTok{=}\FunctionTok{global}\NormalTok{(slanis,}\AttributeTok{fun=}\StringTok{"mean"}\NormalTok{,}\AttributeTok{na.rm=}\ConstantTok{TRUE}\NormalTok{)}
\NormalTok{centrets}\OtherTok{=}\NormalTok{slanis}\SpecialCharTok{{-}}\NormalTok{videjais[,}\DecValTok{1}\NormalTok{]}
\NormalTok{standartnovirze}\OtherTok{=}\NormalTok{terra}\SpecialCharTok{::}\FunctionTok{global}\NormalTok{(centrets,}\AttributeTok{fun=}\StringTok{"rms"}\NormalTok{,}\AttributeTok{na.rm=}\ConstantTok{TRUE}\NormalTok{)}
\NormalTok{merogots}\OtherTok{=}\NormalTok{centrets}\SpecialCharTok{/}\NormalTok{standartnovirze[,}\DecValTok{1}\NormalTok{]}
\FunctionTok{writeRaster}\NormalTok{(merogots,}
      \AttributeTok{filename=}\NormalTok{saglabasanas\_cels,}
      \AttributeTok{overwrite=}\ConstantTok{TRUE}\NormalTok{)}
\end{Highlighting}
\end{Shaded}

\section{SoilChemistry\_ESDAC-CN\_cell}\label{ch06.500}

\textbf{filename:} \texttt{SoilChemistry\_ESDAC-CN\_cell.tif}

\textbf{layername:} \texttt{egv\_500}

\textbf{English name:} Average value of Topsoil Carbon-Nitrogen ratio (ESDAC v2.0)
within the analysis cell (1 ha)

\textbf{Latvian name:} Augsnes virskārtas oglekļa-slāpekļa attiecība (ESDAC v2.0)
analīzes šūnā (1 ha)

\textbf{Procedure:} Directly derived from the \hyperref[Ch04.07.01]{Soil chemistry}. Processed
using the workflow \texttt{egvtools::downscale2egv()} with \texttt{fill\ gaps\ =\ TRUE}, performing
inverse distance weighted (power = 2) filling of gaps at the border
and \texttt{smooth\ =\ FALSE}. This is done to preserve the original values as much as
possible (bilinear interpolation is involved when projecting from a 500 m
resolution to a 100 m resolution in a different CRS). Finally, the layer is
standardised by subtracting the arithmetic mean and dividing by the root mean
squared error.

\begin{Shaded}
\begin{Highlighting}[]
\CommentTok{\# libs {-}{-}{-}{-}}
\ControlFlowTok{if}\NormalTok{(}\SpecialCharTok{!}\FunctionTok{require}\NormalTok{(egvtools)) \{remotes}\SpecialCharTok{::}\FunctionTok{install\_github}\NormalTok{(}\StringTok{"aavotins/egvtools"}\NormalTok{); }\FunctionTok{require}\NormalTok{(egvtools)\}}


\CommentTok{\# CN {-}{-}{-}{-}}

\NormalTok{egv}\OtherTok{=}\FunctionTok{downscale2egv}\NormalTok{(}
 \AttributeTok{template\_path =} \StringTok{"./Templates/TemplateRasters/LV100m\_10km.tif"}\NormalTok{,}
 \AttributeTok{grid\_path   =} \StringTok{"./Templates/TemplateGrids/tikls1km\_sauzeme.parquet"}\NormalTok{,}
 \AttributeTok{rawfile\_path =} \StringTok{"./Geodata/2024/Soils/ESDAC/chemistry/chemistry/CN/CN.tif"}\NormalTok{,}
 \AttributeTok{out\_path   =} \StringTok{"./RasterGrids\_100m/2024/RAW/"}\NormalTok{,}
 \AttributeTok{file\_name   =} \StringTok{"SoilChemistry\_ESDAC{-}CN\_cell.tif"}\NormalTok{,}
 \AttributeTok{layer\_name  =} \StringTok{"egv\_500"}\NormalTok{,}
 \AttributeTok{fill\_gaps   =} \ConstantTok{TRUE}\NormalTok{,}
 \AttributeTok{smooth    =} \ConstantTok{FALSE}\NormalTok{,}
 \AttributeTok{plot\_result  =} \ConstantTok{TRUE}\NormalTok{)}
\NormalTok{egv}

\CommentTok{\# standardisation {-}{-}{-}{-}}
\ControlFlowTok{if}\NormalTok{(}\SpecialCharTok{!}\FunctionTok{require}\NormalTok{(terra)) \{}\FunctionTok{install.packages}\NormalTok{(}\StringTok{"terra"}\NormalTok{); }\FunctionTok{require}\NormalTok{(terra)\}}
\ControlFlowTok{if}\NormalTok{(}\SpecialCharTok{!}\FunctionTok{require}\NormalTok{(tidyverse)) \{}\FunctionTok{install.packages}\NormalTok{(}\StringTok{"tidyverse"}\NormalTok{); }\FunctionTok{require}\NormalTok{(tidyverse)\}}

\NormalTok{nosaukums}\OtherTok{=}\StringTok{"SoilChemistry\_ESDAC{-}CN\_cell.tif"}
\NormalTok{ielasisanas\_cels}\OtherTok{=}\FunctionTok{paste0}\NormalTok{(}\StringTok{"./RasterGrids\_100m/2024/RAW/"}\NormalTok{,nosaukums)}
\NormalTok{saglabasanas\_cels}\OtherTok{=}\FunctionTok{paste0}\NormalTok{(}\StringTok{"./RasterGrids\_100m/2024/Scaled/"}\NormalTok{,nosaukums)}
\NormalTok{slanis}\OtherTok{=}\FunctionTok{rast}\NormalTok{(ielasisanas\_cels)}
\NormalTok{videjais}\OtherTok{=}\FunctionTok{global}\NormalTok{(slanis,}\AttributeTok{fun=}\StringTok{"mean"}\NormalTok{,}\AttributeTok{na.rm=}\ConstantTok{TRUE}\NormalTok{)}
\NormalTok{centrets}\OtherTok{=}\NormalTok{slanis}\SpecialCharTok{{-}}\NormalTok{videjais[,}\DecValTok{1}\NormalTok{]}
\NormalTok{standartnovirze}\OtherTok{=}\NormalTok{terra}\SpecialCharTok{::}\FunctionTok{global}\NormalTok{(centrets,}\AttributeTok{fun=}\StringTok{"rms"}\NormalTok{,}\AttributeTok{na.rm=}\ConstantTok{TRUE}\NormalTok{)}
\NormalTok{merogots}\OtherTok{=}\NormalTok{centrets}\SpecialCharTok{/}\NormalTok{standartnovirze[,}\DecValTok{1}\NormalTok{]}
\FunctionTok{writeRaster}\NormalTok{(merogots,}
      \AttributeTok{filename=}\NormalTok{saglabasanas\_cels,}
      \AttributeTok{overwrite=}\ConstantTok{TRUE}\NormalTok{)}
\end{Highlighting}
\end{Shaded}

\section{SoilChemistry\_ESDAC-CaCo3\_cell}\label{ch06.501}

\textbf{filename:} \texttt{SoilChemistry\_ESDAC-CaCo3\_cell.tif}

\textbf{layername:} \texttt{egv\_501}

\textbf{English name:} Average value of Topsoil Calcium Carbonate Content (ESDAC
v2.0) within the analysis cell (1 ha)

\textbf{Latvian name:} Augsnes virskārtas kalcija karbonātu saturs (ESDAC v2.0)
analīzes šūnā (1 ha)

\textbf{Procedure:} Directly derived from the \hyperref[Ch04.07.01]{Soil chemistry}. Processed
using the workflow \texttt{egvtools::downscale2egv()} with \texttt{fill\ gaps\ =\ TRUE}, performing
inverse distance weighted (power = 2) filling of gaps at the border
and \texttt{smooth\ =\ FALSE}. This is done to preserve the original values as much as
possible (bilinear interpolation is involved when projecting from a 500 m
resolution to a 100 m resolution in a different CRS). Finally, the layer is
standardised by subtracting the arithmetic mean and dividing by the root mean
squared error.

\begin{Shaded}
\begin{Highlighting}[]
\CommentTok{\# libs {-}{-}{-}{-}}
\ControlFlowTok{if}\NormalTok{(}\SpecialCharTok{!}\FunctionTok{require}\NormalTok{(egvtools)) \{remotes}\SpecialCharTok{::}\FunctionTok{install\_github}\NormalTok{(}\StringTok{"aavotins/egvtools"}\NormalTok{); }\FunctionTok{require}\NormalTok{(egvtools)\}}


\CommentTok{\# CaCO3 {-}{-}{-}{-}}


\NormalTok{egv}\OtherTok{=}\FunctionTok{downscale2egv}\NormalTok{(}
 \AttributeTok{template\_path =} \StringTok{"./Templates/TemplateRasters/LV100m\_10km.tif"}\NormalTok{,}
 \AttributeTok{grid\_path   =} \StringTok{"./Templates/TemplateGrids/tikls1km\_sauzeme.parquet"}\NormalTok{,}
 \AttributeTok{rawfile\_path =} \StringTok{"./Geodata/2024/Soils/ESDAC/chemistry/chemistry/Caco3/CaCO3.tif"}\NormalTok{,}
 \AttributeTok{out\_path   =} \StringTok{"./RasterGrids\_100m/2024/RAW/"}\NormalTok{,}
 \AttributeTok{file\_name   =} \StringTok{"SoilChemistry\_ESDAC{-}CaCo3\_cell.tif"}\NormalTok{,}
 \AttributeTok{layer\_name  =} \StringTok{"egv\_501"}\NormalTok{,}
 \AttributeTok{fill\_gaps   =} \ConstantTok{TRUE}\NormalTok{,}
 \AttributeTok{smooth    =} \ConstantTok{FALSE}\NormalTok{,}
 \AttributeTok{plot\_result  =} \ConstantTok{TRUE}\NormalTok{)}
\NormalTok{egv}

\CommentTok{\# standardisation {-}{-}{-}{-}}
\ControlFlowTok{if}\NormalTok{(}\SpecialCharTok{!}\FunctionTok{require}\NormalTok{(terra)) \{}\FunctionTok{install.packages}\NormalTok{(}\StringTok{"terra"}\NormalTok{); }\FunctionTok{require}\NormalTok{(terra)\}}
\ControlFlowTok{if}\NormalTok{(}\SpecialCharTok{!}\FunctionTok{require}\NormalTok{(tidyverse)) \{}\FunctionTok{install.packages}\NormalTok{(}\StringTok{"tidyverse"}\NormalTok{); }\FunctionTok{require}\NormalTok{(tidyverse)\}}

\NormalTok{nosaukums}\OtherTok{=}\StringTok{"SoilChemistry\_ESDAC{-}CaCo3\_cell.tif"}
\NormalTok{ielasisanas\_cels}\OtherTok{=}\FunctionTok{paste0}\NormalTok{(}\StringTok{"./RasterGrids\_100m/2024/RAW/"}\NormalTok{,nosaukums)}
\NormalTok{saglabasanas\_cels}\OtherTok{=}\FunctionTok{paste0}\NormalTok{(}\StringTok{"./RasterGrids\_100m/2024/Scaled/"}\NormalTok{,nosaukums)}
\NormalTok{slanis}\OtherTok{=}\FunctionTok{rast}\NormalTok{(ielasisanas\_cels)}
\NormalTok{videjais}\OtherTok{=}\FunctionTok{global}\NormalTok{(slanis,}\AttributeTok{fun=}\StringTok{"mean"}\NormalTok{,}\AttributeTok{na.rm=}\ConstantTok{TRUE}\NormalTok{)}
\NormalTok{centrets}\OtherTok{=}\NormalTok{slanis}\SpecialCharTok{{-}}\NormalTok{videjais[,}\DecValTok{1}\NormalTok{]}
\NormalTok{standartnovirze}\OtherTok{=}\NormalTok{terra}\SpecialCharTok{::}\FunctionTok{global}\NormalTok{(centrets,}\AttributeTok{fun=}\StringTok{"rms"}\NormalTok{,}\AttributeTok{na.rm=}\ConstantTok{TRUE}\NormalTok{)}
\NormalTok{merogots}\OtherTok{=}\NormalTok{centrets}\SpecialCharTok{/}\NormalTok{standartnovirze[,}\DecValTok{1}\NormalTok{]}
\FunctionTok{writeRaster}\NormalTok{(merogots,}
      \AttributeTok{filename=}\NormalTok{saglabasanas\_cels,}
      \AttributeTok{overwrite=}\ConstantTok{TRUE}\NormalTok{)}
\end{Highlighting}
\end{Shaded}

\section{SoilChemistry\_ESDAC-K\_cell}\label{ch06.502}

\textbf{filename:} \texttt{SoilChemistry\_ESDAC-K\_cell.tif}

\textbf{layername:} \texttt{egv\_502}

\textbf{English name:} Average value of Topsoil Potassium Content (ESDAC v2.0) within
the analysis cell (1 ha)

\textbf{Latvian name:} Augsnes virskārtas kālija saturs (ESDAC v2.0) analīzes šūnā (1
ha)

\textbf{Procedure:} Directly derived from the \hyperref[Ch04.07.01]{Soil chemistry}. Processed
using the workflow \texttt{egvtools::downscale2egv()} with \texttt{fill\ gaps\ =\ TRUE}, performing
inverse distance weighted (power = 2) filling of gaps at the border
and \texttt{smooth\ =\ FALSE}. This is done to preserve the original values as much as
possible (bilinear interpolation is involved when projecting from a 500 m
resolution to a 100 m resolution in a different CRS). Finally, the layer is
standardised by subtracting the arithmetic mean and dividing by the root mean
squared error.

\begin{Shaded}
\begin{Highlighting}[]
\CommentTok{\# libs {-}{-}{-}{-}}
\ControlFlowTok{if}\NormalTok{(}\SpecialCharTok{!}\FunctionTok{require}\NormalTok{(egvtools)) \{remotes}\SpecialCharTok{::}\FunctionTok{install\_github}\NormalTok{(}\StringTok{"aavotins/egvtools"}\NormalTok{); }\FunctionTok{require}\NormalTok{(egvtools)\}}


\CommentTok{\# K {-}{-}{-}{-}}

\NormalTok{egv}\OtherTok{=}\FunctionTok{downscale2egv}\NormalTok{(}
 \AttributeTok{template\_path =} \StringTok{"./Templates/TemplateRasters/LV100m\_10km.tif"}\NormalTok{,}
 \AttributeTok{grid\_path   =} \StringTok{"./Templates/TemplateGrids/tikls1km\_sauzeme.parquet"}\NormalTok{,}
 \AttributeTok{rawfile\_path =} \StringTok{"./Geodata/2024/Soils/ESDAC/chemistry/chemistry/K/K.tif"}\NormalTok{,}
 \AttributeTok{out\_path   =} \StringTok{"./RasterGrids\_100m/2024/RAW/"}\NormalTok{,}
 \AttributeTok{file\_name   =} \StringTok{"SoilChemistry\_ESDAC{-}K\_cell.tif"}\NormalTok{,}
 \AttributeTok{layer\_name  =} \StringTok{"egv\_502"}\NormalTok{,}
 \AttributeTok{fill\_gaps   =} \ConstantTok{TRUE}\NormalTok{,}
 \AttributeTok{smooth    =} \ConstantTok{FALSE}\NormalTok{,}
 \AttributeTok{plot\_result  =} \ConstantTok{TRUE}\NormalTok{)}
\NormalTok{egv}

\CommentTok{\# standardisation {-}{-}{-}{-}}
\ControlFlowTok{if}\NormalTok{(}\SpecialCharTok{!}\FunctionTok{require}\NormalTok{(terra)) \{}\FunctionTok{install.packages}\NormalTok{(}\StringTok{"terra"}\NormalTok{); }\FunctionTok{require}\NormalTok{(terra)\}}
\ControlFlowTok{if}\NormalTok{(}\SpecialCharTok{!}\FunctionTok{require}\NormalTok{(tidyverse)) \{}\FunctionTok{install.packages}\NormalTok{(}\StringTok{"tidyverse"}\NormalTok{); }\FunctionTok{require}\NormalTok{(tidyverse)\}}

\NormalTok{nosaukums}\OtherTok{=}\StringTok{"SoilChemistry\_ESDAC{-}K\_cell.tif"}
\NormalTok{ielasisanas\_cels}\OtherTok{=}\FunctionTok{paste0}\NormalTok{(}\StringTok{"./RasterGrids\_100m/2024/RAW/"}\NormalTok{,nosaukums)}
\NormalTok{saglabasanas\_cels}\OtherTok{=}\FunctionTok{paste0}\NormalTok{(}\StringTok{"./RasterGrids\_100m/2024/Scaled/"}\NormalTok{,nosaukums)}
\NormalTok{slanis}\OtherTok{=}\FunctionTok{rast}\NormalTok{(ielasisanas\_cels)}
\NormalTok{videjais}\OtherTok{=}\FunctionTok{global}\NormalTok{(slanis,}\AttributeTok{fun=}\StringTok{"mean"}\NormalTok{,}\AttributeTok{na.rm=}\ConstantTok{TRUE}\NormalTok{)}
\NormalTok{centrets}\OtherTok{=}\NormalTok{slanis}\SpecialCharTok{{-}}\NormalTok{videjais[,}\DecValTok{1}\NormalTok{]}
\NormalTok{standartnovirze}\OtherTok{=}\NormalTok{terra}\SpecialCharTok{::}\FunctionTok{global}\NormalTok{(centrets,}\AttributeTok{fun=}\StringTok{"rms"}\NormalTok{,}\AttributeTok{na.rm=}\ConstantTok{TRUE}\NormalTok{)}
\NormalTok{merogots}\OtherTok{=}\NormalTok{centrets}\SpecialCharTok{/}\NormalTok{standartnovirze[,}\DecValTok{1}\NormalTok{]}
\FunctionTok{writeRaster}\NormalTok{(merogots,}
      \AttributeTok{filename=}\NormalTok{saglabasanas\_cels,}
      \AttributeTok{overwrite=}\ConstantTok{TRUE}\NormalTok{)}
\end{Highlighting}
\end{Shaded}

\section{SoilChemistry\_ESDAC-N\_cell}\label{ch06.503}

\textbf{filename:} \texttt{SoilChemistry\_ESDAC-N\_cell.tif}

\textbf{layername:} \texttt{egv\_503}

\textbf{English name:} Average value of Topsoil Nitrogen Content (ESDAC v2.0) within
the analysis cell (1 ha)

\textbf{Latvian name:} Augsnes virskārtas slāpekļa saturs (ESDAC v2.0) analīzes šūnā
(1 ha)

\textbf{Procedure:} Directly derived from the \hyperref[Ch04.07.01]{Soil chemistry}. Processed
using the workflow \texttt{egvtools::downscale2egv()} with \texttt{fill\ gaps\ =\ TRUE}, performing
inverse distance weighted (power = 2) filling of gaps at the border
and \texttt{smooth\ =\ FALSE}. This is done to preserve the original values as much as
possible (bilinear interpolation is involved when projecting from a 500 m
resolution to a 100 m resolution in a different CRS). Finally, the layer is
standardised by subtracting the arithmetic mean and dividing by the root mean
squared error.

\begin{Shaded}
\begin{Highlighting}[]
\CommentTok{\# libs {-}{-}{-}{-}}
\ControlFlowTok{if}\NormalTok{(}\SpecialCharTok{!}\FunctionTok{require}\NormalTok{(egvtools)) \{remotes}\SpecialCharTok{::}\FunctionTok{install\_github}\NormalTok{(}\StringTok{"aavotins/egvtools"}\NormalTok{); }\FunctionTok{require}\NormalTok{(egvtools)\}}


\CommentTok{\# N {-}{-}{-}{-}}

\NormalTok{egv}\OtherTok{=}\FunctionTok{downscale2egv}\NormalTok{(}
 \AttributeTok{template\_path =} \StringTok{"./Templates/TemplateRasters/LV100m\_10km.tif"}\NormalTok{,}
 \AttributeTok{grid\_path   =} \StringTok{"./Templates/TemplateGrids/tikls1km\_sauzeme.parquet"}\NormalTok{,}
 \AttributeTok{rawfile\_path =} \StringTok{"./Geodata/2024/Soils/ESDAC/chemistry/chemistry/N/N.tif"}\NormalTok{,}
 \AttributeTok{out\_path   =} \StringTok{"./RasterGrids\_100m/2024/RAW/"}\NormalTok{,}
 \AttributeTok{file\_name   =} \StringTok{"SoilChemistry\_ESDAC{-}N\_cell.tif"}\NormalTok{,}
 \AttributeTok{layer\_name  =} \StringTok{"egv\_503"}\NormalTok{,}
 \AttributeTok{fill\_gaps   =} \ConstantTok{TRUE}\NormalTok{,}
 \AttributeTok{smooth    =} \ConstantTok{FALSE}\NormalTok{,}
 \AttributeTok{plot\_result  =} \ConstantTok{TRUE}\NormalTok{)}
\NormalTok{egv}

\CommentTok{\# standardisation {-}{-}{-}{-}}
\ControlFlowTok{if}\NormalTok{(}\SpecialCharTok{!}\FunctionTok{require}\NormalTok{(terra)) \{}\FunctionTok{install.packages}\NormalTok{(}\StringTok{"terra"}\NormalTok{); }\FunctionTok{require}\NormalTok{(terra)\}}
\ControlFlowTok{if}\NormalTok{(}\SpecialCharTok{!}\FunctionTok{require}\NormalTok{(tidyverse)) \{}\FunctionTok{install.packages}\NormalTok{(}\StringTok{"tidyverse"}\NormalTok{); }\FunctionTok{require}\NormalTok{(tidyverse)\}}

\NormalTok{nosaukums}\OtherTok{=}\StringTok{"SoilChemistry\_ESDAC{-}N\_cell.tif"}
\NormalTok{ielasisanas\_cels}\OtherTok{=}\FunctionTok{paste0}\NormalTok{(}\StringTok{"./RasterGrids\_100m/2024/RAW/"}\NormalTok{,nosaukums)}
\NormalTok{saglabasanas\_cels}\OtherTok{=}\FunctionTok{paste0}\NormalTok{(}\StringTok{"./RasterGrids\_100m/2024/Scaled/"}\NormalTok{,nosaukums)}
\NormalTok{slanis}\OtherTok{=}\FunctionTok{rast}\NormalTok{(ielasisanas\_cels)}
\NormalTok{videjais}\OtherTok{=}\FunctionTok{global}\NormalTok{(slanis,}\AttributeTok{fun=}\StringTok{"mean"}\NormalTok{,}\AttributeTok{na.rm=}\ConstantTok{TRUE}\NormalTok{)}
\NormalTok{centrets}\OtherTok{=}\NormalTok{slanis}\SpecialCharTok{{-}}\NormalTok{videjais[,}\DecValTok{1}\NormalTok{]}
\NormalTok{standartnovirze}\OtherTok{=}\NormalTok{terra}\SpecialCharTok{::}\FunctionTok{global}\NormalTok{(centrets,}\AttributeTok{fun=}\StringTok{"rms"}\NormalTok{,}\AttributeTok{na.rm=}\ConstantTok{TRUE}\NormalTok{)}
\NormalTok{merogots}\OtherTok{=}\NormalTok{centrets}\SpecialCharTok{/}\NormalTok{standartnovirze[,}\DecValTok{1}\NormalTok{]}
\FunctionTok{writeRaster}\NormalTok{(merogots,}
      \AttributeTok{filename=}\NormalTok{saglabasanas\_cels,}
      \AttributeTok{overwrite=}\ConstantTok{TRUE}\NormalTok{)}
\end{Highlighting}
\end{Shaded}

\section{SoilChemistry\_ESDAC-P\_cell}\label{ch06.504}

\textbf{filename:} \texttt{SoilChemistry\_ESDAC-P\_cell.tif}

\textbf{layername:} \texttt{egv\_504}

\textbf{English name:} Average value of Topsoil Phosphorous Content (ESDAC v2.0)
within the analysis cell (1 ha)

\textbf{Latvian name:} Augsnes virskārtas fosfora saturs (ESDAC v2.0) analīzes šūnā
(1 ha)

\textbf{Procedure:} Directly derived from the \hyperref[Ch04.07.01]{Soil chemistry}. Processed
using the workflow \texttt{egvtools::downscale2egv()} with \texttt{fill\ gaps\ =\ TRUE}, performing
inverse distance weighted (power = 2) filling of gaps at the border
and \texttt{smooth\ =\ FALSE}. This is done to preserve the original values as much as
possible (bilinear interpolation is involved when projecting from a 500 m
resolution to a 100 m resolution in a different CRS). Finally, the layer is
standardised by subtracting the arithmetic mean and dividing by the root mean
squared error.

\begin{Shaded}
\begin{Highlighting}[]
\CommentTok{\# libs {-}{-}{-}{-}}
\ControlFlowTok{if}\NormalTok{(}\SpecialCharTok{!}\FunctionTok{require}\NormalTok{(egvtools)) \{remotes}\SpecialCharTok{::}\FunctionTok{install\_github}\NormalTok{(}\StringTok{"aavotins/egvtools"}\NormalTok{); }\FunctionTok{require}\NormalTok{(egvtools)\}}

\CommentTok{\# P {-}{-}{-}{-}}

\NormalTok{egv}\OtherTok{=}\FunctionTok{downscale2egv}\NormalTok{(}
 \AttributeTok{template\_path =} \StringTok{"./Templates/TemplateRasters/LV100m\_10km.tif"}\NormalTok{,}
 \AttributeTok{grid\_path   =} \StringTok{"./Templates/TemplateGrids/tikls1km\_sauzeme.parquet"}\NormalTok{,}
 \AttributeTok{rawfile\_path =} \StringTok{"./Geodata/2024/Soils/ESDAC/chemistry/chemistry/P/P.tif"}\NormalTok{,}
 \AttributeTok{out\_path   =} \StringTok{"./RasterGrids\_100m/2024/RAW/"}\NormalTok{,}
 \AttributeTok{file\_name   =} \StringTok{"SoilChemistry\_ESDAC{-}P\_cell.tif"}\NormalTok{,}
 \AttributeTok{layer\_name  =} \StringTok{"egv\_504"}\NormalTok{,}
 \AttributeTok{fill\_gaps   =} \ConstantTok{TRUE}\NormalTok{,}
 \AttributeTok{smooth    =} \ConstantTok{FALSE}\NormalTok{,}
 \AttributeTok{plot\_result  =} \ConstantTok{TRUE}\NormalTok{)}
\NormalTok{egv}

\CommentTok{\# standardisation {-}{-}{-}{-}}
\ControlFlowTok{if}\NormalTok{(}\SpecialCharTok{!}\FunctionTok{require}\NormalTok{(terra)) \{}\FunctionTok{install.packages}\NormalTok{(}\StringTok{"terra"}\NormalTok{); }\FunctionTok{require}\NormalTok{(terra)\}}
\ControlFlowTok{if}\NormalTok{(}\SpecialCharTok{!}\FunctionTok{require}\NormalTok{(tidyverse)) \{}\FunctionTok{install.packages}\NormalTok{(}\StringTok{"tidyverse"}\NormalTok{); }\FunctionTok{require}\NormalTok{(tidyverse)\}}

\NormalTok{nosaukums}\OtherTok{=}\StringTok{"SoilChemistry\_ESDAC{-}P\_cell.tif"}
\NormalTok{ielasisanas\_cels}\OtherTok{=}\FunctionTok{paste0}\NormalTok{(}\StringTok{"./RasterGrids\_100m/2024/RAW/"}\NormalTok{,nosaukums)}
\NormalTok{saglabasanas\_cels}\OtherTok{=}\FunctionTok{paste0}\NormalTok{(}\StringTok{"./RasterGrids\_100m/2024/Scaled/"}\NormalTok{,nosaukums)}
\NormalTok{slanis}\OtherTok{=}\FunctionTok{rast}\NormalTok{(ielasisanas\_cels)}
\NormalTok{videjais}\OtherTok{=}\FunctionTok{global}\NormalTok{(slanis,}\AttributeTok{fun=}\StringTok{"mean"}\NormalTok{,}\AttributeTok{na.rm=}\ConstantTok{TRUE}\NormalTok{)}
\NormalTok{centrets}\OtherTok{=}\NormalTok{slanis}\SpecialCharTok{{-}}\NormalTok{videjais[,}\DecValTok{1}\NormalTok{]}
\NormalTok{standartnovirze}\OtherTok{=}\NormalTok{terra}\SpecialCharTok{::}\FunctionTok{global}\NormalTok{(centrets,}\AttributeTok{fun=}\StringTok{"rms"}\NormalTok{,}\AttributeTok{na.rm=}\ConstantTok{TRUE}\NormalTok{)}
\NormalTok{merogots}\OtherTok{=}\NormalTok{centrets}\SpecialCharTok{/}\NormalTok{standartnovirze[,}\DecValTok{1}\NormalTok{]}
\FunctionTok{writeRaster}\NormalTok{(merogots,}
      \AttributeTok{filename=}\NormalTok{saglabasanas\_cels,}
      \AttributeTok{overwrite=}\ConstantTok{TRUE}\NormalTok{)}
\end{Highlighting}
\end{Shaded}

\section{SoilChemistry\_ESDAC-phH2O\_cell}\label{ch06.505}

\textbf{filename:} \texttt{SoilChemistry\_ESDAC-phH2O\_cell.tif}

\textbf{layername:} \texttt{egv\_505}

\textbf{English name:} Average value of Topsoil pH reaction in water (ESDAC v2.0)
within the analysis cell (1 ha)

\textbf{Latvian name:} Augsnes virskārtas reakcija (pH) ūdens šķīdumā (ESDAC v2.0)
analīzes šūnā (1 ha)

\textbf{Procedure:} Directly derived from the \hyperref[Ch04.07.01]{Soil chemistry}. Processed
using the workflow \texttt{egvtools::downscale2egv()} with \texttt{fill\ gaps\ =\ TRUE}, performing
inverse distance weighted (power = 2) filling of gaps at the border
and \texttt{smooth\ =\ FALSE}. This is done to preserve the original values as much as
possible (bilinear interpolation is involved when projecting from a 500 m
resolution to a 100 m resolution in a different CRS). Finally, the layer is
standardised by subtracting the arithmetic mean and dividing by the root mean
squared error.

\begin{Shaded}
\begin{Highlighting}[]
\CommentTok{\# libs {-}{-}{-}{-}}
\ControlFlowTok{if}\NormalTok{(}\SpecialCharTok{!}\FunctionTok{require}\NormalTok{(egvtools)) \{remotes}\SpecialCharTok{::}\FunctionTok{install\_github}\NormalTok{(}\StringTok{"aavotins/egvtools"}\NormalTok{); }\FunctionTok{require}\NormalTok{(egvtools)\}}


\CommentTok{\# pH\_H2O {-}{-}{-}{-}}

\NormalTok{egv}\OtherTok{=}\FunctionTok{downscale2egv}\NormalTok{(}
 \AttributeTok{template\_path =} \StringTok{"./Templates/TemplateRasters/LV100m\_10km.tif"}\NormalTok{,}
 \AttributeTok{grid\_path   =} \StringTok{"./Templates/TemplateGrids/tikls1km\_sauzeme.parquet"}\NormalTok{,}
 \AttributeTok{rawfile\_path =} \StringTok{"./Geodata/2024/Soils/ESDAC/chemistry/chemistry/pH\_H2O/pH\_H2O.tif"}\NormalTok{,}
 \AttributeTok{out\_path   =} \StringTok{"./RasterGrids\_100m/2024/RAW/"}\NormalTok{,}
 \AttributeTok{file\_name   =} \StringTok{"SoilChemistry\_ESDAC{-}phH2O\_cell.tif"}\NormalTok{,}
 \AttributeTok{layer\_name  =} \StringTok{"egv\_505"}\NormalTok{,}
 \AttributeTok{fill\_gaps   =} \ConstantTok{TRUE}\NormalTok{,}
 \AttributeTok{smooth    =} \ConstantTok{FALSE}\NormalTok{,}
 \AttributeTok{plot\_result  =} \ConstantTok{TRUE}\NormalTok{)}
\NormalTok{egv}

\CommentTok{\# standardisation {-}{-}{-}{-}}
\ControlFlowTok{if}\NormalTok{(}\SpecialCharTok{!}\FunctionTok{require}\NormalTok{(terra)) \{}\FunctionTok{install.packages}\NormalTok{(}\StringTok{"terra"}\NormalTok{); }\FunctionTok{require}\NormalTok{(terra)\}}
\ControlFlowTok{if}\NormalTok{(}\SpecialCharTok{!}\FunctionTok{require}\NormalTok{(tidyverse)) \{}\FunctionTok{install.packages}\NormalTok{(}\StringTok{"tidyverse"}\NormalTok{); }\FunctionTok{require}\NormalTok{(tidyverse)\}}

\NormalTok{nosaukums}\OtherTok{=}\StringTok{"SoilChemistry\_ESDAC{-}phH2O\_cell.tif"}
\NormalTok{ielasisanas\_cels}\OtherTok{=}\FunctionTok{paste0}\NormalTok{(}\StringTok{"./RasterGrids\_100m/2024/RAW/"}\NormalTok{,nosaukums)}
\NormalTok{saglabasanas\_cels}\OtherTok{=}\FunctionTok{paste0}\NormalTok{(}\StringTok{"./RasterGrids\_100m/2024/Scaled/"}\NormalTok{,nosaukums)}
\NormalTok{slanis}\OtherTok{=}\FunctionTok{rast}\NormalTok{(ielasisanas\_cels)}
\NormalTok{videjais}\OtherTok{=}\FunctionTok{global}\NormalTok{(slanis,}\AttributeTok{fun=}\StringTok{"mean"}\NormalTok{,}\AttributeTok{na.rm=}\ConstantTok{TRUE}\NormalTok{)}
\NormalTok{centrets}\OtherTok{=}\NormalTok{slanis}\SpecialCharTok{{-}}\NormalTok{videjais[,}\DecValTok{1}\NormalTok{]}
\NormalTok{standartnovirze}\OtherTok{=}\NormalTok{terra}\SpecialCharTok{::}\FunctionTok{global}\NormalTok{(centrets,}\AttributeTok{fun=}\StringTok{"rms"}\NormalTok{,}\AttributeTok{na.rm=}\ConstantTok{TRUE}\NormalTok{)}
\NormalTok{merogots}\OtherTok{=}\NormalTok{centrets}\SpecialCharTok{/}\NormalTok{standartnovirze[,}\DecValTok{1}\NormalTok{]}
\FunctionTok{writeRaster}\NormalTok{(merogots,}
      \AttributeTok{filename=}\NormalTok{saglabasanas\_cels,}
      \AttributeTok{overwrite=}\ConstantTok{TRUE}\NormalTok{)}
\end{Highlighting}
\end{Shaded}

\section{SoilTexture\_Clay\_cell}\label{ch06.506}

\textbf{filename:} \texttt{SoilTexture\_Clay\_cell.tif}

\textbf{layername:} \texttt{egv\_506}

\textbf{English name:} Fractional cover of Clay Soils within the analysis cell (1 ha)

\textbf{Latvian name:} Augsnes granulometriskās klases ``māls'' platības īpatsvars
analīzes šūnā (1 ha)

\textbf{Procedure:} Derived from the \hyperref[Ch05.02]{Soil texture product}. First, the layer is
reclassified so that the class of interest is 1 and the other classes are 0. The resulting layer
is then aggregated to EGV resolution using the workflow \texttt{egvtools::input2egv()}, which
calculates the arithmetic mean to determine the cover fraction. During
aggregation, inverse distance weighted (power = 2) gap filling on the output is
applied to ensure no missing values at the edges. Finally, the layer is
standardised by subtracting the arithmetic mean and dividing by the root mean squared
error.

\begin{Shaded}
\begin{Highlighting}[]
\CommentTok{\# libs {-}{-}{-}{-}}
\ControlFlowTok{if}\NormalTok{(}\SpecialCharTok{!}\FunctionTok{require}\NormalTok{(terra)) \{}\FunctionTok{install.packages}\NormalTok{(}\StringTok{"terra"}\NormalTok{); }\FunctionTok{require}\NormalTok{(terra)\}}
\ControlFlowTok{if}\NormalTok{(}\SpecialCharTok{!}\FunctionTok{require}\NormalTok{(egvtools)) \{remotes}\SpecialCharTok{::}\FunctionTok{install\_github}\NormalTok{(}\StringTok{"aavotins/egvtools"}\NormalTok{); }\FunctionTok{require}\NormalTok{(egvtools)\}}

\CommentTok{\# templates {-}{-}{-}{-}}
\NormalTok{template10}\OtherTok{=}\FunctionTok{rast}\NormalTok{(}\StringTok{"./Templates/TemplateRasters/LV10m\_10km.tif"}\NormalTok{)}
\NormalTok{template100}\OtherTok{=}\FunctionTok{rast}\NormalTok{(}\StringTok{"./Templates/TemplateRasters/LV100m\_10km.tif"}\NormalTok{)}

\CommentTok{\# input {-}{-}{-}{-}}
\NormalTok{combtext}\OtherTok{=}\FunctionTok{rast}\NormalTok{(}\StringTok{"./RasterGrids\_10m/2024/SoilTXT\_combined.tif"}\NormalTok{)}

\CommentTok{\# EGVs cell {-}{-}{-}{-}}

\CommentTok{\# SoilTexture\_Clay\_cell.tif egv\_506}

\NormalTok{clay10}\OtherTok{=}\FunctionTok{ifel}\NormalTok{(combtext}\SpecialCharTok{==}\DecValTok{3}\NormalTok{,}\DecValTok{1}\NormalTok{,}\DecValTok{0}\NormalTok{)}

\FunctionTok{input2egv}\NormalTok{(}\AttributeTok{input=}\NormalTok{clay10,}
     \AttributeTok{egv\_template=}\StringTok{"./Templates/TemplateRasters/LV100m\_10km.tif"}\NormalTok{,}
     \AttributeTok{summary\_function =} \StringTok{"average"}\NormalTok{,}
     \AttributeTok{missing\_job =} \StringTok{"FillOutput"}\NormalTok{,}
     \AttributeTok{idw\_weight =} \DecValTok{2}\NormalTok{,}
     \AttributeTok{outlocation =} \StringTok{"./RasterGrids\_100m/2024/RAW/"}\NormalTok{,}
     \AttributeTok{outfilename =} \StringTok{"SoilTexture\_Clay\_cell.tif"}\NormalTok{,}
     \AttributeTok{layername=}\StringTok{"egv\_506"}\NormalTok{,}
     \AttributeTok{return\_visible =} \ConstantTok{TRUE}\NormalTok{)}

\CommentTok{\# standardisation {-}{-}{-}{-}}
\ControlFlowTok{if}\NormalTok{(}\SpecialCharTok{!}\FunctionTok{require}\NormalTok{(terra)) \{}\FunctionTok{install.packages}\NormalTok{(}\StringTok{"terra"}\NormalTok{); }\FunctionTok{require}\NormalTok{(terra)\}}
\ControlFlowTok{if}\NormalTok{(}\SpecialCharTok{!}\FunctionTok{require}\NormalTok{(tidyverse)) \{}\FunctionTok{install.packages}\NormalTok{(}\StringTok{"tidyverse"}\NormalTok{); }\FunctionTok{require}\NormalTok{(tidyverse)\}}

\NormalTok{nosaukums}\OtherTok{=}\StringTok{"SoilTexture\_Clay\_cell.tif"}
\NormalTok{ielasisanas\_cels}\OtherTok{=}\FunctionTok{paste0}\NormalTok{(}\StringTok{"./RasterGrids\_100m/2024/RAW/"}\NormalTok{,nosaukums)}
\NormalTok{saglabasanas\_cels}\OtherTok{=}\FunctionTok{paste0}\NormalTok{(}\StringTok{"./RasterGrids\_100m/2024/Scaled/"}\NormalTok{,nosaukums)}
\NormalTok{slanis}\OtherTok{=}\FunctionTok{rast}\NormalTok{(ielasisanas\_cels)}
\NormalTok{videjais}\OtherTok{=}\FunctionTok{global}\NormalTok{(slanis,}\AttributeTok{fun=}\StringTok{"mean"}\NormalTok{,}\AttributeTok{na.rm=}\ConstantTok{TRUE}\NormalTok{)}
\NormalTok{centrets}\OtherTok{=}\NormalTok{slanis}\SpecialCharTok{{-}}\NormalTok{videjais[,}\DecValTok{1}\NormalTok{]}
\NormalTok{standartnovirze}\OtherTok{=}\NormalTok{terra}\SpecialCharTok{::}\FunctionTok{global}\NormalTok{(centrets,}\AttributeTok{fun=}\StringTok{"rms"}\NormalTok{,}\AttributeTok{na.rm=}\ConstantTok{TRUE}\NormalTok{)}
\NormalTok{merogots}\OtherTok{=}\NormalTok{centrets}\SpecialCharTok{/}\NormalTok{standartnovirze[,}\DecValTok{1}\NormalTok{]}
\FunctionTok{writeRaster}\NormalTok{(merogots,}
      \AttributeTok{filename=}\NormalTok{saglabasanas\_cels,}
      \AttributeTok{overwrite=}\ConstantTok{TRUE}\NormalTok{)}
\end{Highlighting}
\end{Shaded}

\section{SoilTexture\_Clay\_r500}\label{ch06.507}

\textbf{filename:} \texttt{SoilTexture\_Clay\_r500.tif}

\textbf{layername:} \texttt{egv\_507}

\textbf{English name:} Fractional cover of Clay Soils within the 0.5 km landscape

\textbf{Latvian name:} Augsnes granulometriskās klases ``māls'' platības īpatsvars 0,5
km ainavā

\textbf{Procedure:} The cover fraction within a radius of 500 m around the analysis grid cell is
calculated as the area-weighted sum of the \hyperref[ch06.506]{analysis cells} inside the
buffer, using the workflow \texttt{egvtools::radius\_function()}. During the calculation of the landscape metric,
inverse distance weighted (power = 2) gap filling on the output is applied
to ensure no missing values at the edges. Then the layer is rewritten to set
its name. Finally, the layer is standardised by subtracting the arithmetic
mean and dividing by the root mean squared error.

\begin{Shaded}
\begin{Highlighting}[]
\CommentTok{\# libs {-}{-}{-}{-}}
\ControlFlowTok{if}\NormalTok{(}\SpecialCharTok{!}\FunctionTok{require}\NormalTok{(terra)) \{}\FunctionTok{install.packages}\NormalTok{(}\StringTok{"terra"}\NormalTok{); }\FunctionTok{require}\NormalTok{(terra)\}}
\ControlFlowTok{if}\NormalTok{(}\SpecialCharTok{!}\FunctionTok{require}\NormalTok{(egvtools)) \{remotes}\SpecialCharTok{::}\FunctionTok{install\_github}\NormalTok{(}\StringTok{"aavotins/egvtools"}\NormalTok{); }\FunctionTok{require}\NormalTok{(egvtools)\}}

\CommentTok{\# EGVs radii {-}{-}{-}{-}}

\FunctionTok{radius\_function}\NormalTok{(}
 \AttributeTok{kvadrati\_path =} \StringTok{"./Templates/TemplateGrids/tiles/"}\NormalTok{,}
 \AttributeTok{radii\_path   =} \StringTok{"./Templates/TemplateGridPoints/tiles/"}\NormalTok{,}
 \AttributeTok{tikls100\_path =} \StringTok{"./Templates/TemplateGrids/tikls100\_sauzeme.parquet"}\NormalTok{,}
 \AttributeTok{template\_path =} \StringTok{"./Templates/TemplateRasters/LV100m\_10km.tif"}\NormalTok{,}
 \AttributeTok{input\_layers  =} \FunctionTok{c}\NormalTok{(}\StringTok{"./RasterGrids\_100m/2024/RAW/SoilTexture\_Clay\_cell.tif"}\NormalTok{),}
 \AttributeTok{layer\_prefixes =} \FunctionTok{c}\NormalTok{(}\StringTok{"SoilTexture\_Clay"}\NormalTok{),}
 \AttributeTok{output\_dir   =} \StringTok{"./RasterGrids\_100m/2024/RAW/"}\NormalTok{,}
 \AttributeTok{n\_workers   =} \DecValTok{5}\NormalTok{,}
 \AttributeTok{radii     =} \FunctionTok{c}\NormalTok{(}\StringTok{"r500"}\NormalTok{),}
 \AttributeTok{radius\_mode  =} \StringTok{"sparse"}\NormalTok{,}
 \AttributeTok{extract\_fun  =} \StringTok{"mean"}\NormalTok{,}
 \AttributeTok{fill\_missing  =} \ConstantTok{TRUE}\NormalTok{,}
 \AttributeTok{IDW\_weight   =} \DecValTok{2}\NormalTok{,}
 \AttributeTok{future\_max\_size =} \DecValTok{5} \SpecialCharTok{*} \DecValTok{1024}\SpecialCharTok{\^{}}\DecValTok{3}\NormalTok{)}

\CommentTok{\# SoilTexture\_Clay\_r500.tif egv\_507}

\NormalTok{slanis}\OtherTok{=}\FunctionTok{rast}\NormalTok{(}\StringTok{"./RasterGrids\_100m/2024/RAW/SoilTexture\_Clay\_r500.tif"}\NormalTok{)}
\FunctionTok{names}\NormalTok{(slanis)}\OtherTok{=}\StringTok{"egv\_507"}
\NormalTok{slanis2}\OtherTok{=}\FunctionTok{project}\NormalTok{(slanis,template100)}
\FunctionTok{writeRaster}\NormalTok{(slanis2,}
      \StringTok{"./RasterGrids\_100m/2024/RAW/SoilTexture\_Clay\_r500.tif"}\NormalTok{,}
      \AttributeTok{overwrite=}\ConstantTok{TRUE}\NormalTok{)}

\CommentTok{\# standardisation {-}{-}{-}{-}}
\ControlFlowTok{if}\NormalTok{(}\SpecialCharTok{!}\FunctionTok{require}\NormalTok{(terra)) \{}\FunctionTok{install.packages}\NormalTok{(}\StringTok{"terra"}\NormalTok{); }\FunctionTok{require}\NormalTok{(terra)\}}
\ControlFlowTok{if}\NormalTok{(}\SpecialCharTok{!}\FunctionTok{require}\NormalTok{(tidyverse)) \{}\FunctionTok{install.packages}\NormalTok{(}\StringTok{"tidyverse"}\NormalTok{); }\FunctionTok{require}\NormalTok{(tidyverse)\}}

\NormalTok{nosaukums}\OtherTok{=}\StringTok{"SoilTexture\_Clay\_r500.tif"}
\NormalTok{ielasisanas\_cels}\OtherTok{=}\FunctionTok{paste0}\NormalTok{(}\StringTok{"./RasterGrids\_100m/2024/RAW/"}\NormalTok{,nosaukums)}
\NormalTok{saglabasanas\_cels}\OtherTok{=}\FunctionTok{paste0}\NormalTok{(}\StringTok{"./RasterGrids\_100m/2024/Scaled/"}\NormalTok{,nosaukums)}
\NormalTok{slanis}\OtherTok{=}\FunctionTok{rast}\NormalTok{(ielasisanas\_cels)}
\NormalTok{videjais}\OtherTok{=}\FunctionTok{global}\NormalTok{(slanis,}\AttributeTok{fun=}\StringTok{"mean"}\NormalTok{,}\AttributeTok{na.rm=}\ConstantTok{TRUE}\NormalTok{)}
\NormalTok{centrets}\OtherTok{=}\NormalTok{slanis}\SpecialCharTok{{-}}\NormalTok{videjais[,}\DecValTok{1}\NormalTok{]}
\NormalTok{standartnovirze}\OtherTok{=}\NormalTok{terra}\SpecialCharTok{::}\FunctionTok{global}\NormalTok{(centrets,}\AttributeTok{fun=}\StringTok{"rms"}\NormalTok{,}\AttributeTok{na.rm=}\ConstantTok{TRUE}\NormalTok{)}
\NormalTok{merogots}\OtherTok{=}\NormalTok{centrets}\SpecialCharTok{/}\NormalTok{standartnovirze[,}\DecValTok{1}\NormalTok{]}
\FunctionTok{writeRaster}\NormalTok{(merogots,}
      \AttributeTok{filename=}\NormalTok{saglabasanas\_cels,}
      \AttributeTok{overwrite=}\ConstantTok{TRUE}\NormalTok{)}
\end{Highlighting}
\end{Shaded}

\section{SoilTexture\_Clay\_r1250}\label{ch06.508}

\textbf{filename:} \texttt{SoilTexture\_Clay\_r1250.tif}

\textbf{layername:} \texttt{egv\_508}

\textbf{English name:} Fractional cover of Clay Soils within the 1.25 km landscape

\textbf{Latvian name:} Augsnes granulometriskās klases ``māls'' platības īpatsvars 1,25
km ainavā

\textbf{Procedure:} The cover fraction within a radius of 1250 m around the analysis grid cell is
calculated as the area-weighted sum of the \hyperref[ch06.506]{analysis cells} inside the
buffer, using the workflow \texttt{egvtools::radius\_function()}. During the calculation of the landscape metric,
inverse distance weighted (power = 2) gap filling on the output is applied
to ensure no missing values at the edges. Then the layer is rewritten to set
its name. Finally, the layer is standardised by subtracting the arithmetic
mean and dividing by the root mean squared error.

\begin{Shaded}
\begin{Highlighting}[]
\CommentTok{\# libs {-}{-}{-}{-}}
\ControlFlowTok{if}\NormalTok{(}\SpecialCharTok{!}\FunctionTok{require}\NormalTok{(terra)) \{}\FunctionTok{install.packages}\NormalTok{(}\StringTok{"terra"}\NormalTok{); }\FunctionTok{require}\NormalTok{(terra)\}}
\ControlFlowTok{if}\NormalTok{(}\SpecialCharTok{!}\FunctionTok{require}\NormalTok{(egvtools)) \{remotes}\SpecialCharTok{::}\FunctionTok{install\_github}\NormalTok{(}\StringTok{"aavotins/egvtools"}\NormalTok{); }\FunctionTok{require}\NormalTok{(egvtools)\}}

\CommentTok{\# EGVs radii {-}{-}{-}{-}}

\FunctionTok{radius\_function}\NormalTok{(}
 \AttributeTok{kvadrati\_path =} \StringTok{"./Templates/TemplateGrids/tiles/"}\NormalTok{,}
 \AttributeTok{radii\_path   =} \StringTok{"./Templates/TemplateGridPoints/tiles/"}\NormalTok{,}
 \AttributeTok{tikls100\_path =} \StringTok{"./Templates/TemplateGrids/tikls100\_sauzeme.parquet"}\NormalTok{,}
 \AttributeTok{template\_path =} \StringTok{"./Templates/TemplateRasters/LV100m\_10km.tif"}\NormalTok{,}
 \AttributeTok{input\_layers  =} \FunctionTok{c}\NormalTok{(}\StringTok{"./RasterGrids\_100m/2024/RAW/SoilTexture\_Clay\_cell.tif"}\NormalTok{),}
 \AttributeTok{layer\_prefixes =} \FunctionTok{c}\NormalTok{(}\StringTok{"SoilTexture\_Clay"}\NormalTok{),}
 \AttributeTok{output\_dir   =} \StringTok{"./RasterGrids\_100m/2024/RAW/"}\NormalTok{,}
 \AttributeTok{n\_workers   =} \DecValTok{5}\NormalTok{,}
 \AttributeTok{radii     =} \FunctionTok{c}\NormalTok{(}\StringTok{"r1250"}\NormalTok{),}
 \AttributeTok{radius\_mode  =} \StringTok{"sparse"}\NormalTok{,}
 \AttributeTok{extract\_fun  =} \StringTok{"mean"}\NormalTok{,}
 \AttributeTok{fill\_missing  =} \ConstantTok{TRUE}\NormalTok{,}
 \AttributeTok{IDW\_weight   =} \DecValTok{2}\NormalTok{,}
 \AttributeTok{future\_max\_size =} \DecValTok{5} \SpecialCharTok{*} \DecValTok{1024}\SpecialCharTok{\^{}}\DecValTok{3}\NormalTok{)}


\CommentTok{\# SoilTexture\_Clay\_r1250.tif    egv\_508}

\NormalTok{slanis}\OtherTok{=}\FunctionTok{rast}\NormalTok{(}\StringTok{"./RasterGrids\_100m/2024/RAW/SoilTexture\_Clay\_r1250.tif"}\NormalTok{)}
\FunctionTok{names}\NormalTok{(slanis)}\OtherTok{=}\StringTok{"egv\_508"}
\NormalTok{slanis2}\OtherTok{=}\FunctionTok{project}\NormalTok{(slanis,template100)}
\FunctionTok{writeRaster}\NormalTok{(slanis2,}
      \StringTok{"./RasterGrids\_100m/2024/RAW/SoilTexture\_Clay\_r1250.tif"}\NormalTok{,}
      \AttributeTok{overwrite=}\ConstantTok{TRUE}\NormalTok{)}

\CommentTok{\# standardisation {-}{-}{-}{-}}
\ControlFlowTok{if}\NormalTok{(}\SpecialCharTok{!}\FunctionTok{require}\NormalTok{(terra)) \{}\FunctionTok{install.packages}\NormalTok{(}\StringTok{"terra"}\NormalTok{); }\FunctionTok{require}\NormalTok{(terra)\}}
\ControlFlowTok{if}\NormalTok{(}\SpecialCharTok{!}\FunctionTok{require}\NormalTok{(tidyverse)) \{}\FunctionTok{install.packages}\NormalTok{(}\StringTok{"tidyverse"}\NormalTok{); }\FunctionTok{require}\NormalTok{(tidyverse)\}}

\NormalTok{nosaukums}\OtherTok{=}\StringTok{"SoilTexture\_Clay\_r1250.tif"}
\NormalTok{ielasisanas\_cels}\OtherTok{=}\FunctionTok{paste0}\NormalTok{(}\StringTok{"./RasterGrids\_100m/2024/RAW/"}\NormalTok{,nosaukums)}
\NormalTok{saglabasanas\_cels}\OtherTok{=}\FunctionTok{paste0}\NormalTok{(}\StringTok{"./RasterGrids\_100m/2024/Scaled/"}\NormalTok{,nosaukums)}
\NormalTok{slanis}\OtherTok{=}\FunctionTok{rast}\NormalTok{(ielasisanas\_cels)}
\NormalTok{videjais}\OtherTok{=}\FunctionTok{global}\NormalTok{(slanis,}\AttributeTok{fun=}\StringTok{"mean"}\NormalTok{,}\AttributeTok{na.rm=}\ConstantTok{TRUE}\NormalTok{)}
\NormalTok{centrets}\OtherTok{=}\NormalTok{slanis}\SpecialCharTok{{-}}\NormalTok{videjais[,}\DecValTok{1}\NormalTok{]}
\NormalTok{standartnovirze}\OtherTok{=}\NormalTok{terra}\SpecialCharTok{::}\FunctionTok{global}\NormalTok{(centrets,}\AttributeTok{fun=}\StringTok{"rms"}\NormalTok{,}\AttributeTok{na.rm=}\ConstantTok{TRUE}\NormalTok{)}
\NormalTok{merogots}\OtherTok{=}\NormalTok{centrets}\SpecialCharTok{/}\NormalTok{standartnovirze[,}\DecValTok{1}\NormalTok{]}
\FunctionTok{writeRaster}\NormalTok{(merogots,}
      \AttributeTok{filename=}\NormalTok{saglabasanas\_cels,}
      \AttributeTok{overwrite=}\ConstantTok{TRUE}\NormalTok{)}
\end{Highlighting}
\end{Shaded}

\section{SoilTexture\_Clay\_r3000}\label{ch06.509}

\textbf{filename:} \texttt{SoilTexture\_Clay\_r3000.tif}

\textbf{layername:} \texttt{egv\_509}

\textbf{English name:} Fractional cover of Clay Soils within the 3 km landscape

\textbf{Latvian name:} Augsnes granulometriskās klases ``māls'' platības īpatsvars 3 km
ainavā

\textbf{Procedure:} The cover fraction within a radius of 3000 m around the analysis grid cell is
calculated as the area-weighted sum of the \hyperref[ch06.506]{analysis cells} inside the
buffer, using the workflow \texttt{egvtools::radius\_function()}. During the calculation of the landscape metric,
inverse distance weighted (power = 2) gap filling on the output is applied
to ensure no missing values at the edges. Then the layer is rewritten to set
its name. Finally, the layer is standardised by subtracting the arithmetic
mean and dividing by the root mean squared error.

\begin{Shaded}
\begin{Highlighting}[]
\CommentTok{\# libs {-}{-}{-}{-}}
\ControlFlowTok{if}\NormalTok{(}\SpecialCharTok{!}\FunctionTok{require}\NormalTok{(terra)) \{}\FunctionTok{install.packages}\NormalTok{(}\StringTok{"terra"}\NormalTok{); }\FunctionTok{require}\NormalTok{(terra)\}}
\ControlFlowTok{if}\NormalTok{(}\SpecialCharTok{!}\FunctionTok{require}\NormalTok{(egvtools)) \{remotes}\SpecialCharTok{::}\FunctionTok{install\_github}\NormalTok{(}\StringTok{"aavotins/egvtools"}\NormalTok{); }\FunctionTok{require}\NormalTok{(egvtools)\}}

\CommentTok{\# EGVs radii {-}{-}{-}{-}}

\FunctionTok{radius\_function}\NormalTok{(}
 \AttributeTok{kvadrati\_path =} \StringTok{"./Templates/TemplateGrids/tiles/"}\NormalTok{,}
 \AttributeTok{radii\_path   =} \StringTok{"./Templates/TemplateGridPoints/tiles/"}\NormalTok{,}
 \AttributeTok{tikls100\_path =} \StringTok{"./Templates/TemplateGrids/tikls100\_sauzeme.parquet"}\NormalTok{,}
 \AttributeTok{template\_path =} \StringTok{"./Templates/TemplateRasters/LV100m\_10km.tif"}\NormalTok{,}
 \AttributeTok{input\_layers  =} \FunctionTok{c}\NormalTok{(}\StringTok{"./RasterGrids\_100m/2024/RAW/SoilTexture\_Clay\_cell.tif"}\NormalTok{),}
 \AttributeTok{layer\_prefixes =} \FunctionTok{c}\NormalTok{(}\StringTok{"SoilTexture\_Clay"}\NormalTok{),}
 \AttributeTok{output\_dir   =} \StringTok{"./RasterGrids\_100m/2024/RAW/"}\NormalTok{,}
 \AttributeTok{n\_workers   =} \DecValTok{5}\NormalTok{,}
 \AttributeTok{radii     =} \FunctionTok{c}\NormalTok{(}\StringTok{"r3000"}\NormalTok{),}
 \AttributeTok{radius\_mode  =} \StringTok{"sparse"}\NormalTok{,}
 \AttributeTok{extract\_fun  =} \StringTok{"mean"}\NormalTok{,}
 \AttributeTok{fill\_missing  =} \ConstantTok{TRUE}\NormalTok{,}
 \AttributeTok{IDW\_weight   =} \DecValTok{2}\NormalTok{,}
 \AttributeTok{future\_max\_size =} \DecValTok{5} \SpecialCharTok{*} \DecValTok{1024}\SpecialCharTok{\^{}}\DecValTok{3}\NormalTok{)}

\CommentTok{\# SoilTexture\_Clay\_r3000.tif    egv\_509}

\NormalTok{slanis}\OtherTok{=}\FunctionTok{rast}\NormalTok{(}\StringTok{"./RasterGrids\_100m/2024/RAW/SoilTexture\_Clay\_r3000.tif"}\NormalTok{)}
\FunctionTok{names}\NormalTok{(slanis)}\OtherTok{=}\StringTok{"egv\_509"}
\NormalTok{slanis2}\OtherTok{=}\FunctionTok{project}\NormalTok{(slanis,template100)}
\FunctionTok{writeRaster}\NormalTok{(slanis2,}
      \StringTok{"./RasterGrids\_100m/2024/RAW/SoilTexture\_Clay\_r3000.tif"}\NormalTok{,}
      \AttributeTok{overwrite=}\ConstantTok{TRUE}\NormalTok{)}

\CommentTok{\# standardisation {-}{-}{-}{-}}
\ControlFlowTok{if}\NormalTok{(}\SpecialCharTok{!}\FunctionTok{require}\NormalTok{(terra)) \{}\FunctionTok{install.packages}\NormalTok{(}\StringTok{"terra"}\NormalTok{); }\FunctionTok{require}\NormalTok{(terra)\}}
\ControlFlowTok{if}\NormalTok{(}\SpecialCharTok{!}\FunctionTok{require}\NormalTok{(tidyverse)) \{}\FunctionTok{install.packages}\NormalTok{(}\StringTok{"tidyverse"}\NormalTok{); }\FunctionTok{require}\NormalTok{(tidyverse)\}}

\NormalTok{nosaukums}\OtherTok{=}\StringTok{"SoilTexture\_Clay\_r3000.tif"}
\NormalTok{ielasisanas\_cels}\OtherTok{=}\FunctionTok{paste0}\NormalTok{(}\StringTok{"./RasterGrids\_100m/2024/RAW/"}\NormalTok{,nosaukums)}
\NormalTok{saglabasanas\_cels}\OtherTok{=}\FunctionTok{paste0}\NormalTok{(}\StringTok{"./RasterGrids\_100m/2024/Scaled/"}\NormalTok{,nosaukums)}
\NormalTok{slanis}\OtherTok{=}\FunctionTok{rast}\NormalTok{(ielasisanas\_cels)}
\NormalTok{videjais}\OtherTok{=}\FunctionTok{global}\NormalTok{(slanis,}\AttributeTok{fun=}\StringTok{"mean"}\NormalTok{,}\AttributeTok{na.rm=}\ConstantTok{TRUE}\NormalTok{)}
\NormalTok{centrets}\OtherTok{=}\NormalTok{slanis}\SpecialCharTok{{-}}\NormalTok{videjais[,}\DecValTok{1}\NormalTok{]}
\NormalTok{standartnovirze}\OtherTok{=}\NormalTok{terra}\SpecialCharTok{::}\FunctionTok{global}\NormalTok{(centrets,}\AttributeTok{fun=}\StringTok{"rms"}\NormalTok{,}\AttributeTok{na.rm=}\ConstantTok{TRUE}\NormalTok{)}
\NormalTok{merogots}\OtherTok{=}\NormalTok{centrets}\SpecialCharTok{/}\NormalTok{standartnovirze[,}\DecValTok{1}\NormalTok{]}
\FunctionTok{writeRaster}\NormalTok{(merogots,}
      \AttributeTok{filename=}\NormalTok{saglabasanas\_cels,}
      \AttributeTok{overwrite=}\ConstantTok{TRUE}\NormalTok{)}
\end{Highlighting}
\end{Shaded}

\section{SoilTexture\_Clay\_r10000}\label{ch06.510}

\textbf{filename:} \texttt{SoilTexture\_Clay\_r10000.tif}

\textbf{layername:} \texttt{egv\_510}

\textbf{English name:} Fractional cover of Clay Soils within the 10 km landscape

\textbf{Latvian name:} Augsnes granulometriskās klases ``māls'' platības īpatsvars 10
km ainavā

\textbf{Procedure:} The cover fraction within a radius of 10000 m around the analysis grid cell is
calculated as the area-weighted sum of the \hyperref[ch06.506]{analysis cells} inside the
buffer, using the workflow \texttt{egvtools::radius\_function()}. During the calculation of the landscape metric,
inverse distance weighted (power = 2) gap filling on the output is applied
to ensure no missing values at the edges. Then the layer is rewritten to set
its name. Finally, the layer is standardised by subtracting the arithmetic
mean and dividing by the root mean squared error.

\begin{Shaded}
\begin{Highlighting}[]
\CommentTok{\# libs {-}{-}{-}{-}}
\ControlFlowTok{if}\NormalTok{(}\SpecialCharTok{!}\FunctionTok{require}\NormalTok{(terra)) \{}\FunctionTok{install.packages}\NormalTok{(}\StringTok{"terra"}\NormalTok{); }\FunctionTok{require}\NormalTok{(terra)\}}
\ControlFlowTok{if}\NormalTok{(}\SpecialCharTok{!}\FunctionTok{require}\NormalTok{(egvtools)) \{remotes}\SpecialCharTok{::}\FunctionTok{install\_github}\NormalTok{(}\StringTok{"aavotins/egvtools"}\NormalTok{); }\FunctionTok{require}\NormalTok{(egvtools)\}}

\CommentTok{\# EGVs radii {-}{-}{-}{-}}

\FunctionTok{radius\_function}\NormalTok{(}
 \AttributeTok{kvadrati\_path =} \StringTok{"./Templates/TemplateGrids/tiles/"}\NormalTok{,}
 \AttributeTok{radii\_path   =} \StringTok{"./Templates/TemplateGridPoints/tiles/"}\NormalTok{,}
 \AttributeTok{tikls100\_path =} \StringTok{"./Templates/TemplateGrids/tikls100\_sauzeme.parquet"}\NormalTok{,}
 \AttributeTok{template\_path =} \StringTok{"./Templates/TemplateRasters/LV100m\_10km.tif"}\NormalTok{,}
 \AttributeTok{input\_layers  =} \FunctionTok{c}\NormalTok{(}\StringTok{"./RasterGrids\_100m/2024/RAW/SoilTexture\_Clay\_cell.tif"}\NormalTok{),}
 \AttributeTok{layer\_prefixes =} \FunctionTok{c}\NormalTok{(}\StringTok{"SoilTexture\_Clay"}\NormalTok{),}
 \AttributeTok{output\_dir   =} \StringTok{"./RasterGrids\_100m/2024/RAW/"}\NormalTok{,}
 \AttributeTok{n\_workers   =} \DecValTok{5}\NormalTok{,}
 \AttributeTok{radii     =} \FunctionTok{c}\NormalTok{(}\StringTok{"r10000"}\NormalTok{),}
 \AttributeTok{radius\_mode  =} \StringTok{"sparse"}\NormalTok{,}
 \AttributeTok{extract\_fun  =} \StringTok{"mean"}\NormalTok{,}
 \AttributeTok{fill\_missing  =} \ConstantTok{TRUE}\NormalTok{,}
 \AttributeTok{IDW\_weight   =} \DecValTok{2}\NormalTok{,}
 \AttributeTok{future\_max\_size =} \DecValTok{5} \SpecialCharTok{*} \DecValTok{1024}\SpecialCharTok{\^{}}\DecValTok{3}\NormalTok{)}


\CommentTok{\# SoilTexture\_Clay\_r10000.tif   egv\_510}

\NormalTok{slanis}\OtherTok{=}\FunctionTok{rast}\NormalTok{(}\StringTok{"./RasterGrids\_100m/2024/RAW/SoilTexture\_Clay\_r10000.tif"}\NormalTok{)}
\FunctionTok{names}\NormalTok{(slanis)}\OtherTok{=}\StringTok{"egv\_510"}
\NormalTok{slanis2}\OtherTok{=}\FunctionTok{project}\NormalTok{(slanis,template100)}
\FunctionTok{writeRaster}\NormalTok{(slanis2,}
      \StringTok{"./RasterGrids\_100m/2024/RAW/SoilTexture\_Clay\_r10000.tif"}\NormalTok{,}
      \AttributeTok{overwrite=}\ConstantTok{TRUE}\NormalTok{)}

\CommentTok{\# standardisation {-}{-}{-}{-}}
\ControlFlowTok{if}\NormalTok{(}\SpecialCharTok{!}\FunctionTok{require}\NormalTok{(terra)) \{}\FunctionTok{install.packages}\NormalTok{(}\StringTok{"terra"}\NormalTok{); }\FunctionTok{require}\NormalTok{(terra)\}}
\ControlFlowTok{if}\NormalTok{(}\SpecialCharTok{!}\FunctionTok{require}\NormalTok{(tidyverse)) \{}\FunctionTok{install.packages}\NormalTok{(}\StringTok{"tidyverse"}\NormalTok{); }\FunctionTok{require}\NormalTok{(tidyverse)\}}

\NormalTok{nosaukums}\OtherTok{=}\StringTok{"SoilTexture\_Clay\_r10000.tif"}
\NormalTok{ielasisanas\_cels}\OtherTok{=}\FunctionTok{paste0}\NormalTok{(}\StringTok{"./RasterGrids\_100m/2024/RAW/"}\NormalTok{,nosaukums)}
\NormalTok{saglabasanas\_cels}\OtherTok{=}\FunctionTok{paste0}\NormalTok{(}\StringTok{"./RasterGrids\_100m/2024/Scaled/"}\NormalTok{,nosaukums)}
\NormalTok{slanis}\OtherTok{=}\FunctionTok{rast}\NormalTok{(ielasisanas\_cels)}
\NormalTok{videjais}\OtherTok{=}\FunctionTok{global}\NormalTok{(slanis,}\AttributeTok{fun=}\StringTok{"mean"}\NormalTok{,}\AttributeTok{na.rm=}\ConstantTok{TRUE}\NormalTok{)}
\NormalTok{centrets}\OtherTok{=}\NormalTok{slanis}\SpecialCharTok{{-}}\NormalTok{videjais[,}\DecValTok{1}\NormalTok{]}
\NormalTok{standartnovirze}\OtherTok{=}\NormalTok{terra}\SpecialCharTok{::}\FunctionTok{global}\NormalTok{(centrets,}\AttributeTok{fun=}\StringTok{"rms"}\NormalTok{,}\AttributeTok{na.rm=}\ConstantTok{TRUE}\NormalTok{)}
\NormalTok{merogots}\OtherTok{=}\NormalTok{centrets}\SpecialCharTok{/}\NormalTok{standartnovirze[,}\DecValTok{1}\NormalTok{]}
\FunctionTok{writeRaster}\NormalTok{(merogots,}
      \AttributeTok{filename=}\NormalTok{saglabasanas\_cels,}
      \AttributeTok{overwrite=}\ConstantTok{TRUE}\NormalTok{)}
\end{Highlighting}
\end{Shaded}

\section{SoilTexture\_Organic\_cell}\label{ch06.511}

\textbf{filename:} \texttt{SoilTexture\_Organic\_cell.tif}

\textbf{layername:} \texttt{egv\_511}

\textbf{English name:} Fractional cover of Organic Soils within the analysis cell (1
ha)

\textbf{Latvian name:} Augsnes granulometriskās klases ``organiskās augsnes'' platības
īpatsvars analīzes šūnā (1 ha)

\textbf{Procedure:} Derived from the \hyperref[Ch05.02]{Soil texture product}. First, the layer is
reclassified so that the class of interest is 1 and the other classes are 0. The resulting layer
is then aggregated to EGV resolution using the workflow \texttt{egvtools::input2egv()}, which
calculates the arithmetic mean to determine the cover fraction. During
aggregation, inverse distance weighted (power = 2) gap filling on the output is
applied to ensure no missing values at the edges. Finally, the layer is
standardised by subtracting the arithmetic mean and dividing by the root mean squared
error.

\begin{Shaded}
\begin{Highlighting}[]
\CommentTok{\# libs {-}{-}{-}{-}}
\ControlFlowTok{if}\NormalTok{(}\SpecialCharTok{!}\FunctionTok{require}\NormalTok{(terra)) \{}\FunctionTok{install.packages}\NormalTok{(}\StringTok{"terra"}\NormalTok{); }\FunctionTok{require}\NormalTok{(terra)\}}
\ControlFlowTok{if}\NormalTok{(}\SpecialCharTok{!}\FunctionTok{require}\NormalTok{(egvtools)) \{remotes}\SpecialCharTok{::}\FunctionTok{install\_github}\NormalTok{(}\StringTok{"aavotins/egvtools"}\NormalTok{); }\FunctionTok{require}\NormalTok{(egvtools)\}}

\CommentTok{\# templates {-}{-}{-}{-}}
\NormalTok{template10}\OtherTok{=}\FunctionTok{rast}\NormalTok{(}\StringTok{"./Templates/TemplateRasters/LV10m\_10km.tif"}\NormalTok{)}
\NormalTok{template100}\OtherTok{=}\FunctionTok{rast}\NormalTok{(}\StringTok{"./Templates/TemplateRasters/LV100m\_10km.tif"}\NormalTok{)}

\CommentTok{\# input {-}{-}{-}{-}}
\NormalTok{combtext}\OtherTok{=}\FunctionTok{rast}\NormalTok{(}\StringTok{"./RasterGrids\_10m/2024/SoilTXT\_combined.tif"}\NormalTok{)}

\CommentTok{\# EGVs cell {-}{-}{-}{-}}

\CommentTok{\# SoilTexture\_Organic\_cell.tif  egv\_511}

\NormalTok{org10}\OtherTok{=}\FunctionTok{ifel}\NormalTok{(combtext}\SpecialCharTok{==}\DecValTok{4}\NormalTok{,}\DecValTok{1}\NormalTok{,}\DecValTok{0}\NormalTok{)}

\FunctionTok{input2egv}\NormalTok{(}\AttributeTok{input=}\NormalTok{org10,}
     \AttributeTok{egv\_template=}\StringTok{"./Templates/TemplateRasters/LV100m\_10km.tif"}\NormalTok{,}
     \AttributeTok{summary\_function =} \StringTok{"average"}\NormalTok{,}
     \AttributeTok{missing\_job =} \StringTok{"FillOutput"}\NormalTok{,}
     \AttributeTok{idw\_weight =} \DecValTok{2}\NormalTok{,}
     \AttributeTok{outlocation =} \StringTok{"./RasterGrids\_100m/2024/RAW/"}\NormalTok{,}
     \AttributeTok{outfilename =} \StringTok{"SoilTexture\_Organic\_cell.tif"}\NormalTok{,}
     \AttributeTok{layername=}\StringTok{"egv\_511"}\NormalTok{,}
     \AttributeTok{return\_visible =} \ConstantTok{TRUE}\NormalTok{)}

\CommentTok{\# standardisation {-}{-}{-}{-}}
\ControlFlowTok{if}\NormalTok{(}\SpecialCharTok{!}\FunctionTok{require}\NormalTok{(terra)) \{}\FunctionTok{install.packages}\NormalTok{(}\StringTok{"terra"}\NormalTok{); }\FunctionTok{require}\NormalTok{(terra)\}}
\ControlFlowTok{if}\NormalTok{(}\SpecialCharTok{!}\FunctionTok{require}\NormalTok{(tidyverse)) \{}\FunctionTok{install.packages}\NormalTok{(}\StringTok{"tidyverse"}\NormalTok{); }\FunctionTok{require}\NormalTok{(tidyverse)\}}

\NormalTok{nosaukums}\OtherTok{=}\StringTok{"SoilTexture\_Organic\_cell.tif"}
\NormalTok{ielasisanas\_cels}\OtherTok{=}\FunctionTok{paste0}\NormalTok{(}\StringTok{"./RasterGrids\_100m/2024/RAW/"}\NormalTok{,nosaukums)}
\NormalTok{saglabasanas\_cels}\OtherTok{=}\FunctionTok{paste0}\NormalTok{(}\StringTok{"./RasterGrids\_100m/2024/Scaled/"}\NormalTok{,nosaukums)}
\NormalTok{slanis}\OtherTok{=}\FunctionTok{rast}\NormalTok{(ielasisanas\_cels)}
\NormalTok{videjais}\OtherTok{=}\FunctionTok{global}\NormalTok{(slanis,}\AttributeTok{fun=}\StringTok{"mean"}\NormalTok{,}\AttributeTok{na.rm=}\ConstantTok{TRUE}\NormalTok{)}
\NormalTok{centrets}\OtherTok{=}\NormalTok{slanis}\SpecialCharTok{{-}}\NormalTok{videjais[,}\DecValTok{1}\NormalTok{]}
\NormalTok{standartnovirze}\OtherTok{=}\NormalTok{terra}\SpecialCharTok{::}\FunctionTok{global}\NormalTok{(centrets,}\AttributeTok{fun=}\StringTok{"rms"}\NormalTok{,}\AttributeTok{na.rm=}\ConstantTok{TRUE}\NormalTok{)}
\NormalTok{merogots}\OtherTok{=}\NormalTok{centrets}\SpecialCharTok{/}\NormalTok{standartnovirze[,}\DecValTok{1}\NormalTok{]}
\FunctionTok{writeRaster}\NormalTok{(merogots,}
      \AttributeTok{filename=}\NormalTok{saglabasanas\_cels,}
      \AttributeTok{overwrite=}\ConstantTok{TRUE}\NormalTok{)}
\end{Highlighting}
\end{Shaded}

\section{SoilTexture\_Organic\_r500}\label{ch06.512}

\textbf{filename:} \texttt{SoilTexture\_Organic\_r500.tif}

\textbf{layername:} \texttt{egv\_512}

\textbf{English name:} Fractional cover of Organic Soils within the 0.5 km landscape

\textbf{Latvian name:} Augsnes granulometriskās klases ``organiskās augsnes'' platības
īpatsvars 0,5 km ainavā

\textbf{Procedure:} The cover fraction within a radius of 500 m around the analysis grid cell is
calculated as the area-weighted sum of the \hyperref[ch06.511]{analysis cells} inside the
buffer, using the workflow \texttt{egvtools::radius\_function()}. During the calculation of the landscape metric,
inverse distance weighted (power = 2) gap filling on the output is applied
to ensure no missing values at the edges. Then the layer is rewritten to set
its name. Finally, the layer is standardised by subtracting the arithmetic
mean and dividing by the root mean squared error.

\begin{Shaded}
\begin{Highlighting}[]
\CommentTok{\# libs {-}{-}{-}{-}}
\ControlFlowTok{if}\NormalTok{(}\SpecialCharTok{!}\FunctionTok{require}\NormalTok{(terra)) \{}\FunctionTok{install.packages}\NormalTok{(}\StringTok{"terra"}\NormalTok{); }\FunctionTok{require}\NormalTok{(terra)\}}
\ControlFlowTok{if}\NormalTok{(}\SpecialCharTok{!}\FunctionTok{require}\NormalTok{(egvtools)) \{remotes}\SpecialCharTok{::}\FunctionTok{install\_github}\NormalTok{(}\StringTok{"aavotins/egvtools"}\NormalTok{); }\FunctionTok{require}\NormalTok{(egvtools)\}}

\CommentTok{\# EGVs radii {-}{-}{-}{-}}

\FunctionTok{radius\_function}\NormalTok{(}
 \AttributeTok{kvadrati\_path =} \StringTok{"./Templates/TemplateGrids/tiles/"}\NormalTok{,}
 \AttributeTok{radii\_path   =} \StringTok{"./Templates/TemplateGridPoints/tiles/"}\NormalTok{,}
 \AttributeTok{tikls100\_path =} \StringTok{"./Templates/TemplateGrids/tikls100\_sauzeme.parquet"}\NormalTok{,}
 \AttributeTok{template\_path =} \StringTok{"./Templates/TemplateRasters/LV100m\_10km.tif"}\NormalTok{,}
 \AttributeTok{input\_layers  =} \FunctionTok{c}\NormalTok{(}\StringTok{"./RasterGrids\_100m/2024/RAW/SoilTexture\_Organic\_cell.tif"}\NormalTok{),}
 \AttributeTok{layer\_prefixes =} \FunctionTok{c}\NormalTok{(}\StringTok{"SoilTexture\_Organic"}\NormalTok{),}
 \AttributeTok{output\_dir   =} \StringTok{"./RasterGrids\_100m/2024/RAW/"}\NormalTok{,}
 \AttributeTok{n\_workers   =} \DecValTok{5}\NormalTok{,}
 \AttributeTok{radii     =} \FunctionTok{c}\NormalTok{(}\StringTok{"r500"}\NormalTok{),}
 \AttributeTok{radius\_mode  =} \StringTok{"sparse"}\NormalTok{,}
 \AttributeTok{extract\_fun  =} \StringTok{"mean"}\NormalTok{,}
 \AttributeTok{fill\_missing  =} \ConstantTok{TRUE}\NormalTok{,}
 \AttributeTok{IDW\_weight   =} \DecValTok{2}\NormalTok{,}
 \AttributeTok{future\_max\_size =} \DecValTok{5} \SpecialCharTok{*} \DecValTok{1024}\SpecialCharTok{\^{}}\DecValTok{3}\NormalTok{)}

\CommentTok{\# SoilTexture\_Organic\_r500.tif  egv\_512}

\NormalTok{slanis}\OtherTok{=}\FunctionTok{rast}\NormalTok{(}\StringTok{"./RasterGrids\_100m/2024/RAW/SoilTexture\_Organic\_r500.tif"}\NormalTok{)}
\FunctionTok{names}\NormalTok{(slanis)}\OtherTok{=}\StringTok{"egv\_512"}
\NormalTok{slanis2}\OtherTok{=}\FunctionTok{project}\NormalTok{(slanis,template100)}
\FunctionTok{writeRaster}\NormalTok{(slanis2,}
      \StringTok{"./RasterGrids\_100m/2024/RAW/SoilTexture\_Organic\_r500.tif"}\NormalTok{,}
      \AttributeTok{overwrite=}\ConstantTok{TRUE}\NormalTok{)}

\CommentTok{\# standardisation {-}{-}{-}{-}}
\ControlFlowTok{if}\NormalTok{(}\SpecialCharTok{!}\FunctionTok{require}\NormalTok{(terra)) \{}\FunctionTok{install.packages}\NormalTok{(}\StringTok{"terra"}\NormalTok{); }\FunctionTok{require}\NormalTok{(terra)\}}
\ControlFlowTok{if}\NormalTok{(}\SpecialCharTok{!}\FunctionTok{require}\NormalTok{(tidyverse)) \{}\FunctionTok{install.packages}\NormalTok{(}\StringTok{"tidyverse"}\NormalTok{); }\FunctionTok{require}\NormalTok{(tidyverse)\}}

\NormalTok{nosaukums}\OtherTok{=}\StringTok{"SoilTexture\_Organic\_r500.tif"}
\NormalTok{ielasisanas\_cels}\OtherTok{=}\FunctionTok{paste0}\NormalTok{(}\StringTok{"./RasterGrids\_100m/2024/RAW/"}\NormalTok{,nosaukums)}
\NormalTok{saglabasanas\_cels}\OtherTok{=}\FunctionTok{paste0}\NormalTok{(}\StringTok{"./RasterGrids\_100m/2024/Scaled/"}\NormalTok{,nosaukums)}
\NormalTok{slanis}\OtherTok{=}\FunctionTok{rast}\NormalTok{(ielasisanas\_cels)}
\NormalTok{videjais}\OtherTok{=}\FunctionTok{global}\NormalTok{(slanis,}\AttributeTok{fun=}\StringTok{"mean"}\NormalTok{,}\AttributeTok{na.rm=}\ConstantTok{TRUE}\NormalTok{)}
\NormalTok{centrets}\OtherTok{=}\NormalTok{slanis}\SpecialCharTok{{-}}\NormalTok{videjais[,}\DecValTok{1}\NormalTok{]}
\NormalTok{standartnovirze}\OtherTok{=}\NormalTok{terra}\SpecialCharTok{::}\FunctionTok{global}\NormalTok{(centrets,}\AttributeTok{fun=}\StringTok{"rms"}\NormalTok{,}\AttributeTok{na.rm=}\ConstantTok{TRUE}\NormalTok{)}
\NormalTok{merogots}\OtherTok{=}\NormalTok{centrets}\SpecialCharTok{/}\NormalTok{standartnovirze[,}\DecValTok{1}\NormalTok{]}
\FunctionTok{writeRaster}\NormalTok{(merogots,}
      \AttributeTok{filename=}\NormalTok{saglabasanas\_cels,}
      \AttributeTok{overwrite=}\ConstantTok{TRUE}\NormalTok{)}
\end{Highlighting}
\end{Shaded}

\section{SoilTexture\_Organic\_r1250}\label{ch06.513}

\textbf{filename:} \texttt{SoilTexture\_Organic\_r1250.tif}

\textbf{layername:} \texttt{egv\_513}

\textbf{English name:} Fractional cover of Organic Soils within the 1.25 km landscape

\textbf{Latvian name:} Augsnes granulometriskās klases ``organiskās augsnes'' platības
īpatsvars 1,25 km ainavā

\textbf{Procedure:} The cover fraction within a radius of 1250 m around the analysis grid cell is
calculated as the area-weighted sum of the \hyperref[ch06.511]{analysis cells} inside the
buffer, using the workflow \texttt{egvtools::radius\_function()}. During the calculation of the landscape metric,
inverse distance weighted (power = 2) gap filling on the output is applied
to ensure no missing values at the edges. Then the layer is rewritten to set
its name. Finally, the layer is standardised by subtracting the arithmetic
mean and dividing by the root mean squared error.

\begin{Shaded}
\begin{Highlighting}[]
\CommentTok{\# libs {-}{-}{-}{-}}
\ControlFlowTok{if}\NormalTok{(}\SpecialCharTok{!}\FunctionTok{require}\NormalTok{(terra)) \{}\FunctionTok{install.packages}\NormalTok{(}\StringTok{"terra"}\NormalTok{); }\FunctionTok{require}\NormalTok{(terra)\}}
\ControlFlowTok{if}\NormalTok{(}\SpecialCharTok{!}\FunctionTok{require}\NormalTok{(egvtools)) \{remotes}\SpecialCharTok{::}\FunctionTok{install\_github}\NormalTok{(}\StringTok{"aavotins/egvtools"}\NormalTok{); }\FunctionTok{require}\NormalTok{(egvtools)\}}

\CommentTok{\# EGVs radii {-}{-}{-}{-}}

\FunctionTok{radius\_function}\NormalTok{(}
 \AttributeTok{kvadrati\_path =} \StringTok{"./Templates/TemplateGrids/tiles/"}\NormalTok{,}
 \AttributeTok{radii\_path   =} \StringTok{"./Templates/TemplateGridPoints/tiles/"}\NormalTok{,}
 \AttributeTok{tikls100\_path =} \StringTok{"./Templates/TemplateGrids/tikls100\_sauzeme.parquet"}\NormalTok{,}
 \AttributeTok{template\_path =} \StringTok{"./Templates/TemplateRasters/LV100m\_10km.tif"}\NormalTok{,}
 \AttributeTok{input\_layers  =} \FunctionTok{c}\NormalTok{(}\StringTok{"./RasterGrids\_100m/2024/RAW/SoilTexture\_Organic\_cell.tif"}\NormalTok{),}
 \AttributeTok{layer\_prefixes =} \FunctionTok{c}\NormalTok{(}\StringTok{"SoilTexture\_Organic"}\NormalTok{),}
 \AttributeTok{output\_dir   =} \StringTok{"./RasterGrids\_100m/2024/RAW/"}\NormalTok{,}
 \AttributeTok{n\_workers   =} \DecValTok{5}\NormalTok{,}
 \AttributeTok{radii     =} \FunctionTok{c}\NormalTok{(}\StringTok{"r1250"}\NormalTok{),}
 \AttributeTok{radius\_mode  =} \StringTok{"sparse"}\NormalTok{,}
 \AttributeTok{extract\_fun  =} \StringTok{"mean"}\NormalTok{,}
 \AttributeTok{fill\_missing  =} \ConstantTok{TRUE}\NormalTok{,}
 \AttributeTok{IDW\_weight   =} \DecValTok{2}\NormalTok{,}
 \AttributeTok{future\_max\_size =} \DecValTok{5} \SpecialCharTok{*} \DecValTok{1024}\SpecialCharTok{\^{}}\DecValTok{3}\NormalTok{)}


\CommentTok{\# SoilTexture\_Organic\_r1250.tif egv\_513}

\NormalTok{slanis}\OtherTok{=}\FunctionTok{rast}\NormalTok{(}\StringTok{"./RasterGrids\_100m/2024/RAW/SoilTexture\_Organic\_r1250.tif"}\NormalTok{)}
\FunctionTok{names}\NormalTok{(slanis)}\OtherTok{=}\StringTok{"egv\_513"}
\NormalTok{slanis2}\OtherTok{=}\FunctionTok{project}\NormalTok{(slanis,template100)}
\FunctionTok{writeRaster}\NormalTok{(slanis2,}
      \StringTok{"./RasterGrids\_100m/2024/RAW/SoilTexture\_Organic\_r1250.tif"}\NormalTok{,}
      \AttributeTok{overwrite=}\ConstantTok{TRUE}\NormalTok{)}

\CommentTok{\# standardisation {-}{-}{-}{-}}
\ControlFlowTok{if}\NormalTok{(}\SpecialCharTok{!}\FunctionTok{require}\NormalTok{(terra)) \{}\FunctionTok{install.packages}\NormalTok{(}\StringTok{"terra"}\NormalTok{); }\FunctionTok{require}\NormalTok{(terra)\}}
\ControlFlowTok{if}\NormalTok{(}\SpecialCharTok{!}\FunctionTok{require}\NormalTok{(tidyverse)) \{}\FunctionTok{install.packages}\NormalTok{(}\StringTok{"tidyverse"}\NormalTok{); }\FunctionTok{require}\NormalTok{(tidyverse)\}}

\NormalTok{nosaukums}\OtherTok{=}\StringTok{"SoilTexture\_Organic\_r1250.tif"}
\NormalTok{ielasisanas\_cels}\OtherTok{=}\FunctionTok{paste0}\NormalTok{(}\StringTok{"./RasterGrids\_100m/2024/RAW/"}\NormalTok{,nosaukums)}
\NormalTok{saglabasanas\_cels}\OtherTok{=}\FunctionTok{paste0}\NormalTok{(}\StringTok{"./RasterGrids\_100m/2024/Scaled/"}\NormalTok{,nosaukums)}
\NormalTok{slanis}\OtherTok{=}\FunctionTok{rast}\NormalTok{(ielasisanas\_cels)}
\NormalTok{videjais}\OtherTok{=}\FunctionTok{global}\NormalTok{(slanis,}\AttributeTok{fun=}\StringTok{"mean"}\NormalTok{,}\AttributeTok{na.rm=}\ConstantTok{TRUE}\NormalTok{)}
\NormalTok{centrets}\OtherTok{=}\NormalTok{slanis}\SpecialCharTok{{-}}\NormalTok{videjais[,}\DecValTok{1}\NormalTok{]}
\NormalTok{standartnovirze}\OtherTok{=}\NormalTok{terra}\SpecialCharTok{::}\FunctionTok{global}\NormalTok{(centrets,}\AttributeTok{fun=}\StringTok{"rms"}\NormalTok{,}\AttributeTok{na.rm=}\ConstantTok{TRUE}\NormalTok{)}
\NormalTok{merogots}\OtherTok{=}\NormalTok{centrets}\SpecialCharTok{/}\NormalTok{standartnovirze[,}\DecValTok{1}\NormalTok{]}
\FunctionTok{writeRaster}\NormalTok{(merogots,}
      \AttributeTok{filename=}\NormalTok{saglabasanas\_cels,}
      \AttributeTok{overwrite=}\ConstantTok{TRUE}\NormalTok{)}
\end{Highlighting}
\end{Shaded}

\section{SoilTexture\_Organic\_r3000}\label{ch06.514}

\textbf{filename:} \texttt{SoilTexture\_Organic\_r3000.tif}

\textbf{layername:} \texttt{egv\_514}

\textbf{English name:} Fractional cover of Organic Soils within the 3 km landscape

\textbf{Latvian name:} Augsnes granulometriskās klases ``organiskās augsnes'' platības
īpatsvars 3 km ainavā

\textbf{Procedure:} The cover fraction within a radius of 3000 m around the analysis grid cell is
calculated as the area-weighted sum of the \hyperref[ch06.511]{analysis cells} inside the
buffer, using the workflow \texttt{egvtools::radius\_function()}. During the calculation of the landscape metric,
inverse distance weighted (power = 2) gap filling on the output is applied
to ensure no missing values at the edges. Then the layer is rewritten to set
its name. Finally, the layer is standardised by subtracting the arithmetic
mean and dividing by the root mean squared error.

\begin{Shaded}
\begin{Highlighting}[]
\CommentTok{\# libs {-}{-}{-}{-}}
\ControlFlowTok{if}\NormalTok{(}\SpecialCharTok{!}\FunctionTok{require}\NormalTok{(terra)) \{}\FunctionTok{install.packages}\NormalTok{(}\StringTok{"terra"}\NormalTok{); }\FunctionTok{require}\NormalTok{(terra)\}}
\ControlFlowTok{if}\NormalTok{(}\SpecialCharTok{!}\FunctionTok{require}\NormalTok{(egvtools)) \{remotes}\SpecialCharTok{::}\FunctionTok{install\_github}\NormalTok{(}\StringTok{"aavotins/egvtools"}\NormalTok{); }\FunctionTok{require}\NormalTok{(egvtools)\}}

\CommentTok{\# EGVs radii {-}{-}{-}{-}}

\FunctionTok{radius\_function}\NormalTok{(}
 \AttributeTok{kvadrati\_path =} \StringTok{"./Templates/TemplateGrids/tiles/"}\NormalTok{,}
 \AttributeTok{radii\_path   =} \StringTok{"./Templates/TemplateGridPoints/tiles/"}\NormalTok{,}
 \AttributeTok{tikls100\_path =} \StringTok{"./Templates/TemplateGrids/tikls100\_sauzeme.parquet"}\NormalTok{,}
 \AttributeTok{template\_path =} \StringTok{"./Templates/TemplateRasters/LV100m\_10km.tif"}\NormalTok{,}
 \AttributeTok{input\_layers  =} \FunctionTok{c}\NormalTok{(}\StringTok{"./RasterGrids\_100m/2024/RAW/SoilTexture\_Organic\_cell.tif"}\NormalTok{),}
 \AttributeTok{layer\_prefixes =} \FunctionTok{c}\NormalTok{(}\StringTok{"SoilTexture\_Organic"}\NormalTok{),}
 \AttributeTok{output\_dir   =} \StringTok{"./RasterGrids\_100m/2024/RAW/"}\NormalTok{,}
 \AttributeTok{n\_workers   =} \DecValTok{5}\NormalTok{,}
 \AttributeTok{radii     =} \FunctionTok{c}\NormalTok{(}\StringTok{"r3000"}\NormalTok{),}
 \AttributeTok{radius\_mode  =} \StringTok{"sparse"}\NormalTok{,}
 \AttributeTok{extract\_fun  =} \StringTok{"mean"}\NormalTok{,}
 \AttributeTok{fill\_missing  =} \ConstantTok{TRUE}\NormalTok{,}
 \AttributeTok{IDW\_weight   =} \DecValTok{2}\NormalTok{,}
 \AttributeTok{future\_max\_size =} \DecValTok{5} \SpecialCharTok{*} \DecValTok{1024}\SpecialCharTok{\^{}}\DecValTok{3}\NormalTok{)}

\CommentTok{\# SoilTexture\_Organic\_r3000.tif egv\_514}

\NormalTok{slanis}\OtherTok{=}\FunctionTok{rast}\NormalTok{(}\StringTok{"./RasterGrids\_100m/2024/RAW/SoilTexture\_Organic\_r3000.tif"}\NormalTok{)}
\FunctionTok{names}\NormalTok{(slanis)}\OtherTok{=}\StringTok{"egv\_514"}
\NormalTok{slanis2}\OtherTok{=}\FunctionTok{project}\NormalTok{(slanis,template100)}
\FunctionTok{writeRaster}\NormalTok{(slanis2,}
      \StringTok{"./RasterGrids\_100m/2024/RAW/SoilTexture\_Organic\_r3000.tif"}\NormalTok{,}
      \AttributeTok{overwrite=}\ConstantTok{TRUE}\NormalTok{)}

\CommentTok{\# standardisation {-}{-}{-}{-}}
\ControlFlowTok{if}\NormalTok{(}\SpecialCharTok{!}\FunctionTok{require}\NormalTok{(terra)) \{}\FunctionTok{install.packages}\NormalTok{(}\StringTok{"terra"}\NormalTok{); }\FunctionTok{require}\NormalTok{(terra)\}}
\ControlFlowTok{if}\NormalTok{(}\SpecialCharTok{!}\FunctionTok{require}\NormalTok{(tidyverse)) \{}\FunctionTok{install.packages}\NormalTok{(}\StringTok{"tidyverse"}\NormalTok{); }\FunctionTok{require}\NormalTok{(tidyverse)\}}

\NormalTok{nosaukums}\OtherTok{=}\StringTok{"SoilTexture\_Organic\_r3000.tif"}
\NormalTok{ielasisanas\_cels}\OtherTok{=}\FunctionTok{paste0}\NormalTok{(}\StringTok{"./RasterGrids\_100m/2024/RAW/"}\NormalTok{,nosaukums)}
\NormalTok{saglabasanas\_cels}\OtherTok{=}\FunctionTok{paste0}\NormalTok{(}\StringTok{"./RasterGrids\_100m/2024/Scaled/"}\NormalTok{,nosaukums)}
\NormalTok{slanis}\OtherTok{=}\FunctionTok{rast}\NormalTok{(ielasisanas\_cels)}
\NormalTok{videjais}\OtherTok{=}\FunctionTok{global}\NormalTok{(slanis,}\AttributeTok{fun=}\StringTok{"mean"}\NormalTok{,}\AttributeTok{na.rm=}\ConstantTok{TRUE}\NormalTok{)}
\NormalTok{centrets}\OtherTok{=}\NormalTok{slanis}\SpecialCharTok{{-}}\NormalTok{videjais[,}\DecValTok{1}\NormalTok{]}
\NormalTok{standartnovirze}\OtherTok{=}\NormalTok{terra}\SpecialCharTok{::}\FunctionTok{global}\NormalTok{(centrets,}\AttributeTok{fun=}\StringTok{"rms"}\NormalTok{,}\AttributeTok{na.rm=}\ConstantTok{TRUE}\NormalTok{)}
\NormalTok{merogots}\OtherTok{=}\NormalTok{centrets}\SpecialCharTok{/}\NormalTok{standartnovirze[,}\DecValTok{1}\NormalTok{]}
\FunctionTok{writeRaster}\NormalTok{(merogots,}
      \AttributeTok{filename=}\NormalTok{saglabasanas\_cels,}
      \AttributeTok{overwrite=}\ConstantTok{TRUE}\NormalTok{)}
\end{Highlighting}
\end{Shaded}

\section{SoilTexture\_Organic\_r10000}\label{ch06.515}

\textbf{filename:} \texttt{SoilTexture\_Organic\_r10000.tif}

\textbf{layername:} \texttt{egv\_515}

\textbf{English name:} Fractional cover of Organic Soils within the 10 km landscape

\textbf{Latvian name:} Augsnes granulometriskās klases ``organiskās augsnes'' platības
īpatsvars 10 km ainavā

\textbf{Procedure:} The cover fraction within a radius of 10000 m around the analysis grid cell is
calculated as the area-weighted sum of the \hyperref[ch06.511]{analysis cells} inside the
buffer, using the workflow \texttt{egvtools::radius\_function()}. During the calculation of the landscape metric,
inverse distance weighted (power = 2) gap filling on the output is applied
to ensure no missing values at the edges. Then the layer is rewritten to set
its name. Finally, the layer is standardised by subtracting the arithmetic
mean and dividing by the root mean squared error.

\begin{Shaded}
\begin{Highlighting}[]
\CommentTok{\# libs {-}{-}{-}{-}}
\ControlFlowTok{if}\NormalTok{(}\SpecialCharTok{!}\FunctionTok{require}\NormalTok{(terra)) \{}\FunctionTok{install.packages}\NormalTok{(}\StringTok{"terra"}\NormalTok{); }\FunctionTok{require}\NormalTok{(terra)\}}
\ControlFlowTok{if}\NormalTok{(}\SpecialCharTok{!}\FunctionTok{require}\NormalTok{(egvtools)) \{remotes}\SpecialCharTok{::}\FunctionTok{install\_github}\NormalTok{(}\StringTok{"aavotins/egvtools"}\NormalTok{); }\FunctionTok{require}\NormalTok{(egvtools)\}}

\CommentTok{\# EGVs radii {-}{-}{-}{-}}

\FunctionTok{radius\_function}\NormalTok{(}
 \AttributeTok{kvadrati\_path =} \StringTok{"./Templates/TemplateGrids/tiles/"}\NormalTok{,}
 \AttributeTok{radii\_path   =} \StringTok{"./Templates/TemplateGridPoints/tiles/"}\NormalTok{,}
 \AttributeTok{tikls100\_path =} \StringTok{"./Templates/TemplateGrids/tikls100\_sauzeme.parquet"}\NormalTok{,}
 \AttributeTok{template\_path =} \StringTok{"./Templates/TemplateRasters/LV100m\_10km.tif"}\NormalTok{,}
 \AttributeTok{input\_layers  =} \FunctionTok{c}\NormalTok{(}\StringTok{"./RasterGrids\_100m/2024/RAW/SoilTexture\_Organic\_cell.tif"}\NormalTok{),}
 \AttributeTok{layer\_prefixes =} \FunctionTok{c}\NormalTok{(}\StringTok{"SoilTexture\_Organic"}\NormalTok{),}
 \AttributeTok{output\_dir   =} \StringTok{"./RasterGrids\_100m/2024/RAW/"}\NormalTok{,}
 \AttributeTok{n\_workers   =} \DecValTok{5}\NormalTok{,}
 \AttributeTok{radii     =} \FunctionTok{c}\NormalTok{(}\StringTok{"r10000"}\NormalTok{),}
 \AttributeTok{radius\_mode  =} \StringTok{"sparse"}\NormalTok{,}
 \AttributeTok{extract\_fun  =} \StringTok{"mean"}\NormalTok{,}
 \AttributeTok{fill\_missing  =} \ConstantTok{TRUE}\NormalTok{,}
 \AttributeTok{IDW\_weight   =} \DecValTok{2}\NormalTok{,}
 \AttributeTok{future\_max\_size =} \DecValTok{5} \SpecialCharTok{*} \DecValTok{1024}\SpecialCharTok{\^{}}\DecValTok{3}\NormalTok{)}

\CommentTok{\# SoilTexture\_Organic\_r10000.tif    egv\_515}

\NormalTok{slanis}\OtherTok{=}\FunctionTok{rast}\NormalTok{(}\StringTok{"./RasterGrids\_100m/2024/RAW/SoilTexture\_Organic\_r10000.tif"}\NormalTok{)}
\FunctionTok{names}\NormalTok{(slanis)}\OtherTok{=}\StringTok{"egv\_515"}
\NormalTok{slanis2}\OtherTok{=}\FunctionTok{project}\NormalTok{(slanis,template100)}
\FunctionTok{writeRaster}\NormalTok{(slanis2,}
      \StringTok{"./RasterGrids\_100m/2024/RAW/SoilTexture\_Organic\_r10000.tif"}\NormalTok{,}
      \AttributeTok{overwrite=}\ConstantTok{TRUE}\NormalTok{)}

\CommentTok{\# standardisation {-}{-}{-}{-}}
\ControlFlowTok{if}\NormalTok{(}\SpecialCharTok{!}\FunctionTok{require}\NormalTok{(terra)) \{}\FunctionTok{install.packages}\NormalTok{(}\StringTok{"terra"}\NormalTok{); }\FunctionTok{require}\NormalTok{(terra)\}}
\ControlFlowTok{if}\NormalTok{(}\SpecialCharTok{!}\FunctionTok{require}\NormalTok{(tidyverse)) \{}\FunctionTok{install.packages}\NormalTok{(}\StringTok{"tidyverse"}\NormalTok{); }\FunctionTok{require}\NormalTok{(tidyverse)\}}

\NormalTok{nosaukums}\OtherTok{=}\StringTok{"SoilTexture\_Organic\_r10000.tif"}
\NormalTok{ielasisanas\_cels}\OtherTok{=}\FunctionTok{paste0}\NormalTok{(}\StringTok{"./RasterGrids\_100m/2024/RAW/"}\NormalTok{,nosaukums)}
\NormalTok{saglabasanas\_cels}\OtherTok{=}\FunctionTok{paste0}\NormalTok{(}\StringTok{"./RasterGrids\_100m/2024/Scaled/"}\NormalTok{,nosaukums)}
\NormalTok{slanis}\OtherTok{=}\FunctionTok{rast}\NormalTok{(ielasisanas\_cels)}
\NormalTok{videjais}\OtherTok{=}\FunctionTok{global}\NormalTok{(slanis,}\AttributeTok{fun=}\StringTok{"mean"}\NormalTok{,}\AttributeTok{na.rm=}\ConstantTok{TRUE}\NormalTok{)}
\NormalTok{centrets}\OtherTok{=}\NormalTok{slanis}\SpecialCharTok{{-}}\NormalTok{videjais[,}\DecValTok{1}\NormalTok{]}
\NormalTok{standartnovirze}\OtherTok{=}\NormalTok{terra}\SpecialCharTok{::}\FunctionTok{global}\NormalTok{(centrets,}\AttributeTok{fun=}\StringTok{"rms"}\NormalTok{,}\AttributeTok{na.rm=}\ConstantTok{TRUE}\NormalTok{)}
\NormalTok{merogots}\OtherTok{=}\NormalTok{centrets}\SpecialCharTok{/}\NormalTok{standartnovirze[,}\DecValTok{1}\NormalTok{]}
\FunctionTok{writeRaster}\NormalTok{(merogots,}
      \AttributeTok{filename=}\NormalTok{saglabasanas\_cels,}
      \AttributeTok{overwrite=}\ConstantTok{TRUE}\NormalTok{)}
\end{Highlighting}
\end{Shaded}

\section{SoilTexture\_Sand\_cell}\label{ch06.516}

\textbf{filename:} \texttt{SoilTexture\_Sand\_cell.tif}

\textbf{layername:} \texttt{egv\_516}

\textbf{English name:} Fractional cover of Sand Soils within the analysis cell (1 ha)

\textbf{Latvian name:} Augsnes granulometriskās klases ``smilts'' platības īpatsvars
analīzes šūnā (1 ha)

\textbf{Procedure:} Derived from the \hyperref[Ch05.02]{Soil texture product}. First, the layer is
reclassified so that the class of interest is 1 and the other classes are 0. The resulting layer
is then aggregated to EGV resolution using the workflow \texttt{egvtools::input2egv()}, which
calculates the arithmetic mean to determine the cover fraction. During
aggregation, inverse distance weighted (power = 2) gap filling on the output is
applied to ensure no missing values at the edges. Finally, the layer is
standardised by subtracting the arithmetic mean and dividing by the root mean squared
error.

\begin{Shaded}
\begin{Highlighting}[]
\CommentTok{\# libs {-}{-}{-}{-}}
\ControlFlowTok{if}\NormalTok{(}\SpecialCharTok{!}\FunctionTok{require}\NormalTok{(terra)) \{}\FunctionTok{install.packages}\NormalTok{(}\StringTok{"terra"}\NormalTok{); }\FunctionTok{require}\NormalTok{(terra)\}}
\ControlFlowTok{if}\NormalTok{(}\SpecialCharTok{!}\FunctionTok{require}\NormalTok{(egvtools)) \{remotes}\SpecialCharTok{::}\FunctionTok{install\_github}\NormalTok{(}\StringTok{"aavotins/egvtools"}\NormalTok{); }\FunctionTok{require}\NormalTok{(egvtools)\}}

\CommentTok{\# templates {-}{-}{-}{-}}
\NormalTok{template10}\OtherTok{=}\FunctionTok{rast}\NormalTok{(}\StringTok{"./Templates/TemplateRasters/LV10m\_10km.tif"}\NormalTok{)}
\NormalTok{template100}\OtherTok{=}\FunctionTok{rast}\NormalTok{(}\StringTok{"./Templates/TemplateRasters/LV100m\_10km.tif"}\NormalTok{)}

\CommentTok{\# input {-}{-}{-}{-}}
\NormalTok{combtext}\OtherTok{=}\FunctionTok{rast}\NormalTok{(}\StringTok{"./RasterGrids\_10m/2024/SoilTXT\_combined.tif"}\NormalTok{)}

\CommentTok{\# EGVs cell {-}{-}{-}{-}}

\CommentTok{\# SoilTexture\_Sand\_cell.tif egv\_516}

\NormalTok{sand10}\OtherTok{=}\FunctionTok{ifel}\NormalTok{(combtext}\SpecialCharTok{==}\DecValTok{1}\NormalTok{,}\DecValTok{1}\NormalTok{,}\DecValTok{0}\NormalTok{)}
\FunctionTok{plot}\NormalTok{(sand10)}

\FunctionTok{input2egv}\NormalTok{(}\AttributeTok{input=}\NormalTok{sand10,}
     \AttributeTok{egv\_template=}\StringTok{"./Templates/TemplateRasters/LV100m\_10km.tif"}\NormalTok{,}
     \AttributeTok{summary\_function =} \StringTok{"average"}\NormalTok{,}
     \AttributeTok{missing\_job =} \StringTok{"FillOutput"}\NormalTok{,}
     \AttributeTok{idw\_weight =} \DecValTok{2}\NormalTok{,}
     \AttributeTok{outlocation =} \StringTok{"./RasterGrids\_100m/2024/RAW/"}\NormalTok{,}
     \AttributeTok{outfilename =} \StringTok{"SoilTexture\_Sand\_cell.tif"}\NormalTok{,}
     \AttributeTok{layername=}\StringTok{"egv\_516"}\NormalTok{,}
     \AttributeTok{return\_visible =} \ConstantTok{TRUE}\NormalTok{)}

\CommentTok{\# standardisation {-}{-}{-}{-}}
\ControlFlowTok{if}\NormalTok{(}\SpecialCharTok{!}\FunctionTok{require}\NormalTok{(terra)) \{}\FunctionTok{install.packages}\NormalTok{(}\StringTok{"terra"}\NormalTok{); }\FunctionTok{require}\NormalTok{(terra)\}}
\ControlFlowTok{if}\NormalTok{(}\SpecialCharTok{!}\FunctionTok{require}\NormalTok{(tidyverse)) \{}\FunctionTok{install.packages}\NormalTok{(}\StringTok{"tidyverse"}\NormalTok{); }\FunctionTok{require}\NormalTok{(tidyverse)\}}

\NormalTok{nosaukums}\OtherTok{=}\StringTok{"SoilTexture\_Sand\_cell.tif"}
\NormalTok{ielasisanas\_cels}\OtherTok{=}\FunctionTok{paste0}\NormalTok{(}\StringTok{"./RasterGrids\_100m/2024/RAW/"}\NormalTok{,nosaukums)}
\NormalTok{saglabasanas\_cels}\OtherTok{=}\FunctionTok{paste0}\NormalTok{(}\StringTok{"./RasterGrids\_100m/2024/Scaled/"}\NormalTok{,nosaukums)}
\NormalTok{slanis}\OtherTok{=}\FunctionTok{rast}\NormalTok{(ielasisanas\_cels)}
\NormalTok{videjais}\OtherTok{=}\FunctionTok{global}\NormalTok{(slanis,}\AttributeTok{fun=}\StringTok{"mean"}\NormalTok{,}\AttributeTok{na.rm=}\ConstantTok{TRUE}\NormalTok{)}
\NormalTok{centrets}\OtherTok{=}\NormalTok{slanis}\SpecialCharTok{{-}}\NormalTok{videjais[,}\DecValTok{1}\NormalTok{]}
\NormalTok{standartnovirze}\OtherTok{=}\NormalTok{terra}\SpecialCharTok{::}\FunctionTok{global}\NormalTok{(centrets,}\AttributeTok{fun=}\StringTok{"rms"}\NormalTok{,}\AttributeTok{na.rm=}\ConstantTok{TRUE}\NormalTok{)}
\NormalTok{merogots}\OtherTok{=}\NormalTok{centrets}\SpecialCharTok{/}\NormalTok{standartnovirze[,}\DecValTok{1}\NormalTok{]}
\FunctionTok{writeRaster}\NormalTok{(merogots,}
      \AttributeTok{filename=}\NormalTok{saglabasanas\_cels,}
      \AttributeTok{overwrite=}\ConstantTok{TRUE}\NormalTok{)}
\end{Highlighting}
\end{Shaded}

\section{SoilTexture\_Sand\_r500}\label{ch06.517}

\textbf{filename:} \texttt{SoilTexture\_Sand\_r500.tif}

\textbf{layername:} \texttt{egv\_517}

\textbf{English name:} Fractional cover of Sand Soils within the 0.5 km landscape

\textbf{Latvian name:} Augsnes granulometriskās klases ``smilts'' platības īpatsvars
0,5 km ainavā

\textbf{Procedure:} The cover fraction within a radius of 500 m around the analysis grid cell is
calculated as the area-weighted sum of the \hyperref[ch06.516]{analysis cells} inside the
buffer, using the workflow \texttt{egvtools::radius\_function()}. During the calculation of the landscape metric,
inverse distance weighted (power = 2) gap filling on the output is applied
to ensure no missing values at the edges. Then the layer is rewritten to set
its name. Finally, the layer is standardised by subtracting the arithmetic
mean and dividing by the root mean squared error.

\begin{Shaded}
\begin{Highlighting}[]
\CommentTok{\# libs {-}{-}{-}{-}}
\ControlFlowTok{if}\NormalTok{(}\SpecialCharTok{!}\FunctionTok{require}\NormalTok{(terra)) \{}\FunctionTok{install.packages}\NormalTok{(}\StringTok{"terra"}\NormalTok{); }\FunctionTok{require}\NormalTok{(terra)\}}
\ControlFlowTok{if}\NormalTok{(}\SpecialCharTok{!}\FunctionTok{require}\NormalTok{(egvtools)) \{remotes}\SpecialCharTok{::}\FunctionTok{install\_github}\NormalTok{(}\StringTok{"aavotins/egvtools"}\NormalTok{); }\FunctionTok{require}\NormalTok{(egvtools)\}}

\CommentTok{\# EGVs radii {-}{-}{-}{-}}

\FunctionTok{radius\_function}\NormalTok{(}
 \AttributeTok{kvadrati\_path =} \StringTok{"./Templates/TemplateGrids/tiles/"}\NormalTok{,}
 \AttributeTok{radii\_path   =} \StringTok{"./Templates/TemplateGridPoints/tiles/"}\NormalTok{,}
 \AttributeTok{tikls100\_path =} \StringTok{"./Templates/TemplateGrids/tikls100\_sauzeme.parquet"}\NormalTok{,}
 \AttributeTok{template\_path =} \StringTok{"./Templates/TemplateRasters/LV100m\_10km.tif"}\NormalTok{,}
 \AttributeTok{input\_layers  =} \FunctionTok{c}\NormalTok{(}\StringTok{"./RasterGrids\_100m/2024/RAW/SoilTexture\_Sand\_cell.tif"}\NormalTok{),}
 \AttributeTok{layer\_prefixes =} \FunctionTok{c}\NormalTok{(}\StringTok{"SoilTexture\_Sand"}\NormalTok{),}
 \AttributeTok{output\_dir   =} \StringTok{"./RasterGrids\_100m/2024/RAW/"}\NormalTok{,}
 \AttributeTok{n\_workers   =} \DecValTok{5}\NormalTok{,}
 \AttributeTok{radii     =} \FunctionTok{c}\NormalTok{(}\StringTok{"r500"}\NormalTok{),}
 \AttributeTok{radius\_mode  =} \StringTok{"sparse"}\NormalTok{,}
 \AttributeTok{extract\_fun  =} \StringTok{"mean"}\NormalTok{,}
 \AttributeTok{fill\_missing  =} \ConstantTok{TRUE}\NormalTok{,}
 \AttributeTok{IDW\_weight   =} \DecValTok{2}\NormalTok{,}
 \AttributeTok{future\_max\_size =} \DecValTok{5} \SpecialCharTok{*} \DecValTok{1024}\SpecialCharTok{\^{}}\DecValTok{3}\NormalTok{)}

\CommentTok{\# SoilTexture\_Sand\_r500.tif egv\_517}

\NormalTok{slanis}\OtherTok{=}\FunctionTok{rast}\NormalTok{(}\StringTok{"./RasterGrids\_100m/2024/RAW/SoilTexture\_Sand\_r500.tif"}\NormalTok{)}
\FunctionTok{names}\NormalTok{(slanis)}\OtherTok{=}\StringTok{"egv\_517"}
\NormalTok{slanis2}\OtherTok{=}\FunctionTok{project}\NormalTok{(slanis,template100)}
\FunctionTok{writeRaster}\NormalTok{(slanis2,}
      \StringTok{"./RasterGrids\_100m/2024/RAW/SoilTexture\_Sand\_r500.tif"}\NormalTok{,}
      \AttributeTok{overwrite=}\ConstantTok{TRUE}\NormalTok{)}

\CommentTok{\# standardisation {-}{-}{-}{-}}
\ControlFlowTok{if}\NormalTok{(}\SpecialCharTok{!}\FunctionTok{require}\NormalTok{(terra)) \{}\FunctionTok{install.packages}\NormalTok{(}\StringTok{"terra"}\NormalTok{); }\FunctionTok{require}\NormalTok{(terra)\}}
\ControlFlowTok{if}\NormalTok{(}\SpecialCharTok{!}\FunctionTok{require}\NormalTok{(tidyverse)) \{}\FunctionTok{install.packages}\NormalTok{(}\StringTok{"tidyverse"}\NormalTok{); }\FunctionTok{require}\NormalTok{(tidyverse)\}}

\NormalTok{nosaukums}\OtherTok{=}\StringTok{"SoilTexture\_Sand\_r500.tif"}
\NormalTok{ielasisanas\_cels}\OtherTok{=}\FunctionTok{paste0}\NormalTok{(}\StringTok{"./RasterGrids\_100m/2024/RAW/"}\NormalTok{,nosaukums)}
\NormalTok{saglabasanas\_cels}\OtherTok{=}\FunctionTok{paste0}\NormalTok{(}\StringTok{"./RasterGrids\_100m/2024/Scaled/"}\NormalTok{,nosaukums)}
\NormalTok{slanis}\OtherTok{=}\FunctionTok{rast}\NormalTok{(ielasisanas\_cels)}
\NormalTok{videjais}\OtherTok{=}\FunctionTok{global}\NormalTok{(slanis,}\AttributeTok{fun=}\StringTok{"mean"}\NormalTok{,}\AttributeTok{na.rm=}\ConstantTok{TRUE}\NormalTok{)}
\NormalTok{centrets}\OtherTok{=}\NormalTok{slanis}\SpecialCharTok{{-}}\NormalTok{videjais[,}\DecValTok{1}\NormalTok{]}
\NormalTok{standartnovirze}\OtherTok{=}\NormalTok{terra}\SpecialCharTok{::}\FunctionTok{global}\NormalTok{(centrets,}\AttributeTok{fun=}\StringTok{"rms"}\NormalTok{,}\AttributeTok{na.rm=}\ConstantTok{TRUE}\NormalTok{)}
\NormalTok{merogots}\OtherTok{=}\NormalTok{centrets}\SpecialCharTok{/}\NormalTok{standartnovirze[,}\DecValTok{1}\NormalTok{]}
\FunctionTok{writeRaster}\NormalTok{(merogots,}
      \AttributeTok{filename=}\NormalTok{saglabasanas\_cels,}
      \AttributeTok{overwrite=}\ConstantTok{TRUE}\NormalTok{)}
\end{Highlighting}
\end{Shaded}

\section{SoilTexture\_Sand\_r1250}\label{ch06.518}

\textbf{filename:} \texttt{SoilTexture\_Sand\_r1250.tif}

\textbf{layername:} \texttt{egv\_518}

\textbf{English name:} Fractional cover of Sand Soils within the 1.25 km landscape

\textbf{Latvian name:} Augsnes granulometriskās klases ``smilts'' platības īpatsvars
1,25 km ainavā

\textbf{Procedure:} The cover fraction within a radius of 1250 m around the analysis grid cell is
calculated as the area-weighted sum of the \hyperref[ch06.516]{analysis cells} inside the
buffer, using the workflow \texttt{egvtools::radius\_function()}. During the calculation of the landscape metric,
inverse distance weighted (power = 2) gap filling on the output is applied
to ensure no missing values at the edges. Then the layer is rewritten to set
its name. Finally, the layer is standardised by subtracting the arithmetic
mean and dividing by the root mean squared error.

\begin{Shaded}
\begin{Highlighting}[]
\CommentTok{\# libs {-}{-}{-}{-}}
\ControlFlowTok{if}\NormalTok{(}\SpecialCharTok{!}\FunctionTok{require}\NormalTok{(terra)) \{}\FunctionTok{install.packages}\NormalTok{(}\StringTok{"terra"}\NormalTok{); }\FunctionTok{require}\NormalTok{(terra)\}}
\ControlFlowTok{if}\NormalTok{(}\SpecialCharTok{!}\FunctionTok{require}\NormalTok{(egvtools)) \{remotes}\SpecialCharTok{::}\FunctionTok{install\_github}\NormalTok{(}\StringTok{"aavotins/egvtools"}\NormalTok{); }\FunctionTok{require}\NormalTok{(egvtools)\}}

\CommentTok{\# EGVs radii {-}{-}{-}{-}}

\FunctionTok{radius\_function}\NormalTok{(}
 \AttributeTok{kvadrati\_path =} \StringTok{"./Templates/TemplateGrids/tiles/"}\NormalTok{,}
 \AttributeTok{radii\_path   =} \StringTok{"./Templates/TemplateGridPoints/tiles/"}\NormalTok{,}
 \AttributeTok{tikls100\_path =} \StringTok{"./Templates/TemplateGrids/tikls100\_sauzeme.parquet"}\NormalTok{,}
 \AttributeTok{template\_path =} \StringTok{"./Templates/TemplateRasters/LV100m\_10km.tif"}\NormalTok{,}
 \AttributeTok{input\_layers  =} \FunctionTok{c}\NormalTok{(}\StringTok{"./RasterGrids\_100m/2024/RAW/SoilTexture\_Sand\_cell.tif"}\NormalTok{),}
 \AttributeTok{layer\_prefixes =} \FunctionTok{c}\NormalTok{(}\StringTok{"SoilTexture\_Sand"}\NormalTok{),}
 \AttributeTok{output\_dir   =} \StringTok{"./RasterGrids\_100m/2024/RAW/"}\NormalTok{,}
 \AttributeTok{n\_workers   =} \DecValTok{5}\NormalTok{,}
 \AttributeTok{radii     =} \FunctionTok{c}\NormalTok{(}\StringTok{"r1250"}\NormalTok{),}
 \AttributeTok{radius\_mode  =} \StringTok{"sparse"}\NormalTok{,}
 \AttributeTok{extract\_fun  =} \StringTok{"mean"}\NormalTok{,}
 \AttributeTok{fill\_missing  =} \ConstantTok{TRUE}\NormalTok{,}
 \AttributeTok{IDW\_weight   =} \DecValTok{2}\NormalTok{,}
 \AttributeTok{future\_max\_size =} \DecValTok{5} \SpecialCharTok{*} \DecValTok{1024}\SpecialCharTok{\^{}}\DecValTok{3}\NormalTok{)}

\CommentTok{\# SoilTexture\_Sand\_r1250.tif    egv\_518}

\NormalTok{slanis}\OtherTok{=}\FunctionTok{rast}\NormalTok{(}\StringTok{"./RasterGrids\_100m/2024/RAW/SoilTexture\_Sand\_r1250.tif"}\NormalTok{)}
\FunctionTok{names}\NormalTok{(slanis)}\OtherTok{=}\StringTok{"egv\_518"}
\NormalTok{slanis2}\OtherTok{=}\FunctionTok{project}\NormalTok{(slanis,template100)}
\FunctionTok{writeRaster}\NormalTok{(slanis2,}
      \StringTok{"./RasterGrids\_100m/2024/RAW/SoilTexture\_Sand\_r1250.tif"}\NormalTok{,}
      \AttributeTok{overwrite=}\ConstantTok{TRUE}\NormalTok{)}

\CommentTok{\# standardisation {-}{-}{-}{-}}
\ControlFlowTok{if}\NormalTok{(}\SpecialCharTok{!}\FunctionTok{require}\NormalTok{(terra)) \{}\FunctionTok{install.packages}\NormalTok{(}\StringTok{"terra"}\NormalTok{); }\FunctionTok{require}\NormalTok{(terra)\}}
\ControlFlowTok{if}\NormalTok{(}\SpecialCharTok{!}\FunctionTok{require}\NormalTok{(tidyverse)) \{}\FunctionTok{install.packages}\NormalTok{(}\StringTok{"tidyverse"}\NormalTok{); }\FunctionTok{require}\NormalTok{(tidyverse)\}}

\NormalTok{nosaukums}\OtherTok{=}\StringTok{"SoilTexture\_Sand\_r1250.tif"}
\NormalTok{ielasisanas\_cels}\OtherTok{=}\FunctionTok{paste0}\NormalTok{(}\StringTok{"./RasterGrids\_100m/2024/RAW/"}\NormalTok{,nosaukums)}
\NormalTok{saglabasanas\_cels}\OtherTok{=}\FunctionTok{paste0}\NormalTok{(}\StringTok{"./RasterGrids\_100m/2024/Scaled/"}\NormalTok{,nosaukums)}
\NormalTok{slanis}\OtherTok{=}\FunctionTok{rast}\NormalTok{(ielasisanas\_cels)}
\NormalTok{videjais}\OtherTok{=}\FunctionTok{global}\NormalTok{(slanis,}\AttributeTok{fun=}\StringTok{"mean"}\NormalTok{,}\AttributeTok{na.rm=}\ConstantTok{TRUE}\NormalTok{)}
\NormalTok{centrets}\OtherTok{=}\NormalTok{slanis}\SpecialCharTok{{-}}\NormalTok{videjais[,}\DecValTok{1}\NormalTok{]}
\NormalTok{standartnovirze}\OtherTok{=}\NormalTok{terra}\SpecialCharTok{::}\FunctionTok{global}\NormalTok{(centrets,}\AttributeTok{fun=}\StringTok{"rms"}\NormalTok{,}\AttributeTok{na.rm=}\ConstantTok{TRUE}\NormalTok{)}
\NormalTok{merogots}\OtherTok{=}\NormalTok{centrets}\SpecialCharTok{/}\NormalTok{standartnovirze[,}\DecValTok{1}\NormalTok{]}
\FunctionTok{writeRaster}\NormalTok{(merogots,}
      \AttributeTok{filename=}\NormalTok{saglabasanas\_cels,}
      \AttributeTok{overwrite=}\ConstantTok{TRUE}\NormalTok{)}
\end{Highlighting}
\end{Shaded}

\section{SoilTexture\_Sand\_r3000}\label{ch06.519}

\textbf{filename:} \texttt{SoilTexture\_Sand\_r3000.tif}

\textbf{layername:} \texttt{egv\_519}

\textbf{English name:} Fractional cover of Sand Soils within the 3 km landscape

\textbf{Latvian name:} Augsnes granulometriskās klases ``smilts'' platības īpatsvars 3
km ainavā

\textbf{Procedure:} The cover fraction within a radius of 3000 m around the analysis grid cell is
calculated as the area-weighted sum of the \hyperref[ch06.516]{analysis cells} inside the
buffer, using the workflow \texttt{egvtools::radius\_function()}. During the calculation of the landscape metric,
inverse distance weighted (power = 2) gap filling on the output is applied
to ensure no missing values at the edges. Then the layer is rewritten to set
its name. Finally, the layer is standardised by subtracting the arithmetic
mean and dividing by the root mean squared error.

\begin{Shaded}
\begin{Highlighting}[]
\CommentTok{\# libs {-}{-}{-}{-}}
\ControlFlowTok{if}\NormalTok{(}\SpecialCharTok{!}\FunctionTok{require}\NormalTok{(terra)) \{}\FunctionTok{install.packages}\NormalTok{(}\StringTok{"terra"}\NormalTok{); }\FunctionTok{require}\NormalTok{(terra)\}}
\ControlFlowTok{if}\NormalTok{(}\SpecialCharTok{!}\FunctionTok{require}\NormalTok{(egvtools)) \{remotes}\SpecialCharTok{::}\FunctionTok{install\_github}\NormalTok{(}\StringTok{"aavotins/egvtools"}\NormalTok{); }\FunctionTok{require}\NormalTok{(egvtools)\}}

\CommentTok{\# EGVs radii {-}{-}{-}{-}}

\FunctionTok{radius\_function}\NormalTok{(}
 \AttributeTok{kvadrati\_path =} \StringTok{"./Templates/TemplateGrids/tiles/"}\NormalTok{,}
 \AttributeTok{radii\_path   =} \StringTok{"./Templates/TemplateGridPoints/tiles/"}\NormalTok{,}
 \AttributeTok{tikls100\_path =} \StringTok{"./Templates/TemplateGrids/tikls100\_sauzeme.parquet"}\NormalTok{,}
 \AttributeTok{template\_path =} \StringTok{"./Templates/TemplateRasters/LV100m\_10km.tif"}\NormalTok{,}
 \AttributeTok{input\_layers  =} \FunctionTok{c}\NormalTok{(}\StringTok{"./RasterGrids\_100m/2024/RAW/SoilTexture\_Sand\_cell.tif"}\NormalTok{),}
 \AttributeTok{layer\_prefixes =} \FunctionTok{c}\NormalTok{(}\StringTok{"SoilTexture\_Sand"}\NormalTok{),}
 \AttributeTok{output\_dir   =} \StringTok{"./RasterGrids\_100m/2024/RAW/"}\NormalTok{,}
 \AttributeTok{n\_workers   =} \DecValTok{5}\NormalTok{,}
 \AttributeTok{radii     =} \FunctionTok{c}\NormalTok{(}\StringTok{"r3000"}\NormalTok{),}
 \AttributeTok{radius\_mode  =} \StringTok{"sparse"}\NormalTok{,}
 \AttributeTok{extract\_fun  =} \StringTok{"mean"}\NormalTok{,}
 \AttributeTok{fill\_missing  =} \ConstantTok{TRUE}\NormalTok{,}
 \AttributeTok{IDW\_weight   =} \DecValTok{2}\NormalTok{,}
 \AttributeTok{future\_max\_size =} \DecValTok{5} \SpecialCharTok{*} \DecValTok{1024}\SpecialCharTok{\^{}}\DecValTok{3}\NormalTok{)}

\CommentTok{\# SoilTexture\_Sand\_r3000.tif    egv\_519}

\NormalTok{slanis}\OtherTok{=}\FunctionTok{rast}\NormalTok{(}\StringTok{"./RasterGrids\_100m/2024/RAW/SoilTexture\_Sand\_r3000.tif"}\NormalTok{)}
\FunctionTok{names}\NormalTok{(slanis)}\OtherTok{=}\StringTok{"egv\_519"}
\NormalTok{slanis2}\OtherTok{=}\FunctionTok{project}\NormalTok{(slanis,template100)}
\FunctionTok{writeRaster}\NormalTok{(slanis2,}
      \StringTok{"./RasterGrids\_100m/2024/RAW/SoilTexture\_Sand\_r3000.tif"}\NormalTok{,}
      \AttributeTok{overwrite=}\ConstantTok{TRUE}\NormalTok{)}

\CommentTok{\# standardisation {-}{-}{-}{-}}
\ControlFlowTok{if}\NormalTok{(}\SpecialCharTok{!}\FunctionTok{require}\NormalTok{(terra)) \{}\FunctionTok{install.packages}\NormalTok{(}\StringTok{"terra"}\NormalTok{); }\FunctionTok{require}\NormalTok{(terra)\}}
\ControlFlowTok{if}\NormalTok{(}\SpecialCharTok{!}\FunctionTok{require}\NormalTok{(tidyverse)) \{}\FunctionTok{install.packages}\NormalTok{(}\StringTok{"tidyverse"}\NormalTok{); }\FunctionTok{require}\NormalTok{(tidyverse)\}}

\NormalTok{nosaukums}\OtherTok{=}\StringTok{"SoilTexture\_Sand\_r3000.tif"}
\NormalTok{ielasisanas\_cels}\OtherTok{=}\FunctionTok{paste0}\NormalTok{(}\StringTok{"./RasterGrids\_100m/2024/RAW/"}\NormalTok{,nosaukums)}
\NormalTok{saglabasanas\_cels}\OtherTok{=}\FunctionTok{paste0}\NormalTok{(}\StringTok{"./RasterGrids\_100m/2024/Scaled/"}\NormalTok{,nosaukums)}
\NormalTok{slanis}\OtherTok{=}\FunctionTok{rast}\NormalTok{(ielasisanas\_cels)}
\NormalTok{videjais}\OtherTok{=}\FunctionTok{global}\NormalTok{(slanis,}\AttributeTok{fun=}\StringTok{"mean"}\NormalTok{,}\AttributeTok{na.rm=}\ConstantTok{TRUE}\NormalTok{)}
\NormalTok{centrets}\OtherTok{=}\NormalTok{slanis}\SpecialCharTok{{-}}\NormalTok{videjais[,}\DecValTok{1}\NormalTok{]}
\NormalTok{standartnovirze}\OtherTok{=}\NormalTok{terra}\SpecialCharTok{::}\FunctionTok{global}\NormalTok{(centrets,}\AttributeTok{fun=}\StringTok{"rms"}\NormalTok{,}\AttributeTok{na.rm=}\ConstantTok{TRUE}\NormalTok{)}
\NormalTok{merogots}\OtherTok{=}\NormalTok{centrets}\SpecialCharTok{/}\NormalTok{standartnovirze[,}\DecValTok{1}\NormalTok{]}
\FunctionTok{writeRaster}\NormalTok{(merogots,}
      \AttributeTok{filename=}\NormalTok{saglabasanas\_cels,}
      \AttributeTok{overwrite=}\ConstantTok{TRUE}\NormalTok{)}
\end{Highlighting}
\end{Shaded}

\section{SoilTexture\_Sand\_r10000}\label{ch06.520}

\textbf{filename:} \texttt{SoilTexture\_Sand\_r10000.tif}

\textbf{layername:} \texttt{egv\_520}

\textbf{English name:} Fractional cover of Sand Soils within the 10 km landscape

\textbf{Latvian name:} Augsnes granulometriskās klases ``smilts'' platības īpatsvars 10
km ainavā

\textbf{Procedure:} The cover fraction within a radius of 10000 m around the analysis grid cell is
calculated as the area-weighted sum of the \hyperref[ch06.516]{analysis cells} inside the
buffer, using the workflow \texttt{egvtools::radius\_function()}. During the calculation of the landscape metric,
inverse distance weighted (power = 2) gap filling on the output is applied
to ensure no missing values at the edges. Then the layer is rewritten to set
its name. Finally, the layer is standardised by subtracting the arithmetic
mean and dividing by the root mean squared error.

\begin{Shaded}
\begin{Highlighting}[]
\CommentTok{\# libs {-}{-}{-}{-}}
\ControlFlowTok{if}\NormalTok{(}\SpecialCharTok{!}\FunctionTok{require}\NormalTok{(terra)) \{}\FunctionTok{install.packages}\NormalTok{(}\StringTok{"terra"}\NormalTok{); }\FunctionTok{require}\NormalTok{(terra)\}}
\ControlFlowTok{if}\NormalTok{(}\SpecialCharTok{!}\FunctionTok{require}\NormalTok{(egvtools)) \{remotes}\SpecialCharTok{::}\FunctionTok{install\_github}\NormalTok{(}\StringTok{"aavotins/egvtools"}\NormalTok{); }\FunctionTok{require}\NormalTok{(egvtools)\}}

\CommentTok{\# EGVs radii {-}{-}{-}{-}}

\FunctionTok{radius\_function}\NormalTok{(}
 \AttributeTok{kvadrati\_path =} \StringTok{"./Templates/TemplateGrids/tiles/"}\NormalTok{,}
 \AttributeTok{radii\_path   =} \StringTok{"./Templates/TemplateGridPoints/tiles/"}\NormalTok{,}
 \AttributeTok{tikls100\_path =} \StringTok{"./Templates/TemplateGrids/tikls100\_sauzeme.parquet"}\NormalTok{,}
 \AttributeTok{template\_path =} \StringTok{"./Templates/TemplateRasters/LV100m\_10km.tif"}\NormalTok{,}
 \AttributeTok{input\_layers  =} \FunctionTok{c}\NormalTok{(}\StringTok{"./RasterGrids\_100m/2024/RAW/SoilTexture\_Sand\_cell.tif"}\NormalTok{),}
 \AttributeTok{layer\_prefixes =} \FunctionTok{c}\NormalTok{(}\StringTok{"SoilTexture\_Sand"}\NormalTok{),}
 \AttributeTok{output\_dir   =} \StringTok{"./RasterGrids\_100m/2024/RAW/"}\NormalTok{,}
 \AttributeTok{n\_workers   =} \DecValTok{5}\NormalTok{,}
 \AttributeTok{radii     =} \FunctionTok{c}\NormalTok{(}\StringTok{"r10000"}\NormalTok{),}
 \AttributeTok{radius\_mode  =} \StringTok{"sparse"}\NormalTok{,}
 \AttributeTok{extract\_fun  =} \StringTok{"mean"}\NormalTok{,}
 \AttributeTok{fill\_missing  =} \ConstantTok{TRUE}\NormalTok{,}
 \AttributeTok{IDW\_weight   =} \DecValTok{2}\NormalTok{,}
 \AttributeTok{future\_max\_size =} \DecValTok{5} \SpecialCharTok{*} \DecValTok{1024}\SpecialCharTok{\^{}}\DecValTok{3}\NormalTok{)}

\CommentTok{\# SoilTexture\_Sand\_r10000.tif   egv\_520}

\NormalTok{slanis}\OtherTok{=}\FunctionTok{rast}\NormalTok{(}\StringTok{"./RasterGrids\_100m/2024/RAW/SoilTexture\_Sand\_r10000.tif"}\NormalTok{)}
\FunctionTok{names}\NormalTok{(slanis)}\OtherTok{=}\StringTok{"egv\_520"}
\NormalTok{slanis2}\OtherTok{=}\FunctionTok{project}\NormalTok{(slanis,template100)}
\FunctionTok{writeRaster}\NormalTok{(slanis2,}
      \StringTok{"./RasterGrids\_100m/2024/RAW/SoilTexture\_Sand\_r10000.tif"}\NormalTok{,}
      \AttributeTok{overwrite=}\ConstantTok{TRUE}\NormalTok{)}

\CommentTok{\# standardisation {-}{-}{-}{-}}
\ControlFlowTok{if}\NormalTok{(}\SpecialCharTok{!}\FunctionTok{require}\NormalTok{(terra)) \{}\FunctionTok{install.packages}\NormalTok{(}\StringTok{"terra"}\NormalTok{); }\FunctionTok{require}\NormalTok{(terra)\}}
\ControlFlowTok{if}\NormalTok{(}\SpecialCharTok{!}\FunctionTok{require}\NormalTok{(tidyverse)) \{}\FunctionTok{install.packages}\NormalTok{(}\StringTok{"tidyverse"}\NormalTok{); }\FunctionTok{require}\NormalTok{(tidyverse)\}}

\NormalTok{nosaukums}\OtherTok{=}\StringTok{"SoilTexture\_Sand\_r10000.tif"}
\NormalTok{ielasisanas\_cels}\OtherTok{=}\FunctionTok{paste0}\NormalTok{(}\StringTok{"./RasterGrids\_100m/2024/RAW/"}\NormalTok{,nosaukums)}
\NormalTok{saglabasanas\_cels}\OtherTok{=}\FunctionTok{paste0}\NormalTok{(}\StringTok{"./RasterGrids\_100m/2024/Scaled/"}\NormalTok{,nosaukums)}
\NormalTok{slanis}\OtherTok{=}\FunctionTok{rast}\NormalTok{(ielasisanas\_cels)}
\NormalTok{videjais}\OtherTok{=}\FunctionTok{global}\NormalTok{(slanis,}\AttributeTok{fun=}\StringTok{"mean"}\NormalTok{,}\AttributeTok{na.rm=}\ConstantTok{TRUE}\NormalTok{)}
\NormalTok{centrets}\OtherTok{=}\NormalTok{slanis}\SpecialCharTok{{-}}\NormalTok{videjais[,}\DecValTok{1}\NormalTok{]}
\NormalTok{standartnovirze}\OtherTok{=}\NormalTok{terra}\SpecialCharTok{::}\FunctionTok{global}\NormalTok{(centrets,}\AttributeTok{fun=}\StringTok{"rms"}\NormalTok{,}\AttributeTok{na.rm=}\ConstantTok{TRUE}\NormalTok{)}
\NormalTok{merogots}\OtherTok{=}\NormalTok{centrets}\SpecialCharTok{/}\NormalTok{standartnovirze[,}\DecValTok{1}\NormalTok{]}
\FunctionTok{writeRaster}\NormalTok{(merogots,}
      \AttributeTok{filename=}\NormalTok{saglabasanas\_cels,}
      \AttributeTok{overwrite=}\ConstantTok{TRUE}\NormalTok{)}
\end{Highlighting}
\end{Shaded}

\section{SoilTexture\_Silt\_cell}\label{ch06.521}

\textbf{filename:} \texttt{SoilTexture\_Silt\_cell.tif}

\textbf{layername:} \texttt{egv\_521}

\textbf{English name:} Fractional cover of Silt Soils within the analysis cell (1 ha)

\textbf{Latvian name:} Augsnes granulometriskās klases ``smilšmāls un mālsmilts''
platības īpatsvars analīzes šūnā (1 ha)

\textbf{Procedure:} Derived from the \hyperref[Ch05.02]{Soil texture product}. First, the layer is
reclassified so that the class of interest is 1 and the other classes are 0. The resulting layer
is then aggregated to EGV resolution using the workflow \texttt{egvtools::input2egv()}, which
calculates the arithmetic mean to determine the cover fraction. During
aggregation, inverse distance weighted (power = 2) gap filling on the output is
applied to ensure no missing values at the edges. Finally, the layer is
standardised by subtracting the arithmetic mean and dividing by the root mean squared
error.

\begin{Shaded}
\begin{Highlighting}[]
\CommentTok{\# libs {-}{-}{-}{-}}
\ControlFlowTok{if}\NormalTok{(}\SpecialCharTok{!}\FunctionTok{require}\NormalTok{(terra)) \{}\FunctionTok{install.packages}\NormalTok{(}\StringTok{"terra"}\NormalTok{); }\FunctionTok{require}\NormalTok{(terra)\}}
\ControlFlowTok{if}\NormalTok{(}\SpecialCharTok{!}\FunctionTok{require}\NormalTok{(egvtools)) \{remotes}\SpecialCharTok{::}\FunctionTok{install\_github}\NormalTok{(}\StringTok{"aavotins/egvtools"}\NormalTok{); }\FunctionTok{require}\NormalTok{(egvtools)\}}

\CommentTok{\# templates {-}{-}{-}{-}}
\NormalTok{template10}\OtherTok{=}\FunctionTok{rast}\NormalTok{(}\StringTok{"./Templates/TemplateRasters/LV10m\_10km.tif"}\NormalTok{)}
\NormalTok{template100}\OtherTok{=}\FunctionTok{rast}\NormalTok{(}\StringTok{"./Templates/TemplateRasters/LV100m\_10km.tif"}\NormalTok{)}

\CommentTok{\# input {-}{-}{-}{-}}
\NormalTok{combtext}\OtherTok{=}\FunctionTok{rast}\NormalTok{(}\StringTok{"./RasterGrids\_10m/2024/SoilTXT\_combined.tif"}\NormalTok{)}

\CommentTok{\# EGVs cell {-}{-}{-}{-}}

\CommentTok{\# SoilTexture\_Silt\_cell.tif egv\_521}

\NormalTok{silt10}\OtherTok{=}\FunctionTok{ifel}\NormalTok{(combtext}\SpecialCharTok{==}\DecValTok{2}\NormalTok{,}\DecValTok{1}\NormalTok{,}\DecValTok{0}\NormalTok{)}

\FunctionTok{input2egv}\NormalTok{(}\AttributeTok{input=}\NormalTok{silt10,}
     \AttributeTok{egv\_template=}\StringTok{"./Templates/TemplateRasters/LV100m\_10km.tif"}\NormalTok{,}
     \AttributeTok{summary\_function =} \StringTok{"average"}\NormalTok{,}
     \AttributeTok{missing\_job =} \StringTok{"FillOutput"}\NormalTok{,}
     \AttributeTok{idw\_weight =} \DecValTok{2}\NormalTok{,}
     \AttributeTok{outlocation =} \StringTok{"./RasterGrids\_100m/2024/RAW/"}\NormalTok{,}
     \AttributeTok{outfilename =} \StringTok{"SoilTexture\_Silt\_cell.tif"}\NormalTok{,}
     \AttributeTok{layername=}\StringTok{"egv\_521"}\NormalTok{,}
     \AttributeTok{return\_visible =} \ConstantTok{TRUE}\NormalTok{)}

\CommentTok{\# standardisation {-}{-}{-}{-}}
\ControlFlowTok{if}\NormalTok{(}\SpecialCharTok{!}\FunctionTok{require}\NormalTok{(terra)) \{}\FunctionTok{install.packages}\NormalTok{(}\StringTok{"terra"}\NormalTok{); }\FunctionTok{require}\NormalTok{(terra)\}}
\ControlFlowTok{if}\NormalTok{(}\SpecialCharTok{!}\FunctionTok{require}\NormalTok{(tidyverse)) \{}\FunctionTok{install.packages}\NormalTok{(}\StringTok{"tidyverse"}\NormalTok{); }\FunctionTok{require}\NormalTok{(tidyverse)\}}

\NormalTok{nosaukums}\OtherTok{=}\StringTok{"SoilTexture\_Silt\_cell.tif"}
\NormalTok{ielasisanas\_cels}\OtherTok{=}\FunctionTok{paste0}\NormalTok{(}\StringTok{"./RasterGrids\_100m/2024/RAW/"}\NormalTok{,nosaukums)}
\NormalTok{saglabasanas\_cels}\OtherTok{=}\FunctionTok{paste0}\NormalTok{(}\StringTok{"./RasterGrids\_100m/2024/Scaled/"}\NormalTok{,nosaukums)}
\NormalTok{slanis}\OtherTok{=}\FunctionTok{rast}\NormalTok{(ielasisanas\_cels)}
\NormalTok{videjais}\OtherTok{=}\FunctionTok{global}\NormalTok{(slanis,}\AttributeTok{fun=}\StringTok{"mean"}\NormalTok{,}\AttributeTok{na.rm=}\ConstantTok{TRUE}\NormalTok{)}
\NormalTok{centrets}\OtherTok{=}\NormalTok{slanis}\SpecialCharTok{{-}}\NormalTok{videjais[,}\DecValTok{1}\NormalTok{]}
\NormalTok{standartnovirze}\OtherTok{=}\NormalTok{terra}\SpecialCharTok{::}\FunctionTok{global}\NormalTok{(centrets,}\AttributeTok{fun=}\StringTok{"rms"}\NormalTok{,}\AttributeTok{na.rm=}\ConstantTok{TRUE}\NormalTok{)}
\NormalTok{merogots}\OtherTok{=}\NormalTok{centrets}\SpecialCharTok{/}\NormalTok{standartnovirze[,}\DecValTok{1}\NormalTok{]}
\FunctionTok{writeRaster}\NormalTok{(merogots,}
      \AttributeTok{filename=}\NormalTok{saglabasanas\_cels,}
      \AttributeTok{overwrite=}\ConstantTok{TRUE}\NormalTok{)}
\end{Highlighting}
\end{Shaded}

\section{SoilTexture\_Silt\_r500}\label{ch06.522}

\textbf{filename:} \texttt{SoilTexture\_Silt\_r500.tif}

\textbf{layername:} \texttt{egv\_522}

\textbf{English name:} Fractional cover of Silt Soils within the 0.5 km landscape

\textbf{Latvian name:} Augsnes granulometriskās klases ``smilšmāls un mālsmilts''
platības īpatsvars 0,5 km ainavā

\textbf{Procedure:} The cover fraction within a radius of 500 m around the analysis grid cell is
calculated as the area-weighted sum of the \hyperref[ch06.521]{analysis cells} inside the
buffer, using the workflow \texttt{egvtools::radius\_function()}. During the calculation of the landscape metric,
inverse distance weighted (power = 2) gap filling on the output is applied
to ensure no missing values at the edges. Then the layer is rewritten to set
its name. Finally, the layer is standardised by subtracting the arithmetic
mean and dividing by the root mean squared error.

\begin{Shaded}
\begin{Highlighting}[]
\CommentTok{\# libs {-}{-}{-}{-}}
\ControlFlowTok{if}\NormalTok{(}\SpecialCharTok{!}\FunctionTok{require}\NormalTok{(terra)) \{}\FunctionTok{install.packages}\NormalTok{(}\StringTok{"terra"}\NormalTok{); }\FunctionTok{require}\NormalTok{(terra)\}}
\ControlFlowTok{if}\NormalTok{(}\SpecialCharTok{!}\FunctionTok{require}\NormalTok{(egvtools)) \{remotes}\SpecialCharTok{::}\FunctionTok{install\_github}\NormalTok{(}\StringTok{"aavotins/egvtools"}\NormalTok{); }\FunctionTok{require}\NormalTok{(egvtools)\}}

\CommentTok{\# EGVs radii {-}{-}{-}{-}}

\FunctionTok{radius\_function}\NormalTok{(}
 \AttributeTok{kvadrati\_path =} \StringTok{"./Templates/TemplateGrids/tiles/"}\NormalTok{,}
 \AttributeTok{radii\_path   =} \StringTok{"./Templates/TemplateGridPoints/tiles/"}\NormalTok{,}
 \AttributeTok{tikls100\_path =} \StringTok{"./Templates/TemplateGrids/tikls100\_sauzeme.parquet"}\NormalTok{,}
 \AttributeTok{template\_path =} \StringTok{"./Templates/TemplateRasters/LV100m\_10km.tif"}\NormalTok{,}
 \AttributeTok{input\_layers  =} \FunctionTok{c}\NormalTok{(}\StringTok{"./RasterGrids\_100m/2024/RAW/SoilTexture\_Silt\_cell.tif"}\NormalTok{),}
 \AttributeTok{layer\_prefixes =} \FunctionTok{c}\NormalTok{(}\StringTok{"SoilTexture\_Silt"}\NormalTok{),}
 \AttributeTok{output\_dir   =} \StringTok{"./RasterGrids\_100m/2024/RAW/"}\NormalTok{,}
 \AttributeTok{n\_workers   =} \DecValTok{5}\NormalTok{,}
 \AttributeTok{radii     =} \FunctionTok{c}\NormalTok{(}\StringTok{"r500"}\NormalTok{),}
 \AttributeTok{radius\_mode  =} \StringTok{"sparse"}\NormalTok{,}
 \AttributeTok{extract\_fun  =} \StringTok{"mean"}\NormalTok{,}
 \AttributeTok{fill\_missing  =} \ConstantTok{TRUE}\NormalTok{,}
 \AttributeTok{IDW\_weight   =} \DecValTok{2}\NormalTok{,}
 \AttributeTok{future\_max\_size =} \DecValTok{5} \SpecialCharTok{*} \DecValTok{1024}\SpecialCharTok{\^{}}\DecValTok{3}\NormalTok{)}

\CommentTok{\# SoilTexture\_Silt\_r500.tif egv\_522}

\NormalTok{slanis}\OtherTok{=}\FunctionTok{rast}\NormalTok{(}\StringTok{"./RasterGrids\_100m/2024/RAW/SoilTexture\_Silt\_r500.tif"}\NormalTok{)}
\FunctionTok{names}\NormalTok{(slanis)}\OtherTok{=}\StringTok{"egv\_522"}
\NormalTok{slanis2}\OtherTok{=}\FunctionTok{project}\NormalTok{(slanis,template100)}
\FunctionTok{writeRaster}\NormalTok{(slanis2,}
      \StringTok{"./RasterGrids\_100m/2024/RAW/SoilTexture\_Silt\_r500.tif"}\NormalTok{,}
      \AttributeTok{overwrite=}\ConstantTok{TRUE}\NormalTok{)}

\CommentTok{\# standardisation {-}{-}{-}{-}}
\ControlFlowTok{if}\NormalTok{(}\SpecialCharTok{!}\FunctionTok{require}\NormalTok{(terra)) \{}\FunctionTok{install.packages}\NormalTok{(}\StringTok{"terra"}\NormalTok{); }\FunctionTok{require}\NormalTok{(terra)\}}
\ControlFlowTok{if}\NormalTok{(}\SpecialCharTok{!}\FunctionTok{require}\NormalTok{(tidyverse)) \{}\FunctionTok{install.packages}\NormalTok{(}\StringTok{"tidyverse"}\NormalTok{); }\FunctionTok{require}\NormalTok{(tidyverse)\}}

\NormalTok{nosaukums}\OtherTok{=}\StringTok{"SoilTexture\_Silt\_r500.tif"}
\NormalTok{ielasisanas\_cels}\OtherTok{=}\FunctionTok{paste0}\NormalTok{(}\StringTok{"./RasterGrids\_100m/2024/RAW/"}\NormalTok{,nosaukums)}
\NormalTok{saglabasanas\_cels}\OtherTok{=}\FunctionTok{paste0}\NormalTok{(}\StringTok{"./RasterGrids\_100m/2024/Scaled/"}\NormalTok{,nosaukums)}
\NormalTok{slanis}\OtherTok{=}\FunctionTok{rast}\NormalTok{(ielasisanas\_cels)}
\NormalTok{videjais}\OtherTok{=}\FunctionTok{global}\NormalTok{(slanis,}\AttributeTok{fun=}\StringTok{"mean"}\NormalTok{,}\AttributeTok{na.rm=}\ConstantTok{TRUE}\NormalTok{)}
\NormalTok{centrets}\OtherTok{=}\NormalTok{slanis}\SpecialCharTok{{-}}\NormalTok{videjais[,}\DecValTok{1}\NormalTok{]}
\NormalTok{standartnovirze}\OtherTok{=}\NormalTok{terra}\SpecialCharTok{::}\FunctionTok{global}\NormalTok{(centrets,}\AttributeTok{fun=}\StringTok{"rms"}\NormalTok{,}\AttributeTok{na.rm=}\ConstantTok{TRUE}\NormalTok{)}
\NormalTok{merogots}\OtherTok{=}\NormalTok{centrets}\SpecialCharTok{/}\NormalTok{standartnovirze[,}\DecValTok{1}\NormalTok{]}
\FunctionTok{writeRaster}\NormalTok{(merogots,}
      \AttributeTok{filename=}\NormalTok{saglabasanas\_cels,}
      \AttributeTok{overwrite=}\ConstantTok{TRUE}\NormalTok{)}
\end{Highlighting}
\end{Shaded}

\section{SoilTexture\_Silt\_r1250}\label{ch06.523}

\textbf{filename:} \texttt{SoilTexture\_Silt\_r1250.tif}

\textbf{layername:} \texttt{egv\_523}

\textbf{English name:} Fractional cover of Silt Soils within the 1.25 km landscape

\textbf{Latvian name:} Augsnes granulometriskās klases ``smilšmāls un mālsmilts''
platības īpatsvars 1,25 km ainavā

\textbf{Procedure:} The cover fraction within a radius of 1250 m around the analysis grid cell is
calculated as the area-weighted sum of the \hyperref[ch06.521]{analysis cells} inside the
buffer, using the workflow \texttt{egvtools::radius\_function()}. During the calculation of the landscape metric,
inverse distance weighted (power = 2) gap filling on the output is applied
to ensure no missing values at the edges. Then the layer is rewritten to set
its name. Finally, the layer is standardised by subtracting the arithmetic
mean and dividing by the root mean squared error.

\begin{Shaded}
\begin{Highlighting}[]
\CommentTok{\# libs {-}{-}{-}{-}}
\ControlFlowTok{if}\NormalTok{(}\SpecialCharTok{!}\FunctionTok{require}\NormalTok{(terra)) \{}\FunctionTok{install.packages}\NormalTok{(}\StringTok{"terra"}\NormalTok{); }\FunctionTok{require}\NormalTok{(terra)\}}
\ControlFlowTok{if}\NormalTok{(}\SpecialCharTok{!}\FunctionTok{require}\NormalTok{(egvtools)) \{remotes}\SpecialCharTok{::}\FunctionTok{install\_github}\NormalTok{(}\StringTok{"aavotins/egvtools"}\NormalTok{); }\FunctionTok{require}\NormalTok{(egvtools)\}}

\CommentTok{\# EGVs radii {-}{-}{-}{-}}

\FunctionTok{radius\_function}\NormalTok{(}
 \AttributeTok{kvadrati\_path =} \StringTok{"./Templates/TemplateGrids/tiles/"}\NormalTok{,}
 \AttributeTok{radii\_path   =} \StringTok{"./Templates/TemplateGridPoints/tiles/"}\NormalTok{,}
 \AttributeTok{tikls100\_path =} \StringTok{"./Templates/TemplateGrids/tikls100\_sauzeme.parquet"}\NormalTok{,}
 \AttributeTok{template\_path =} \StringTok{"./Templates/TemplateRasters/LV100m\_10km.tif"}\NormalTok{,}
 \AttributeTok{input\_layers  =} \FunctionTok{c}\NormalTok{(}\StringTok{"./RasterGrids\_100m/2024/RAW/SoilTexture\_Silt\_cell.tif"}\NormalTok{),}
 \AttributeTok{layer\_prefixes =} \FunctionTok{c}\NormalTok{(}\StringTok{"SoilTexture\_Silt"}\NormalTok{),}
 \AttributeTok{output\_dir   =} \StringTok{"./RasterGrids\_100m/2024/RAW/"}\NormalTok{,}
 \AttributeTok{n\_workers   =} \DecValTok{5}\NormalTok{,}
 \AttributeTok{radii     =} \FunctionTok{c}\NormalTok{(}\StringTok{"r1250"}\NormalTok{),}
 \AttributeTok{radius\_mode  =} \StringTok{"sparse"}\NormalTok{,}
 \AttributeTok{extract\_fun  =} \StringTok{"mean"}\NormalTok{,}
 \AttributeTok{fill\_missing  =} \ConstantTok{TRUE}\NormalTok{,}
 \AttributeTok{IDW\_weight   =} \DecValTok{2}\NormalTok{,}
 \AttributeTok{future\_max\_size =} \DecValTok{5} \SpecialCharTok{*} \DecValTok{1024}\SpecialCharTok{\^{}}\DecValTok{3}\NormalTok{)}

\CommentTok{\# SoilTexture\_Silt\_r1250.tif    egv\_523}

\NormalTok{slanis}\OtherTok{=}\FunctionTok{rast}\NormalTok{(}\StringTok{"./RasterGrids\_100m/2024/RAW/SoilTexture\_Silt\_r1250.tif"}\NormalTok{)}
\FunctionTok{names}\NormalTok{(slanis)}\OtherTok{=}\StringTok{"egv\_523"}
\NormalTok{slanis2}\OtherTok{=}\FunctionTok{project}\NormalTok{(slanis,template100)}
\FunctionTok{writeRaster}\NormalTok{(slanis2,}
      \StringTok{"./RasterGrids\_100m/2024/RAW/SoilTexture\_Silt\_r1250.tif"}\NormalTok{,}
      \AttributeTok{overwrite=}\ConstantTok{TRUE}\NormalTok{)}

\CommentTok{\# standardisation {-}{-}{-}{-}}
\ControlFlowTok{if}\NormalTok{(}\SpecialCharTok{!}\FunctionTok{require}\NormalTok{(terra)) \{}\FunctionTok{install.packages}\NormalTok{(}\StringTok{"terra"}\NormalTok{); }\FunctionTok{require}\NormalTok{(terra)\}}
\ControlFlowTok{if}\NormalTok{(}\SpecialCharTok{!}\FunctionTok{require}\NormalTok{(tidyverse)) \{}\FunctionTok{install.packages}\NormalTok{(}\StringTok{"tidyverse"}\NormalTok{); }\FunctionTok{require}\NormalTok{(tidyverse)\}}

\NormalTok{nosaukums}\OtherTok{=}\StringTok{"SoilTexture\_Silt\_r1250.tif"}
\NormalTok{ielasisanas\_cels}\OtherTok{=}\FunctionTok{paste0}\NormalTok{(}\StringTok{"./RasterGrids\_100m/2024/RAW/"}\NormalTok{,nosaukums)}
\NormalTok{saglabasanas\_cels}\OtherTok{=}\FunctionTok{paste0}\NormalTok{(}\StringTok{"./RasterGrids\_100m/2024/Scaled/"}\NormalTok{,nosaukums)}
\NormalTok{slanis}\OtherTok{=}\FunctionTok{rast}\NormalTok{(ielasisanas\_cels)}
\NormalTok{videjais}\OtherTok{=}\FunctionTok{global}\NormalTok{(slanis,}\AttributeTok{fun=}\StringTok{"mean"}\NormalTok{,}\AttributeTok{na.rm=}\ConstantTok{TRUE}\NormalTok{)}
\NormalTok{centrets}\OtherTok{=}\NormalTok{slanis}\SpecialCharTok{{-}}\NormalTok{videjais[,}\DecValTok{1}\NormalTok{]}
\NormalTok{standartnovirze}\OtherTok{=}\NormalTok{terra}\SpecialCharTok{::}\FunctionTok{global}\NormalTok{(centrets,}\AttributeTok{fun=}\StringTok{"rms"}\NormalTok{,}\AttributeTok{na.rm=}\ConstantTok{TRUE}\NormalTok{)}
\NormalTok{merogots}\OtherTok{=}\NormalTok{centrets}\SpecialCharTok{/}\NormalTok{standartnovirze[,}\DecValTok{1}\NormalTok{]}
\FunctionTok{writeRaster}\NormalTok{(merogots,}
      \AttributeTok{filename=}\NormalTok{saglabasanas\_cels,}
      \AttributeTok{overwrite=}\ConstantTok{TRUE}\NormalTok{)}
\end{Highlighting}
\end{Shaded}

\section{SoilTexture\_Silt\_r3000}\label{ch06.524}

\textbf{filename:} \texttt{SoilTexture\_Silt\_r3000.tif}

\textbf{layername:} \texttt{egv\_524}

\textbf{English name:} Fractional cover of Silt Soils within the 3 km landscape

\textbf{Latvian name:} Augsnes granulometriskās klases ``smilšmāls un mālsmilts''
platības īpatsvars 3 km ainavā

\textbf{Procedure:} The cover fraction within a radius of 3000 m around the analysis grid cell is
calculated as the area-weighted sum of the \hyperref[ch06.521]{analysis cells} inside the
buffer, using the workflow \texttt{egvtools::radius\_function()}. During the calculation of the landscape metric,
inverse distance weighted (power = 2) gap filling on the output is applied
to ensure no missing values at the edges. Then the layer is rewritten to set
its name. Finally, the layer is standardised by subtracting the arithmetic
mean and dividing by the root mean squared error.

\begin{Shaded}
\begin{Highlighting}[]
\CommentTok{\# libs {-}{-}{-}{-}}
\ControlFlowTok{if}\NormalTok{(}\SpecialCharTok{!}\FunctionTok{require}\NormalTok{(terra)) \{}\FunctionTok{install.packages}\NormalTok{(}\StringTok{"terra"}\NormalTok{); }\FunctionTok{require}\NormalTok{(terra)\}}
\ControlFlowTok{if}\NormalTok{(}\SpecialCharTok{!}\FunctionTok{require}\NormalTok{(egvtools)) \{remotes}\SpecialCharTok{::}\FunctionTok{install\_github}\NormalTok{(}\StringTok{"aavotins/egvtools"}\NormalTok{); }\FunctionTok{require}\NormalTok{(egvtools)\}}

\CommentTok{\# EGVs radii {-}{-}{-}{-}}

\FunctionTok{radius\_function}\NormalTok{(}
 \AttributeTok{kvadrati\_path =} \StringTok{"./Templates/TemplateGrids/tiles/"}\NormalTok{,}
 \AttributeTok{radii\_path   =} \StringTok{"./Templates/TemplateGridPoints/tiles/"}\NormalTok{,}
 \AttributeTok{tikls100\_path =} \StringTok{"./Templates/TemplateGrids/tikls100\_sauzeme.parquet"}\NormalTok{,}
 \AttributeTok{template\_path =} \StringTok{"./Templates/TemplateRasters/LV100m\_10km.tif"}\NormalTok{,}
 \AttributeTok{input\_layers  =} \FunctionTok{c}\NormalTok{(}\StringTok{"./RasterGrids\_100m/2024/RAW/SoilTexture\_Silt\_cell.tif"}\NormalTok{),}
 \AttributeTok{layer\_prefixes =} \FunctionTok{c}\NormalTok{(}\StringTok{"SoilTexture\_Silt"}\NormalTok{),}
 \AttributeTok{output\_dir   =} \StringTok{"./RasterGrids\_100m/2024/RAW/"}\NormalTok{,}
 \AttributeTok{n\_workers   =} \DecValTok{5}\NormalTok{,}
 \AttributeTok{radii     =} \FunctionTok{c}\NormalTok{(}\StringTok{"r3000"}\NormalTok{),}
 \AttributeTok{radius\_mode  =} \StringTok{"sparse"}\NormalTok{,}
 \AttributeTok{extract\_fun  =} \StringTok{"mean"}\NormalTok{,}
 \AttributeTok{fill\_missing  =} \ConstantTok{TRUE}\NormalTok{,}
 \AttributeTok{IDW\_weight   =} \DecValTok{2}\NormalTok{,}
 \AttributeTok{future\_max\_size =} \DecValTok{5} \SpecialCharTok{*} \DecValTok{1024}\SpecialCharTok{\^{}}\DecValTok{3}\NormalTok{)}


\CommentTok{\# SoilTexture\_Silt\_r3000.tif    egv\_524}

\NormalTok{slanis}\OtherTok{=}\FunctionTok{rast}\NormalTok{(}\StringTok{"./RasterGrids\_100m/2024/RAW/SoilTexture\_Silt\_r3000.tif"}\NormalTok{)}
\FunctionTok{names}\NormalTok{(slanis)}\OtherTok{=}\StringTok{"egv\_524"}
\NormalTok{slanis2}\OtherTok{=}\FunctionTok{project}\NormalTok{(slanis,template100)}
\FunctionTok{writeRaster}\NormalTok{(slanis2,}
      \StringTok{"./RasterGrids\_100m/2024/RAW/SoilTexture\_Silt\_r3000.tif"}\NormalTok{,}
      \AttributeTok{overwrite=}\ConstantTok{TRUE}\NormalTok{)}

\CommentTok{\# standardisation {-}{-}{-}{-}}
\ControlFlowTok{if}\NormalTok{(}\SpecialCharTok{!}\FunctionTok{require}\NormalTok{(terra)) \{}\FunctionTok{install.packages}\NormalTok{(}\StringTok{"terra"}\NormalTok{); }\FunctionTok{require}\NormalTok{(terra)\}}
\ControlFlowTok{if}\NormalTok{(}\SpecialCharTok{!}\FunctionTok{require}\NormalTok{(tidyverse)) \{}\FunctionTok{install.packages}\NormalTok{(}\StringTok{"tidyverse"}\NormalTok{); }\FunctionTok{require}\NormalTok{(tidyverse)\}}

\NormalTok{nosaukums}\OtherTok{=}\StringTok{"SoilTexture\_Silt\_r3000.tif"}
\NormalTok{ielasisanas\_cels}\OtherTok{=}\FunctionTok{paste0}\NormalTok{(}\StringTok{"./RasterGrids\_100m/2024/RAW/"}\NormalTok{,nosaukums)}
\NormalTok{saglabasanas\_cels}\OtherTok{=}\FunctionTok{paste0}\NormalTok{(}\StringTok{"./RasterGrids\_100m/2024/Scaled/"}\NormalTok{,nosaukums)}
\NormalTok{slanis}\OtherTok{=}\FunctionTok{rast}\NormalTok{(ielasisanas\_cels)}
\NormalTok{videjais}\OtherTok{=}\FunctionTok{global}\NormalTok{(slanis,}\AttributeTok{fun=}\StringTok{"mean"}\NormalTok{,}\AttributeTok{na.rm=}\ConstantTok{TRUE}\NormalTok{)}
\NormalTok{centrets}\OtherTok{=}\NormalTok{slanis}\SpecialCharTok{{-}}\NormalTok{videjais[,}\DecValTok{1}\NormalTok{]}
\NormalTok{standartnovirze}\OtherTok{=}\NormalTok{terra}\SpecialCharTok{::}\FunctionTok{global}\NormalTok{(centrets,}\AttributeTok{fun=}\StringTok{"rms"}\NormalTok{,}\AttributeTok{na.rm=}\ConstantTok{TRUE}\NormalTok{)}
\NormalTok{merogots}\OtherTok{=}\NormalTok{centrets}\SpecialCharTok{/}\NormalTok{standartnovirze[,}\DecValTok{1}\NormalTok{]}
\FunctionTok{writeRaster}\NormalTok{(merogots,}
      \AttributeTok{filename=}\NormalTok{saglabasanas\_cels,}
      \AttributeTok{overwrite=}\ConstantTok{TRUE}\NormalTok{)}
\end{Highlighting}
\end{Shaded}

\section{SoilTexture\_Silt\_r10000}\label{ch06.525}

\textbf{filename:} \texttt{SoilTexture\_Silt\_r10000.tif}

\textbf{layername:} \texttt{egv\_525}

\textbf{English name:} Fractional cover of Silt Soils within the 10 km landscape

\textbf{Latvian name:} Augsnes granulometriskās klases ``smilšmāls un mālsmilts''
platības īpatsvars 10 km ainavā

\textbf{Procedure:} The cover fraction within a radius of 10000 m around the analysis grid cell is
calculated as the area-weighted sum of the \hyperref[ch06.521]{analysis cells} inside the
buffer, using the workflow \texttt{egvtools::radius\_function()}. During the calculation of the landscape metric,
inverse distance weighted (power = 2) gap filling on the output is applied
to ensure no missing values at the edges. Then the layer is rewritten to set
its name. Finally, the layer is standardised by subtracting the arithmetic
mean and dividing by the root mean squared error.

\begin{Shaded}
\begin{Highlighting}[]
\CommentTok{\# libs {-}{-}{-}{-}}
\ControlFlowTok{if}\NormalTok{(}\SpecialCharTok{!}\FunctionTok{require}\NormalTok{(terra)) \{}\FunctionTok{install.packages}\NormalTok{(}\StringTok{"terra"}\NormalTok{); }\FunctionTok{require}\NormalTok{(terra)\}}
\ControlFlowTok{if}\NormalTok{(}\SpecialCharTok{!}\FunctionTok{require}\NormalTok{(egvtools)) \{remotes}\SpecialCharTok{::}\FunctionTok{install\_github}\NormalTok{(}\StringTok{"aavotins/egvtools"}\NormalTok{); }\FunctionTok{require}\NormalTok{(egvtools)\}}

\CommentTok{\# EGVs radii {-}{-}{-}{-}}

\FunctionTok{radius\_function}\NormalTok{(}
 \AttributeTok{kvadrati\_path =} \StringTok{"./Templates/TemplateGrids/tiles/"}\NormalTok{,}
 \AttributeTok{radii\_path   =} \StringTok{"./Templates/TemplateGridPoints/tiles/"}\NormalTok{,}
 \AttributeTok{tikls100\_path =} \StringTok{"./Templates/TemplateGrids/tikls100\_sauzeme.parquet"}\NormalTok{,}
 \AttributeTok{template\_path =} \StringTok{"./Templates/TemplateRasters/LV100m\_10km.tif"}\NormalTok{,}
 \AttributeTok{input\_layers  =} \FunctionTok{c}\NormalTok{(}\StringTok{"./RasterGrids\_100m/2024/RAW/SoilTexture\_Silt\_cell.tif"}\NormalTok{),}
 \AttributeTok{layer\_prefixes =} \FunctionTok{c}\NormalTok{(}\StringTok{"SoilTexture\_Silt"}\NormalTok{),}
 \AttributeTok{output\_dir   =} \StringTok{"./RasterGrids\_100m/2024/RAW/"}\NormalTok{,}
 \AttributeTok{n\_workers   =} \DecValTok{5}\NormalTok{,}
 \AttributeTok{radii     =} \FunctionTok{c}\NormalTok{(}\StringTok{"r10000"}\NormalTok{),}
 \AttributeTok{radius\_mode  =} \StringTok{"sparse"}\NormalTok{,}
 \AttributeTok{extract\_fun  =} \StringTok{"mean"}\NormalTok{,}
 \AttributeTok{fill\_missing  =} \ConstantTok{TRUE}\NormalTok{,}
 \AttributeTok{IDW\_weight   =} \DecValTok{2}\NormalTok{,}
 \AttributeTok{future\_max\_size =} \DecValTok{5} \SpecialCharTok{*} \DecValTok{1024}\SpecialCharTok{\^{}}\DecValTok{3}\NormalTok{)}

\CommentTok{\# SoilTexture\_Silt\_r10000.tif   egv\_525}

\NormalTok{slanis}\OtherTok{=}\FunctionTok{rast}\NormalTok{(}\StringTok{"./RasterGrids\_100m/2024/RAW/SoilTexture\_Silt\_r10000.tif"}\NormalTok{)}
\FunctionTok{names}\NormalTok{(slanis)}\OtherTok{=}\StringTok{"egv\_525"}
\NormalTok{slanis2}\OtherTok{=}\FunctionTok{project}\NormalTok{(slanis,template100)}
\FunctionTok{writeRaster}\NormalTok{(slanis2,}
      \StringTok{"./RasterGrids\_100m/2024/RAW/SoilTexture\_Silt\_r10000.tif"}\NormalTok{,}
      \AttributeTok{overwrite=}\ConstantTok{TRUE}\NormalTok{)}

\CommentTok{\# standardisation {-}{-}{-}{-}}
\ControlFlowTok{if}\NormalTok{(}\SpecialCharTok{!}\FunctionTok{require}\NormalTok{(terra)) \{}\FunctionTok{install.packages}\NormalTok{(}\StringTok{"terra"}\NormalTok{); }\FunctionTok{require}\NormalTok{(terra)\}}
\ControlFlowTok{if}\NormalTok{(}\SpecialCharTok{!}\FunctionTok{require}\NormalTok{(tidyverse)) \{}\FunctionTok{install.packages}\NormalTok{(}\StringTok{"tidyverse"}\NormalTok{); }\FunctionTok{require}\NormalTok{(tidyverse)\}}

\NormalTok{nosaukums}\OtherTok{=}\StringTok{"SoilTexture\_Silt\_r10000.tif"}
\NormalTok{ielasisanas\_cels}\OtherTok{=}\FunctionTok{paste0}\NormalTok{(}\StringTok{"./RasterGrids\_100m/2024/RAW/"}\NormalTok{,nosaukums)}
\NormalTok{saglabasanas\_cels}\OtherTok{=}\FunctionTok{paste0}\NormalTok{(}\StringTok{"./RasterGrids\_100m/2024/Scaled/"}\NormalTok{,nosaukums)}
\NormalTok{slanis}\OtherTok{=}\FunctionTok{rast}\NormalTok{(ielasisanas\_cels)}
\NormalTok{videjais}\OtherTok{=}\FunctionTok{global}\NormalTok{(slanis,}\AttributeTok{fun=}\StringTok{"mean"}\NormalTok{,}\AttributeTok{na.rm=}\ConstantTok{TRUE}\NormalTok{)}
\NormalTok{centrets}\OtherTok{=}\NormalTok{slanis}\SpecialCharTok{{-}}\NormalTok{videjais[,}\DecValTok{1}\NormalTok{]}
\NormalTok{standartnovirze}\OtherTok{=}\NormalTok{terra}\SpecialCharTok{::}\FunctionTok{global}\NormalTok{(centrets,}\AttributeTok{fun=}\StringTok{"rms"}\NormalTok{,}\AttributeTok{na.rm=}\ConstantTok{TRUE}\NormalTok{)}
\NormalTok{merogots}\OtherTok{=}\NormalTok{centrets}\SpecialCharTok{/}\NormalTok{standartnovirze[,}\DecValTok{1}\NormalTok{]}
\FunctionTok{writeRaster}\NormalTok{(merogots,}
      \AttributeTok{filename=}\NormalTok{saglabasanas\_cels,}
      \AttributeTok{overwrite=}\ConstantTok{TRUE}\NormalTok{)}
\end{Highlighting}
\end{Shaded}

\section{Terrain\_ASL-average\_cell}\label{ch06.526}

\textbf{filename:} \texttt{Terrain\_ASL-average\_cell.tif}

\textbf{layername:} \texttt{egv\_526}

\textbf{English name:} Average value of height Above Sea Level (m) within the
analysis cell (1 ha)

\textbf{Latvian name:} Augstums virs jūras līmeņa (m) analīzes šūnā (1 ha)

\textbf{Procedure:} Derived from the \hyperref[Ch04.15]{Digital elevation/terrain models}.
Processed using the workflow \texttt{egvtools::input2egv()}. Inverse distance
weighted (power = 2) gap filling is implemented to protect against potential data
loss at edge cells. Finally, the layer is standardised by subtracting the
arithmetic mean and dividing by the root mean squared error.

\begin{Shaded}
\begin{Highlighting}[]
\CommentTok{\# libs {-}{-}{-}{-}}
\ControlFlowTok{if}\NormalTok{(}\SpecialCharTok{!}\FunctionTok{require}\NormalTok{(terra)) \{}\FunctionTok{install.packages}\NormalTok{(}\StringTok{"terra"}\NormalTok{); }\FunctionTok{require}\NormalTok{(terra)\}}
\ControlFlowTok{if}\NormalTok{(}\SpecialCharTok{!}\FunctionTok{require}\NormalTok{(egvtools)) \{remotes}\SpecialCharTok{::}\FunctionTok{install\_github}\NormalTok{(}\StringTok{"aavotins/egvtools"}\NormalTok{); }\FunctionTok{require}\NormalTok{(egvtools)\}}

\CommentTok{\# templates {-}{-}{-}{-}}
\NormalTok{template100}\OtherTok{=}\FunctionTok{rast}\NormalTok{(}\StringTok{"./Templates/TemplateRasters/LV100m\_10km.tif"}\NormalTok{)}

\CommentTok{\# Terrain\_ASL{-}average\_cell.tif  egv\_526}

\FunctionTok{input2egv}\NormalTok{(}\AttributeTok{input=}\StringTok{"./Geodata/2024/DEM/mozDEM\_10m.tif"}\NormalTok{,}
     \AttributeTok{egv\_template=}\StringTok{"./Templates/TemplateRasters/LV100m\_10km.tif"}\NormalTok{,}
     \AttributeTok{summary\_function =} \StringTok{"average"}\NormalTok{,}
     \AttributeTok{missing\_job =} \StringTok{"FillOutput"}\NormalTok{,}
     \AttributeTok{idw\_weight =} \DecValTok{2}\NormalTok{,}
     \AttributeTok{outlocation =} \StringTok{"./RasterGrids\_100m/2024/RAW/"}\NormalTok{,}
     \AttributeTok{outfilename =} \StringTok{"Terrain\_ASL{-}average\_cell.tif"}\NormalTok{,}
     \AttributeTok{layername=}\StringTok{"egv\_526"}\NormalTok{,}
     \AttributeTok{return\_visible =} \ConstantTok{TRUE}\NormalTok{,}
     \AttributeTok{plot\_final =} \ConstantTok{TRUE}\NormalTok{)}

\CommentTok{\# standardisation {-}{-}{-}{-}}
\ControlFlowTok{if}\NormalTok{(}\SpecialCharTok{!}\FunctionTok{require}\NormalTok{(terra)) \{}\FunctionTok{install.packages}\NormalTok{(}\StringTok{"terra"}\NormalTok{); }\FunctionTok{require}\NormalTok{(terra)\}}
\ControlFlowTok{if}\NormalTok{(}\SpecialCharTok{!}\FunctionTok{require}\NormalTok{(tidyverse)) \{}\FunctionTok{install.packages}\NormalTok{(}\StringTok{"tidyverse"}\NormalTok{); }\FunctionTok{require}\NormalTok{(tidyverse)\}}

\NormalTok{nosaukums}\OtherTok{=}\StringTok{"Terrain\_ASL{-}average\_cell.tif"}
\NormalTok{ielasisanas\_cels}\OtherTok{=}\FunctionTok{paste0}\NormalTok{(}\StringTok{"./RasterGrids\_100m/2024/RAW/"}\NormalTok{,nosaukums)}
\NormalTok{saglabasanas\_cels}\OtherTok{=}\FunctionTok{paste0}\NormalTok{(}\StringTok{"./RasterGrids\_100m/2024/Scaled/"}\NormalTok{,nosaukums)}
\NormalTok{slanis}\OtherTok{=}\FunctionTok{rast}\NormalTok{(ielasisanas\_cels)}
\NormalTok{videjais}\OtherTok{=}\FunctionTok{global}\NormalTok{(slanis,}\AttributeTok{fun=}\StringTok{"mean"}\NormalTok{,}\AttributeTok{na.rm=}\ConstantTok{TRUE}\NormalTok{)}
\NormalTok{centrets}\OtherTok{=}\NormalTok{slanis}\SpecialCharTok{{-}}\NormalTok{videjais[,}\DecValTok{1}\NormalTok{]}
\NormalTok{standartnovirze}\OtherTok{=}\NormalTok{terra}\SpecialCharTok{::}\FunctionTok{global}\NormalTok{(centrets,}\AttributeTok{fun=}\StringTok{"rms"}\NormalTok{,}\AttributeTok{na.rm=}\ConstantTok{TRUE}\NormalTok{)}
\NormalTok{merogots}\OtherTok{=}\NormalTok{centrets}\SpecialCharTok{/}\NormalTok{standartnovirze[,}\DecValTok{1}\NormalTok{]}
\FunctionTok{writeRaster}\NormalTok{(merogots,}
      \AttributeTok{filename=}\NormalTok{saglabasanas\_cels,}
      \AttributeTok{overwrite=}\ConstantTok{TRUE}\NormalTok{)}
\end{Highlighting}
\end{Shaded}

\section{Terrain\_Aspect-average\_cell}\label{ch06.527}

\textbf{filename:} \texttt{Terrain\_Aspect-average\_cell.tif}

\textbf{layername:} \texttt{egv\_527}

\textbf{English name:} Average value of Terrain Aspect (degree) within the analysis
cell (1 ha)

\textbf{Latvian name:} Nogāzes vidējais vērsuma virziens (grādi) analīzes šūnā (1 ha)

\textbf{Procedure:} Derived from the \hyperref[Ch05.01]{Terrain products}.
Processed using the workflow \texttt{egvtools::input2egv()}. Inverse distance
weighted (power = 2) gap filling is implemented to protect against potential data
loss at edge cells. Finally, the layer is standardised by subtracting the
arithmetic mean and dividing by the root mean squared error.

\begin{Shaded}
\begin{Highlighting}[]
\CommentTok{\# libs {-}{-}{-}{-}}
\ControlFlowTok{if}\NormalTok{(}\SpecialCharTok{!}\FunctionTok{require}\NormalTok{(terra)) \{}\FunctionTok{install.packages}\NormalTok{(}\StringTok{"terra"}\NormalTok{); }\FunctionTok{require}\NormalTok{(terra)\}}
\ControlFlowTok{if}\NormalTok{(}\SpecialCharTok{!}\FunctionTok{require}\NormalTok{(egvtools)) \{remotes}\SpecialCharTok{::}\FunctionTok{install\_github}\NormalTok{(}\StringTok{"aavotins/egvtools"}\NormalTok{); }\FunctionTok{require}\NormalTok{(egvtools)\}}

\CommentTok{\# templates {-}{-}{-}{-}}
\NormalTok{template100}\OtherTok{=}\FunctionTok{rast}\NormalTok{(}\StringTok{"./Templates/TemplateRasters/LV100m\_10km.tif"}\NormalTok{)}


\CommentTok{\# Terrain\_Aspect{-}average\_cell.tif   egv\_527}
\FunctionTok{input2egv}\NormalTok{(}\AttributeTok{input=}\StringTok{"./RasterGrids\_10m/2024/Terrain\_Aspect\_udeni2\_10m.tif"}\NormalTok{,}
     \AttributeTok{egv\_template=}\StringTok{"./Templates/TemplateRasters/LV100m\_10km.tif"}\NormalTok{,}
     \AttributeTok{summary\_function =} \StringTok{"average"}\NormalTok{,}
     \AttributeTok{missing\_job =} \StringTok{"FillOutput"}\NormalTok{,}
     \AttributeTok{idw\_weight =} \DecValTok{2}\NormalTok{,}
     \AttributeTok{outlocation =} \StringTok{"./RasterGrids\_100m/2024/RAW/"}\NormalTok{,}
     \AttributeTok{outfilename =} \StringTok{"Terrain\_Aspect{-}average\_cell.tif"}\NormalTok{,}
     \AttributeTok{layername=}\StringTok{"egv\_527"}\NormalTok{,}
     \AttributeTok{return\_visible =} \ConstantTok{TRUE}\NormalTok{,}
     \AttributeTok{plot\_final =} \ConstantTok{TRUE}\NormalTok{)}

\CommentTok{\# standardisation {-}{-}{-}{-}}
\ControlFlowTok{if}\NormalTok{(}\SpecialCharTok{!}\FunctionTok{require}\NormalTok{(terra)) \{}\FunctionTok{install.packages}\NormalTok{(}\StringTok{"terra"}\NormalTok{); }\FunctionTok{require}\NormalTok{(terra)\}}
\ControlFlowTok{if}\NormalTok{(}\SpecialCharTok{!}\FunctionTok{require}\NormalTok{(tidyverse)) \{}\FunctionTok{install.packages}\NormalTok{(}\StringTok{"tidyverse"}\NormalTok{); }\FunctionTok{require}\NormalTok{(tidyverse)\}}

\NormalTok{nosaukums}\OtherTok{=}\StringTok{"Terrain\_Aspect{-}average\_cell.tif"}
\NormalTok{ielasisanas\_cels}\OtherTok{=}\FunctionTok{paste0}\NormalTok{(}\StringTok{"./RasterGrids\_100m/2024/RAW/"}\NormalTok{,nosaukums)}
\NormalTok{saglabasanas\_cels}\OtherTok{=}\FunctionTok{paste0}\NormalTok{(}\StringTok{"./RasterGrids\_100m/2024/Scaled/"}\NormalTok{,nosaukums)}
\NormalTok{slanis}\OtherTok{=}\FunctionTok{rast}\NormalTok{(ielasisanas\_cels)}
\NormalTok{videjais}\OtherTok{=}\FunctionTok{global}\NormalTok{(slanis,}\AttributeTok{fun=}\StringTok{"mean"}\NormalTok{,}\AttributeTok{na.rm=}\ConstantTok{TRUE}\NormalTok{)}
\NormalTok{centrets}\OtherTok{=}\NormalTok{slanis}\SpecialCharTok{{-}}\NormalTok{videjais[,}\DecValTok{1}\NormalTok{]}
\NormalTok{standartnovirze}\OtherTok{=}\NormalTok{terra}\SpecialCharTok{::}\FunctionTok{global}\NormalTok{(centrets,}\AttributeTok{fun=}\StringTok{"rms"}\NormalTok{,}\AttributeTok{na.rm=}\ConstantTok{TRUE}\NormalTok{)}
\NormalTok{merogots}\OtherTok{=}\NormalTok{centrets}\SpecialCharTok{/}\NormalTok{standartnovirze[,}\DecValTok{1}\NormalTok{]}
\FunctionTok{writeRaster}\NormalTok{(merogots,}
      \AttributeTok{filename=}\NormalTok{saglabasanas\_cels,}
      \AttributeTok{overwrite=}\ConstantTok{TRUE}\NormalTok{)}
\end{Highlighting}
\end{Shaded}

\section{Terrain\_Aspect-iqr\_cell}\label{ch06.528}

\textbf{filename:} \texttt{Terrain\_Aspect-iqr\_cell.tif}

\textbf{layername:} \texttt{egv\_528}

\textbf{English name:} Variability of Terrain Aspect (degree) within the analysis
cell (1 ha)

\textbf{Latvian name:} Nogāzes vērsuma (grādi) variabilitāte analīzes šūnā (1 ha)

\textbf{Procedure:} Derived from the \hyperref[Ch05.01]{Terrain products}. The
workflow \texttt{egvtools::input2egv()} is used to calculate Q1 and Q3 for every cell.
To protect against potential data loss at the edges, inverse distance
weighted (power = 2) gap filling is implemented. Next, Q1 is subtracted from Q3.
Finally, the layer is standardised by subtracting the arithmetic mean and
dividing by the root mean squared error.

\begin{Shaded}
\begin{Highlighting}[]
\CommentTok{\# libs {-}{-}{-}{-}}
\ControlFlowTok{if}\NormalTok{(}\SpecialCharTok{!}\FunctionTok{require}\NormalTok{(terra)) \{}\FunctionTok{install.packages}\NormalTok{(}\StringTok{"terra"}\NormalTok{); }\FunctionTok{require}\NormalTok{(terra)\}}
\ControlFlowTok{if}\NormalTok{(}\SpecialCharTok{!}\FunctionTok{require}\NormalTok{(egvtools)) \{remotes}\SpecialCharTok{::}\FunctionTok{install\_github}\NormalTok{(}\StringTok{"aavotins/egvtools"}\NormalTok{); }\FunctionTok{require}\NormalTok{(egvtools)\}}

\CommentTok{\# templates {-}{-}{-}{-}}
\NormalTok{template100}\OtherTok{=}\FunctionTok{rast}\NormalTok{(}\StringTok{"./Templates/TemplateRasters/LV100m\_10km.tif"}\NormalTok{)}


\CommentTok{\# Terrain\_Aspect{-}iqr\_cell.tif   egv\_528}
\NormalTok{p25rez}\OtherTok{=}\FunctionTok{input2egv}\NormalTok{(}\AttributeTok{input=}\StringTok{"./RasterGrids\_10m/2024/Terrain\_Aspect\_udeni2\_10m.tif"}\NormalTok{,}
         \AttributeTok{egv\_template=} \StringTok{"./Templates/TemplateRasters/LV100m\_10km.tif"}\NormalTok{,}
         \AttributeTok{summary\_function =} \StringTok{"q1"}\NormalTok{,}
         \AttributeTok{missing\_job =} \StringTok{"FillOutput"}\NormalTok{,}
         \AttributeTok{outlocation =} \StringTok{"./RasterGrids\_100m/2024/"}\NormalTok{,}
         \AttributeTok{outfilename =} \StringTok{"draza\_p25.tif"}\NormalTok{,}
         \AttributeTok{layername =} \StringTok{"egv\_528"}\NormalTok{,}
         \AttributeTok{idw\_weight =} \DecValTok{2}\NormalTok{)}
\NormalTok{p25rez\_r}\OtherTok{=}\FunctionTok{rast}\NormalTok{(}\StringTok{"./RasterGrids\_100m/2024/draza\_p25.tif"}\NormalTok{)}


\NormalTok{p75rez}\OtherTok{=}\FunctionTok{input2egv}\NormalTok{(}\AttributeTok{input=}\StringTok{"./RasterGrids\_10m/2024/Terrain\_Aspect\_udeni2\_10m.tif"}\NormalTok{,}
         \AttributeTok{egv\_template=} \StringTok{"./Templates/TemplateRasters/LV100m\_10km.tif"}\NormalTok{,}
         \AttributeTok{summary\_function =} \StringTok{"q3"}\NormalTok{,}
         \AttributeTok{missing\_job =} \StringTok{"FillOutput"}\NormalTok{,}
         \AttributeTok{outlocation =} \StringTok{"./RasterGrids\_100m/2024/"}\NormalTok{,}
         \AttributeTok{outfilename =} \StringTok{"draza\_p75.tif"}\NormalTok{,}
         \AttributeTok{layername =} \StringTok{"egv\_528"}\NormalTok{,}
         \AttributeTok{idw\_weight =} \DecValTok{2}\NormalTok{)}
\NormalTok{p75rez\_r}\OtherTok{=}\FunctionTok{rast}\NormalTok{(}\StringTok{"./RasterGrids\_100m/2024/draza\_p75.tif"}\NormalTok{)}

\NormalTok{iqr\_rez}\OtherTok{=}\NormalTok{p75rez\_r}\SpecialCharTok{{-}}\NormalTok{p25rez\_r}
\NormalTok{iqr\_rez}
\FunctionTok{plot}\NormalTok{(iqr\_rez)}

\FunctionTok{writeRaster}\NormalTok{(iqr\_rez,}
      \StringTok{"./RasterGrids\_100m/2024/RAW/Terrain\_Aspect{-}iqr\_cell.tif"}\NormalTok{,}
      \AttributeTok{overwrite=}\ConstantTok{TRUE}\NormalTok{)}

\FunctionTok{unlink}\NormalTok{(}\StringTok{"./RasterGrids\_100m/2024/draza\_p75.tif"}\NormalTok{)}
\FunctionTok{unlink}\NormalTok{(}\StringTok{"./RasterGrids\_100m/2024/draza\_p25.tif"}\NormalTok{)}

\CommentTok{\# standardisation {-}{-}{-}{-}}
\ControlFlowTok{if}\NormalTok{(}\SpecialCharTok{!}\FunctionTok{require}\NormalTok{(terra)) \{}\FunctionTok{install.packages}\NormalTok{(}\StringTok{"terra"}\NormalTok{); }\FunctionTok{require}\NormalTok{(terra)\}}
\ControlFlowTok{if}\NormalTok{(}\SpecialCharTok{!}\FunctionTok{require}\NormalTok{(tidyverse)) \{}\FunctionTok{install.packages}\NormalTok{(}\StringTok{"tidyverse"}\NormalTok{); }\FunctionTok{require}\NormalTok{(tidyverse)\}}

\NormalTok{nosaukums}\OtherTok{=}\StringTok{"Terrain\_Aspect{-}iqr\_cell.tif"}
\NormalTok{ielasisanas\_cels}\OtherTok{=}\FunctionTok{paste0}\NormalTok{(}\StringTok{"./RasterGrids\_100m/2024/RAW/"}\NormalTok{,nosaukums)}
\NormalTok{saglabasanas\_cels}\OtherTok{=}\FunctionTok{paste0}\NormalTok{(}\StringTok{"./RasterGrids\_100m/2024/Scaled/"}\NormalTok{,nosaukums)}
\NormalTok{slanis}\OtherTok{=}\FunctionTok{rast}\NormalTok{(ielasisanas\_cels)}
\NormalTok{videjais}\OtherTok{=}\FunctionTok{global}\NormalTok{(slanis,}\AttributeTok{fun=}\StringTok{"mean"}\NormalTok{,}\AttributeTok{na.rm=}\ConstantTok{TRUE}\NormalTok{)}
\NormalTok{centrets}\OtherTok{=}\NormalTok{slanis}\SpecialCharTok{{-}}\NormalTok{videjais[,}\DecValTok{1}\NormalTok{]}
\NormalTok{standartnovirze}\OtherTok{=}\NormalTok{terra}\SpecialCharTok{::}\FunctionTok{global}\NormalTok{(centrets,}\AttributeTok{fun=}\StringTok{"rms"}\NormalTok{,}\AttributeTok{na.rm=}\ConstantTok{TRUE}\NormalTok{)}
\NormalTok{merogots}\OtherTok{=}\NormalTok{centrets}\SpecialCharTok{/}\NormalTok{standartnovirze[,}\DecValTok{1}\NormalTok{]}
\FunctionTok{writeRaster}\NormalTok{(merogots,}
      \AttributeTok{filename=}\NormalTok{saglabasanas\_cels,}
      \AttributeTok{overwrite=}\ConstantTok{TRUE}\NormalTok{)}
\end{Highlighting}
\end{Shaded}

\section{Terrain\_DiS-area\_cell}\label{ch06.529}

\textbf{filename:} \texttt{Terrain\_DiS-area\_cell.tif}

\textbf{layername:} \texttt{egv\_529}

\textbf{English name:} Fractional cover of Terrain Sinks within the analysis cell (1
ha)

\textbf{Latvian name:} Reljefa depresiju bez virszemes noteces platības īpatsvars
analīzes šūnā (1 ha)

\textbf{Procedure:} Derived from the \hyperref[Ch05.01]{Terrain products} depth-in-sinks
layer, which is reclassified to a value of 1 in every cell with a positive value.
The resulting layer
is then aggregated to EGV resolution using the workflow \texttt{egvtools::input2egv()}, which
calculates the arithmetic mean to determine the cover fraction. During
aggregation, inverse distance weighted (power = 2) gap filling on the output is
applied to ensure no missing values at the edges. Finally, the layer is
standardised by subtracting the arithmetic mean and dividing by the root mean squared
error.

\begin{Shaded}
\begin{Highlighting}[]
\CommentTok{\# libs {-}{-}{-}{-}}
\ControlFlowTok{if}\NormalTok{(}\SpecialCharTok{!}\FunctionTok{require}\NormalTok{(terra)) \{}\FunctionTok{install.packages}\NormalTok{(}\StringTok{"terra"}\NormalTok{); }\FunctionTok{require}\NormalTok{(terra)\}}
\ControlFlowTok{if}\NormalTok{(}\SpecialCharTok{!}\FunctionTok{require}\NormalTok{(egvtools)) \{remotes}\SpecialCharTok{::}\FunctionTok{install\_github}\NormalTok{(}\StringTok{"aavotins/egvtools"}\NormalTok{); }\FunctionTok{require}\NormalTok{(egvtools)\}}

\CommentTok{\# templates {-}{-}{-}{-}}
\NormalTok{template100}\OtherTok{=}\FunctionTok{rast}\NormalTok{(}\StringTok{"./Templates/TemplateRasters/LV100m\_10km.tif"}\NormalTok{)}


\CommentTok{\# Terrain\_DiS{-}area\_cell.tif egv\_529}
\NormalTok{dis}\OtherTok{=}\FunctionTok{rast}\NormalTok{(}\StringTok{"./RasterGrids\_10m/2024/Terrain\_DiS\_udeni2\_10m.tif"}\NormalTok{)}
\NormalTok{dis2}\OtherTok{=}\FunctionTok{ifel}\NormalTok{(dis}\SpecialCharTok{\textgreater{}}\DecValTok{0}\NormalTok{,}\DecValTok{1}\NormalTok{,dis)}

\FunctionTok{input2egv}\NormalTok{(}\AttributeTok{input=}\NormalTok{dis2,}
     \AttributeTok{egv\_template=}\StringTok{"./Templates/TemplateRasters/LV100m\_10km.tif"}\NormalTok{,}
     \AttributeTok{summary\_function =} \StringTok{"average"}\NormalTok{,}
     \AttributeTok{missing\_job =} \StringTok{"FillOutput"}\NormalTok{,}
     \AttributeTok{idw\_weight =} \DecValTok{2}\NormalTok{,}
     \AttributeTok{outlocation =} \StringTok{"./RasterGrids\_100m/2024/RAW/"}\NormalTok{,}
     \AttributeTok{outfilename =} \StringTok{"Terrain\_DiS{-}area\_cell.tif"}\NormalTok{,}
     \AttributeTok{layername=}\StringTok{"egv\_529"}\NormalTok{,}
     \AttributeTok{return\_visible =} \ConstantTok{TRUE}\NormalTok{,}
     \AttributeTok{plot\_final =} \ConstantTok{TRUE}\NormalTok{)}

\CommentTok{\# standardisation {-}{-}{-}{-}}
\ControlFlowTok{if}\NormalTok{(}\SpecialCharTok{!}\FunctionTok{require}\NormalTok{(terra)) \{}\FunctionTok{install.packages}\NormalTok{(}\StringTok{"terra"}\NormalTok{); }\FunctionTok{require}\NormalTok{(terra)\}}
\ControlFlowTok{if}\NormalTok{(}\SpecialCharTok{!}\FunctionTok{require}\NormalTok{(tidyverse)) \{}\FunctionTok{install.packages}\NormalTok{(}\StringTok{"tidyverse"}\NormalTok{); }\FunctionTok{require}\NormalTok{(tidyverse)\}}

\NormalTok{nosaukums}\OtherTok{=}\StringTok{"Terrain\_DiS{-}area\_cell.tif"}
\NormalTok{ielasisanas\_cels}\OtherTok{=}\FunctionTok{paste0}\NormalTok{(}\StringTok{"./RasterGrids\_100m/2024/RAW/"}\NormalTok{,nosaukums)}
\NormalTok{saglabasanas\_cels}\OtherTok{=}\FunctionTok{paste0}\NormalTok{(}\StringTok{"./RasterGrids\_100m/2024/Scaled/"}\NormalTok{,nosaukums)}
\NormalTok{slanis}\OtherTok{=}\FunctionTok{rast}\NormalTok{(ielasisanas\_cels)}
\NormalTok{videjais}\OtherTok{=}\FunctionTok{global}\NormalTok{(slanis,}\AttributeTok{fun=}\StringTok{"mean"}\NormalTok{,}\AttributeTok{na.rm=}\ConstantTok{TRUE}\NormalTok{)}
\NormalTok{centrets}\OtherTok{=}\NormalTok{slanis}\SpecialCharTok{{-}}\NormalTok{videjais[,}\DecValTok{1}\NormalTok{]}
\NormalTok{standartnovirze}\OtherTok{=}\NormalTok{terra}\SpecialCharTok{::}\FunctionTok{global}\NormalTok{(centrets,}\AttributeTok{fun=}\StringTok{"rms"}\NormalTok{,}\AttributeTok{na.rm=}\ConstantTok{TRUE}\NormalTok{)}
\NormalTok{merogots}\OtherTok{=}\NormalTok{centrets}\SpecialCharTok{/}\NormalTok{standartnovirze[,}\DecValTok{1}\NormalTok{]}
\FunctionTok{writeRaster}\NormalTok{(merogots,}
      \AttributeTok{filename=}\NormalTok{saglabasanas\_cels,}
      \AttributeTok{overwrite=}\ConstantTok{TRUE}\NormalTok{)}
\end{Highlighting}
\end{Shaded}

\section{Terrain\_DiS-area\_r500}\label{ch06.530}

\textbf{filename:} \texttt{Terrain\_DiS-area\_r500.tif}

\textbf{layername:} \texttt{egv\_530}

\textbf{English name:} Fractional cover of Terrain Sinks within the 0.5 km landscape

\textbf{Latvian name:} Reljefa depresiju bez virszemes noteces platības īpatsvars 0,5
km ainavā

\textbf{Procedure:} The cover fraction within a radius of 500 m around the analysis grid cell is
calculated as the area-weighted sum of the \hyperref[ch06.529]{analysis cells} inside the
buffer, using the workflow \texttt{egvtools::radius\_function()}. During the calculation of the landscape metric,
inverse distance weighted (power = 2) gap filling on the output is applied
to ensure no missing values at the edges. Then the layer is rewritten to set
its name. Finally, the layer is standardised by subtracting the arithmetic
mean and dividing by the root mean squared error.

\begin{Shaded}
\begin{Highlighting}[]
\CommentTok{\# libs {-}{-}{-}{-}}
\ControlFlowTok{if}\NormalTok{(}\SpecialCharTok{!}\FunctionTok{require}\NormalTok{(terra)) \{}\FunctionTok{install.packages}\NormalTok{(}\StringTok{"terra"}\NormalTok{); }\FunctionTok{require}\NormalTok{(terra)\}}
\ControlFlowTok{if}\NormalTok{(}\SpecialCharTok{!}\FunctionTok{require}\NormalTok{(egvtools)) \{remotes}\SpecialCharTok{::}\FunctionTok{install\_github}\NormalTok{(}\StringTok{"aavotins/egvtools"}\NormalTok{); }\FunctionTok{require}\NormalTok{(egvtools)\}}

\CommentTok{\# templates {-}{-}{-}{-}}
\NormalTok{template100}\OtherTok{=}\FunctionTok{rast}\NormalTok{(}\StringTok{"./Templates/TemplateRasters/LV100m\_10km.tif"}\NormalTok{)}


\CommentTok{\# radii}
\FunctionTok{radius\_function}\NormalTok{(}
 \AttributeTok{kvadrati\_path =} \StringTok{"./Templates/TemplateGrids/tiles/"}\NormalTok{,}
 \AttributeTok{radii\_path   =} \StringTok{"./Templates/TemplateGridPoints/tiles/"}\NormalTok{,}
 \AttributeTok{tikls100\_path =} \StringTok{"./Templates/TemplateGrids/tikls100\_sauzeme.parquet"}\NormalTok{,}
 \AttributeTok{template\_path =} \StringTok{"./Templates/TemplateRasters/LV100m\_10km.tif"}\NormalTok{,}
 \AttributeTok{input\_layers  =} \FunctionTok{c}\NormalTok{(}\StringTok{"./RasterGrids\_100m/2024/RAW/Terrain\_DiS{-}area\_cell.tif"}\NormalTok{),}
 \AttributeTok{layer\_prefixes =} \FunctionTok{c}\NormalTok{(}\StringTok{"Terrain\_DiS{-}area"}\NormalTok{),}
 \AttributeTok{output\_dir   =} \StringTok{"./RasterGrids\_100m/2024/RAW/"}\NormalTok{,}
 \AttributeTok{n\_workers   =} \DecValTok{5}\NormalTok{,}
 \AttributeTok{radii     =} \FunctionTok{c}\NormalTok{(}\StringTok{"r500"}\NormalTok{),}
 \AttributeTok{radius\_mode  =} \StringTok{"sparse"}\NormalTok{,}
 \AttributeTok{extract\_fun  =} \StringTok{"mean"}\NormalTok{,}
 \AttributeTok{fill\_missing  =} \ConstantTok{TRUE}\NormalTok{,}
 \AttributeTok{IDW\_weight   =} \DecValTok{2}\NormalTok{,}
 \AttributeTok{future\_max\_size =} \DecValTok{5} \SpecialCharTok{*} \DecValTok{1024}\SpecialCharTok{\^{}}\DecValTok{3}\NormalTok{)}


\CommentTok{\# Terrain\_DiS{-}area\_r500.tif egv\_530}
\NormalTok{slanis}\OtherTok{=}\FunctionTok{rast}\NormalTok{(}\StringTok{"./RasterGrids\_100m/2024/RAW/Terrain\_DiS{-}area\_r500.tif"}\NormalTok{)}
\FunctionTok{names}\NormalTok{(slanis)}\OtherTok{=}\StringTok{"egv\_530"}
\NormalTok{slanis2}\OtherTok{=}\FunctionTok{project}\NormalTok{(slanis,template100)}
\FunctionTok{writeRaster}\NormalTok{(slanis2,}
      \StringTok{"./RasterGrids\_100m/2024/RAW/Terrain\_DiS{-}area\_r500.tif"}\NormalTok{,}
      \AttributeTok{overwrite=}\ConstantTok{TRUE}\NormalTok{)}

\CommentTok{\# standardisation {-}{-}{-}{-}}
\ControlFlowTok{if}\NormalTok{(}\SpecialCharTok{!}\FunctionTok{require}\NormalTok{(terra)) \{}\FunctionTok{install.packages}\NormalTok{(}\StringTok{"terra"}\NormalTok{); }\FunctionTok{require}\NormalTok{(terra)\}}
\ControlFlowTok{if}\NormalTok{(}\SpecialCharTok{!}\FunctionTok{require}\NormalTok{(tidyverse)) \{}\FunctionTok{install.packages}\NormalTok{(}\StringTok{"tidyverse"}\NormalTok{); }\FunctionTok{require}\NormalTok{(tidyverse)\}}

\NormalTok{nosaukums}\OtherTok{=}\StringTok{"Terrain\_DiS{-}area\_r500.tif"}
\NormalTok{ielasisanas\_cels}\OtherTok{=}\FunctionTok{paste0}\NormalTok{(}\StringTok{"./RasterGrids\_100m/2024/RAW/"}\NormalTok{,nosaukums)}
\NormalTok{saglabasanas\_cels}\OtherTok{=}\FunctionTok{paste0}\NormalTok{(}\StringTok{"./RasterGrids\_100m/2024/Scaled/"}\NormalTok{,nosaukums)}
\NormalTok{slanis}\OtherTok{=}\FunctionTok{rast}\NormalTok{(ielasisanas\_cels)}
\NormalTok{videjais}\OtherTok{=}\FunctionTok{global}\NormalTok{(slanis,}\AttributeTok{fun=}\StringTok{"mean"}\NormalTok{,}\AttributeTok{na.rm=}\ConstantTok{TRUE}\NormalTok{)}
\NormalTok{centrets}\OtherTok{=}\NormalTok{slanis}\SpecialCharTok{{-}}\NormalTok{videjais[,}\DecValTok{1}\NormalTok{]}
\NormalTok{standartnovirze}\OtherTok{=}\NormalTok{terra}\SpecialCharTok{::}\FunctionTok{global}\NormalTok{(centrets,}\AttributeTok{fun=}\StringTok{"rms"}\NormalTok{,}\AttributeTok{na.rm=}\ConstantTok{TRUE}\NormalTok{)}
\NormalTok{merogots}\OtherTok{=}\NormalTok{centrets}\SpecialCharTok{/}\NormalTok{standartnovirze[,}\DecValTok{1}\NormalTok{]}
\FunctionTok{writeRaster}\NormalTok{(merogots,}
      \AttributeTok{filename=}\NormalTok{saglabasanas\_cels,}
      \AttributeTok{overwrite=}\ConstantTok{TRUE}\NormalTok{)}
\end{Highlighting}
\end{Shaded}

\section{Terrain\_DiS-area\_r1250}\label{ch06.531}

\textbf{filename:} \texttt{Terrain\_DiS-area\_r1250.tif}

\textbf{layername:} \texttt{egv\_531}

\textbf{English name:} Fractional cover of Terrain Sinks within the 1.25 km landscape

\textbf{Latvian name:} Reljefa depresiju bez virszemes noteces platības īpatsvars
1,25 km ainavā

\textbf{Procedure:} The cover fraction within a radius of 1250 m around the analysis grid cell is
calculated as the area-weighted sum of the \hyperref[ch06.529]{analysis cells} inside the
buffer, using the workflow \texttt{egvtools::radius\_function()}. During the calculation of the landscape metric,
inverse distance weighted (power = 2) gap filling on the output is applied
to ensure no missing values at the edges. Then the layer is rewritten to set
its name. Finally, the layer is standardised by subtracting the arithmetic
mean and dividing by the root mean squared error.

\begin{Shaded}
\begin{Highlighting}[]
\CommentTok{\# libs {-}{-}{-}{-}}
\ControlFlowTok{if}\NormalTok{(}\SpecialCharTok{!}\FunctionTok{require}\NormalTok{(terra)) \{}\FunctionTok{install.packages}\NormalTok{(}\StringTok{"terra"}\NormalTok{); }\FunctionTok{require}\NormalTok{(terra)\}}
\ControlFlowTok{if}\NormalTok{(}\SpecialCharTok{!}\FunctionTok{require}\NormalTok{(egvtools)) \{remotes}\SpecialCharTok{::}\FunctionTok{install\_github}\NormalTok{(}\StringTok{"aavotins/egvtools"}\NormalTok{); }\FunctionTok{require}\NormalTok{(egvtools)\}}

\CommentTok{\# templates {-}{-}{-}{-}}
\NormalTok{template100}\OtherTok{=}\FunctionTok{rast}\NormalTok{(}\StringTok{"./Templates/TemplateRasters/LV100m\_10km.tif"}\NormalTok{)}


\CommentTok{\# radii}
\FunctionTok{radius\_function}\NormalTok{(}
 \AttributeTok{kvadrati\_path =} \StringTok{"./Templates/TemplateGrids/tiles/"}\NormalTok{,}
 \AttributeTok{radii\_path   =} \StringTok{"./Templates/TemplateGridPoints/tiles/"}\NormalTok{,}
 \AttributeTok{tikls100\_path =} \StringTok{"./Templates/TemplateGrids/tikls100\_sauzeme.parquet"}\NormalTok{,}
 \AttributeTok{template\_path =} \StringTok{"./Templates/TemplateRasters/LV100m\_10km.tif"}\NormalTok{,}
 \AttributeTok{input\_layers  =} \FunctionTok{c}\NormalTok{(}\StringTok{"./RasterGrids\_100m/2024/RAW/Terrain\_DiS{-}area\_cell.tif"}\NormalTok{),}
 \AttributeTok{layer\_prefixes =} \FunctionTok{c}\NormalTok{(}\StringTok{"Terrain\_DiS{-}area"}\NormalTok{),}
 \AttributeTok{output\_dir   =} \StringTok{"./RasterGrids\_100m/2024/RAW/"}\NormalTok{,}
 \AttributeTok{n\_workers   =} \DecValTok{5}\NormalTok{,}
 \AttributeTok{radii     =} \FunctionTok{c}\NormalTok{(}\StringTok{"r1250"}\NormalTok{),}
 \AttributeTok{radius\_mode  =} \StringTok{"sparse"}\NormalTok{,}
 \AttributeTok{extract\_fun  =} \StringTok{"mean"}\NormalTok{,}
 \AttributeTok{fill\_missing  =} \ConstantTok{TRUE}\NormalTok{,}
 \AttributeTok{IDW\_weight   =} \DecValTok{2}\NormalTok{,}
 \AttributeTok{future\_max\_size =} \DecValTok{5} \SpecialCharTok{*} \DecValTok{1024}\SpecialCharTok{\^{}}\DecValTok{3}\NormalTok{)}

\CommentTok{\# Terrain\_DiS{-}area\_r1250.tif    egv\_531}
\NormalTok{slanis}\OtherTok{=}\FunctionTok{rast}\NormalTok{(}\StringTok{"./RasterGrids\_100m/2024/RAW/Terrain\_DiS{-}area\_r1250.tif"}\NormalTok{)}
\FunctionTok{names}\NormalTok{(slanis)}\OtherTok{=}\StringTok{"egv\_531"}
\NormalTok{slanis2}\OtherTok{=}\FunctionTok{project}\NormalTok{(slanis,template100)}
\FunctionTok{writeRaster}\NormalTok{(slanis2,}
      \StringTok{"./RasterGrids\_100m/2024/RAW/Terrain\_DiS{-}area\_r1250.tif"}\NormalTok{,}
      \AttributeTok{overwrite=}\ConstantTok{TRUE}\NormalTok{)}

\CommentTok{\# standardisation {-}{-}{-}{-}}
\ControlFlowTok{if}\NormalTok{(}\SpecialCharTok{!}\FunctionTok{require}\NormalTok{(terra)) \{}\FunctionTok{install.packages}\NormalTok{(}\StringTok{"terra"}\NormalTok{); }\FunctionTok{require}\NormalTok{(terra)\}}
\ControlFlowTok{if}\NormalTok{(}\SpecialCharTok{!}\FunctionTok{require}\NormalTok{(tidyverse)) \{}\FunctionTok{install.packages}\NormalTok{(}\StringTok{"tidyverse"}\NormalTok{); }\FunctionTok{require}\NormalTok{(tidyverse)\}}

\NormalTok{nosaukums}\OtherTok{=}\StringTok{"Terrain\_DiS{-}area\_r1250.tif"}
\NormalTok{ielasisanas\_cels}\OtherTok{=}\FunctionTok{paste0}\NormalTok{(}\StringTok{"./RasterGrids\_100m/2024/RAW/"}\NormalTok{,nosaukums)}
\NormalTok{saglabasanas\_cels}\OtherTok{=}\FunctionTok{paste0}\NormalTok{(}\StringTok{"./RasterGrids\_100m/2024/Scaled/"}\NormalTok{,nosaukums)}
\NormalTok{slanis}\OtherTok{=}\FunctionTok{rast}\NormalTok{(ielasisanas\_cels)}
\NormalTok{videjais}\OtherTok{=}\FunctionTok{global}\NormalTok{(slanis,}\AttributeTok{fun=}\StringTok{"mean"}\NormalTok{,}\AttributeTok{na.rm=}\ConstantTok{TRUE}\NormalTok{)}
\NormalTok{centrets}\OtherTok{=}\NormalTok{slanis}\SpecialCharTok{{-}}\NormalTok{videjais[,}\DecValTok{1}\NormalTok{]}
\NormalTok{standartnovirze}\OtherTok{=}\NormalTok{terra}\SpecialCharTok{::}\FunctionTok{global}\NormalTok{(centrets,}\AttributeTok{fun=}\StringTok{"rms"}\NormalTok{,}\AttributeTok{na.rm=}\ConstantTok{TRUE}\NormalTok{)}
\NormalTok{merogots}\OtherTok{=}\NormalTok{centrets}\SpecialCharTok{/}\NormalTok{standartnovirze[,}\DecValTok{1}\NormalTok{]}
\FunctionTok{writeRaster}\NormalTok{(merogots,}
      \AttributeTok{filename=}\NormalTok{saglabasanas\_cels,}
      \AttributeTok{overwrite=}\ConstantTok{TRUE}\NormalTok{)}
\end{Highlighting}
\end{Shaded}

\section{Terrain\_DiS-area\_r3000}\label{ch06.532}

\textbf{filename:} \texttt{Terrain\_DiS-area\_r3000.tif}

\textbf{layername:} \texttt{egv\_532}

\textbf{English name:} Fractional cover of Terrain Sinks within the 3 km landscape

\textbf{Latvian name:} Reljefa depresiju bez virszemes noteces platības īpatsvars 3
km ainavā

\textbf{Procedure:} The cover fraction within a radius of 3000 m around the analysis grid cell is
calculated as the area-weighted sum of the \hyperref[ch06.529]{analysis cells} inside the
buffer, using the workflow \texttt{egvtools::radius\_function()}. During the calculation of the landscape metric,
inverse distance weighted (power = 2) gap filling on the output is applied
to ensure no missing values at the edges. Then the layer is rewritten to set
its name. Finally, the layer is standardised by subtracting the arithmetic
mean and dividing by the root mean squared error.

\begin{Shaded}
\begin{Highlighting}[]
\CommentTok{\# libs {-}{-}{-}{-}}
\ControlFlowTok{if}\NormalTok{(}\SpecialCharTok{!}\FunctionTok{require}\NormalTok{(terra)) \{}\FunctionTok{install.packages}\NormalTok{(}\StringTok{"terra"}\NormalTok{); }\FunctionTok{require}\NormalTok{(terra)\}}
\ControlFlowTok{if}\NormalTok{(}\SpecialCharTok{!}\FunctionTok{require}\NormalTok{(egvtools)) \{remotes}\SpecialCharTok{::}\FunctionTok{install\_github}\NormalTok{(}\StringTok{"aavotins/egvtools"}\NormalTok{); }\FunctionTok{require}\NormalTok{(egvtools)\}}

\CommentTok{\# templates {-}{-}{-}{-}}
\NormalTok{template100}\OtherTok{=}\FunctionTok{rast}\NormalTok{(}\StringTok{"./Templates/TemplateRasters/LV100m\_10km.tif"}\NormalTok{)}


\CommentTok{\# radii}
\FunctionTok{radius\_function}\NormalTok{(}
 \AttributeTok{kvadrati\_path =} \StringTok{"./Templates/TemplateGrids/tiles/"}\NormalTok{,}
 \AttributeTok{radii\_path   =} \StringTok{"./Templates/TemplateGridPoints/tiles/"}\NormalTok{,}
 \AttributeTok{tikls100\_path =} \StringTok{"./Templates/TemplateGrids/tikls100\_sauzeme.parquet"}\NormalTok{,}
 \AttributeTok{template\_path =} \StringTok{"./Templates/TemplateRasters/LV100m\_10km.tif"}\NormalTok{,}
 \AttributeTok{input\_layers  =} \FunctionTok{c}\NormalTok{(}\StringTok{"./RasterGrids\_100m/2024/RAW/Terrain\_DiS{-}area\_cell.tif"}\NormalTok{),}
 \AttributeTok{layer\_prefixes =} \FunctionTok{c}\NormalTok{(}\StringTok{"Terrain\_DiS{-}area"}\NormalTok{),}
 \AttributeTok{output\_dir   =} \StringTok{"./RasterGrids\_100m/2024/RAW/"}\NormalTok{,}
 \AttributeTok{n\_workers   =} \DecValTok{5}\NormalTok{,}
 \AttributeTok{radii     =} \FunctionTok{c}\NormalTok{(}\StringTok{"r3000"}\NormalTok{),}
 \AttributeTok{radius\_mode  =} \StringTok{"sparse"}\NormalTok{,}
 \AttributeTok{extract\_fun  =} \StringTok{"mean"}\NormalTok{,}
 \AttributeTok{fill\_missing  =} \ConstantTok{TRUE}\NormalTok{,}
 \AttributeTok{IDW\_weight   =} \DecValTok{2}\NormalTok{,}
 \AttributeTok{future\_max\_size =} \DecValTok{5} \SpecialCharTok{*} \DecValTok{1024}\SpecialCharTok{\^{}}\DecValTok{3}\NormalTok{)}


\CommentTok{\# Terrain\_DiS{-}area\_r3000.tif    egv\_532}
\NormalTok{slanis}\OtherTok{=}\FunctionTok{rast}\NormalTok{(}\StringTok{"./RasterGrids\_100m/2024/RAW/Terrain\_DiS{-}area\_r3000.tif"}\NormalTok{)}
\FunctionTok{names}\NormalTok{(slanis)}\OtherTok{=}\StringTok{"egv\_532"}
\NormalTok{slanis2}\OtherTok{=}\FunctionTok{project}\NormalTok{(slanis,template100)}
\FunctionTok{writeRaster}\NormalTok{(slanis2,}
      \StringTok{"./RasterGrids\_100m/2024/RAW/Terrain\_DiS{-}area\_r3000.tif"}\NormalTok{,}
      \AttributeTok{overwrite=}\ConstantTok{TRUE}\NormalTok{)}

\CommentTok{\# standardisation {-}{-}{-}{-}}
\ControlFlowTok{if}\NormalTok{(}\SpecialCharTok{!}\FunctionTok{require}\NormalTok{(terra)) \{}\FunctionTok{install.packages}\NormalTok{(}\StringTok{"terra"}\NormalTok{); }\FunctionTok{require}\NormalTok{(terra)\}}
\ControlFlowTok{if}\NormalTok{(}\SpecialCharTok{!}\FunctionTok{require}\NormalTok{(tidyverse)) \{}\FunctionTok{install.packages}\NormalTok{(}\StringTok{"tidyverse"}\NormalTok{); }\FunctionTok{require}\NormalTok{(tidyverse)\}}

\NormalTok{nosaukums}\OtherTok{=}\StringTok{"Terrain\_DiS{-}area\_r3000.tif"}
\NormalTok{ielasisanas\_cels}\OtherTok{=}\FunctionTok{paste0}\NormalTok{(}\StringTok{"./RasterGrids\_100m/2024/RAW/"}\NormalTok{,nosaukums)}
\NormalTok{saglabasanas\_cels}\OtherTok{=}\FunctionTok{paste0}\NormalTok{(}\StringTok{"./RasterGrids\_100m/2024/Scaled/"}\NormalTok{,nosaukums)}
\NormalTok{slanis}\OtherTok{=}\FunctionTok{rast}\NormalTok{(ielasisanas\_cels)}
\NormalTok{videjais}\OtherTok{=}\FunctionTok{global}\NormalTok{(slanis,}\AttributeTok{fun=}\StringTok{"mean"}\NormalTok{,}\AttributeTok{na.rm=}\ConstantTok{TRUE}\NormalTok{)}
\NormalTok{centrets}\OtherTok{=}\NormalTok{slanis}\SpecialCharTok{{-}}\NormalTok{videjais[,}\DecValTok{1}\NormalTok{]}
\NormalTok{standartnovirze}\OtherTok{=}\NormalTok{terra}\SpecialCharTok{::}\FunctionTok{global}\NormalTok{(centrets,}\AttributeTok{fun=}\StringTok{"rms"}\NormalTok{,}\AttributeTok{na.rm=}\ConstantTok{TRUE}\NormalTok{)}
\NormalTok{merogots}\OtherTok{=}\NormalTok{centrets}\SpecialCharTok{/}\NormalTok{standartnovirze[,}\DecValTok{1}\NormalTok{]}
\FunctionTok{writeRaster}\NormalTok{(merogots,}
      \AttributeTok{filename=}\NormalTok{saglabasanas\_cels,}
      \AttributeTok{overwrite=}\ConstantTok{TRUE}\NormalTok{)}
\end{Highlighting}
\end{Shaded}

\section{Terrain\_DiS-area\_r10000}\label{ch06.533}

\textbf{filename:} \texttt{Terrain\_DiS-area\_r10000.tif}

\textbf{layername:} \texttt{egv\_533}

\textbf{English name:} Fractional cover of Terrain Sinks within the 10 km landscape

\textbf{Latvian name:} Reljefa depresiju bez virszemes noteces platības īpatsvars 10
km ainavā

\textbf{Procedure:} The cover fraction within a radius of 10000 m around the analysis grid cell is
calculated as the area-weighted sum of the \hyperref[ch06.529]{analysis cells} inside the
buffer, using the workflow \texttt{egvtools::radius\_function()}. During the calculation of the landscape metric,
inverse distance weighted (power = 2) gap filling on the output is applied
to ensure no missing values at the edges. Then the layer is rewritten to set
its name. Finally, the layer is standardised by subtracting the arithmetic
mean and dividing by the root mean squared error.

\begin{Shaded}
\begin{Highlighting}[]
\CommentTok{\# libs {-}{-}{-}{-}}
\ControlFlowTok{if}\NormalTok{(}\SpecialCharTok{!}\FunctionTok{require}\NormalTok{(terra)) \{}\FunctionTok{install.packages}\NormalTok{(}\StringTok{"terra"}\NormalTok{); }\FunctionTok{require}\NormalTok{(terra)\}}
\ControlFlowTok{if}\NormalTok{(}\SpecialCharTok{!}\FunctionTok{require}\NormalTok{(egvtools)) \{remotes}\SpecialCharTok{::}\FunctionTok{install\_github}\NormalTok{(}\StringTok{"aavotins/egvtools"}\NormalTok{); }\FunctionTok{require}\NormalTok{(egvtools)\}}

\CommentTok{\# templates {-}{-}{-}{-}}
\NormalTok{template100}\OtherTok{=}\FunctionTok{rast}\NormalTok{(}\StringTok{"./Templates/TemplateRasters/LV100m\_10km.tif"}\NormalTok{)}


\CommentTok{\# radii}
\FunctionTok{radius\_function}\NormalTok{(}
 \AttributeTok{kvadrati\_path =} \StringTok{"./Templates/TemplateGrids/tiles/"}\NormalTok{,}
 \AttributeTok{radii\_path   =} \StringTok{"./Templates/TemplateGridPoints/tiles/"}\NormalTok{,}
 \AttributeTok{tikls100\_path =} \StringTok{"./Templates/TemplateGrids/tikls100\_sauzeme.parquet"}\NormalTok{,}
 \AttributeTok{template\_path =} \StringTok{"./Templates/TemplateRasters/LV100m\_10km.tif"}\NormalTok{,}
 \AttributeTok{input\_layers  =} \FunctionTok{c}\NormalTok{(}\StringTok{"./RasterGrids\_100m/2024/RAW/Terrain\_DiS{-}area\_cell.tif"}\NormalTok{),}
 \AttributeTok{layer\_prefixes =} \FunctionTok{c}\NormalTok{(}\StringTok{"Terrain\_DiS{-}area"}\NormalTok{),}
 \AttributeTok{output\_dir   =} \StringTok{"./RasterGrids\_100m/2024/RAW/"}\NormalTok{,}
 \AttributeTok{n\_workers   =} \DecValTok{5}\NormalTok{,}
 \AttributeTok{radii     =} \FunctionTok{c}\NormalTok{(}\StringTok{"r10000"}\NormalTok{),}
 \AttributeTok{radius\_mode  =} \StringTok{"sparse"}\NormalTok{,}
 \AttributeTok{extract\_fun  =} \StringTok{"mean"}\NormalTok{,}
 \AttributeTok{fill\_missing  =} \ConstantTok{TRUE}\NormalTok{,}
 \AttributeTok{IDW\_weight   =} \DecValTok{2}\NormalTok{,}
 \AttributeTok{future\_max\_size =} \DecValTok{5} \SpecialCharTok{*} \DecValTok{1024}\SpecialCharTok{\^{}}\DecValTok{3}\NormalTok{)}


\CommentTok{\# Terrain\_DiS{-}area\_r10000.tif   egv\_533}
\NormalTok{slanis}\OtherTok{=}\FunctionTok{rast}\NormalTok{(}\StringTok{"./RasterGrids\_100m/2024/RAW/Terrain\_DiS{-}area\_r10000.tif"}\NormalTok{)}
\FunctionTok{names}\NormalTok{(slanis)}\OtherTok{=}\StringTok{"egv\_533"}
\NormalTok{slanis2}\OtherTok{=}\FunctionTok{project}\NormalTok{(slanis,template100)}
\FunctionTok{writeRaster}\NormalTok{(slanis2,}
      \StringTok{"./RasterGrids\_100m/2024/RAW/Terrain\_DiS{-}area\_r10000.tif"}\NormalTok{,}
      \AttributeTok{overwrite=}\ConstantTok{TRUE}\NormalTok{)}

\CommentTok{\# standardisation {-}{-}{-}{-}}
\ControlFlowTok{if}\NormalTok{(}\SpecialCharTok{!}\FunctionTok{require}\NormalTok{(terra)) \{}\FunctionTok{install.packages}\NormalTok{(}\StringTok{"terra"}\NormalTok{); }\FunctionTok{require}\NormalTok{(terra)\}}
\ControlFlowTok{if}\NormalTok{(}\SpecialCharTok{!}\FunctionTok{require}\NormalTok{(tidyverse)) \{}\FunctionTok{install.packages}\NormalTok{(}\StringTok{"tidyverse"}\NormalTok{); }\FunctionTok{require}\NormalTok{(tidyverse)\}}

\NormalTok{nosaukums}\OtherTok{=}\StringTok{"Terrain\_DiS{-}area\_r10000.tif"}
\NormalTok{ielasisanas\_cels}\OtherTok{=}\FunctionTok{paste0}\NormalTok{(}\StringTok{"./RasterGrids\_100m/2024/RAW/"}\NormalTok{,nosaukums)}
\NormalTok{saglabasanas\_cels}\OtherTok{=}\FunctionTok{paste0}\NormalTok{(}\StringTok{"./RasterGrids\_100m/2024/Scaled/"}\NormalTok{,nosaukums)}
\NormalTok{slanis}\OtherTok{=}\FunctionTok{rast}\NormalTok{(ielasisanas\_cels)}
\NormalTok{videjais}\OtherTok{=}\FunctionTok{global}\NormalTok{(slanis,}\AttributeTok{fun=}\StringTok{"mean"}\NormalTok{,}\AttributeTok{na.rm=}\ConstantTok{TRUE}\NormalTok{)}
\NormalTok{centrets}\OtherTok{=}\NormalTok{slanis}\SpecialCharTok{{-}}\NormalTok{videjais[,}\DecValTok{1}\NormalTok{]}
\NormalTok{standartnovirze}\OtherTok{=}\NormalTok{terra}\SpecialCharTok{::}\FunctionTok{global}\NormalTok{(centrets,}\AttributeTok{fun=}\StringTok{"rms"}\NormalTok{,}\AttributeTok{na.rm=}\ConstantTok{TRUE}\NormalTok{)}
\NormalTok{merogots}\OtherTok{=}\NormalTok{centrets}\SpecialCharTok{/}\NormalTok{standartnovirze[,}\DecValTok{1}\NormalTok{]}
\FunctionTok{writeRaster}\NormalTok{(merogots,}
      \AttributeTok{filename=}\NormalTok{saglabasanas\_cels,}
      \AttributeTok{overwrite=}\ConstantTok{TRUE}\NormalTok{)}
\end{Highlighting}
\end{Shaded}

\section{Terrain\_DiS-max\_cell}\label{ch06.534}

\textbf{filename:} \texttt{Terrain\_DiS-max\_cell.tif}

\textbf{layername:} \texttt{egv\_534}

\textbf{English name:} Maximum Depth in Terrain Sink within the analysis cell (1 ha)

\textbf{Latvian name:} Reljefa depresiju lielākais dziļums analīzes šūnā (1 ha)

\textbf{Procedure:} Derived from the \hyperref[Ch05.01]{Terrain products}.
Processed using the workflow \texttt{egvtools::input2egv()}. Inverse distance
weighted (power = 2) gap filling is implemented to protect against potential data
loss at edge cells. Finally, the layer is standardised by subtracting the
arithmetic mean and dividing by the root mean squared error.

\begin{Shaded}
\begin{Highlighting}[]
\CommentTok{\# libs {-}{-}{-}{-}}
\ControlFlowTok{if}\NormalTok{(}\SpecialCharTok{!}\FunctionTok{require}\NormalTok{(terra)) \{}\FunctionTok{install.packages}\NormalTok{(}\StringTok{"terra"}\NormalTok{); }\FunctionTok{require}\NormalTok{(terra)\}}
\ControlFlowTok{if}\NormalTok{(}\SpecialCharTok{!}\FunctionTok{require}\NormalTok{(egvtools)) \{remotes}\SpecialCharTok{::}\FunctionTok{install\_github}\NormalTok{(}\StringTok{"aavotins/egvtools"}\NormalTok{); }\FunctionTok{require}\NormalTok{(egvtools)\}}

\CommentTok{\# templates {-}{-}{-}{-}}
\NormalTok{template100}\OtherTok{=}\FunctionTok{rast}\NormalTok{(}\StringTok{"./Templates/TemplateRasters/LV100m\_10km.tif"}\NormalTok{)}


\CommentTok{\# Terrain\_DiS{-}max\_cell.tif  egv\_534}
\FunctionTok{input2egv}\NormalTok{(}\AttributeTok{input=}\StringTok{"./RasterGrids\_10m/2024/Terrain\_DiS\_udeni2\_10m.tif"}\NormalTok{,}
     \AttributeTok{egv\_template=}\StringTok{"./Templates/TemplateRasters/LV100m\_10km.tif"}\NormalTok{,}
     \AttributeTok{summary\_function =} \StringTok{"max"}\NormalTok{,}
     \AttributeTok{missing\_job =} \StringTok{"FillOutput"}\NormalTok{,}
     \AttributeTok{idw\_weight =} \DecValTok{2}\NormalTok{,}
     \AttributeTok{outlocation =} \StringTok{"./RasterGrids\_100m/2024/RAW/"}\NormalTok{,}
     \AttributeTok{outfilename =} \StringTok{"Terrain\_DiS{-}max\_cell.tif"}\NormalTok{,}
     \AttributeTok{layername=}\StringTok{"egv\_534"}\NormalTok{,}
     \AttributeTok{return\_visible =} \ConstantTok{TRUE}\NormalTok{,}
     \AttributeTok{plot\_final =} \ConstantTok{TRUE}\NormalTok{)}

\CommentTok{\# standardisation {-}{-}{-}{-}}
\ControlFlowTok{if}\NormalTok{(}\SpecialCharTok{!}\FunctionTok{require}\NormalTok{(terra)) \{}\FunctionTok{install.packages}\NormalTok{(}\StringTok{"terra"}\NormalTok{); }\FunctionTok{require}\NormalTok{(terra)\}}
\ControlFlowTok{if}\NormalTok{(}\SpecialCharTok{!}\FunctionTok{require}\NormalTok{(tidyverse)) \{}\FunctionTok{install.packages}\NormalTok{(}\StringTok{"tidyverse"}\NormalTok{); }\FunctionTok{require}\NormalTok{(tidyverse)\}}

\NormalTok{nosaukums}\OtherTok{=}\StringTok{"Terrain\_DiS{-}max\_cell.tif"}
\NormalTok{ielasisanas\_cels}\OtherTok{=}\FunctionTok{paste0}\NormalTok{(}\StringTok{"./RasterGrids\_100m/2024/RAW/"}\NormalTok{,nosaukums)}
\NormalTok{saglabasanas\_cels}\OtherTok{=}\FunctionTok{paste0}\NormalTok{(}\StringTok{"./RasterGrids\_100m/2024/Scaled/"}\NormalTok{,nosaukums)}
\NormalTok{slanis}\OtherTok{=}\FunctionTok{rast}\NormalTok{(ielasisanas\_cels)}
\NormalTok{videjais}\OtherTok{=}\FunctionTok{global}\NormalTok{(slanis,}\AttributeTok{fun=}\StringTok{"mean"}\NormalTok{,}\AttributeTok{na.rm=}\ConstantTok{TRUE}\NormalTok{)}
\NormalTok{centrets}\OtherTok{=}\NormalTok{slanis}\SpecialCharTok{{-}}\NormalTok{videjais[,}\DecValTok{1}\NormalTok{]}
\NormalTok{standartnovirze}\OtherTok{=}\NormalTok{terra}\SpecialCharTok{::}\FunctionTok{global}\NormalTok{(centrets,}\AttributeTok{fun=}\StringTok{"rms"}\NormalTok{,}\AttributeTok{na.rm=}\ConstantTok{TRUE}\NormalTok{)}
\NormalTok{merogots}\OtherTok{=}\NormalTok{centrets}\SpecialCharTok{/}\NormalTok{standartnovirze[,}\DecValTok{1}\NormalTok{]}
\FunctionTok{writeRaster}\NormalTok{(merogots,}
      \AttributeTok{filename=}\NormalTok{saglabasanas\_cels,}
      \AttributeTok{overwrite=}\ConstantTok{TRUE}\NormalTok{)}
\end{Highlighting}
\end{Shaded}

\section{Terrain\_DiS-mean\_cell}\label{ch06.535}

\textbf{filename:} \texttt{Terrain\_DiS-mean\_cell.tif}

\textbf{layername:} \texttt{egv\_535}

\textbf{English name:} Average Depth in Terrain Sink within the analysis cell (1 ha)

\textbf{Latvian name:} Reljefa depresiju vidējais dziļums analīzes šūnā (1 ha)

\textbf{Procedure:} Derived from the \hyperref[Ch05.01]{Terrain products}.
Processed using the workflow \texttt{egvtools::input2egv()}. Inverse distance
weighted (power = 2) gap filling is implemented to protect against potential data
loss at edge cells. Finally, the layer is standardised by subtracting the
arithmetic mean and dividing by the root mean squared error.

\begin{Shaded}
\begin{Highlighting}[]
\CommentTok{\# libs {-}{-}{-}{-}}
\ControlFlowTok{if}\NormalTok{(}\SpecialCharTok{!}\FunctionTok{require}\NormalTok{(terra)) \{}\FunctionTok{install.packages}\NormalTok{(}\StringTok{"terra"}\NormalTok{); }\FunctionTok{require}\NormalTok{(terra)\}}
\ControlFlowTok{if}\NormalTok{(}\SpecialCharTok{!}\FunctionTok{require}\NormalTok{(egvtools)) \{remotes}\SpecialCharTok{::}\FunctionTok{install\_github}\NormalTok{(}\StringTok{"aavotins/egvtools"}\NormalTok{); }\FunctionTok{require}\NormalTok{(egvtools)\}}

\CommentTok{\# templates {-}{-}{-}{-}}
\NormalTok{template100}\OtherTok{=}\FunctionTok{rast}\NormalTok{(}\StringTok{"./Templates/TemplateRasters/LV100m\_10km.tif"}\NormalTok{)}


\CommentTok{\# Terrain\_DiS{-}mean\_cell.tif egv\_535}
\FunctionTok{input2egv}\NormalTok{(}\AttributeTok{input=}\StringTok{"./RasterGrids\_10m/2024/Terrain\_DiS\_udeni2\_10m.tif"}\NormalTok{,}
     \AttributeTok{egv\_template=}\StringTok{"./Templates/TemplateRasters/LV100m\_10km.tif"}\NormalTok{,}
     \AttributeTok{summary\_function =} \StringTok{"average"}\NormalTok{,}
     \AttributeTok{missing\_job =} \StringTok{"FillOutput"}\NormalTok{,}
     \AttributeTok{idw\_weight =} \DecValTok{2}\NormalTok{,}
     \AttributeTok{outlocation =} \StringTok{"./RasterGrids\_100m/2024/RAW/"}\NormalTok{,}
     \AttributeTok{outfilename =} \StringTok{"Terrain\_DiS{-}mean\_cell.tif"}\NormalTok{,}
     \AttributeTok{layername=}\StringTok{"egv\_535"}\NormalTok{,}
     \AttributeTok{return\_visible =} \ConstantTok{TRUE}\NormalTok{,}
     \AttributeTok{plot\_final =} \ConstantTok{TRUE}\NormalTok{)}

\CommentTok{\# standardisation {-}{-}{-}{-}}
\ControlFlowTok{if}\NormalTok{(}\SpecialCharTok{!}\FunctionTok{require}\NormalTok{(terra)) \{}\FunctionTok{install.packages}\NormalTok{(}\StringTok{"terra"}\NormalTok{); }\FunctionTok{require}\NormalTok{(terra)\}}
\ControlFlowTok{if}\NormalTok{(}\SpecialCharTok{!}\FunctionTok{require}\NormalTok{(tidyverse)) \{}\FunctionTok{install.packages}\NormalTok{(}\StringTok{"tidyverse"}\NormalTok{); }\FunctionTok{require}\NormalTok{(tidyverse)\}}

\NormalTok{nosaukums}\OtherTok{=}\StringTok{"Terrain\_DiS{-}mean\_cell.tif"}
\NormalTok{ielasisanas\_cels}\OtherTok{=}\FunctionTok{paste0}\NormalTok{(}\StringTok{"./RasterGrids\_100m/2024/RAW/"}\NormalTok{,nosaukums)}
\NormalTok{saglabasanas\_cels}\OtherTok{=}\FunctionTok{paste0}\NormalTok{(}\StringTok{"./RasterGrids\_100m/2024/Scaled/"}\NormalTok{,nosaukums)}
\NormalTok{slanis}\OtherTok{=}\FunctionTok{rast}\NormalTok{(ielasisanas\_cels)}
\NormalTok{videjais}\OtherTok{=}\FunctionTok{global}\NormalTok{(slanis,}\AttributeTok{fun=}\StringTok{"mean"}\NormalTok{,}\AttributeTok{na.rm=}\ConstantTok{TRUE}\NormalTok{)}
\NormalTok{centrets}\OtherTok{=}\NormalTok{slanis}\SpecialCharTok{{-}}\NormalTok{videjais[,}\DecValTok{1}\NormalTok{]}
\NormalTok{standartnovirze}\OtherTok{=}\NormalTok{terra}\SpecialCharTok{::}\FunctionTok{global}\NormalTok{(centrets,}\AttributeTok{fun=}\StringTok{"rms"}\NormalTok{,}\AttributeTok{na.rm=}\ConstantTok{TRUE}\NormalTok{)}
\NormalTok{merogots}\OtherTok{=}\NormalTok{centrets}\SpecialCharTok{/}\NormalTok{standartnovirze[,}\DecValTok{1}\NormalTok{]}
\FunctionTok{writeRaster}\NormalTok{(merogots,}
      \AttributeTok{filename=}\NormalTok{saglabasanas\_cels,}
      \AttributeTok{overwrite=}\ConstantTok{TRUE}\NormalTok{)}
\end{Highlighting}
\end{Shaded}

\section{Terrain\_Slope-average\_cell}\label{ch06.536}

\textbf{filename:} \texttt{Terrain\_Slope-average\_cell.tif}

\textbf{layername:} \texttt{egv\_536}

\textbf{English name:} Average value of Terrain Slope (degree) within the analysis
cell (1 ha)

\textbf{Latvian name:} Nogāzes slīpuma (grādi) vidējā vērtība analīzes šūnā (1 ha)

\textbf{Procedure:} Derived from the \hyperref[Ch05.01]{Terrain products}.
Processed using the workflow \texttt{egvtools::input2egv()}. Inverse distance
weighted (power = 2) gap filling is implemented to protect against potential data
loss at edge cells. Finally, the layer is standardised by subtracting the
arithmetic mean and dividing by the root mean squared error.

\begin{Shaded}
\begin{Highlighting}[]
\CommentTok{\# libs {-}{-}{-}{-}}
\ControlFlowTok{if}\NormalTok{(}\SpecialCharTok{!}\FunctionTok{require}\NormalTok{(terra)) \{}\FunctionTok{install.packages}\NormalTok{(}\StringTok{"terra"}\NormalTok{); }\FunctionTok{require}\NormalTok{(terra)\}}
\ControlFlowTok{if}\NormalTok{(}\SpecialCharTok{!}\FunctionTok{require}\NormalTok{(egvtools)) \{remotes}\SpecialCharTok{::}\FunctionTok{install\_github}\NormalTok{(}\StringTok{"aavotins/egvtools"}\NormalTok{); }\FunctionTok{require}\NormalTok{(egvtools)\}}

\CommentTok{\# templates {-}{-}{-}{-}}
\NormalTok{template100}\OtherTok{=}\FunctionTok{rast}\NormalTok{(}\StringTok{"./Templates/TemplateRasters/LV100m\_10km.tif"}\NormalTok{)}


\CommentTok{\# Terrain\_Slope{-}average\_cell.tif    egv\_536}
\FunctionTok{input2egv}\NormalTok{(}\AttributeTok{input=}\StringTok{"./RasterGrids\_10m/2024/Terrain\_Slope\_udeni2\_10m.tif"}\NormalTok{,}
     \AttributeTok{egv\_template=}\StringTok{"./Templates/TemplateRasters/LV100m\_10km.tif"}\NormalTok{,}
     \AttributeTok{summary\_function =} \StringTok{"average"}\NormalTok{,}
     \AttributeTok{missing\_job =} \StringTok{"FillOutput"}\NormalTok{,}
     \AttributeTok{idw\_weight =} \DecValTok{2}\NormalTok{,}
     \AttributeTok{outlocation =} \StringTok{"./RasterGrids\_100m/2024/RAW/"}\NormalTok{,}
     \AttributeTok{outfilename =} \StringTok{"Terrain\_Slope{-}average\_cell.tif"}\NormalTok{,}
     \AttributeTok{layername=}\StringTok{"egv\_536"}\NormalTok{,}
     \AttributeTok{return\_visible =} \ConstantTok{TRUE}\NormalTok{,}
     \AttributeTok{plot\_final =} \ConstantTok{TRUE}\NormalTok{)}

\CommentTok{\# standardisation {-}{-}{-}{-}}
\ControlFlowTok{if}\NormalTok{(}\SpecialCharTok{!}\FunctionTok{require}\NormalTok{(terra)) \{}\FunctionTok{install.packages}\NormalTok{(}\StringTok{"terra"}\NormalTok{); }\FunctionTok{require}\NormalTok{(terra)\}}
\ControlFlowTok{if}\NormalTok{(}\SpecialCharTok{!}\FunctionTok{require}\NormalTok{(tidyverse)) \{}\FunctionTok{install.packages}\NormalTok{(}\StringTok{"tidyverse"}\NormalTok{); }\FunctionTok{require}\NormalTok{(tidyverse)\}}

\NormalTok{nosaukums}\OtherTok{=}\StringTok{"Terrain\_Slope{-}average\_cell.tif"}
\NormalTok{ielasisanas\_cels}\OtherTok{=}\FunctionTok{paste0}\NormalTok{(}\StringTok{"./RasterGrids\_100m/2024/RAW/"}\NormalTok{,nosaukums)}
\NormalTok{saglabasanas\_cels}\OtherTok{=}\FunctionTok{paste0}\NormalTok{(}\StringTok{"./RasterGrids\_100m/2024/Scaled/"}\NormalTok{,nosaukums)}
\NormalTok{slanis}\OtherTok{=}\FunctionTok{rast}\NormalTok{(ielasisanas\_cels)}
\NormalTok{videjais}\OtherTok{=}\FunctionTok{global}\NormalTok{(slanis,}\AttributeTok{fun=}\StringTok{"mean"}\NormalTok{,}\AttributeTok{na.rm=}\ConstantTok{TRUE}\NormalTok{)}
\NormalTok{centrets}\OtherTok{=}\NormalTok{slanis}\SpecialCharTok{{-}}\NormalTok{videjais[,}\DecValTok{1}\NormalTok{]}
\NormalTok{standartnovirze}\OtherTok{=}\NormalTok{terra}\SpecialCharTok{::}\FunctionTok{global}\NormalTok{(centrets,}\AttributeTok{fun=}\StringTok{"rms"}\NormalTok{,}\AttributeTok{na.rm=}\ConstantTok{TRUE}\NormalTok{)}
\NormalTok{merogots}\OtherTok{=}\NormalTok{centrets}\SpecialCharTok{/}\NormalTok{standartnovirze[,}\DecValTok{1}\NormalTok{]}
\FunctionTok{writeRaster}\NormalTok{(merogots,}
      \AttributeTok{filename=}\NormalTok{saglabasanas\_cels,}
      \AttributeTok{overwrite=}\ConstantTok{TRUE}\NormalTok{)}
\end{Highlighting}
\end{Shaded}

\section{Terrain\_Slope-iqr\_cell}\label{ch06.537}

\textbf{filename:} \texttt{Terrain\_Slope-iqr\_cell.tif}

\textbf{layername:} \texttt{egv\_537}

\textbf{English name:} Variability of Terrain Slope (degree) within the analysis cell
(1 ha)

\textbf{Latvian name:} Nogāzes slīpuma (grādi) variabilitāte analīzes šūnā (1 ha)

\textbf{Procedure:} Derived from the \hyperref[Ch05.01]{Terrain products}. The
workflow \texttt{egvtools::input2egv()} is used to calculate Q1 and Q3 for every cell.
To protect against potential data loss at the edges, inverse distance
weighted (power = 2) gap filling is implemented. Next, Q1 is subtracted from Q3.
Finally, the layer is standardised by subtracting the arithmetic mean and
dividing by the root mean squared error.

\begin{Shaded}
\begin{Highlighting}[]
\CommentTok{\# libs {-}{-}{-}{-}}
\ControlFlowTok{if}\NormalTok{(}\SpecialCharTok{!}\FunctionTok{require}\NormalTok{(terra)) \{}\FunctionTok{install.packages}\NormalTok{(}\StringTok{"terra"}\NormalTok{); }\FunctionTok{require}\NormalTok{(terra)\}}
\ControlFlowTok{if}\NormalTok{(}\SpecialCharTok{!}\FunctionTok{require}\NormalTok{(egvtools)) \{remotes}\SpecialCharTok{::}\FunctionTok{install\_github}\NormalTok{(}\StringTok{"aavotins/egvtools"}\NormalTok{); }\FunctionTok{require}\NormalTok{(egvtools)\}}

\CommentTok{\# templates {-}{-}{-}{-}}
\NormalTok{template100}\OtherTok{=}\FunctionTok{rast}\NormalTok{(}\StringTok{"./Templates/TemplateRasters/LV100m\_10km.tif"}\NormalTok{)}


\CommentTok{\# Terrain\_Slope{-}iqr\_cell.tif    egv\_537}
\NormalTok{p25rez}\OtherTok{=}\FunctionTok{input2egv}\NormalTok{(}\AttributeTok{input=}\StringTok{"./RasterGrids\_10m/2024/Terrain\_Slope\_udeni2\_10m.tif"}\NormalTok{,}
         \AttributeTok{egv\_template=} \StringTok{"./Templates/TemplateRasters/LV100m\_10km.tif"}\NormalTok{,}
         \AttributeTok{summary\_function =} \StringTok{"q1"}\NormalTok{,}
         \AttributeTok{missing\_job =} \StringTok{"FillOutput"}\NormalTok{,}
         \AttributeTok{outlocation =} \StringTok{"./RasterGrids\_100m/2024/"}\NormalTok{,}
         \AttributeTok{outfilename =} \StringTok{"draza\_p25.tif"}\NormalTok{,}
         \AttributeTok{layername =} \StringTok{"egv\_537"}\NormalTok{,}
         \AttributeTok{idw\_weight =} \DecValTok{2}\NormalTok{)}
\NormalTok{p25rez\_r}\OtherTok{=}\FunctionTok{rast}\NormalTok{(}\StringTok{"./RasterGrids\_100m/2024/draza\_p25.tif"}\NormalTok{)}


\NormalTok{p75rez}\OtherTok{=}\FunctionTok{input2egv}\NormalTok{(}\AttributeTok{input=}\StringTok{"./RasterGrids\_10m/2024/Terrain\_Slope\_udeni2\_10m.tif"}\NormalTok{,}
         \AttributeTok{egv\_template=} \StringTok{"./Templates/TemplateRasters/LV100m\_10km.tif"}\NormalTok{,}
         \AttributeTok{summary\_function =} \StringTok{"q3"}\NormalTok{,}
         \AttributeTok{missing\_job =} \StringTok{"FillOutput"}\NormalTok{,}
         \AttributeTok{outlocation =} \StringTok{"./RasterGrids\_100m/2024/"}\NormalTok{,}
         \AttributeTok{outfilename =} \StringTok{"draza\_p75.tif"}\NormalTok{,}
         \AttributeTok{layername =} \StringTok{"egv\_537"}\NormalTok{,}
         \AttributeTok{idw\_weight =} \DecValTok{2}\NormalTok{)}
\NormalTok{p75rez\_r}\OtherTok{=}\FunctionTok{rast}\NormalTok{(}\StringTok{"./RasterGrids\_100m/2024/draza\_p75.tif"}\NormalTok{)}

\NormalTok{iqr\_rez}\OtherTok{=}\NormalTok{p75rez\_r}\SpecialCharTok{{-}}\NormalTok{p25rez\_r}
\NormalTok{iqr\_rez}
\FunctionTok{plot}\NormalTok{(iqr\_rez)}

\FunctionTok{writeRaster}\NormalTok{(iqr\_rez,}
      \StringTok{"./RasterGrids\_100m/2024/RAW/Terrain\_Slope{-}iqr\_cell.tif"}\NormalTok{,}
      \AttributeTok{overwrite=}\ConstantTok{TRUE}\NormalTok{)}

\FunctionTok{unlink}\NormalTok{(}\StringTok{"./RasterGrids\_100m/2024/draza\_p75.tif"}\NormalTok{)}
\FunctionTok{unlink}\NormalTok{(}\StringTok{"./RasterGrids\_100m/2024/draza\_p25.tif"}\NormalTok{)}

\CommentTok{\# standardisation {-}{-}{-}{-}}
\ControlFlowTok{if}\NormalTok{(}\SpecialCharTok{!}\FunctionTok{require}\NormalTok{(terra)) \{}\FunctionTok{install.packages}\NormalTok{(}\StringTok{"terra"}\NormalTok{); }\FunctionTok{require}\NormalTok{(terra)\}}
\ControlFlowTok{if}\NormalTok{(}\SpecialCharTok{!}\FunctionTok{require}\NormalTok{(tidyverse)) \{}\FunctionTok{install.packages}\NormalTok{(}\StringTok{"tidyverse"}\NormalTok{); }\FunctionTok{require}\NormalTok{(tidyverse)\}}

\NormalTok{nosaukums}\OtherTok{=}\StringTok{"Terrain\_Slope{-}iqr\_cell.tif"}
\NormalTok{ielasisanas\_cels}\OtherTok{=}\FunctionTok{paste0}\NormalTok{(}\StringTok{"./RasterGrids\_100m/2024/RAW/"}\NormalTok{,nosaukums)}
\NormalTok{saglabasanas\_cels}\OtherTok{=}\FunctionTok{paste0}\NormalTok{(}\StringTok{"./RasterGrids\_100m/2024/Scaled/"}\NormalTok{,nosaukums)}
\NormalTok{slanis}\OtherTok{=}\FunctionTok{rast}\NormalTok{(ielasisanas\_cels)}
\NormalTok{videjais}\OtherTok{=}\FunctionTok{global}\NormalTok{(slanis,}\AttributeTok{fun=}\StringTok{"mean"}\NormalTok{,}\AttributeTok{na.rm=}\ConstantTok{TRUE}\NormalTok{)}
\NormalTok{centrets}\OtherTok{=}\NormalTok{slanis}\SpecialCharTok{{-}}\NormalTok{videjais[,}\DecValTok{1}\NormalTok{]}
\NormalTok{standartnovirze}\OtherTok{=}\NormalTok{terra}\SpecialCharTok{::}\FunctionTok{global}\NormalTok{(centrets,}\AttributeTok{fun=}\StringTok{"rms"}\NormalTok{,}\AttributeTok{na.rm=}\ConstantTok{TRUE}\NormalTok{)}
\NormalTok{merogots}\OtherTok{=}\NormalTok{centrets}\SpecialCharTok{/}\NormalTok{standartnovirze[,}\DecValTok{1}\NormalTok{]}
\FunctionTok{writeRaster}\NormalTok{(merogots,}
      \AttributeTok{filename=}\NormalTok{saglabasanas\_cels,}
      \AttributeTok{overwrite=}\ConstantTok{TRUE}\NormalTok{)}
\end{Highlighting}
\end{Shaded}

\section{Terrain\_TWI-average\_cell}\label{ch06.538}

\textbf{filename:} \texttt{Terrain\_TWI-average\_cell.tif}

\textbf{layername:} \texttt{egv\_538}

\textbf{English name:} Average value of Topographic Wetness Index (TWI) within the
analysis cell (1 ha)

\textbf{Latvian name:} Topogrāfiskā mitruma indeksa (TWI) vidējā vērtība analīzes šūnā (1
ha)

\textbf{Procedure:} Derived from the \hyperref[Ch05.01]{Terrain products}.
Processed using the workflow \texttt{egvtools::input2egv()}. Inverse distance
weighted (power = 2) gap filling is implemented to protect against potential data
loss at edge cells. Finally, the layer is standardised by subtracting the
arithmetic mean and dividing by the root mean squared error.

\begin{Shaded}
\begin{Highlighting}[]
\CommentTok{\# libs {-}{-}{-}{-}}
\ControlFlowTok{if}\NormalTok{(}\SpecialCharTok{!}\FunctionTok{require}\NormalTok{(terra)) \{}\FunctionTok{install.packages}\NormalTok{(}\StringTok{"terra"}\NormalTok{); }\FunctionTok{require}\NormalTok{(terra)\}}
\ControlFlowTok{if}\NormalTok{(}\SpecialCharTok{!}\FunctionTok{require}\NormalTok{(egvtools)) \{remotes}\SpecialCharTok{::}\FunctionTok{install\_github}\NormalTok{(}\StringTok{"aavotins/egvtools"}\NormalTok{); }\FunctionTok{require}\NormalTok{(egvtools)\}}

\CommentTok{\# templates {-}{-}{-}{-}}
\NormalTok{template100}\OtherTok{=}\FunctionTok{rast}\NormalTok{(}\StringTok{"./Templates/TemplateRasters/LV100m\_10km.tif"}\NormalTok{)}


\CommentTok{\# Terrain\_TWI{-}average\_cell.tif  egv\_538}
\FunctionTok{input2egv}\NormalTok{(}\AttributeTok{input=}\StringTok{"./RasterGrids\_10m/2024/Terrain\_TWI\_udeni2\_10m.tif"}\NormalTok{,}
     \AttributeTok{egv\_template=}\StringTok{"./Templates/TemplateRasters/LV100m\_10km.tif"}\NormalTok{,}
     \AttributeTok{summary\_function =} \StringTok{"average"}\NormalTok{,}
     \AttributeTok{missing\_job =} \StringTok{"FillOutput"}\NormalTok{,}
     \AttributeTok{idw\_weight =} \DecValTok{2}\NormalTok{,}
     \AttributeTok{outlocation =} \StringTok{"./RasterGrids\_100m/2024/RAW/"}\NormalTok{,}
     \AttributeTok{outfilename =} \StringTok{"Terrain\_TWI{-}average\_cell.tif"}\NormalTok{,}
     \AttributeTok{layername=}\StringTok{"egv\_538"}\NormalTok{,}
     \AttributeTok{return\_visible =} \ConstantTok{TRUE}\NormalTok{,}
     \AttributeTok{plot\_final =} \ConstantTok{TRUE}\NormalTok{)}

\CommentTok{\# standardisation {-}{-}{-}{-}}
\ControlFlowTok{if}\NormalTok{(}\SpecialCharTok{!}\FunctionTok{require}\NormalTok{(terra)) \{}\FunctionTok{install.packages}\NormalTok{(}\StringTok{"terra"}\NormalTok{); }\FunctionTok{require}\NormalTok{(terra)\}}
\ControlFlowTok{if}\NormalTok{(}\SpecialCharTok{!}\FunctionTok{require}\NormalTok{(tidyverse)) \{}\FunctionTok{install.packages}\NormalTok{(}\StringTok{"tidyverse"}\NormalTok{); }\FunctionTok{require}\NormalTok{(tidyverse)\}}

\NormalTok{nosaukums}\OtherTok{=}\StringTok{"Terrain\_TWI{-}average\_cell.tif"}
\NormalTok{ielasisanas\_cels}\OtherTok{=}\FunctionTok{paste0}\NormalTok{(}\StringTok{"./RasterGrids\_100m/2024/RAW/"}\NormalTok{,nosaukums)}
\NormalTok{saglabasanas\_cels}\OtherTok{=}\FunctionTok{paste0}\NormalTok{(}\StringTok{"./RasterGrids\_100m/2024/Scaled/"}\NormalTok{,nosaukums)}
\NormalTok{slanis}\OtherTok{=}\FunctionTok{rast}\NormalTok{(ielasisanas\_cels)}
\NormalTok{videjais}\OtherTok{=}\FunctionTok{global}\NormalTok{(slanis,}\AttributeTok{fun=}\StringTok{"mean"}\NormalTok{,}\AttributeTok{na.rm=}\ConstantTok{TRUE}\NormalTok{)}
\NormalTok{centrets}\OtherTok{=}\NormalTok{slanis}\SpecialCharTok{{-}}\NormalTok{videjais[,}\DecValTok{1}\NormalTok{]}
\NormalTok{standartnovirze}\OtherTok{=}\NormalTok{terra}\SpecialCharTok{::}\FunctionTok{global}\NormalTok{(centrets,}\AttributeTok{fun=}\StringTok{"rms"}\NormalTok{,}\AttributeTok{na.rm=}\ConstantTok{TRUE}\NormalTok{)}
\NormalTok{merogots}\OtherTok{=}\NormalTok{centrets}\SpecialCharTok{/}\NormalTok{standartnovirze[,}\DecValTok{1}\NormalTok{]}
\FunctionTok{writeRaster}\NormalTok{(merogots,}
      \AttributeTok{filename=}\NormalTok{saglabasanas\_cels,}
      \AttributeTok{overwrite=}\ConstantTok{TRUE}\NormalTok{)}
\end{Highlighting}
\end{Shaded}

\chapter{Data access}\label{Ch07}

When using code or data disclosed in this document, please cite our article and
data repository:

\begin{itemize}
\item
  article reference: \texttt{to\ be\ added}
\item
  repository reference: \texttt{to\ be\ added}
\end{itemize}

Standardised ecogeographical variables are available for download from the project's
Zenodo \href{https://doi.org/10.5281/zenodo.17428601}{repository}.

Layers can be interacted with in the Google Earth Engine \href{https://aavj.users.earthengine.app/view/egvs-2024}{application}.

\chapter*{References}\label{references}
\addcontentsline{toc}{chapter}{References}

\phantomsection\label{refs}
\begin{CSLReferences}{1}{0}
\bibitem[\citeproctext]{ref-DynWorld}
Brown, C.F., Brumby, S.P., Guzder-Williams, B., Birch, T., Hyde, S.B., Mazzariello, J., Czerwinski, W., Pasquarella, V.J., Haertel, R., Ilyushchenko, S., Schwehr, K., Weisse, M., Stolle, F., Hanson, C., Guinan, O., Moore, R., Tait, A.M., 2022. {Dynamic} {World}, {Near} real-time global 10 m land use land cover mapping. Scientific Data 9, 251. \url{https://doi.org/10.1038/s41597-022-01307-4}

\bibitem[\citeproctext]{ref-HydroClim}
Domisch, S., Amatulli, G., Jetz, W., 2015. Near-global freshwater-specific environmental variables for biodiversity analyses in 1 km resolution. Scientific Data 2:150073, 1--13. \url{https://doi.org/10.1038/sdata.2015.73}

\bibitem[\citeproctext]{ref-GEEpaper}
Gorelick, N., Hancher, M., Dixon, M., Ilyushchenko, S., Thau, D., Moore, R., 2017. {Google} {Earth} {Engine}: {Planetary-scale} geospatial analysis for everyone. Remote Sensing of Environment 202, 18--27. \url{https://doi.org/10.1016/j.rse.2017.06.031}

\bibitem[\citeproctext]{ref-theGFW}
Hansen, M.C., Potapov, P.V., Moore, R., Hancher, M., Turubanova, S.A., Tyukavina, A., Thau, D., Stehman, S.V., Goetz, S.J., Loveland, T.R., Kommareddy, A., Egorov, A., Chini, L., Justice, C.O., Townshend, J.R.G., 2013. {High}-resolution {Global} maps of 21st-century forest cover change. Science 342, 850--853. \url{https://doi.org/10.1126/science.1244693}

\bibitem[\citeproctext]{ref-worldclim_hijmans}
Hijmans, R.J., Cameron, S.E., Parra, J.L., Jones, P.G., Jarvis, A., 2005. Very high resolution interpolated climate surfaces for global land areas. International Journal of Climatology 25, 1965--1978. \url{https://doi.org/10.1002/joc.1276}

\bibitem[\citeproctext]{ref-CHELSA}
Karger, D.N., Conrad, O., Böhner, J., Kawohl, T., Kreft, H., Soria-Auza, R.W., Zimmermann, N.E., Linder, H.P., Kessler, M., 2017. {Data} {Descriptor}: {Climatologies} at high resolution for the earth's land surface areas. Scientific Data 4:170122. \url{https://doi.org/10.1038/sdata.2017.122}

\bibitem[\citeproctext]{ref-HydroBasins}
Lehner, B., Grill, G., 2013. Global river hydrography and network routing: Baseline data and new approaches to study the world's large river systems. Hydrological Processes 27, 2171--2186. \url{https://doi.org/10.1002/hyp.9740}

\bibitem[\citeproctext]{ref-HydroSheds}
Lehner, B., Verdin, K., Jarvis, A., 2008. New global hydrography derived from spaceborne elevation data. Eos, Transactions, American Geophysical Union 89, 93--94. \url{https://doi.org/10.1029/2008EO100001}

\bibitem[\citeproctext]{ref-esdac2}
Panagos, P., Liedekerke, M.V., Borrelli, P., Köninger, J., Ballabio, C., Orgiazzi, A., Lugato, E., Liakos, L., Hervas, J., Jones, A., Montanarella, L., 2022. {European} {Soil} {Data} {Centre} 2.0: {Soil} data and knowledge in support of the {EU} policies. European Journal of Soil Science 73, e13315. \url{https://doi.org/10.1111/ejss.13315}

\bibitem[\citeproctext]{ref-PALSARForest}
Shimada, M., Itoh, T., Motooka, T., Watanabe, M., Shiraishi, T., Thapa, R., Lucas, R., 2013. New global forest/non-forest maps from {ALOS PALSAR} data (2007--2010). Remote Sensing of Environment 155, 13--31. \url{https://doi.org/10.1016/j.rse.2014.04.014}

\bibitem[\citeproctext]{ref-WangLiu2006}
Wang, L., Liu, H., 2006. An efficient method for identifying and filling surface depressions in digital elevation models for hydrologic analysis and modelling. International Journal of Geographical Information Science 20, 193--213. \url{https://doi.org/10.1080/13658810500433453}

\end{CSLReferences}

\end{document}
